\section{Conclusion} 
\label{sec:concl-future-work}

In this paper we have shown a MTSMS model, including the requirements
for these special systems and how they can be applied to an example
time series. The main objective is to store compactly a time series
and manage consistently its temporal dimension.

Our MTSMS model proposes to store a time series split into time
subseries, which we call resolution discs.  Each resolution disc has a
different resolution and is compacted with an attribute aggregate
function. Therefore, in a multiresolution database the configuration
parameters are the quantity of resolution discs and each of their
three parameters: the consolidation step, the attribute aggregate
function and the capacity.

The data model shown is the first step to develop a complete model for
a MTSMS. In the future the operations will be defined. In this
context, there is a need for a model collecting generic properties for
the TSMS, as it can be the time series union operation or the time
interval operations. Then, the multiresolution model would be build
upon the generic TSMS model.

Concluding, in this paper we show that using TSMS facilitates
substantially time series management. The current field interest makes
us optimistic to expect soon an adequate management in DBMS.

%%% Local Variables:
%%% TeX-master: "main-wseas"
%%% ispell-local-dictionary: "british"
%%% End:

% LocalWords: multiresolution MTSMS DBMS RDBMS TSMS

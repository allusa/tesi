
\begin{frame}{Time Series Management Systems (TSMS)}

  A \acro{TSMS} is a special purpose \acro{DBMS} devoted to store and
  manage time series. The main objective of \acro{TSMS} is to put
  together two areas of study: time series analysis and \acro{DBMS}.
  Main features:
  \begin{itemize}
  \item Continuous nature
  \item Large data sets
  \item Temporal coherence, sometimes time not equi-spaced.
  \end{itemize}

  Nomenclature:

  \begin{enumerate}

  \item Measure $m=(v,t)$. Value measured in a time instant.
    \begin{itemize}
    \item The time value defines the canonical order between measures
    \end{itemize}

  \item Time series $S=\{m_0,\ldots,m_k\}$. Sequence of measures.
    \begin{itemize}
    \item Same phenomena
    \item No repeated time values
    \item Regular time series when measures are equi-spaced.
    \end{itemize}

  \end{enumerate}

\end{frame}


%%% Local Variables: 
%%% ispell-local-dictionary: "british"
%%% mode: latex
%%% TeX-master: "presentacio"
%%% End: 

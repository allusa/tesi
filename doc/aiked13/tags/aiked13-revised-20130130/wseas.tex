\documentclass[twocolumn,11pt]{article}
\usepackage{times}
%
% DO NOT CHANGE THE FOLLOWING PART
%
\setlength{\textwidth}{6.9in}
\setlength{\textheight}{9.5in}
\setlength{\oddsidemargin}{-0.25in}
\setlength{\evensidemargin}{-0pt}
\setlength{\topmargin}{-0.25in}
\setlength{\columnsep}{0.4in}
\setlength{\parindent}{4ex}
%
\newtheorem{definition}{Definition}
\newtheorem{remark}[definition]{Remark}
\newtheorem{lemma}[definition]{Lemma}
\newtheorem{theorem}[definition]{Theorem}
\newtheorem{proposition}[definition]{Proposition}
\newtheorem{corollary}[definition]{Corollary}
%
%
% THIS IS THE PLACE FOR YOUR OWN DEFINITIONS
%
\newcommand{\R}{{\rm I}\!{\rm R}} % the set of real numbers
% boldface characters in mathematical formulas
\newcommand{\bff}{\mbox{\boldmath $f$}}
\newcommand{\bfl}{\mbox{\boldmath $l$}}
\newcommand{\bfn}{\mbox{\boldmath $n$}}
\newcommand{\bfu}{\mbox{\boldmath $u$}}
\newcommand{\bfx}{\mbox{\boldmath $x$}}
\newcommand{\bfy}{\mbox{\boldmath $y$}}
\newcommand{\bfzero}{\mbox{\boldmath $0$}}
\newcommand{\bfcurl}{{\bf curl}} % curl of a vector field
\newcommand{\rmdiv}{{\rm div}} % divergence of a vector field
\newcommand{\eop}{\hfill $\sqcap\!\!\!\!\sqcup$} % end of proof
%
%
% THE BEGINNING OF THE DOCUMENT
%
\begin{document}
\global\def\refname{{\normalsize \it References:}}
%
\baselineskip 12.5pt
%
%
% TITLE, AUTHOR, ABSTRACT, KEYWORDS
%
\title{\LARGE \bf Instructions for Preparation of Manuscripts for
WSEAS Proceedings and Journals by Means of LaTeX}

\date{}

\author{\hspace*{-10pt}
\begin{minipage}[t]{2.7in} \normalsize \baselineskip 12.5pt
\centerline{JOHN BROWN}
\centerline{Name of the Institution}
\centerline{Department of Applied Mathematics}
\centerline{Long Street 25, 121 35 City}
\centerline{COUNTRY}
\centerline{brown@math.univ.az}
\end{minipage} \kern 0in
\begin{minipage}[t]{2.7in} \normalsize \baselineskip 12.5pt
\centerline{SECOND AUTHOR}
\centerline{Name of the University}
\centerline{Institute of Mechanical Engineering}
\centerline{47 West Lincoln Avenue, 87 115 City}
\centerline{COUNTRY}
\centerline{second.author@math.univ.ab}
\end{minipage}
%
% If you are three authors then you can use three mini--pages
% instead of two. Their horizontal size must be less than 2.7in
% indicated above. It can be e.g. 2.3in. However, you must pay
% attention that you do not exceed the total width of the text.
%
\\ \\ \hspace*{-10pt}
\begin{minipage}[b]{6.9in} \normalsize
\baselineskip 12.5pt {\it Abstract:}
% The text of the abstract follows.
This paper contains brief instructions for authors who wish to
prepare their manuscripts for WSEAS proceedings or journals by
means of LaTeX. You will find the format you have to choose,
fonts, how to type the title of your paper, the titles of
sections, examples of definitions, lemmas, theorems, equations
etc.
%
\\ [4mm] {\it Key--Words:}
% The key-words follow.
Typing manuscripts, \LaTeX
\end{minipage}
\vspace{-10pt}}

\maketitle

\thispagestyle{empty} \pagestyle{empty}
% numbers of pages are supplemented by the editor
%
% THE BEGINNING OF THE TEXT
%
\section{Introduction}
\label{S1} \vspace{-4pt}

Although WSEAS proceedings and journals accept papers in the pdf
or in a ps format, authors of the papers are encouraged to prepare
their manuscripts in LaTeX. This concerns especially papers which
contain many mathematical symbols, formulas, equations and
similar. LaTeX is a very powerful tool specially developed for
preparation of such manuscripts and it represents, together with
Plain TeX and AMS TeX, a very popular and leading text processor
among wide mathematical community.

The package ``times'' (the declaration on line 2 in the source tex
file) is used because it produces fonts which are mostly similar
to the required WSEAS format. You can use the package ``times''
only if you have a LaTeX 2e compiler.

If you do not have the LaTeX 2e compiler then you must replace the
declaration ``documentclass'' on line 1 of the source text by
``documentstyle'' and omit the whole line 2. In that case, your
fonts will slightly differ from the font ``times new roman'' which
is required by WSEAS in another example prepared by means of MS
Word. This concerns especially the title of the paper and the
titles of sections. The same happens if you use ``documentclass''
on line 1, but omit ``usepackage\{times\}'' on line 2.

The author has the experience that the package ``times'' causes
that the dvi file contains some incorrect characters on some LaTeX
installations. Then the dvi viewer for example shows the
connection ``fi'' (even inside a word like ``file'') as the Greek
letter $\Theta$, italic letters look same as roman letters etc.
Consequently, if you create a pdf file from the dvi file, it also
looks bad. However, the perfect pdf file is always produced by
using the PDFLatex.exe instead of Latex.exe compiler (or the
PDFLaTeX command instead of the LaTeX command) and creating the
pdf file directly from the tex file.

\subsection{Subsection}
\vspace{-4pt}

When including a subsection, LaTeX automatically produces for its
heading smaller letters as here.

Section \ref{S2} contains a sample of a mathematical text.
Mathematical equations must be numbers as follows: (1), (2),
$\dots$ and not (1.1), (1.2) $\dots$ depending on your sections.
LaTeX does it automatically if you do not change it by a special
command.

\section{Definitions of Function Spaces and Notation}
\label{S2} \vspace{-4pt}

Assume that $\Omega$ is a bounded simply connected domain in
$\R^3$ whose boundary $\partial\Omega$ is a smooth surface. We
shall use the notation:

\begin{list}{$\circ$}
{\setlength{\topsep 1pt}
\setlength{\itemsep 1pt}
\setlength{\leftmargin 8pt}
\setlength{\labelwidth 6pt}}

\item
$L_{\sigma}^2(\Omega)^3$ is a subspace of $L^2(\Omega)^3$ which
contains functions $\bfu$ whose divergence equals zero in $\Omega$
in the sense of distributions and $(\bfu\cdot\bfn)|_{\partial
\Omega}=0$ in the sense of traces.

\item
$D^1$ is the set of functions $\bfu\in W^{1,2}(\Omega)^3\cap
L_{\sigma}^2(\Omega)^3$ such that $(\bfcurl\, \bfu\cdot
\bfn)|_{\partial\Omega}=0$ in the sense of traces. It is a closed
subspace of $W^{1,2}(\Omega)^3$.
\end{list}

Let $T>0$. We deal with the initial--boundary value problem which
is defined by the equations
\begin{eqnarray}
\frac{\partial\bfu}{\partial t}+(\bfu\cdot\nabla)\bfu &=& -\nabla
p+\nu\Delta\bfu+\bff \label{5} \\
\rmdiv\, \bfu &=& 0 \label{6}
\end{eqnarray}
in $Q_T\equiv\Omega\times(0,T)$, by the initial condition
\begin{equation}
\bfu(\bfx,0)=\bfu^*(\bfx) \qquad \hbox{for}\ \bfx\in\Omega
\label{7}
\end{equation}
and by the generalized impermeability boundary conditions
\begin{equation}
\bfu\cdot\bfn=0,\ \ \bfcurl\, \bfu\cdot\bfn=0,\ \
\bfcurl^2\bfu\cdot\bfn=0 \label{4}
\end{equation}
on $\partial\Omega\times(0,T)$. (These boundary conditions were
introduced in \cite{BNP}.)

\begin{definition} \label{D1}
This is the example of a definition. In order to stress the
defined notion, you can e.g.~\underline{underline} it or type it
by {\bf boldface} letters. \rm Or you can write the whole
definition by roman letters and type just the defined notion by
{\it italics}.
\end{definition}

\begin{lemma} \label{L1}
A function $\bfu\in W^{1,2}(\Omega)^3\cap L^2_{\sigma}(\Omega)^3$
sa\-tisfies the homogeneous Dirichlet boundary condition
\begin{equation}
\bfu(\bfx)=\bfzero \qquad \hbox{for}\ \bfx\in\partial\Omega
\label{1}
\end{equation}
if and only if it satisfies
\begin{eqnarray}
\bfu\cdot\bfn=0, \quad \bfcurl\, \bfu\cdot\bfn=0, \quad
\frac{\partial\bfu}{\partial\bfn}\cdot\bfn=0 \label{2}
\end{eqnarray}
on the boundary $\partial\Omega$ of domain $\Omega$.
\end{lemma}

\noindent
{\bf Proof:} \ Assume that $\bfu$ is a smooth vector function in
$L^2_{\sigma}(\Omega)^3$ at first.

If $\bfu$ satisfies (\ref{1}) then $\bfu$ and $\bfcurl\, \bfu$
obviously satisfy the first two conditions in (\ref{2}). Let us
verify the third condition. Let $\bfx_0\in\partial\Omega$. The
cartesian system of coordinates can be chosen so that the origin
is at point $\bfx_0$ and $\bfn$ shows the direction of the
$x_3$--axis. Since $u_1=u_2=0$ on $\partial\Omega$ and $\dots$
etc. \eop

\vspace{2pt}
Lemma \ref{L1} confirms that the generalized impermeability
boundary conditions (\ref{4}) differ from boundary condition
(\ref{1}) only in the third condition in (\ref{2}).

\begin{theorem} \label{T1}
This is the example of a theorem. Texts of lemmas and theorems are
usually typed by italic or slanted fonts in mathematical texts and
so LaTeX does it automatically, until you do not change it by a
special command.
\end{theorem}

\section{Conclusion}
\label{S3} \vspace{-4pt}

The results explained in the previous sections show that the
generalized impermeability boundary conditions (\ref{4}) represent
an acceptable alternative to (\ref{1}). Their relevance on
boundaries of various smoothness $\dots$ etc.


\vspace{10pt} \noindent
{\bf Acknowledgements:} \ The research was supported by the
University of ABC and in the case of the first author, it was also
supported by the Grant Agency of DEF (grant No. 000/05/ 0000).

\begin{thebibliography}{11}

\vspace{-7pt}
\bibitem{BNP}
H.~Bellout, J.~Neustupa and P.~Penel, On the Navier-Stokes
Equation with Boundary Condi\-tions Based on Vorticity, {\em
Math.~Nachr.} 269--270, 2004, pp.~59--72.

\vspace{-7pt}
\bibitem{GR}
V.~Girault and P.--A.~Raviart, {\em Finite Element Methods for
Navier--Stokes Equations,} Springer --Verlag,
Berlin--Heidelberg--New York--Tokyo 1986

\vspace{-7pt}
\bibitem{Ho}
E.~Hopf, \"Uber die Anfangswertaufgabe f\"ur die Hydrodynamischen
Grundgleichungen, {\em Math. Nachr.}~4, 1951, pp.~213--231.

\vspace{-7pt}
\bibitem{Ka}
T.~Kato, Non--stationary flows of viscous and ideal fluids in
$\R^3$, {\em J.~Func.~Anal.}~9, 1972, pp. 296--305.

\vspace{-7pt}
\bibitem{Se}
J.~Serrin, On the interior regularity of weak solutions of the
Navier--Stokes equations, {\em Arch.~Rat. Mech.~Anal.}~9, 1962,
pp.~187--195.

\end{thebibliography}

\end{document}

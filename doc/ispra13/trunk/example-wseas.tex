\section{Example}
\label{sec:example}

Next we show an example database for a real time series data.  It is
situated in a distributed sensor monitoring system where temperature
data is captured from various sensors \cite{alippi10}. We focus on one
sensor data.

Original data is shown in figure~\ref{fig:exemple:original} for a
period of one year and a half. %, from 29th April 2010 to 18th October 2011.
 The time series is plotted interpolating linearly its measures.
In this plot we can see that there is missing data and some outlying
observations, and there are 146709 stored values.


\begin{figure}[tp]
  \centering
% \tikzset{
% every node/.style={transform shape}
% }
\tikzset{every picture/.style={scale=0.8}}
  \usetikzlibrary{dateplot}    
\begin{tikzpicture}
    \begin{axis}[
        date coordinates in=x,
%        xticklabel={\pgfcalendar{tickcal}{\tick}{\tick}{\pgfcalendarshorthand{m}{.}}},
        xticklabel={\pgfcalendarmonthshortname{\month} \year},
        xticklabel style= {rotate=15,anchor=east},
        xlabel=Time,
        ylabel=Temperature (K),
        ymax = 320,
        clip=false,
        y filter/.code = { \pgfmathparse{(#1>320)*322+(#1<320)*#1}},
        ]
       \addplot[blue] file {dades/matriu0.originalbyday.dat};

%      \node[right] at (axis cs:2011-10-12,330) {\footnotesize(2938)};
       \node (break) at (axis cs:2011-10-02,318)[inner sep=0pt,minimum width=0.75em, minimum height=0.5ex,fill=white] {};
    \draw [fill=red,color=blue] (break.north east) -- (break.north west) (break.south west) -- (break.south east);



  \end{axis}
\end{tikzpicture}



%%% Local Variables:
%%% TeX-master: "../main"
%%% ispell-local-dictionary: "british"
%%% End:

  %\includegraphics[width=12cm]{\experiment/original.pdf}
  \caption{Example of a temperature time series data}
  \label{fig:exemple:original}
\end{figure}



%\subsection{Schema}

\vspace{5pt} \noindent {\bf Schema:} \ We design a multiresolution
time series database (MTSDB) that stores a time series with high
resolution at recent times and with low resolution at older times. The
schema is illustrated in figure~\ref{fig:exemple:window}. At the top
there are four \emph{discs} with different number of \emph{measures}
and at the bottom there is a timeline showing the time series chopped
along time. For recent times, every 5 hours a \emph{measure} is stored
in the fourth \emph{disc} which has a capacity of 24 \emph{measures}
so that results in a 5 day piece. For low mid times, every 2 days in
the third \emph{disc} with a capacity of 20 resulting in a 40 days
piece. For high mid times, every 15 in the second \emph{disc} with a
capacity of 12 resulting 180 days. For old times, every 50 days in the
first \emph{disc} with a capacity of 12 resulting results in 600 days.

\begin{figure}[tp]
\centering
\setlength{\unitlength}{1.3mm}
%mrd.afegeix_disc(h5,24,mitjana,zero)
%mrd.afegeix_disc(d2,20,mitjana,zero)
%mrd.afegeix_disc(d15,12,mitjana,zero)
%mrd.afegeix_disc(d50,12,mitjana,zero)
\tiny
\begin{center}
%\begin{multicols}{4} 


    \begin{picture}(14,12)(-7,-6)
    \put(0,-1){\makebox(0,0)[c]{{\color{blue}50 days}}}
      \put(0,0){\circle{10}}
      \put(5,0){\circle{0.8}}
      \put(4.33,2.5){\circle{0.8}}
      \put(2.5,4.33){\circle{0.8}}
      \put(0,5){\circle{0.8}}
      \put(-2.5,4.33){\circle{0.8}}   
      \put(-4.33,2.5){\circle{0.8}}
      \put(-5,0){\circle{0.8}}
      \put(-4.33,-2.5){\circle{0.8}}
      \put(-2.5,-4.33){\circle{0.8}} 
      \put(0,-5){\circle{0.8}}
      \put(2.5,-4.33){\circle{0.8}} 
      \put(4.33,-2.5){\circle{0.8}}
      \put(0,0){\vector(0,1){5}}
      \put(0,0){\oval(5,5)[t]}
      \put(-2.5,0){\makebox(0,0)[c]{$\vee$}}
    \end{picture}
%
    \begin{picture}(14,12)(-7,-6)
    \put(0,-1){\makebox(0,0)[c]{{\color{brown}15 days}}}
      \put(0,0){\circle{10}}
      \put(5,0){\circle{0.8}}
      \put(4.33,2.5){\circle{0.8}}
      \put(2.5,4.33){\circle{0.8}}
      \put(0,5){\circle{0.8}}
      \put(-2.5,4.33){\circle{0.8}}   
      \put(-4.33,2.5){\circle{0.8}}
      \put(-5,0){\circle{0.8}}
      \put(-4.33,-2.5){\circle{0.8}}
      \put(-2.5,-4.33){\circle{0.8}} 
      \put(0,-5){\circle{0.8}}
      \put(2.5,-4.33){\circle{0.8}} 
      \put(4.33,-2.5){\circle{0.8}}
      \put(0,0){\vector(0,1){5}}
      \put(0,0){\oval(5,5)[t]}
      \put(-2.5,0){\makebox(0,0)[c]{$\vee$}}
    \end{picture}
%
    \begin{picture}(14,12)(-7,-6)
    \put(0,-1){\makebox(0,0)[c]{{\color{red}2 days}}}
      \put(0,0){\circle{10}}
      %\put(5,0){\circle{0.8}}
      \put(4.82,1.29){\circle{0.8}}
      \put(4.33,2.5){\circle{0.8}}
     \put(3.5,3.5){\circle{0.8}}
      \put(2.5,4.33){\circle{0.8}}
      \put(1.29,4.82){\circle{0.8}}
      %\put(0,5){\circle{0.8}}
      \put(-1.29,4.82){\circle{0.8}}
      \put(-2.5,4.33){\circle{0.8}}
       \put(-3.5,3.5){\circle{0.8}} 
      \put(-4.33,2.5){\circle{0.8}}
    \put(-4.82,1.29){\circle{0.8}}
      %\put(-5,0){\circle{0.8}}
    \put(-4.82,-1.29){\circle{0.8}}
      \put(-4.33,-2.5){\circle{0.8}}
      \put(-3.5,-3.5){\circle{0.8}} 
      \put(-2.5,-4.33){\circle{0.8 } } 
      \put(-1.29,-4.82){\circle{0.8 }}
      % \put(0,-5){\circle{0.8 }}
     \put(1.29,-4.82){\circle{0.8 }}
      \put(2.5,-4.33){\circle{0.8}}
      \put(3.5,-3.5){\circle{0.8}} 
      \put(4.33,-2.5){\circle{0.8}}
  \put(4.82,-1.29){\circle{0.8}}
      \put(0,0){\vector(0,1){5}}
      \put(0,0){\oval(5,5)[t]}
      \put(-2.5,0){\makebox(0,0)[c]{$\vee$}}
    \end{picture}
%
    \begin{picture}(14,12)(-7,-6)
    \put(0,-1){\makebox(0,0)[c]{{\color{cyan}5 hours}}}
      \put(0,0){\circle{10}}
      \put(5,0){\circle{0.8}}
      \put(4.82,1.29){\circle{0.8}}
      \put(4.33,2.5){\circle{0.8}}
     \put(3.5,3.5){\circle{0.8}}
      \put(2.5,4.33){\circle{0.8}}
      \put(1.29,4.82){\circle{0.8}}
      \put(0,5){\circle{0.8}}
      \put(-1.29,4.82){\circle{0.8}}
      \put(-2.5,4.33){\circle{0.8}}
       \put(-3.5,3.5){\circle{0.8}} 
      \put(-4.33,2.5){\circle{0.8}}
    \put(-4.82,1.29){\circle{0.8}}
      \put(-5,0){\circle{0.8}}
    \put(-4.82,-1.29){\circle{0.8}}
      \put(-4.33,-2.5){\circle{0.8}}
      \put(-3.5,-3.5){\circle{0.8}} 
      \put(-2.5,-4.33){\circle{0.8 } } 
      \put(-1.29,-4.82){\circle{0.8 }}
\put(0,-5){\circle{0.8 }}
     \put(1.29,-4.82){\circle{0.8 }}
      \put(2.5,-4.33){\circle{0.8}}
      \put(3.5,-3.5){\circle{0.8}} 
      \put(4.33,-2.5){\circle{0.8}}
  \put(4.82,-1.29){\circle{0.8}}
      \put(0,0){\vector(0,1){5}}
      \put(0,0){\oval(5,5)[t]}
      \put(-2.5,0){\makebox(0,0)[c]{$\vee$}}
    \end{picture}


%\end{multicols}

\vspace{-10pt}

\setlength{\unitlength}{900sp}
\begin{picture}(14460,5066)(7322,-7148)
\thinlines
{\color[rgb]{0,0,0}\put(7300,-6271){\line( 0,-1){386}}
}%
{\color[rgb]{0,0,0}\put(7782,-6271){\line( 0,-1){386}}
}%
{\color[rgb]{0,0,0}\put(8263,-6271){\line( 0,-1){386}}
}%
{\color[rgb]{0,0,0}\put(8745,-6271){\line( 0,-1){386}}
}%
{\color[rgb]{0,0,0}\put(9227,-6271){\line( 0,-1){386}}
}%
{\color[rgb]{0,0,0}\put(9709,-6271){\line( 0,-1){386}}
}%
{\color[rgb]{0,0,0}\put(10191,-6271){\line( 0,-1){386}}
}%
{\color[rgb]{0,0,0}\put(10673,-6271){\line( 0,-1){386}}
}%
{\color[rgb]{0,0,0}\put(11155,-6271){\line( 0,-1){386}}
}%
{\color[rgb]{0,0,0}\put(11637,-6271){\line( 0,-1){386}}
}%
{\color[rgb]{0,0,0}\put(12119,-6271){\line( 0,-1){386}}
}%
{\color[rgb]{0,0,0}\put(12600,-6271){\line( 0,-1){386}}
}%
{\color[rgb]{0,0,0}\put(13082,-6271){\line( 0,-1){386}}
}%
{\color[rgb]{0,0,0}\put(13564,-6271){\line( 0,-1){386}}
}%
{\color[rgb]{0,0,0}\put(14046,-6271){\line( 0,-1){386}}
}%
{\color[rgb]{0,0,0}\put(14528,-6271){\line( 0,-1){386}}
}%
{\color[rgb]{0,0,0}\put(15010,-6271){\line( 0,-1){386}}
}%
{\color[rgb]{0,0,0}\put(15492,-6271){\line( 0,-1){386}}
}%
{\color[rgb]{0,0,0}\put(15974,-6271){\line( 0,-1){386}}
}%
{\color[rgb]{0,0,0}\put(16456,-6271){\line( 0,-1){386}}
}%
{\color[rgb]{0,0,0}\put(16938,-6271){\line( 0,-1){386}}
}%
{\color[rgb]{0,0,0}\put(17419,-6271){\line( 0,-1){386}}
}%
{\color[rgb]{0,0,0}\put(17901,-6271){\line( 0,-1){386}}
}%
{\color[rgb]{0,0,0}\put(18383,-6271){\line( 0,-1){386}}
}%
{\color[rgb]{0,0,0}\put(18865,-6271){\line( 0,-1){386}}
}%
{\color[rgb]{0,0,0}\put(19347,-6271){\line( 0,-1){386}}
}%
{\color[rgb]{0,0,0}\put(19829,-6271){\line( 0,-1){386}}
}%
{\color[rgb]{0,0,0}\put(20311,-6271){\line( 0,-1){386}}
}%
{\color[rgb]{0,0,0}\put(20793,-6271){\line( 0,-1){386}}
}%
{\color[rgb]{0,0,0}\put(21275,-6271){\line( 0,-1){386}}
}%
{\color[rgb]{0,0,0}\put(7300,-6271){\line( 0,-1){1157}}
}%
{\color[rgb]{0,0,0}\put(9709,-6271){\line( 0,-1){1157}}
}%
{\color[rgb]{0,0,0}\put(12119,-6271){\line( 0,-1){1157}}
}%
{\color[rgb]{0,0,0}\put(14528,-6271){\line( 0,-1){1157}}
}%
{\color[rgb]{0,0,0}\put(16938,-6271){\line( 0,-1){1157}}
}%
{\color[rgb]{0,0,0}\put(19347,-6271){\line( 0,-1){1157}}
}%
{\color[rgb]{0,0,0}\put(21756,-6271){\line( 0,-1){1157}}
}%
{\color[rgb]{0,0,0}\put(7300,-6271){\line( 1, 0){14456}}
}%

\put(7322,-6271){\line( 0,1){3000}}
\put(21756,-7783){\makebox(0,0)[b]{now}}%
\put(7322,-7783){\makebox(0,0)[b]{600 days back}}%

\color{blue}
\put(21782,-5928){\line( -1,0){14460}}
\put(21782,-5928){\line( 0,1){779}}
\put(21782,-5149){\line( -1,0){14460}}
\put(7322,-5928){\line( 0,1){779}}
\put(14530,-5450){\makebox(0,0)[c]{600 days}}

\color{brown}
\put(21782,-5149){\line( 0,1){779}}
\put(21782,-4370){\line( -1,0){4438}}
\put(17344,-5149){\line( 0,1){779}}
\put(19563,-4800){\makebox(0,0)[c]{180 days}}

\color{red}
\put(21782,-4370){\line( 0,1){779}}
\put(21782,-3591){\line( -1,0){964}}
\put(20818,-4370){\line( 0,1){779}}
\put(21300,-3950){\makebox(0,0)[c]{40d}}

\color{cyan}
\put(21782,-3591){\line( 0,1){779}}
\put(21782,-2812){\line( -1,0){120}}
\put(21661,-3591){\line( 0,1){779}}
\put(21300,-3201){\makebox(0,0)[c]{5d}}
\end{picture}%


\normalsize

\end{center}
\caption{Schema of resolutions in a MTSDB}
\label{fig:exemple:window}
\end{figure}


%\subsection{Attribute aggregate functions}
\vspace{5pt} \noindent
{\bf Attribute aggregate functions:} \
In order to illustrate this example we consolidate all the resolution
discs by zohe arithmetic mean aggregate function and the highest
resolution disc by zohe maximum aggregate function, owing to their
simplicity. Next, we show the process in designing both aggregate
functions.

% In this example we show a possible family of attribute aggregate
% functions for time series represented by a staircase function, that is
% with a piecewise constant representation. 

We define a new
representation for time series called \emph{zero-order hold backwards}
(zohe%from \emph{zero-order hold everted}
) consisting in holding each value until the preceding value, which a
similar representation is used by RRDtool \cite{lisa98:oetiker}.

Let $S=\{m_0,\ldots,m_k\}$ be a time series, we define
$S(t)^{\text{zohe}}$ as its continuous representation along time $t$:
%continuous definition of the time series using left-continuous step functions.
$
\forall t \in \mathbb{R}  ,\forall m \in S:
S(t)^{\text{zohe}} =  
\begin{cases}
  \infty & \text{if } t > T(\max S) \\%\text{not defined}
  V(m) & \text{if }  t\in (T(\prev_S m),T(m)]
\end{cases}
$


We now define the \emph{zohe} attribute
aggregate function family as the one interpreting the
consolidation time interval left-continuous $i=(T_0,T_f]$
%
, let $S'=(S(T_0,T_f] \cup \{\min(S-S(-\infty,Tf))\}$:
% and the
% resulting aggregated measure's time always being $T_f$, in
% accordance to the \emph{zero-order hold backwards} representation
% being defined using left-continuous step functions.  Let
% $S=\{m_0,\ldots,m_k\}$ be a time series and $i=[T_0,T_f]$ be a time
% interval, the attribute aggregate function $f^{\text{zohe}}\in f$
% summarises $S$ with a measure, $f^{zohes}: S=\{m_0,\ldots,m_k\} \times
% i=[T_0,T_f] \mapsto m'$ where $m'=(v',T_f)$ and the resulting value
% $v'$ depends on the attribute aggregate function calculated from
% the subset $S^{zohe}=(S(T_0,T_f] \cup \{\min(S-S(-\infty,Tf))\}$.
%
% Let $S'=S^{zohe}$:
\begin{itemize}
\item maximum$^{zohe}$: $S \times i \mapsto m'$ where $V(m') =
  \max_{\forall m \in S'}(V(m))$ and $T(m')=T_f$.
\item arithmetic mean$^{zohe}$: $S \times i \mapsto m'$ where $V(m')
  = \frac{1}{|S'|} \sum\limits_{\forall m\in S'} V(m)$ and
  $T(m')=T_f$. 
\end{itemize}







\subsection{Results}

The resulting time series after consolidating the MTSDB are shown in
figure~\ref{fig:exemple:4mrd}, where each graphic corresponds to a
resolution disc's time series. Each title shows the
resolutions disc's delta and cardinal values, and each attribute
aggregate function has different colour.  Each time series
is plotted with zohe continuous representation. Time axis has
UTC units rounded to nearest time points
% in order to reduce printing space, for example the timestamp December 2010 is a rounding for a day between 1 and 31
and temperature axis has Kelvin
units. Outlayers are marked as discontinuities, for instance see
fourth plot's 2938 K maximum.


\begin{figure}[tp]
  \centering
  \tikzset{
    every picture/.style={scale=0.7},
%    every node/.style={transform shape}
  }
  
  \begin{tikzpicture}[scale=0.6, every node/.style={transform shape}]
    \begin{axis}[
        multiresoluciodate,
        title={$R_1$: 5h $|24|$},
        xticklabel={\day--\hour:\minute},
        clip=false,
        ]
       \addplot[const plot mark right, blue] table[col sep=comma] {imatges/exemple/dades-matriu0/R18000mean_zohe.csv};
     \node[left] at (axis cs:2011-10-19,274) {\footnotesize oct.~2011};
  \end{axis}
\end{tikzpicture}
%
  \begin{tikzpicture}[scale=0.6, every node/.style={transform shape}]
    \begin{axis}[
        multiresoluciodate,
        title={$R_2$: 2d $|20|$},
        xticklabel={\day~\pgfcalendarmonthshortname{\month}},
        clip=false,
        ]
       \addplot[const plot mark right, blue] table[col sep=comma] {imatges/exemple/dades-matriu0/R172800mean_zohe.csv};
     \node[left] at (axis cs:2011-10-21,279) {\footnotesize 2011};
  \end{axis}
\end{tikzpicture}
%
  \begin{tikzpicture}[scale=0.6, every node/.style={transform shape}]
    \begin{axis}[
        multiresoluciodate,
        title={$R_3$: 15d $|12|$},
        xticklabel={\day~\pgfcalendarmonthshortname{\month}},
        y filter/.code = { \pgfmathparse{(#1>320)*330+(#1<320)*#1}},
        ymax = 320,
        clip=false,
        ]

       \addplot[const plot mark right, blue] table[col sep=comma] {imatges/exemple/dades-matriu0/R1296000mean_zohe.csv};

      \addplot[const plot mark right, orange] table[col sep=comma] {imatges/exemple/dades-matriu0/R1296000maximum_zohe.csv};

      \node[right] at (axis cs:2011-10-07,330) {\footnotesize(2938)};
       \node (break) at (axis cs:2011-09-23,325)[inner sep=0pt,minimum width=0.75em, minimum height=0.5ex,fill=white] {};
    \draw [fill=red,color=orange] (break.north east) -- (break.north west) (break.south west) -- (break.south east);

     \node[left] at (axis cs:2011-10-27,273) {\footnotesize 2011};

  \end{axis}
\end{tikzpicture}
%
\begin{tikzpicture}[scale=0.6, every node/.style={transform shape}]
    \begin{axis}[
        multiresoluciodate,
        xticklabel={\pgfcalendarmonthshortname{\month}~\year},
        title={$R_4$: 50d $|12|$},
        xlabel={Temps (UTC)},
%        ylabel={Temperatura (K)},
        ymax = 320,
        clip=false,
%v1.6     restrict y to domain=0:320,
        y filter/.code = { \pgfmathparse{(#1>320)*330+(#1<320)*#1}},
        ]

       \addplot[const plot mark right, blue] table[col sep=comma] {imatges/exemple/dades-matriu0/R4320000mean_zohe.csv};
       \addlegendentry{mitjana};

       \addplot[const plot mark right, orange] table[col sep=comma] {imatges/exemple/dades-matriu0/R4320000maximum_zohe.csv};
       \addlegendentry{màxim};

       \node[right] at (axis cs:2011-10-12,330) {\footnotesize(2938)};
       \node (break) at (axis cs:2011-08-24,325)[inner sep=0pt,minimum width=0.75em, minimum height=0.5ex,fill=white] {};
    \draw [fill=red,color=orange] (break.north east) -- (break.north west) (break.south west) -- (break.south east);

  \end{axis}
\end{tikzpicture}




%%% Local Variables:
%%% TeX-master: "../../main"
%%% End:

  \caption{Resolution discs' time series in a MTSDB}
  \label{fig:exemple:4mrd}
\end{figure}


In all the four plots, we can see that arithmetic mean aggregate
function has filled missing data and filtered outlayers
observations. This is due to the aggregate function coming from a zohe
interpretation, that is known values are time left hold. 
% Otherwise, we
% could be stricter by using a non filling aggregate such as an unknown
% marking aggregate function as stated in
% section~\ref{sec:model:interpolador}.

% Moreover, in the MTSDB the number of stored values has been reduced to
% 92, which is the sum of disc's cardinalities. However, now there is
% only high resolution for recent data. Note that 18th October 2011
% corresponds to the 'now ' point showed in figure
% \ref{fig:exemple:window} and May 2010 corresponds to the oldest data
% known.


% We  can  easily  have a  bigger  MTSDB  by  doubling the  capacity  of
% resolution discs.  Then we  have 184 stored  values and  the resulting
% time series are  shown in figure~\ref{fig:exemple:4mrdbigger}.  Now we
% can preserve more historic but no more resolution. Although plots seem
% smoother, this is only a visualisation  issue owing to the fact we are
% showing more points in the same width. Fourth plot is the same in both
% figures as there is no more known historic.

% \begin{figure}[tp]
%   \centering
%   \tikzset{
%     every picture/.style={scale=0.8},
%     every node/.style={transform shape}
%   }
%   \usetikzlibrary{dateplot}  
%\usetikzlibrary{pgfplots.groupplots}

\pgfplotsset{
   petit/.style={
        ylabel=Temperature (K),
        width=\textwidth,
        height=3.5cm,
        legend style={font=\footnotesize},
        tick label style={font=\footnotesize},
        label style={font=\tiny},
        title style={font=\small,below, anchor=north,fill=white},
        xticklabel style= {rotate=15,anchor=east},
%        every axis title shift=0pt,
%        max space between ticks=15,
        every mark/.append style={mark size=6},
        major tick length=0.1cm,
        minor tick length=0.066cm,
        very thin,
    }
}

  \begin{tikzpicture}
    \begin{axis}[
        petit,
        title={RD: 5h $|48|$},
        date coordinates in=x,
        xticklabel={\day\ },%\hour:\minute},
        ]
       \addplot[const plot mark right, blue, mark=*] table[col sep=comma] {dades/mrdb2-matriu0/0.csv};
  \end{axis}
\end{tikzpicture}


  \begin{tikzpicture}
    \begin{axis}[
        petit,
        title={RD: 2d $|40|$},
        date coordinates in=x,
        xticklabel={\pgfcalendarmonthshortname{\month} \day},
        ]
       \addplot[const plot mark right, blue, mark=*] table[col sep=comma] {dades/mrdb2-matriu0/1.csv};
  \end{axis}
\end{tikzpicture}


  \begin{tikzpicture}
    \begin{axis}[
        petit,
        title={RD: 15d $|24|$},
        date coordinates in=x,
%        xticklabel={\pgfcalendar{tickcal}{\tick}{\tick}{\pgfcalendarshorthand{m}{.}}},
        xticklabel={\pgfcalendarmonthshortname{\month} \day},
%        xticklabel style= {rotate=15,anchor=east},
        title style = {below, anchor=north,fill=white},
%v>1.4        unbounded coords=jump,
        ]
       \addplot[const plot mark right, blue, mark=*] table[col sep=comma] {dades/mrdb2-matriu0/2.csv};
  \end{axis}
\end{tikzpicture}
\begin{tikzpicture}
    \begin{axis}[
        petit,
        date coordinates in=x,
        xticklabel={\pgfcalendarmonthshortname{\month} \year},
        title={RD: 50d $|24|$},
        xlabel=Time (UTC),
%        unbounded coords=jump, %v>1.4
        legend style = {anchor = north, draw = none},
        legend columns = 4,
        ymax = 320,
        clip=false,
%v1.6     restrict y to domain=0:320,
        y filter/.code = { \pgfmathparse{(#1>320)*330+(#1<320)*#1}},
        ]
       \addplot[const plot mark right, blue, mark=*] table[col sep=comma] {dades/mrdb2-matriu0/4.csv};
       \addlegendentry{mean};

       \addplot[const plot mark right, orange, mark=*] table[col sep=comma] {dades/mrdb2-matriu0/3.csv};
       \addlegendentry{max};

       \node[right] at (axis cs:2011-10-12,330) {\mbox{(2938)}};
       \node (break) at (axis cs:2011-08-23,325)[inner sep=0pt,minimum width=0.75em, minimum height=0.5ex,fill=white] {};
    \draw [fill=red,color=orange] (break.north east) -- (break.north west) (break.south west) -- (break.south east);

  \end{axis}
\end{tikzpicture}
%http://tex.stackexchange.com/questions/46422/axis-break-in-pgfplots

%http://tex.stackexchange.com/questions/52409/insert-a-separate-mark-inside-a-pgfplots-graph



%%% Local Variables:
%%% TeX-master: "../main"
%%% ispell-local-dictionary: "british"
%%% End:

%   \caption{Resolution discs' time series in a bigger MTSDB}
%   \label{fig:exemple:4mrdbigger}
% \end{figure}



With the multiresolution database filled now we may want to ask
queries, in which we are currently working in order to develop a model
for multiresolution operators.  For example, one query would be the
union of the four time subseries choosing the one with the highest
resolution as shown in figure~\ref{fig:exemple:4mrdtot}.  
%Therefore we obtain a piecewise function where each piece has a different resolution.  
Each time series is plotted interpolating linearly its measures, note
that this linearly visualisation seems right time displaced as time
series comes from a zohe aggregation.  Comparing this figure with the
original figure~\ref{fig:exemple:original}, as we have applied mean
aggregation it resembles an incremental low-pass filter and the
maximum aggregation resembles an envelope function.

\begin{figure}[tp]
  \centering
  \tikzset{every picture/.style={scale=0.8}}
  \usetikzlibrary{dateplot}  
%\usetikzlibrary{pgfplots.groupplots}

\pgfplotsset{
   petit/.style={
        ylabel=Temperature (K),
%        width=\textwidth,
%        height=3.5cm,
        legend style={font=\footnotesize},
        tick label style={font=\footnotesize},
        label style={font=\tiny},
        title style={font=\small,below, anchor=north,fill=white},
        xticklabel style= {rotate=15,anchor=east},
%        every axis title shift=0pt,
%        max space between ticks=15,
        every mark/.append style={mark size=6},
        major tick length=0.1cm,
        minor tick length=0.066cm,
        very thin,
        every axis legend/.append style={
          at={(1,0.02)},
          anchor=south east,
          draw = none},
        legend columns = 4,
    }
}

\begin{tikzpicture}
    \begin{axis}[
        petit,
        date coordinates in=x,
        xticklabel={\pgfcalendarmonthshortname{\month} \year},
        xlabel=Time (UTC),
%        unbounded coords=jump, %v>1.4
%        unbounded coords=discard, %v>1.4
        ymax = 320,
        clip=false,
%v1.6     restrict y to domain=0:320,
        y filter/.code = { \pgfmathparse{(#1>320)*330+(#1<320)*#1}},
        ]

       \addplot[black!15] file {dades/matriu0.originalbyday.dat};
       \addlegendentry{original};

       \addplot[blue] table[col sep=comma] {dades/mrdb-matriu0/union1.csv};
       \addlegendentry{mean};

       \addplot[orange] table[col sep=comma] {dades/mrdb-matriu0/union0.csv};
       \addlegendentry{max};

%       \node[right] at (axis cs:2011-10-12,330) {\mbox{(2938)}};
       \node (break) at (axis cs:2011-09-25,325)[inner sep=0pt,minimum width=0.9em, minimum height=0.4ex,fill=white] {};
    \draw [fill=red,color=orange] (break.north east) -- (break.north west) (break.south west) -- (break.south east);


  \end{axis}
\end{tikzpicture}
%http://tex.stackexchange.com/questions/46422/axis-break-in-pgfplots

%http://tex.stackexchange.com/questions/52409/insert-a-separate-mark-inside-a-pgfplots-graph



%%% Local Variables:
%%% TeX-master: "../main"
%%% ispell-local-dictionary: "british"
%%% End:

  %\includegraphics[width=12cm]{\experiment/isense3-tot.pdf}
  \caption{All time series united from the MTSDB}
  \label{fig:exemple:4mrdtot}
\end{figure}


Concluding, with this MTSDB schema we achieve not an
approximation to the original function but an extraction of some
interesting information.



%%% Local Variables:
%%% TeX-master: "main-wseas"
%%% ispell-local-dictionary: "british"
%%% End:

% LocalWords:  multiresolution MTSDB

\chapter{Implementació amb para\l.lelisme}


El disseny de la funció de multiresolució \todo{ref al capítol}
defineix una funció sobre una sèrie temporal. Aquesta funció té
bàsicament dues parts: un plec sobre un esquema de multiresolució i
mapes sobre la sèrie temporal. Així doncs, plantegem de resoldre
aquesta funció mitjançant computació para\l.lela. 


Una tècnica de computació para\l.lela és MapReduce, la qual s'adequa
bé al problema ja que es basa en aplicar operacions de mapa
(\emph{map}) i posteriorment plegar-les (\emph{reduce}). Un \gls{SGBD}
que es basa exclusivament en aquesta tècnica és Hadoop.\todo{refs}


En primer lloc, estudiem Hadoop i la tècnica MapReduce.  En segon
lloc, implementem usant Hadoop un \gls{SGSTM} anomenat
\emph{RoundRobindoop}. Aquest és un \gls{SGSTM} específic amb
l'objectiu de mostrar una implementació que resolgui la multiresolució
d'una sèrie temporal en temps diferit (\emph{offline}) i computant
para\l.lelament.



\section{Hadoop i MapReduce}

\todo{sobre Hadoop}



\todo{sobre MapReduce}

MapReduce és un algoritme per a processar unes dades que es basa en
resoldre les operacions en dues etapes --primer en una etapa de mapes
i segon en una etapa de plecs-- de tal manera que els mapes i els
plecs es puguin computar para\l.lelament. Aquestes dues etapes són
l'algoritme bàsic de l'algoritme MapReduce, però hi ha variacions que
afegeixen més etapes.  L'etapa de mapa correspon al nom en anglès Map,
i l'etapa de plec correspon a l'anglès fold que també és equivalent a
Reduce.\todo{abordar aquesta qüestió de notació}







\section{RoundRobindoop}






%%% Local Variables:
%%% TeX-master: "main"
%%% End:

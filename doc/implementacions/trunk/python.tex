

\chapter{Python}


La implementació de referència dels models de SGST i SGSTM s'ha
realitzat amb Python \todo{referència}. L'objectiu d'aquesta
implementació de referència és mantenir la fidelitat al model per tal
de poder experimentar-hi amb tota la potència matemàtica .

En la implementació s'han afegit alguns operadors que en el model ja
estaven definits perquè són propis de l'àlgebra de conjunts. Alguns
dels operadors principals que s'han d'afegir són els relacionats amb
la notació de creació de conjunts, els quals en els SGBD es coneixen
com a \emph{llenguatge de definició de dades} (DDL, de l'anglès
\emph{data definition language}). 


Implementem els dos models de SGST i SGSTM com a dues biblioteques
diferents: \emph{Pytsms} i \emph{RoundRobinson} respectivament. La
\emph{RoundRobinson}, però, té una forta dependència en la
\emph{Pytsms} de la mateixa manera que hem definit els SGSTM en base
als SGST.


Utilitzem orientació a objectes. Ens permet fer explicita la relació
entre la implementació i el model. \todo{més explicat}

Utilitzem diagrames UML per a definir l'estructura de classes. \todo{més explicat}


Implementem la part essencial, és a dir l'àlgebra definida en els
models lògics. No implementem els complements habituals dels SGBD, els
quals són necessaris en entorns d'explotació, com per exemple gestió
d'usuaris i permisos, còpies de seguretat, llenguatges estàndards de
consulta, etc.




\section{Pytsms}

La biblioteca \emph{Pytsms} implementa un SGST de referència. Així
doncs, seguint el model, els objectes principals són les mesures, les
sèries temporals i les representacions de les sèries temporals. Tots
tres s'implementen respectivament com a classes \emph{Measure},
\emph{TimeSeries} i \emph{Representation}.


\begin{figure}[tp]
  \centering
  \begin{tikzpicture}

  %Timeseries
  \umlclass[x=0,y=0] {TimeSeries}{}{}  

  % Measure
  \umlclass[x=-4] {Measure}{}{}
  \umluniaggreg[mult=0..*]  {TimeSeries}{Measure}
  \umlclass[x=-5.2,y=-2] {MFloat}{}{}
  \umlclass[x=-2.8,y=-2] {MString}{}{}
  \umlinherit {MFloat}{Measure}
  \umlinherit {MString}{Measure}

  %Repr
  \umlclass[x=4,type=abstract] {Representation}{}{}
  \umlassoc[mult1=1,mult2=1]  {TimeSeries}{Representation}
  \umlclass[x=3,y=-2] {Zohe}{}{}
  \umlclass[x=5,y=-2] {Discret}{}{}
  \umlinherit {Zohe}{Representation}
  \umlinherit {Discret}{Representation}

  %Associacions
  \umlclass[x=-1.5,y=-4] {RegularProp}{}{}
  \umluniassoc  {RegularProp}{TimeSeries}
  \umlclass[x=1.5,y=-4] {Storage}{}{}
  \umluniassoc {Storage}{TimeSeries}
  \end{tikzpicture}


  \caption{Diagrama UML de pyTsms}
  \label{fig:implementacio:pytsms-uml}
\end{figure}


\begin{figure}
  \centering

\begin{tikzpicture}

  %Timeseries
  \umlclass[x=0,y=0] {TimeSeries}{}{}  
  %Realisations 
  \umlclass[x=-3.5,y=-3] {Structure}{}{}
  \umlclass[x=-1.2,y=-3] {OpSet}{}{}
  \umlclass[x=1.2,y=-3] {OpSeq}{}{}
  \umlclass[x=3.5,y=-3] {OpFunc}{}{}
  \umlreal[geometry=|-|]{Structure}{TimeSeries}
  \umlreal[geometry=|-|]{OpSet}{TimeSeries}
  \umlreal[geometry=|-|]{OpSeq}{TimeSeries}
  \umlreal[geometry=|-|]{OpFunc}{TimeSeries}
  %Subrealisations
  \umlclass[x=-3,y=-6] {SetNoTemporal}{}{}
  \umlclass[x=0.7,y=-6] {SetTemporal}{}{}
  \umlclass[x=4,y=-6] {SetRelacional}{}{}
  \umlreal[geometry=|-|]{SetNoTemporal}{OpSet}
  \umlreal[geometry=|-|]{SetTemporal}{OpSet}
  \umlreal[geometry=|-|]{SetRelacional}{OpSet}
  \umlclass[x=-6,y=-6] {Set}{}{}
  \umlinherit{Structure}{Set}

\end{tikzpicture}

  \caption{Diagrama UML de la realització de sèries temporals a pyTsms}
  \label{fig:implementacio:pytsms-uml-ts}
\end{figure}





La \autoref{fig:implementacio:pytsms-uml} mostra amb un diagrama
UML la relació entre aquests tres objectes principals. Així, per una
banda, una \emph{TimeSeries} té una relació d'agregació amb les
\emph{Measure}, és a dir que una sèrie temporal conté cap, una o més
d'una mesura.  Per altra banda, les sèries temporals i les
representacions són ortogonals i això s'implementa mitjançant una
relació d'associació bidireccional entre una \emph{TimeSeries} i una
\emph{Representation}, és a dir que una instància de sèrie temporal té
associada una representació i una instància de representació coneix la
sèrie temporal que representa.




Una \emph{TimeSeries} és un objecte amb una gran quantitat de
mètodes. Com a conseqüència, la implementació de funcionalitats
essencials s'ha dividit en diversos mòduls, el qual es mostra a la
\autoref{fig:implementacio:pytsms-uml-ts}. Els mètodes que implementen
el model estructural i el model d'operacions bàsiques s'han agrupat en
objectes segons la seva funcionalitat. Així hi ha l'objecte
\emph{Structure} que implementa el model estructural de les sèries
temporals, l'\emph{OpSet} pel model d'operacions de conjunts,
l'\emph{OpSeq} pel model d'operacions de seqüències i l'\emph{OpFunc}
pel model d'operacions de funció temporal.  Aleshores el
\emph{TimeSeries} hereta les funcionalitats d'aquests quatre objectes.
L'objecte \emph{OpSet} també té una gran quantitat de mètodes i, de la
mateixa manera, hereta la funcionalitat de tres objectes: el
\emph{SetNoTemporal} per les operacions basades en l'orde parcial de
les sèries temporals, el \emph{SetTemporal} basat en l'ordre temporal
i el \emph{SetRelacional} per les operacions específiques de l'àlgebra
relacional.



La implementació de funcionalitats complementàries s'ha implementat
amb relacions d'associació unidireccionals, el qual es mostra a la
\autoref{fig:implementacio:pytsms-uml}.



La \autoref{fig:implementacio:pytsms-uml} també mostra
especialitzacions de les mesures i de les representacions.




En el cas de les representacions, una \emph{Representation} és una
classe abstracta, és a dir

La representació és abstracta, sempre ha de ser una de concreta? I la discreta pura?
 Cada usuari ha de definir la representació
que vol associar, com a exemple se n'implementen algunes.



\subsubsection{Encaix a Python}

Heretem de sets, implementem els mètodes especials de sets

implementem els mètodes especials de seqüències







\section{RoundRobinson}

\begin{tikzpicture}

  %MultiTimeseries
  \umlclass[x=0,y=0] {MultiresolutionSeries}{}{}  
  \umlclass[x=4,y=0] {Set}{}{}
  \umlinherit{MultiresolutionSeries}{Set}
  %Components 
  \umlclass[x=0,y=-3] {Resolution}{}{}
  \umluniaggreg  {MultiresolutionSeries}{Resolution}
  %SubComponents 
  \umlclass[x=-1.2,y=-6] {Buffer}{}{}
  \umlclass[x=1.2,y=-6] {Disc}{}{}
  \umlclass[x=-3,y=-9] {Aggregators}{}{}
  \umlunicompo[mult=1]  {Resolution}{Buffer}
  \umlunicompo[mult=1]  {Resolution}{Disc}
  \umluniassoc[mult=1]  {Buffer}{Aggregators}

  %TimeSeries
  \begin{umlpackage}[x=0,y=-9]{pyTsms}
    \umlclass{TimeSeries}{}{}  
  \end{umlpackage}
  \umluniassoc[mult=1]  {Buffer}{TimeSeries}
  \umluniassoc[mult=1]  {Disc}{TimeSeries}

  %Associacions
  \umlclass[x=5,y=2] {Storage}{}{}
  \umluniassoc {Storage}{MultiresolutionSeries}
  \umlclass[x=2,y=2] {Plot}{}{}
  \umluniassoc {Plot}{MultiresolutionSeries}

\end{tikzpicture}




%%% Local Variables:
%%% TeX-master: "main"
%%% End:
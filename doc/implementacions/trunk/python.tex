

\chapter{Python}


La implementació de referència dels models de SGST i SGSTM s'ha
realitzat amb Python. L'objectiu d'aquesta implementació de referència és mantenir la fidelitat al model per tal de poder experimentar amb el model amb tota la seva potència matemàtica. 

En la implementació cal afegir alguns operadors que en el model no
feien falta com perquè ja són propis de l'àlgebra de conjunts. Els
operadors principals que s'han d'afegir són els relacionats amb la
creació de conjunts, els quals en els SGBD es coneixen com a
\emph{llenguatge de definició de dades} (DDL, de l'anglès \emph{data
  definition language}).


Implementem els dos models de SGST i SGSTM com a dues biblioteques
diferents: \emph{Pytsms} i \emph{RoundRobinson} respectivament. La
\emph{RoundRobinson}, però, té una forta dependència en la
\emph{Pytsms} de la mateixa manera que hem definit els SGSTM en base
als SGST.


Utilitzem orientació a objectes. Ens permet fer explicita la relació entre la implementació i el model.



Implementem la part essencial, és a dir l'àlgebra definida en els
models lògics. No implementem els complements habituals dels SGBD, els
quals són necessaris en entorns d'explotació, com per exemple gestió
d'usuaris i permisos, còpies de seguretat, llenguatges estàndards de
consulta, etc.




\section{Pytsms}

La biblioteca \emph{Pytsms} implementa un SGST de referència. Així
doncs, seguint el model, els objectes principals són les mesures, les
sèries temporals i les representacions de les sèries temporals.




\begin{tikzpicture}

  %Timeseries
  \umlclass[x=0,y=0] {TimeSeries}{}{}  
  %Realisations 
  \umlclass[x=-3.5,y=-3] {Structure}{}{}
  \umlclass[x=-1.2,y=-3] {OpSet}{}{}
  \umlclass[x=1.2,y=-3] {OpSeq}{}{}
  \umlclass[x=3.5,y=-3] {OpFunc}{}{}
  \umlreal[geometry=|-|]{Structure}{TimeSeries}
  \umlreal[geometry=|-|]{OpSet}{TimeSeries}
  \umlreal[geometry=|-|]{OpSeq}{TimeSeries}
  \umlreal[geometry=|-|]{OpFunc}{TimeSeries}
  %Subrealisations
  \umlclass[x=-3,y=-6] {SetNoTemporal}{}{}
  \umlclass[x=0.7,y=-6] {SetTemporal}{}{}
  \umlclass[x=4,y=-6] {SetRelacional}{}{}
  \umlreal[geometry=|-|]{SetNoTemporal}{OpSet}
  \umlreal[geometry=|-|]{SetTemporal}{OpSet}
  \umlreal[geometry=|-|]{SetRelacional}{OpSet}
  \umlclass[x=-6,y=-6] {Set}{}{}
  \umlinherit{Structure}{Set}


  % Measure
  \umlclass[x=-4] {Measure}{}{}
  \umluniaggreg  {TimeSeries}{Measure}

  %Repr
  \umlclass[x=8,type=abstract] {Representation}{}{}
  \umlassoc  {TimeSeries}{Representation}
  \umlclass[x=7,y=-4] {Zohe}{}{}
  \umlclass[x=9,y=-4] {Discret}{}{}
  \umlinherit {Zohe}{Representation}
  \umlinherit {Discret}{Representation}


  %Associacions
  \umlclass[x=2,y=2] {RegularProp}{}{}
  \umluniassoc  {RegularProp}{TimeSeries}
  \umlclass[x=5,y=2] {Storage}{}{}
  \umluniassoc {Storage}{TimeSeries}

\end{tikzpicture}










Les sèries temporals i les representacions són ortogonals i això
s'implementa mitjançant una associació entre ambdós. Cada usuari ha de
definir la representació que vol associar, com a exemple se
n'implementen algunes. 




El concepte de sèrie temporal s'ha implementat com un objecte
\emph{TimeSeries}.  Una sèrie temporal és un objecte amb una gran
quantitat de mètodes. Com a conseqüència s'ha dividit la seva
implementació en diversos mòduls. Els mètodes que implementen el model
estructural i el model d'operacions bàsiques s'han agrupat en objectes
segons la seva funcionalitat. Així hi ha l'objecte \emph{Structure}
que implementa el model estructural de les sèries temporals,
l'\emph{OpSet} pel model d'operacions de conjunts, l'\emph{OpSeq} pel
model d'operacions de seqüències i l'\emph{OpFunc} pel model
d'operacions de funció temporal.  Aleshores el \emph{TimeSeries}
hereta les funcionalitats d'aquests quatre objectes.  L'objecte
\emph{OpSet} també té una gran quantitat de mètodes i, de la mateixa
manera, hereta la funcionalitat de tres objectes: el
\emph{SetNoTemporal} per les operacions basades en l'orde parcial de
les sèries temporals, el \emph{SetTemporal} basat en l'ordre temporal
i el \emph{SetRelacional} per les operacions específiques de l'àlgebra
relacional.





La representació és abstracta, sempre ha de ser una de concreta? I la discreta pura?






\section{RoundRobinson}

\begin{tikzpicture}

  %MultiTimeseries
  \umlclass[x=0,y=0] {MultiresolutionSeries}{}{}  
  \umlclass[x=4,y=0] {Set}{}{}
  \umlinherit{MultiresolutionSeries}{Set}
  %Components 
  \umlclass[x=0,y=-3] {Resolution}{}{}
  \umluniaggreg  {MultiresolutionSeries}{Resolution}
  %SubComponents 
  \umlclass[x=-1.2,y=-6] {Buffer}{}{}
  \umlclass[x=1.2,y=-6] {Disc}{}{}
  \umlclass[x=-3,y=-9] {Aggregators}{}{}
  \umlunicompo[mult=1]  {Resolution}{Buffer}
  \umlunicompo[mult=1]  {Resolution}{Disc}
  \umluniassoc[mult=1]  {Buffer}{Aggregators}

  %TimeSeries
  \begin{umlpackage}[x=0,y=-9]{pyTsms}
    \umlclass{TimeSeries}{}{}  
  \end{umlpackage}
  \umluniassoc[mult=1]  {Buffer}{TimeSeries}
  \umluniassoc[mult=1]  {Disc}{TimeSeries}

  %Associacions
  \umlclass[x=5,y=2] {Storage}{}{}
  \umluniassoc {Storage}{MultiresolutionSeries}
  \umlclass[x=2,y=2] {Plot}{}{}
  \umluniassoc {Plot}{MultiresolutionSeries}

\end{tikzpicture}




%%% Local Variables:
%%% TeX-master: "main"
%%% End:
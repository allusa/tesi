%\part{Experimentació}


\chapter{Introducció a les implementacions}


Es realitzen implementacions a alt nivell per a observar el funcionament a nivell acadèmic: Python. Implementem els models de SGST i SGSTM amb Python, aquesta és la nostra implementació de referència, la qual usem per als exemples amb dades.

Es realitzen implementacions a alt nivell per a observar l'encaix amb el model relacional: Tutorial D.

Es realitzen implementacions a baix nivell d'estructures específiques: VHDL o circuit digital.


Es realitzen implementacions específiques per a la resolució offline de la multiresolució com a consulta sobre els SGST: MapReduce


S'avaluen implementacions específiques com RRDtool?

% We implement the TSMS and MTSMS models into three different approaches:

% \begin{itemize}
% \item Pytsms+RoundRobinson, a Python implementation. This is our referent
%   implementation, which we use for example data.
% \item Reltsms, a Tutorial D implementation. This shows a TSMS
%   implemented on a relational language.
% \item Roundrobindoop, a MapReduce implementation. This is a specific
%   implementation for offline computing multiresolution with
%   parallelism approaches.
% \end{itemize}

% RRDtool can also be seen as a MTSMS implementation in a specific
% field. We give more references for RRDtool in
% Section~\ref{sec:related-work}.





%el model que presentem a continuació pertany al nivell lògic
%les implementacions són nivell físic
%no definim nivell d'usuari, el que significaria definir un llenguatge de consultes per al SGST i al SGSTM. No el definim perquè les implementacions que fem són totalment fidedignes al nivell lògic i el nivell d'usuari que se'n deriva és igual que el nivell lògic; només li faltarien les facilitats d'un llenguatge de programació (poder fer bucles, condicionals, declarar variables, etc.) Aquestes facilitats en el nivell lògic no s'expliciten ja que es consideren inherents a les matemàtiques.

















%%% Local Variables:
%%% TeX-master: "main"
%%% End:

%\part{Experimentació}


\chapter{Introducció a les implementacions}


Implementem els models definits de \gls{SGST} i de \gls{SGSTM}


\todo{on oferirem els codis de les implementacions?}
Podem posar un enllaç a l'escriny? 

Idealment, en un \gls{SGBD} l'usuari executa una consulta i
l'optimitzador s'encarrega d'executar les operacions físiques més
eficients. El model relacional permet trobar expressions equivalents
donada una determinada consulta. Això no obstant, i com s'ha detallat
a l'estat actual dels \gls{SGBD} \todo{ref}, una implementació de
\gls{SGBD} no pot abastar i ser eficient en tots els àmbits. Així
doncs, no és tan senzill mantenir totalment la independència entre
l'usuari i la implementació ja que cal que aquest decideixi un
\gls{SGBD} adequat per a cada context i fins i tot que declari com
resoldre algunes operacions. Sí que es pot mantenir la independència
del nivell lògic, respecte a les implementacions; i és en aquest
sentit que a continuació avaluem diferents implementacions per als
models definits de \gls{SGST} i de \gls{SGSTM}, on cada implementació
està pensada per a un context determinat.





Es realitzen implementacions a alt nivell per a observar el funcionament a nivell acadèmic: Python. Implementem els models de \gls{SGST} i \gls{SGSTM} amb Python, aquesta és la nostra implementació de referència, la qual usem per als exemples amb dades.

Es realitzen implementacions a alt nivell per a observar l'encaix amb el model relacional: Tutorial D.

Es realitzen implementacions a baix nivell d'estructures específiques: VHDL o circuit digital.


Es realitzen implementacions específiques per a la resolució offline de la multiresolució com a consulta sobre els \gls{SGST}: MapReduce


S'avaluen implementacions específiques com RRDtool?




% We implement the TSMS and MTSMS models into three different approaches:

% \begin{itemize}
% \item Pytsms+RoundRobinson, a Python implementation. This is our referent
%   implementation, which we use for example data.
% \item Reltsms, a Tutorial D implementation. This shows a TSMS
%   implemented on a relational language.
% \item Roundrobindoop, a MapReduce implementation. This is a specific
%   implementation for offline computing multiresolution with
%   parallelism approaches.
% \end{itemize}

% RRDtool can also be seen as a MTSMS implementation in a specific
% field. We give more references for RRDtool in
% Section~\ref{sec:related-work}.






\section{Particularitats de les implementacions}



Els models d'implementacions pertanyen al nivell físic dels \gls{SGBD}
i són una realització d'un model lògic, en el nostre cas dels models
lògics de \gls{SGST} i de \gls{SGSTM}. Per a les implementacions se
sol definir el nivell d'usuari, que és el llenguatge que serà visible
per als usuaris. Les implementacions que realitzem volen ser molt
properes al model lògic i per tant el nivell d'usuari que se'n deriva
és molt similar. Per a ser un llenguatge d'usuari complet es
requereixen facilitats de llenguatge de programació --bucles,
condicionals, declarar variables, \dots--, per a la qual cosa ens
basem en els recursos particulars de cada implementació: Python,
Tutorial~D, etc.

% Aquestes facilitats en el nivell lògic no s'expliciten ja que es
% consideren inherents a les matemàtiques.



En la implementació s'afegeixen alguns operadors que en el model
estructural no estaven explícitament definits perquè són propis de
l'àlgebra de conjunts. Així, alguns d'aquests operadors que s'han
d'implementar són els relacionats amb la notació de creació de
conjunts, %set-builder notation (set comprehension)
els quals en els \gls{SGBD} s'inclouen en el que es coneix com a
\emph{llenguatge de definició de dades} (DDL, de l'anglès \emph{data
  definition language}), o els relacionats amb la manipulació de les
dades amb assignació, inserció, modificació o esborrament, els quals
en els SGBD es coneixen com a \emph{llenguatge de manipulació de
  dades} (DML, de l'anglès \emph{data manipulation language}) .  En el
cas del model de \gls{SGSTM} sí que hem definit operacions de DDL i DML per
a l'esquema multiresolució ja que aquest requereix ser manipulat
coherentment.

Pel que fa al llenguatge de consultes (\emph{query language}),
s'implementa seguint el model d'operacions de consulta de cada cas.












%%% Local Variables:
%%% TeX-master: "main"
%%% End:

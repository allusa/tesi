%\part{Experimentació}


\chapter{Introducció a les implementacions}


Implementem els models definits de \gls{SGST} i de \gls{SGSTM}


Es realitzen implementacions a alt nivell per a observar el funcionament a nivell acadèmic: Python. Implementem els models de \gls{SGST} i \gls{SGSTM} amb Python, aquesta és la nostra implementació de referència, la qual usem per als exemples amb dades.

Es realitzen implementacions a alt nivell per a observar l'encaix amb el model relacional: Tutorial D.

Es realitzen implementacions a baix nivell d'estructures específiques: VHDL o circuit digital.


Es realitzen implementacions específiques per a la resolució offline de la multiresolució com a consulta sobre els \gls{SGST}: MapReduce


S'avaluen implementacions específiques com RRDtool?

% We implement the TSMS and MTSMS models into three different approaches:

% \begin{itemize}
% \item Pytsms+RoundRobinson, a Python implementation. This is our referent
%   implementation, which we use for example data.
% \item Reltsms, a Tutorial D implementation. This shows a TSMS
%   implemented on a relational language.
% \item Roundrobindoop, a MapReduce implementation. This is a specific
%   implementation for offline computing multiresolution with
%   parallelism approaches.
% \end{itemize}

% RRDtool can also be seen as a MTSMS implementation in a specific
% field. We give more references for RRDtool in
% Section~\ref{sec:related-work}.





%el model que presentem a continuació pertany al nivell lògic
%les implementacions són nivell físic
%no definim nivell d'usuari, el que significaria definir un llenguatge de consultes per al SGST i al SGSTM. No el definim perquè les implementacions que fem són totalment fidedignes al nivell lògic i el nivell d'usuari que se'n deriva és igual que el nivell lògic; només li faltarien les facilitats d'un llenguatge de programació (poder fer bucles, condicionals, declarar variables, etc.) Aquestes facilitats en el nivell lògic no s'expliciten ja que es consideren inherents a les matemàtiques.





\section{Particularitats de les implementacions}


En la implementació s'afegeixen alguns operadors que en el model
estructural no estaven explícitament definits perquè són propis de
l'àlgebra de conjunts. Així, alguns d'aquests operadors que s'han
d'implementar són els relacionats amb la notació de creació de
conjunts, %set-builder notation (set comprehension)
els quals en els \gls{SGBD} s'inclouen en el que es coneix com a
\emph{llenguatge de definició de dades} (DDL, de l'anglès \emph{data
  definition language}), o els relacionats amb la manipulació de les
dades amb assignació, inserció, modificació o esborrament, els quals
en els SGBD es coneixen com a \emph{llenguatge de manipulació de
  dades} (DML, de l'anglès \emph{data manipulation language}) .  En el
cas del model de \gls{SGSTM} sí que hem definit operacions de DDL i DML per
a l'esquema multiresolució ja que aquest requereix ser manipulat
coherentment.

Pel que fa al llenguatge de consultes (\emph{query language}),
s'implementa seguint el model d'operacions de consulta de cada cas.



% El model s'ha descrit utilitzant àlgebra de conjunts. Així moltes operacions no s'han hagut de definir perquè formen part de la notació de conjunts: definició de nous conjunts (set builder notation), operació d'assignació, manipulació de les dades amb inserció, modificació, esborrar (perquè en àlgebra de conjunts es treballa amb immutabilitat ja que a l'àlgebra crear un nou conjunt no té cap cost), etc.  A les implementacions, però, totes aquestes operacions s'han de definir en cas que es vulguin.










%%% Local Variables:
%%% TeX-master: "main"
%%% End:

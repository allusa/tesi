\chapter{Implementació relacional}





\section{SGST relacional}



\begin{verbatim}
VAR timeseries BASE RELATION
    { t RATIONAL, v RATIONAL }  KEY { t } ;

OPERATOR ts.t(m SAME_TYPE_AS  (timeseries)) RETURNS RATIONAL;
return t FROM TUPLE FROM m;
END OPERATOR;

OPERATOR ts.v(m SAME_TYPE_AS  (timeseries)) RETURNS RATIONAL;
return v FROM TUPLE FROM m;
END OPERATOR;
\end{verbatim}



\subsection{Màxim i suprem}

Tutorial D:
\begin{verbatim}
OPERATOR ts.max(s1 SAME_TYPE_AS  (timeseries)) RETURNS RELATION SAME_HEADING_AS  (timeseries);
return s1 JOIN ( SUMMARIZE s1 {t} PER (s1 {}) ADD (MAX (t) AS t));
END OPERATOR;

OPERATOR ts.sup(s1 SAME_TYPE_AS  (timeseries)) RETURNS RELATION SAME_HEADING_AS  (timeseries);
return ts.max(ts.union(s1,(RELATION { TUPLE {t -1.0/0.0, v 1.0/0.0} })));
END OPERATOR;
\end{verbatim}



Exemple:
\begin{verbatim}
WITH RELATION {
TUPLE { t 2.0, v 3.0 },
TUPLE { t 4.0, v 2.0 },
TUPLE { t 6.0, v 4.0 }
 } AS ts1: 
ts.max(ts1)
\end{verbatim}
\begin{verbatim}
RELATION {
TUPLE { t 1.0, v 2.0 },
TUPLE { t 5.0, v 3.0 },
TUPLE { t 6.0, v 5.0 },
TUPLE { t 1.0/0.0, v 1.0 }  //1.0/0.0 infinit
 } AS ts2: 
ts.max(ts2)
\end{verbatim}
\begin{verbatim}
WITH RELATION {
TUPLE { t 2.0, v 3.0 },
TUPLE { t 4.0, v 2.0 },
TUPLE { t 6.0, v 4.0 }
 } AS ts1: 
ts.sup(ts1)
\end{verbatim}
\begin{verbatim}
WITH RELATION {
TUPLE { t 2.0, v 3.0 },
TUPLE { t 4.0, v 2.0 },
TUPLE { t 6.0, v 4.0 }
 } AS ts1: 
ts.sup(timeseries)
\end{verbatim}



\subsection{Unió}

TutorialD:
\begin{verbatim}
OPERATOR ts.union(s1 SAME_TYPE_AS  (timeseries), s2 SAME_TYPE_AS  (timeseries)) RETURNS RELATION SAME_HEADING_AS  (timeseries);
return s1 UNION (s2 JOIN (s2 {t} MINUS s1 {t}));
END OPERATOR;
\end{verbatim}


Exemple:
\begin{verbatim}
WITH RELATION {
TUPLE { t 2.0, v 3.0 },
TUPLE { t 4.0, v 2.0 },
TUPLE { t 6.0, v 4.0 }
 } AS ts1,
RELATION {
TUPLE { t 1.0, v 2.0 },
TUPLE { t 5.0, v 3.0 },
TUPLE { t 6.0, v 5.0 },
TUPLE { t 10.0, v 1.0 }
 } AS ts2: 
ts.union(ts1,ts2)
\end{verbatim}



\subsection{Unió exclusiva}


TutorialD:
\begin{verbatim}
OPERATOR ts.xunion(s1 SAME_TYPE_AS  (timeseries), s2 SAME_TYPE_AS  (timeseries)) RETURNS RELATION SAME_HEADING_AS  (timeseries);
return ts.union(s1,s2) MINUS ts.intersect(s1,s2) ;
END OPERATOR;
\end{verbatim}


Exemple:
\begin{verbatim}
WITH RELATION {
TUPLE { t 2.0, v 3.0 },
TUPLE { t 4.0, v 2.0 },
TUPLE { t 6.0, v 4.0 }
 } AS ts1,
RELATION {
TUPLE { t 1.0, v 2.0 },
TUPLE { t 5.0, v 3.0 },
TUPLE { t 6.0, v 5.0 },
TUPLE { t 10.0, v 1.0 }
 } AS ts2: 
ts.xunion(ts1,ts2)
\end{verbatim}


\subsection{Selecció temporal}



TutorialD:
\begin{verbatim}
OPERATOR ts.temporal.select.zohe(s SAME_TYPE_AS  (timeseries), l RATIONAL, h RATIONAL ) RETURNS RELATION SAME_HEADING_AS  (timeseries);
BEGIN;
VAR x RATIONAL init(0.0);
VAR sp PRIVATE SAME_TYPE_AS ( timeseries) KEY { t };
x := ts.v(ts.inf(s MINUS ts.interval.ni(s,h)));
sp := RELATION {
TUPLE {t h, v x}
};
return ts.union(ts.interval(s,l,h),sp);
END;
END OPERATOR;
\end{verbatim}

Exemple:
\begin{verbatim}
WITH RELATION {
TUPLE { t 2.0, v 3.0 },
TUPLE { t 4.0, v 2.0 },
TUPLE { t 6.0, v 4.0 }
 } AS ts1,
RELATION {
TUPLE { t 1.0, v 2.0 },
TUPLE { t 5.0, v 3.0 },
TUPLE { t 6.0, v 5.0 },
TUPLE { t 10.0, v 1.0 }
 } AS ts2:
ts.temporal.select.zohe(ts1,1.0,5.0)
\end{verbatim}
\begin{verbatim}
WITH RELATION {
TUPLE { t 2.0, v 3.0 },
TUPLE { t 4.0, v 2.0 },
TUPLE { t 6.0, v 4.0 }
 } AS ts1,
RELATION {
TUPLE { t 1.0, v 2.0 },
TUPLE { t 5.0, v 3.0 },
TUPLE { t 6.0, v 5.0 },
TUPLE { t 10.0, v 1.0 }
 } AS ts2:
ts.temporal.select.zohe(ts1,-1.0/0.0,-1.0/0.0)
\end{verbatim}



\subsection{Map i fold}



TutorialD:
\begin{verbatim}
WITH RELATION {
TUPLE { t 2.0, v 3.0 },
TUPLE { t 4.0, v 2.0 },
TUPLE { t 6.0, v 4.0 }
} AS ts1: 
ts.map(ts1,'t','t*v/2.0')
\end{verbatim}

TutorialD:
\begin{verbatim}
WITH RELATION {
TUPLE { t 2.0, v 3.0 },
TUPLE { t 4.0, v 2.0 },
TUPLE { t 6.0, v 4.0 }
 } AS ts1,
RELATION {
TUPLE { t 0.0, v 0.0}
} AS mi: 
ts.fold(ts1,mi,'t','v+vi')
\end{verbatim}

antM TutorialD:
\begin{verbatim}
WITH RELATION {
TUPLE { t 2.0, v 3.0 },
TUPLE { t 4.0, v 2.0 },
TUPLE { t 6.0, v 4.0 }
 } AS ts1,
RELATION {
TUPLE { t 5.0, v 0.0}
} AS m: 
ts.sup(ts1 WHERE t < ts.t(m))
\end{verbatim}

sup fold TutorialD:
\begin{verbatim}
WITH RELATION {
TUPLE { t 2.0, v 3.0 },
TUPLE { t 4.0, v 2.0 },
TUPLE { t 6.0, v 4.0 }
 } AS ts1,
RELATION {
TUPLE { t -1.0/0.0, v 1.0/0.0}
} AS mi: 
ts.fold(ts1,mi,'max {t,ti}','v FROM TUPLE FROM (RELATION { TUPLE {t t, v v, e True}, TUPLE {t ti, v vi,  e False} } JOIN RELATION { TUPLE {e t > ti}})')
\end{verbatim}

predecessors mapfold TutorialD:
\begin{verbatim}
WITH RELATION {
TUPLE { t 2.0, v 3.0 },
TUPLE { t 4.0, v 2.0 },
TUPLE { t 6.0, v 4.0 }
 } AS ts1,
EXTEND ts1 {t} ADD ( -1.0/0.0 as v)
AS si: 
ts.fold(ts1,si,'t','v FROM TUPLE FROM (RELATION { TUPLE {v ti, e True}, TUPLE {v v,  e False} } JOIN RELATION { TUPLE {e v < ti and ti<t}})')
\end{verbatim}


\section{SGSTM relacional}














%%% Local Variables:
%%% TeX-master: "main"
%%% End:

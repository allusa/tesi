%ABREVIACIONS (abreviatures,sigles,símbols)


%ABREVIATURES


%ús
% \gls{vegeu} %= v.
% \acrshort{vegeu} %= v.
% \acrlong{vegeu} %= vegeu
% \acrfull{vegeu} %= vegeu (v.)
% \glsentryshort{vegeu} %=v.

%angl. anglès
%fr. francès


%v. vegeu
%\S secció
%s. v. sub voce ('sota l'entrada', en els diccionaris)


\newglossaryentry{angl}{type=abreviatura, name={angl.}, description={anglès},short={angl.},long={anglès}}
\newglossaryentry{fr}{type=abreviatura, name={fr.}, description={francès},short={fr.},long={francès}}


\newglossaryentry{perexemple}{type=abreviatura, name={p.ex.}, description={per exemple},short={p.ex.},long={per exemple}}



\newglossaryentry{seccio}{type=abreviatura, name={\S}, description={secció o paràgraf},short={\S},long={secció},sort=S}
\newglossaryentry{capitol}{type=abreviatura, name={cap.}, description={capítol},short={cap.},long={capítol}}
\newglossaryentry{apendix}{type=abreviatura, name={ap.}, description={apèndix},short={ap.},long={apèndix}}

\newglossaryentry{subvoce}{type=abreviatura, name={s.~v.}, description={sub voce, `sota l'entrada' en els diccionaris}, short={s~.v.},long={sub voce}}

\newglossaryentry{vegeu}{type=abreviatura, name={v.}, description={vegeu}, short={v.},long={vegeu}}








%SIGLES
%inclou acrònims
\newcommand{\acro}[1]{\textsc{\lowercase{#1}}}



%català

\newglossaryentry{SGBD}{type=\acronymtype, name={SGBD}, description={sistema de gestió de base de dades (\emph{Data Base Management System})}, text={\acro{SGBD}}, first={sistema de gestió de base de dades (\acro{SGBD})}, plural={\acro{SGBD}}, firstplural={sistemes de gestió de bases de dades (\acro{SGBD})}}

\newglossaryentry{SGBDR}{type=\acronymtype, name={SGBDR}, description={sistema de gestió de base de dades relacional (\emph{Relational Data Base Management System})}, text={\acro{SGBDR}}, first={sistema de gestió de base de dades relacional (\acro{SGBDR})}, plural={\acro{SGBDR}}, firstplural={sistemes de gestió de bases de dades relacionals (\acro{SGBDR})}}

\newglossaryentry{SQL}{type=\acronymtype, name={SQL}, description={\emph{Structured Query Language}, llenguatge habitual de consulta en els \gls{SGBDR} }, text={\acro{SQL}}, first={  \emph{Structured Query Language} (\acro{SQL})}}

\newglossaryentry{DDL}{type=\acronymtype, name={DDL}, description={llenguatge de definició de dades (\acro{DDL}, de l'\gls{angl}~\emph{data definition language})}, text={\acro{DDL}}, first={llenguatge de definició de dades (\acro{DDL}, de l'\gls{angl}~\emph{data definition language})}}

\newglossaryentry{DML}{type=\acronymtype, name={DML}, description={llenguatge de manipulació de dades (\acro{DML}, de l'\gls{angl}~\emph{data manipulation language})}, text={\acro{DML}}, first={llenguatge de manipulació de dades (\acro{DML}, de l'\gls{angl}~\emph{data manipulation language})}}


\newglossaryentry{SGST}{type=\acronymtype, name={SGST}, description={sistema de gestió de base de dades per a sèries temporals (\emph{Time Series Data Base Management System})}, text={\acro{SGST}}, first={sistema de gestió de base de dades per a sèries temporals (\acro{SGST})},firstplural={sistemes de gestió de bases de dades per a sèries temporals (\acro{SGST})}, plural={\acro{SGST}}}

\newglossaryentry{SGSTM}{type=\acronymtype, name={SGSTM}, description={sistema de gestió de base de dades per a sèries temporals multiresolució (\emph{Multiresolution Time Series Data Base Management System})}, text={\acro{SGSTM}}, first={sistema de gestió de base de dades per a sèries temporals multiresolució (\acro{SGSTM})},firstplural={sistemes de gestió de bases de dades per a sèries temporals multiresolució (\acro{SGSTM})}, plural={\acro{SGSTM}}}

\newglossaryentry{BDSTM}{type=\acronymtype, name={BDSTM}, description={base de dades per a sèries temporals multiresolució (\emph{Multiresolution Time Series Data Base})}, text={\acro{BDSTM}}, first={base de dades per a sèries temporals multiresolució (\acro{BDSTM})},firstplural={bases de dades per a sèries temporals multiresolució (\acro{BDSTM})}, plural={\acro{BDSTM}}}

\newglossaryentry{VLDB}{type=\acronymtype, name={VLDB}, description={\emph{very large databases}}, text={\acro{VLDB}}, first={\emph{very large databases} (\acro{VLDB})}}



\newglossaryentry{CSV}{type=\acronymtype, name={CSV}, description={format de fitxer de text que emmagatzema els valors separats per comes (de l'\gls{angl}~\emph{comma-separated values})}, text={\acro{CSV}}, first={valors separats per comes (\acro{CSV}, de l'\gls{angl}~\emph{comma-separated values})}}


\newglossaryentry{HDFS}{type=\acronymtype, name={HDFS}, description={\emph{Hadoop Distributed File System} \parencite{hadoop}}, text={\acro{HDFS}}, first={\emph{Hadoop Distributed File System} (\acro{HDFS})}}



\newglossaryentry{PLR}{type=\acronymtype, name={PLR}, description={\emph{Piecewise Linear Representation} \cite{last:keogh}}, text={\acro{PLR}}, first={\emph{Piecewise Linear Representation}} (\acro{PLR})}




\newglossaryentry{SCADA}{type=\acronymtype, name={SCADA}, description={sistema de supervisió, control i adquisició (de l'\gls{angl}~\emph{Supervisory Control And Data Acquisition})}, text={\acro{SCADA}}, first={sistema de supervisió, control i adquisició (\acro{SCADA}, de l'\gls{angl}~\emph{Supervisory Control And Data Acquisition})},firstplural={sistemes de supervisió, control i adquisició (\acro{SCADA}, de l'\gls{angl}~\emph{Supervisory Control And Data Acquisition})},plural={\acro{SCADA}}}




\newglossaryentry{SIG}{type=\acronymtype, name={SIG}, description={sistema d'informació geogràfica}, text={\acro{SIG}}, first={sistema d'informació geogràfica (\acro{SIG})},firstplural={sistemes d'informació geogràfica (\acro{SIG})}, plural={\acro{SIG}}}





\newglossaryentry{UML}{type=\acronymtype, name={UML}, description={Llenguatge unificat de modelització (de l'\gls{angl}~\emph{Unified Modeling Language})}, text={\acro{UML}}, first={Llenguatge unificat de modelització (\acro{UML}, de l'\gls{angl}~\emph{Unified Modeling Language})}}

\newglossaryentry{XML}{type=\acronymtype, name={XML}, description={\emph{Extensible Markup Language}}, text={\acro{XML}}, first={\emph{Extensible Markup Language}} (\acro{XML})}




\newglossaryentry{UTC}{type=\acronymtype, name={UTC}, description={temps universal coordinat (del \gls{fr}~\emph{temps universel coordonné} i de l'\gls{angl}~\emph{Coordinated Universal Time})}, text={\acro{UTC}}, first={temps universal coordinat (\acro{UTC}, del \gls{fr}~\emph{temps universel coordonné} i de l'\gls{angl}~\emph{Coordinated Universal Time})}}

\newglossaryentry{TAI}{type=\acronymtype, name={TAI}, description={temps atòmic internacional (del \gls{fr}~\emph{temps atomique international})}, text={\acro{TAI}}, first={temps atòmic internacional (\acro{TAI}, del \gls{fr}~\emph{temps atomique international})}}



%sigles mètodes de representació
\newglossaryentry{pd}{type=\acronymtype, name={PD}, description={mètode de representació parcial discreta}, text={\acro{PD}}}
\newglossaryentry{dd}{type=\acronymtype, name={DD}, description={mètode de representació delta de Dirac}, text={\acro{DD}}}
\newglossaryentry{zoh}{type=\acronymtype, name={ZOH}, description={mètode de representació zero-order hold}, text={\acro{ZOH}}}
\newglossaryentry{zohe}{type=\acronymtype, name={ZOHE}, description={mètode de representació zero-order hold cap enrere}, text={\acro{ZOHE}}}
\newglossaryentry{foh}{type=\acronymtype, name={FOH}, description={mètode de representació first-order hold}, text={\acro{FOH}}}




%anglès
\newacronym[see=SGBD]{DBMS}{DBMS}{\nopostdesc}
\newacronym[see=SGBDR]{RDBMS}{RDBMS}{\nopostdesc}

\newacronym[see=SGSTM]{MTSMS}{MTSMS}{Multiresolution Time Series Data Base management system}
\newacronym[see=SGST]{TSMS}{TSMS}{Time Series Data Base management system}





% %acrònims
% \newacronym{relvar}{relvar}{relation variable}%generally, relation will mean relation value
% \newacronym{relcon}{relcon}{relation constant}% a relvar that never changes relation value


%SÍMBOLS

%inclou codis i xifres





%%% Local Variables: 
%%% mode: latex
%%% TeX-master: "../main"
%%% End: 

\chapter{Estat actual} 
\label{cap:estat}


\todo{resum del capítol}

%Base teòrica i avantguarda

En aquest capítol resumim la base teòrica per als models que proposem i l'avantguarda de l'estat de la qüestió. 


1. Sèries temporals

2. \gls{SGBD}

3. Sistemes i projectes similars





%Estat de l'art

% * no n'hi ha d'específic del tema, potser el que més s'hi assembla són els SGST que hi ha (Cougar, RRDtool, ...)

% * Hi ha temes colaterals (monitoratge,anàlisis)

% * Temes para\l.lels que ens serveixen d'inspiració (SGBD relacionals)

%Cal introduir bé el forat de coneixement que hi ha en els SGST. Forat entre les sèries temporals i els SGBD.



%Capítol:

% * Sèries temporals
  
%   - mineria
%   - aplicacions
%   - monitoratge de sèries temporals i problemes
%      * censura
%      * mostreig

%   - sgst: 
%       ficar aquí els sgbd per sèries temporals i més endavant ja es parlarà dels sgbd en general i com modelar-los i implementar-los.


% * SGBD
%  - model relacional
%  - implementacions
%  - temporal data


  % * Sèries temporals (històrics, predicció, diagnosis, prognosis, etc.)
  % * Mostreig: docs quan període de mostreig no regular
  % * Bases de dades (docs d'emmagatzematge quan la memòria és finita, docs quan període de mostreig no és regular, altres sistemes semblants (comercials,prototips))




% El capítol comença resumint l'estat de les sèries temporals en aquest camp de mineria; és a dir d'emmagatzematge i tractament. A continuació es llisten algunes aplicacions informàtiques que han implementat models de la mineria de sèries temporals. Finalment, es descriu l'estat actual de l'aplicació RRDtool, la qual també es classifica en aquest camp.

% This paper focuses on Data Base Management Systems (DBMS) that store
% and treat data as time series.   Other DBMS are not adequate for these cases as they do not have enough facilities to manage and retrieve time series
% information \parencite{schmidt95}.

% DBMS are based from formal models that define the objects and
% operations of the abstract machine to which users interact, such is
% the relational model \parencite{date}. TSMS lack a consolidated formal
% model, although special properties and requirements for a TSMS
% have been proposed \parencite{dreyer94}.








%%% Local Variables: 
%%% mode: latex
%%% TeX-master: "main"
%%% End: 

% LocalWords:  monitoratge

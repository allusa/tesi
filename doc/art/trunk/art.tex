\chapter{Estat actual}



\section{Sèries temporals}


Important! Diferència entre dades temporals i sèries temporals.




\section{SGBD}

Els sistemes de gestió de bases de dades (SGBD) 



L'any tal Codd va proposar un model relacional pels SGBD (SGBDR) que va revolucionar aquest àmbit. A partir de llavors els SGBDR han anat evolucionat, amb Date com a principal divulgador, fins a tenir una gran solidesa.

Avui dia, no hi cap implementació comercial que segueixi fidelment el model relacional \todo{citar date}. Això ha contribuït a que per una banda hi hagi hagut una sèrie de malentesos i errors que Date s'ha encarregat de desmentir en nombroses publicacions \todo{citar, sobretot ddbunk} i per altra banda també ha contribuït en gran mesura a voler explorar altres models.              


Alguns SQL squirks (the possibility of nulls, the possibility of duplicate rows, the
fact of left-to-right column ordering, and so on)

Alguns malentesos també es deuen en l'ús de la mateixa terminologia amb significat diferent entre el model relacional i l'orientació a objectes, sobretot pel que fa als termes valor i variable. 

Relacional: tipus | representació |  valor, objecte, instància  | variable  | referència (adreça continguda en una variable) | operadors
Objectes: tipus, classe (tipus amb atributs i mètodes) | atributs,propietats  |  valor, estat, objecte/instància immutable |  objecte/instància mutable  | variable | funcions,mètodes (funcions dins de classes)

Hi han hagut algunes propostes per apartar-se del model relacional, com pot ser recuperant models antics de xarxa (Xquery) o bé utilitzant el paradigma de programació d'objectes \todo{cites}, tot i que Date i altres consideren que no estan ben fundades i cap d'aquestes té la mateixa potència que el model relacional. Actualment s'està treballant en una implementació fidel al model relacional \todo{citar}. 

Date i altres consideren la possibilitat, tot i que remota, que es pugui definir un model més potent que el relacional però que no hi ha cap indici que cap definició ho estigui aconseguint ni de fet explorant. Per tant, aconsellen que cap aplicació de gestió de bases de dades no s'allunyi del model relacional.


El model relacional es basa en la lògica i la teoria de conjunts.





Com s'ha d'estendre el model relacional?

Date i altres consideren que el model relacional va evolucionat i no consideren que hi hagi hagut cap revolució des de la seva aparició. (Tot i que han definit una nova àlgebra relacional anomenada 'A' que fa molt bona pinta). Consideren que el model relacional està bastant completat i l'evoluciona en petites modificacions de comprensió. 

Només hi ha hagut un àmbit en el que han admès que s'havia d'estendre el model relacional i aquest és el de les dades temporals \todo{citar}.


Així doncs no hi ha necessitat d'estendre el model relacional. Sí que hi ha necessitat, però, de tenir disponibles diversos tipus de dades. La creació de nous tipus de dades ha estat un tema de molta confusió ja que les 'suposades' implementacions del model relacional no ho han permès i sobretot ha fet que les implementacions basades en orientació a objectes tinguessin un gran impacte \todo{citar ODMG (object databases)}. Date ha rebutjat aquesta idea argumentant que el model relacional mai no ha fixat els tipus de dades disponibles sinó que ben al contrari mai n'ha parlat i per tant els tipus de dades han tingut total llibertat de creació; si bé el model relacional defineix que com a mínim hi ha d'haver el tipus booleà i el tipus relació.


Així doncs, tenint en compte que segons Date el model relacional és complet, que no hi ha cap de tant potent i que l'ampliació de funcionalitat dels SGBDR s'ha de fer mitjançant la creació de nous tipus, el SGST hauria de contemplar aquestes idees. 

Primer s'hauria de veure que el cas de les sèries temporals no sigui com el cas de les dades temporals a on sí que s'ha necessitat estendre el model relacional.

Segon, s'hauria de considerar que el SGST són un nou tipus de dades en el model relacional i per tant el model relacional ja té tota la potència per una banda constituir SGBD i per altra banda definir i incorporar nous tipus.


Com es defineixen nous tipus complexos als SGBD?

En el cas que es descarti que el cas dels SGST presenta els mateixos problemes que les dades temporals i per tant els SGST han d'esdevenir un tipus de dades, cal preguntar-se com són els tipus de dades al model relacional.

Concretament el model relacional només defineix què es un tipus de dades però dóna llibertat a la seva creació. Això és un gran avantatge. En els cas de tipus de dades senzills es defineixen amb una bona estructura i ja està però què passa quan es vol definir un nou tipus de dades complex?

Cal recercar com s'han definit nous tipus de dades complexos. Els principals problemes es donen que el tipus és complex i forma una entitat de per sí. És a dir que definir un tipus sèrie temporal no és trivial. Aleshores, com cal procedir?



Sobre la necessitat de modelar els tipus.

Cal definir un model pel tipus que volem dissenyar.
De fet, volem dissenyar un SGST. És a dir, un tipus que conformi pròpiament un SGBD. Per tant, utilitzarem les mateixes eines que es fan servir per modelar els SGBDR per a poder modelar el nostre SGST. Com que s'hauran utilitzat les mateixes eines, el SGST podrà esdevenir perfectament un tipus de dades pels SGBDR.

En resum, per a definir nous tipus complexes cal modelar-los com a entitat pròpia, fent un símil amb el model relacional. Això no vol dir, però, que un cop modelats constitueixin de per sí un nou model per als SGBD, sinó que queden dins dels SGBDR. \todo{caldria} elaborar més i trobar alguna pista de com definir nous tipus complexos, stonebraker86 només indica com es va estendre postgresql amb operadors de creació de nous tipus però no com s'han de modelar els nous tipus.

Volem crear un nou model de SGBD, Date ens diu que només hi ha el model relacional, per tant hem d'utilitzar el model relacional per a definir el nostre nou model.











%%% Local Variables: 
%%% mode: latex
%%% TeX-master: "main"
%%% End: 

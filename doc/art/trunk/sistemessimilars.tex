
\section{Implementacions actuals}

Hi ha hagut vàries implementacions de sistemes específics per a sèries
temporals. Algunes són només l'aplicació d'un algoritme d'anàlisi per
un problema concret de sèries temporals però altres són més elaborades
i es defineixen com a SGBD per a sèries temporals.  En aquest apartat
resumim algunes aplicacions que considerem que implementem conceptes
dels SGST.



\begin{description}

\item[Calanda] \textcite{dreyer94} proposen els requeriments de propòsit específic que han de complir els SGST i basen el model en quatre elements estructurals bàsics: esdeveniments, sèries temporals, grups i metadades, a banda de les bases de dades per sèries temporals. Implementen un SGST anomenat Calanda \parencite{dreyer94b,dreyer95,dreyer95b} que té operacions de calendari, pot agrupar sèries temporals i respondre consultes simples i ho exemplifiquen amb dades econòmiques. A \cite{schmidt95} es compara Calanda amb els SGBD temporals que operen amb sèries temporals. 




\item[T-Time] \textcite{assfalg08:thesis} mostra un sistema que pot cercar similituds calculades com a distàncies entre sèries temporals. Principalment, dues sèries temporals es marquen com a similars si la seva distància és menor que un llindar en cada interval. A partir d'aquest mètode dissenya algoritmes eficients que implementa en un programa anomenat T-Time \parencite{assfalg08:ttime}.


 
\item[iSAX] \textcite{keogh08:isax,keogh10:isax} estudien l'anàlisi i l'indexat de co\l.lecions massives de sèries temporals. Descriuen que el problema principal del tractament rau en l'indexat de les sèries temporals i proposen mètodes per calcular-lo de manera eficient. El mètode principal que proposen està basat en l'aproximació a trossos constants de la sèrie temporal \parencite{keogh00}.  Ho implementen en una estructura de gestió de dades que anomenen \emph{indexable Symbolic Aggregate approXimation} (iSAX) \parencite{isax}. Les representacions de sèries temporals que s'obtenen amb aquesta eina permeten reduir l'espai emmagatzemat i indexar tant bé com altres mètodes de representació més complexos.




\item[TSDS]
\textcite{weigel10} noten la necessitat de mostrar les dades en tot el seu rang temporal i no només en un subconjunt com normalment s'ofereixen. Desenvolupen el paquet informàtic \emph{Time Series Data Server} (TSDS) \parencite{tsds} a on es poden introduir les dades de sèries temporals per posteriorment consultar-les per rangs temporals o aplicant-hi filtres i operacions.





\item[RRDtool]
RRDtool \parencite{rrdtool} {é}s un SGBD molt usat per la comunitat de programari lliure. Projectes en diversos camps l'utilitzen com a SGBD, en els quals hi ha sistemes de monitoratge professionals, també en l'àmbit de programari lliure, com Nagios/Icinga \parencite{nagios,icinga} o el Multi Router Traffic Grapher (MRTG) \parencite{mrtg}. Aquests monitors transfereixen a RRDtool la responsabilitat de gestionar l'emmagatzematge i d'operar amb les dades, i així es poden centrar en l'adquisició de dades i la gestió d'alarmes. 
En l'evolució de RRDtool hi ha dues millores destacables. En primer lloc, \textcite{lisa98:oetiker} va separar el sistema de gestió de RRDtool de MRTG i el va dissenyar amb una estructura característica de Round Robin. En segon lloc,  \textcite{lisa00:brutlag} va estendre RRDtool amb algoritmes de predicció i detecció de comportaments aberrants. 

Actualment, s'està estudiant l'eficiència i rapidesa de RRDtool a processar sèries temporals. \textcite{carder:rrdcached} ha dissenyat una aplicació, \emph{rrdcached}, que millora el rendiment de RRDtool amb la qual s'aconsegueix fer funcionar  simultàniament sistemes amb grans quantitats de bases de dades RRDtool \parencite{lisa07:plonka}. \textcite{jrobin} han dissenyat una adaptació de RRDtool anomenada \emph{JRobin}. 
Finalment, és destacable l'ús emergent de RRDtool en entorns d'experimentació, com és el cas de \textcite{zhang07} i \textcite{chilingaryan10} que hi emmagatzemen dades experimentals per posteriorment predir o validar-les.


\item[Cougar]
\textcite{cougar,fung02} proposen Cougar com un SGBD per xarxes de sensors (\emph{sensor database systems}). El sistema té dues estructures \parencite{bonnet01}: una basada en relacions per les característiques dels sensors i una basada en seqüències per les dades dels sensors, les quals són sèries temporals.
Les consultes es processen de manera distribuïda: cada sensor és un node amb capacitat de processament que pot resoldre una part de la consulta i fusionar-la amb les altres. D'aquesta manera es minimitza l'ús de comunicacions però l'estructura i estratègia de comunicació dels nodes esdevé una part crítica a configurar \parencite{demers03}.

\item[TinyDB]
Un altre prototip de SGBD per xarxes de sensors desenvolupat para\l.lelament a Cougar és TinyDB \parencite{tinyDB,madden05}. A part de les característiques descrites per Cougar, aquest sistema  modifica i s'implica en parts del procés d'adquisició de les dades com és el temps, la freqüència o l'ordre de mostreig. Per exemple donada una consulta que vol correlacionar les dades de dos sensors, el sistema indica als sensors implicats que han d'adquirir amb la mateixa freqüència.

\item[SciDB]
\textcite{stonebraker09:scidb} estudien els SGBD científiques amb models  de dades basats en matrius. Estan desenvolupant SciDB \parencite{scidb}, un SGBD productiu i optimitzat per treballar amb matrius.


\item[SciQL]
\textcite{kersten11} descriuen SciQL, un llenguatge per a SGBD científiques basades en matrius. Hi ha un prototip en desenvolupament de SciQL \parencite{sciql}.


\end{description}


%SETL http://setl.org/setl/ un llenguatge de programació d'alt nivell que té els conjunts i els mapes de primer ordre com a parts fonamentals. Els tipus bàsics són conjunts, conjunts desordenats i seqüències (també anomenades tuples). Els mapes són conjunts de parelles (tuples de mida dos). Les operacions bàsiques inclouen la pertinença, la unió, la intersecció, etc.


\todo{}


OpenTSDB \cite{deri12:tsdb_compressed_database}
\url{http://opentsdb.net/}
Han fet anàlisis del rendiment de RRDtool, MySQL i la seva implementació TSDB i conclouen que RRDtool és el que pitjor funciona per a sèries temporals. La seva implementació, TSDB, es basa en la compressió de dades. Assumeixen que les sèries temporals són regulars i totes tenen el mateix patró de mostreig, fet que els permet implementar les gestió de les sèries temporals de manera més senzilla.
\todo{Potser hauria de classificar els SGST segons si es basen en compressió de dades (aquest tsdb, iSAX..), en Round Robin (RRDtool), vectors (SciQL), SQL (?) ...}




\url{http://pandas.pydata.org/pandas-docs/stable/index.html}

\url{http://pytseries.sourceforge.net/}



http://stackoverflow.com/questions/4814167/storing-time-series-data-relational-or-non



\url{http://2013.nosql-matters.org/bcn/abstracts/#abstract_gianmarco}

Streaming data analysis in real time is becoming the fastest and most efficient way to obtain useful knowledge from what is happening now, allowing organizations to react quickly when problems appear or to detect new trends helping to improve their performance. In this talk, we present SAMOA, an upcoming platform for mining big data streams. SAMOA is a platform for online mining in a cluster/cloud environment. It features a pluggable architecture that allows it to run on several distributed stream processing engines such as S4 and Storm. SAMOA includes algorithms for the most common machine learning tasks such as classification and clustering. 






\todo{} També hi ha molts sistemes propis d'empreses que van lligats
amb els seus productes. Ara bé ofereixen molt poques capacitats de
SGST i les que ofereixen són molt restringides a l'àmbit a on estan
dirigits els productes; és a dir que no són genèrics i són més aviat
controladors del procés d'adquisició. Per exemple Keller
\url{http://www.catsensors.com/ca/productes/varis__software/logger_4x}, permet desar dades cada un cert període amb estructura d'anell (és a dir eliminant les més antigues quan és ple) però només té un anell. A banda permet detectar certs esdeveniments i aleshores canviar el període de mostreig. A banda permet també emmatgazemar alguns estadístics de les dades: mitjana i rang cada certs segons.



\subsection{Conclusió}

Els SGST actuals bàsicament resolen alguns problemes d'anàlisis de sèries temporals.
Però no solen atendre la relació entre la base de dades i el sistema de monitoratge, és a dir la manera com s'adquireixen les dades. En aquest pas intermig hi ha un sèrie de problemes, com per exemple forats, dades falses o irregularitat en els temps de mostreig, que cal gestionar correctament. Concretament un dels problemes que no s'atén és el de mostreig irregular ja que es considera que les mostres estan a intervals regulars (equi-espaiades) encara que els sistemes de monitoratge informàtics sovint no són capaços de complir-ho amb exactitud sinó que presenten una certa variació en els temps de mesura. 

RRDtool n'és una excepció ja que, per ser un sistema productiu, el processament de dades i emmagatzematge és més proper als sistemes de monitoratge. No obstant, està centrat en un tipus de dades particulars, les magnituds i els comptadors, i no té tantes operacions generals per les sèries temporals com els altres SGST.

També Cougar i TinyDB que exploren l'encaix dels SGBD en entorns distribuïts de xarxes de sensors. Proposen noves estratègies de comunicació amb l'objectiu d'ajustar el consum d'energia. 


SciQL, un model recent per SGBD  basat en matrius, és el que més es pot considerar com a SGST, ja que s'està desenvolupant per complir-ne algunes propietats.










%%% Local Variables: 
%%% mode: latex
%%% TeX-master: "main"
%%% End: 



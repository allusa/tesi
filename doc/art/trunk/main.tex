%--------------------
% document principal
%--------------------
% cal compilar amb `pdflatex -shell-escape main.tex`
%--------------------
\documentclass[paper=a4,fontsize=12pt,twoside,parskip=half,BCOR12mm]{scrbook}
%%%%BCOR12mm  factor de correcció per enquadernació

%------------- capçalera ----------------------
\input{capçalera.default}
\graphicspath{{imatges/}} %{{imatges/}{imatges/model/}}
%---------- Mode esborrany --------------------
%\includeonly{resum}
\usepackage[catalan]{todonotes} %%ús: \todo{text} \missingfigure{text}
\usepackage{fancyhdr}\pagestyle{fancyplain}\chead{\fancyplain{--- esborrany \today\ ---}{\footnotesize\today}}
%%\renewcommand{\headrulewidth}{0pt}
%----------------------------------------------


%------------- format -------------------------
%%ús coma decimal sense espais:  2{,}5

\DeclareMathOperator*{\seg}{seg}
\DeclareMathOperator*{\ant}{ant}

%anotacions a la bibliografia
% \newboolean{bbx@annotation}% (same as biblatex-dw)
% \DeclareBibliographyOption{annotation}[true]{%
% \setboolean{bbx@annotation}{#1}}

\renewbibmacro{finentry}{%
\finentry%
\iffieldundef{annotation}%
{}%
{%\ifbool{bbx@annotation}%
{\color{blue}
\begin{quotation}\noindent%
\printfield{annotation}%
\end{quotation}}%
{}}%
}


%cites bibliogràfiques: números de citacions entre comes
%http://tex.stackexchange.com/questions/26401/help-to-develop-a-textcite-command-to-be-used-with-verbose-citation-styles-in-b
%http://tex.stackexchange.com/questions/28461/biblatex-tighter-integration-of-textcite-in-the-flow-of-text


\makeatletter

\newcommand{\citacomes}[4]{
\blx@addpunct{comma}\space\cite[#1][#2]{#3}%
%\blx@imc@ifpunctmark{#4}{\blx@addpunct{comma}\space#4}{#4}%
\blx@imc@ifnumerals{#4}{%per separar accents
\blx@imc@ifpunctmark{#4}{\blx@addpunct{comma}\space#4}{#4}}%
{\blx@addpunct{comma}\space#4}%
}

\renewrobustcmd*{\textcite}{\blx@citeargs\cbx@textcite}
\newcommand{\cbx@textcite}[4]{%
\citeauthor{#3}%
\citacomes{#1}{#2}{#3}{#4}%
}

\renewrobustcmd*{\parencite}{\blx@citeargs\cbx@parencite}
\newcommand{\cbx@parencite}[4]{%
\citacomes{#1}{#2}{#3}{#4}%
}

\DeclareCiteCommand{\citeauthor}
  {\usebibmacro{cite:init}%
    \boolfalse{citetracker}%
    \boolfalse{pagetracker}%
    \usebibmacro{prenote}}%pre
  {\ifciteindex
     {\indexnames{labelname}}
     {}%
     \iffieldequals{namehash}{\cbx@lasthash}%
     {}%repetit 
     {\ifnumequal{\value{citecount}}{1}{}{\multicitedelim}%
       \printnames{labelname}}%
     \savefield{namehash}{\cbx@lasthash}%
}%post
  {}%\multicitedelim}%sep
  {\usebibmacro{postnote}}


\makeatother
%cal escapar els accents: \parencite{tal} {é}s

%http://tex.stackexchange.com/questions/28461/biblatex-tighter-integration-of-textcite-in-the-flow-of-text
%http://tex.stackexchange.com/questions/19627/biblatex-idiom-for-testing-contents-of-list-field

%-------------- dades --------------------------
\usepackage[bookmarks,pdfborder={0 0 0},pdfusetitle,
    pdftitle={Model dels SGBD Round Robin pel tractament de sèries temporals},
    pdfauthor={Aleix Llusà Serra},
    pdfcreator={DiPSE--UPC},
    pdfsubject={SGBD RRD},
    pdfkeywords={sèries temporals; adquisició de dades; SGBD per a sèries temporals; model de dades Round Robin (RRD); SGBD RRDtool},
    pdflang=ca,
    ]{hyperref}

\def\figureautorefname{figura}

\title{Model dels SGBD Round Robin}
\author{Aleix Llusà Serra}
%----------------------------------------------


\begin{document}


%------------- Pàgina de títol ------------
%\maketitle
%%------------- pàgina de portada -----------
\begin{titlepage}
  \begin{center} 

   

    {\Large \scshape Universitat Politècnica de Catalunya} \vskip 1cm 

    {Programa de Doctorat:} \vskip 0.5cm 
    
    {\scshape Automàtica, Robòtica i Visió} \vfill%\vskip 4cm 

    {Tesi Doctoral} \vskip 1cm 
    
    {\scshape \bfseries \Large Disseny i modelització d'un sistema de gestió\\
 multiresolució per a sèries temporals} \vskip 2cm

    {\bfseries Aleix Llusà Serra} \vfill%\vskip 4cm 

    {Direcció:}
       
    {Teresa Escobet Canal i
    Sebastià Vila-Marta}  \vskip 1cm 
    %\vfill 

    {Juny de 2015}

\end{center}
\end{titlepage}


%------------- pàgina de crèdits -----------
{
  \thispagestyle{empty}

  \mbox{}

  \vfill

  Primera edició: setembre de 2015. %Enquadernació en espiral, primera impressió.
  \\
  {\small Primera versió: 1.0.0 (composta a \today).} 

  \mbox{}

  {\footnotesize
  Amb el suport de la Universitat Politècnica de Catalunya (UPC).
  

  }

  \cc\bysa

  {\small
  Copyright (C) 2015 Aleix Llusà Serra.
  

  {\footnotesize
    Aquest document està sotmès a una llicència de Reconeixement-CompartirIgual 3.0 No adaptada de Creative Commons. Per veure una còpia de la llicència, visiteu \url{http://creativecommons.org/licenses/by-sa/3.0/deed.ca} o envieu una carta a Creative Commons, 444 Castro Street, Suite 900, Mountain View, California, 94041, USA.
  }

    Aleix Llusà Serra\\
    Departament de Disseny i Programació de Sistemes Electrònics
      de la Universitat Politècnica de Catalunya (DiPSE--UPC)\\
    Escola Politècnica Superior d'Enginyeria de Manresa (EPSEM),
    Av.\ de les Bases de Manresa, 61-73,
    08242 Manresa (Barcelona),
    CATALUNYA 
    }\\
    \url{aleix@dipse.upc.edu}

    {\footnotesize
      El codi font \LaTeX\ del document es troba a 
      \url{http://escriny.epsem.upc.edu/projects/rrb/}
    }
}





%%% Local Variables: 
%%% mode: latex
%%% TeX-master: "main"
%%% End: 

%----------------------------------------------

%------------- Abstract ------------
%\begin{abstract}
%\chapter*{Resum}


Les xarxes de sensors capturen dades de l'entorn, les quals s'han
d'emmagatzemar en bases de dades per a poder-les tractar
posteriorment. Hi ha models que descriuen com han de ser aquestes
bases de dades per a sèries temporals i esquemes que solucionen alguns
dels seus problemes. 


Una sèrie temporal és un conjunt de parelles de temps i valor que
provenen de l'evolució d'una variable al llarg del temps. 

A causa d'aquesta naturalesa de variable capturada al llarg del temps,
en l'adquisició i tractament de les sèries temporals apareixen
propietats problemàtiques que anomenem patologies.
Algunes d'aquestes patologies són:
\begin{itemize}
\item La sincronització dels rellotges en els diferents sistemes
  d'adquisició.
\item L'aparició de dades desconegudes perquè no s'han pogut adquirir
  o perquè són errònies.
\item La gestió d'una quantitat enorme de dades i que a més segueix
  creixent al llarg del temps.
\item L'operació amb dades que no s'han recollit de manera uniforme en
  el temps.
\end{itemize}


Els sistemes informàtics que saben emmagatzemar i tractar les sèries
temporals s'anomenen sistemes de gestió de bases de dades per a sèries
temporals (SGST). Els SGST han de saber gestionar les patologies de
les sèries temporals. 

Una solució per a aquestes patologies es pot aconseguir afegint
esquemes de multiresolució per a les sèries temporals. Aleshores
s'obtenen SGST específics anomenats SGST multiresolució (SGSTM).  La
multiresolució és un sistema d'emmagatzematge que selecciona la
informació prèviament a ser guardada i en descarta la que no es
considera important.




Un SGSTM és una solució d'emmagatzematge per a sèries temporals a on,
resumint, la informació es distribueix mitjançant diferents
resolucions temporals.  Una sèrie temporal amb multiresolució és una
co\l.lecció de subsèries resolució, les quals acumulen temporalment
mesures en un buffer on són processades i finalment emmagatzemades
en un disc. El processament de les dades té per objectiu canviar els
intervals de temps entre les mesures per tal de compactar la
informació de les sèries temporals. D'aquesta manera, les sèries
temporals queden emmagatzemades en diferents resolucions temporals
distribuïdes en els discs.  L'arquitectura d'aquests sistemes es pot
veure a la figura~\ref{fig:vhdl:bdstm}.

Els discs tenen la mida limitada i només poden contenir un nombre
fixat de mesures. Quan un disc no té més capacitat ha d'eliminar una
mesura. Com a conseqüència en un SGSTM la mida és fixada i les sèries
temporals hi queden emmagatzemades a trossos; és a dir com a subsèries
temporals.





* Ja no només importa el temps de computació, també tenir en compte altres recursos limitats --capacitat, transmissió per la xarxa, etc. Sobretot en xarxes de sensor i sistemes integrats petits


* Com que com diu Stonebraker, one size does not fit all, dissenyem
diverses implementacions del nostre model de SGBD. Explorem altres
tècniques de computació: computació para\l.lela, computació de fluxos
de dades.

%\end{abstract}
%----------------------------------------------

%------------- Índexs ------------
%\listoftodos
\cleardoublepage\pdfbookmark{\contentsname}{bookmark:index}\tableofcontents
%\addtocontents{toc}{\protect\enlargethispage{1cm}}
%\listoffigures
%\listoftables
%\lstlistoflistings
%----------------------------------------------



%------------- Cos ------------
%\frontmatter
%\mainmatter

\chapter{Estat actual}
\label{cap:estat}


SGBD: sistema de gestió de bases de dades

SGBDR: SGBD relacional

SGST: SGBD per sèries temporals



%Estat de l'art

% * no n'hi ha d'específic del tema, potser el que més s'hi assembla són els SGST que hi ha (Cougar, RRDtool, ...)

% * Hi ha temes colaterals (monitoratge,anàlisis)

% * Temes para\l.lels que ens serveixen d'inspiració (SGBD relacionals)

%Cal introduir bé el forat de coneixement que hi ha en els SGST. Forat entre les sèries temporals i els SGBD.



%Capítol:

% * Sèries temporals
  
%   - mineria
%   - aplicacions
%   - monitoratge de sèries temporals i problemes
%      * censura
%      * mostreig

%   - sgst: 
%       ficar aquí els sgbd per sèries temporals i més endavant ja es parlarà dels sgbd en general i com modelar-los i implementar-los.


% * SGBD
%  - model relacional
%  - implementacions
%  - temporal data


  % * Sèries temporals (històrics, predicció, diagnosis, prognosis, etc.)
  % * Mostreig: docs quan període de mostreig no regular
  % * Bases de dades (docs d'emmagatzematge quan la memòria és finita, docs quan període de mostreig no és regular, altres sistemes semblants (comercials,prototips))




% El capítol comença resumint l'estat de les sèries temporals en aquest camp de mineria; és a dir d'emmagatzematge i tractament. A continuació es llisten algunes aplicacions informàtiques que han implementat models de la mineria de sèries temporals. Finalment, es descriu l'estat actual de l'aplicació RRDtool, la qual també es classifica en aquest camp.

% This paper focuses on Data Base Management Systems (DBMS) that store
% and treat data as time series.   Other DBMS are not adequate for these cases as they do not have enough facilities to manage and retrieve time series
% information \parencite{schmidt95}.

% DBMS are based from formal models that define the objects and
% operations of the abstract machine to which users interact, such is
% the relational model \parencite{date}. TSMS lack a consolidated formal
% model, although special properties and requirements for a TSMS
% have been proposed \parencite{dreyer94}.








%%% Local Variables: 
%%% mode: latex
%%% TeX-master: "main"
%%% End: 

% LocalWords:  monitoratge




%------- Annexos ------
%\appendix

%\include{}


%------- Bibliografia ------
\cleardoublepage
%\phantomsection\addcontentsline{toc}{chapter}{\bibname}
\pdfbookmark{\bibname}{bookmark:bibliografia}
\printbibliography
%----------------------------------------------

%\backmatter

\end{document}


%%%%%%%%%%%%%%%%%%%%%%%%%%%%%%%%%%%%%%%%%%%%%%%%%%%%%%%%%%%%%%%%%%%%%%%%%%  
% Model dels  sistemes de gestió de bases de dades per sèries temporals.
%
% Copyright (C) 2011-2012 Aleix Llusà Serra.
% 
% This LaTeX document is free software: you can redistribute it and/or
% modify it under the terms of the GNU General Public License as
% published by the Free Software Foundation, either version 3 of the
% License, or (at your option) any later version.
%
% This document is distributed in the hope that it will be useful, but
% WITHOUT ANY WARRANTY; without even the implied warranty of
% MERCHANTABILITY or FITNESS FOR A PARTICULAR PURPOSE. See the GNU
% General Public License for more details.
%
% You should have received a copy of the GNU General Public License
% along with this document. If not, see <http://www.gnu.org/licenses/>.
%
%
% Aleix Llusà Serra
% Departament de Disseny i Programació de Sistemes Electrònics de la Universitat Politècnica de Catalunya (DiPSE-UPC)
% Escola Politècnica Superior d'Enginyeria de Manresa (EPSEM)
% Av. de les Bases de Manresa, 61-73
% 08242 Manresa (Barcelona)
% PAÏSOS CATALANS 
%
% aleix (a) dipse.upc.edu
% 
% El codi font LaTeX del document es troba a 
% <http://escriny.epsem.upc.edu/projects/rrb/>
%%%%%%%%%%%%%%%%%%%%%%%%%%%%%%%%%%%%%%%%%%%%%%%%%%%%%%%%%%%%%%%%%%%%%%%%%%  


%--------------------
% document principal
%--------------------
% cal compilar amb `pdflatex main.tex`
%--------------------
\documentclass[paper=a4,fontsize=11pt,twoside,parskip=half,BCOR12mm]{scrbook}
%%%%BCOR12mm  factor de correcció per enquadernació
%------------- capçalera ----------------------
\input{capçalera.default}
\bibliography{bibliografia}
\ExecuteBibliographyOptions{
%  annotation=true,
  backref=true,
%  isbn=false,url=false,doi=false,alldates=terse,firstinits=true,abbreviate=true
}
%\bibitemsep 0cm \biblabelsep 0cm \bibhang 0cm \renewcommand{\bibfont}{\normalfont\footnotesize}
%---------- Mode esborrany --------------------
%\includeonly{sgbd}
\usepackage[catalan]{todonotes} %%ús: \todo{text} \missingfigure{text}
\usepackage{fancyhdr}\pagestyle{fancyplain}\chead{\fancyplain{--- esborrany \today\ ---}{\footnotesize\today}}
%%\renewcommand{\headrulewidth}{0pt}
%----------------------------------------------

%------------- format -------------------------
%%ús coma decimal sense espais:  2{,}5
%per tal que trenqui la coma en math inline mode 
%http://tex.stackexchange.com/questions/19094/allowing-line-break-at-in-inline-math-mode-breaks-citations/
\AtBeginDocument{%
  \mathchardef\mathcomma\mathcode`\,
  \mathcode`\,="8000 
}
{\catcode`,=\active
  \gdef,{\mathcomma\discretionary{}{}{}}
}
\newtheorem{definition}{Definició}
\def\figureautorefname{figura} %ús: \autoref{}
\def\tableautorefname{taula} %ús: \autoref{}
\def\sectionautorefname{secció} %ús: \autoref{}
\numberwithin{equation}{chapter}



\usepackage[
          acronym,
          %%nonumberlist,
          %%toc,
          section,
          numberedsection=autolabel,
          sanitize=none, %pels accents en el vegeu
          ]{glossaries}
%\renewcommand*{\glspostdescription}{}%anul·la el punt final
\renewcommand*{\acronymname}{Sigles}%{Índex de sigles}??si té refs pàgines 
%Índex d'abreviacions?? si conté abreviatures o símbols
% \short<type>name,
\newglossary{notation}{not}{ntn}{Notació}
\newglossarystyle{estil-notation}{%
  \renewcommand{\glsgroupskip}{}% make nothing happen between groups
  \renewenvironment{theglossary}
  {\begin{longtable}{lll}
      % \caption{Notació dels SGSTM \label{tab:sgstm-simbols}}
      % \endfirsthead
      % \caption[]{Notació dels SGSTM (continuació)}
      % \endhead
          % \endfoot
          % \endlastfoot
    }{\end{longtable}}
  \renewcommand*{\glossarysubentryfield}[6]{%
    %\glstarget{##2}{##3}% the entry name
    \glstarget{##2}{\Glsentryname{##2}}% the entry name
    &
     %\space (##5)% the symbol in brackets
    \space ##4% the description
    &
    \space [##6]% the number list in square brackets
    \\
  }%
  \renewcommand*{\glossaryentryfield}[5]{%
    \\\pagebreak[3]\hline
    \glossarysubentryfield{##2}{##1}{##2}{##3}{##4}{##5}
    \hline
  }
}


\renewcommand{\seename}{vegeu}
\renewcommand{\entryname}{Notació}
\renewcommand{\descriptionname}{Descripció}

\makeglossaries


%\renewcommand{\glossarypreamble}{Text com a préambul}



%TERMES


%temps real
\newglossaryentry{TempsReal}{name={temps real}, description={(\emph{real time}), sistemes que han de respondre amb un temps determinat. A vegades també s'utilitza el terme com a adjectiu per a designar sincronització real amb el rellotge o per a indicar que l'usuari no percep retards. Allà on pugui causar confusió, utilitzarem sincronitzat o en línia (\emph{online}) per al segon significat.}
}





\newglossaryentry{SistemaGestioBaseDades}{name={sistema de gesti{ó} de base de dades}, description={(\emph{Data Base Management System})} }




%terme:SGBDR

\newglossaryentry{terme:SGBDR}{name={sistema de gestió de base de dades relacional}, description={(\emph{Relational Data Base Management System}). 
També anomenat 'object/relational' DBMS \parencite{date06}.
Totes les definicions són coherents amb \textcite{date:introduction} } }



%model, implementació
%Els SGBD es basen en teories matemàtiques que reben el nom de model de dades, un SGBD és una implementació d'un model de dades.
%Segons \citeauthor{date:introduction}, ``un model de dades és una definició abstracta, auto continguda i lògica dels objectes, de les operacions i  de la resta que conjuntament constitueixen la màquina abstracta amb la que els usuaris interactuen. Els objectes permeten modelar l'estructura de les dades. Les operacions permeten modelar el comportament''. Ara bé, \citeauthor{date:introduction} avisa que el concepte model de dades també s'usa per a definir una estructura persistent de dades concreta i per tant cal distingir adequadament la confusió entre els dos conceptes.
% Tal com fa Date, parlarem de model de dades en el primer sentit de màquina abstracta i a vegades ho abreviarem com a model.


%tipus,valor,variable,operador

\newglossaryentry{terme:SGBDR:domini}{see={terme:SGBDR},name={domini}, description = {(\emph{domain}), equivalent a tipus de dades.
Conjunt de valors. Cada domini té associat un conjunt d'operadors, en alguns casos fins i tot s'entén que el domini inclou els operadors (concepte de classe a orientació a objecte). Els tipus tenen una representació (estructura) o més d'una, és a dir els seus valors poden estar denotats per més d'un literal} }
\newglossaryentry{terme:SGBDR:tipus}{see={terme:SGBDR:domini}, name={tipus de dades}, description = {(\emph{data type}), a vegades solament 'tipus' (\emph{type}) o bé 'tipus de dades abstracte' (\emph{abstract data type}). Segons \textcite{date:introduction} en el context de model tots els tipus de dades han de ser abstractes} }

\newglossaryentry{terme:SGBDR:escalar}{parent={terme:SGBDR:domini}, name={escalar}, description = {Un tipus és escalar (\emph{scalar}) quan no té components visibles a l'usuari i és no escalar (\emph{nonscalar}) en cas contrari; no obstant, tant els escalars com els no escalars tenen representació, la qual pot contenir components} }


\newglossaryentry{terme:SGBDR:valor}{see={terme:SGBDR},name={valor}, description = {(\emph{value}), equivalent a objecte i instància.
'Constant individual' que és d'un tipus de dades. A vegades s'utilitza 'constant' per designar una  variable que mai canvia de valor, però aquest no és el cas d'aquesta definició} }
\newglossaryentry{terme:SGBDR:objecte}{see={terme:SGBDR:valor}, name={objecte}, description = {(\emph{object})} }
\newglossaryentry{terme:SGBDR:instancia}{see={terme:SGBDR:valor}, name={instància}, description = {(\emph{instance})} }

\newglossaryentry{terme:SGBDR:literal}{see={terme:SGBDR},name={literal}, description = {(\emph{literal}).
Símbol que denota un valor. Un valor pot estar denotat per més d'un literal. Segons aquesta definició literal no és equivalent a valor} }


\newglossaryentry{terme:SGBDR:variable}{see={terme:SGBDR},name={variable}, description = {(\emph{variable}).
Contenidor d'una aparició d'un valor. El valor que conté la variable pot ser canviat mitjançant l'operador d'assignació. En canvi els valors, per si mateixos, no poden ser actualitzats} } %A l'esquerra de l'operador d'assignació sempre hi ha variables, tot i que s'admeten simplificacions mitjançant expressions que són pseudovariables (p.ex. s[1] := 3 és equivalent a s := [s[0],3,s[2],..]).
%Les variables tenen adreces (\emph{addresses}) i per tant es pot apuntar (\emph{point to}) a les variables mitjançant els operadors de referència (\emph{referencing}), el qual retorna l'adreça d'una variable, i de desreferència (\emph{dereferencing}), el qual retorna la variable a partir de l'adreça. Els valors adreces pertanyen al tipus apuntador, però el model relacional prohibeix els valors de tipus apuntador i per tant no té REF ni DEREF; les relvar s'identifiquen pel seu nom i no cal que tinguin adreça. (Compte que en orientació a objectes una variable és el contenidor d'un valor que és un ID d'objecte, és a dir és el contenidor d'una referència).



%relació
\newglossaryentry{terme:SGBDR:relacio}{%
  see={terme:SGBDR},%
  name={relació},%
  plural={relacions},%
  sort={relacio},%
  description = {(\emph{relation}). Pot referir-se tant a tipus,
    valor, literal o variable relació. És l'objecte principal d'estudi
    en els SGBDR i de manera popular s'anomena taula. \emph{Nota}:
    hi ha certes diferències lògiques entre les relacions del model
    relacional i les relacions tal com es defineixen en matemàtiques.
  }%
}










%terme:tipus

% %reals projectius
% \newglossaryentry{terme:tipus:real-projectiu}{%
%   see={terme:SGBDR:tipus},%
%   name={real projectiu},%
%   plural={reals projectius},%
%   symbol={\ensuremath{\bar\mathbb{R}}},%
%   description = {(\emph{projective extended real
%       numbers}). 
% %$\bar\mathbb{R}\in\mathbb{R}\cup$
% %\{-\infty,+\infty\}$.
%   }%
% }





% [date2005]
% The original version of the model also omitted a few things I now consider vital. For example, it excluded any
% mention—at least, any explicit mention—of all of the following: predicates, constraints (other than candidate
% and foreign key constraints), relation variables, relational comparisons, relation type inference and associated
% features, certain algebraic operators (especially rename, extend, summarize, semijoin, and semidifference),
% and the important relations TABLE_DUM and TABLE_DEE.




%pendent: falta posar el name

% \newglossaryentry{SGBD-model}{ description = {Un model és}, name={Model de SGBD} }



% \newglossaryentry{SBDR-cap}{ description = {La capçalera d'un SGBDR}, name={Capçalera}, parent={SGBD-model} }



% \newglossaryentry{heading}{ description = {Equivalent to intension and relation schema} }
% \newglossaryentry{intension}{ description = {}, see=heading }
% \newglossaryentry{relation schema}{ description = {}, see=heading }

% \newglossaryentry{body}{ description = {Equivalent to extension} }
% \newglossaryentry{extension}{ description = {buit}, see=body}


% \newglossaryentry{DBMS data model}{ description = {A data model (first sense) is an abstract, self-contained, logical definition of the
% objects, operators, and so forth, that together constitute the abstract machine with which
% users interact. The objects allow us to model the structure of data. The operators allow us
% to model its behavior.\cite{date}. Sometimes it is referred as architecture.
% } }

% \newglossaryentry{data model}{ description = { A data model (second sense) is a model of the persistent data of some particular
% enterprise. [date06]. }}


% \newglossaryentry{DBMS implementation}{ description = {An implementation of a given data model is a physical realization on a real
% machine of the components of the abstract machine that together constitute that model.\cite{date}} }


% \newglossaryentry{data independence}{ description = {model and implementation kept separated}}




% \newglossaryentry{relationships}{
% description={relationships are semantic. relationships are entities.}}








%%% Local Variables: 
%%% mode: latex
%%% TeX-master: "../main"
%%% End: 

\loadglsentries{vocabulari/abreviacions.tex}
\loadglsentries[notation]{vocabulari/notacio.tex}
\newcommand{\glssymboldef}{\glssymbol[format=hyperbf,counter=definition]}
\newcommand{\glsdispdef}{\glsdisp[format=hyperbf,counter=definition]}
\newcommand{\hyperbfsec}[1]{\textbf{\S\hypersf{#1}}}
\newcommand{\hyperbfex}[1]{\textbf{ex.\hypersf{#1}}}
\newcommand{\glsaddsec}{\glsadd[format=hyperbfsec,counter=subsection]}
\newcommand{\glsaddchap}{\glsadd[format=hyperbfsec,counter=chapter]}
\newcommand{\glsaddsection}{\glsadd[format=hyperbfsec,counter=section]}
\newcommand{\glsdispsec}{\glsdisp[format=hyperbfsec,counter=subsection]}
\newcommand{\glssymbolsec}{\glssymbol[format=hyperbfsec,counter=subsection]}
\newcommand{\glssymbolex}{\glssymbol[format=hyperbfex,counter=example]}










%-------------- dades --------------------------
\hypersetup{
    pdftitle={Model dels SGBD per sèries temporals},
    pdfauthor={Aleix Llusà Serra},
    pdfcreator={DiPSE--UPC},
    pdfsubject={SGST},
    pdfkeywords={sèries temporals; adquisició de dades; SGBD per a sèries temporals; model de dades Round Robin (RRD); SGBD RRDtool},
    pdflang=ca,
}
\title{Model dels SGBD per sèries temporals}
\author{Aleix Llusà Serra}
%----------------------------------------------

%\includeonly{sgbd}

\begin{document}

\tableofcontents{}

\chapter{Estat actual}
\label{cap:estat}


SGBD: sistema de gestió de bases de dades

SGBDR: SGBD relacional

SGST: SGBD per sèries temporals



%Estat de l'art

% * no n'hi ha d'específic del tema, potser el que més s'hi assembla són els SGST que hi ha (Cougar, RRDtool, ...)

% * Hi ha temes colaterals (monitoratge,anàlisis)

% * Temes para\l.lels que ens serveixen d'inspiració (SGBD relacionals)

%Cal introduir bé el forat de coneixement que hi ha en els SGST. Forat entre les sèries temporals i els SGBD.



%Capítol:

% * Sèries temporals
  
%   - mineria
%   - aplicacions
%   - monitoratge de sèries temporals i problemes
%      * censura
%      * mostreig

%   - sgst: 
%       ficar aquí els sgbd per sèries temporals i més endavant ja es parlarà dels sgbd en general i com modelar-los i implementar-los.


% * SGBD
%  - model relacional
%  - implementacions
%  - temporal data


  % * Sèries temporals (històrics, predicció, diagnosis, prognosis, etc.)
  % * Mostreig: docs quan període de mostreig no regular
  % * Bases de dades (docs d'emmagatzematge quan la memòria és finita, docs quan període de mostreig no és regular, altres sistemes semblants (comercials,prototips))




% El capítol comença resumint l'estat de les sèries temporals en aquest camp de mineria; és a dir d'emmagatzematge i tractament. A continuació es llisten algunes aplicacions informàtiques que han implementat models de la mineria de sèries temporals. Finalment, es descriu l'estat actual de l'aplicació RRDtool, la qual també es classifica en aquest camp.

% This paper focuses on Data Base Management Systems (DBMS) that store
% and treat data as time series.   Other DBMS are not adequate for these cases as they do not have enough facilities to manage and retrieve time series
% information \parencite{schmidt95}.

% DBMS are based from formal models that define the objects and
% operations of the abstract machine to which users interact, such is
% the relational model \parencite{date}. TSMS lack a consolidated formal
% model, although special properties and requirements for a TSMS
% have been proposed \parencite{dreyer94}.








%%% Local Variables: 
%%% mode: latex
%%% TeX-master: "main"
%%% End: 

% LocalWords:  monitoratge

\todo{dir}
dir que en el nostre model no ens ocupem de l'etapa d'adquisició/mostreig de les sèries temporals sinó que tenim unes mesures que s'han capturat d'alguna manera. Si bé cal destacar que alguns SGST sí que influeixen al procés d'adquisició, per exemple poden controlar obtenir més mostres si veuen coses rares, no mostrejar més si és tranquil, canviar paràmetres, etc.


\todo{}
explicar el data mining com a background per a les oepracions que es poden fer amb les sèries temporals: pattern search, similiraties, etc.

\section{Sèries temporals}





Una sèrie temporal és un conjunt de valors cadascun dels quals té
associat un instant de temps diferent.  Tradicionalment s'anomenen
sèries temporals tot i que també s'accepta la denominació de
seqüències temporals, per exemple a \cite{last:hetland}.

Les sèries temporals s'emmarquen dins l'àmbit més genèric del que es
coneix com a \emph{dades temporals}. Les dades temporals són
co\l.leccions de dades arbitràries que estan associades a la dimensió
temps.  Dins del concepte de dades temporals s'hi encabeixen
co\l.leccions de dades de diversa natura. En funció de com un valor
queda vinculat amb el temps, es poden diferenciar dues
categories \parencite{assfalg08:thesis}.
\begin{enumerate}
\item La primera la formen les sèries temporals tal i com s'han
  definit prèviament, en la qual la dada està associada a un instant
  de temps.
\item La segona, que anomena \emph{bitemporal data}, la formen
  co\l.leccions de dades en que cada element té dos atributs
  temporals: el rang de validesa, que indica l'intèrval de temps en
  que la dada és vàlida, i el temps de transacció, que indica quan es
  va desar la dada a la base de dades.  
  %\citeauthor{assfalg08:thesis} a \cite{assfalg08:thesis} assegura que aquesta categoria de dades temporals es poden expressar en termes de la primera.\todo{Teresa5}
\end{enumerate}
Aquestes dues categories de dades temporals, tot i tenir aspectes en
comú, no poden ser tractades amb els mateixos
sistemes, \parencite{schmidt95}.


Les sèries temporals s'utilitzen en camps molt diversos i amb
objectius molt diferents. L'ús generalitzat és per a l'anàlisi i la
comprensió del comportament temporal de variables. L'evolució d'una
sèrie temporal es pot representar amb un model. Aquests models, en
l'àmbit de l'enginyeria, permeten realitzar tasques relacionades amb
validació de dades, diagnòstic i prognosis.  Per exemple, trobem
aplicacions de sèries temporals en el camp de l'avaluació de la
degradació de components \parencite{yu11}, anàlisi de l'estat dels
sensors d'un vaixell \parencite{palmer07}, validació i reconstrucció
de dades en xarxes de distribució d'aigua \parencite{quevedo10},
classificació de valors econòmics \parencite{dreyer95}, optimització
de la planificació semafòrica \parencite{last11}, estimació del temps
de viatge en autopistes \parencite{soriguera10} o transmissió
d'informació en xarxes de
sensors \parencite{jainagrawal05,yaogehrke02}.
\todo{també sèries temporals van de bracet amb GIS, exemple en dades hidrològiques [bollaert06:thesis]}


Aquest apartat té com a objectiu mostrar l'estat de l'art dels
principals processos vinculats en el treball amb sèries temporals. A
tal efecte s'organitza en tres subapartats.
 

El primer subapartat se centra en l'adquisició de dades. El primer
requeriment d'una sèrie temporal és l'adquisició de dades. Els
sistemes de monitoratge s'encarreguen de recollir dades dels sensors,
periòdicament o en base a esdeveniments.  Els problemes que es donen
durant l'adquisició generen defectes específics en les sèries
temporals que cal analitzar i tractar convenientment.


El segon subapartat tracta de l'anàlisi de sèries temporals, que és
la formalització de les tècniques que s'utilitzen per extreure
informació. A vegades aquesta extracció també es coneix com
descobriment de coneixement i s'emmarca dins de l'inte\l.ligència
artificial.


El tercer subapartat es dedica als sistemes d'emmagatzemat de sèries
temporals. L'emmagatzematge de les dades i la implementació de les
tècniques d'anàlisi té lloc en els sistemes de gestió de bases de
dades. Aquests s'encarreguen de l'organització correcte de la
informació i de respondre a les operacions de consulta. Les sèries
temporals necessiten un tractament específic per part d'aquests
sistemes.






\subsection{Adquisició de sèries temporals}

Els sistemes de monitoratge són un part important d'interacció entre
un procés i els usuaris, entenent com a procés qualsevol sistema
físic, químic, ambiental, etc.\ del qual es pugui recollir informació
continuada, ja sigui de forma periòdica o en funció
d'esdeveniments. Principalment, aquests sistemes s'encarreguen de
recollir dades, conèixer l'estat actual del procés i informar a
l'usuari. Els sistemes de monitoratge constitueixen la part principal
dels sistemes SCADA (\emph{Supervisory Control And Data
  Acquisition}). Un SCADA és el sistema encarregat de recollir i
centralitzar les dades de manera periòdica en el temps.



\begin{figure}[tp]
  \begin{center}
    \scriptsize 
    ../../../imatges/aplicacions/monitoratge.tex
  \end{center}
  \caption{Sistema de monitoratge: de l'adquisició de dades fins a informar l'usuari}
  \label{fig:sistema_monitoratge}
\end{figure}


El monitoratge es pot dividir en diferents blocs principals, els quals
es mostren a la \autoref{fig:sistema_monitoratge}. Un monitor
adquireix dades dels sensors. Les dades poden ser valors de mesures o
estats del procés adquirits com a esdeveniments. Fent referència a la
classificació de dades temporals de \textcite{assfalg08:thesis}, en
general les mesures es poden entendre com a sèries temporals i els
esdeveniments com a dades bitemporals.

En el cas de sistemes controlats o automatitzats, les dades adquirides
poden ser utilitzades per comandar o modificar el funcionament del
procés. Aleshores s'incideix en diferents nivells des de llaços de
control modificant directament un accionament, fet que no sol ser
habitual ja que els llaços de control solen realitzar-se els sistemes
electrònics que resideixen prop dels sistemes controlats, fins a
gestió de modes de funcionament i coordinació entre màquines, fet més
habitual.

L'ús generalitzat dels sistemes de monitoratge és el de proporcionar
informació de l'estat actual del procés. També disposen de la
possibilitat de generar alarmes senzilles com per exemple que no s'han
pogut adquirir les dades o que el sensor ha assolit un valor
crític. Per a usuari ens referim tant a un usuari humà com a un altre
sistema supervisor dotat amb inte\l.ligència artificial.

Per a càlculs més complicats amb les dades, els sistemes de
monitoratge utilitzen sistemes de gestió de bases de dades
(SGBD). Mitjançant els SGBD, s'emmagatzemen les dades en bases de
dades i posteriorment l'usuari les consulta per observar els històrics
o per obtenir informació i elaborar coneixement a partir de les dades
emmagatzemades.

La \autoref{fig:sistema_monitoratge} presenta una visió centralitzada
de l'adquisició de dades. Ara bé, els sistemes de monitoratge
internament poden tenir estructura distribuïda quan els sensors tenen
suficient capacitat de processament, com per exemple les xarxes de
sensors. En aquests casos els monitors cedeixen parts al sensors,
sobretot pel que fa als SGBD que passen a tenir un paper més rellevant
en la comunicació.


Un dels camps recents on l'adquisició de sèries temporals hi juga un
paper fonamental és el de les xarxes de sensors. L'abaratiment del
maquinari permet monitorar el procés amb grans quantitats de sensors
inte\l.ligents \parencite{jainagrawal05,yaogehrke02}, els quals tenen
processador i ràdio incorporats però tenen recursos limitats pel que
fa a transmissió, energia i processament i estan sotmesos a la
incertesa dels sensors. Així doncs, el problema de les xarxes de
sensors rau en estudiar l'ús eficient d'aquests recursos, pel qual
actualment trobem dues propostes.  Una solució consisteix en
transmetre la informació a un node central comprimint-la tant amb
agregacions (estadístics) com amb
aproximacions \parencite{deligiannakis07}.  Una altra solució
consisteix en tenir les dades distribuïdes en diferents sensors i
decidir com s'ha de resoldre cada consulta tenint en compte que el
processament local és més barat que la
comunicació \parencite{yaogehrke02,gehrkemadden04,bonnet01}.

\todo{el tema d'agregar informació en xarxes de sensors és interessant, potser hauria de tenir una secció pròpia}

\todo{citar kim12}:
Research Article
Aggregate Queries in Wireless Sensor Networks
Jeong-Joon Kim,1 In-Su Shin,1 Yan-Sheng Zhang,2 Dong-Oh Kim,3 and Ki-Joon Han1
Hindawi Publishing Corporation
International Journal of Distributed Sensor Networks
Volume 2012, Article ID 625798, 15 pages
doi:10.1155/2012/625798





\subsubsection{Problemes en el monitoratge}

Els sistemes de monitoratge habitualment presenten problemes derivats
de la reco\l.lecció de dades. Principalment distingim tres problemes.

\todo{parlar també del problema de la sincronització de rellotges}
\textcite[cap.~3]{kopetz11:realtime}


\begin{enumerate}
\item El primer problema és la gestió d'una quantitat enorme de dades. 

Un sistema de monitoratge recull una gran quantitat de dades. Ara bé, l'usuari només en pot observar una petita part sincronitzat (\emph{online}) amb el procés i les dades emmagatzemades esdevenen massa grans per a ser processades posteriorment \parencite{keogh97}. No obstant, les dades han de ser analitzades ja que contenen informació interessant per a les aplicacions de les sèries temporals descrites a l'apartat anterior. S'observa que en el context de monitoratge les dades recollides es poden considerar com a sèries temporals ja que abstractament són una co\l.lecció de mesures.


\item El segon problema és el de la necessitat de censurar les dades, és a dir validar que les dades siguin correctes i en cas contrari rebutjar-les o reconstruir-les. 

\textcite{quevedo10} mostren la quantitat d'informació que hi ha en els sistemes complexos de telecontrol. Aquesta informació s'obté de diversos sensors distribuïts pel camp de mesura.
En el moment de reco\l.lecció de dades apareixen dos problemes: valors que en un instant de temps prefixat no s'han pogut recollir i valors que són falsos. En el procés de gestió de dades no es poden emmagatzemar les dades amb aquests dos tipus de problema ja que aleshores els registres històrics serien inconsistents. 
Així doncs, cal comprovar que les dades emmagatzemades són correctes, mitjançant un procés de validació, i modificar-les en el cas que siguin falses, mitjançant un procés de reconstrucció que estimi els valors correctes. Per exemple, \citeauthor{quevedo10} apliquen aquests processos de validació i reconstrucció a xarxes de distribució d'aigua.


\item El tercer problema es dóna quan el període de mostreig no és regular, és a dir que les dades no es recullen de manera uniforme en el temps, però les aplicacions no ho contemplen o volen treballar amb dades a intervals regulars, també anomenat dades equi-espaiades.

Una causa de la irregularitat es deu a que els sistemes de monitoratge informàtics sovint no són capaços de complir amb exactitud el temps de mesura sinó que presenten una certa variació, ja sigui deguda a retards en els sensors, les comunicacions o la planificació del monitoratge amb altres tasques concurrents del sistema operatiu. Aquesta causa, però, es pot atenuar si els sensors envien el temps de mesura juntament amb el valor mesurat. Aleshores, el problema recau en la sincronització dels rellotges dels sensors. \todo{busca jitter en el periodic sampling de control}

Una altra causa es deu a que l'adquisició de dades prové de processos sotmesos a sistemes de control, els quals prenen el control de l'adquisició de dades. És a dir, el sistema de monitoratge ha d'obeir a les restriccions de temps imposades pels llaços de control. Aquestes restriccions són especialment crítiques en els sistemes de control en temps real ja que, aleshores, el sistema de monitoratge no pot imposar restriccions de temps diferents de les que s'han calculat per als llaços de control.  \textcite{lozoya08} mostren que s'ha de vigilar amb les entrades i sortides de les tasques periòdiques als sistemes en temps real. L'actuació dels sistemes de control es degrada quan no es té en compte que les operacions d'entrada i sortida estan subjectes a fluctuacions degudes al mostreig i a latències. Aquest problema afecta als sistemes de monitoratge en dues vessants.
Per una banda, els sistemes de monitoratge tenen una part de l'adquisició controlada per les aplicacions de control en temps real i per tant el període de mostreig resultant que veu el monitor no és regular. 
Per altra banda, les aplicacions que analitzen les dades obtingudes del monitoratge poden veure com la seva actuació es degrada si no consideren que l'adquisició de dades és irregular, el qual és similar a la regressió que s'observa \parencite{lozoya08} quan en el disseny d'un controlador discret es considera que es mostreja i s'actua periòdicament però en la implementació amb un sistema en temps real aquest pot fluctuar la periodicitat.
\todo{event based sampling/control}

\end{enumerate}



En conclusió, per tal de gestionar la complexitat derivada de la recollida de dades i també la complexitat de les consultes posteriors per part de l'usuari, els sistemes de monitoratge es recolzen en sistemes de gestió de bases de dades per gestionar l'emmagatzematge de les dades i la recuperació d'informació.





\subsection{Anàlisi de sèries temporals}

\todo{dir que una aplicació emergent és l'anàlisi de dades en temps real/en línia amb el procés per a prendre decisions}

L'anàlisi de sèries temporals consisteix en l'aplicació de
metodologies i d'algoritmes que permeten tasques com per exemple
l'extracció de característiques o obtenció de models.  Aquestes
tècniques es recullen en el que es coneix amb el nom de mineria de
sèries temporals (\emph{time series data mining}). La mineria de
dades, en la qual s'inscriu la de sèries temporals, és l'estudi
d'algoritmes específics per a extreure patrons de comportament de les
dades i s'inclou com un pas del procés general de descobriment de
coneixement a les bases de dades (\emph{knowledge discovery in
  databases}) \parencite{fayyad96,last01}.
% [M. E. Mueller, Relational Knowledge Discovery, Cambridge 2012. sec1.1p7]
%knowledge discovery: process of extracting new knowledge from a set of data about that set of data, This means that the acquisition of new lnowledge requires us to build a new model of the data.
% Data mining: refers mostly to the extraction of parts of information with respect to a given model. 
%Exemple: correlació, si dues coses tenen correlació no vol dir que necessàriament hi hagi una dependència causal entre les dades.

Actualment, les sèries temporals es consideren com un dels deu problemes
prioritaris en la mineria de dades \parencite{yangwu06}. Tal com
esmenta \textcite{fu11} en un article recent, la recerca en mineria de
sèries temporals s'ha incrementat en la darrera dècada. L'objectiu
principal és reduir la mida de les sèries temporals per tal de
disminuir el temps de processat de les dades.  \citeauthor{fu11}
resumeix l'estat actual de la mineria de sèries temporals de forma
exhaustiva i conclou que encara queden molts problemes per investigar
i resoldre. La recerca en tasques de mineria ha estat intensa però es
necessita millorar la representació de sèries temporals, ja que es
considera el pas que redueix la mida de les dades.

Segons \textcite{keogh02}, les quatre tasques que centren l'atenció de
la recerca actual de sèries temporals són l'indexat (\emph{indexing}),
que treballa amb una estructura comprimida de les dades; l'agrupament
(\emph{clustering}), que agrupa les dades segons la similitud entre
elles per tal de descobrir patrons; la classificació
(\emph{classification}), que etiqueta les dades segons les
característiques que presentin; i la segmentació
(\emph{segmentation}), que parteix una sèrie temporal en
subseqüències.  A més, \citeauthor{keogh02} comparen alguns algoritmes
experimentals duts a terme en aquests camps per diversos
autors. Recomanen a la comunitat de mineria de sèries temporals que
segueixi el seu estudi com a punt de referència per avaluar el
rendiment d'algoritmes similars.

Un pas comú previ a les quatre tasques anteriors és el de representació de la sèrie temporal. Les sèries temporals són discretes, són valors en punts de temps discrets, i la representació és el model de funció que aproxima la sèrie temporal a la seva naturalesa contínua original. La mineria de sèries temporals aprofita la representació per reduir la mida de les sèries temporals.
La representació de sèries temporals a trossos lineals (PLR, \emph{Piecewise Linear Representation}) \parencite{keogh97,keogh98} {é}s la més habitual actualment per ser més propera als usuaris ja que la visió de l'ésser humà segmenta les corbes en línies rectes.
Després de definir la PLR, \textcite{keogh00,keogh01} exploren altres representacions de sèries temporals per tal de reduir la mida d'una sèrie temporal i poder-la indexar més fàcilment. Proposen dues tècniques eficients en el càlcul: la \emph{Piecewise Aggregate Aproximation} i la \emph{Adaptive Piecewise Constant Approximation}, ambdues basades en la representació a trossos constants de la sèrie temporal. 
D'aquestes dues tècniques, \citeauthor{keogh00,keogh01} conclouen que mantenen una bona aproximació a la sèrie temporal i que a més  tenen molt menys cost de càlcul que altres de més complicades, com ara la \emph{Discrete Fourier Transform},  la  \emph{Singular Value Decomposition} o la \emph{Discrete Wavelet Transform}.

\todo{arreglar perquè quedi ben explicat; després en el model fem menció d'aquestes representacions}

\todo{potser posar-ho a estat de l'art tot això?}


In the design of the attribute interpolation function we can interpret
a time series in different ways, that is what we call the
representation of a time series. Keogh et al.\ \cite{last:keogh} cite
some possible representations for time series such as \emph{Fourier
  Transforms}, \emph{Wavelets}, \emph{Symbolic Mappings} or
\emph{Piecewise Linear Representation} (PLR). This last is remarked as
the most used owing to the most common representation is with linear
functions \cite{keogh01}.


\todo{last01} també fa una mena d'agregacions amb l'objectiu de trobar un a funció que s'aproximi a un interval de la sèrie temporal. Seria un mètode d'agregació amb aproximació?


\paragraph{Representació de sèries temporals}

\textcite{last:keogh}, cita vàries representacions per les sèries temporals com per exemple \emph{Fourier Transforms}, \emph{Wavelets}, \emph{Symbolic Mappings} o \emph{Piecewise Linear Representation} (PLR), però assenyala aquesta última com la representació més utilitzada. 
La PLR, funció definida a trossos lineal, és l'aproximació d'una sèrie temporal $S$, de llargada $n$, amb $K$ segments rectes. Els segments podrien ser polinomis de qualsevol grau, però la manera més comuna de representar sèries temporals és amb funcions lineals, segons Keogh, \cite{keogh02}.
Per aproximar el segment $S(t_a,t_b]$ d'una sèrie $S$, Keogh defineix dues tècniques: interpolació lineal, la recta que connecta $t_a$ i $t_b$, i regressió lineal, la millor recta que aproxima per mínims quadrats el segment entre $t_a$ i $t_b$.

Però també es pot representar una sèrie temporal amb una funció graó (\emph{step} o \emph{staircase function}); és a dir, amb una funció definida a trossos constant (\emph{piecewise constant representation}).
La representació a trossos constant és utilitzada en electrònica als convertidors digital-analògic (DAC, \emph{digital-to-analog converter}). En aquest cas, un senyal discret es considera una sèrie temporal i per reconstruir el senyal continu típicament s'aplica el model de \emph{zero-order hold}, equivalent a la representació a trossos constant,  o el de \emph{first-order hold},  equivalent a la representació a trossos lineal.
El model de \emph{zero-order hold} consisteix en mantenir constant cada valor fins al proper. S'obté una representació a trossos constant que en electrònica s'anomena seqüència de pulsos rectangulars (\emph{rectangular pulses}).

%http://en.wikipedia.org/wiki/Piecewise

%http://ca.wikipedia.org/wiki/Funció_definida_a_trossos

%http://en.wikipedia.org/wiki/Rectangular_function

%http://en.wikipedia.org/wiki/Step_function

% Piecewise Aggregate Approximation (PAA) \cite{keogh00}: aproxima una sèrie temporal partint-la en segments de la mateixa mida i emmagatzemant la mitjana dels punts que cauen dins del segment. Redueix de dimensió $n$ a dimensió $N$

% Adaptive Piecewise Constant Approximation (APCA) \cite{keogh01}: com el PAA però amb segments de mida variable.




\subsection{Emmagatzematge i gestió de sèries temporals}


Els sistemes de gestió de bases de dades (SGBD) són els sistemes informàtics que s'encarreguen d'emmagatzemar informació i de permetre a l'usuari consultar-la. A la secció \ref{sec:art:sgbd} es descriu com es formalitzen els SGBD, en aquest apartat ens centrarem en les necessitats que tenen les sèries temporals dels SGBD.


Les sèries temporals es diferencien d'altres tipus de dades en que els seus valors són dependents d'una variable: el temps. Com a conseqüència, qualsevol SGBD que hi vulgui tractar no ho pot fer de manera independent pels valors i pel temps; ha de conservar la coherència temporal.

Per poder aplicar les tècniques d'anàlisis de les sèries temporals de manera eficient cal disposar de SGBD específics. 
Durant l'última dècada, el maquinari informàtic ha millorat tant des del punt de vista tecnològic com de l'econòmic \parencite{deligiannakis07}, el qual ha facilitat l'adquisició de dades, per exemple amb xarxes de sensors, i alhora ha ampliat la capacitat per emmagatzemar les dades. 
Per tant, el volum de dades a tractar  en els SGBD cada cop esdevé més crític.

 
En els SGBD, el problema de grans quantitats de dades també es troba en altres camps com demostren \textcite{mylopoulos96} sobre la necessitat de grans bases de dades de coneixement. Els SGBD que tracten amb aquestes dades s'anomenen \emph{very large databases} (VLDB) i han de construir, accedir i gestionar la quantitat de dades de manera eficient.

\textcite{ogras06} consideren que les aproximacions que fan les VLDB
estan pensades per a bases de dades estàtiques. No obstant, observen
que les sèries temporal normalment són dinàmiques, és a dir de
naturalesa contínua i de mida no fitada. Conseqüentment, conclouen que
les solucions tradicionals, les quals analitzen a posteriori i sense
tenir en compte l'ordre, no es poden aplicar a causa de l'arribada
seqüencial i contínua de les dades.  Com a solució proposen resumir
dinàmicament les sèries temporals amb les tècniques de compressió que
s'apliquen en altres aplicacions on hi ha bases de dades grans.




\textcite{dreyer94} proposen desenvolupar SGBD que implementin operacions específiques per les sèries temporals, aleshores els anomenen sistemes de gestió de bases de dades per sèries temporals (SGST, \emph{time series database management systems}). Consideren que els altres SGBD no són adequats per tractar sèries temporals, tot i que després de comparar els SGBD per dades temporals i els SGST \parencite{schmidt95} troben que hi ha aspectes comuns entre els dos sistemes.
Els SGST estan optimitzats per gestionar les dades segons les operacions de temps i rotació, les quals són molt comunes en la gestió de les sèries temporals.  A més també cal controlar el creixement de la base de dades i la consulta ha de ser flexible i d'alta velocitat \parencite{keogh10:isax}. 
No obstant, fins a on coneixem, les propietats d'un model de SGST no s'han investigat més enllà  ja que la recerca s'ha concentrat en tasques de mineria de dades. Per exemple \textcite{last01} estudien una metodologia general per descobrir coneixement en els SGST, tant pel que fa a 
patrons temporals %(groups of events ordered by time)
com a regles temporals%(cause-effect relationships between events)
, i breument noten l'existència del model \cite{dreyer94} pels SGST.


Altres estudis proposen tractar les sèries temporals com a tipus que tenen ordre, per exemple seqüències o matrius.

\textcite{seshadri96:thesis} proposa que les sèries temporals són un subconjunt de les seqüències i per tant el model i les operacions per les seqüències \parencite{seshadri95} serveixen per les sèries temporals. 
\textcite{bonnet01} utilitzen el model de seqüències en SGBD distribuïts per xarxes de sensors, aleshores l'estratègia de comunicació inclou agregacions de les sèries temporals en els sensors \parencite{demers03}.
També es relaciona el model de seqüències de les sèries temporals amb els \emph{data streams} \parencite{babcock02,jagadish95,ogras06}. Els \emph{data streams} són dades que arriben contínuament i amb ordre temporal i es modelen com una seqüència on només s'hi poden afegir elements. Aleshores les consultes poden ser contínues, és a dir cada cop que arriba una dada nova s'actualitza incrementalment la informació. Per les sèries temporals s'utilitza en el càlcul de correlacions i prediccions de forma incremental \parencite{yi00} i en la cerca de patrons \parencite{bai05}.
%Data Stream Management System (DSMS) is an extension of Data Base Management System  

\todo{tot i així sembla que el concepte de bases de dades científiques ja apareix des de fa força temps \cite{segev87:sigmod}} 
En els SGBD per matrius \emph{arrays} destaquen els anomenats sistemes
de gestió de bases de dades científiques, camp en el qual les sèries
temporals hi tenen un paper de primer
ordre \parencite{zhang11}. \textcite{stonebraker09:scidb} estudien les
necessitats d'aquests sistemes sobretot en l'àmbit de la
ciència. \textcite{kersten11} proposen un sistema molt semblant però a
més integren el seu llenguatge, anomenat SciQL (\emph{SQL for
  science applications}), amb la sintaxi de SQL (\emph{Structured Query Language},
vegeu apartat \ref{sec:estat:sgbdr}). \textcite{zhang11} exemplifiquen
detalladament l'ús de SciQL en les sèries temporals per a algunes de
les seves propietats: regularitat, interpolació i cerca de
correlacions.




\todo{big data}

\url{http://cacm.acm.org/magazines/2014/7/176204-big-data-and-its-technical-challenges/fulltext}



\subsubsection{Implementacions actuals}

Hi ha hagut vàries implementacions de sistemes específics per a sèries
temporals. Algunes són només l'aplicació d'un algoritme d'anàlisi per
un problema concret de sèries temporals però altres són més elaborades
i es defineixen com a SGBD per a sèries temporals.  En aquest apartat
resumim algunes aplicacions que considerem que implementem conceptes
dels SGST.



\begin{description}

\item[Calanda] \textcite{dreyer94} proposen els requeriments de propòsit específic que han de complir els SGST i basen el model en quatre elements estructurals bàsics: esdeveniments, sèries temporals, grups i metadades, a banda de les bases de dades per sèries temporals. Implementen un SGST anomenat Calanda \parencite{dreyer94b,dreyer95,dreyer95b} que té operacions de calendari, pot agrupar sèries temporals i respondre consultes simples i ho exemplifiquen amb dades econòmiques. A \cite{schmidt95} es compara Calanda amb els SGBD temporals que operen amb sèries temporals. 




\item[T-Time] \textcite{assfalg08:thesis} mostra un sistema que pot cercar similituds calculades com a distàncies entre sèries temporals. Principalment, dues sèries temporals es marquen com a similars si la seva distància és menor que un llindar en cada interval. A partir d'aquest mètode dissenya algoritmes eficients que implementa en un programa anomenat T-Time \parencite{assfalg08:ttime}.


 
\item[iSAX] \textcite{keogh08:isax,keogh10:isax} estudien l'anàlisi i l'indexat de co\l.lecions massives de sèries temporals. Descriuen que el problema principal del tractament rau en l'indexat de les sèries temporals i proposen mètodes per calcular-lo de manera eficient. El mètode principal que proposen està basat en l'aproximació a trossos constants de la sèrie temporal \parencite{keogh00}.  Ho implementen en una estructura de gestió de dades que anomenen \emph{indexable Symbolic Aggregate approXimation} (iSAX) \parencite{isax}. Les representacions de sèries temporals que s'obtenen amb aquesta eina permeten reduir l'espai emmagatzemat i indexar tant bé com altres mètodes de representació més complexos.




\item[TSDS]
\textcite{weigel10} noten la necessitat de mostrar les dades en tot el seu rang temporal i no només en un subconjunt com normalment s'ofereixen. Desenvolupen el paquet informàtic \emph{Time Series Data Server} (TSDS) \parencite{tsds} a on es poden introduir les dades de sèries temporals per posteriorment consultar-les per rangs temporals o aplicant-hi filtres i operacions.





\item[RRDtool]
RRDtool \parencite{rrdtool} {é}s un SGBD molt usat per la comunitat de programari lliure. Projectes en diversos camps l'utilitzen com a SGBD, en els quals hi ha sistemes de monitoratge professionals, també en l'àmbit de programari lliure, com Nagios/Icinga \parencite{nagios,icinga} o el Multi Router Traffic Grapher (MRTG) \parencite{mrtg}. Aquests monitors transfereixen a RRDtool la responsabilitat de gestionar l'emmagatzematge i d'operar amb les dades, i així es poden centrar en l'adquisició de dades i la gestió d'alarmes. 
En l'evolució de RRDtool hi ha dues millores destacables. En primer lloc, \textcite{lisa98:oetiker} va separar el sistema de gestió de RRDtool de MRTG i el va dissenyar amb una estructura característica de Round Robin. En segon lloc,  \textcite{lisa00:brutlag} va estendre RRDtool amb algoritmes de predicció i detecció de comportaments aberrants. 

Actualment, s'està estudiant l'eficiència i rapidesa de RRDtool a processar sèries temporals. \textcite{carder:rrdcached} ha dissenyat una aplicació, \emph{rrdcached}, que millora el rendiment de RRDtool amb la qual s'aconsegueix fer funcionar  simultàniament sistemes amb grans quantitats de bases de dades RRDtool \parencite{lisa07:plonka}. \textcite{jrobin} han dissenyat una adaptació de RRDtool anomenada \emph{JRobin}. 
Finalment, és destacable l'ús emergent de RRDtool en entorns d'experimentació, com és el cas de \textcite{zhang07} i \textcite{chilingaryan10} que hi emmagatzemen dades experimentals per posteriorment predir o validar-les.


\item[Cougar]
\textcite{cougar,fung02} proposen Cougar com un SGBD per xarxes de sensors (\emph{sensor database systems}). El sistema té dues estructures \parencite{bonnet01}: una basada en relacions per les característiques dels sensors i una basada en seqüències per les dades dels sensors, les quals són sèries temporals.
Les consultes es processen de manera distribuïda: cada sensor és un node amb capacitat de processament que pot resoldre una part de la consulta i fusionar-la amb les altres. D'aquesta manera es minimitza l'ús de comunicacions però l'estructura i estratègia de comunicació dels nodes esdevé una part crítica a configurar \parencite{demers03}.

\item[TinyDB]
Un altre prototip de SGBD per xarxes de sensors desenvolupat para\l.lelament a Cougar és TinyDB \parencite{tinyDB,madden05}. A part de les característiques descrites per Cougar, aquest sistema  modifica i s'implica en parts del procés d'adquisició de les dades com és el temps, la freqüència o l'ordre de mostreig. Per exemple donada una consulta que vol correlacionar les dades de dos sensors, el sistema indica als sensors implicats que han d'adquirir amb la mateixa freqüència.

\item[SciDB]
\textcite{stonebraker09:scidb} estudien els SGBD científiques amb models  de dades basats en matrius. Estan desenvolupant SciDB \parencite{scidb}, un SGBD productiu i optimitzat per treballar amb matrius.


\item[SciQL]
\textcite{kersten11} descriuen SciQL, un llenguatge per a SGBD científiques basades en matrius. Hi ha un prototip en desenvolupament de SciQL \parencite{sciql}.


\end{description}


%SETL http://setl.org/setl/ un llenguatge de programació d'alt nivell que té els conjunts i els mapes de primer ordre com a parts fonamentals. Els tipus bàsics són conjunts, conjunts desordenats i seqüències (també anomenades tuples). Els mapes són conjunts de parelles (tuples de mida dos). Les operacions bàsiques inclouen la pertinença, la unió, la intersecció, etc.


\todo{}


OpenTSDB \cite{deri12:tsdb_compressed_database}
\url{http://opentsdb.net/}
Han fet anàlisis del rendiment de RRDtool, MySQL i la seva implementació TSDB i conclouen que RRDtool és el que pitjor funciona per a sèries temporals. La seva implementació, TSDB, es basa en la compressió de dades. Assumeixen que les sèries temporals són regulars i totes tenen el mateix patró de mostreig, fet que els permet implementar les gestió de les sèries temporals de manera més senzilla.
\todo{Potser hauria de classificar els SGST segons si es basen en compressió de dades (aquest tsdb, iSAX..), en Round Robin (RRDtool), vectors (SciQL), SQL (?) ...}




\url{http://pandas.pydata.org/pandas-docs/stable/index.html}

\url{http://pytseries.sourceforge.net/}



http://stackoverflow.com/questions/4814167/storing-time-series-data-relational-or-non



\url{http://2013.nosql-matters.org/bcn/abstracts/#abstract_gianmarco}

Streaming data analysis in real time is becoming the fastest and most efficient way to obtain useful knowledge from what is happening now, allowing organizations to react quickly when problems appear or to detect new trends helping to improve their performance. In this talk, we present SAMOA, an upcoming platform for mining big data streams. SAMOA is a platform for online mining in a cluster/cloud environment. It features a pluggable architecture that allows it to run on several distributed stream processing engines such as S4 and Storm. SAMOA includes algorithms for the most common machine learning tasks such as classification and clustering. 






\todo{} També hi ha molts sistemes propis d'empreses que van lligats
amb els seus productes. Ara bé ofereixen molt poques capacitats de
SGST i les que ofereixen són molt restringides a l'àmbit a on estan
dirigits els productes; és a dir que no són genèrics i són més aviat
controladors del procés d'adquisició. Per exemple Keller
\url{http://www.catsensors.com/ca/productes/varis__software/logger_4x}, permet desar dades cada un cert període amb estructura d'anell (és a dir eliminant les més antigues quan és ple) però només té un anell. A banda permet detectar certs esdeveniments i aleshores canviar el període de mostreig. A banda permet també emmatgazemar alguns estadístics de les dades: mitjana i rang cada certs segons.


\subsection{Conclusió}

Els SGST actuals bàsicament resolen alguns problemes d'anàlisis de sèries temporals.
Però no solen atendre la relació entre la base de dades i el sistema de monitoratge, és a dir la manera com s'adquireixen les dades. En aquest pas intermig hi ha un sèrie de problemes, com per exemple forats, dades falses o irregularitat en els temps de mostreig, que cal gestionar correctament. Concretament un dels problemes que no s'atén és el de mostreig irregular ja que es considera que les mostres estan a intervals regulars (equi-espaiades) encara que els sistemes de monitoratge informàtics sovint no són capaços de complir-ho amb exactitud sinó que presenten una certa variació en els temps de mesura. 

RRDtool n'és una excepció ja que, per ser un sistema productiu, el processament de dades i emmagatzematge és més proper als sistemes de monitoratge. No obstant, està centrat en un tipus de dades particulars, les magnituds i els comptadors, i no té tantes operacions generals per les sèries temporals com els altres SGST.

També Cougar i TinyDB que exploren l'encaix dels SGBD en entorns distribuïts de xarxes de sensors. Proposen noves estratègies de comunicació amb l'objectiu d'ajustar el consum d'energia. 


SciQL, un model recent per SGBD  basat en matrius, és el que més es pot considerar com a SGST, ja que s'està desenvolupant per complir-ne algunes propietats.




\subsection{Sèries temporals vs. senyals digitals}
\todo{}

Un senyal digital és un senyal (una quantitat que té informació) representat per una seqüència de valors discrets de la quantitat. Un senyal analògic és un senyal representat per una quantitat que varia contínuament.

La similitud és gran entre els conceptes de sèrie temporal i de senyal digital. La possible diferència rau en els principis assumits en els senyals digitals:
\begin{itemize}
\item Un senyal digital generalment es considera com a dades equi-espaiades en el temps.
\item La posició absoluta en el temps de les mostres d'un senyal digital no és rellevant. A més el temps no es considera continu (com a les sèries temporals) sinó que s'assumeix en el sentit estricte de seqüència a on els elements tenen ordre entre ells i una distància segons el ja dit de dades equi-espaiades.
\item Un senyal digital s'entén com un senyal amb una certa periodicitat (conseqüentment té sentit estudiar-los en el domini de la freqüència).
\item El teorema de mostreig de Shanon-Nyquist és vital en els senyals digitals. A les sèries temporals l'inframostreig pot ser tolerat en alguns casos.
\end{itemize}

Quan una sèrie temporal compleix els paràmetres anteriors aleshores és totalment semblant a un senyal digital. Moltes vegades en el treball amb sèries temporals s'assumeixen aquests principis i per tant realment s'estan aplicant operacions del processat digital de senyal.



%%% Local Variables: 
%%% mode: latex
%%% TeX-master: "main"
%%% End: 

% LocalWords:  monitoratge

\section{Sistemes de gestió de bases de dades}
\sectionmark{SGBD}
\label{sec:art:sgbd}


Segons \textcite{date:introduction}, ``una base de dades és un
contenidor informàtic persistent per a una co\l.lecció de dades''. El
sistemes informàtics que tracten amb bases de dades s'anomenen
sistemes de gestió de bases de dades (SGBD, \emph{Data Base Management
  Systems}) i tenen els objectius d'emmagatzemar informació i permetre
consultar i modificar aquesta informació per part dels usuaris.  Per
complir aquests objectius, els SGBD ofereixen a l'usuari diferents
operacions com per exemple crear una base de dades, afegir dades o
operar amb les dades emmagatzemades. 

Els àmbits d'aplicació dels SGBD són varis: operacions repetitives
i rutinàries de producció, anomenades \emph{online transaction
  processing}; sistemes per a prendre decisions empresarials, a
vegades anomenats \emph{data warehouse}; processament de dades
científiques; etc.  Alguns dels avantatges de gestionar aquestes dades
en bases de dades són: evitar la disgregació de la informació i
tenir-la perfectament organitzada, poder compartir la mateixa
informació entre diverses aplicacions, garantir la consistència i la
integritat de les dades i evitar redundàncies innecessàries, o afegir
seguretat a la gestió de les dades.


Els SGBD es poden descriure mitjançant teories matemàtiques que reben
el nom de \emph{model de dades}.  Segons
\citeauthor{date:introduction}, ``un model de dades és una definició
abstracta, auto continguda i lògica dels objectes, de les operacions i
de la resta que conjuntament constitueixen la màquina abstracta amb
què els usuaris interaccionen. Els objectes permeten modelar
l'estructura de les dades. Les operacions permeten modelar el
comportament''. Ara bé, \citeauthor{date:introduction} avisa que el
concepte \emph{model de dades} també s'usa per a definir una
estructura o esquema persistent de dades concreta i, per tant, cal distingir
adequadament entre tots dos significats.  Tal com fa Date, en aquest
document parlarem de model de dades, o simplement de model, en el
primer sentit de màquina abstracta. També distingeix entre els
conceptes de \emph{dades} --allò que està emmagatzemat a la base de
dades-- i \emph{informació} --el significat que algú dóna a aquestes
dades.


Un model de SGBD que ha s'ha consolidat i ha esdevingut un referent és
el model relacional (\emph{relational model}). L'èxit d'aquest model
és degut principalment que es fonamenta en teories matemàtiques
consolidades: la lògica de predicats i la teoria de
conjunts \parencite{date:introduction}. En base al model relacional es
va definir el llenguatge \emph{Structured Query Language} (SQL) per
operar amb bases de dades que ha esdevingut un estàndard en molts
SGBD.


Els SGBD se solen dissenyar amb una arquitectura de tres nivells: el
físic, el lògic i el d'usuari \parencite{date:introduction}. 
\begin{itemize}

\item El nivell d'usuari o extern agrupa les eines que tenen
  disponibles els usuaris per a interactuar amb la base de dades, per
  exemple en el cas relacional SQL pertany a aquest nivell.

\item El nivell lògic o conceptual és l'abstracció formal dels
  conceptes dels SGBD. En aquest nivell hi pertanyen els models de
  dades, per exemple el mateix model relacional en el cas relacional.

\item El nivell físic o intern agrupa la programació informàtica de
  com s'han d'emmagatzemar físicament les base dades i de com s'has
  d'executar les operacions. Per exemple en aquest nivell apareixen
  els registres de memòria, punters, mètodes d'accés als fitxers,
  etc., conceptes que no tenen res de relacional.
\end{itemize}


Una bona diferenciació entre els tres nivells d'arquitectura aporta
independència a les dades (\emph{data
  independence}) \parencite{date:dictionary}. Date considera que és
una de les propietats més importants que han de complir els SGBD. De
forma resumida, la independència a les dades significa que el nivell
lògic no ha de contenir detalls d'implementació ni parlar d'objectius
de rendiment sinó que aquests són part del nivell físic. Així ha de
ser possible canviar el nivell físic sense afectar el nivell lògic.
mantenint.  Així doncs, un model de dades concret pot tenir diverses
implementacions en el nivell físic, per exemple
\emph{PostgreSQL} \parencite{postgresql} per al model
relacional. \textcite{dbdebunk} detallen algunes confusions actuals
sobre la independència entre el model i la implementació.

 


\subsection{Sistemes relacionals}
\label{sec:estat:sgbdr}

El model relacional va ser proposat per \textcite{codd70} com una
teoria abstracta de dades per tal de formalitzar els SGBD amb teories
matemàtiques consolidades.  El model relacional va significar un gran
canvi en la recerca en SGBD ja que, a diferència dels models antics,
possibilitava l'estudi dels problemes amb teories matemàtiques: la
lògica i l'àlgebra de
conjunts \parencite{atzeni13:relational_model_dead}.  A partir de
llavors el model relacional ha evolucionat fins a aconseguir una gran
solidesa, amb \textcite{date04:introduction8,date06,date:dictionary} com
a principal divulgador.  Quan els SGBD es basen en el model relacional
s'anomenen relacionals (SGBDR).



El model relacional defineix el nucli dels SGBD en tres parts:
\begin{itemize}
\item Estructural: les relacions com a estructura principal per a
  representar les dades. Els tipus de dades també són necessaris per a
  representar les dades però no es defineixen en l'estructura
  principal sinó que són considerats ortogonals. \todo{vegeu més
    endavant?}

\item Manipulació: operacions sobre les relacions i que resulten en
  noves relacions. Són definides dualment a partir de l'àlgebra de
  conjunts i de la lògica i anomenades àlgebra relacional i càlcul
  lògic respectivament, totes dues tenen una definició totalment
  independent però s'han dissenyat a la vegada i són coherents entre
  elles.

\item Integritat: regles d'integritat o restriccions que han de
  complir les variables relació. És una aplicació particular de les
  operacions. Per exemple la clau primària \todo{Date no ho considera
    del nucli}

\end{itemize}


La definició de les relacions inicialment es basava en el concepte
matemàtic homònim en el sentit de producte cartesià de conjunts, però
el model relacional ha anat evolucionat i ara ja no són exactament el
mateix.  Les relacions es defineixen com una parella de capçalera
(\emph{heading}) i cos (\emph{body}). El cos és un conjunt de tuples
on cada tuple és un conjunt de parelles atribut i valor. Tots els
tuples d'una mateixa relació tenen els mateixos atributs, així es
distingeix entre la capçalera de la relació --els atributs-- i el cos
de la relació --els tuples.

Per exemple una relació entre un nom i una edat és
$r_1=(\{\text{nom},\text{edat} \}, \{
\{(\text{nom},\text{a}),(\text{edat},21)\},
\{(\text{nom},\text{b}),(\text{edat},23) \} \})$.  Simplificant i
sense explicitar que els atributs no tenen ordre, la mateixa relació
es pot expressar de forma més compacta $r_1=(
(\text{nom},\text{edat}), \{ (\text{a},21),(\text{b},23) \})$.  En la
definició estructural del model relacional, els valors sempre
pertanyen a un tipus de dades i cada atribut és restringit a un únic
tipus de dades. Així, de manera més completa hauríem d'escriure la
capçalera de la relació $r1$ com $\{\text{nom}: \text{text},\,
\text{edat}:\text{enter} \}$.


En el context lògic del model relacional, les relacions tenen la
interpretació següent: les capçaleres són predicats i els tuples són
proposicions certes per al predicat. En un context informàtic també es
pot interpretar que les capçaleres són funcions amb paràmetres i els
tuples contenen els arguments que fan certes les funcions.  Aquesta
interpretació lògica és la que realment estableix el significat de les
relacions en un context determinat.  Així, per exemple, la capçalera
de la relació $r_1$ podria correspondre al predicat ``L'estudiant
\emph{nom} té \emph{edat} anys'' i les proposicions certes són:
``L'estudiant a té 21 anys'' i ``L'estudiant b té 23 anys''.  Les
relacions es defineixen segons el principi de \emph{Closed World
  Assumption}; és a dir que els tuples que apareixen són proposicions
certes i els que no apareixen són proposicions falses. Així, per
exemple, podem dir que la proposició ``L'estudiant a té 22 anys'' és
falsa.



Les relacions es poden representar gràficament com a taules, per
exemple a la \autoref{fig:art:relacio:taula} es visualitza la relació
$r_1$.  Així, els conceptes de taules s'associen als de relacions i,
informalment, les relacions s'anomenen taules, els tuples, files o
registres, i els atributs, columnes o camps.

\begin{figure}[tp]
  \centering
  \begin{tabular}[c]{|c|c|}
    \multicolumn{2}{c}{$r_1$} \\ \hline
    nom  & edat \\ \hline
    a  & 21 \\
    b  & 23 \\ \hline
  \end{tabular} 
  \caption{Visualització com a taula d'una relació}
  \label{fig:art:relacio:taula}
\end{figure}

El nombre de tuples d'una relació s'anomena cardinal
(\emph{cardinality}) i el nombre d'atributs, grau
(\emph{degree}). Així doncs, la relació $r_1$ té cardinal 2 i grau 2.
Hi ha només dues relacions que tenen grau zero. Tenen un nom
específic, són la relació amb la capçalera i el cos buits
$\text{table\_DUM} = (\{\},\{\})$ i la relació amb la capçalera buida
i un tuple buit $\text{table\_DEE} = (\{\},\{\{\}\})$. Aquestes, però, no
tenen una representació clara com a taula.  




\todo{}

Pel que fa als operadors, en el cas de l'àlgebra relacional estan
fortament relacionats amb l'àlgebra de conjunts. Així hi ha els
operadors habituals de conjunts, com per exemple la unió, la
diferència, la intersecció o el producte; i altres d'específics per a
les relacions, com per exemple la projecció, la selecció, la junció o
el reanomena \parencite[cap.~7]{date04:introduction8}.

\todo{dir alguna cosa del càlcul relacional?}
En el cas del càlcul lògic, també anomenat càlcul relacional, estan relacionats amb la lògica. \parencite[cap.~8]{date04:introduction8}. 





\todo{alguna cosa sobre les variables relació i l'assignació?}

 En el model relacional
quan es parla de relacions es refereix a valors relació (relation value). Aquests valors es poden assignar a variables, l'operació d'assignació ja és de l'àlgebra. Aleshores aquestes variables es poden anomenar també variables relació (relvar). 
Les relvar són interessants perquè en els llenguatges que manipulen les bases de dades és còmode tenir operacions sobre les relvar (delete, update, insert, etc.) que són dreceres a operacions d'assignació sobre les relacions valor. . Aquestes operacions sobre variables relació no apareixen en el nucli del model relacional ? I també hi ha les regles d'integritat sobre les relvar. 
Unes relvar especials són les vistes. Les vistes són relvar derivades a partir d'operacions a altres relvar; és a dir són àlies d'expressions relacionals i actuen com a relvar en altres expressions. A causa d'això a vegades també s'anomenen relvar virtuals. 

La igualtat algebraica (=) entre relvars i valors relació  es pot entendre inicialment però després les relvar perden el sentit de variable algebraica i l'assignació (:=) s'ha d'entendre com una altra cosa.
Prova d'això és que les relvar pertanyen al catàleg de la base de dades.


\todo{}

Una darrera nota:
 Cal no confondre el
concepte de taula-relació (\emph{relation}) amb el de parentiu
(\emph{relationship}), els quals en català tots dos s'anomenen relació. 
Alguns autors ho anomenen taula relacional (relational table) Pascal? \parencite{dbdebunk}
\todo{podríem anomenar-lo taula-relació, taula relacional, taulaR per no confondre-ho?}
El parentiu es pot representar amb les relacions.






\subsubsection{Extensió del model amb nous tipus}
%Com s'ha d'estendre el model relacional?

El model relacional ha evolucionat però no es considera que hi hagi hagut
cap revolució des de la seva aparició
\parencite[cap.~19]{date06}. %[date06ch19pp254]
Consideren que el model relacional és bastant complet i que segueix
evolucionant en la comprensió de les teories i els conceptes que hi
intervenen, com per exemple la recent àlgebra relacional
'A' \parencite[ap.~A]{date06:_datab_types_relat_model}.  En aquest
context d'evolució, es preveuen les investigacions que poden estendre
el model relacional. Aquestes investigacions estudien propietats de
les dades, com per exemple seguretat, redundància o optimitzacions de
les consultes, a partir del nucli del model relacional i permeten
aconseguir abstraccions més generals de les
dades \parencite[cap.~25]{date06}. %[date06ch25pp441]


En el sentit d'extensió cal destacar la definició de nous tipus de
dades, els quals estenen els SGBD en funcionalitat.  Els tipus de
dades (\emph{data type}, també anomenats dominis, tipus de dades
abstracte o solament tipus, són la definició d'un conjunt de
valors. Cada tipus té associat un conjunt d'operadors, en alguns casos
fins i tot s'entén que la definició tipus inclou aquests operadors.
De manera informal \parencite{date04:introduction8} fa correspondre
els conceptes de tipus i de relacions amb els conceptes lingüístics de
noms i frases.


Com s'ha dit anteriorment, la teoria de tipus i el model relacional
són ortogonals: el model relacional requereix que hi hagi un sistema
de tipus de dades però diu molt poc de la naturalesa d'aquest sistema,
si bé el model relacional defineix que com a mínim hi ha d'haver el
tipus booleà i el tipus
relació \parencite{date:thethirdmanifesto}. 
Pel que fa a implementar
els tipus de dades en els SGBD, destaquen les primeres propostes fetes per
\textcite{stonebraker86} per tal que els usuaris puguin definir els
seus propis tipus de dades i les de \textcite{seshadri98:_enhan} que
estudia la definició de tipus de dades complexos per tal que es puguin
tractar eficientment.


Normalment els SGBD tenen uns tipus predefinits, com els enters, els
reals o els caràcters. Això no obstant, els tipus de dades poden
definir qualssevol nous valors, com per exemple matrius, documents de
text, imatges o fins i tot relacions.  Aquests nous tipus de dades
poden afegir estructures i operadors que ja siguin expressables amb
l'àlgebra relacional o bé també poden definir-se a partir de l'àlgebra
relacional. No obstant això, disposar d'un bon model d'un tipus de
dades serveix per augmentar el nivell d'abstracció en el tractament
dels conceptes relacionats amb aquestes dades
\parencite{date02:_tempor_data_relat_model}. %[date02:_tempor_data_relat_model:prefaceppxix]


El tipus de dades d'una relació és determinat per la seva capçalera.
Així doncs, la relació $r1$ és de tipus relació $\{\text{nom}:
\text{text},\, \text{edat}:\text{enter} \}$.  El tipus relació pot ser
usat a qualsevol definició on puguin ser usats els altres tipus de
dades: definicions de variables, operadors, nous tipus de dades,
etc. Tot i així la definició del tipus relació és molt rígida quant
als tipus dels seus atributs, cosa que, per exemple, no permet definir
nous operadors genèrics per a qualsevol relació.  Recentment ha
aparegut una proposta preliminar de
\textcite{darwen13:generic_relation_type} per a solucionar aquest
problema. Aquesta proposta permetria definir capçaleres genèriques de
relacions mitjançant el símbol asterisc; és a dir capçaleres amb
atributs i tipus genèrics. Així per exemple permetria usar el tipus
relació $\{ * \}$ que determinaria una relació de qualsevol tipus; és
a dir que el conjunt de valors del tipus relació $\{ * \}$ contindria
totes les relacions possibles: la table\_DUM, la table\_DEE, la $r1$,
etc. Això no obstant, encara cal flexibilitzar més les definicions
dels tipus relació. Per exemple no és possible definir nous tipus de
dades que siguin subtipus del tipus relació, cosa que permetria que
els valors d'aquests subtipus funcionessin com a arguments en els
operadors predefinits de l'àlgebra relacional.



\todo{}


Finalment, cal destacar una extensió dels SGBD amb nous tipus que ha
tingut un gran impacte en el seu àmbit: GIS


\todo{ampliació de tipus als SGBD molt important a GIS [bollaert06:thesis]}


Una altra extensió de tipus important és la de les dades temporals,
les quals detallem a la \autoref{sec:art:dades_temporals}.






\subsubsection{Implementacions relacionals}


\todo{}

 Les implementacions més populars de SGBDR són les que
ofereixen a l'usuari el llenguatge SQL per a operar amb les bases de
dades, a continuació ens hi referim com a SGBD SQL.  Ara bé, segons
Date els SGBD SQL es desvien considerablement del model relacional:
permeten files duplicades, tenen ordre en les columnes, permeten
valors nuls \parencite{date08:nulls}, etc.

Les diferències entre els SGBD SQL i el model relacional han
contribuït que hi hagi hagut diversos malentesos i errors, alguns dels
quals han estat avaluats i desmentits \parencite{dbdebunk,date06}.
  

\textcite[cap.~2]{date06} %ch2pp21-22
considera que no hi cap implementació comercial que segueixi fidelment
el model relacional, tot i que esmenta algunes implementacions
prometedores com \emph{Dataphor} o la seva proposta tecnològica
\emph{TransRelational} \parencite{date:transrelational}. A banda,
també cal destacar \emph{Rel} \parencite{rel} com un SGBDR bastant
consolidat.



Actualment \textcite{date:thethirdmanifesto} estan treballant en el
'\emph{Third Manifesto}' com a proposta per a obtenir SGBDR purament
relacionals. Destaquen que, en el model relacional, els tipus de dades
i les relacions són necessaris i suficients per representar qualssevol
dades a nivell lògic. %[date06ch21,369]
Defineixen dos principis bàsics dels SGBDR: l'\emph{Information
  Principle} o \emph{The Principle of Uniform
  Representation} \parencite{date:dictionary}, segons el qual una base
de dades només conté variables relacions, i el principi
d'ortogonalitat entre la teoria de tipus i el model
relacional \parencite[cap.~6]{date06}, segons el qual relacions i
tipus de dades són independents i per tant els atributs de les
relacions admeten qualsevol tipus.  Segons aquest punt de vista, els
tipus de dades són el conjunt de coses de les que podem parlar mentre que les
relacions són proposicions certes sobre aquestes coses.
%In other words, types give us our vocabulary the things we can talk about and relations give us the ability to say things about the things we can talk about. (There's a nice analogy here that might help: Types are to relations as nouns are to sentences.) %[date05ch4secMore on Relations Versus Types]

En la proposta per a obtenir SGBDR purament relacionals
\textcite{date06:_datab_types_relat_model,date:tutoriald} classifiquen
com a \emph{D} els llenguatges que segueixin els principis del
\emph{Third Manifesto}. Concretament, com a exemple d'un llenguatge
\emph{D} estan definint les regles de \emph{Tutorial D}, que ha de
servir pels estudis del model relacional a nivell acadèmic. Aquest
llenguatge ja s'utilitza en alguns SGBDR, com per exemple a
\emph{Rel} \parencite{rel}.


El model relacional ha incorporat conceptes d'altres disciplines. En
destaca sobretot la incorporació de conceptes dels models d'orientació
a objectes com és el cas dels tipus de dades i de l'herència.
Aleshores s'entén que els SGBDR també es puguin anomenar SGBD
objecte/relacionals (\emph{object/relational})
\parencite{date02:foundation}.  Tot i així, \textcite[cap.~6]{date06}
manifesta i avisa de l'ús de la mateixa terminologia amb significat
diferent entre el model relacional i l'orientació a objectes, sobretot
pel que fa als termes valor i variable. %[date06ch6pp91]
La seva hipòtesi a aquestes diferències és que el model relacional és
un model de dades i el model d'orientació a objectes és més proper a
un model
d'emmagatzematge. % 'the object model' is closer to being a model of
                  % storage than it is to being a model of
                  % data. [date06ch6pp92]

% A la \autoref{tab:sgbd:relacional-objectes} es resumeix la possible
% equivalència lògica dels conceptes entre el model relacional i
% l'orientació a objectes tal com Date exposa al capítol 6, tot i que
% cal tenir en compte que la semblança és difusa.

% \begin{table}
% \centering
% \begin{tabular}[ht]{ll}
%   relacional & objectes \\\hline \hline
%   tipus & tipus, classe, interfície \\\hline
%   representació & classe, atributs, propietats \\\hline
%   valor, objecte, instància & valor, estat, objecte/instància immutable/estàtic \\\hline
%   variable & valor, objecte/instància mutable/dinàmic \\\hline
%   referència & variable \\\hline
%   operador & funció, mètode \\\hline
% \end{tabular}
% \caption{Possible equivalència lògica de termes entre el model relacional i l'orientació a objectes \parencite[cap.~6]{date06}.}
% \label{tab:sgbd:relacional-objectes}
% \end{table}

% Relacional: tipus | representació |  valor, objecte, instància  | variable  | referència (adreça continguda en una variable) | operadors (de lectura i de modificació)
% Objectes: tipus, classe (tipus amb atributs i mètodes), interfície | classe, atributs,propietats  |  valor, estat, objecte/instància immutable/estàtic |  valor, objecte/instància mutable/dinàmic  | variable | funcions,mètodes (funcions dins de classes) (purs o modificadors)










\subsection{Altres sistemes}

\todo{}

\todo{models de graf: aquests fan bona pinta, tenen una bona base teòrica. Però sembla que és un model que va molt bé per a representar dades de tipus relació parentiu (relationship) i no es tan genèric per a les altres dades com el model relacional}


Fabian Pascal %\url{http://www.dbdebunk.com/2013/11/more-on-erm-still-not-data-model.html?utm_source=feedburner&utm_medium=feed&utm_campaign=Feed%3A+blogspot%2FTuHQT+%28DATABASE+DEBUNKINGS%29}:
% ``I never claimed that "only RM models data". In fact, the hierarchic and network data models also do, but were discarded decades ago because they were, for various reasons, inferior to the RM. Other than those two, I am unaware of any other proposed data model that is formal and complete.'' 
% ``There have been attempts to structure and manipulate data based on other logics/theories e.g. graph theory or 2nd order logic, but they have proved more complex and less flexible and comprehensible than RT. In my modeling paper I provide a criterion for comparing data models on superiority, but doing it right is non-trivial and nobody has ever claimed superiority to RT on sound grounds.''



\todo{Encara avui hi ha discussions en l'àmbit de SGBD}
Es pot dir que hi ha quatre corrents d'opinió: Els SQL, els NoSQL, els NewSQL i els RelacionalsPurs (els teòrics com Date, Darwen i Pascal).
És molt difícil valorar per què tenir un model fortament matemàtic és millor que no tenir-lo, això s'ha demostrat a través de l'experiència. Per tant algunes discussions giren al voltant d'això. Alhora també és molt difícil aconseguir una implementació totalment fidedigna al model matemàtic, a causa de la gran potència que aconsegueixen les matemàtiques amb l'abstracció, i molt menys que sigui una implementació eficient. Per tant altres discussions giren al voltant d'això. 

Tenen raó quan DateDarwenPascal diuen que molts dels productes NoSQL són retornar a models pre-relacional fallits (jeràrquic o  altres) però també tenen raó els NoSQL quan diuen que fin a l'arribada dels seus productes no hi havia cap sistema informàtic capaç de resoldre eficientment determinades aplicacions. Potser s'ha de passar per uns moments de transició d'implementacions NoSQL de baix nivell (més properes al nivell físic) per a després agafar les bones idees i millorar els sistemes relacionals. De fet en certa manera ja està succeint: el corrent NoSQL ha esperonat la millora dels productes SQL (p.ex. cas de MapReduce esperona al NewSQL de Stonebraker).

Per altra banda, la definició del concepte de SGBD ha hagut de canviar. Primer anava acompanyada de requisits particulars (propietats ACID, transaccions, seguretat, optimització de les consultes) però ara s'ha hagut de generalitzar i centrar-se en el nucli dels SGBD: de fet és exactament el que descriu el model relacional. En canvi els requisits particulars es descriuen com a complements d'aquest nucli. (alguns fins i tot diuen que el sistema de fitxers es pot veure com un SGBD).


\todo{ha d'aparèixer MapReduce i també ha d'aparèixer que Stonebraker està dissenyant el mateix esquema però amb SGBDR}


\todo{}





Les crítiques als SGBDR, sobretot les degudes als SGBD
\emph{SQL}, han contribuït a voler explorar altres models de
SGBD \parencite{stonebraker09}. Aquests models presenten diferents
maneres de representar les dades: llistes, seqüències, enllaços,
matrius, etc.
%[date06pp116,134]

Tot i així, \textcite[cap.~21--25]{date06} considera que els nous
models de SGBD, a vegades anomenats post-relacionals, no estan fundats
tant sòlidament en teories matemàtiques i la lògica de predicats com
el model relacional i pronostica que ens els propers cent anys els
SGBD encara estaran basats en el model
relacional. %[date06ch19pp354,date06ch20pp365]
Considera la possibilitat, tot i que remota, que es pugui definir un
model més potent que el relacional però que no hi ha cap indici que
cap definició dels nous model tingui la mateixa potència que el
relacional. Per tant, aconsella que per ara els SGBD no s'allunyin del
model relacional. %[date06ch25,ch21pp379-380]



\todo{} La comunitat de SGBD dóna la benvinguda al NoSQL
\cite{atzeni13:relational_model_dead}. És una bona notícia, a veure si
a partir d'ara veiem recerca i formalització dels models dels SGBD
NoSQL i es poden entendre i comprendre les diferències amb els SGBDR.

\todo{més recentment ha aparegut el corrent NewSQL \url{http://readwrite.com/2011/04/06/the-newsql-movement}}
\url{http://readwrite.com/2011/04/06/the-newsql-movement}


Recentment ha aparegut un nou corrent en l'àmbit dels SGBD que
s'anomena NoSQL (\emph{Not Only SQL}) amb l'objectiu de sobrepassar
les limitacions dels SGBDR \parencite{edlich:nosql,stonebraker10}.  A
l'espera que Date valori aquest corrent, cal tenir present els seus
apunts sobre sistemes relacionals contra sistemes no
relacionals \parencite[part 7]{date06}, sobretot pel que fa al
concepte de l'\emph{Information Principle} i que SQL no és un bon
referent pels SGBDR. És a dir, s'ha d'entendre NoSQL com una crítica a
les implementacions comercials actuals del model relacional, una
crítica que pot estar motivada per l'ús de SQL per part d'aquest
productes.  No es pot entendre, però, el NoSQL com una crítica al
model relacional ja que els objectius parlen de millorar el rendiment
dels SGBD, cosa només atribuïble a les implementacions però no al
model. Precisament, actualment del model relacional destaca la
proposta de \citeauthor{date:tutoriald} d'un llenguatge,
\emph{Tutorial D}, que no és SQL.


El corrent de NoSQL també critica l'adequació dels productes actuals.
Els SGBD NoSQL apunten els SGBDR actuals per voler ser \emph{one size
  fits all} \parencite{stonebraker07,stonebraker09} però que cada
aplicació té els seus requisits i per tant una mateixa implementació
no pot ser bona per a tots el camps.  En aquesta mateixa línia els
models pels SGBD prenen més sentit que mai ja que permeten mantenir
una definició comuna per a moltes implementacions.


Segons es desprèn de \textcite{date06} fins a l'actualitat només hi ha
hagut un model consolidat pels SGBD: el model relacional.  Ara, en el
corrent NoSQL també es parla de nous models de
SGBD \parencite{edlich:nosql,stonebraker09:scidb}.
\citeauthor{date06} ha avaluat que alguns nous models recuperen
intents fallits en el passat tot i que es poden representar amb el
model relacional, per exemple els SGBD XML basats en estructures
d'arbre \parencite[cap.~14]{date06} \parencite[cap.~27]{date04:introduction8}
o els ODMG basats en objectes \parencite[cap.~27]{date06}. Tot i així,
en un futur cal estar atents per si alguns d'aquests models joves de
SGBD arriben a consolidar-se i poden esdevenir tant potents com el
model relacional.


\todo{}
Diferència entre SGBD i llenguatges de programació: independència de les dades quant al model lògic i al nivell físic: \emph{``It appears that
current NoSQL systems make no distinction be-
tween the logical and physical schema. Thus, the
fundamental advantages of the ANSI SPARC ar-
chitecture have been voided, which complicates the
maintenance of these databases. Storing objects as
they are programmed essentially negates the data
independence requirement that then remains to be
adequately addressed for NoSQL database systems.''}  \textcite{atzeni13:relational_model_dead}.








\section{Dades temporals}
\label{sec:art:dades_temporals}

\todo{Convertir en una secció les dades temporals i parlar-ne més extensament}
parlar de docs anteriors de dades temporals

Refs a estudiar:

\cite{jensen99:temporaldata}
\cite{jensen00:thesis}
\cite{jensen98:temporal_database_glossary}
\cite{tansel93:temporal_databases}


Em alguns llocs les bases de dades temporals també s'anomenem multiversió (multi-version)


%Precisament,
Una de les extensions importants del model relacional s'ha produït amb
l'estudi de les dades
temporals \parencite{date02:_tempor_data_relat_model}. Amb el model de
dades temporals basat en el model relacional s'obtenen SGBDR capaços
d'emmagatzemar i consultar dades històriques.

El model de les dades
temporals \parencite{date02:_tempor_data_relat_model} es basa en
relacions i intervals per representar les dades històriques. Cada
relació s'estén amb un atribut que és un interval temporal que indica
el rang temporal de validesa de les proposicions. A més el model també
defineix les operacions necessàries per tractar amb les dades
temporals. Aquestes operacions són extensions de l'àlgebra relacional.
El principal objectiu del model de dades temporals és poder tenir
suport per a les dades històriques en els SGBDR.
\textcite[cap.~28]{date06} compara aquest model per dades temporals
amb altres aproximacions que s'han fet no basades en el model
relacional.



El model de dades temporals representa les dades històriques amb el
temps vàlid i temps de
transacció \parencite[cap.~15]{date02:_tempor_data_relat_model}, el
que es coneix com a dades bitemporals.  Així doncs, els SGBD per dades
bitemporals no es consideren adequats com a SGBD per sèries temporals
ja que els primers estan pensats per a històrics, es descriuen amb
intervals temporals, i els segons per anàlisi d'observacions
seqüencials, es descriuen amb instants temporals.
\textcite{schmidt95} arriben a aquesta conclusió després de comparar
els SGBD bitemporals amb els de sèries temporals, tot i que per a
desenvolupaments futurs observen que hi aspectes temporals comuns
entre els dos sistemes i es pregunten si es podran trobar sistemes que
els englobin a tots dos o cadascun necessitarà sistemes específics.


%a algun lloc deia Date que el seu model per dades temporals no era del tot adequat per les sèries temporals?, potser perquè els intervals no estaven pensat per a instants? date02ch16.8?




\todo{llibre que apareixerà aviat}

Time and Relational Theory, 2nd Edition
Temporal Databases in the Relational Model and SQL
 
Author(s) : Date,     Lorentzos,    Darwen.    Expected Release Date: 26 Sep 2014Imprint:Morgan KaufmannPrint Book ISBN :9780128006313 Pages: 560Dimensions: 235 X 191
\url{http://store.elsevier.com/product.jsp?isbn=9780128006313&pagename=search}


\subsection{Conclusió}

\todo{}

Actualment l'àmbit informàtic de SGBD se centra en les
implementacions, com ho demostra el nou corrent NoSQL concentrat en
trobar models d'implementacions que tinguin bon rendiment. A tal
efecte la recerca es concentra en temes de garantia de propietats ACID
(\emph{atomicity, consistency, isolation, durability}), d'optimització
de consultes, d'emmagatzematge de grans volums de dades, de consultes
via web, de distribució de bases de dades, de reduir la despesa en
energia, etc. \parencite{stonebraker07,stonebraker10}, la qual cosa és
exce\l.lent per a disposar d'un SGBD adequat a cada aplicació.
\textcite{haerder05:_dbms_archit} descriu diferents models
d'implementació per als SGBD, ja que indica que per obtenir bon
rendiment la implementació d'un SGBD s'ha d'estudiar per cada
aplicació. És més, una implementació d'un SGBD que vulgui obtenir un
bon rendiment en una determinada aplicació potser no pot implementar
el model de dades complet sinó que només una part, com per exemple en
els sistemes
encastats \parencite{saake09:_downs_data_manag_embed_system}.

Per altra banda, l'àmbit matemàtic de SGBD, amb el model relacional
com a màxim exponent, se centra en els conceptes teòrics, és a dir
respon a la pregunta de què són els SGBD. Recerca en millorar-ne la
comprensió, en obtenir la màxima potència i facilitat de cara a la
gestió de dades per part de l'usuari o en obtenir nous models. Tot i
així, actualment encara no s'ha trobat cap altre model que tingui la
mateixa potència que el relacional. Cal destacar que tot i que el
model relacional té conceptes madurs i consolidats, i que a més han
tingut èxit amb els SGBD SQL, s'obre una nova perspectiva amb
l'evolució de conceptes que proposa el \emph{Third Manifesto},
especialment amb \emph{Tutorial D} i les implementacions que comencen a prendre
cos a nivell acadèmic.



















%%% Local Variables: 
%%% mode: latex
%%% TeX-master: "main"
%%% End: 

% LocalWords:  monitoratge SGBD SGBDR SQL


\section{Sistemes i projectes similars}




Hi ha vàries implementacions de sistemes per a gestionar sèries
temporals. Algunes són només l'aplicació d'un algoritme d'anàlisi per
a un problema concret de sèries temporals però altres són més
elaborades i es defineixen com a \gls{SGBD} específics per a sèries
temporals. En aquests secció resumim algunes aplicacions que
considerem que implementem conceptes dels \gls{SGST}.



Explorem l'estat de la recerca en sistemes i projectes
similars a l'objectiu dels nostres models: gestionar les sèries
temporals i aplicar-hi alguna tècnica, com la multiresolució, per tal
de solucionar algunes de les propietats problemàtiques. Cal notar que
hi ha una gran quantitat de sistemes propis de productes, sobretot
lligats a la reco\l.lecció de dades de sensors, que gestionen algunes
característiques de les dades adquirides. Ara bé, ofereixen capacitats
molt restringides a l'àmbit on es dirigeixen els productes, és a dir
que no són genèrics i són més aviat controladors del procés
d'adquisició. Per exemple Keller \parencite{keller} permet desar dades
cada un cert període amb estructura d'anell, és a dir elimina les més
antigues quan és ple, però només té un anell, a banda també permet
detectar certs esdeveniments i emmagatzemar alguns estadístics de les
dades.  Aquests sistemes, però, són tancats i no s'especifica amb
detall el seu funcionament ni la seva estructura i per tant són
difícils d'avaluar.
% A banda permet detectar certs esdeveniments i aleshores canviar el període de mostreig. A banda permet també emmatgazemar alguns estadístics de les dades: mitjana i rang cada certs segons.






Classifiquem els sistemes en quatre apartats segons la característica
principal que els defineixi, tot i que no és una classificació
absoluta ja que alguns en poden tenir més d'una:
\begin{itemize}
\item Sistemes genèrics
\item Compressió i aproximació
\item Processament en flux
\item Emmagatzematge massiu
\end{itemize}






 


\subsection{Sistemes genèrics}


La recerca en dades bitemporals formalitza de forma adequada els
\gls{SGBD} per a poder tractar històrics i esdeveniments
temporals \parencite{jensen99:temporaldata,date02:_tempor_data_relat_model}. Això
no obstant, com ja hem notat, les sèries temporals i les dades
bitemporals no són exactament el mateix i no poden ser tractats de la
mateixa manera \parencite{schmidt95}. Hi ha, però, certes similituds
que es poden tenir en compte, per exemple les nocions de temps
discret. A més, formalitzarem les sèries temporals de manera similar a
com les dades bitemporals es formalitzen en els \gls{SGBDR}.



Per altra banda, alguns autors descriuen sistemes genèrics per a
tractar sèries temporals, és a dir amb un model adequat per a sèries
temporals però sense cap tècnica específica per a processar-les. A
continuació en descrivim alguns breument.




\begin{description}


\item[TDM] \textcite{segev87:sigmod} presenten un model, que anomenen
  \emph{Temporal Data Management} (TDM), per a dades temporals amb un
  llenguatge molt semblant a \gls{SQL}. Les seqüències temporals que
  presenten són similars a les que definim en el model de
  \gls{SGST}, inclouen la noció de regularitat i representació
  temporal, tot i que molt lligades a un tercer atribut que indica
  l'objecte de referència. Principalment estudien les operacions
  d'agregació sobre les sèries temporals.


\item[Calanda] \textcite{dreyer94} proposen els requeriments de
  propòsit específic que han de complir els \gls{SGST} i basen el
  model en quatre elements estructurals bàsics: esdeveniments, sèries
  temporals, grups i metadades, a banda de les bases de dades per
  sèries temporals. Implementen un \gls{SGST} anomenat
  Calanda \parencite{dreyer94b,dreyer95,dreyer95b} que té operacions
  de calendari, pot agrupar sèries temporals i respondre consultes
  simples i ho exemplifiquen amb dades econòmiques. A \cite{schmidt95}
  es compara Calanda amb els \gls{SGBD} per a dades bitemporals.




\item[Pandas] Pandas \parencite{pandas} és una eina d'anàlisis de
  dades. Tot i no ser un \gls{SGBD} sí que en té una forta orientació
  ja que gestiona les dades a partir d'una estructura tabular i amb
  molts conceptes relacionals.  Una de les principals aplicacions és
  en les sèries temporals, inclou per exemple la regularització de
  sèries temporals. Així, Pandas és semblant a altres eines d'anàlisi
  estadística per a computació científica però incorpora la gestió de
  sèries temporals i dades similars. Un sistema similar d'anàlisi de
  sèries temporals, scikits.timeseries \parencite{pytseries}, ja no es
  desenvolupa més i està previst que s'incorpori a Pandas.



\end{description}



%SETL http://setl.org/setl/ un llenguatge de programació d'alt nivell que té els conjunts i els mapes de primer ordre com a parts fonamentals. Els tipus bàsics són conjunts, conjunts desordenats i seqüències (també anomenades tuples). Els mapes són conjunts de parelles (tuples de mida dos). Les operacions bàsiques inclouen la pertinença, la unió, la intersecció, etc.




\subsection{Tècniques de compressió i aproximació}



% As \acro{TSMS} suffer from problematic properties of time
% series, like the ones we describe in
% Section~\ref{sec:model:properties} mainly the huge data volume,
% compression techniques are used.  Next, we summarise some current work
% in \acro{TSMS} with compression.

Els \gls{SGST} han de gestionar les propietats problemàtiques de les
sèries temporals, com les descrites a
la~\autoref{sec:art:problemes}. Principalment, el gran volum de dades
comporta que s'explorin tècniques de compressió de les dades o de
treballar amb dades que s'aproximin a la informació original.  La
compressió i aproximació es pot explorar tant amb emmagatzematge amb
pèrdua de les dades originals o sense pèrdua o fins i tot intentant
resoldre el problema de trobar el compromís entre les mínimes dades
que poden reconstruir el senyal original amb el mínim error. A
continuació descrivim breument els projectes que exploren la
compressió i aproximació de sèries temporals.



\begin{description}


\item[T-Time] \textcite{assfalg08:thesis} mostra un sistema que pot
  cercar similituds entre sèries temporals, calculades segons funcions
  de distàncies entre sèries temporals. Principalment, dues sèries
  temporals es marquen com a similars si la seva distància és menor
  a un llindar per cada interval de temps. A partir d'aquest mètode dissenya
  algoritmes eficients que implementa en un programa anomenat
  T-Time \parencite{assfalg08:ttime}.

 
\item[iSAX] \textcite{keogh08:isax,keogh10:isax} estudien l'anàlisi i
  l'indexat de co\l.lecions massives de sèries temporals. Descriuen
  que el problema principal del tractament rau en l'indexat de les
  sèries temporals i proposen mètodes per calcular-lo de manera
  eficient. El mètode principal que proposen està basat en
  l'aproximació a trossos de la sèrie temporal \parencite{keogh00}.
  Ho implementen en una estructura de gestió de dades que anomenen
  \emph{indexable Symbolic Aggregate approXimation}
  (iSAX) \parencite{isax}. Les representacions de sèries temporals que
  s'obtenen amb aquesta eina permeten reduir l'espai emmagatzemat i
  indexar tant bé com altres mètodes de representació més complexos.
  Aquestes tècniques de compressió són candidates per a ser usades com
  a funcions d'agregació d'atributs en el model de \gls{SGSTM} que
  definim, així seria interessant poder definir agregacions en el
  domini freqüencial de les sèries temporals.

% Piecewise Aggregate Approximation (PAA) \cite{keogh00}: aproxima una sèrie temporal partint-la en segments de la mateixa mida i emmagatzemant la mitjana dels punts que cauen dins del segment. Redueix de dimensió $n$ a dimensió $N$

% Adaptive Piecewise Constant Approximation (APCA) \cite{keogh01}: com el PAA però amb segments de mida variable.




\item[RRDtool]
  \parencite{rrdtool,lisa98:oetiker} desenvolupa un \gls{SGBD}
  anomenat RRDtool que és molt usat per la comunitat de programari
  lliure en l'àmbit dels sistemes de monitoratge. A causa d'això es
  focalitza en unes dades en particular, les magnituds i els
  comptadors, i hi manquen operacions genèriques de sèries temporals.
  La principal característica és l'emmagatzematge de les dades amb la
  tècnica que anomenen Round Robin, la qual consisteix en emmagatzemar
  més resolució per als temps recents i en perdre resolució per als
  temps més antics tot gestionant els registres d'emmagatzematge de
  manera circular.

  Hi ha diversos projectes que utilitzen RRDtool com a \gls{SGBD}, en
  els quals hi ha sistemes de monitoratge professionals, també en
  l'àmbit de programari lliure, com
  Nagios/Icinga \parencite{nagios,icinga} o el Multi Router Traffic
  Grapher (MRTG) \parencite{mrtg}. Aquests monitors transfereixen a
  RRDtool la responsabilitat de gestionar l'emmagatzematge i d'operar
  amb les dades, i així es poden centrar en l'adquisició de dades i la
  gestió d'alarmes. Altres projectes adapten la tècnica de RRDtool en
  altres llenguatges, com per exemple
  JRobin \parencite{jrobin}. També és destacable l'ús emergent
  de RRDtool en entorns d'experimentació, com és el cas de
  \textcite{zhang07} i \textcite{chilingaryan10} que hi emmagatzemen
  dades experimentals per posteriorment predir o validar-les.  A causa
  del gran ús que es fa de RRDtool, sobretot en la comunitat de
  programari lliure, ens ha inspirat per a desenvolupar un model a
  partir de les principals característiques, la qual cosa és el que
  anomenem multiresolució.


  En l'evolució de RRDtool hi ha dues millores destacables. En primer
  lloc, \textcite{lisa98:oetiker} va separar el sistema de gestió de
  RRDtool d'un sistema de monitoratge particular, MRTG, i el va
  dissenyar amb l'estructura característica de Round Robin. En segon
  lloc, \textcite{lisa00:brutlag} va estendre RRDtool amb algoritmes
  de predicció i detecció de comportaments aberrants.  Actualment,
  s'està estudiant l'eficiència i rapidesa de RRDtool en processar les
  sèries temporals.  RRDtool pot emmagatzemar múltiples resolucions de
  les dades, però \textcite{lisa07:plonka} troben limitacions de
  rendiment quan s'han d'emmagatzemar grans quantitats de sèries
  temporals diferents. Una solució que observen per a aquest problema
  és l'aplicació de \emph{cache} dissenyada per
  \textcite{carder:rrdcached}, anomenada RRDcached, que permet
  fer funcionar simultàniament sistemes amb grans quantitats de bases
  de dades RRDtool.




\item[Whisper] Una eina per a visualitzar gràfics de dades que tenen
  forma de sèries temporals és Graphite \parencite{graphite}. Graphite
  utilitza un \gls{SGBD} anomenat Whisper que té un disseny molt
  similar a RRDtool, de fet inicialment Graphite usava RRDtool com a
  sistema d'emmagatzematge.





\item[Tsdb] \textcite{deri12:tsdb_compressed_database} desenvolupen
  Tsdb, un \gls{SGST} d'emmagatzematge amb compressió sense pèrdua per
  a les sèrie temporals. Les sèries temporals han de compartir
  exactament els mateixos instants de temps d'adquisició i aleshores
  tots els valors s'emmagatzemen agrupats per temps en comptes de
  tenir cada sèrie temporal aïllada.  Així doncs, assumeixen que les
  sèries temporals són regular i tenen el mateix patró de
  mostreig. Els valors s'emmagatzemen aplicant tècniques de compressió
  sense pèrdua, a diferència d'altres sistemes que també emmagatzemen
  tota la sèrie temporal original però amb tècniques massives, com per
  exemple OpenTSDB del qual comenten que té una arquitectura massa
  complicada i només és útil per a sistemes distribuïts.

  Comparen el rendiment de Tsdb amb RRDtool i un producte
  \gls{SQL}. Gràcies a l'estructura de Tsdb aconsegueixen un millor
  temps d'addició de les mesures però un pitjor temps de recuperació
  de les dades ja que per obtenir una sèrie temporal s'han de
  reagrupar els valors. Tot i així, és una aproximació interessant per
  a ser aplicada en els \gls{SGSTM} quan cal agrupar sèries temporals
  que comparteixen els mateixos instants d'adquisició: aleshores es
  podria dissenyar una implementació amb aquesta arquitectura en què
  els valors fossin vectors i les operacions es processessin alhora
  per a totes les sèries temporals en el mateix moment de l'addició.





\item[Emmagatzematge en memòries Flash]
  \textcite{dou14:historic_queries_flash_storage} se centren en
  l'àmbit de l'emmagatzematge de sèries temporals en memòries de tipus
  Flash, de les quals noten que tenen propietats diferents a
  l'emmagatzematge tradicional en discs.  Proposen emmagatzemar
  informació de cada sèrie temporal per a poder resoldre tres tipus de
  consultes: agregacions temporals, històrics basats en mostrejos
  aleatoris i cerca de patrons similars.  La tècnica d'agregacions
  temporals que utilitzen és molt semblant a la de RRDtool, és a dir
  agregar i emmagatzemar les dades amb diferents resolucions, tot i
  que implementada i particularitzada per a les memòries Flash, amb
  registre i punters. Per a la cerca de patrons similars indexen les
  sèries temporals de manera similar als algoritmes de iSAX.


\end{description}



\subsection{Processament en flux}


Les sèries temporals també es tracten com a fluxos de dades
(\emph{data stream}) per tal de resoldre consultes d'agregació
estadística de les dades mitjançant consultes aproximades.  Com a
fluxos de dades, s'exploren tècniques per a processar les consultes de
forma incremental cada cop que arriba una dada nova.
\textcite{cormode08:pods} exploren tècniques d'agregació en flux per a
sèries temporals que consideren donar més pes a les dades més recents,
és a dir de manera molt similar a la multiresolució que proposem però
només per a una resolució i per a unes funcions d'agregació
determinades.

El processament en flux  s'usa sobretot en l'àmbit de les xarxes de
sensors, del qual a continuació en descrivim alguns projectes.


\begin{description}

\item[Cougar] \textcite{cougar,fung02} proposen Cougar com un
  \gls{SGBD} per a xarxes de sensors (\emph{sensor database
    systems}). El sistema té dues estructures \parencite{bonnet01}:
  una per a les característiques dels sensors emmagatzemades com a
  taules relacionals i una altra per a les sèries temporals dels
  sensor emmagatzemades com a seqüències de dades.  Les consultes es
  processen de manera distribuïda. Cada sensor és un node amb
  capacitat de processament que pot resoldre una part de la consulta i
  fusionar-la amb les altres. D'aquesta manera les dades
  s'emmagatzemen distribuïdes en els sensors i les consultes es
  resolen combinant les dades amb orientació de flux, cosa que millora
  el rendiment del processament i es minimitza l'ús de les
  comunicacions.  Això no obstant, l'estructura i l'estratègia de
  comunicació dels nodes esdevé una part crítica a configurar en
  aquests sistemes \parencite{demers03}.


\item[TinyDB] Un altre prototip de \gls{SGBD} per a xarxes de sensors
  desenvolupat para\l.lelament a Cougar és
  TinyDB \parencite{tinyDB,madden05}. A més de les característiques
  descrites per Cougar, aquest sistema s'implica i modifica el procés
  d'adquisició de les dades com ara els instants de temps, la
  freqüència o l'ordre de mostreig. Per exemple donada una consulta
  que vol correlacionar les dades de dos sensors diferents, el sistema
  indica als sensors implicats que han d'adquirir amb la mateixa
  freqüència.


\end{description}




% \url{http://2013.nosql-matters.org/bcn/abstracts/#abstract_gianmarco}

% Streaming data analysis in real time is becoming the fastest and most efficient way to obtain useful knowledge from what is happening now, allowing organizations to react quickly when problems appear or to detect new trends helping to improve their performance. In this talk, we present SAMOA, an upcoming platform for mining big data streams. SAMOA is a platform for online mining in a cluster/cloud environment. It features a pluggable architecture that allows it to run on several distributed stream processing engines such as S4 and Storm. SAMOA includes algorithms for the most common machine learning tasks such as classification and clustering. 




\subsection{Emmagatzematge massiu}
\label{art:massius}

Hi ha sistemes que aborden l'emmagatzematge massiu de les
sèries temporals, és a dir de grans volums de dades, seguint
l'enfocament de les \gls{VLDB}.  A continuació en descrivim alguns
projectes.



\begin{description}



\item[TSDS] \textcite{weigel10} noten la necessitat de mostrar les
  dades en tot el seu rang temporal i no només en un subconjunt com
  molts altres sistemes ofereixen. Desenvolupen el paquet informàtic
  \emph{Time Series Data Server} (TSDS) \parencite{tsds} en què es
  poden introduir les dades de sèries temporals per posteriorment
  consultar-les per rangs temporals o aplicant-hi filtres i
  operacions. La particularitat de TSDS és el fet que incorpora un
  sistema de \emph{cache} per a les consultes que, de forma similar a
  la tècnica descrita per RRDtool, emmagatzema els resultats de les
  consultes segons la resolució i agregació realitzada. D'aquesta
  manera els resultats es poden aprofitar per a altres consultes
  similars. Això no obstant, aquestes consultes s'han de basar en els
  operadors predefinits de TSDS.







\item[SciDB] \textcite{stonebraker09:scidb} estudien l'emmagatzematge
  de dades científiques en \gls{SGBD} basats en models de matrius.
  Les sèries temporals són les dades científiques per exce\l.lència, i
  per tant són les aplicacions que principalment exploren.  Dissenyen
  SciDB, un \gls{SGBD} que implementa les sèries temporals com a
  matrius i permet aconseguir anàlisis multidimensonals amb més bon
  rendiment. Les altres dades que acompanyen les sèries temporals les
  emmagatzemen en taules. Això no obstant, la diferència entre taules
  i matrius sembla massa del nivell físic i comporta ambigüitat per a
  representar les sèries temporals.


% However,
% difference between tables and arrays seems too physical and leads to
% ambiguity when representing time series.  
% Our TSMS model proposes time
% series as firmly based on relational algebra, clarifying this
% ambiguity and describing them coherently in terms of information
% systems theory.



\item[SciQL] \textcite{kersten11,zhang11} descriuen SciQL, un
  llenguatge per a \gls{SGBD} de dades científiques basades en
  matrius, del qual n'estan desenvolupant un
  prototip \parencite{sciql}. És molt semblant a la proposta de SciDB,
  però a diferència SciQL defineix les sèries temporals com una mescla
  de matrius, conjunts i seqüències. A més mostren com gestionar
  algunes característiques de sèries temporals com per exemple la
  regularitat, la interpolació o les consultes de correlació.








\item[OpenTSDB] OpenTSDB \parencite{opentsdb} és un sistema
  d'emmagatzematge distribuït de sèries temporals. Basa
  l'emmagatzematge en Apache Hadoop i HBase, els quals permeten
  distribuir les dades ens diferents nodes. Gràcies a aquests
  sistemes, pot emmagatzemar totes les dades originals ja que és una
  estructura en què és ràpid d'escriure-hi i localitzar les dades, cal
  destacar que HBase crea uns índex potents de les dades i això
  s'aprofita per a indexar l'atribut de temps de les sèries temporals.
  Per a consultar les dades defineixen el concepte d'agregadors, tot i
  que només per a interpolacions lineals, i les operacions d'agregació
  es processen en el mateix moment d'executar la consulta.  Així
  doncs, si bé pot recuperar les dades de forma molt ràpida,
  restringeix les consultes a intervals temporals petits per tal que
  les execucions siguin ràpides. Per tant, és un sistema útil sobretot
  per a visualitzar i comparar intervals temporals petits de diferents
  sèries temporals.




\end{description}





% http://stackoverflow.com/questions/4814167/storing-time-series-data-relational-or-non
%\url{http://en.wikipedia.org/wiki/Time_series_database}






% \todo{wavelet}

% També hi ha l'anàlisi de les sèries temporals amb wavelet analysis. Aquest es basa en anàlisis de la freqüència dels senyals. 

% A multiresolution analysis (MRA) or multiscale approximation (MSA) is the design method of most of the practically relevant discrete wavelet transforms (DWT) and the justification for the algorithm of the fast wavelet transform (FWT). It was introduced in this context in 1988/89 by Stephane Mallat and Yves Meyer and has predecessors in the microlocal analysis in the theory of differential equations (the ironing method) and the pyramid methods of image processing as introduced in 1981/83 by Peter J. Burt, Edward H. Adelson and James Crowley.















%%% Local Variables: 
%%% mode: latex
%%% TeX-master: "main"
%%% End: 







\chapter{Abreviacions i nomenclatura}
%\glsaddall[counter=section]
%\printglossary

\printglossary[type=abreviatura]

\printglossary[type=\acronymtype]

\printglossary[type=notation,style=estil-notation]




%------- Bibliografia ------
\cleardoublepage
%\phantomsection\addcontentsline{toc}{chapter}{\bibname}
\pdfbookmark{\bibname}{bookmark:bibliografia}
\printbibliography
%----------------------------------------------

%\backmatter

\end{document}


%%%%%%%%%%%%%%%%%%%%%%%%%%%%%%%%%%%%%%%%%%%%%%%%%%%%%%%%%%%%%%%%%%%%%%%%%%  
% Model dels sistemes de gestió de bases de dades per sèries temporals.
%
% Copyright (C) 2011-2012 Aleix Llusà Serra.
% 
% This LaTeX document is free software: you can redistribute it and/or
% modify it under the terms of the GNU General Public License as
% published by the Free Software Foundation, either version 3 of the
% License, or (at your option) any later version.
%
% This document is distributed in the hope that it will be useful, but
% WITHOUT ANY WARRANTY; without even the implied warranty of
% MERCHANTABILITY or FITNESS FOR A PARTICULAR PURPOSE. See the GNU
% General Public License for more details.
%
% You should have received a copy of the GNU General Public License
% along with this document. If not, see <http://www.gnu.org/licenses/>.
%
%
% Aleix Llusà Serra
% Departament de Disseny i Programació de Sistemes Electrònics de la Universitat Politècnica de Catalunya (DiPSE-UPC)
% Escola Politècnica Superior d'Enginyeria de Manresa (EPSEM)
% Av. de les Bases de Manresa, 61-73
% 08242 Manresa (Barcelona)
% PAÏSOS CATALANS 
%
% aleix (a) dipse.upc.edu
% 
% El codi font LaTeX del document es troba a 
% <http://escriny.epsem.upc.edu/projects/rrb/>
%%%%%%%%%%%%%%%%%%%%%%%%%%%%%%%%%%%%%%%%%%%%%%%%%%%%%%%%%%%%%%%%%%%%%%%%%%  


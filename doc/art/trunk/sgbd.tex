\section{Sistemes de gestió de bases de dades}
\sectionmark{SGBD}
\label{sec:art:sgbd}


Segons \textcite{date:introduction}, ``una base de dades és un
contenidor informàtic per a una co\l.lecció de dades''. El sistemes
informàtics que tracten amb bases de dades s'anomenen sistemes de
gestió de bases de dades (SGBD, \emph{Data Base Management Systems}) i
tenen els objectius d'emmagatzemar informació i permetre consultar i
afegir aquesta informació per part dels usuaris.  Per complir aquests
objectius, els SGBD ofereixen a l'usuari diferents operacions com per
exemple crear una base de dades, afegir dades o consultar informació a
partir de les dades emmagatzemades.

Els SGBD es poden descriure mitjançant teories matemàtiques que reben
el nom de model de dades, per tant es poden veure els SGBD com una
implementació d'un model de dades.  Segons
\citeauthor{date:introduction}, ``un model de dades és una definició
abstracta, auto continguda i lògica dels objectes, de les operacions i
de la resta que conjuntament constitueixen la màquina abstracta amb la
que els usuaris interaccionen. Els objectes permeten modelar
l'estructura de les dades. Les operacions permeten modelar el
comportament''. Ara bé, \citeauthor{date:introduction} avisa que el
concepte model de dades també s'usa per a definir una estructura
persistent de dades concreta i, per tant, cal distingir adequadament
entre els dos conceptes.  Tal com fa Date, en aquest document parlarem
de model de dades, o simplement de model, en el primer sentit de
màquina abstracta.


Un model de SGBD que ha s'ha consolidat i esdevingut un referent és el
model relacional (\emph{relational}). L'èxit d'aquest model és degut a
que es fonamenta en teories matemàtiques consolidades: la lògica de
predicats i la teoria de conjunts \parencite{date:introduction}.



\textcite{date:introduction} diferencia amb detall els conceptes de
model i d'implementació.  El model d'un SGBD és el model matemàtic tal
com s'ha descrit anteriorment, en canvi un SGBD és la implementació
d'un model de dades, per exemple \emph{PostgreSQL} \parencite{postgresql}.
%Segons l'esquema de comunicació, també es poden anomenar com a sistema servidor de bases de dades, per exemple postgresql.
En aquest context una base de dades és una instància d'un SGBD,
per exemple la base de dades dels estudiants.


Aquesta diferència entre implementació i model aporta independència de dades (\emph{data independence}) \parencite{date:dictionary}. En altres paraules, els models no han de tenir detalls d'implementació ni parlar d'objectius de rendiment. 
\textcite{dbdebunk} detallen algunes confusions actuals sobre la independència entre el model i la implementació.







\subsection{Sistemes relacionals}
\label{sec:estat:sgbdr}

El model relacional va ser proposat per \textcite{codd70} per a
formalitzar els SGBD, els quals quan es basen en aquestes teories
s'anomenen relacionals (SGBDR). A partir de llavors els SGBDR han anat
evolucionat fins a aconseguir una gran solidesa, amb
\textcite{date:introduction,date06,date:dictionary} com a principal
divulgador.



Les implementacions més populars de SGBDR són les que s'anomenen
\emph{SQL} ja que tenen en comú el llenguatge \emph{Structured Query
  Language}. Ara bé, els SGBD \emph{SQL} es desvien considerablement
del model relacional: permeten files duplicades, tenen ordre en les
columnes, permeten valors nuls, etc, sent aquest últim un tema
actualment discutit \parencite{date08:nulls}.

Les diferències entre els SGBD \emph{SQL} i el model relacional han
contribuït a que hi hagi hagut una sèrie de malentesos i
errors, alguns dels quals han estat avaluats i desmentits per
\citeauthor{dbdebunk} en vàries
publicacions \parencite{dbdebunk,date06}.
  

\textcite[cap.~2]{date06} %ch2pp21-22
considera que no hi cap implementació comercial que segueixi fidelment
el model relacional, tot i que esmenta algunes implementacions
prometedores com \emph{Dataphor} o la seva proposta tecnològica
\emph{TransRelational} \parencite{date:transrelational}. A banda,
també cal destacar \emph{Rel} \parencite{rel} com un SGBDR bastant
consolidat.



Actualment \textcite{date:thethirdmanifesto} estan treballant en el
'\emph{Third Manifesto}' com a proposta per a obtenir SGBDR purament
relacionals. Destaquen que, en el model relacional, els tipus de dades
i les relacions són necessaris i suficients per representar qualssevol
dades a nivell lògic. %[date06ch21,369]
Defineixen dos principis bàsics dels SGBDR: l'\emph{Information
  Principle} o \emph{The Principle of Uniform
  Representation} \parencite{date:dictionary}, segons el qual una base
de dades només conté variables relacions, i el principi
d'ortogonalitat entre la teoria de tipus i el model
relacional \parencite[cap.~6]{date06}, segons el qual relacions i
tipus de dades són independents i per tant els atributs de les
relacions admeten qualsevol tipus.  Segons aquest punt de vista, els
tipus de dades són el conjunt de coses de les que podem parlar mentre que les
relacions són proposicions certes sobre aquestes coses.
%In other words, types give us our vocabulary the things we can talk about and relations give us the ability to say things about the things we can talk about. (There's a nice analogy here that might help: Types are to relations as nouns are to sentences.) %[date05ch4secMore on Relations Versus Types]

En la proposta per a obtenir SGBDR purament relacionals
\textcite{date06:_datab_types_relat_model,date:tutoriald} classifiquen
com a \emph{D} els llenguatges que segueixin els principis del
\emph{Third Manifesto}. Concretament, com a exemple d'un llenguatge
\emph{D} estan definint les regles de \emph{Tutorial D}, que ha de
servir pels estudis del model relacional a nivell acadèmic. Aquest
llenguatge ja s'utilitza en alguns SGBDR, com per exemple a
\emph{Rel} \parencite{rel}.


El model relacional ha incorporat conceptes d'altres disciplines. En
destaca sobretot la incorporació de conceptes dels models d'orientació
a objectes com és el cas de l'herència.  Aleshores s'entén que els
SGBDR també es puguin anomenar SGBD objecte/relacionals
(\emph{object/relational})
\parencite{date02:foundation}.  Tot i així, \textcite[cap.~6]{date06}
manifesta i avisa de l'ús de la mateixa terminologia amb significat
diferent entre el model relacional i l'orientació a objectes, sobretot
pel que fa als termes valor i variable. %[date06ch6pp91]
La seva hipòtesi a aquestes diferències és que el model relacional és
un model de dades i el model d'orientació a objectes és més proper a
un model
d'emmagatzematge. % 'the object model' is closer to being a model of
                  % storage than it is to being a model of
                  % data. [date06ch6pp92]

% A la \autoref{tab:sgbd:relacional-objectes} es resumeix la possible
% equivalència lògica dels conceptes entre el model relacional i
% l'orientació a objectes tal com Date exposa al capítol 6, tot i que
% cal tenir en compte que la semblança és difusa.

% \begin{table}
% \centering
% \begin{tabular}[ht]{ll}
%   relacional & objectes \\\hline \hline
%   tipus & tipus, classe, interfície \\\hline
%   representació & classe, atributs, propietats \\\hline
%   valor, objecte, instància & valor, estat, objecte/instància immutable/estàtic \\\hline
%   variable & valor, objecte/instància mutable/dinàmic \\\hline
%   referència & variable \\\hline
%   operador & funció, mètode \\\hline
% \end{tabular}
% \caption{Possible equivalència lògica de termes entre el model relacional i l'orientació a objectes \parencite[cap.~6]{date06}.}
% \label{tab:sgbd:relacional-objectes}
% \end{table}

% Relacional: tipus | representació |  valor, objecte, instància  | variable  | referència (adreça continguda en una variable) | operadors (de lectura i de modificació)
% Objectes: tipus, classe (tipus amb atributs i mètodes), interfície | classe, atributs,propietats  |  valor, estat, objecte/instància immutable/estàtic |  valor, objecte/instància mutable/dinàmic  | variable | funcions,mètodes (funcions dins de classes) (purs o modificadors)



\subsubsection{Extensió del model}
%Com s'ha d'estendre el model relacional?

El model relacional ha evolucionat però no es considera que hi hagi hagut
cap revolució des de la seva aparició
\parencite[cap.~19]{date06}. %[date06ch19pp254]
Consideren que el model relacional és bastant complet i que segueix
evolucionant en la comprensió de les teories i els conceptes que hi
intervenen, com per exemple la recent àlgebra relacional
'A' \parencite[ap.~A]{date06:_datab_types_relat_model}.  En aquest
context d'evolució, es contemplen les investigacions que poden
estendre el model relacional, és a dir, aconseguir abstraccions més
generals de les dades \parencite[cap.~25]{date06}. %[date06ch25pp441]

% En l'extensió hi ha el què després s'aplica a sobre: optimitzacions,
% redundàncies, etc.

En el sentit d'extensió, també cal contemplar la definició de nous
tipus de dades, els quals estenen els SGBD en funcionalitat.  Aquests
nous tipus de dades poden afegir estructures i operadors que ja siguin
expressables amb l'àlgebra relacional. No obstant, un bon model d'un
tipus de dades serveix per augmentar el nivell d'abstracció en el
tractament dels conceptes relacionats amb aquestes
dades
\parencite{date02:_tempor_data_relat_model}. %[date02:_tempor_data_relat_model:prefaceppxix]

Com s'ha dit anteriorment, la teoria de tipus i el model relacional
són ortogonals: el model relacional requereix que hi hagi un 'sistema'
de tipus de dades però diu molt poc de la naturalesa d'aquest sistema,
si bé el model relacional defineix que com a mínim hi ha d'haver el
tipus booleà i el tipus
relació \parencite{date:thethirdmanifesto}. Pel que fa a implementar
el tipus de dades en els SGBD, els quals aleshores també s'anomenen
SGBD objecte/relacionals, destaquen les primeres propostes fetes per
\textcite{stonebraker86} per tal que els usuaris puguin definir els
seus propis tipus de dades i les de \textcite{seshadri98:_enhan} que
estudia la definició de tipus de dades complexos per tal que es puguin
tractar eficientment.

\todo{ampliació de tipus als SGBD molt important a GIS [bollaert06:thesis]}

 




\subsection{Altres sistemes}

Les crítiques als SGBDR, sobretot les degudes als SGBD
\emph{SQL}, han contribuït a voler explorar altres models de
SGBD \parencite{stonebraker09}. Aquests models presenten diferents
maneres de representar les dades: llistes, seqüències, enllaços,
matrius, etc.
%[date06pp116,134]

Tot i així, \textcite[cap.~21--25]{date06} considera que els nous
models de SGBD, a vegades anomenats post-relacionals, no estan fundats
tant sòlidament en teories matemàtiques i la lògica de predicats com
el model relacional i pronostica que ens els propers cent anys els
SGBD encara estaran basats en el model
relacional. %[date06ch19pp354,date06ch20pp365]
Considera la possibilitat, tot i que remota, que es pugui definir un
model més potent que el relacional però que no hi ha cap indici que
cap definició dels nous model tingui la mateixa potència que el
relacional. Per tant, aconsella que per ara els SGBD no s'allunyin del
model relacional. %[date06ch25,ch21pp379-380]




Recentment ha aparegut un nou corrent en l'àmbit dels SGBD que
s'anomena NoSQL (\emph{Not Only SQL}) amb l'objectiu de sobrepassar
les limitacions dels SGBDR \parencite{edlich:nosql,stonebraker10}.  A
l'espera que Date valori aquest corrent, cal tenir present els seus
apunts sobre sistemes relacionals contra sistemes no
relacionals \parencite[part 7]{date06}, sobretot pel que fa al
concepte de l'\emph{Information Principle} i que SQL no és un bon
referent pels SGBDR. És a dir, s'ha d'entendre NoSQL com una crítica a
les implementacions comercials actuals del model relacional, una
crítica que pot estar motivada per l'ús de SQL per part d'aquest
productes.  No es pot entendre, però, el NoSQL com una crítica al
model relacional ja que els objectius parlen de millorar el rendiment
dels SGBD, cosa només atribuïble a les implementacions però no al
model. Precisament, actualment del model relacional destaca la
proposta de \citeauthor{date:tutoriald} d'un llenguatge,
\emph{Tutorial D}, que no és SQL.


El corrent de NoSQL també critica l'adequació dels productes actuals.
Els SGBD NoSQL apunten els SGBDR actuals per voler ser \emph{one size
  fits all} \parencite{stonebraker07,stonebraker09} però que cada
aplicació té els seus requisits i per tant una mateixa implementació
no pot ser bona per a tots el camps.  En aquesta mateixa línia els
models pels SGBD prenen més sentit que mai ja que permeten mantenir
una definició comuna per a moltes implementacions.


Segons es desprèn de \textcite{date06} fins a l'actualitat només hi ha
hagut un model consolidat pels SGBD: el model relacional.  Ara, en el
corrent NoSQL també es parla de nous models de
SGBD \parencite{edlich:nosql,stonebraker09:scidb}.
\citeauthor{date06} ha avaluat que alguns nous models recuperen
intents fallits en el passat tot i que es poden representar amb el
model relacional, per exemple els SGBD XML basats en estructures
d'arbre \parencite[cap.~14]{date06} \parencite[cap.~27]{date04:introduction8}
o els ODMG basats en objectes \parencite[cap.~27]{date06}. Tot i així,
en un futur cal estar atents per si alguns d'aquests models joves de
SGBD arriben a consolidar-se i poden esdevenir tant potents com el
model relacional.





\subsection{Dades temporals}

%parlar de docs anteriors de dades temporals

%Precisament,
Una de les extensions importants del model relacional s'ha produït amb
l'estudi de les dades
temporals \parencite{date02:_tempor_data_relat_model}. Amb el model de
dades temporals basat en el model relacional s'obtenen SGBDR capaços
d'emmagatzemar i consultar dades històriques.

El model de les dades
temporals \parencite{date02:_tempor_data_relat_model} es basa en
relacions i intervals per representar les dades històriques. Cada
relació s'estén amb un atribut que és un interval temporal que indica
el rang temporal de validesa de les proposicions. A més el model també
defineix les operacions necessàries per tractar amb les dades
temporals. Aquestes operacions són extensions de l'àlgebra relacional.
El principal objectiu del model de dades temporals és poder tenir
suport per a les dades històriques en els SGBDR.
\textcite[cap.~28]{date06} compara aquest model per dades temporals
amb altres aproximacions que s'han fet no basades en el model
relacional.



El model de dades temporals representa les dades històriques amb el
temps vàlid i temps de
transacció \parencite[cap.~15]{date02:_tempor_data_relat_model}, el
que es coneix com a dades bitemporals.  Així doncs, els SGBD per dades
bitemporals no es consideren adequats com a SGBD per sèries temporals
ja que els primers estan pensats per a històrics, es descriuen amb
intervals temporals, i els segons per anàlisi d'observacions
seqüencials, es descriuen amb instants temporals.
\textcite{schmidt95} arriben a aquesta conclusió després de comparar
els SGBD bitemporals amb els de sèries temporals, tot i que per a
desenvolupaments futurs observen que hi aspectes temporals comuns
entre els dos sistemes i es pregunten si es podran trobar sistemes que
els englobin a tots dos o cadascun necessitarà sistemes específics.


%a algun lloc deia Date que el seu model per dades temporals no era del tot adequat per les sèries temporals?, potser perquè els intervals no estaven pensat per a instants? date02ch16.8?



\subsection{Conclusió}


Actualment l'àmbit informàtic de SGBD se centra en les
implementacions, com ho demostra el nou corrent NoSQL concentrat en
trobar models d'implementacions que tinguin bon rendiment. A tal
efecte la recerca es concentra en temes de garantia de propietats ACID
(\emph{atomicity, consistency, isolation, durability}), d'optimització
de consultes, d'emmagatzematge de grans volums de dades, de consultes
via web, de distribució de bases de dades, de reduir la despesa en
energia, etc. \parencite{stonebraker07,stonebraker10}, la qual cosa és
exce\l.lent per a disposar d'un SGBD adequat a cada aplicació.
\textcite{haerder05:_dbms_archit} descriu diferents models
d'implementació per als SGBD, ja que indica que per obtenir bon
rendiment la implementació d'un SGBD s'ha d'estudiar per cada
aplicació. És més, una implementació d'un SGBD que vulgui obtenir un
bon rendiment en una determinada aplicació potser no pot implementar
el model de dades complet sinó que només una part, com per exemple en
els sistemes
encastats \parencite{saake09:_downs_data_manag_embed_system}.

Per altra banda, l'àmbit matemàtic de SGBD, amb el model relacional
com a màxim exponent, se centra en els conceptes teòrics, és a dir
respon a la pregunta de què són els SGBD. Recerca en millorar-ne la
comprensió, en obtenir la màxima potència i facilitat de cara a la
gestió de dades per part de l'usuari o en obtenir nous models. Tot i
així, actualment encara no s'ha trobat cap altre model que tingui la
mateixa potència que el relacional. Cal destacar que tot i que el
model relacional té conceptes madurs i consolidats, i que a més han
tingut èxit amb els SGBD SQL, s'obre una nova perspectiva amb
l'evolució de conceptes que proposa el \emph{Third Manifesto},
especialment amb \emph{Tutorial D} i les implementacions que comencen a prendre
cos a nivell acadèmic.



















%%% Local Variables: 
%%% mode: latex
%%% TeX-master: "main"
%%% End: 

% LocalWords:  monitoratge SGBD SGBDR

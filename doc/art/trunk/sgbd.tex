\section{Sistemes de gestió de bases de dades}
\sectionmark{SGBD}
\label{sec:art:sgbd}

\todo{faltar canviar sigles a gls}


Segons \textcite{date:introduction}, ``una base de dades és un
contenidor informàtic persistent per a una co\l.lecció de dades''. El
sistemes informàtics que tracten amb bases de dades s'anomenen
sistemes de gestió de bases de dades (SGBD, \emph{Data Base Management
  Systems}) i tenen els objectius d'emmagatzemar informació i permetre
consultar i modificar aquesta informació per part dels usuaris.  Per
complir aquests objectius, els SGBD ofereixen a l'usuari diferents
operacions com per exemple crear una base de dades, afegir dades o
operar amb les dades emmagatzemades. 

Els àmbits d'aplicació dels SGBD són varis: operacions repetitives i
rutinàries de producció, anomenades \emph{online transaction
  processing}; sistemes per a prendre decisions empresarials, a
vegades anomenats \emph{data warehouse}; processament de dades
científiques; etc.  Alguns dels avantatges de gestionar aquestes dades
en bases de dades són: evitar la disgregació de la informació i
tenir-la perfectament organitzada, poder compartir la mateixa
informació entre diverses aplicacions, garantir la consistència i la
integritat de les dades i evitar redundàncies innecessàries, afegir
seguretat a la gestió de les dades o optimitzar les consultes que
l'usuari so\l.licita.


Els SGBD es poden descriure mitjançant teories matemàtiques que reben
el nom de \emph{model de dades}.  Segons
\citeauthor{date:introduction}, ``un model de dades és una definició
abstracta, auto continguda i lògica dels objectes, de les operacions i
de la resta que conjuntament constitueixen la màquina abstracta amb
què els usuaris interaccionen. Els objectes permeten modelar
l'estructura de les dades. Les operacions permeten modelar el
comportament''. Ara bé, \citeauthor{date:introduction} avisa que el
concepte \emph{model de dades} també s'usa per a definir una
estructura o esquema persistent de dades concreta i, per tant, cal distingir
adequadament entre tots dos significats.  Tal com fa Date, en aquest
document parlarem de model de dades, o simplement de model, en el
primer sentit de màquina abstracta. També distingeix entre els
conceptes de \emph{dades} --allò que està emmagatzemat a la base de
dades-- i \emph{informació} --el significat que algú dóna a aquestes
dades.


Un model de SGBD que ha s'ha consolidat i ha esdevingut un referent és
el model relacional (\emph{relational model}). L'èxit d'aquest model
és degut principalment que es fonamenta en teories matemàtiques
consolidades: la lògica de predicats i la teoria de
conjunts \parencite{date:introduction}. En base al model relacional es
va definir el llenguatge \emph{Structured Query Language} (SQL) per
operar amb bases de dades que ha esdevingut un estàndard en molts
SGBD.


Els SGBD se solen dissenyar amb una arquitectura de tres nivells: el
físic, el lògic i el d'usuari \parencite{date:introduction}. 
\begin{itemize}

\item El nivell d'usuari o extern agrupa les eines que tenen
  disponibles els usuaris per a interactuar amb la base de dades, per
  exemple en el cas relacional SQL pertany a aquest nivell.

\item El nivell lògic o conceptual és l'abstracció formal dels
  conceptes dels SGBD. En aquest nivell hi pertanyen els models de
  dades, per exemple el mateix model relacional en el cas relacional.

\item El nivell físic o intern agrupa la programació informàtica de
  com s'han d'emmagatzemar físicament les base dades i de com s'has
  d'executar les operacions. Per exemple en aquest nivell apareixen
  els registres de memòria, punters, mètodes d'accés als fitxers,
  etc., conceptes que no tenen res de relacional.
\end{itemize}


Una bona diferenciació entre els tres nivells d'arquitectura aporta
independència a les dades (\emph{data
  independence}) \parencite{date:dictionary}. Date considera que és
una de les propietats més importants que han de complir els SGBD. De
forma resumida, la independència a les dades significa que el nivell
lògic no ha de contenir detalls d'implementació ni parlar d'objectius
de rendiment sinó que aquests són part del nivell físic. Així ha de
ser possible canviar el nivell físic sense afectar el nivell lògic.
mantenint.  Així doncs, un model de dades concret pot tenir diverses
implementacions en el nivell físic, per exemple
\emph{PostgreSQL} \parencite{postgresql} per al model
relacional. \textcite{dbdebunk} detallen algunes confusions actuals
sobre la independència entre el model i la implementació.

 


\subsection{Sistemes relacionals}
\label{sec:estat:sgbdr}

El model relacional va ser proposat per \textcite{codd69,codd70} com una
teoria abstracta de dades per tal de formalitzar els SGBD amb teories
matemàtiques consolidades.  El model relacional va significar un gran
canvi en la recerca en SGBD ja que, a diferència dels models antics,
possibilitava l'estudi dels problemes amb teories matemàtiques: la
lògica i l'àlgebra de
conjunts \parencite{atzeni13:relational_model_dead}.  A partir de
llavors el model relacional ha evolucionat fins a aconseguir una gran
solidesa, amb \textcite{date04:introduction8,date06,date:dictionary} com
a principal divulgador.  Quan els SGBD es basen en el model relacional
s'anomenen relacionals (SGBDR).



El model relacional defineix el nucli dels SGBD en tres parts:
\begin{itemize}
\item Estructural: les relacions com a estructura principal per a
  representar les dades. Els tipus de dades també són necessaris per a
  representar les dades però no es defineixen en l'estructura
  principal sinó que són considerats ortogonals
  (v.~\autoref{sec:art:relacional-tipus}).

\item Manipulació: operacions sobre les relacions i que resulten en
  noves relacions. Són definides dualment a partir de l'àlgebra de
  conjunts i de la lògica, anomenades àlgebra relacional i càlcul
  lògic respectivament. Totes dues tenen una definició independent
  però són equivalents.

\item Integritat: regles d'integritat o restriccions que han de
  complir les variables relació. La integritat es basa en aplicar
  operacions que han de retornar el valor cert. La integritat s'ha de
  complir sempre, normalment es comprova durant les assignacions.  Per
  exemple la clau primària és una regla d'integritat.
\end{itemize}


La definició de les relacions inicialment es basava en el concepte
matemàtic homònim en el sentit de producte cartesià de conjunts, però
el model relacional ha anat evolucionat i ara ja no són exactament el
mateix. Tampoc no s'ha de confondre el terme relació (\emph{relation})
del model relacional amb el terme relació de parentiu
(\emph{relationship}). El primer és el concepte basat en conjunts que
definim a continuació mentre que el segon és el concepte de parentiu
entre entitats: una a molts, molts a molts, etc. De fet, les
estructures de parentiu és una de les moltes dades que poden ser
expressades en el model relacional. Alguns autors del model relacional
prefereixen el terme taula relacional (\emph{relational table}) en
comptes de relació \parencite{dbdebunk}.



Les relacions es defineixen com una parella de capçalera
(\emph{heading}) i cos (\emph{body}). El cos és un conjunt de tuples
on cada tuple és un conjunt de parelles atribut i valor. Tots els
tuples d'una mateixa relació tenen els mateixos atributs, així es
distingeix entre la capçalera de la relació --els atributs-- i el cos
de la relació --els tuples.

Per exemple una relació entre un nom i una edat és
$r_1=(\{\text{nom},\text{edat} \}, \{
\{(\text{nom},\text{a}),(\text{edat},21)\},
\{(\text{nom},\text{b}),(\text{edat},23) \} \})$.  Simplificant i
sense explicitar que els atributs no tenen ordre, la mateixa relació
es pot expressar de forma més compacta $r_1=(
(\text{nom},\text{edat}), \{ (\text{a},21),(\text{b},23) \})$.  En la
definició estructural del model relacional, els valors sempre
pertanyen a un tipus de dades i cada atribut és restringit a un únic
tipus de dades. Així, de manera més completa hauríem d'escriure la
capçalera de la relació $r1$ com $\{\text{nom}: \text{text},\,
\text{edat}:\text{enter} \}$.


En el context lògic del model relacional, les relacions tenen la
interpretació següent: les capçaleres són predicats i els tuples són
proposicions certes per al predicat. En un context informàtic també es
pot interpretar que les capçaleres són funcions amb paràmetres i els
tuples contenen els arguments que fan certes les funcions. 
%També a vegades definit com a intensió/extensió de la relació
Aquesta interpretació lògica és la que realment estableix el
significat de les relacions en un context determinat.  Així, per
exemple, la capçalera de la relació $r_1$ podria correspondre al
predicat ``L'estudiant \emph{nom} té \emph{edat} anys'' i les
proposicions certes són: ``L'estudiant a té 21 anys'' i ``L'estudiant
b té 23 anys''.  Les relacions es defineixen segons el principi de
\emph{Closed World Assumption}; és a dir que els tuples que apareixen
són proposicions certes i els que no apareixen són proposicions
falses. Així, per exemple, podem dir que la proposició ``L'estudiant a
té 22 anys'' és falsa.



Les relacions es poden representar gràficament com a taules, per
exemple a la \autoref{fig:art:relacio:taula} es visualitza la relació
$r_1$.  Així, els conceptes de taules s'associen als de relacions i,
informalment, les relacions s'anomenen taules, els tuples, files o
registres, i els atributs, columnes o camps.

\begin{figure}[tp]
  \centering
  \begin{tabular}[c]{|c|c|}
    \multicolumn{2}{c}{$r_1$} \\ \hline
    nom  & edat \\ \hline
    a  & 21 \\
    b  & 23 \\ \hline
  \end{tabular} 
  \caption{Visualització com a taula d'una relació}
  \label{fig:art:relacio:taula}
\end{figure}

El nombre de tuples d'una relació s'anomena cardinal
(\emph{cardinality}) i el nombre d'atributs, grau
(\emph{degree}). Així doncs, la relació $r_1$ té cardinal 2 i grau 2.
Hi ha només dues relacions que tenen grau zero. Tenen un nom
específic, són la relació amb la capçalera i el cos buits
$\text{table\_DUM} = (\{\},\{\})$ i la relació amb la capçalera buida
i un tuple buit $\text{table\_DEE} = (\{\},\{\{\}\})$. Aquestes, però, no
tenen una representació clara com a taula.  



Pel que fa als operadors, en el cas de l'àlgebra relacional estan
fortament relacionats amb l'àlgebra de conjunts. Així hi ha els
operadors habituals de conjunts, com per exemple la unió, la
diferència, la intersecció o el producte; i altres d'específics per a
les relacions, com per exemple la projecció, la selecció, la junció o
el reanomena \parencite[cap.~7]{date04:introduction8}.  En el cas del
càlcul lògic, també anomenat càlcul relacional, estan relacionats amb
la lògica de predicats. Així per exemple hi ha un operador de rang per
recórrer el conjunt de tuples, el quantificador existencial o el
quantificador universal.\parencite[cap.~8]{date04:introduction8}.



En el model relacional es distingeix entre les variables relació o
relvar (\emph{relation variable}) i els valors relació (\emph{relation
  value}). El valor relació s'anomena simplement relació, com ja hem
definit fins ara.  Una relvar és una variable a les qual s'assigna una
relació.  L'assignació a les relvar es defineix amb el símbol $:=$ a
diferència de la igualtat algebraica o la definició de variables
algebraiques de símbol $=$, com ja hem usat.  Les relvar són els
objectes bàsics d'emmagatzematge a les bases de dades i per tant són
els objectes bàsics als quals s'apliquen les regles
d'integritat. També hi ha operacions que treballen sobre les relvar,
per exemple la inserció, l'actualització o l'esborrat, que són àlies
de combinacions de l'assignació amb altres operacions relacionals.
Unes relvar especials són les vistes. Les vistes són relvar derivades
a partir d'operacions a altres relvar; és a dir són àlies
d'expressions relacionals i actuen com a relvar en altres
expressions. A causa d'això les vistes també s'anomenen relvar
virtuals mentre que les relvar que no són derivades s'anomenen relvar
base. Les relvar s'emmagatzemen en una relació especial de les bases
de dades: el catàleg.






\subsubsection{Extensió del model amb nous tipus}
\label{sec:art:relacional-tipus}
%Com s'ha d'estendre el model relacional?

El model relacional ha evolucionat però no es considera que hi hagi hagut
cap revolució des de la seva aparició
\parencite[cap.~19]{date06}. %[date06ch19pp254]
Consideren que el model relacional és bastant complet i que segueix
evolucionant en la comprensió de les teories i els conceptes que hi
intervenen, com per exemple la recent àlgebra relacional
'A' \parencite[ap.~A]{date06:_datab_types_relat_model}.  En aquest
context d'evolució, es preveuen les investigacions que poden estendre
el model relacional. Aquestes investigacions estudien propietats de
les dades, com per exemple seguretat, redundància o optimitzacions de
les consultes, a partir del nucli del model relacional i permeten
aconseguir abstraccions més generals de les
dades \parencite[cap.~25]{date06}. %[date06ch25pp441]


En el sentit d'extensió cal destacar la definició de nous tipus de
dades, els quals estenen els SGBD en funcionalitat.  Els tipus de
dades (\emph{data type}, també anomenats dominis, tipus de dades
abstracte o solament tipus, són la definició d'un conjunt de
valors. Cada tipus té associat un conjunt d'operadors, en alguns casos
fins i tot s'entén que la definició tipus inclou aquests operadors.
De manera informal \parencite{date04:introduction8} fa correspondre
els conceptes de tipus i de relacions amb els conceptes lingüístics de
noms i frases.


Com s'ha dit anteriorment, la teoria de tipus i el model relacional
són ortogonals: el model relacional requereix que hi hagi un sistema
de tipus de dades però diu molt poc de la naturalesa d'aquest sistema,
si bé el model relacional defineix que com a mínim hi ha d'haver el
tipus booleà i el tipus
relació \parencite{date:thethirdmanifesto}. 
Pel que fa a implementar
els tipus de dades en els SGBD, destaquen les primeres propostes fetes per
\textcite{stonebraker86} per tal que els usuaris puguin definir els
seus propis tipus de dades i les de \textcite{seshadri98:_enhan} que
estudia la definició de tipus de dades complexos per tal que es puguin
tractar eficientment.


Normalment els SGBD tenen uns tipus predefinits, com els enters, els
reals o els caràcters. Això no obstant, els tipus de dades poden
definir qualssevol nous valors, com per exemple matrius, documents de
text, imatges o fins i tot relacions.  Aquests nous tipus de dades
poden afegir estructures i operadors que ja siguin expressables amb
l'àlgebra relacional o bé també poden definir-se a partir de l'àlgebra
relacional. No obstant això, disposar d'un bon model d'un tipus de
dades serveix per augmentar el nivell d'abstracció en el tractament
dels conceptes relacionats amb aquestes dades
\parencite{date02:_tempor_data_relat_model}. %[date02:_tempor_data_relat_model:prefaceppxix]


El tipus de dades d'una relació és determinat per la seva capçalera.
Així doncs, la relació $r1$ és de tipus relació $\{\text{nom}:
\text{text},\, \text{edat}:\text{enter} \}$.  El tipus relació pot ser
usat a qualsevol definició on puguin ser usats els altres tipus de
dades: definicions de variables, operadors, nous tipus de dades,
etc. Tot i així la definició del tipus relació és molt rígida quant
als tipus dels seus atributs, cosa que, per exemple, no permet definir
nous operadors genèrics per a qualsevol relació.  Recentment ha
aparegut una proposta preliminar de
\textcite{darwen13:generic_relation_type} per a solucionar aquest
problema. Aquesta proposta permetria definir capçaleres genèriques de
relacions mitjançant el símbol asterisc; és a dir capçaleres amb
atributs i tipus genèrics. Així per exemple permetria usar el tipus
relació $\{ * \}$ que determinaria una relació de qualsevol tipus; és
a dir que el conjunt de valors del tipus relació $\{ * \}$ contindria
totes les relacions possibles: la table\_DUM, la table\_DEE, la $r1$,
etc. Això no obstant, encara cal flexibilitzar més les definicions
dels tipus relació. Per exemple no és possible definir nous tipus de
dades que siguin subtipus del tipus relació, cosa que permetria que
els valors d'aquests subtipus funcionessin com a arguments en els
operadors predefinits de l'àlgebra relacional.




En algunes extensions, el model relacional ha incorporat conceptes
d'altres disciplines. En destaca sobretot la incorporació de conceptes
dels models d'orientació a objectes en el cas dels tipus de dades i de
l'herència.  Des d'aquesta perspectiva, els SGBDR també s'anomenen
SGBD objecte/relacionals (\emph{object/relational})
\parencite{date02:foundation}.  Tot i així, \textcite[cap.~6]{date06}
avisa de l'ús de la mateixa terminologia amb significat lleugerament
diferent entre el model relacional i l'orientació a objectes, sobretot
pel que fa als termes valor, variable i tipus. %[date06ch6pp91]
Les diferències provenen principalment del fet que el model relacional
és un model de dades i el model d'orientació a objectes és un
paradigma de programació i és més proper a un model d'emmagatzematge.
% 'the object model' is closer to being a model of storage than it is
% to being a model of data. [date06ch6pp92]





% A la \autoref{tab:sgbd:relacional-objectes} es resumeix la possible
% equivalència lògica dels conceptes entre el model relacional i
% l'orientació a objectes tal com Date exposa al capítol 6, tot i que
% cal tenir en compte que la semblança és difusa.

% \begin{table}
% \centering
% \begin{tabular}[ht]{ll}
%   relacional & objectes \\\hline \hline
%   tipus & tipus, classe, interfície \\\hline
%   representació & classe, atributs, propietats \\\hline
%   valor relació, objecte, instància & valor, estat, objecte/instància immutable/estàtic \\\hline
%   relvar & valor, objecte/instància mutable/dinàmic \\\hline
%   relvar??? & variable \\\hline
%   operador & funció, mètode \\\hline
% \end{tabular}
% \caption{Possible equivalència lògica de termes entre el model relacional i l'orientació a objectes \parencite[cap.~6]{date06}.}
% \label{tab:sgbd:relacional-objectes}
% \end{table}

% Relacional: tipus | representació |  valor, objecte, instància  | variable  | referència (adreça continguda en una variable) | operadors (de lectura i de modificació)
% Objectes: tipus, classe (tipus amb atributs i mètodes), interfície | classe, atributs,propietats  |  valor, estat, objecte/instància immutable/estàtic |  valor, objecte/instància mutable/dinàmic  | variable | funcions,mètodes (funcions dins de classes) (purs o modificadors)











Finalment, com a exemple de nous tipus de dades, cal destacar que en
l'àmbit dels sistemes d’informació geogràfica (SIG) hi va haver un
gran impacte quan els SGBDR es van estendre amb el tipus de dades
espacials \parencite{nunes13:icc_geospacial}.  Inicialment els SIG
usaven programes informàtics específics per a la gestió de les dades
geogràfiques. Hi va haver una recerca intensa en els SGBDR per tal
d'establir models que permetessin emmagatzemar i consultar dades
geomètriques i les propietats espacials de la informació
geogràfica. D'aquesta manera es van definir el que es coneix com a
bases de dades espacials o geoespacials, les quals han permès
desenvolupar tots els avantatges dels SGBD en els SIG. %[bollaert06:thesis]

Una altra extensió de tipus important en els SGBDR és la de les dades
temporals, les quals detallem a la \autoref{sec:art:dades_temporals}.






\subsubsection{Implementacions relacionals}


Les implementacions més populars de SGBDR són les que ofereixen a
l'usuari el llenguatge SQL per a operar amb les bases de dades, per
exemple MySQL o PostgreSQL, a continuació ens hi referim com a SGBD
SQL.  

Segons \textcite{datedarwen13:notosql_notonosql} els SGBD SQL es
desvien considerablement del model relacional: permeten files
duplicades, tenen ordre en les columnes, permeten valors
nuls \parencite{date08:nulls}, etc.  Les diferències entre els SGBD
SQL i el model relacional han contribuït que hi hagi hagut diversos
malentesos i errors, alguns dels quals han estat avaluats i
desmentits \parencite{dbdebunk,date06}.
  

Actualment \textcite{date:thethirdmanifesto} estan treballant en el
'\emph{Third Manifesto}' com a proposta per a obtenir SGBDR purament
relacionals. Destaquen que, en el model relacional, els tipus de dades
i les relacions són necessaris i suficients per representar qualssevol
dades a nivell lògic. %[date06ch21,369]
Defineixen dos principis bàsics dels SGBDR: l'\emph{Information
  Principle} o \emph{The Principle of Uniform
  Representation} \parencite{date:dictionary}, segons el qual una base
de dades només conté variables relacions, i el principi
d'ortogonalitat entre la teoria de tipus i el model
relacional \parencite[cap.~6]{date06}, segons el qual relacions i
tipus de dades són independents i a meś els atributs de les relacions
admeten qualsevol tipus.  
%In other words, types give us our vocabulary the things we can talk about and relations give us the ability to say things about the things we can talk about. (There's a nice analogy here that might help: Types are to relations as nouns are to sentences.) %[date05ch4secMore on Relations Versus Types]

En la proposta per a obtenir SGBDR purament relacionals
\textcite{date06:_datab_types_relat_model,date:tutoriald} classifiquen
com a \emph{D} els llenguatges que segueixin els principis del Third
Manifesto. Particularment, defineixen un llenguatge que compleix amb
els principis D anomenat \emph{Tutorial D}, amb l'objectiu que s'usi
pels estudis del model relacional a nivell acadèmic. Utilitzen aquest
llenguatge en les seves obres per a exemplificar els conceptes
del model relacional, tot i que en alguns casos també ofereixen
exemples amb SQL.


\textcite[cap.~2]{date06} %ch2pp21-22
considera que actualment no hi cap implementació comercial, per
exemple les de SGBD SQL, que segueixi fidelment el model relacional.
Si bé recull diverses implementacions que
S'estan desenvolupant i que segueixen les
especificacions del Third
Manifesto \parencite[Projects]{date:thethirdmanifesto.com}.  Algunes
d'aquestes implementacions són:
\begin{itemize}
\item Rel \parencite{rel}, que és un dels més consolidats i usa
  Tutorial D com a llenguatge
\item Dee \parencite{dee}, que és una implementació molt diferent a
  les altres ja que defineix el llenguatge relacional com una extensió
  de Python
% Dee és una implementació en Python de D
% http://www.quicksort.co.uk/ 
%És interessant, no es basa en fer una implementació declarativa sinó en fer una implementació relacional integrada a Python
\end{itemize}
% tot i que esmenta algunes
% implementacions prometedores com \emph{Dataphor} o la seva proposta
% tecnològica \emph{TransRelational} \parencite{date:transrelational}.










\subsection{Recerca actual}

Actualment es poden distingir quatre corrents majoritaris d'opinió en
l'àmbit dels SGBD: els SGBD SQL, el NoSQL, el Third Manifesto i el
NewSQL.

Els SGBD SQL són els productes tradicionals que van implementar el
model relacional inicial. Aquests productes estan molt consolidats i
són els més usats. Algunes propietats que tenen són les següents:
garantia de propietats ACID (\emph{atomicity, consistency, isolation,
  durability}) amb transaccions, optimització de consultes,
emmagatzematge de grans volums de dades o gestió de seguretat i
permisos.  Això no obstant, ja no són considerats com l'única solució
per a tots els problemes de bases dades, cosa que se sentencia amb el
lema \emph{one size does not fit
  all} \parencite{stonebraker07,stonebraker09}; és a dir que en cada
camp d'aplicació les bases de dades tenen uns requisits diferents i
per tant una determinada implementació pot ser eficient en un camp
concret però no en tots. A més, cada camp d'aplicació pot requerir uns
requisits diferents d'eficiència: temps d'execució ràpid, poca despesa
d'energia, poques dades transmeses per la xarxa, etc. Per exemple en
els sistemes encastats potser no es pot implementar tot el model de
dades sinó que només una
part \parencite{saake09:_downs_data_manag_embed_system}.



El NoSQL és un corrent modern en l'àmbit dels SGBD que té com a
objectiu millorar les limitacions d'eficiència dels SGBD
SQL \parencite{edlich:nosql,stonebraker10}. El terme NoSQL és un nom
propi lexicalitzat per al lema del corrent \emph{Not Only SQL}.  Tot i
que hi ha molta varietat en els productes NoSQL, s'ha d'entendre NoSQL
com una crítica a les implementacions comercials actuals del model
relacional, principalment els SGBD SQL, i no com una crítica al model
relacional ja que els objectius parlen de millorar el rendiment dels
SGBD, cosa que és només atribuïble al nivell físic però no al nivell
lògic.  Alguns investigadors de SGBDR veuen compatible la coexistència
dels SGBD SQL amb els NoSQL perquè tenen objectius molt diferents
\parencite{atzeni13:relational_model_dead}.  En el corrent NoSQL hi ha
molts productes diferents i s'agrupen segons el model que utilitzen,
alguns exemples de models NoSQL són \parencite{edlich:nosql}: grafs, 
objectes, clau/valor, columna o semiestructurat en arbres com XML.
Per exemple ZODB \parencite{zodb} és un sistema amb model d'objectes o
Hadoop \parencite{hadoop} té model de columna i a més és basa en un
model de programació para\l.lela de les consultes anomenat
MapReduce. %[lemel08]


El Third Manifesto, que ja hem comentat, principalment defineix els
requisits D per a les implementacions. Defensa que els models teòrics
dels SGBD tenen més sentit que mai ja que permeten mantenir una
definició comuna per a les diverses implementacions que hi pugui
haver.  És un corrent molt crític amb els SGBD SQL i els NoSQL perquè
s'allunyen del model relacional
teòric \parencite{datedarwen13:notosql_notonosql}, sobretot pel que fa
al concepte de l'\emph{Information Principle} \parencite[part
7]{date06}. Consideren que alguns models del NoSQL recuperen models
obsolets, com el jeràrquic o el de xarxa, ja explorats en el passat;
en aquest sentit avaluen alguns productes NoSQL com els SGBD XML
basats en estructures
d'arbre \parencite[cap.~14]{date06} \parencite[cap.~27]{date04:introduction8}
o els ODMG basats en objectes \parencite[cap.~27]{date06}.
\textcite[cap.~21--25]{date06} considera que els nous models de SGBD,
a vegades anomenats post-relacionals, no estan fundats tan sòlidament
en teories matemàtiques i la lògica de predicats com el model
relacional ni tenen el mateix nivell teòric de
formalització. %[date06ch19pp354,date06ch20pp365]
Considera, però, la possibilitat que es pugui definir un model més
potent que el relacional tot i que no veu cap indici que sigui el cas
dels nous model proposats. Per tant, aconsella que per ara els SGBD no
s'allunyin del model relacional. %[date06ch25,ch21pp379-380]
Precisament, actualment del Third Manifesto destaca la proposta del
llenguatge Tutorial D, fidel al model relacional i que a més no és
SQL. Tot i així, actualment no hi ha completada cap implementació
totalment fidedigna al model relacional del Third Manifesto. Aquest és
un model matemàtic de gran potència i molta abstracció i per tant és
difícil obtenir-ne una implementació completa. A més les
implementacions que s'estan desenvolupant són per a usos acadèmics,
encara no hi ha cap intent per a aconseguir implementacions
productives o comercials que tinguin en compte aspectes d'eficiència.




El NewSQL és el corrent més recent i apareix com a contrapartida dels
SGBD SQL tradicionals i el NoSQL. Si bé critiquen els SGBD SQL actuals
per voler ser \emph{one size fits all}, proposen el disseny de noves
implementacions dels SGBD SQL que tinguin en compte els requisits
d'eficiència dels problemes
actuals \parencite{stonebraker07,stonebraker10}.  És un corrent crític
amb alguns productes NoSQL perquè no solucionen cap problema que no
estigui previst en els SGBD actuals i perquè defineixen models de
programació que tenen poc a veure amb la teoria de SGBD, en canvi
veuen amb bons ulls la recerca NoSQL en camps on els SGBD SQL no hi
funcionen gaire bé com la gestió de documents o dades
semiestructurades \parencite{stonebraker11:nocoug}.  Per exemple,
avaluen que les aportacions del model de programació MapReduce ja
estan suportades pels SGBD para\l.lels des de fa molt
temps \parencite{pavlo09:sigmod}.  Alguns exemples de sistemes NewSQL
són: SciDB \parencite{stonebraker09:scidb}, els autors del qual
critiquen que els classifiquen incorrectament com a NoSQL, o
H-Store \parencite{hstore}, el qual és un SGBD amb programació
para\l.lela.
%http://www.theregister.co.uk/Print/2010/09/13/michael_stonebraker_interview/


%\subsubsection{Resum}

En conclusió, per una banda el Third Manifesto considera que molts
dels productes NoSQL retornen a models pre-relacional fallits. Per
altra banda, els NoSQL reclamen que fins a l'arribada dels seus
productes no hi havia cap sistema de bases de dades capaç de resoldre
eficientment determinades aplicacions. Si bé és cert que els productes
NoSQL són implementacions amb models propers al nivell físic, gràcies
a això disposen de més facilitats per a investigar noves estructures
eficients ja que no han de tenir en compte tota la potència del model
relacional. Per una tercera banda, el NewSQL recull aquestes
estructures que milloren l'eficiència i intenta expressar-les en el
model relacional; així doncs els NoSQL motiven noves millores en els
SGBD SQL, com per exemple el NoSQL de MapReduce ha esperonat el NewSQL
de HStore. En els sistemes NoSQL, el model de grafs és el que parteix
de més bona base teòrica i per tant és un candidat a formalitzar-se
amb un nivell semblant que el model relacional, tot i així les
aplicacions actuals estan restringides per a representar dades de
tipus relació de parentiu (\emph{relationship}).  
% sobretot molt bo per al problema de
% regles d'integritat
% amb claus:
% primàries, foranes,
% etc.
Finalment, cal destacar que, tot i aquests nous productes, els SGBD
SQL tradicionals encara són el més usats perquè en moltes aplicacions
segueixen sent l'opció més eficient.  A més, el model relacional té
molt prestigi acadèmic com a teoria dels sistemes d'informació.



% Per altra banda, la definició del concepte de SGBD ha hagut de canviar. Primer anava acompanyada de requisits particulars (propietats ACID, transaccions, seguretat, optimització de les consultes) però ara s'ha hagut de generalitzar i centrar-se en el nucli dels SGBD: de fet és exactament el que descriu el model relacional. En canvi els requisits particulars es descriuen com a complements d'aquest nucli. (alguns fins i tot diuen que el sistema de fitxers es pot veure com un SGBD).



%%GRAFS
%Fabian Pascal %\url{http://www.dbdebunk.com/2013/11/more-on-erm-still-not-data-model.html?utm_source=feedburner&utm_medium=feed&utm_campaign=Feed%3A+blogspot%2FTuHQT+%28DATABASE+DEBUNKINGS%29}:
% ``There have been attempts to structure and manipulate data based on other logics/theories e.g. graph theory or 2nd order logic, but they have proved more complex and less flexible and comprehensible than RT. In my modeling paper I provide a criterion for comparing data models on superiority, but doing it right is non-trivial and nobody has ever claimed superiority to RT on sound grounds.''

% \url{http://stackoverflow.com/questions/19570654/do-graph-databases-deprecate-relational-databases/19572483#19572483}
% Graph databases were deprecated by relational-ish technology some 20 to 30 years ago.

% The major theoretical disadvantage is that graph databases use TWO basic concepts to represent information (nodes and edges), whereas a relational database uses only one (the relation). This bleeds over into the language for data manipulation, in that a graph-based language must provide two distinct sets of operators : one for operating on nodes, and one for operating on edges. The relational model can suffice with only one.

% More operators means more operators to implement for the DBMS builder, more opportunity for bugs, and for the user it means more distinct language constructs to learn. For example, adding information to a database is just INSERT in relational, in graph-based it can be either STORE (nodes) or CONNECT (edges). Removing information is just DELETE (relational), as opposed to either ERASE (nodes) or DISCONNECT (edges).

% share|improve this answer
% answered Oct 24 '13 at 17:21

% Erwin Smout


% \todo{}
% Diferència entre SGBD i llenguatges de programació: independència de les dades quant al model lògic i al nivell físic: \emph{``It appears that
% current NoSQL systems make no distinction be-
% tween the logical and physical schema. Thus, the
% fundamental advantages of the ANSI SPARC ar-
% chitecture have been voided, which complicates the
% maintenance of these databases. Storing objects as
% they are programmed essentially negates the data
% independence requirement that then remains to be
% adequately addressed for NoSQL database systems.''}  \textcite{atzeni13:relational_model_dead}.




% \subsection{Conclusió}

% \todo{}

% Actualment l'àmbit informàtic de SGBD se centra en les
% implementacions, com ho demostra el nou corrent NoSQL concentrat en
% trobar models d'implementacions que tinguin bon rendiment. A tal
% efecte la recerca es concentra en temes de garantia de propietats ACID
% (\emph{atomicity, consistency, isolation, durability}), d'optimització
% de consultes, d'emmagatzematge de grans volums de dades, de consultes
% via web, de distribució de bases de dades, de reduir la despesa en
% energia, etc. \parencite{stonebraker07,stonebraker10}, la qual cosa és
% exce\l.lent per a disposar d'un SGBD adequat a cada aplicació.
% \textcite{haerder05:_dbms_archit} descriu diferents models
% d'implementació per als SGBD, ja que indica que per obtenir bon
% rendiment la implementació d'un SGBD s'ha d'estudiar per cada
% aplicació. És més, una implementació d'un SGBD que vulgui obtenir un
% bon rendiment en una determinada aplicació potser no pot implementar
% el model de dades complet sinó que només una part, com per exemple en
% els sistemes
% encastats \parencite{saake09:_downs_data_manag_embed_system}.

% Per altra banda, l'àmbit matemàtic de SGBD, amb el model relacional
% com a màxim exponent, se centra en els conceptes teòrics, és a dir
% respon a la pregunta de què són els SGBD. Recerca en millorar-ne la
% comprensió, en obtenir la màxima potència i facilitat de cara a la
% gestió de dades per part de l'usuari o en obtenir nous models. Tot i
% així, actualment encara no s'ha trobat cap altre model que tingui la
% mateixa potència que el relacional. Cal destacar que tot i que el
% model relacional té conceptes madurs i consolidats, i que a més han
% tingut èxit amb els SGBD SQL, s'obre una nova perspectiva amb
% l'evolució de conceptes que proposa el \emph{Third Manifesto},
% especialment amb \emph{Tutorial D} i les implementacions que comencen a prendre
% cos a nivell acadèmic.



















%%% Local Variables: 
%%% mode: latex
%%% TeX-master: "main"
%%% End: 

% LocalWords:  monitoratge SGBD SGBDR SQL

\section{Sistemes de gestió de bases de dades}
\sectionmark{SGBD}
\label{sec:art:sgbd}


Segons \textcite{date:introduction}, ``una base de dades és un
contenidor informàtic per a una co\l.lecció de dades''. El sistemes
informàtics que tracten amb bases de dades s'anomenen sistemes de
gestió de bases de dades (SGBD, \emph{Data Base Management Systems}) i
tenen l'objectiu d'emmagatzemar informació i permetre consultar i
afegir aquesta informació per part dels usuaris.  Per complir aquests
objectius, els SGBD ofereixen a l'usuari diferents operacions a fer
amb la base de dades, com per exemple crear-la, afegir dades o
consultar informació a partir de les dades emmagatzemades.

Els SGBD es basen en teories matemàtiques que reben el nom de model de
dades, un SGBD és una implementació d'un model de dades.  Segons
\citeauthor{date:introduction}, ``un model de dades és una definició
abstracta, auto continguda i lògica dels objectes, de les operacions i
de la resta que conjuntament constitueixen la màquina abstracta amb la
que els usuaris interaccionen. Els objectes permeten modelar
l'estructura de les dades. Les operacions permeten modelar el
comportament''. Ara bé, \citeauthor{date:introduction} avisa que el
concepte model de dades també s'usa per a definir una estructura
persistent de dades concreta i per tant cal distingir adequadament la
confusió entre els dos conceptes.
% Tal com fa Date, parlarem de model de dades, o simplement de model, en el primer sentit de màquina abstracta.

Un model de SGBD que ha s'ha consolidat i esdevingut un referent és el
model relacional (\emph{relational}). Principalment l'èxit d'aquest
model es basa en teories matemàtiques consolidades: la lògica de
predicats i la teoria de conjunts \parencite{date:introduction}.



\subsection{Model relacional}

El model relacional va ser proposat per \textcite{codd70} per a
formalitzar els SGBD, els quals quan es basen en aquestes teories
s'anomenen relacionals (SGBDR). A partir de llavors els SGBDR han anat
evolucionat fins a tenir una gran solidesa, amb
\textcite{date:introduction,date06,date:dictionary} com a principal
divulgador.


\textcite[cap.~2]{date06} %ch2pp21-22
considera que no hi cap implementació comercial que segueixi fidelment
el model relacional, tot i que esmenta algunes implementacions
prometedores com \emph{Dataphor} o la seva proposta tecnològica
\emph{TransRelational} \parencite{date:transrelational}. A banda,
també cal destacar \emph{Rel} \parencite{rel} com un SGBDR bastant
consolidat.

Les implementacions més populars de SGBDR són les que s'anomenen
\emph{SQL} ja que tenen en comú el llenguatge \emph{Structured Query
  Language}. Ara bé, els SGBD \emph{SQL} es desvien considerablement
del model relacional: permeten files duplicades, tenen ordre en les
columnes, permeten valors nuls, etc, sent aquest últim un tema
modernament discutit \parencite{date08:nulls}.

Les diferències entre els SGBD \emph{SQL} i el model relacional han
contribuït a que per una banda hi hagi hagut una sèrie de malentesos i
errors, alguns dels quals han estat avaluats i desmentits per
\citeauthor{dbdebunk} en vàries
publicacions \parencite{dbdebunk,date06}, i que per altra banda han
contribuït en gran mesura a voler explorar altres models de
SGBD \parencite{stonebraker09}.

Tot i així, \textcite[cap.~21--25]{date06} considera que els nous
models de SGBD, a vegades anomenats post-relacionals, no estan fundats
tant sòlidament en teories matemàtiques i la lògica de predicats com
el model relacional i pronostica que ens els propers cent anys els
SGBD encara estaran basats en el model
relacional. %[date06ch19pp354,date06ch20pp365]
Considera la possibilitat, tot i que remota, que es pugui definir un
model més potent que el relacional però que no hi ha cap indici que
cap definició dels nous model tingui la mateixa potència que el
relacional. Per tant, aconsella que per ara els SGBD no s'allunyin del
model relacional. %[date06ch25,ch21pp379-380]

  
Actualment \textcite{date:thethirdmanifesto} estan treballant en el
'\emph{Third Manifesto}' com a proposta per a obtenir SGBDR purament
relacionals. Destaquen que, en el model relacional, els tipus i les
relacions són necessaris i suficients per representar absolutament
qualsevol tipus de dades a nivell lògic. %[date06ch21,370]
Amb aquesta definició s'entén que els SGBDR també es puguin anomenar
SGBD objecte/relacionals (\emph{object/relational}).  Com a proposta
per a obtenir SGBDR purament relacionals
\textcite{date05,date:tutoriald} estan definint les regles d'un nou
llenguatge, \emph{Tutorial D}, que ha de servir pels estudis del model
relacional a nivell acadèmic.


\textcite[cap.~6]{date06} manifesta i avisa de l'ús de la mateixa
terminologia amb significat diferent entre el model relacional i
l'orientació a objectes, sobretot pel que fa als termes valor i
variable. %[date06ch6pp91]
La seva hipòtesi és que el model relacional és un model de dades i el
model d'orientació a objectes és més proper a un model
d'emmagatzematge. % 'the object model' is closer to being a model of
                  % storage than it is to being a model of
                  % data. [date06ch6pp92]
A la \autoref{tab:sgbd:relacional-objectes} es resumeix la possible equivalència lògica dels
conceptes entre el model relacional i l'orientació a objectes tal com
Date exposa al capítol 6, tot i que cal tenir en compte que
la semblança és difusa.


\begin{table}
\centering
\begin{tabular}[ht]{ll}
  relacional & objectes \\\hline \hline
  tipus & tipus, classe\footnote{tipus amb atributs i mètodes}, interfície \\\hline
  representació & classe, atributs, propietats \\\hline
  valor, objecte, instància & valor, estat, objecte/instància immutable/estàtic \\\hline
  variable & valor, objecte/instància mutable/dinàmic \\\hline
  referència\footnote{adreça continguda en una variable} & variable \\\hline
  operador & funció, mètode \\\hline
\end{tabular}
\caption{Possible equivalència lògica de termes entre el model relacional i l'orientació a objectes \parencite[cap.~6]{date06}.}
\label{tab:sgbd:relacional-objectes}
\end{table}
% Relacional: tipus | representació |  valor, objecte, instància  | variable  | referència (adreça continguda en una variable) | operadors (de lectura i de modificació)
% Objectes: tipus, classe (tipus amb atributs i mètodes), interfície | classe, atributs,propietats  |  valor, estat, objecte/instància immutable/estàtic |  valor, objecte/instància mutable/dinàmic  | variable | funcions,mètodes (funcions dins de classes) (purs o modificadors)




\subsection{Implementació i model}

\textcite{date:introduction} diferencia amb detall els conceptes d'implementació i model.
El model d'un SGBD és el model matemàtic tal com s'ha descrit anteriorment, en canvi un SGBD és la implementació d'un model de dades, per exemple \emph{Rel}.
%Segons l'esquema de comunicació, també es poden anomenar com a sistema servidor de bases de dades, per exemple postgresql.
Una base de dades és una instància d'un SGBD, per exemple la base de dades dels estudiants.



Aquesta diferència entre implementació i model aporta independència de dades (\emph{data independence}) \parencite{date:dictionary}. En altres paraules, els models no han de tenir detalls d'implementació ni parlar d'objectius de rendiment. 
\textcite{dbdebunk} detallen algunes confusions actuals sobre la independència entre el model i la implementació.


Com s'ha dit anteriorment, Date descriu els SGBD \emph{SQL} com a implementacions no fidels al model relacional, encara que aquests productes comercialment es coneguin com a SGBDR. Com a proposta per a obtenir SGBDR purament relacionals \textcite{date05,date:tutoriald} han definit el llenguatge \emph{Tutorial D} i ja està donant resultats amb SGBDR com \emph{Rel} .


Recentment ha aparegut un nou corrent en l'àmbit dels SGBD que s'anomena NoSQL (\emph{Not Only SQL}) amb l'objectiu de sobrepassar les limitacions dels SGBDR \parencite{edlich:nosql,stonebraker10}. 
A l'espera que Date valori aquest corrent, cal tenir present els seus apunts sobre sistemes relacionals contra sistemes no relacionals \parencite[part 7]{date06}, sobretot pel que fa al concepte de l'\emph{Information Principle} i que SQL no és un bon referent pels SGBDR. És a dir, s'ha d'entendre NoSQL com una crítica a les implementacions comercials actuals del model relacional, una crítica que pot estar motivada per l'ús de SQL per part d'aquest productes. 
No es pot entendre, però, el NoSQL com una crítica al model relacional ja que els objectius parlen de millorar el rendiment dels SGBD, cosa només atribuïble a les implementacions però no al model. Precisament, actualment del model relacional destaca la proposta de \citeauthor{date:tutoriald} d'un llenguatge, \emph{Tutorial D}, que no és SQL.


El corrent de NoSQL també critica l'adequació dels productes actuals.
Els SGBD NoSQL apunten els SGBDR actuals per voler ser \emph{one size fits all} \parencite{stonebraker07,stonebraker09} però que cada aplicació té els seus requisits i per tant una mateixa implementació no pot ser bona per a tots el camps.
En aquesta mateixa línia els models pels SGBD prenen més sentit que mai ja que permeten mantenir una definició comuna per a moltes implementacions.


Segons es desprèn de \textcite{date06} fins a l'actualitat només hi ha hagut un model consolidat pels SGBD: el model relacional. 
Ara, en el corrent NoSQL també es parla de nous models de SGBD \parencite{edlich:nosql,stonebraker09:scidb}.
\citeauthor{date06} ha avaluat que alguns nous models recuperen intents fallits en el passat tot i que es poden representar amb el model relacional, per exemple els SGBD XML basats en estructures d'arbre \parencite[cap.~14]{date06} o els ODMG basats en objectes \parencite[cap.~27]{date06}. Tot i així, en un futur cal estar atents per si alguns d'aquests models joves de SGBD arriben a consolidar-se o no poden esdevenir tant potents com el model relacional.


En resum, l'àmbit informàtic de SGBD se centra en les implementacions. Recerca en temes de garantia de propietats ACID (\emph{atomicity, consistency, isolation, durability}), d'optimització de consultes, d'emmagatzematge de grans volums de dades, de consultes via web, de distribució de bases de dades, de reduir la despesa en energia, etc. \parencite{stonebraker07,stonebraker10}, la qual cosa és excel·lent per a disposar un SGBD adequat a cada aplicació. 

Per altra banda, l'àmbit matemàtic de SGBD, amb el model relacional com a màxim exponent, se centra en els conceptes dels SGBD. Recerca en millorar-ne la comprensió, en obtenir la màxima potència de cara a la gestió de dades per part de l'usuari o en obtenir nous models, encara que  no s'ha trobat cap altre model que tingui la mateixa potència que el relacional. Cal destacar que tot i tenir conceptes antics consolidats i que han tingut èxit amb els SGBD SQL, s'obre una nova perspectiva amb Tutorial D i les implementacions que comencen a prendre cos  a nivell acadèmic. La idea bàsica dels SGBDR és que ho són realment quan la base de dades conté i només conté variables relacions els atributs de les quals admeten qualsevol tipus, respectivament definit per Date com a \emph{Information Principle} o \emph{The Principle of Uniform Representation} \parencite{date:dictionary} i que la teoria de tipus i el model relacional són ortogonals és a dir independents \parencite[cap.~6]{date06}.


%Ara bé, una implementació d'un SGBD que vulgui  obtenir un bon rendiment en una determinada aplicació potser no pot implementar el model relacional complet sinó que només una part




%Com s'ha d'estendre el model relacional?
\subsection{Extensió del model relacional}



                                And those investigations in turn can lead to (compatible!)
extensions to the original model; that is, the model can grow over time to become an
ever more faithful abstraction of data “as it really is.” In other words, a model of data is
really a theory of data (or at least the beginnings of one), and it’s not a static thing [date06ch25pp441]



Date considera que el model relacional va evolucionat i no considera que hi hagi hagut cap revolució des de la seva aparició [date06ch19pp254]. Tot i que hi ha noves maneres per explicar-lo, com és el cas de l'àlgebra relacional 'A' \todo{cite apxA dates and types o thirdmanifesto}. Consideren que el model relacional està bastant completat i que evoluciona en la comprensió de les teories i els conceptes que hi intervenen. 

Només hi ha hagut un àmbit en el que han admès que s'havia d'estendre el model relacional i aquest és el de les dades temporals \todo{citar date temporal}.


  'Support for interval-valued attributes (and hence for temporal databases) involves among other things support for
%generalized versions of the usual relational operators.' [date]            Note: Those “U_” operators are all defined in terms of two new relational operators called pack and unpack,
and those latter operators in turn are defined in terms of relation-valued attributes. As already noted, therefore, support
for interval-valued attributes relies on support for relation-valued attributes, at least conceptually. 

Third, I’d like to see good support for temporal data. I think support for the time dimen-
sion is important already and due to become much more so. Now, several researchers
have proposed approaches to this problem—approaches that are, however, fundamentally
flawed for the most part (in my opinion), because they violate fundamental relational
principles. By contrast, it’s my belief that the relational model already includes what’s
needed to support the time dimension properly. All that’s needed is a set of well chosen
and carefully designed shorthands. Along with two coauthors, Hugh Darwen and Nikos
Lorentzos, I’ve written about this topic at some length in another book, Temporal Data
and the Relational Model (Morgan Kaufmann, 2003). [date06ch2pp22]



%When people say "object-relational", this is  fundamentally what they're talking about, user-defined types. 
%Of course, I and CJ Date would argue that features of "object-relational" are really just features of a fully-implemented "relational", meaning user-defined types, and so for that reason saying "object" is just a noise-word, helpful only for marketing and nothing else. 


Així doncs no hi ha necessitat d'estendre el model relacional. Sí que hi ha necessitat, però, de tenir disponibles diversos tipus de dades. La creació de nous tipus de dades ha estat un tema de molta confusió ja que les 'suposades' implementacions del model relacional no ho han permès, o les extensions proposades per \textcite{stonebraker86} es consideren que són a nivell d'implementació i no les que es desitjarien a nivell de model (silbershatz96 en parla a l'apartat 6.research a on diu que els ADT han de ser més flexible i conformar tipus de primer nivell. però veure seshadriVLDBJournal1998:enhanced abstract data types in object-relational databases), i sobretot ha fet que les implementacions basades en orientació a objectes tinguessin un gran impacte \todo{citar ODMG (object databases)}. Date ha rebutjat aquesta idea argumentant que el model relacional mai no ha fixat els tipus de dades disponibles sinó que ben al contrari mai n'ha parlat i per tant els tipus de dades han tingut total llibertat de creació; si bé el model relacional defineix que com a mínim hi ha d'haver el tipus booleà i el tipus relació.

El model relacional requereix que hi hagi un 'sistema' de tipus però diu molt poc de la naturalesa d'aquest sistema de tipus; això es coneix com a ortogonalitat entre les relacions i els tipus.



Així doncs, tenint en compte que segons Date el model relacional és complet, que no hi ha cap de tant potent i que l'ampliació de funcionalitat dels SGBDR s'ha de fer mitjançant la creació de nous tipus, el SGST hauria de contemplar aquestes idees. 

Primer s'hauria de veure que el cas de les sèries temporals no sigui com el cas de les dades temporals a on sí que s'ha necessitat estendre el model relacional.

Segon, s'hauria de considerar que el SGST són un nou tipus de dades en el model relacional i per tant el model relacional ja té tota la potència per una banda constituir SGBD i per altra banda definir i incorporar nous tipus.


Com es defineixen nous tipus complexos als SGBD?

En el cas que es descarti que el cas dels SGST presenta els mateixos problemes que les dades temporals i per tant els SGST han d'esdevenir un tipus de dades, cal preguntar-se com són els tipus de dades al model relacional.

Concretament el model relacional només defineix què es un tipus de dades però dóna llibertat a la seva creació. Això és un gran avantatge. En els cas de tipus de dades senzills es defineixen amb una bona estructura i ja està però què passa quan es vol definir un nou tipus de dades complex?

Cal recercar com s'han definit nous tipus de dades complexos. Els principals problemes es donen que el tipus és complex i forma una entitat de per sí. És a dir que definir un tipus sèrie temporal no és trivial. Aleshores, com cal procedir?



Sobre la necessitat de modelar els tipus.

Cal definir un model pel tipus que volem dissenyar.
De fet, volem dissenyar un SGST. És a dir, un tipus que conformi pròpiament un SGBD. Per tant, utilitzarem les mateixes eines que es fan servir per modelar els SGBDR per a poder modelar el nostre SGST. Com que s'hauran utilitzat les mateixes eines, el SGST podrà esdevenir perfectament un tipus de dades pels SGBDR.

En resum, per a definir nous tipus complexes cal modelar-los com a entitat pròpia, fent un símil amb el model relacional. Això no vol dir, però, que un cop modelats constitueixin de per sí un nou model per als SGBD, sinó que queden dins dels SGBDR. \todo{caldria} elaborar més i trobar alguna pista de com definir nous tipus complexos, stonebraker86 només indica com es va estendre postgresql amb operadors de creació de nous tipus però no com s'han de modelar els nous tipus.

Volem crear un nou model de SGBD, Date ens diu que només hi ha el model relacional, per tant hem d'utilitzar el model relacional per a definir el nostre nou model.


A la xerrada [Date02] va dir que els object/relational són relacionals.


Sobre sèries temporals i temporal data\cite{assfalg08:thesis}

vocabulari proposat per darwen [transparències]
Stated times = "valid times"
Logged times = "transaction times"


The Book’s Aims:
Describe a foundation for inclusion of support for temporal data in a truly
relational database management system (TRDBMS)
Focussing on problems related to data representing beliefs that hold throughout
given intervals (usually, of time).
Propose additional operators on relations and relation variables ("relvars")
having interval-valued attributes.
Propose additional constraints on relation variables having interval-valued
attributes.
transparencies darwen]


Els SGBD relacionals són capaços d'implementar el primer tipus de coherència, les \emph{bitemporal data}; llavors es classifiquen sota el nom de bases de dades temporals, \cite{date:introduction,wiki:temporal_database}. Però el model relacional no és suficient pel segon tipus: les sèries temporals. Tot i que en principi no hi hauria cap problema a utilitzar una base de dades relacional per a sèries temporals, enteses com a dades històriques, la pròpia naturalesa dels sistemes relacionals  dificulta les operacions necessàries. 


Relational DBMS can implement \emph{bitemporal data}. Then they are known as temporal databases \parencite[ch.\ 22]{date:introduction}. However, relational DBMS are not adequate for time series. The relational model is capable to describe time series when they are thought as historical data but the design of relational DBMS would difficult the operations  needed by time series \parencite{schmidt95}. Theses time series operations are mainly based in time ranges and need time zones conversions, rotations of table registers and file size maintained at bounded levels.


A TSMS could be implemented in a relational system considering the improvement proposed by \textcite{stonebraker86} that allows the inclusion of new types and operations to relational DBMS. However, it is difficult to evaluate this solution as there is no consolidated data model  for time series. 

After comparing temporal DBMS and TSMS \parencite{schmidt95} they ask whether temporal DBMS and TSMS will meet or they need to be manage by different systems despite both working with time aspects.


he fundamental difficulty is that the relational model does not take advantage of the order of rows. Whereas one can perform an ``order by'' query and manipulate the data in some other language, one cannot natively manipulate the ordered data using select, from, and where. 
Arguably, this is good for data independence, but it is bad for time series.
[http://cs.nyu.edu/shasha/papers/jagtalk.html]

        However, there are major differences between sen-
sor networks and traditional distributed database systems.
Most relevant to sensor networks is existing work on dis-
tributed aggregation [38, 42], but these approaches do not
consider the physical limitations of sensor networks. Mad-
den et al. [30] give an extended classification of aggregate
operators with properties relevant to sensor network aggre-
gation. %\cite{demers03}


Seq model [seshadri95] vs relacional: veure capítol 14 Tree-Structured Data [date06] a on Date descriu com modelar estructures d'arbre amb relacions. Potser es podrien aplicar conceptes semblants per les seqüències? En tot cas: el model estructural ha de contemplar explícitament l'ordre o ho han de fer les operacions?

[date06]:
                                                Now, the relational model provides just one way to
represent data—namely, by means of relations themselves—and the sole essential information
carriers in a relational database are thus necessarily relations, a fortiori. By contrast, other data
models typically provide many distinct ways to represent data (lists, bags, links, sets, arrays,
and so on), and all or any of those ways can be used as essential information carriers in a
nonrelational database. One way of representing data is both necessary and sufficient; more
than one introduces complexity, but no additional power.
     As a matter of fact, it’s the concept of essentiality that forms the underpinning for the well-
known, and important, Information Principle:
     The entire information content of a relational database is represented in one and only
     one way: namely, as attribute values within tuples within relations.

[date06,pàg116]:
Note that what they don’t do is introduce any more power. To be more specific, there’s no data that can
be represented with arrays, lists, and so on, that can’t be represented without them. Likewise, there’s no
useful processing that can be done on data represented with arrays, lists, and so on, that can’t be done
on data represented without them.
[pàg 134]
about the continual references in the body of this chapter to the effect that we might have
array-valued attributes, list-valued attributes, and so on. Don’t such possibilities constitute an
obvious violation of The Information Principle?
     Well, no, they don’t. What that principle says is that the only kind of variable permitted in
the database (considered in turn as a variable) is the relation variable. But a relation variable
can have attributes of any type whatsoever ... including array types, list types, and so on.
Thus, the database at any given time might include a relation value that in turn includes array
values, list values, and so on ... But there are still no array or list variables (etc.) anywhere in sight.









% \section{Preguntes}

% El llenguatge dels SGBD, per exemple D, és declaratiu (what) i no procedural/imperatiu (how). Com lliguen els SGBD amb els llenguatges de programació declaratius, per exemple amb la programació lògica amb Prolog? Concretament, com lliga Prolog amb el concepte de SGBD? Té la mateixa potència un SGBD que Prolog?
%  Es pot veure Prolog com un SGBD?

% http://stackoverflow.com/questions/2117651/difference-between-sql-and-prolog

% diferències entre model semàntic i programació orientada a objectes [hull86]?







%%% Local Variables: 
%%% mode: latex
%%% TeX-master: "main"
%%% End: 

% LocalWords:  monitoratge SGBD SGBDR




\section{Dades temporals}
\label{sec:art:dades_temporals}

\todo{Convertir en una secció les dades temporals i parlar-ne més extensament}
parlar de docs anteriors de dades temporals

Refs a estudiar:

\cite{jensen99:temporaldata}
\cite{jensen00:thesis}
\cite{jensen98:temporal_database_glossary}
\cite{tansel93:temporal_databases}


Em alguns llocs les bases de dades temporals també s'anomenem multiversió (multi-version)


%Precisament,
Una de les extensions importants del model relacional s'ha produït amb
l'estudi de les dades
temporals \parencite{date02:_tempor_data_relat_model}. Amb el model de
dades temporals basat en el model relacional s'obtenen SGBDR capaços
d'emmagatzemar i consultar dades històriques.

El model de les dades
temporals \parencite{date02:_tempor_data_relat_model} es basa en
relacions i intervals per representar les dades històriques. Cada
relació s'estén amb un atribut que és un interval temporal que indica
el rang temporal de validesa de les proposicions. A més el model també
defineix les operacions necessàries per tractar amb les dades
temporals. Aquestes operacions són extensions de l'àlgebra relacional.
El principal objectiu del model de dades temporals és poder tenir
suport per a les dades històriques en els SGBDR.
\textcite[cap.~28]{date06} compara aquest model per dades temporals
amb altres aproximacions que s'han fet no basades en el model
relacional.



El model de dades temporals representa les dades històriques amb el
temps vàlid i temps de
transacció \parencite[cap.~15]{date02:_tempor_data_relat_model}, el
que es coneix com a dades bitemporals.  Així doncs, els SGBD per dades
bitemporals no es consideren adequats com a SGBD per sèries temporals
ja que els primers estan pensats per a històrics, es descriuen amb
intervals temporals, i els segons per anàlisi d'observacions
seqüencials, es descriuen amb instants temporals.
\textcite{schmidt95} arriben a aquesta conclusió després de comparar
els SGBD bitemporals amb els de sèries temporals, tot i que per a
desenvolupaments futurs observen que hi aspectes temporals comuns
entre els dos sistemes i es pregunten si es podran trobar sistemes que
els englobin a tots dos o cadascun necessitarà sistemes específics.


%a algun lloc deia Date que el seu model per dades temporals no era del tot adequat per les sèries temporals?, potser perquè els intervals no estaven pensat per a instants? date02ch16.8?




\todo{llibre que apareixerà aviat}

Time and Relational Theory, 2nd Edition
Temporal Databases in the Relational Model and SQL
 
Author(s) : Date,     Lorentzos,    Darwen.    Expected Release Date: 26 Sep 2014Imprint:Morgan KaufmannPrint Book ISBN :9780128006313 Pages: 560Dimensions: 235 X 191
\url{http://store.elsevier.com/product.jsp?isbn=9780128006313&pagename=search}





%%% Local Variables: 
%%% mode: latex
%%% TeX-master: "main"
%%% End: 

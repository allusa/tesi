\documentclass[
english
]{scrartcl}
\usepackage[utf8]{inputenc}


\usepackage{url}

% \usepackage[cmex10]{amsmath}
% \usepackage{amssymb}



\title{
  A Model for a Multiresolution Time Series\\
  Database Management System }


%[A. Llusà Serra; T. Escobet Canal, S. Vila-Marta]
\author
{
  {
    A.\ Llusà Serra,
    T.\ Escobet Canal
    and S.\ Vila-Marta
  }\\
  {\{aleix,sebas\}@dipse.upc.edu, teresa.escobet@upc.edu}\\
  {Department of Electronic System Design and Programming (DiPSE)}\\
  {Universitat Politècnica de Catalunya, 08242 Manresa, Spain}
}



\begin{document}


\maketitle


\begin{abstract}
The abstract goes here.
Manuscripts should be written in English and include a 100-300 word abstract.
\end{abstract}

{\bfseries Keywords:} 



\section{Introduction}

This paper focuses on Data Base Management Systems (DBMS) that store
and treat data as time series. Traditional DBMS, as is ones derived
from relational model, are not adequate for these cases as they do not
have enough facilities to manage and retrieve time series
information.



* Temporal databases. Basades en esdeveniments. Data mining basat en sèries temporals definides per parelles temps-valor; calen TSMS

* Alta dimensió sèries temporals, cal reduir-la. Es conserven els segments de temps més interessants; multiresolució

* Multiresolució, diferents resolucions, es pot treballar amb més o menys dades segons convingui

*Cal saber canviar de resolució, exemple transformar dades periòdiques d'un mes a un any.

* Aggregates, una sèrie temporal pot estar mostrant diferent informació. ex: mitjana, màxim, valor al final del període, ...

* Les sèries temmporals tenen una metainformació que cal guardar en una base de dades relacional (localització, etiquetes de classificació, últim valor mesurat, unitats, etc.)

* Disseny del model de TSMS, aleshores veurem si una TSMS pot ser implementada com a camp d'una altra DBMS o si els DBMS no són capaços de manipular TS adequadament i cal implementar TSMS específics.

* Calendari, passa a segon terme. El temps es defineix com universal i constant (semblant a Unix Time Epoch). Aquests temps es pot convertir a calendari. Cal definir la interacció usuari/calendari amb temps universal.

* El temps és un nom donat al camp, qualsevol objecte que tingui la mateixa interfície que el temps pot funcionar. En el cas del valor pot ser qualsevol objecte, s'exemplifica amb reals per facilitar-ne la comprensió i per ser el més proper al time series analysis: statistical methods focused on sequences of values representing a single numeric variable [llibre-last].


* Representació: Entre dos punts de mesura, quin valor pren la sèrie temporal?.





 P.ex. a la Tesis Soriguera, APPENDIX A2 , 
Requiem for Freeway Travel Time Estimation Methods Based
on Blind Speed Interpolations between Point Measurements, parla d'interpolació entre mesures (compte, EN l'ESPAI NO EN EL TEMPS!! però seria un exemple de que els links compleixen amb la interfície temps?) amb diferents mètodes:  interpolacions constants (endavant, endarrere, optimistica), a trossos, lineal, quadràtica;    tot i que conclou que The present paper shows that all speed interpolation methods that omit traffic
dynamics and queue evolution do not contribute to better travel time estimations. 
%http://www.fundacioabertis.org/pdf/Editarium_FSoriguera.pdf
Unlike in fusion 1, in the fusion 2 process the information is not provided by the same
data source, and hence, the data will not be equally located in space and time. 
Therefore, a spatial and temporal alignment is needed before the data can be fused.


* Xarxa de sensors, tsms distribuïda. Sensor dades recents, màquina grossa històrics. Quan es llança una consulta, es llança distribuïdament: si es té prou resolució es respon sinó s'envia la consulta al sensor. (consultar si en deligiannakis en diu alguna cosa).

%Multi-Sensor Data Fusion: An Introduction. Harvey B. Mitchell





%Paper structure


\section{The Model}


%\section{Results and Discussion}






\section{Related Work}


Per una banda, hi ha alguns que exploren SGBD similars



RRDtool from Oetiker \cite{rrdtool} is a professional database
management system extremely used by the free software community. It is
used in professional monitoring systems and its efficiency and speed
when processing time series is being improved. Nevertheless, it is
focused to a particular kind of data, gauges and counters, and it has
not general time series operations.

Cougar 


% item[Cougar]
% \textcite{cougar,fung02} proposen Cougar com un SGBD per xarxes de sensors (\emph{sensor database systems}). El sistema té dues estructures \parencite{bonnet01}: una basada en relacions per les característiques dels sensors i una basada en seqüències per les dades dels sensors, les quals són sèries temporals.
% Les consultes es processen de manera distribuïda: cada sensor és un node amb capacitat de processament que pot resoldre una part de la consulta i fusionar-la amb les altres. D'aquesta manera es minimitza l'ús de comunicacions però l'estructura i estratègia de comunicació dels nodes esdevé una part crítica a configurar \parencite{demers03}.

% item[SciDB]
% \textcite{stonebraker09:scidb} estudien els SGBD científiques amb models  de dades basats en matrius. Estan desenvolupant SciDB \parencite{scidb}, un SGBD productiu i optimitzat per treballar amb matrius.


% item[SciQL]
% \textcite{kersten11} descriuen SciQL, un llenguatge per a SGBD científiques basades en matrius. Hi ha un prototip en desenvolupament de SciQL \parencite{sciql}.



Per altra banda, hi ha les dades bitemporals que exploren històrics mitjançant intervals temporals. Hi ha hagut tot un recorregut d'investigació de com encaixar  els intervals temporals en els SGBD, finalment DDL2002 marca una fita.



\section{Conclusions} 
The conclusion goes here.





\section*{Acknowledgements}

This work was supported by Universitat Polit\`{e}cnica de Catalunya (UPC).



\begin{thebibliography}{9}


\bibitem{llusa11:tfm} A.\ Llusà Serra, ``Estudi i modelització
  dels SGBD Round Robin pel tractament de sèries temporals''
  (catalan), Master Thesis, Universitat Politècnica de Catalunya,
  2011. 

\bibitem{llusa12:ptd} A.\ Llusà Serra, ``Disseny i modelització
  d'un sistema de gestió multiresolució de sèries temporals''
  (catalan), Thesis Proposal, Universitat Politècnica de Catalunya,
  2012.


\bibitem{rrdtool} T.\ Oetiker, ``RRDtool, Round Robin database'',
  1998--2011.  \url{http://oss.oetiker.ch/rrdtool/}


\bibitem{fung02} W.\ Fu Fung, D.\ Sun and J.\ Gehrke, ``COUGAR: the network is the database'', Proceedings of the 2002 ACM SIGMOD international conference on Management of Data, Madison, 4--6 June 2002, pp. 621--621. doi: 10.1145/564691.564775 



\end{thebibliography}




\end{document}




%%% Local Variables: 
%%% ispell-local-dictionary: "british"
%%% End: 
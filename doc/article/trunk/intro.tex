\section{Introduction}

Modern society depends on the smooth operation of many complex
engineering systems which provides products and services. Examples
include electric power systems, water distributions networks,
transportation systems, manufacturing processes, intelligent
buildings, communication systems, etc. The emergence of networked
embedded systems and sensor/actuator networks has made possible the
collection of large amounts of data for monitoring and control of such
complex systems. In most application, these data need to be processed
and synthesised efficiently to provide relevant information to
engineers, researchers, accident investigators, operators, and many
other users.  

Nevertheless, before using the collected signals, it is of primary
importance to promptly detecting eventual sensor failures or
malfunctions and possibly reconstructing the incorrect signals in
order to avoid processing misleading information which may lead to
unsafe and/or inefficient actions.  In all these instances a measure
is associated with a time stamp, which implies that the correctness of
those data depends not only on the measured value but also on the time
as it is collected. When observations are collected at specific time
interval, large data sets in the form of time series are
generated~\cite{basu07:_autom}.  

Time series are defined as a collection of observations made
chronologically \cite{fu11}, in \cite{last:hetland} they are also
called time sequences.  Time series are usually stored in a
database. Usually the managing software to store this data are
relational database management systems (RDBMS). However, using a RDBMS
as a time series backend suffers some drawbacks
\cite{schmidt95,stonebraker09:scidb,zhang11}. Time series come from a
continuous nature in which they are recorded at regular intervals,
such as hourly or daily, or at irregular intervals, such as recording
when a pump is open or closed. 

One problem when dealing with time series data results from the fact
that these data are often voluminous \cite{fu11}. As a result, storing
and accessing them can be complicated. Moreover, it is specially
critical when developing small embedded systems, whose resources
(capacity, energy, processing, and communications) suffer a genuine
restriction \cite{yaogehrke02}.  Another problem is that the procedure
of processing and synthesising information becomes complicated if data
is not equi-time spaced.



%TSMS

This paper focuses on Data Base Management Systems (DBMS) that store
and treat data as time series. These are usually known as Time Series
Data Base Management Systems (TSMS).  A TSMS is a special purpose DBMS
aimed at storing and managing time series

The main objective of TSMS is to put together two areas of study: time
series analysis and DBMS.  Time series analysis theory studies
formally a great amount of algorithms and methodologies that apply to
time series, focusing on efficiency improving. DBMS theory studies
systems that store and operate with data; currently the relation model
\cite{date:introduction} is the referent.

%TSMS features

In time series analysis there are some common operations that can be
generalised when treating time series.

The main attribute of time series is the time, therefore dealing with
time is a common operation, such as querying time intervals, finding
time correlations, or calculating distances between two time
series. TSMS must respect the temporal coherence of the time series.
In the context of statistics, aggregation of time series is also
common operation. Aggregation consists in summarising a time series
subset with an statistic such as the mean, the maximum, or the mode.

A particular feature of time series is representation. A time series
is discrete in the set sense, that is a set of value and time
pairs. Representation is the function model approximating the time
series to its continuous nature. TSMS operate on time series
respecting the representation coherence. Furthermore, the values of a
time series can be of any type; for simplicity examples are presented
with integers or real numbers but can also be strings or structures
such as arrays.
% , preferably using piecewise operations in the set domain rather
% than solving numerical methods for the continuous domain.


%Related work
\todo{avantatges i incovenients de cada un}

There are some prior works concerning TSMS. RRDtool from Oetiker
\cite{rrdtool} is a free software database management system. It is
designed to be used in monitoring systems. Because of this, it is
focused to a particular kind of data, gauges and counters, and it
lacks general time series operations. RRDtool can store multiple time
resolution data. The work in this paper is partially inspired in
RRDtool.

Cougar \cite{bonnet01} is a sensor database system. It has two
structures: one for sensor properties stored into relations and
another for time series stored into data sequences from sensors.  Time
series have specific operations and can combine relations and
sequences. %Molt específic per a xarxes de sensors a on aplica data streams: avantages per a càlcul eficient però inconvenients que no són TSMS generalitzables a totes les sèries temporals

SciDB \cite{stonebraker09:scidb} and SciQL \cite{zhang11} are array
database systems. These systems are intended for science applications,
in which time series play a principal role. They structure time series
into arrays in order to achieve multidimensional analysis. %sciql barreja relacions i arrays d'una manera estranya que embolica més que tneir-ne uns o els altres

Bitemporal data and time series data are not exactly the same and so
can not be treated interchangeably \cite{schmidt95}. However, there
are some similarities between time series and bitemporal data that can
be considered. First, extending a relational database model to manage
bitemporal data shows the way to extend relational DBMS with new types
and how to model them. Second, bitemporal data modelling settles some
time-related concepts that can be extended to time series.

The recent bitemporal data research in relational DBMS model terms
\cite{date02:_tempor_data_relat_model} marks a promising
foundation. It models bitemporal data as relations extended with time
intervals attributes and extends relational operations in order to
deal with related time aspects.

%MTSMS

In this paper, we introduce a new data model for a TSMS: a TSMS with
multiresolution capabilities. This model allows to store time series
using different time resolutions and organised in a
hierarchical\todo{?} way. The model is specifically designed to cope
well with bounded storage computers like those found in sensor
systems.

MTSMS improve TSMS features in various aspects:

\begin{itemize}

\item Voluminous data. 

\item Regular time data.

\item Validate data.

* Alta dimensió sèries temporals, cal reduir-la. Es conserven els segments de temps més interessants; multiresolució

* Multiresolució, diferents resolucions, es pot treballar amb més o menys dades segons convingui * Visualització adequada de les sèries temporals: no cal emmagatzemar resolucions i informacions sinó es volen mostrar

* Cal saber canviar de resolució, exemple transformar dades periòdiques d'un mes a un any.

* Principals problemes provenen del monitoratge. Cal censurar les dades.

* Cal regularitzar les sèries temporals, o saber operar amb elles quan no són regulars. La no regularitat en el temps de mostreig pot provenir per exemple de jitter en mostrejos periòdics (problemes estudiats en el control) o d'un event-based sampling/control.

* recerca actual molt centrada a recuperar el senyal original: potser nocal. n an example we have shown a possible application of a MTSMS. The
resulting database has the information we have extracted with the
attribute interpolation function. We show that in this example we want
not an approximation to the original function but an extraction of
some interesting information. Then the database is ready to answer
time series questions keeping in mind that it holds this information
summary.

\end{itemize}





** Paper structure **








%%% Local Variables:
%%% TeX-master: "main"
%%% ispell-local-dictionary: "british"
%%% End:

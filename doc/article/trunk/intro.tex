\begin{abstract}
  Current monitoring systems are an essential part of supervising
  control as they manage a large amount of information. Data
  recollection is normally studied as a time series owing to it fits
  into sequence of values.  Thanks to the facility of designing
  monitoring hardware, the measurement of data has increased the last
  decade and there is not enough capacity to store nor process all the
  time series. Therefore, we need to design database management
  systems capable of storing and processing efficiently the time
  series. Moreover, this systems have to cope with the measurements
  not happening at regular time intervals as it is a restriction
  imposed by some time series treatment algorithms.

  In this paper a formal model for a time series database management
  system is designed.  It is called Multiresoltion Time Series
  Database Management Systems model (MTSDB) \todo{canvi de nom:
    Multiresoltion Time Series Database Management Systems
    (MTSDB)}. A Time series is compactly stored in the database and
  the information is summarised by different interpolation
  functions. From this model this kind of DBMS will be better
  understood, new implementations will be possible and we will be able
  to enhance its potential.
\end{abstract}

\section{Introduction}

Monitoring systems are an important part of control and interaction with processes. Mainly, they are in charge of recollecting data, being aware of the current state and informing the user.  It can also be thought as the core part of SCADA systems (\emph{Supervisory Control And Data Acquisition}). 

\begin{figure}[tp]
  \begin{center}
    \scriptsize 
    \begin{tikzpicture}[node distance=0.5cm]  
      \tikzset{
        mynode/.style={rectangle,rounded corners,draw=black, 
          very thick, inner sep=1em, minimum size=3em, text centered,
          groc},
        myarrow/.style={->, >=latex', shorten >=1pt, thick},
        mylabel/.style={text width=7em, text centered},
        groc/.style={top color=white, bottom color=yellow!50},
        verd/.style={top color=white, bottom color=green!50},
        roig/.style={top color=white, bottom color=red!50},
      }  

      \node[mynode]                                       (monitor)   {Monitoring system};  
      \node[mynode, below right=2cm and -0.5cm of monitor]  (bd)        {DBMS}; 
      \node[mynode, below=2cm of monitor, left=2cm and 2cm of bd]     (control)   {Controllers}; 
      \node[mynode, roig, below right=0.5cm and 3cm of monitor] (usuari)    {User};  
      \node[mynode, verd, left=2cm of control]            (actuador)  {Drives};
      \node[mynode, verd, left=3cm of monitor]            (sensor)    {Sensors};  
      
      
      \draw[myarrow] (monitor.east) --   (usuari.north)	
         node [above,sloped,midway] {Alarms}
         node [below,sloped,midway] {Current state};
      \draw[myarrow] (bd.east) --   (usuari.south)
         node [above,sloped,midway] {Historic}
         node [below,sloped,midway] {Information retrieval};
      \draw[myarrow] (sensor.east) --   (monitor.west) 
         node [above,midway] {Measures}
         node [below,midway] {Events};
      \draw[myarrow] (control.west) -- (actuador.east)
         node [above,midway] {Input};
      \draw[myarrow] (monitor.south) -- (bd.north)
         node [above,sloped,midway] {Storage};
      \draw[myarrow] (monitor.south) -- (control.north)
         node [above,sloped,midway] {Control loops};
      
    \end{tikzpicture} 
  \end{center}
  \caption{Monitoring system: from data acquisition to user information}
  \label{fig:monitoring_system}
\end{figure}

The main parts involved in monitoring can be seen in figure \ref{fig:monitoring_system}. A monitor acquires data from sensors. Data can be either measures values or process states acquired as events. 
On the one hand, data is used as the process output for the controllers, which will calculate the input values for drives. Control loops do not need to be so centralised and normally they reside nearer its process.
On the other hand, the monitoring system can inform the user with the current state of the process or generate some simple alarms such as data not acquired or a critical event has been triggered. In this figure the user can be either a human or another system such a supervisor with artificial intelligence skills.

A monitoring system recollects a
large amount of data so only a little can be observed online and the
data stored is also too big to process \parencite{keogh97}. Nevertheless, data must be analysed as there is interesting information achieve  diagnosis, prognosis, prediction, aberrant behaviour detection, and other common control tasks. Monitoring systems have databases management systems (DBMS) in order to manage data storage and information retrieval. 


In the context of monitoring, the data collected can be considered as a time
series. The research in time series data mining has
increased last decade \parencite{fu11} and the key to success is
reducing the dimension of time series in order to be able process the
data in a reasonable time.

This paper focuses on DBMS that store and treat data as time series.
Traditional DBMS, as is ones derived from relational model, are not adequate for these cases as they do not have enough facilities to manage and retrieve time series information. 

Some DBMS have already taken into account the
specialities of time series.  Time Series Data Server (TSDS) from
\textcite{weigel10} allows to select a data range from a time series and
to apply a filter when the data is retrieved.  RRDtool from
\textcite{rrdtool} applies filters and stores different data ranges when
data is stored. Then it has in mind that the sampling times can not be
equally spaced, the temporal order is essential and the value and time
must be stored together. However, this time series DBMS lack the complete definition of the relation between the three main fields involved: time series, monitoring systems and DBMS.


DBMS are based from formal models that define
the objects and operations of the abstract machine to which users
interact, such is the relational model \parencite{date}. In this paper
a model is proposed for an storage system that will keep a time series
in a multiresolution and bounded way.  In this first proposal there
are six definitions that are related to the data storage mechanism:
measure, time series, buffer, disc, resolution disc, and multiresolution
database. Some of this concepts are familiar with RRDtool
operating mode.





%%% Local Variables: 
%%% mode: latex
%%% TeX-master: "article"
%%% ispell-local-dictionary: "british"
%%% End: 

\begin{abstract}
  A time series is a collection of data measured in different instants
  in time. Thanks to the facility of designing monitoring hardware,
  the measurement of data has increased the last decade and there is
  not enough capacity to store nor process all the time
  series. Therefore, we need to design database management systems
  capable of storing and processing efficiently the time
  series. Moreover, this systems have to cope with the measurements
  not happening at regular time intervals as it is a restriction
  imposed by some time series treatment algorithms.

  In this paper a formal model for a time series database management
  system is designed.  It is called Round Robin databases model (RRD)
  \todo{pensar nom: multiresoltion and bounded} as the concepts are
  inspired by the database management system RRDtool. Time series are
  compactly stored in the database and the information is summarised
  by different interpolation functions.
\end{abstract}

\section{Introduction}

Monitoring sensor data and processing the data to achieve diagnosis,
prognosis and prediction is a common task in many fields such as
prognosis in degradation models \parencite{yu11}, navy
vessels \parencite{palmer07} or water distribution
networks \parencite{quevedo10}.  A monitoring system recollects a
large amount of data so only a little can be observed online and the
data stored is also too big to process \parencite{keogh97}.

In this context, the data collected can be considered as a time
series. Consequently, the research in time series data mining has
increased last decade \parencite{fu11}.  The key to success is
reducing the dimension of time series in order to be able process the
data in a reasonable time.

Some database management systems have taken into account the
specialities of time series.  Time Series Data Server (TSDS) from
\textcite{weigel10} allows to select a data range from a time series and
to apply a filter when the data is retrieved.  RRDtool from
\textcite{rrdtool} applies filters and stores different data ranges when
data is stored. Then it has in mind that the sampling times can not be
equally spaced, the temporal order is essential and the value and time
must be stored together. 

Database management systems are based from formal models that define
the objects and operations of the abstract machine to which users
interact, such is the relational model \parencite{date}. In this paper
a model is proposed for an storage system that will keep a time series
in a multiresolution and bounded way.  In this first proposal there
are six definitions that are related to the data storage mechanism:
measure, time series, buffer, disc, Round Robin disc, and Round Robin
database. All this concepts have been abstracted from the RRDtool
operating mode.





%%% Local Variables: 
%%% mode: latex
%%% TeX-master: "article"
%%% ispell-local-dictionary: "british"
%%% End: 

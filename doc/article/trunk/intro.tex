\begin{abstract}
  In this paper a formal model for a time series database management
  system is designed.  It is called Multiresoltion Time Series
  Database Management Systems model (MTSMS). In a MTSMS a time series
  is compactly stored in the database and the information is
  summarised by different interpolation functions. From this model
  this kind of Data Base Management Systems (DBMS) will be better
  understood, new implementations will be possible and we will be able
  to enhance its potential.

  Thanks to the facility of designing monitoring hardware, the
  measurement of data has increased the last decade and there is not
  enough capacity to store nor process all the time series. Therefore,
  we need to design DBMS capable of storing and processing efficiently
  the time series. Moreover, this systems have to cope with the
  measurements not happening at regular time intervals or holes in
  data and must achieve prognosis, data fusion, and other knowledge
  discovery techniques.

\bigskip
{\bf Keywords:} Time series, database management systems, data mining, knowledge discovery in databases, database applications.
\end{abstract}

% \bigskip
% \hrule
% {\scriptsize
%   %Report. December 23, 2011.\\
%   \{aleix,teresa,sebas\}@dipse.upc.edu\\
%   Department of Disseny i Programació de Sistemes
%   Electrònics from Universitat Politècnica de
%   Catalunya, EPSEM, 08242 Manresa, ES.  Research supported
%   by Universitat Politècnica de Catalunya (UPC).  }


% \begin{center}
%   {\footnotesize \cc\bysa}
%   {\tiny This work is licensed under a Creative Commons Attribution-ShareAlike 3.0 License.\\
%     \LaTeX\ source available at
%     \url{http://escriny.epsem.upc.edu/projects/rrb/}.
%   }
%   \end{center}




\section{Introduction}

This paper focuses on Data Base Management Systems (DBMS) that store
and treat data as time series.  Traditional DBMS, as is ones derived
from relational model, are not adequate for these cases as they do not
have enough facilities to manage and retrieve time series
information \parencite{schmidt95}.

Some DBMS have already taken into account the specificities of time
series, then called Time Series Data Base Management Systems
(TSMS) \parencite{dreyer94}.  Time Series Data
Server \parencite{weigel10} allows to select a data range from a time
series and to apply a filter when the data is retrieved.
RRDtool \parencite{rrdtool} applies filters and stores different data
ranges when data is stored, moreover it considers that the sampling
times can not be equally spaced, the temporal order is essential and
the value and time must be stored together. However, this TSMS lack
the complete definition of the relation between the three main fields
involved: time series, monitoring systems and DBMS.

Monitoring sensor data and processing this data to achieve diagnosis,
prognosis, prediction, data fusion or other time series analysis tasks are common in many fields such as prognosis in degradation models \parencite{yu11}, qualify sensor health in navy vessels \parencite{palmer07}, validate and reconstruct data in water distribution networks \parencite{quevedo10}, sensor networks information dissemination \parencite{deligiannakis07} or economic stock classification \parencite{dreyer95}. Time series data mining formalises the knowledge discovery in time series databases \parencite{last01}. 


DBMS are based from formal models that define the objects and
operations of the abstract machine to which users interact, such is
the relational model \parencite{date}. TSMS lack a consolidated formal
model, although special properties and requirements for a TSMS
have been proposed \parencite{dreyer94}.

In this paper
a model is proposed for an storage system that will keep a time series
in a multiresolution and bounded way.  In this first proposal there
are six definitions that are related to the data storage mechanism:
measure, time series, buffer, disc, resolution disc, and multiresolution
database. Some of this concepts are familiar with RRDtool
operating mode.



The paper is organised as follows. In \autoref{sec:preliminaries} the preliminaries concerning TSMS are summarised. In \autoref{sec:TSMS} the state of TSMS is shown as well as known similar systems that are designed with some TSMS properties. In \autoref{sec:MTSMS} a model for a multiresolution TSMS is presented for the first time. Possible reference implementations of this model are explored in \autoref{sec:implementation}. In sections \ref{sec:conclusion} and \ref{sec:future}  the MTSMS is concluded and future work is proposed. In \autoref{sec:notation} the main nomenclature and notation used in the paper is summarised. 




%%% Local Variables: 
%%% mode: latex
%%% TeX-master: "article"
%%% ispell-local-dictionary: "british"
%%% End: 

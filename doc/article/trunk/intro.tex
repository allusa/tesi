\section{Introduction}

Modern society depends on the smooth operation of many complex
engineering systems which provides products and services. Examples
include electric power systems, water distributions networks,
transportation systems, manufacturing processes, intelligent
buildings, communication systems, etc. The emergence of networked
embedded systems and sensor/actuator networks has made possible the
collection of large amounts of real-time data for monitoring and
control of such complex systems.  In most application, these data need
to be processed and synthesized efficiently to provide relevant
information to engineers, researchers, accident investigators,
operators, and many other users.
%
Nevertheless, before using the collected signals, it is of primary
importance to promptly detecting eventual sensor failures or
malfunctions and possibly reconstructing the incorrect signals in
order to avoid processing misleading information which may lead to
unsafe and/or inefficient actions. 
%
The correctness of those real-time data depends not only on the
measured value, but also on the time as it is collected. When
observations are collected at frequent time interval, large data sets
in the form of time series are generated~\cite{basu07:_autom}.  In
\cite{fu11} a time series is defined as a collection of observations
made chronologically, in \cite{last:hetland} they are also called time
sequences.  The procedure of processing and synthesizing information
becomes complicated if the behavior of the series varies over time.
The research in time series data mining has increased last decade and
the key to success is reducing the dimension of time series in order
to be able process the data in a reasonable time.

% Habitualment les ST es guarden en RDBS però això té problemes
Time series are usually stored in a database. Most of the time series
managing software store the data in relational database management
systems. However, using a RDBMS as a time series backend suffers some
inconvenients...

% Si les dades cal emmagartzemarles en sistemes petits (sensors, etc)
% el problemes s'agreugen i apareix la necessitat de limitar la capacitat


This paper focuses on Data Base Management Systems (DBMS) that store
and treat data as time series. These are usually known as Time Series
Data Base Management Systems (TSMS). We introduce a new data model for
a TSMS. This model allows to store time series using different time
resolutions and organized in a hierarchical way. The model is
specifically designed to cope well with bounded storage computers like
those found in sensor systems.

There are some prior works concerning TSMS. RRDtool from Oetiker
\cite{rrdtool} is a free software database management system. It is
designed to be used in monitoring systems. Because of this, it is
focused to a particular kind of data, gauges and counters, and it
lacks general time series operations. RRDtool can store multiple time
resolution data. The work in this paper is partially inspired in
RRDtool.

Cougar \cite{bonnet01} is a sensor database system. It has two
structures: one for sensor properties stored into relations and
another for time series stored into data sequences from sensors.  Time
series have specific operations and can combine relations and
sequences.

SciDB \cite{stonebraker09:scidb} and SciQL \cite{zhang11} are array
database systems. These systems are intended for science applications,
in which time series play a principal role. They structure time series
into arrays in order to achieve multidimensional analysis.

Bitemporal data and time series data are not exactly the same and so
can not be treated interchangeably \cite{schmidt95}. However, there
ara some similarities between time series and bitemporal data that can
be considered. First, extending a relational database model to manage
bitemporal data shows the way to extend relational DBMS with new types
and how to model them. Second, bitemporal data modelling settles some
time-related concepts that can be extended to time series.

The recent bitemporal data research in relational DBMS model terms
\cite{date02:_tempor_data_relat_model} marks a promising
foundation. It models bitemporal data as relations extended with time
intervals attributes and extends relational operations in order to
deal with related time aspects.




Unió entre els dos camps d'anàlisis de sèries temporals i de models de SGBD. 

- Multitud d'algoritmes i metodologies. Problemes en el monitoratge.
- Model relacional és referent




Motivació: vam observar RRDtool [citar tfm?] i volíem entendre els conceptes i modelar-los per tal de veure'n la flexibilitat: RRDtool és molt restringit amb el tractament amb el temps i amb el tipus de dades que pot tractar.



** Paper structure **



\section{Requirements for a MTSMS}\todo{o dir-li features?}

A TSMS is a special purpose DBMS aimed at storing and managing time
series and MTSMS is a TSMS with multiresolution capabilities. Next we
describe the main requirements that a TSMS must achieve, specifically
remarking the improvements where the MTSMS can contribute.



* Alta dimensió sèries temporals, cal reduir-la. Es conserven els segments de temps més interessants; multiresolució


* Multiresolució, diferents resolucions, es pot treballar amb més o menys dades segons convingui

* Cal saber canviar de resolució, exemple transformar dades periòdiques d'un mes a un any.

* Visualització adequada de les sèries temporals: no cal emmagatzemar resolucions i informacions sinó es volen mostrar


* Temporal databases. Basades en esdeveniments. Data mining basat en sèries temporals definides per parelles temps-valor; calen TSMS [schmidt i dreyer] 



* Cal censurar les dades.

* Cal regularitzar les sèries temporals, o saber operar amb elles quan no són regulars. La no regularitat en el temps de mostreig pot provenir per exemple de jitter en mostrejos periòdics (problemes estudiats en el control) o d'un event-based sampling/control.


* Aggregates, una sèrie temporal pot estar mostrant diferent informació. ex: mitjana, màxim, valor al final del període, ...

* Les sèries temmporals tenen una metainformació que cal guardar en una base de dades relacional (localització, etiquetes de classificació, últim valor mesurat, unitats, etc.) [dreyer]



* Calendari, passa a segon terme, (en contraposició a Dreyer). Es necessiten time scales estàndards (http://support.ntp.org/bin/view/Support/TimeScales): El temps es defineix com universal i constant (semblant a Unix Time Epoch, el qual representa UTC sense tenir en compte els leap seconds, tot i que potser millor seria usar TAI ja que és una representació lineal del temps). Aquests temps es pot convertir a calendari. Cal definir la interacció usuari/calendari amb temps universal.

* El temps és un nom donat al camp, qualsevol objecte que tingui la mateixa interfície que el temps pot funcionar. En el cas del valor pot ser qualsevol objecte, s'exemplifica amb reals per facilitar-ne la comprensió i per ser el més proper al time series analysis: statistical methods focused on sequences of values representing a single numeric variable [llibre-last].




* Representació: Entre dos punts de mesura, quin valor pren la sèrie temporal?.


* Disseny del model de TSMS, aleshores veurem si una TSMS pot ser implementada com a camp d'una altra DBMS o si els DBMS no són capaços de manipular TS adequadament i cal implementar TSMS específics.

* Xarxa de sensors, tsms distribuïda. Sensor dades recents, màquina grossa històrics. Quan es llança una consulta, es llança distribuïdament: si es té prou resolució es respon sinó s'envia la consulta al sensor. [bonnet01?]

* S'ha de poder calcular incrementalment, citar data streams

* Necessitem les fórmules (els interpoladors) a trossos, en el domini dels conjunts?: sí perquè la fórmula contínua necessita mètodes númerics per calcular-se? per exemple calcular l'àrea de S(t): amb integral o definida amb conjunts?
A més els interpoladors han de poder existir per a dades no númeriques com per exemple els strings.





%%% Local Variables:
%%% TeX-master: "main"
%%% ispell-local-dictionary: "british"
%%% End:

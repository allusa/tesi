%\section{Multiresolution TSMS model definition}
\label{sec:MTSMS}

In this section we design a mathematical model for the multiresolution time series database management systems (MTSMS). Some concepts come from an abstraction of RRDtool operations \cite{rrdtool}. 

A MTSMS manages time series. A time series is regarded as a chronological data collection, so it needs an appropriate management by the DBMS.
The MTSMS model is an storage solution for a time series where, in short,  the time series information is spread in different time resolutions. 

The main objects of a MTSMS are \emph{measures} and \emph{time series}. A \emph{measure} is a value measured in an instant in time and a  \emph{time series} is a collection of \emph{measures}. A database from a MTSMS contains one time series, which internally is stored in multiple resolutions of itself.

\begin{figure}[tp]
\centering
\setlength{\unitlength}{0.3mm}
../../../imatges/model/locales/mtsdb-internal_architecture.tex
\caption{Architecture of MTSMS model}
\label{fig:model:mtsdb}
\end{figure}

The general schema of the MTSMS model can be seen in figure \ref{fig:model:mtsdb}.  A \emph{multiresolution database} is a collection of \emph{resolution discs}, which temporarily accumulate the \emph{measures} in a \emph{buffer} where they are processed and finally stored in a \emph{disc}. Mainly, the data process is intended to change the time intervals between \emph{measures} in order to compact the time series information. In this way, the time series gets stored in different time resolutions, which are spread in the \emph{discs}.

\emph{Discs} are size bounded so they only contain a fixed amount of \emph{measures}. When a \emph{disc} gets full it discards a \emph{measure}. In this way, the multiresolution database is bounded in size and the time series gets stored in pieces, that is 'time subseries'.







Regarding operations, MTSMS model needs operators to change the time intervals between measures. Mainly, these operators are \emph{interpolation} functions and \emph{consolidation} actions. In this section we design the basic MTSMS model, that is six basic data model definitions -- \emph{measure}, \emph{time series}, \emph{buffer}, \emph{disc}, \emph{resolution disc}, and \emph{multiresolution database} -- and the operations to create a \emph{multiresolution database}, to add \emph{measures} and to \emph{interpolate} and \emph{consolidate} time series.







\subsection{Measure and Time Series}

A \emph{measure} is a tuple $(v,t)$ where $v\in{\mathbb{R}}$ is the
value of the measure and $t \in \mathbb{R}$ is the instant in time of
measurement. Let $m = (v,t)$ be a measure, $v$ is written as $V(m)$ and $t$
is written as $T(m)$.

The time value defines the order between measures.  Let $m = (v_m, t_m)$
and $n = (v_n, t_n)$ be two measures, then $m\geq n$ if and only if
$t_m\geq t_n$.


Time series are sequences of measures that are ordered in time. They
are also called time sequences such as in \cite{last:hetland}.
\begin{definition}[Time series]
  A \emph{time series} $S$ is defined as a set of measures
  $S = \{m_0, \ldots, m_k\}$ without repeated time values $\forall i,j:
  i\leq k, j\leq k, i\neq j : T(m_i)\neq T(m_j)$. We will refer to the
  number of measures of the time series as the cardinality of the set
  $|S|$.
\end{definition}

The order defined by measures implies a total order in a time
series. As a time series is a finite set, if it is not empty it has a
maximum and a minimum.  Let $S=\{m_0,\ldots,m_k\}$ be a time series
and $n\in S$ be a measure. The time series' maximum is $n=\max(S)$ if
and only if $\forall m \in S: n \geq m $.  Similarly, the time series'
minimum is $n=\min(S)$ if and only if $\forall m \in S: n \leq m$.

Given the order defined by time, in a time series we define the
sequence interval similarly as it is done in
\cite{last:keogh,last:hetland}.
  Let $S=\{m_0, \ldots, m_k\}$ be a time series. We define the subset
  $S(r,t] \subseteq S$ as the time series $S(r,t]=\{m\in S |
  r<T(m)\leq t\}$, where $r$ and $t$ are two instants in time.
  We also define the subset $S(r,+\infty)\subseteq S$ as the time
  series $S(r,+\infty) = \{m\in S | r< T(m) \leq T(\max(S))\}$ and the
  subset $S(-\infty,t)\subseteq S$ as the time series $S(-\infty,t) =
  \{m\in S | T(\min(S))\leq T(m) < t\}$.


The time order in time series also implies the sequence concept of
next and previous measure.
  Let $S=\{m_0, \ldots, m_k\}$ be a time series and $l\in S$ and $n$
  two measures. We define the next measure of $n$ in $S$ as
  $l=\nex_S(n)$ where $l = \min(S(T(n),+\infty))$. We define
  the previous measure of $n$ in $S$ as $l=\prev_S(n)$ where $l =
  \max(S(-\infty,T(n)))$.



\subsubsection{Regularity of time series} 

Let $S=\{m_0,\ldots,m_k\}$ be a time series, $t$ an instant in temps
and $\delta$ a duration of time, the time series' measures can be
situated in the time interval $i_0=[t,t+\delta]$ and its multiples
$i_j=[t+j\delta \,,\, t+(j+1)\delta]: j=0,1,2,\ldots$. When the
measures are equally spaced the time series is called regular. In
signal processing, these time intervals are called sampling intervals,
$\delta$ is called sampling period and $t$ is called sampling initial
time.

\begin{definition}[Regular time series]
  Let $S=\{m_0,\ldots,m_k\}$ be a time series, $t$ an instant in time
  and $\delta$ a duration of time. We define $S$ as regular if and
  only if $\forall m \in S(T(\min(S),+\infty):T(m) - T(\prev_S(m)) =
  \delta$ and $T(\min(S))=t$.
\end{definition}




\subsection{Buffer}\label{sec:model:buffer}

A buffer is a container for a regular or a no-regular time series. The
buffer regularises the time series with a function to a constant
sampling period. We call consolidation to this action. In the context
of buffers, the constant sampling period is called consolidation step
and the function is called interpolation.

\begin{definition}[Buffer]
  A \emph{buffer} is defined as the tuple $(S,\tau,\delta,f)$ where
  $S$ is a time series, $\tau$ is the last consolidation instant in
  time, $\delta$ is the duration of the consolidation step and $f$ is
  an interpolation.
\end{definition}

An empty buffer, or the initial buffer, $B_{\emptyset} =
(\emptyset,t_0, \delta_0, f)$ is a buffer that has an empty time
series, the initial time of consolidation a duration indicating the
consolidation step and an interpolation. From the empty buffer all the
instants in time of consolidation can be calculated as $t_0+k\delta,
k\in\mathbb{N}$. 

We define the operator \emph{add} to add a measure to the time series
of a buffer, 
$\text{add}:
B = (S,\tau,\delta,f) \times m = (v,t) \mapsto B'=
(S',\tau,\delta,f), S' = S \cup \{m\}
$.
\[
\text{add}: \text{Buffer} \times \text{Measure} \longrightarrow \text{Buffer}
\]



We say the buffer is ready to consolidate when a time of a measure is
bigger than the buffer's next instant in time of consolidation.  Let
$B=(S,\tau,\delta,f)$ be a buffer and $m=\max(S)$ the maximum measure,
$B$ is ready to consolidate if and only if $T(m) \geq \tau+\delta$.


\subsubsection{Interpolation}

When a buffer is ready to consolidate an interpolation functions is
used in order to consolidate the buffer.  Let $S$ be a time series and
$t_0$ and $t_f$ two instants in time, an interpolation function $f$
calculates a measure that is a summary of $S$ in the time interval
$i=[T_0,T_f]$:
\[
f: \text{Time series} \times \text{time interval} \longrightarrow
\text{Measure}
\]


In the design of the interpolation function we can interpret a time
series in different ways. \cite{last:keogh}, cites
some possible representations for time series such as \emph{Fourier
  Transforms}, \emph{Wavelets}, \emph{Symbolic Mappings} or
\emph{Piecewise Linear Representation} (PLR), remarking this last as
the most used. The most common representation is with linear
functions \cite{keogh01}.

It is also possible to represent a time series with staircase
function, that is with a piecewise constant representation.  We define
a new interpretation for time series called \emph{zero-order hold
  backwards} that is the representation used by RRDtool
\cite{lisa98:oetiker}, consisting in holding each value until the
preceding value. We exemplify it with a continuous definition of a
time series using left-continuous step functions.  Let
$S=\{m_0,\ldots,m_k\}$ be a time series, $S(t)$ is the continuous
representations along time $t$:
$
\forall t \in \mathbb{R}  ,\forall m \in S:
S(t) =  
\begin{cases}
  \text{not defined} & \text{if } t > T(\max S) \\
  V(m) & \text{if }  t\in (T(\prev_S m),T(m)]
\end{cases}
$



There can be different interpolation functions in order to summarise a
time series. The representation of the time series can also make
differences to the same interpolation functions. Next, we show some
examples for time series represented by \emph{zero-order hold
  backwards}, where the time interval of consolidation is interpreted
left-continuous $(T_0,T_f]$.


  Let $S=\{m_0,\ldots,m_k\}$ be a time series interpreted
  \emph{zero-order hold backwards} and $i=[T_0,T_f]$ be a time
  interval, the interpolation function summarises $S$ with a measure
  that is calculated from the measure values belonging to the subset
  $S(T_0,T_f]$:
  \[
  f: \text{Time series} \times \text{time interval}
  \longrightarrow \text{Measure}
  \]
  \[
  S=\{m_0,\ldots,m_k\} \times i=[T_0,T_f] \mapsto m'=(v',T_f)
  \]

  The resulting value $v'$ depends on the interpolation function, let
  $S'=S(T_0:T_f]$:

\begin{itemize}

\item \emph{Arithmetic mean interpolation} function summarises $S$
  with the mean of the measure values.
  $
  v' = \frac{1}{|S'|} \sum\limits_{\forall m\in S'} V(m)
  $

\item \emph{Maximum interpolation} function summarises $S$ with the
  maximum of the measure values.
  $
  v' = \max_{\forall m \in S'}(V(m))
  $
 %Note: \emph{Minimum interpolation} can be defined dually.
\item \emph{Last interpolation} function summarises $S$ with the
  maximum measure.
  $
  v' = \max(S')
  $

\item \emph{Area interpolation} function summarises $S$ keeping the
  area of the region.
  \[
  o=\min(S(T_0:T_f]),
  S'= S(T_0:T_f] - \{o\},
  n=\nex_S \max(S'): 
  \]\todo{S' definit general}
  \[
  \begin{split}
  :v'  = & \frac{1}{T_f-T_0} 
  \big[ (T(o)-T_0)V(o) +( T_f- T(\prev_S n) )V(n) \\
    & {}+\sum\limits_{\forall m \in S'}( T(m)- T(\prev_S m) )V(m) \big]   
   \end{split}
  \]\todo{repassar a model que sigui correcte}
  
  Note: if we consider the continuous definition of $S$ we can see the
  area interpolation function as $v' = \frac{\int_{T_0}^{T_f} S(t) dt}{T_f
    - T_0}$

\end{itemize}




\subsubsection{Consolidation}

Let $B=(S,\tau,\delta,f)$ be a buffer ready to consolidate, the
consolidation from $B$ in the time interval $i=[\tau,\tau+\delta]$
results in a measure $m'=(v,\tau+\delta)$ where $m'=f(S,i)$ and $f$ is
an interpolation function.

\begin{definition}
  The operator \emph{consolidate} calculates the measure of
  consolidation and reduces the consolidated part of the time series
  from the buffer. In a simplified, the \emph{consolidate} is only
  applied to the present consolidation interval.
  \[
  \text{consolidate}: \text{Buffer} \longrightarrow \text{Buffer}
  \times \text{Measure}
  \]
  \[
  B=(S,\tau,\delta,f) \mapsto B' \times m'
  \]
  \[
  B'= (S',\tau+\delta,\delta,f)
  \]
  \[
  S' = S(\tau+\delta,\infty)
  \]
  \[
  m' = f(S,[\tau,\tau+\delta]): f \text{ is an interpolation}
  \]
\end{definition}



\subsection{Disc}\label{sec:model:disc}

A disc is a container for a time series where its
cardinal is bounded. When the cardinal of the times series is to overcome the limit, some measures need to be discarded. 

\begin{definition}[Disc]
  A \emph{disc} is defined as the tuple $(S,k)$ where $S$ is a time
  series and $k\in\mathbb{N}$ is the maximum cardinal of $S$.
\end{definition}

An empty disc $D_{\emptyset} = (\emptyset,k)$ is a disc with an empty
time series and the $k$ maximum cardinal that $S$ is allowed to take.

 The cardinal of the times series is kept under control by the add operator. 

\begin{definition}
  The operator \emph{add} add a measure to the time series of the
  disc. If the limited cardinal of the times series is exceeded, the
  minimum measure is discarded.
  \[
  \text{add}: \text{Disc} \times \text{Measure} \longrightarrow \text{Disc}
  \]
  \[
  D=(S,k) \times m \mapsto D'= (S',k)
  \]
  \[
  S' =  
  \begin{cases}
      S\cup\{m\} &\text{if }  |S|<k\\
      (S-\{\min(S)\}) \cup \{m\} & \text{else }
    \end{cases}  
  \]
\end{definition}


\subsection{Resolution disc}

A resolution disc is a disc with a regular time series and a buffer
with a time series pending regularisation.

\begin{definition}[Resolution disc]
  A \emph{resolution disc} is a tuple $(B,D)$ where $B$
  is a buffer and $D$ is a disc.
\end{definition}
 
On the one hand, the definitions of empty buffer and empty disc imply
an empty resolution disc $R_{\emptyset} = (B_{\emptyset},D_{\emptyset})$.

On the other hand, the operators of a resolution disc are related to
the buffer and disc ones.

\begin{definition}
  Operator \emph{add} adds a measure  to the buffer of the resolution disc:
  \[
  \text{add}: \text{resolution disc} \times \text{Measure}
  \longrightarrow \text{resolution disc}
  \]
  \[
  R=(B,D) \times m \mapsto R'= (B',D)
  \]
  \[
  B'= B \text{ add } m
  \]
\end{definition}

\begin{definition}
  Let $R=(B,D)$ be a resolution disc, $R$ is ready to consolidate if
  and only if $B$ is ready to consolidate.
\end{definition}

If the resolution disc is ready to consolidate, it can be consolidated.

\begin{definition}
  Operator \emph{consolidate} calculates a consolidation measure from
  the buffer and adds it to the disc.
  \[
  \text{consolidate}: \text{resolution disc} \longrightarrow
  \text{resolution disc}
  \]
  \[
  R=(B,D) \mapsto R'= (B',D')
  \]
  \[
  B' \times m'= \text{ consolidate } B 
  \]
  \[
  D'= D \text{ add } m'
  \]
\end{definition}



\subsection{Multiresolution Database}\label{sec:model:rrd}

A multiresolution database is an archive of resolution discs which
share the input of measures, that is they store the same time
series. A time series is stored regularised and distributed with
different resolutions in the various resolution discs, as has been seen
in figure~\ref{fig:model:mtsdb}.

\begin{definition}[Multiresolution Database]
  A \emph{Multiresolution Database} is a set of resolution discs
  $M=\{R_0,\dotsc,R_d\}$.
\end{definition}

An empty multiresolution database has empty resolution discs $M_{\emptyset}=\{R_{0_{\emptyset}},\dotsc,R_{d_{\emptyset}\}}$. 
 
Generally, in a multiresolution database there are not two resolution discs
with the same information. That is, let $R_a = (B_a, D_a)$ and $R_b =
(B_b, D_b)$ be two resolution discs, its buffers
$B_a=(S_a,\tau_a,\delta_a,f_a)$ and $B_b=(S_b,\tau_b,\delta_b,f_b)$
have different consolidation interval and interpolation function
$\delta_a \neq \delta_b \wedge f_a \neq f_b$.


With reference to the operators, the add and consolidate in a multiresolution database are applied to every resolution disc it contains.


\begin{definition}
  Operator \emph{add} adds a measure to every resolution disc:
  \[
  \text{add}: \text{multiresolution database} \times \text{Measure}
  \longrightarrow \text{multiresolution database}
  \]
  \[
  M=\{R_0,\dotsc,R_d\} \times m \mapsto M' 
  \]
  \[  
  M'= \{ \forall R_i\in M: R_i \text{ add } m \}
  \]
\end{definition}


\begin{definition}
  Operator \emph{consolidate} consolidates the resolution discs that
  are ready to consolidate.
  \[
  \text{consolidate}: \text{multiresolution database} \longrightarrow
  \text{multiresolution database}
  \]
  \[
  M=\{R_0,\dotsc,R_d\} \mapsto M'
  \]
  \[
  M'= \big\{
  \forall R_i\in M: 
  \begin{cases}
    \text{ consolidate } R_i & \text{if } R_i \text{ ready to consolidate} \\
    R_i & \text{else }
  \end{cases}\big\}
  \]
\end{definition}

%%% Local Variables:
%%% TeX-master: "main"
%%% ispell-local-dictionary: "british"
%%% End:
% LocalWords: buffer buffers  MTSMS multiresolution DBMS






\section{Conclusions and future work} 

In this paper we have shown a MTSMS model, including the requirements
for these special systems and how they can be applied to an example
time series. The main objective is to store compactly a time series
and manage consistently its temporal dimension.

Our MTSMS model proposes to store a time series split into time
subseries, which we call resolution discs.  Each resolution disc has a
different resolution and is compacted with an attribute interpolation
function. Therefore, in a multiresolution database the configuration
parameters are the quantity of resolution discs and the three
parameters associated with each: the consolidation step, the attribute
interpolation function and the capacity.

The data model shown is the first step to develop a complete model for
a MTSMS but in future the operations will be defined. In this context,
there is a need for a model collecting generic properties for the
TSMS, as it can be the time series union operation or the time
interval operations. Then, the multiresolution model would be build
upon the generic TSMS model.

In an example we have shown a possible application of a MTSMS. The
resulting database has the information we have extracted with the
attribute interpolation function. We show that in this example we want
not an approximation to the original function but an extraction of
some interesting information. Then the database is ready to answer
time series questions keeping in mind that it holds this information
summary.

\todo{treballs futurs: xarxes de sensors distribuïdes}
* Xarxa de sensors, tsms distribuïda. Sensor dades recents, màquina grossa històrics. Quan es llança una consulta, es llança distribuïdament: si es té prou resolució es respon sinó s'envia la consulta al sensor.
* S'ha de poder calcular incrementalment, citar data streams
\todo{multiresolució seleccionable: no només visió més resolució més recent sinó que potser conservar bona resolució de períodes interessants}
\todo{interrelació entre el món de RDBMS i TSMS}
* Les sèries temmporals tenen una metainformació que cal guardar en una base de dades relacional (localització, etiquetes de classificació, últim valor mesurat, unitats, etc.) [dreyer]

Amb aquesta recerca volem demostrar que l'ús de SGBD per a les sèries temporals facilitarà enormement el treball amb aquestes.
L'interès actual ens fa ser optimistes per aventurar que aviat podrem gestionar les sèries temporals adequadament amb els SGBD.\todo{acabar així?}



\section*{Acknowledgements}

This work was supported by Universitat Polit\`{e}cnica de Catalunya (UPC).

Data comes from iSense project \todo{isense}.







%%% Local Variables:
%%% TeX-master: "main"
%%% ispell-local-dictionary: "british"
%%% End:

% LocalWords:  DBMS

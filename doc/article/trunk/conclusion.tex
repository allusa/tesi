\section{Related Work}

The research in time series data management has increased last decade
as the procedure of processing and synthesizing information becomes
complicated if data varies over time. From this point of view we focus
on DBMS aimed to time series where we find two main perspectives in
current research.


On the one hand, there are DBMS with some properties and operations
for time series.

RRDtool from Oetiker \cite{rrdtool} is a professional database
management system extremely used by the free software community. It is
used in professional monitoring systems and its efficiency and speed
when processing time series is being improved. Nevertheless, it is
focused to a particular kind of data, gauges and counters, and it has
not general time series operations.

Cougar \cite{bonnet01} is a sensor database system. It has two
structures: one for sensor properties stored into relations and
another for time series stored into data sequences from sensors.  Time
series have specific operations and can combine relations and
sequences.


SciDB \cite{stonebraker09:scidb} and SciQL \cite{zhang11} are array
database systems. These systems are intended for science applications,
in which time series play a principal role. They structure time series
into arrays in order to achieve multidimensional analysis.



On the other hand, there is the relational DBMS model as the common
study for DBMS theories. The relational DBMS model is continuously
evolving \cite{date:thethirdmanifesto}.

Particularly regarding time data,
intensive research has been carried in the bitemporal data field, that
is the management of history using time intervals.  The recent
temporal data research in relational DBMS model terms
\cite{date02:_tempor_data_relat_model} marks a promising
foundation. It models bitemporal data as relations extended with time
intervals attributes and extends relational operations in order to
deal with related time aspects.

Although bitemporal data and time series data are not exactly the same
and so can not be treated interchangeably \cite{schmidt95},
time series research can benefit from two aspects of this bitemporal
data research. First, it shows the way to extend relational DBMS with
new types and how to model them. Second, it settles some time-related
concepts that can apply well to time series.






\section{Conclusions and future work} 
The conclusion goes here.

Hem vist un model de MTSMS, quins requeriments tenen aquests sistemes i com es podria aplicar en un exemple. L'objectiu és emmagatzemar les sèries temporals de forma compacta i gestionar-les amb coherència amb la dimensió temporal. 


El model MTSMS es basa en emmagatezamar una sèries temporal partida en subsèries a on cadascuna té una resolució diferent i s'ha compactat amb un interpolador d'atribut. En una base de dades multiresolució els paràmetres a configurar són el nombre de subsèries i de cadascuna el període i funció de compactació associat i la mida màxima que pot prendre.

Hem vist el primer pas per definir el model de dades per a un MTSMS, en un futur caldrà definir-ne les operacions i un model de dades que agrupi propietats generals dels TSMS com poden ser la unió de sèries temporals, el càlcul d'itervals temporals, etc. So we are claiming for a need of a TSMS model where the multiresoltion model can be build.


 In an example
we have shown that with this attribute interpolation function we want
not an approximation to the original function but an extraction of
some interesting information.

Amb aquesta recerca volem demostrar que l'ús de SGBD per a les sèries temporals facilitarà enormement el treball amb aquestes.
L'interès actual ens fa ser optimistes per aventurar que aviat podrem gestionar les sèries temporals adequadament amb els SGBD.



\section*{Acknowledgements}

This work was supported by Universitat Polit\`{e}cnica de Catalunya (UPC).

Data comes from iSense project \todo{isense}.







%%% Local Variables:
%%% TeX-master: "main"
%%% ispell-local-dictionary: "british"
%%% End:

% LocalWords:  DBMS

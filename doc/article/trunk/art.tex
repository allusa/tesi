\section{State of the art}


\todo{
  * Sèries temporals (històrics, predicció, diagnosis, prognosis, etc.)
  * Mostreig: docs quan període de mostreig no regular
  * Bases de dades (docs d'emmagatzematge quan la memòria és finita, docs quan període de mostreig no és regular, altres sistemes semblants (comercials,prototips))
}

En aquest capítol se situen els sistemes de gestió de bases de dades (SGBD) per sèries temporals en el context de la mineria de dades de sèries temporals (\emph{time series data mining}), el qual també es considerat com mineria de dades per  detectar automàticament coneixement (\emph{knowledge discovery databases}). Els SGBD de model Round Robin (RRD) pertanyen a aquest context ja que  emmagatzemen sèries temporals  de les quals es vol aconseguir informació rellevant.


El capítol comença resumint l'estat de les sèries temporals en aquest camp de mineria; és a dir d'emmagatzematge i tractament. A continuació es llisten algunes aplicacions informàtiques que han implementat models de la mineria de sèries temporals. Finalment, es descriu l'estat actual de l'aplicació RRDtool, la qual també es classifica en aquest camp.





\subsection{Monitoring}

Monitoring sensor data and processing the data to achieve diagnosis,
prognosis and prediction is a common task in many fields such as
prognosis in degradation models \parencite{yu11}, navy
vessels \parencite{palmer07} or water distribution
networks \parencite{quevedo10}.  A monitoring system recollects a
large amount of data so only a little can be observed online and the
data stored is also too big to process \parencite{keogh97}.

\textcite{quevedo10} shows the enormous information residing in complex remote systems. This information comes from many sensors that are distributed in the area. A SCADA (\emph{Supervisory Control And Data Acquisition}) is the periodically recollecting system responsible for the centralisation of the sensing values. During the recollection time two problems may appear: some values that have not been obtained and values that are erroneous. The database managers can not cope with such data as the historical information would be inconsistent. Therefore, a validation process must be taken in order to assure that stored data is correct. If data is invalid, a reconstruction process may take place in order to modify values to estimated ones. In \cite{quevedo10} validation and reconstruction techniques are applied to water distribution networks. 



In the monitoring context, the data collected can be considered as a time
series.  Time series theory formalises the analysis methods such as the validations and reconstruction processes.



\subsection{Time series data mining}

The analysis of time series includes different activities such as prediction in economics, weather forecasts, quality control in business, etc. In this context, time series data mining is the study and management of large collections of chronological data. Furthermore, this data is continually increasing in size as time goes on.

Mining research in time series has increased the last decade as shown in \textcite{fu11}. Fu summarises exhaustively the current state of the art in time series data mining  and concludes that there is still room for us to further investigate and develop'. Mining tasks have had intensively research but the problem of  time series representation needs better solutions because of high dimensionality of time data. Furthermore, it is said to be one of the ten challenging problems in data mining \parencite{yangwu06}.
 
Time series data mining research  is given at mainly four tasks \parencite{keogh02}: indexing, clustering, classification and segmentation. In order to solve these tasks, lots of experimental algorithms have been proposed by different authors. \textcite{keogh02} compare and evaluate some of these algorithms with the same datasets and they recommend the time series data mining community to follow the same benchmark when evaluating the performance of similar algorithms. 

Representation of time series is a common step done before the above four tasks.
\emph{Piecewise Linear Representation} (PLR) \parencite{keogh97,keogh98}  is one representation widely used as it is noted that human vision segments curvatures into lineal segments. In \textcite{keogh00,keogh01} time series representation is studied in order to reduce the high dimension component and to index more efficiently. Two representation based on PLR are proposed: \emph{Piecewise Aggregate Aproximation} (PAA) and \emph{Adaptive Piecewise Constant Approximation} (APCA). Keogh concludes that both approximations are closer to the original time series and they have better computing time than other complicated techniques such as \emph{Discrete Fourier Transform} (DFT), \emph{Singular Value Decomposition} (SVD), or \emph{Discrete Wavelet Transform} (DWT).



\subsection{Database management systems}

Database management systems (DBMS)\todo{intro general als DBMS, explicant que els relacionals són els típics}


\subsubsection{Time series DBMS}

Time series analysis techniques must be computed in specific DBMS in order to process data efficiently. Moreover, the amount of data is increasing as data acquisition becomes easier and there is more storage capacity. 
Mainly, time series data is different from other data types in the sense that the series values are dependent from another variable: the time. Consequently, DBMS can not process independently values and times; it can be generally stated that they must keep the temporal coherence. 


As said in \textcite{assfalg08:thesis}, the temporal coherence can be divided in two types. Firstly, there is  \emph{bitemporal data} which consists in storing the valid time during which an event is true and storing the transaction time in which the event is stored in the database. That is, tho states -- true or false -- are annotated for each observation.  Secondly, there is \emph{time series data} which consists in describing a collection of data that depends from time. He also says that the first type can be expressed in terms of second type.

Relational DBMS can implement \emph{bitemporal data}. Then they are known as temporal databases \parencite[ch.\ 22]{date}. However, relational DBMS are not adequate for time series. The relational model is capable to describe time series when they are thought as historical data but the design of relational DBMS would difficult the operations  needed by time series. Theses time series operations are mainly based in time ranges and need time zones conversions, rotations of table registers and file size maintained at bounded levels.



Els SGBD que implementen operacions per a sèries temporals es poden anomenar \emph{Time Series Database Systems} (TSDS),~\cite{tsds}. Les TSDS Estan optimitzades per gestionar les dades segons les operacions de temps i rotació, les quals són molt comunes en la gestió de les sèries temporals.  A més també cal controlar el creixement de la base de dades i la consulta ha de ser flexible i d'alta velocitat,~\cite{keogh10:isax}. Per exemple, s'han de poder visualitzar les evolucions tant d'una setmana com d'un any sense haver de fer càlculs complicats amb els valors emmagatzemats. 


A continuació es llisten  bases de dades optimitzades per a sèries temporals.

A{\ss}falg,~\cite{assfalg08:thesis}, presenta un TSDS que és capaç de
cercar similituds, també anomenades distàncies, entre sèries temporals. Principalment utilitza llindars per comparar en cada interval si les dues sèries temporals s'assemblen. A partir d'aquest mètode desenvolupa algoritmes que calculen de manera eficient per a les sèries temporals i en concret els implementa en una aplicació anomenada T-Time, la qual descriu a~\cite{assfalg08:ttime}.

Keogh i Camerra~\cite{keogh08:isax,keogh10:isax}, 
estudien l'anàlisi i l'indexat de co\l.lecions massives de sèries temporals. Descriuen que el problema principal del tractament rau en l'indexat de les sèries temporals i proposen mètodes per calcular-lo de manera eficient. El mètode principal que desenvolupen està basat en l'aproximació a trossos constants de la sèrie temporal (PAA,~\cite{keogh00}) i ho implementen en una estructura de dades que anomenen iSAX (\emph{indexable Symbolic Aggregate approXimation}),~\cite{isax}. Amb aquesta eina s'obtenen representacions de sèries temporals que permeten reduir l'espai emmagatzemat i indexar tant bé com altres mètodes de representació més complexos.




En resum, aquests SGBD per sèries temporals bàsicament resolen els problemes d'anàlisis de sèries temporals.
Però cap d'aquestes sol atendre la relació entre la base de dades i el sistema de monitoratge, és a dir la manera com s'adquireixen les dades. En aquest pas intermig hi ha un sèrie de problemes, com per exemple forats, dades falses, irregularitat en els temps de mostreig, que cal gestionar correctament. Concretament un dels problemes que no s'atén és el de mostreig irregular ja que es considera que les mostres estan a intervals regulars (equi-espaiades) encara que els sistemes de monitoratge informàtics sovint no són capaços de complir-ho amb exactitud sinó que presenten una certa variació en els temps de mesura. 

Així doncs, quan es prenen mesures d'un sistema productiu, aquests problemes apareixen i són de difícil solució.
Les bases de dades RRDtool tenen en compte aquests problemes intermitjos entre el sistema de monitoratge i el sistema d'emmagatzematge i tractament. 




%\subsection{Base de dades RRDtool}

RRDtool és un SGBD per sèries temporals més. \todo{sap tractar molt bé les sèries temporals però està molt afitat a un tipus de dades concrets i no permet operacions tant generals com els altres SGBD}

RRDtool ha inspirat aquest treball


RRDtool és un SGBD per a sèries temporals que despunta en l'àmbit de programari lliure. Hi ha una llista de projectes que utilitzen RRDtool que poden trobar-se indicats a l'apartat \emph{Projects using RRDtool} de~\cite{rrdtool}.
Entre d'altres, s'utilitza en sistemes de monitoratge professionals com per exemple Nagios,~\cite{nagios}, o Icinga,~\cite{icinga}, també populars dins del programari lliure, o en el montior MRTG (The Multi Router Traffic Grapher),~\cite{mrtg}, del mateix creador que RRDtool. Aquests monitors fan un ús complet de les possibilitats de RRDtool i li cedeixen tot el control de l'emmagatzematge de mesures i el posterior tractament i representació gràfica de les dades. 
L'ús de RRDtool permets a aquestes aplicacions centrar-se plenament en la problemàtica de l'adquisició de dades i la gestió d'alarmes.

En l'evolució de RRDtool destaquen dues millores significatives.
La primera, descrita per Oetiker a~\cite{lisa98:oetiker}, va consistir en independitzar la base de dades RRDtool del sistema de monitoratge MRTG i dissenyar-la amb l'estructura Round Robin que la caracteritza. La segona, feta per Brutlag,~\cite{lisa00:brutlag}, ha aportat la possibilitat de fer prediccions i detecció de comportaments aberrants basant-se en algoritmes de predicció exponencials i de Holt-Winters. 


L'evolució actual de RRDtool se centra en aspectes informàtics i consisteix a millorar la rapidesa i eficiència en el processament de les sèries temporals. És el cas de Plonka i Carder que a~\cite{carder:rrdcached,lisa07:plonka} dissenyen l'aplicació \verb+rrdcached+ per incrementar el rendiment de RRDtool, la qual demostren en un sistema de monitoratge amb moltes bases de dades funcionant simultàniament.  També \verb+JRobin+,~\cite{jrobin}, que és una implementació en Java de RRDtool que millora els accessos de lectura i escriptura a la base de dades i té una eina de gràfics més perfeccionada.
És significatiu l'ús incipient d'aquest sistema en experimentació. Zhang,~\cite{zhang07}, i Chilingaryan,~\cite{chilingaryan10}, per exemple, usen RRDtool per emmagatzemar de dades experimentals i posteriorment fer predicció o validació.
  

En l'àmbit dels SGBD els sistemes relacionals van fixar una fita que ha tingut una transcendència posterior de  primer ordre. En bona part aquest èxit dels SGBD relacionals es deu al fet que es basen en un model matemàtic sòlid,~\cite{date}.
En el cas de RRDtool no existeix  cap model que descrigui el sistema i es objectiu d'aquest treball proposar-ne un. El model per a SGBD Round Robin es dissenya  al capítol






%%% Local Variables: 
%%% mode: latex
%%% TeX-master: "article"
%%% ispell-local-dictionary: "british"
%%% End: 


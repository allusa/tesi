  % * Sèries temporals (històrics, predicció, diagnosis, prognosis, etc.)
  % * Mostreig: docs quan període de mostreig no regular
  % * Bases de dades (docs d'emmagatzematge quan la memòria és finita, docs quan període de mostreig no és regular, altres sistemes semblants (comercials,prototips))

\section{State of the art}

Obtaining information from environment is a complex task that has a user, either human or machine, as the final target as stated in figure \ref{fig:monitoring_system}. Mainly, the processes between data acquisition and user information can be divided into three focus.


Monitoring systems recollect data from sensors eventually from events and periodically from measures, in the latter case some problems arise when monitoring can not be done at regular intervals. Then the recollected data is used in different application fields. There is intensive research in applying techniques to obtain information from this data.

Time series data mining analyses the storage of chronological data from which relevant information want to be retrieved. Sometimes, this extraction of information is also referred as knowledge discovery or artificial intelligence.
In the monitoring context, the data collected can be considered as a time series.

Database management systems are the computer systems implementing storage and retrieval of information. Time series data mining techniques must be implemented in these systems paying attention to their especial needs. Then, they are known as time series database management systems.



\subsection{Monitoring and applications}

Monitoring sensor data and processing the data to achieve diagnosis,
prognosis and prediction is a common task in many fields such as
prognosis in degradation models \parencite{yu11}, navy
vessels \parencite{palmer07} or water distribution
networks \parencite{quevedo10}.  A monitoring system recollects a
large amount of data so only a little can be observed online and the
data stored is also too big to process \parencite{keogh97}.

\textcite{quevedo10} shows the enormous information residing in complex remote systems. This information comes from many sensors that are distributed in the area. A SCADA (\emph{Supervisory Control And Data Acquisition}) is the periodically recollecting system responsible for the centralisation of the sensing values. During the recollection time two problems may appear: some values that have not been obtained and values that are erroneous. The database managers can not cope with such data as the historical information would be inconsistent. Therefore, a validation process must be taken in order to assure that stored data is correct. If data is invalid, a reconstruction process may take place in order to modify values to estimated ones. In \cite{quevedo10} validation and reconstruction techniques are applied to water distribution networks. 

In the monitoring context, the data collected can be considered as a time
series.  Time series theory formalises the analysis methods such as the validations and reconstruction processes.



\todo{docs quan període de mostreig no és regular}





\subsection{Time series data mining}

The analysis of time series includes different activities such as prediction in economics, weather forecasts, quality control in business, etc. In this context, time series data mining is the study and management of large collections of chronological data. Furthermore, this data is continually increasing in size as time goes on.

Mining research in time series has increased the last decade as shown in \textcite{fu11}. Fu summarises exhaustively the current state of the art in time series data mining  and concludes that there is still room for us to further investigate and develop'. Mining tasks have had intensively research but the problem of  time series representation needs better solutions because of high dimensionality of time data. Furthermore, it is said to be one of the ten challenging problems in data mining \parencite{yangwu06}.
 
Time series data mining research  is given at mainly four tasks \parencite{keogh02}: indexing, clustering, classification and segmentation. In order to solve these tasks, lots of experimental algorithms have been proposed by different authors. \textcite{keogh02} compare and evaluate some of these algorithms with the same datasets and they recommend the time series data mining community to follow the same benchmark when evaluating the performance of similar algorithms. 

Representation of time series is a common step done before the above four tasks.
\emph{Piecewise Linear Representation} (PLR) \parencite{keogh97,keogh98}  is one representation widely used as it is noted that human vision segments curvatures into lineal segments. In \textcite{keogh00,keogh01} time series representation is studied in order to reduce the high dimension component and to index more efficiently. Two representation based on PLR are proposed: \emph{Piecewise Aggregate Aproximation} (PAA) and \emph{Adaptive Piecewise Constant Approximation} (APCA). Keogh concludes that both approximations are closer to the original time series and they have better computing time than other complicated techniques such as \emph{Discrete Fourier Transform} (DFT), \emph{Singular Value Decomposition} (SVD), or \emph{Discrete Wavelet Transform} (DWT).



\subsection{Database management systems}

Database management systems (DBMS)\todo{intro general als DBMS, explicant que els relacionals són els típics}

En l'àmbit dels SGBD els sistemes relacionals van fixar una fita que ha tingut una transcendència posterior de  primer ordre. En bona part aquest èxit dels SGBD relacionals es deu al fet que es basen en un model matemàtic sòlid,~\cite{date}.

Els SGBD habituals són els relacionals,  disposen d'un model matemàtic consolidat.

%\section{Introducció}

Segons Date,~\cite{date}, ``una base de dades és un contenidor informàtic per a una co\l.lecció de dades''. El sistemes informàtics que tracten amb bases de dades s'anomenen sistemes de gestió de bases de dades (SGBD) i tenen l'objectiu d'emmagatzemar informació i permetre consultar i afegir aquesta informació  per part dels usuaris.
Per complir aquests objectius, els SGBD ofereixen a l'usuari diferents operacions a fer amb la base de dades, com per exemple crear-la, afegir dades o consultar informació a partir de les dades emmagatzemades.

Els SGBD es basen en teories matemàtiques que reben el nom de model de dades, un SGBD és una implementació d'un model de dades.
Segons Date, ``un model de dades és una definició abstracta, auto continguda i lògica dels objectes, de les operacions i  de la resta que conjuntament constitueixen la màquina abstracta amb la que els usuaris interactuen. Els objectes permeten modelar l'estructura de les dades. Les operacions permeten modelar el comportament''.

Les bases de dades Round Robin són bases de dades que contenen sèries temporals. Les sèries temporals són una co\l.lecció de dades mesurades en diferents instants de temps i necessiten un tractament adequat per part de la base de dades. 
El model de dades Round Robin és una solució d'emmagatzematge per a les sèries temporals que, resumint, consisteix a repartir la informació d'una sèrie temporal en intervals de temps diferents.

Segons Date, \cite{date}, ``una base de dades és un contenidor
informàtic per a una co\l.lecció de dades''. En el context dels SGBD's
cal diferenciar:
\begin{itemize}
\item El model d'una bases de dades, que és el model matemàtic a on es
  descriu teòricament l'estructura de les dades, per exemple el model
  relacional o model Round Robin.

\item El sistema de gestió de bases de dades, que és la implementació
  d'un model de dades, per exemple postgresql o
  RRDtool. %Segons l'esquema de comunicació, també es poden anomenar com a sistema servidor de bases de dades, per exemple postgresql.

\item La base de dades, que és una instància d'un sistema de gestió de
  bases de dades, per exemple la base de dades dels estudiants o la
  base de dades de la temperatura de l'escola.
\end{itemize}


Extensió de tipus i operacions \parencite{stonebraker86}

\todo{docs quan la memòria és finita}

\subsubsection{Time series DBMS}

Time series analysis techniques must be computed in specific DBMS in order to process data efficiently. Moreover, the amount of data is increasing as data acquisition becomes easier and there is more storage capacity. 
Mainly, time series data is different from other data types in the sense that the series values are dependent from another variable: the time. Consequently, DBMS can not process independently values and times; it can be generally stated that they must keep the temporal coherence. 


As said in \textcite{assfalg08:thesis}, the temporal coherence can be divided in two types. Firstly, there is  \emph{bitemporal data} which consists in storing the valid time during which an event is true and storing the transaction time in which the event is stored in the database. That is, tho states -- true or false -- are annotated for each observation.  Secondly, there is \emph{time series data} which consists in describing a collection of data that depends from time. He also says that the first type can be expressed in terms of second type.

Relational DBMS can implement \emph{bitemporal data}. Then they are known as temporal databases \parencite[ch.\ 22]{date}. However, relational DBMS are not adequate for time series. The relational model is capable to describe time series when they are thought as historical data but the design of relational DBMS would difficult the operations  needed by time series. Theses time series operations are mainly based in time ranges and need time zones conversions, rotations of table registers and file size maintained at bounded levels.

When a DBMS implements especial operations for time series they can be called \emph{Time Series Database Management Systems} (TSDB). TSDB are optimised to process data with time and rotation operations, which are very common in time series management. Moreover the growth of the data must be under control and the fetch operations must be flexible and fast \parencite{keogh10:isax}. As instance, the temporal changes either for a week or for a year must be processed and it can not take very long.




\paragraph{Similar systems} Next, we summarise some current TSDB.

\paragraph{T-Time}  \textcite{assfalg08:thesis} shows a TSDB that can do similarity search, which is calculated as distances between time series. Mainly, two time series are marked as similar if they distance is less than a threshold in each interval. From this method efficient algorithms are developed and implemented in a program called T-Time, which is described in \cite{assfalg08:ttime}.

 
\paragraph{iSAX} \textcite{keogh08:isax,keogh10:isax} analyse and index massive collections of time series. They describe that the main problem is in the indexing of time series and they propose methods that process efficiently. The first method proposed is based on the constant piecewise approximation, the PAA \parencite{keogh00}. It is implemented in a data management structure called \emph{indexable Symbolic Aggregate approXimation} (iSAX) \parencite{isax}. The time series representation obtained with this tool allows to reduce the stored space and to index faster and with the same quality as other more complex representation methods.
 

\paragraph{TSDS} \textcite{weigel10} notice the necessity to show the data over its full time range and not only subsets of data as it is usually provided. 
They develop  the software package \emph{Time Series Data Server} (TSDS) \parencite{tsds} where time series data can be entered and then requested by date ranges or by applying different filters and operations to the time series data.


\paragraph{RRDtool} RRDtool from \textcite{rrdtool} is a professional TSDB extremely used by the free software community. \emph{Projects using RRDtool} in \cite{rrdtool} lists some projects that are currently using RRDtool in different fields. Among others, it is used in professional monitoring systems such as Nagios \parencite{nagios} or Icinga \parencite{icinga}, both also popular in the free software community, or  in the Multi Router Traffic Grapher (MRTG) \parencite{mrtg}, which is a monitor from the same creator of RRDtool. 
These monitors explore the possibilities of RRDtool to its limits and they transfer to RRDtool the data storing and operations of the data stored . 
Using RRDtool allow the monitors to focus on the data acquisition and warnings management. RRDTool has also an adaptation into Java called \emph{JRobin} \parencite{jrobin}.


In the evolution of RRDtool there are two improvements that must be noticed.
Firstly, the RRDtool management system was separated from MRTG monitoring system by \textcite{lisa98:oetiker} and designed with his characteristically Round Robin structure. Secondly, \textcite{lisa00:brutlag} extended RRDtool with smoothing and Holt-Winters algorithms which allows RRDtool to predict and detect aberrant behaviour in data.  
Nowadays, RRDtool efficiency and speed when processing time series is being improved. \textcite{carder:rrdcached} have designed \emph{rrdcached}, an apllication to improve RRDtool's performance, which in \cite{lisa07:plonka} is shown  in systems with large amount of databases working simultaneously.
It is noticeable the emerging use of RRDtool in testing environments. As instance, \textcite{zhang07} and \textcite{chilingaryan10} use RRDtool for storing experimental data and then predicting or validating the data.


\subsubsection{Overview}

TSDB are mainly concerned with the time series analysis problems. However, the relation between database and monitoring system are not addressed. That is mainly speaking of the problems inferred from data acquisition which  must be treated properly, such as holes in data, false data or sampling time irregularities. 
The latter is one of the most overlooked problem by TSDB as samples are typically considered equi-spaced, although computer monitoring systems present some variation and delays in sampling times. 

RRDtool is an exception as being a productive system makes its data processing and storing closer to monitoring systems. Nevertheless, it is focused to a particular kind of data, gauges and counters, and it has not so general time series operations as the others TSDB.







%%% Local Variables: 
%%% mode: latex
%%% TeX-master: "article"
%%% ispell-local-dictionary: "british"
%%% End: 


% LocalWords:  RRDtool

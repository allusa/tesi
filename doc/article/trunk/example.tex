\todo{Teresa: usar dades iSense}

Next we show an example database for a real time series data. Original
data is shown in figure~\ref{fig:exemple:original} for a period of one
year and a half. In this plot we can see that there is missing
data and some outlying observations.


\begin{figure}[tp]
\centering
\includegraphics{\experiment/original.pdf}
\caption{Example of a time series data}
\label{fig:exemple:original}
\end{figure}



\begin{figure}[tp]
\centering
%mrd.afegeix_disc(h5,24,mitjana,zero)
%mrd.afegeix_disc(d2,20,mitjana,zero)
%mrd.afegeix_disc(d15,12,mitjana,zero)
%mrd.afegeix_disc(d50,12,mitjana,zero)
\tiny
\begin{center}
%\begin{multicols}{4} 


    \begin{picture}(14,12)(-7,-6)
    \put(0,-1){\makebox(0,0)[c]{{\color{blue}50 days}}}
      \put(0,0){\circle{10}}
      \put(5,0){\circle{0.8}}
      \put(4.33,2.5){\circle{0.8}}
      \put(2.5,4.33){\circle{0.8}}
      \put(0,5){\circle{0.8}}
      \put(-2.5,4.33){\circle{0.8}}   
      \put(-4.33,2.5){\circle{0.8}}
      \put(-5,0){\circle{0.8}}
      \put(-4.33,-2.5){\circle{0.8}}
      \put(-2.5,-4.33){\circle{0.8}} 
      \put(0,-5){\circle{0.8}}
      \put(2.5,-4.33){\circle{0.8}} 
      \put(4.33,-2.5){\circle{0.8}}
      \put(0,0){\vector(0,1){5}}
      \put(0,0){\oval(5,5)[t]}
      \put(-2.5,0){\makebox(0,0)[c]{$\vee$}}
    \end{picture}
%
    \begin{picture}(14,12)(-7,-6)
    \put(0,-1){\makebox(0,0)[c]{{\color{brown}15 days}}}
      \put(0,0){\circle{10}}
      \put(5,0){\circle{0.8}}
      \put(4.33,2.5){\circle{0.8}}
      \put(2.5,4.33){\circle{0.8}}
      \put(0,5){\circle{0.8}}
      \put(-2.5,4.33){\circle{0.8}}   
      \put(-4.33,2.5){\circle{0.8}}
      \put(-5,0){\circle{0.8}}
      \put(-4.33,-2.5){\circle{0.8}}
      \put(-2.5,-4.33){\circle{0.8}} 
      \put(0,-5){\circle{0.8}}
      \put(2.5,-4.33){\circle{0.8}} 
      \put(4.33,-2.5){\circle{0.8}}
      \put(0,0){\vector(0,1){5}}
      \put(0,0){\oval(5,5)[t]}
      \put(-2.5,0){\makebox(0,0)[c]{$\vee$}}
    \end{picture}
%
    \begin{picture}(14,12)(-7,-6)
    \put(0,-1){\makebox(0,0)[c]{{\color{red}2 days}}}
      \put(0,0){\circle{10}}
      %\put(5,0){\circle{0.8}}
      \put(4.82,1.29){\circle{0.8}}
      \put(4.33,2.5){\circle{0.8}}
     \put(3.5,3.5){\circle{0.8}}
      \put(2.5,4.33){\circle{0.8}}
      \put(1.29,4.82){\circle{0.8}}
      %\put(0,5){\circle{0.8}}
      \put(-1.29,4.82){\circle{0.8}}
      \put(-2.5,4.33){\circle{0.8}}
       \put(-3.5,3.5){\circle{0.8}} 
      \put(-4.33,2.5){\circle{0.8}}
    \put(-4.82,1.29){\circle{0.8}}
      %\put(-5,0){\circle{0.8}}
    \put(-4.82,-1.29){\circle{0.8}}
      \put(-4.33,-2.5){\circle{0.8}}
      \put(-3.5,-3.5){\circle{0.8}} 
      \put(-2.5,-4.33){\circle{0.8 } } 
      \put(-1.29,-4.82){\circle{0.8 }}
      % \put(0,-5){\circle{0.8 }}
     \put(1.29,-4.82){\circle{0.8 }}
      \put(2.5,-4.33){\circle{0.8}}
      \put(3.5,-3.5){\circle{0.8}} 
      \put(4.33,-2.5){\circle{0.8}}
  \put(4.82,-1.29){\circle{0.8}}
      \put(0,0){\vector(0,1){5}}
      \put(0,0){\oval(5,5)[t]}
      \put(-2.5,0){\makebox(0,0)[c]{$\vee$}}
    \end{picture}
%
    \begin{picture}(14,12)(-7,-6)
    \put(0,-1){\makebox(0,0)[c]{{\color{cyan}5 hours}}}
      \put(0,0){\circle{10}}
      \put(5,0){\circle{0.8}}
      \put(4.82,1.29){\circle{0.8}}
      \put(4.33,2.5){\circle{0.8}}
     \put(3.5,3.5){\circle{0.8}}
      \put(2.5,4.33){\circle{0.8}}
      \put(1.29,4.82){\circle{0.8}}
      \put(0,5){\circle{0.8}}
      \put(-1.29,4.82){\circle{0.8}}
      \put(-2.5,4.33){\circle{0.8}}
       \put(-3.5,3.5){\circle{0.8}} 
      \put(-4.33,2.5){\circle{0.8}}
    \put(-4.82,1.29){\circle{0.8}}
      \put(-5,0){\circle{0.8}}
    \put(-4.82,-1.29){\circle{0.8}}
      \put(-4.33,-2.5){\circle{0.8}}
      \put(-3.5,-3.5){\circle{0.8}} 
      \put(-2.5,-4.33){\circle{0.8 } } 
      \put(-1.29,-4.82){\circle{0.8 }}
\put(0,-5){\circle{0.8 }}
     \put(1.29,-4.82){\circle{0.8 }}
      \put(2.5,-4.33){\circle{0.8}}
      \put(3.5,-3.5){\circle{0.8}} 
      \put(4.33,-2.5){\circle{0.8}}
  \put(4.82,-1.29){\circle{0.8}}
      \put(0,0){\vector(0,1){5}}
      \put(0,0){\oval(5,5)[t]}
      \put(-2.5,0){\makebox(0,0)[c]{$\vee$}}
    \end{picture}


%\end{multicols}

\vspace{-10pt}

\setlength{\unitlength}{900sp}
\begin{picture}(14460,5066)(7322,-7148)
\thinlines
{\color[rgb]{0,0,0}\put(7300,-6271){\line( 0,-1){386}}
}%
{\color[rgb]{0,0,0}\put(7782,-6271){\line( 0,-1){386}}
}%
{\color[rgb]{0,0,0}\put(8263,-6271){\line( 0,-1){386}}
}%
{\color[rgb]{0,0,0}\put(8745,-6271){\line( 0,-1){386}}
}%
{\color[rgb]{0,0,0}\put(9227,-6271){\line( 0,-1){386}}
}%
{\color[rgb]{0,0,0}\put(9709,-6271){\line( 0,-1){386}}
}%
{\color[rgb]{0,0,0}\put(10191,-6271){\line( 0,-1){386}}
}%
{\color[rgb]{0,0,0}\put(10673,-6271){\line( 0,-1){386}}
}%
{\color[rgb]{0,0,0}\put(11155,-6271){\line( 0,-1){386}}
}%
{\color[rgb]{0,0,0}\put(11637,-6271){\line( 0,-1){386}}
}%
{\color[rgb]{0,0,0}\put(12119,-6271){\line( 0,-1){386}}
}%
{\color[rgb]{0,0,0}\put(12600,-6271){\line( 0,-1){386}}
}%
{\color[rgb]{0,0,0}\put(13082,-6271){\line( 0,-1){386}}
}%
{\color[rgb]{0,0,0}\put(13564,-6271){\line( 0,-1){386}}
}%
{\color[rgb]{0,0,0}\put(14046,-6271){\line( 0,-1){386}}
}%
{\color[rgb]{0,0,0}\put(14528,-6271){\line( 0,-1){386}}
}%
{\color[rgb]{0,0,0}\put(15010,-6271){\line( 0,-1){386}}
}%
{\color[rgb]{0,0,0}\put(15492,-6271){\line( 0,-1){386}}
}%
{\color[rgb]{0,0,0}\put(15974,-6271){\line( 0,-1){386}}
}%
{\color[rgb]{0,0,0}\put(16456,-6271){\line( 0,-1){386}}
}%
{\color[rgb]{0,0,0}\put(16938,-6271){\line( 0,-1){386}}
}%
{\color[rgb]{0,0,0}\put(17419,-6271){\line( 0,-1){386}}
}%
{\color[rgb]{0,0,0}\put(17901,-6271){\line( 0,-1){386}}
}%
{\color[rgb]{0,0,0}\put(18383,-6271){\line( 0,-1){386}}
}%
{\color[rgb]{0,0,0}\put(18865,-6271){\line( 0,-1){386}}
}%
{\color[rgb]{0,0,0}\put(19347,-6271){\line( 0,-1){386}}
}%
{\color[rgb]{0,0,0}\put(19829,-6271){\line( 0,-1){386}}
}%
{\color[rgb]{0,0,0}\put(20311,-6271){\line( 0,-1){386}}
}%
{\color[rgb]{0,0,0}\put(20793,-6271){\line( 0,-1){386}}
}%
{\color[rgb]{0,0,0}\put(21275,-6271){\line( 0,-1){386}}
}%
{\color[rgb]{0,0,0}\put(7300,-6271){\line( 0,-1){1157}}
}%
{\color[rgb]{0,0,0}\put(9709,-6271){\line( 0,-1){1157}}
}%
{\color[rgb]{0,0,0}\put(12119,-6271){\line( 0,-1){1157}}
}%
{\color[rgb]{0,0,0}\put(14528,-6271){\line( 0,-1){1157}}
}%
{\color[rgb]{0,0,0}\put(16938,-6271){\line( 0,-1){1157}}
}%
{\color[rgb]{0,0,0}\put(19347,-6271){\line( 0,-1){1157}}
}%
{\color[rgb]{0,0,0}\put(21756,-6271){\line( 0,-1){1157}}
}%
{\color[rgb]{0,0,0}\put(7300,-6271){\line( 1, 0){14456}}
}%

\put(7322,-6271){\line( 0,1){3000}}
\put(21756,-7783){\makebox(0,0)[b]{now}}%
\put(7322,-7783){\makebox(0,0)[b]{600 days back}}%

\color{blue}
\put(21782,-5928){\line( -1,0){14460}}
\put(21782,-5928){\line( 0,1){779}}
\put(21782,-5149){\line( -1,0){14460}}
\put(7322,-5928){\line( 0,1){779}}
\put(14530,-5450){\makebox(0,0)[c]{600 days}}

\color{brown}
\put(21782,-5149){\line( 0,1){779}}
\put(21782,-4370){\line( -1,0){4438}}
\put(17344,-5149){\line( 0,1){779}}
\put(19563,-4800){\makebox(0,0)[c]{180 days}}

\color{red}
\put(21782,-4370){\line( 0,1){779}}
\put(21782,-3591){\line( -1,0){964}}
\put(20818,-4370){\line( 0,1){779}}
\put(21300,-3950){\makebox(0,0)[c]{40d}}

\color{cyan}
\put(21782,-3591){\line( 0,1){779}}
\put(21782,-2812){\line( -1,0){120}}
\put(21661,-3591){\line( 0,1){779}}
\put(21300,-3201){\makebox(0,0)[c]{5d}}
\end{picture}%


\normalsize

\end{center}
\caption{Schema of different resolutions in a MTSDB}
\label{fig:exemple:window}
\end{figure}


In a multiresolution time series database (MTSDB) a time series is
stored in different resolution pieces, that is in different 'time
subseries'.  In this example we store the time series with high
resolution at recent times and with low resolution at older times. The
schema of this example is illustrated in figure
\ref{fig:exemple:window}. At the top there are four \emph{discs} with
different number of \emph{measures} and at the bottom there is a
timeline showing the time series chopped along time. For recent times,
every 5 hours a \emph{measure} is stored in the fourth \emph{disc}
which has a capacity of 24 \emph{measures} so that results in a 5 day
piece. For low mid times, every 2 days a \emph{measure} is stored in
the third \emph{disc} which has a capacity of 20 \emph{measures} so
that results in a 40 days piece. For high mid times, every 15 days a
\emph{measure} is stored in the second \emph{disc} which has a
capacity of 12 \emph{measures} so that results in a 180 days
piece. For old times, every 50 days a \emph{measure} is stored in the
first \emph{disc} which has a capacity of 12 \emph{measures} so that
results in a 600 days piece.

All the resolution discs are consolidated by average interpolation function\todo{fer referència allà on s'expliqui}. The resulting time series in the MTSDB are shown in figure~\ref{fig:exemple:4mrd}, where each graphic corresponds to a resolution disc's time series. Missing data and outlayers observations have been eliminated, the number of stored values has been reduced from \todo{quants} to 92 but now there i only high resolution for the recent data. Note that October 18th corresponds to the 'now ' point showed in figure \ref{fig:exemple:window}.


\begin{figure}[tp]
\centering
\includegraphics{\experiment/isense2.pdf}
\caption{Different resolutions for a time series in a MTSDB}
\label{fig:exemple:4mrd}
\end{figure}









%%% Local Variables:
%%% TeX-master: "main"
%%% ispell-local-dictionary: "british"
%%% End:

\usetikzlibrary{dateplot}  
%\usetikzlibrary{pgfplots.groupplots}

\pgfplotsset{
   petit/.style={
        width=\textwidth,
        height=3.5cm,
        legend style={font=\footnotesize},
        tick label style={font=\tiny},
        label style={font=\tiny},
        title style={font=\small,below, anchor=north,fill=white},
%        every axis title shift=0pt,
%        max space between ticks=15,
        every mark/.append style={mark size=6},
        major tick length=0.1cm,
        minor tick length=0.066cm,
        very thin,
    }
}

  \begin{tikzpicture}
    \begin{axis}[
        petit,
        title={RD: 5h $|24|$},
        ylabel=Temperature (K),
%
        date coordinates in=x,
        xticklabel={\pgfcalendarmonthshortname{\month} \day},
        ]
       \addplot[const plot mark right, blue, mark=*] table[col sep=comma] {dades/mrdb-matriu0/0.csv};
  \end{axis}
\end{tikzpicture}


  \begin{tikzpicture}
    \begin{axis}[
        petit,
        title={RD: 2d $|20|$},
        ylabel=Temperature (K),
%
        date coordinates in=x,
        xticklabel={\pgfcalendarmonthshortname{\month} \day},
        ]
       \addplot[const plot mark right, blue, mark=*] table[col sep=comma] {dades/mrdb-matriu0/1.csv};
  \end{axis}
\end{tikzpicture}


  \begin{tikzpicture}
    \begin{axis}[
        width=\textwidth,
        height=3.5cm,
        title={RD: 15d $|12|$},
        ylabel=Temperature (K),
%
        date coordinates in=x,
%        xticklabel={\pgfcalendar{tickcal}{\tick}{\tick}{\pgfcalendarshorthand{m}{.}}},
        xticklabel={\pgfcalendarmonthshortname{\month} \day},
%        xticklabel style= {rotate=15,anchor=east},
        title style = {below, anchor=north,fill=white},
%v>1.4        unbounded coords=jump,
        ]
       \addplot[const plot mark right, blue, mark=*] table[col sep=comma] {dades/mrdb-matriu0/2.csv};
  \end{axis}
\end{tikzpicture}
\begin{tikzpicture}
    \begin{axis}[
        width=\textwidth,
        height=3.5cm,
        date coordinates in=x,
%        xticklabel={\pgfcalendar{tickcal}{\tick}{\tick}{\pgfcalendarshorthand{m}{.}}},
        xticklabel={\pgfcalendarmonthshortname{\month} \year},
        xticklabel style= {rotate=15,anchor=east},
        title={RD: 50d $|12|$},
        title style = {below, anchor=north,fill=white},
        xlabel=Time (UTC),
        ylabel=Temperature (K),
%v>1.4        unbounded coords=jump,
        legend style = {anchor = north, draw = none},
        legend columns = 4,
        ymax = 320,
        clip=false,
%v1.6     restrict y to domain=0:320,
        y filter/.code = { \pgfmathparse{(#1>320)*330+(#1<320)*#1}},
        ]
       \addplot[const plot mark right, blue, mark=*] table[col sep=comma] {dades/mrdb-matriu0/4.csv};
       \addlegendentry{mean};

       \addplot[const plot mark right, orange, mark=*] table[col sep=comma] {dades/mrdb-matriu0/3.csv};
       \addlegendentry{max};

       \node at (axis cs:2011-10-12,330) {(2938)};
       \node (break) at (axis cs:2011-08-23,325)[inner sep=0pt,minimum width=0.75em, minimum height=0.5ex,fill=white] {};
    \draw [fill=red,color=orange] (break.north east) -- (break.north west) (break.south west) -- (break.south east);

  \end{axis}
\end{tikzpicture}
%http://tex.stackexchange.com/questions/46422/axis-break-in-pgfplots

%http://tex.stackexchange.com/questions/52409/insert-a-separate-mark-inside-a-pgfplots-graph



%%% Local Variables:
%%% TeX-master: "../main"
%%% ispell-local-dictionary: "british"
%%% End:

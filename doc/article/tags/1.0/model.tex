\section{Preliminaries}
\label{sec:model:preliminaries}

In this section we clarify the time series' definitions and
nomenclature which we use next when defining the MTSMS model.  The
main objects of a MTSMS are \emph{measures} and \emph{time series}. A
\emph{measure} is a value measured in an instant in time and a
\emph{time series} is a collection of \emph{measures}.

A \emph{measure} is a tuple $(v,t)$ where $v$ is the value of the
measure and $t \in \mathbb{R}$ is the instant in time of measurement.
The values of a time series can be of any type; for simplicity
examples are presented generally with integers or real numbers but can
also be strings or structures such as arrays.  Let $m = (v,t)$ be a
measure, $v$ is written as $V(m)$ and $t$ is written as $T(m)$.


The time value defines the order between measures.  Let $m = (v_m,
t_m)$ and $n = (v_n, t_n)$ be two measures, then $m\geq n$ if and only
if $t_m\geq t_n$.

A time series is sequence of measures that are ordered in time. 
\begin{definition}[Time series]
  A \emph{time series} $S$ is a a set of measures $S = \{m_0, \ldots,
  m_k\}$ without repeated time values $\forall i,j: i\leq k, j\leq k,
  i\neq j : T(m_i)\neq T(m_j)$, where $k+1=|S|$ is the cardinality of
  the set.
\end{definition}

The order defined by measures implies a total order in a time
series. As a time series is a finite set, if it is not empty it has a
maximum and a minimum.  Let $S=\{m_0,\ldots,m_k\}$ be a time series
and $n\in S$ be a measure. The time series' maximum is $n=\max(S)$ if
and only if $\forall m \in S: n \geq m $.  Similarly, the time series'
minimum is $n=\min(S)$ if and only if $\forall m \in S: n \leq m$.

Given the order defined by time, in a time series we define the
sequence interval similarly as it is done in
\cite{last:keogh,last:hetland}.  Let $S=\{m_0, \ldots, m_k\}$ be a
time series. We define the subset $S(r,t] \subseteq S$ as the time
series $S(r,t]=\{m\in S | r<T(m)\leq t\}$, where $r$ and $t$ are two
instants in time.  We also define the subset $S(r,+\infty)\subseteq S$
as the time series $S(r,+\infty) = \{m\in S | r< T(m) \leq
T(\max(S))\}$ and the subset $S(-\infty,t)\subseteq S$ as the time
series $S(-\infty,t) = \{m\in S | T(\min(S))\leq T(m) < t\}$.


The time order in time series also implies the sequence concept of
next and previous measure.  Let $S=\{m_0, \ldots, m_k\}$ be a time
series and $l\in S$ and $n$ be two measures. We define the next
measure of $n$ in $S$ as $l=\nex_S(n)$ where $l =
\min(S(T(n),+\infty))$. We define the previous measure of $n$ in $S$
as $l=\prev_S(n)$ where $l = \max(S(-\infty,T(n)))$.


%\subsubsection{Regularity of time series}

Let $S=\{m_0,\ldots,m_k\}$ be a time series, $t$ an time instant and
$\delta$ a time duration, the time series' measures can be located in
the time interval $i_0=[t,t+\delta]$ and its multiples $i_j=[t+j\delta
\,,\, t+(j+1)\delta]: j=0,1,2,\ldots$. In signal processing, these
time intervals are called sampling intervals, $\delta$ is called
sampling period and $t$ is called initial time. When the measures are
equally spaced the time series is called regular.

\begin{definition}[Regular time series]
  Let $S=\{m_0,\ldots,m_k\}$ be a time series and $\delta$ a time
  duration. $S$ is a \emph{regular time series} if and only if $\forall m \in
  S(T(\min(S),+\infty): T(m) - T(\prev_S(m)) =
  \delta$. %Property: for a regular time series there is a initial
           %time instant, let $t_0$ be a time instant then $T(\min(S))=t_0$.
\end{definition}



\section{The multiresolution TSMS data model}
\label{sec:MTSMS}

In this section we design a mathematical model for the multiresolution
time series database management systems (MTSMS). Some concepts come
from an abstraction of RRDtool operations \cite{rrdtool}.

A MTSMS manages time series. A time series is regarded as a
chronological data collection, so it needs an appropriate management
by the DBMS.  The MTSMS model is an storage solution for a time series
where, in short, the time series information is spread in different
time resolutions.  The objects of a MTSMS are \emph{measures} and
\emph{time series} as defined in section~\ref{sec:model:preliminaries}
and each database from a MTSMS contains only one time series.

\begin{figure}[tp]
\centering
%\setlength{\unitlength}{0.3mm}
%../../../imatges/model/locales/mtsdb-internal_architecture.tex
\begin{tikzpicture}
 \tikzset{
        myarrow/.style={->, >=latex',  thick},
      }
      

  \node[rectangle,draw,minimum height=6cm,minimum width=9cm] (m) {};
  \draw[shift=( m.south west)]   
  node[above right] {base de dades multiresolució};


  %discmig
  \node (m.center) (discr1) {...};

  %discr
  
  \node[ellipse,draw,minimum height=3.5cm,minimum width=2.5cm,alias=discr0] [left=of discr1] {};
  \node[above=0cm of discr0.north] {R0};
  \node[below=0cm of discr0] {disc resolució};

  \node[cylinder, draw, shape border rotate=90, aspect=0.25,alias=buffer0] [below=3mm of discr0.north] {buffer};
  \node[circle, draw,alias=disc0]  [above=3mm of discr0.south] {disc} ;
  \draw [->] (disc0.center)++(.4:.4cm) arc(0:180:.4cm);
  \draw[myarrow] (buffer0.bottom) -- (disc0.north);


  %discrd

  \node[ellipse,draw,minimum height=3.5cm,minimum width=2.5cm,alias=discrd] [right=of discr1] {};
  \node[above=0cm of discrd] {Rd};
  \node[below=0cm of discrd] {disc resolució};

  \node[cylinder, draw, shape border rotate=90, aspect=0.25,alias=bufferd] [below=3mm of discrd.north] {buffer};
  \node[circle, draw,alias=discd]  [above=3mm of discrd.south] {disc} ;
  \draw [->] (discd.center)++(.4:.4cm) arc(0:180:.4cm);
  \draw[myarrow] (bufferd.bottom) -- (discd.north);



  %mesura 
  \node[above=1cm of m.north] (m0) {};

  \draw[myarrow] (m0) -- (m.north) 
  node[right,midway] {mesura};

  \draw[myarrow] (m.north) -- (buffer0);
  \draw[myarrow] (m.north) -- (bufferd);
  \draw[myarrow] (m.north) -- (discr1);

\end{tikzpicture}
\caption{Architecture of MTSMS model}
\label{fig:model:mtsdb}
\end{figure}

The general schema of the MTSMS model can be seen in figure
\ref{fig:model:mtsdb}.  A \emph{multiresolution database} is a
collection of \emph{resolution discs}, which temporarily accumulate
the \emph{measures} in a \emph{buffer} where they are processed and
finally stored in a \emph{disc}. Mainly, the data process is intended
to change the time intervals between \emph{measures} in order to
compact the time series information. In this way, the time series gets
stored in different time resolutions, which are spread in the
\emph{discs}.

\emph{Discs} are size bounded so they only contain a fixed amount of
\emph{measures}. When a \emph{disc} gets full it discards a
\emph{measure}. In this way, the multiresolution database is bounded
in size and the time series gets stored in pieces, that is 'time
subseries'.




Regarding operations, MTSMS model needs operators to change the time
intervals between measures. Mainly, these operators are
\emph{attribute aggregate} functions and \emph{consolidation}
actions. In this section we design the basic MTSMS model, that is four
basic data model definitions -- \emph{buffer}, \emph{disc},
\emph{resolution disc}, and \emph{multiresolution database} -- and the
operations to create a \emph{multiresolution database}, to add
\emph{measures} and to \emph{consolidate} time series. \emph{Attribute
  aggregate} functions are required but not linked to the model,so we
define them apart in section~\ref{sec:model:interpolador}.






\subsection{Buffer}\label{sec:model:buffer}

A buffer is a container for a regular or a no-regular time series. The
buffer objective is to regularise the time series with predetermined $\delta$ 
and attribute function. We call consolidation to this action. In the context
of buffers, the sampling period is called consolidation step
and the function is called attribute aggregate function.

\begin{definition}[Buffer]
  A \emph{buffer} is defined as the tuple $(S,\tau,\delta,f)$ where
  $S$ is a time series, $\tau$ is the last consolidation time, $\delta$
  is the duration of the consolidation step and $f$ is an
  attribute aggregate function.
\end{definition}

An empty buffer, or the initial buffer, $B_{\emptyset} =
(\emptyset,t_0, \delta, f)$ is a buffer that has an empty time series,
an initial consolidation time $t_0$ and predetermined $\delta$ and $f$. From
an empty buffer all the consolidation time instants can be calculated
as $t_0+i\delta, i\in\mathbb{N}$.

We define the operator \emph{addBuffer} as the addition of a new
measure to buffer's time series, $\text{addBuffer}: B =
(S,\tau,\delta,f) \times m = (v,t) \mapsto B'$ where
$B'=(S',\tau,\delta,f)$ and $S' = S \cup \{m\} $.
\[
\text{addBuffer}: \text{Buffer} \times \text{Measure} \longrightarrow
\text{Buffer}
\]



We say the buffer is ready to consolidate when a time of a measure is
bigger than the buffer's next consolidation time.  Let
$B=(S,\tau,\delta,f)$ be a buffer and $m=\max(S)$ the maximum measure,
$B$ is ready to consolidate if and only if $T(m) \geq \tau+\delta$.


%\subsubsection{Consolidation}

Let $B=(S,\tau,\delta,f)$ be a buffer ready to consolidate, the
consolidation from $B$ in the time interval $i=[\tau,\tau+\delta]$
results in a measure $m'=(v,\tau+\delta)$ where $m'=f(S,i)$. Attribute
aggregate function $f$ is described next in
section~\ref{sec:model:interpolador}.  We define the operator
\emph{consolidateBuffer} that calculates the measure of consolidation
and reduces the consolidated part of the time series from the
buffer. In a simplified, the \emph{consolidateBuffer} is only applied
to the present consolidation interval, $\text{consolidateBuffer}:
B=(S,\tau,\delta,f) \mapsto B' \times m' $ where $ B'=
(S',\tau+\delta,\delta,f)$, $ S' = S$ and $m' =
f(S,[\tau,\tau+\delta])$. Note that the resulting buffer's time series
can be reduced, such as $ S' = S(\tau+\delta,\infty)$, when the
historic is no more needed.
  \[
  \text{consolidateBuffer}: \text{Buffer} \longrightarrow \text{Buffer}
  \times \text{Measure}
  \]








\subsection{Disc}\label{sec:model:disc}

A disc is a container with a finite capacity. Each time series stored
in a disc has its cardinal bounded. When the cardinal of the time
series is to overcome the limit, some measures need to be discarded.

\begin{definition}[Disc]
  A \emph{disc} is a tuple $(S,k)$ where $S$ is a time
  series and $k\in\mathbb{N}$ is the maximum allowed cardinal of $S$.
\end{definition}

An empty disc $D_{\emptyset} = (\emptyset,k)$ is a disc with an empty
time series and the $k$ maximum cardinal that $S$ is allowed to take.

The cardinal of the times series is kept under control by the add
operator.  The operator \emph{addDisc} is defined as the procedure to
add a measure to the time series stored on the disc. If the predefined
capacity is exceeded, the minimum measure of the time series is
discarded, $\text{addDisc}: D=(S,k) \times m \mapsto D'$ where $ D' =
(S',k)$, $ S' =
  \begin{cases}
      S\cup\{m\} &\text{if }  |S|<k\\
      (S-\{\min(S)\}) \cup \{m\} & \text{else }
    \end{cases}  
    $
  \[
  \text{addDisc}: \text{Disc} \times \text{Measure} \longrightarrow \text{Disc}
  \]



\subsection{Resolution disc}

A resolution disc is a disc which stores a regular time series. It is
composed of a buffer, with the time series to be regularised, and a
disc, with the time series regularised.

\begin{definition}[Resolution disc]
  A \emph{resolution disc} is a tuple $(B,D)$ where $B$
  is a buffer and $D$ is a disc.
\end{definition}
 
An empty buffer and empty disc imply an empty resolution disc
$R_{\emptyset} = (B_{\emptyset},D_{\emptyset})$.

The operators of a resolution disc are related to
the buffer and disc ones.

Operator \emph{addRD} is the addition of a measure to the
buffer of the resolution disc, $\text{addRD } : R=(B,D) \times m \mapsto R'$ where
$R'= (B',D)$ and $B'= B \text{ addBuffer } m$.
\[
\text{addRD}: \text{Resolution Disc} \times \text{Measure}
\longrightarrow \text{Resolution Disc}
\]

Let $R=(B,D)$ be a resolution disc, $R$ is ready to consolidate if and
only if $B$ is ready to consolidate.

If the resolution disc is ready to consolidate, it can be
consolidated.  Operator \emph{consolidateRD} calculates a
consolidation measure from the buffer and adds it to the disc,
$\text{consolidateRD } : R=(B,D) \mapsto R'$ where $R'=
(B',D')$, $B' \times m'= \text{ consolidateBuffer } B $ and $ D'= D
\text{ addDisc } m'$.
\[
\text{consolidateRD}: \text{Resolution Disc} \longrightarrow
\text{Resolution Disc}
\]




\subsection{Multiresolution Database}\label{sec:model:rrd}

A multiresolution database is a set of resolution discs which
share the input of measures, that is they store the same time
series. A time series is stored regularised and distributed with
different resolutions in the various resolution discs, as has been seen
in figure~\ref{fig:model:mtsdb}.

\begin{definition}[Multiresolution Database]
  A \emph{Multiresolution Database} is a set of resolution discs
  $M=\{R_0,\dotsc,R_d\}$.
\end{definition}

An empty multiresolution database has empty resolution discs $M_{\emptyset}=\{R_{0_{\emptyset}},\dotsc,R_{d_{\emptyset}\}}$. 
 
Generally, in a multiresolution database there are not two resolution discs
with the same information. That is, let $R_a = (B_a, D_a)$ and $R_b =
(B_b, D_b)$ be two resolution discs, its buffers
$B_a=(S_a,\tau_a,\delta_a,f_a)$ and $B_b=(S_b,\tau_b,\delta_b,f_b)$
have different consolidation interval and attribute aggregate function
$\delta_a \neq \delta_b \wedge f_a \neq f_b$.


With reference to the operators, the add and consolidate in a
multiresolution database are applied to every resolution disc it
contains.


Operator \emph{addMD} is defined as the addition of a measure to every
resolution disc, $\text{addMD } : M=\{R_0,\dotsc,R_d\} \times m
\mapsto M' $ where $M'= \{ \forall R_i\in M: R_i \text{
  addRD } m \}$.
\[
\text{addMD}: \text{Multiresolution Database} \times \text{Measure}
\longrightarrow \text{Multiresolution database}
\]


Operator \emph{consolidateMD} consolidates the resolution discs that
are ready to consolidate, $\text{consolidateMD } :
M=\{R_0,\dotsc,R_d\} \mapsto M'$ where $ M'= \big\{ \forall R_i\in M:$

  $\begin{cases}
    \text{ consolidateRD } R_i & \text{if } R_i \text{ ready to consolidate} \\
    R_i & \text{else }
  \end{cases}$ $\big\}
  $.
\[
\text{consolidateMD}: \text{Multiresolution Database} \longrightarrow
\text{Multiresolution Database}
  \]








\section{Attribute aggregate function}
\label{sec:model:interpolador}


An attribute aggregate function is used when a buffer is
consolidated in order to summarise information from the time series.
Let $S$ be a time series and $t_0$ and $t_f$ two time instants, an
attribute aggregate function $f$ calculates a measure that
summarises an attribute of $S$ in the time interval $i=[T_0,T_f]$:
\[
f: \text{Time series} \times \text{time interval} \longrightarrow
\text{Measure}
\]
\[
f: S=\{m_0,\ldots,m_k\} \times i=[T_0,T_f] \mapsto m'
\]

Different types of attribute aggregate functions may be utilised in
order to summarise a time series. As instance, we may calculate
attribute statistics from a time series such as the maximum value, the
average or apply digital signal processing operations as is done in
\cite{zhang11}.  Furthermore, attribute aggregate functions copes with
data validation, which we will discuss latter in this section.

Next, we exemplify some discrete operators. For simplicity, we present
examples with real numbers but they could also be integers, strings or
more elaborated structures such as arrays. Let $S'=S(T_0,T_f]$:
\begin{itemize}
\item maximum$^d$: $S \times i \mapsto m'$ where $V(m') = \max_{\forall m
    \in S'}(V(m))$. It summarises $S'$ with the maximum of
  the measure values.
\item last$^d$: $S \times i \mapsto m'$ where $V(m') = \max(S')$. It
  summarises $S'$ with the maximum measure.
\item arithmetic mean$^d$: $S \times i \mapsto m'$ where $V(m') =
  \frac{1}{|S'|} \sum\limits_{\forall m\in S'} V(m)$. It
  summarises $S'$ with the mean of the measure values.
\end{itemize}



With reference to data validation, attribute aggregate functions
can cope with this process. When data has not been captured or has
been captured erroneously, it must be treated as unknown data.
\begin{itemize}
\item When data has not been captured it is unknown by nature. For
  example, we try to capture data from a sensor and there is no
  response.
\item When data is erroneously it must be marked as unknown. For
  example, we capture data from a sensor but it responses in a not
  reasonable time or we capture data that is clearly outside a
  reasonable limits.
\end{itemize}
As a consequence, attribute aggregate functions deals with these two
subprocesses: treating unknown data and marking data as
unknown. Following with real numbers example, we extend the
domain with a value that means 'unknown', let this unknown value be
represented by the improper element infinity ($\infty$).

An attribute aggregate functions treating unknown
data is a one that can calculate a result when there are unknown
values in the original time series, $f^u: S \times i \mapsto m'$ where
$\exists m \in S: V(m)=\infty$. Although from a strict point of view
operating with unknown data makes unknown result, aggregate functions
are free to calculate whatever is needed such as time series analysis
does with data reconstruction.

For example, arithmetic mean$^{d}$ aggregate function returns
$V(m')=\infty$ if $\exists m \in S: V(m)=\infty$.  We can define a new
mean function, based on the original arithmetic mean$^{d}$ aggregate,
that naively treats unknown values by keeping the
known mean; in other words, it ignores unknown values found in the time
interval: arithmetic mean$^{du}$: $S \times i \mapsto m'$ where $m' =
\text{arithmetic mean}^{d}(S'',i)$ and $S''= \{m''\in S':V(m'')\neq
\infty\}$.
% ignore$^{u}$: $S \mapsto S'$ where $S'= \{m''\in S':V(m'')\neq
% \infty\}$,
% arithmetic mean$^{du}$: $S \times i \mapsto m'$ where $m' =
% \text{arithmetic mean}^{d}(\text{ignore}^u(S),i)$.

An attribute aggregate functions marking data as unknown is a one
that can give unknown value as the resulting measure's value, $f^{mu}:
S \times i \mapsto m'$ where $V(m')\in \mathbb{R}\cup\{\infty\}$.

For example, we can define a maximum aggregate, based on the
maximum$^d$ aggregate, that returns unknown if there is a
measure's value bigger than 2:  maximum$^{dmu2}$: $S \times i
\mapsto m'$ where $V(m') = 
\begin{cases}
  \infty &\text{if }  m''>2\\
  m'' & \text{else }
\end{cases}$ and $m''=\text{maximum}^d(S,i)$.

%Per exemple definim un termini, si les dades estan més espaiades que 2 es marca com a desconeguda

Summarising, in the design of the attribute aggregate function we can
interpret a time series in different ways, that is what we call the
representation of a time series. Keogh et al.\ \cite{last:keogh} cite
some possible representations for time series such as \emph{Fourier
  Transforms}, \emph{Wavelets}, \emph{Symbolic Mappings} or
\emph{Piecewise Linear Representation} (PLR). This last is remarked as
the most used owing to the most common representation is with linear
functions \cite{keogh01}.

The variety of time series representations results in a variety of the
same attribute aggregate functions. As instance, a maximum attribute
aggregate function may give different values if we consider a linear
or a constant piecewise representation. 

In conclusion, a huge amount of attribute aggregate functions can be
defined and no global assumptions can be made.  Therefore, the time
series representation results in attribute aggregate functions
families.  MTSMS must give freedom to the users to define their own
functions.



%%% Local Variables:
%%% TeX-master: "main"
%%% ispell-local-dictionary: "british"
%%% End:
% LocalWords: buffer buffers  MTSMS multiresolution DBMS zohe






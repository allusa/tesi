\section{The multiresolution TSMS model}
\label{sec:MTSMS}

In this section we design a mathematical model for the multiresolution time series database management systems (MTSMS). Some concepts come from an abstraction of RRDtool operations \cite{rrdtool}. 

A MTSMS manages time series. A time series is regarded as a chronological data collection, so it needs an appropriate management by the DBMS.
The MTSMS model is an storage solution for a time series where, in short,  the time series information is spread in different time resolutions. 

The main objects of a MTSMS are \emph{measures} and \emph{time series}. A \emph{measure} is a value measured in an instant in time and a  \emph{time series} is a collection of \emph{measures}. A database from a MTSMS contains one time series, which internally is stored in multiple resolutions of itself.

\begin{figure}[tp]
\centering
\setlength{\unitlength}{0.3mm}
../../../imatges/model/locales/mtsdb-internal_architecture.tex
\caption{Architecture of MTSMS model}
\label{fig:model:mtsdb}
\end{figure}

The general schema of the MTSMS model can be seen in figure \ref{fig:model:mtsdb}.  A \emph{multiresolution database} is a collection of \emph{resolution discs}, which temporarily accumulate the \emph{measures} in a \emph{buffer} where they are processed and finally stored in a \emph{disc}. Mainly, the data process is intended to change the time intervals between \emph{measures} in order to compact the time series information. In this way, the time series gets stored in different time resolutions, which are spread in the \emph{discs}.

\emph{Discs} are size bounded so they only contain a fixed amount of \emph{measures}. When a \emph{disc} gets full it discards a \emph{measure}. In this way, the multiresolution database is bounded in size and the time series gets stored in pieces, that is 'time subseries'.







Regarding operations, MTSMS model needs operators to change the time intervals between measures. Mainly, these operators are \emph{interpolation} functions and \emph{consolidation} actions. In this section we design the basic MTSMS model, that is six basic data model definitions -- \emph{measure}, \emph{time series}, \emph{buffer}, \emph{disc}, \emph{resolution disc}, and \emph{multiresolution database} -- and the operations to create a \emph{multiresolution database}, to add \emph{measures} and to \emph{interpolate} and \emph{consolidate} time series.







\subsection{Measure and Time Series}

A \emph{measure} is a tuple $(v,t)$ where $v\in{\mathbb{R}}$ is the
value of the measure and $t \in \mathbb{R}$ is the instant in time of
measurement. Let $m = (v,t)$ be a measure, $v$ is written as $V(m)$
and $t$ is written as $T(m)$.

The time value defines the order between measures.  Let $m = (v_m,
t_m)$ and $n = (v_n, t_n)$ be two measures, then $m\geq n$ if and only
if $t_m\geq t_n$.


A time series is sequence of measures that are ordered in time. 
\begin{definition}[Time series]
  A \emph{time series} $S$ is a a set of measures $S = \{m_0, \ldots,
  m_k\}$ without repeated time values $\forall i,j: i\leq k, j\leq k,
  i\neq j : T(m_i)\neq T(m_j)$, where $k+1=|S|$ is the cardinality of
  the set.
\end{definition}

The order defined by measures implies a total order in a time
series. As a time series is a finite set, if it is not empty it has a
maximum and a minimum.  Let $S=\{m_0,\ldots,m_k\}$ be a time series
and $n\in S$ be a measure. The time series' maximum is $n=\max(S)$ if
and only if $\forall m \in S: n \geq m $.  Similarly, the time series'
minimum is $n=\min(S)$ if and only if $\forall m \in S: n \leq m$.

Given the order defined by time, in a time series we define the
sequence interval similarly as it is done in
\cite{last:keogh,last:hetland}.  Let $S=\{m_0, \ldots, m_k\}$ be a
time series. We define the subset $S(r,t] \subseteq S$ as the time
series $S(r,t]=\{m\in S | r<T(m)\leq t\}$, where $r$ and $t$ are two
instants in time.  We also define the subset $S(r,+\infty)\subseteq S$
as the time series $S(r,+\infty) = \{m\in S | r< T(m) \leq
T(\max(S))\}$ and the subset $S(-\infty,t)\subseteq S$ as the time
series $S(-\infty,t) = \{m\in S | T(\min(S))\leq T(m) < t\}$.


The time order in time series also implies the sequence concept of
next and previous measure.  Let $S=\{m_0, \ldots, m_k\}$ be a time
series and $l\in S$ and $n$ be two measures. We define the next
measure of $n$ in $S$ as $l=\nex_S(n)$ where $l =
\min(S(T(n),+\infty))$. We define the previous measure of $n$ in $S$
as $l=\prev_S(n)$ where $l = \max(S(-\infty,T(n)))$.


%\subsubsection{Regularity of time series}

Let $S=\{m_0,\ldots,m_k\}$ be a time series, $t$ an time instant and
$\delta$ a time duration, the time series' measures can be located in
the time interval $i_0=[t,t+\delta]$ and its multiples $i_j=[t+j\delta
\,,\, t+(j+1)\delta]: j=0,1,2,\ldots$. In signal processing, these
time intervals are called sampling intervals, $\delta$ is called
sampling period and $t$ is called initial time. When the measures are
equally spaced the time series is called regular.

\begin{definition}[Regular time series]
  Let $S=\{m_0,\ldots,m_k\}$ be a time series, $t$ a time instant
  and $\delta$ a time duration. $S$ is regular if and
  only if $\forall m \in S(T(\min(S),+\infty):T(m) - T(\prev_S(m)) =
  \delta$ and $T(\min(S))=t$.
\end{definition}




\subsection{Buffer}\label{sec:model:buffer}

A buffer is a container for a regular or a no-regular time series. The
buffer objective is to regularise the time series with predetermined $\delta$ 
and attribute function. We call consolidation to this action. In the context
of buffers, the sampling period is called consolidation step
and the function is called attribute interpolation function.

\begin{definition}[Buffer]
  A \emph{buffer} is defined as the tuple $(S,\tau,\delta,f)$ where
  $S$ is a time series, $\tau$ is the last consolidation time, $\delta$
  is the duration of the consolidation step and $f$ is an
  attribute interpolation function.
\end{definition}

An empty buffer, or the initial buffer, $B_{\emptyset} =
(\emptyset,t_0, \delta, f)$ is a buffer that has an empty time
series, an initial consolidation time and predetermined $\delta$ and
$f$. From an empty buffer all the consolidation time instants
can be calculated as $t_0+k\delta, k\in\mathbb{N}$.

We define the operator \emph{addBuffer} as the addition of a new measure to
buffer's time series, $\text{addBuffer}: B = (S,\tau,\delta,f) \times m =
(v,t) \mapsto B'$ where $B'=(S',\tau,\delta,f)$ and $S' = S \cup \{m\} $.
\[
\text{addBuffer}: \text{Buffer} \times \text{Measure} \longrightarrow \text{Buffer}
\]



We say the buffer is ready to consolidate when a time of a measure is
bigger than the buffer's next consolidation time.  Let
$B=(S,\tau,\delta,f)$ be a buffer and $m=\max(S)$ the maximum measure,
$B$ is ready to consolidate if and only if $T(m) \geq \tau+\delta$.


%\subsubsection{Consolidation}

Let $B=(S,\tau,\delta,f)$ be a buffer ready to consolidate, the
consolidation from $B$ in the time interval $i=[\tau,\tau+\delta]$
results in a measure $m'=(v,\tau+\delta)$ where $m'=f(S,i)$. Attribute
interpolation function $f$ is described next in
section~\ref{sec:model:interpolador}.  We define the operator
\emph{consolidateBuffer} that calculates the measure of consolidation and
reduces the consolidated part of the time series from the buffer. In a
simplified, the \emph{consolidateBuffer} is only applied to the present
consolidation interval, $\text{consolidateBuffer}: B=(S,\tau,\delta,f)
\mapsto B' \times m' $ where $ B'= (S',\tau+\delta,\delta,f)$, $ S' =
S(\tau+\delta,\infty)$\todo{cal reduir la sèrie o en el model es pot conservar?} and $m' = f(S,[\tau,\tau+\delta])$.
  \[
  \text{consolidateBuffer}: \text{Buffer} \longrightarrow \text{Buffer}
  \times \text{Measure}
  \]








\subsection{Disc}\label{sec:model:disc}

A disc is a container with a finite capacity. Each time series stored
in a disc has its cardinal bounded. When the cardinal of the time
series is to overcome the limit, some measures need to be discarded.

\begin{definition}[Disc]
  A \emph{disc} is a tuple $(S,k)$ where $S$ is a time
  series and $k\in\mathbb{N}$ is the maximum allowed cardinal of $S$.
\end{definition}

An empty disc $D_{\emptyset} = (\emptyset,k)$ is a disc with an empty
time series and the $k$ maximum cardinal that $S$ is allowed to take.

The cardinal of the times series is kept under control by the add
operator.  The operator \emph{addDisc} is defined as the procedure to
add a measure to the time series stored on the disc. If the predefined
capacity is exceeded, the minimum measure of the time series is
discarded, $\text{addDisc}: D=(S,k) \times m \mapsto D'$ where $ D' =
(S',k)$, $ S' =
  \begin{cases}
      S\cup\{m\} &\text{if }  |S|<k\\
      (S-\{\min(S)\}) \cup \{m\} & \text{else }
    \end{cases}  
    $
  \[
  \text{addDisc}: \text{Disc} \times \text{Measure} \longrightarrow \text{Disc}
  \]



\subsection{Resolution disc}

A resolution disc is a disc which stores a regular time series. It is
composed of a disc and a buffer with the time series to be
regularised.

\begin{definition}[Resolution disc]
  A \emph{resolution disc} is a tuple $(B,D)$ where $B$
  is a buffer and $D$ is a disc.
\end{definition}
 
On the one hand, the definitions of empty buffer and empty disc imply
an empty resolution disc $R_{\emptyset} = (B_{\emptyset},D_{\emptyset})$.

On the other hand, the operators of a resolution disc are related to
the buffer and disc ones.

Operator \emph{addResolutionDisc} is the addition of a measure to the
buffer of the resolution disc, $\text{addResolutionDisc } : R=(B,D) \times m \mapsto R'$ where
$R'= (B',D)$ and $B'= B \text{ addBuffer } m$.
\[
\text{addResolutionDisc}: \text{Resolution Disc} \times \text{Measure}
\longrightarrow \text{Resolution Disc}
\]

Let $R=(B,D)$ be a resolution disc, $R$ is ready to consolidate if and
only if $B$ is ready to consolidate.

If the resolution disc is ready to consolidate, it can be
consolidated.  Operator \emph{consolidateResolutionDisc} calculates a
consolidation measure from the buffer and adds it to the disc,
$\text{consolidateResolutionDisc } : R=(B,D) \mapsto R'$ where $R'=
(B',D')$, $B' \times m'= \text{ consolidateBuffer } B $ and $ D'= D
\text{ addDisc } m'$.
\[
\text{consolidate}: \text{Resolution Disc} \longrightarrow
\text{Resolution Disc}
\]




\subsection{Multiresolution Database}\label{sec:model:rrd}

A multiresolution database is an archive of resolution discs which
share the input of measures, that is they store the same time
series. A time series is stored regularised and distributed with
different resolutions in the various resolution discs, as has been seen
in figure~\ref{fig:model:mtsdb}.

\begin{definition}[Multiresolution Database]
  A \emph{Multiresolution Database} is a set of resolution discs
  $M=\{R_0,\dotsc,R_d\}$.
\end{definition}

An empty multiresolution database has empty resolution discs $M_{\emptyset}=\{R_{0_{\emptyset}},\dotsc,R_{d_{\emptyset}\}}$. 
 
Generally, in a multiresolution database there are not two resolution discs
with the same information. That is, let $R_a = (B_a, D_a)$ and $R_b =
(B_b, D_b)$ be two resolution discs, its buffers
$B_a=(S_a,\tau_a,\delta_a,f_a)$ and $B_b=(S_b,\tau_b,\delta_b,f_b)$
have different consolidation interval and attribute interpolation function
$\delta_a \neq \delta_b \wedge f_a \neq f_b$.


With reference to the operators, the add and consolidate in a
multiresolution database are applied to every resolution disc it
contains.


Operator \emph{addMD} is defined as the addition of a measure to every
resolution disc, $\text{addMD } : M=\{R_0,\dotsc,R_d\} \times m
\mapsto M' $ where $M'= \{ \forall R_i\in M: R_i \text{
  addResolutionDisc } m \}$.
\[
\text{addMD}: \text{Multiresolution Database} \times \text{Measure}
\longrightarrow \text{Multiresolution database}
\]


Operator \emph{consolidateMD} consolidates the resolution discs that
are ready to consolidate, $\text{consolidateMD } :
M=\{R_0,\dotsc,R_d\} \mapsto M'$ where $ M'= \big\{ \forall R_i\in M:
  \begin{cases}
    \text{ consolidateResolutionDisc } R_i & \text{if } R_i \text{ ready to consolidate} \\
    R_i & \text{else }
  \end{cases}\big\}
  $.
\[
\text{consolidate}: \text{Multiresolution Database} \longrightarrow
\text{Multiresolution Database}
  \]








\section{Attribute interpolation function}\todo{potser dir-li aggregate? en català com en direm?}
\label{sec:model:interpolador}
\todo{Teresa: cal repensar si aquest apartat es fa més global. Proposa que les aif sigui una secció nova, la 4, si tenim més coses a dir}


{\color{red}

  An attribute interpolation function is used when a buffer is
  consolidated in order to summarise information from the time series.
  Let $S$ be a time series and $t_0$ and $t_f$ two time instants, an
  attribute interpolation function $f$ calculates a measure that
  summarises an attribute of $S$ in the time interval $i=[T_0,T_f]$:
\[
f: \text{Time series} \times \text{time interval} \longrightarrow
\text{Measure}
\]


There can be different attribute interpolation functions in order to summarise a
time series. As instance, we may want to calculate statistics from a time series such as the maximum value, the average or apply digital signal processing operations as is done in \cite{zhang11}. 


We can globally define the behaviour of attribute interpolation functions in the discrete (set) and continuous forms, except consolidation time which is subject to interpretation.

Purely discrete interpolation with typical set operators. Let $S'=S(T_0,T_f]$\todo{?}:

\begin{itemize}
\item maximum:  $S \times i \mapsto m'$ where $V(m') = \max_{\forall m \in S'}(V(m))$ and $T(m')$ depends.
\item arithmetic mean: $S \times i \mapsto m'$ where $V(m') = \frac{1}{|S'|} \sum\limits_{\forall m\in S'} V(m)$ and $T(m')$ depends.
\end{itemize}


Continuous interpolation, we need continuous function $S(t)$. Let $t\in\mathbb{R}$ :

\begin{itemize}

\item maximum: $S \times i \mapsto m'$ where $V(m') = \max_{\forall t \in [T_0,T_f]}(S(t))$ and $T(m')$ depends. 

\item last: $S \times i \mapsto m'$ where $V(m') = S(T_f)$ and $T(m')$ depends. 

\item average function

If $f$ is continuous on a closed interval $[a,b]$, then there is at least one number $x^*$ in $[a,b]$ such that
$$
\int_a^b f(x)dx = f(x^*)(b-a)
$$

The average value of the function ($\bar f$)  on this interval is then given by  $f(x^*)$.

Weisstein, Eric W. "Average Function." From MathWorld--A Wolfram Web Resource. http://mathworld.wolfram.com/AverageFunction.html


\end{itemize}


In the design of the attribute interpolation function we can interpret
a time series in different ways, that is what we call the
representation of a time series. Keogh et al.\ \cite{last:keogh} cite some
possible representations for time series such as \emph{Fourier
  Transforms}, \emph{Wavelets}, \emph{Symbolic Mappings} or
\emph{Piecewise Linear Representation} (PLR), remarking this last as
the most used. The most common representation is with linear functions
\cite{keogh01}.  

The variety of time series representations results in a variety of the
same attribute interpolation functions. As instance, a maximum
attribute interpolation function may give different values if we
consider a linear or a constant piecewise representation. Therefore,
the time series representation results in attribute interpolation
functions families.


\todo{dos exemples}
Next, we show two possible families for attribute interpolation
functions: one for time series represented by a purely discrete
function and another by a staircase function.


}


% Attribute interpolation functions for time series represented by a
% purely discrete function are similar to set operators as the discrete representation is  

% $$
% \forall t \in \mathbb{R}  ,\forall m \in S:
% S(t)^{\text{discrete}} =  
% \begin{cases}
%   V(m) & \text{if }  t=T(m) \\
%   \text{not defined} & \text{else} 
% \end{cases}
% $$


% \todo{família discreta: existeix $S(t)$?}
% no pot existir, podem fer com una delta comb (The Dirac delta function, or $\delta$ function, is (informally) a generalized function on the real number line that is zero everywhere except at zero) on els nd siguin zero això va bé possiblement per avg, max i area (és zero) però això ens porta problemes amb min i altres

% no pot existir, fem els nd com $\infty$: s'aplica le màtematiques de Projectively Extended Real Numbers on $\infty$ no té signe ni ordre. Per tant max i min podríen funcionar, àrea seria $\infty$ i avg?


% compte amb l'interpolador àrea: From a purely mathematical viewpoint, any extended-real function that is equal to zero everywhere but a single point must have total integral zero but the Dirac delta is not strictly a function: with an integral of one over the entire real line. Per tant, compte a no definir la família discreta amb delta de Dirac.










Next, we show some attribute interpolation functions for time series
represented by a staircase function, that is with a piecewise constant
representation.  We define a new representation for time series called
\emph{zero-order hold backwards} (zohe%from \emph{zero-order hold everted}
) consisting in holding each value until the preceding
value, which a similar representation is used by RRDtool
\cite{lisa98:oetiker}.

Let $S=\{m_0,\ldots,m_k\}$ be a time series,
we define $S(t)^{\text{zohe}}$ as its \emph{zero-order hold backwards} continuous representation along time $t$:
%continuous definition of the time series using left-continuous step functions.
$$
\forall t \in \mathbb{R}  ,\forall m \in S:
S(t)^{\text{zohe}} =  
\begin{cases}
  \text{not defined} & \text{if } t > T(\max S) \\
  V(m) & \text{if }  t\in (T(\prev_S m),T(m)]
\end{cases}
$$\todo{nd o $\infty$?}


We now define the \emph{zero-order hold backwards} attribute
interpolation function family as the one interpreting the
consolidation time interval left-continuous $i=(T_0,T_f]$ and the
resulting interpolated measure's time always being $T_f$, in
accordance to the \emph{zero-order hold backwards} representation
being defined using left-continuous step functions.  Let
$S=\{m_0,\ldots,m_k\}$ be a time series and $i=[T_0,T_f]$ be a time
interval, the attribute interpolation function $f^{\text{zohes}}\in f$
summarises $S$ with a measure that is calculated from the measure
values belonging to the subset $S(T_0,T_f]$, $f^{zohes}:
S=\{m_0,\ldots,m_k\} \times i=[T_0,T_f] \mapsto m'$ where
$m'=(v',T_f)$ and the resulting value $v'$ depends on the attribute
interpolation function. Let $S'=S(T_0,T_f]$: \todo{realment $S'$ hauria de ser $S(T_0,T_f] + \nex_S(\infty,T_f)$? és a dir usar selecció temporal $S[t_0,t_f]^{zohe}$, sinó usant només l'interval sobre la seqüència es fa interpolació sobre el conjunt discret. Potser fer l'exemple amb la família d'interpoladors discrets i amb la d'inteporladors zohe?} 




\begin{itemize}

\item \emph{Arithmetic mean interpolation} function summarises $S'$
  with the mean of the measure values.
  $
  v' = \frac{1}{|S'|} \sum\limits_{\forall m\in S'} V(m)
  $

\item \emph{Maximum interpolation} function summarises $S'$ with the
  maximum of the measure values.
  $
  v' = \max_{\forall m \in S'}(V(m))
  $
 %Note: \emph{Minimum interpolation} can be defined dually.
\item \emph{Last interpolation} function summarises $S'$ with the
  maximum measure.
  $
  v' = \max(S')
  $

\item \emph{Area interpolation} function summarises $S'$ keeping the
  area of the region. Considering the continuous approach of $S$ we
  can see the area interpolation function as $v' =
  \frac{\int_{T_0}^{T_f} S(t) dt}{T_f - T_0}$. \todo{potser en les
    altres també s'ha de definir de forma contínua? La forma contínua
    és independent de la representació?, aleshores potser posar-ho
    abans com a exemples generals d'interpoladors. Necessitem les fórmules a trossos?: sí perquè la fórmula contínua necessita mètodes númerics per calcular-se?} 

In the piecewise  approach it becomes:
  \[
  o=\min(S'),
  S''= S' - \{o\},
  n=\nex_{S} \max(S''): 
  \]\todo{S' definit general}
  \[
  \begin{split}
  :v'  = & \frac{1}{T_f-T_0} 
  \big[ (T(o)-T_0)V(o) +( T_f- T(\prev_{S} n) )V(n) \\
    & {}+\sum\limits_{\forall m \in S''}( T(m)- T(\prev_S m) )V(m) \big]   
   \end{split}
  \]\todo{repassar a model que sigui correcte}
  

\end{itemize}










%%% Local Variables:
%%% TeX-master: "main"
%%% ispell-local-dictionary: "british"
%%% End:
% LocalWords: buffer buffers  MTSMS multiresolution DBMS






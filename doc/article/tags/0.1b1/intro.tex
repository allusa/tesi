\section{Introduction}

Modern society depends on the expert operation of many complex
engineering systems which provides products and services. Examples
include electric power systems, water distributions networks,
transportation systems, manufacturing processes, intelligent
buildings, communication systems, etc. The emergence of embedded
systems and sensor networks has made possible the collection of large
amounts of data for monitoring and control of such complex systems. In
most application, these data need to be processed and synthesised
efficiently to provide relevant information to engineers, researchers,
accident investigators, operators, and many other users.

Nevertheless, before using the collected signals, it is of primary
importance to promptly detecting eventual sensor failures or
malfunctions and possibly reconstructing the incorrect signals in
order to avoid processing misleading information which may lead to
unsafe or inefficient actions.  In all these instances, acquired data
is associated with a time stamp, which implies that the correctness of
those data depends not only on the measured value but also on the time
as it is collected. When observations are collected at specific time
interval, large data sets in the form of time series are
generated~\cite{basu07:_autom}.  

Time series are defined as a collection of observations made
chronologically \cite{fu11}, accordingly they are also called time
sequences \cite{last:hetland}.  Time series are usually stored in a
database. Usually the managing software to store this data are
relational database management systems (RDBMS). However, using a RDBMS
as a time series backend suffers some drawbacks
\cite{dreyer94,schmidt95,stonebraker09:scidb,zhang11}. Time series
come from a continuous nature in which they are recorded at regular
intervals, such as hourly or daily, or at irregular intervals, such as
recording when a pump is open or closed.

One problem when dealing with time series data results from the fact
that these data are often voluminous \cite{fu11}. As a result, storing
and accessing them can be complicated. Moreover, it is specially
critical when developing small embedded systems, whose resources
(capacity, energy, processing, and communications) suffer a genuine
restriction \cite{yaogehrke02}.  Another problem is that the procedure
of processing and synthesising information becomes complicated if data
is not equi-time spaced.



%TSMS

This paper focuses on Data Base Management Systems (DBMS) that store
and treat data as time series. These are usually known as Time Series
Data Base Management Systems (TSMS) \cite{dreyer94}.  We introduce a
new data model for a multiresolution TSMS (MTSMS). This model allows
to store time series using different time resolutions and organised in
an aggregated way. The model is specifically designed to cope well
with bounded storage computers like those found in sensor systems.


This paper is organised as follows. Continuing this introduction, we
summarise the TSMS and the MTSMS features and we show some related
work concerning these systems.  Next, we formalise the MTSMS model: in
section~\ref{sec:model:preliminaries} we introduce nomenclature
preliminaries, mainly about measures and time series; in
section~\ref{sec:MTSMS} we define the data structure, the main part of
the model; and in section~\ref{sec:model:interpolador} we extend for
the summarising operators, which we call attribute aggregate
functions. In section~\ref{sec:example} we show a multiresolution
database example for real data. Finally in
section~\ref{sec:concl-future-work} we summarise our MTSMS model
proposal and we consider some future working directions.



\subsection{TSMS features}

A TSMS is a special purpose DBMS aimed at storing and managing time
series. The main objective of TSMS is to put together two areas of study: time
series analysis and DBMS.  Time series analysis theory studies
formally a great amount of algorithms and methodologies that apply to
time series, focusing on efficiency improving. DBMS theory studies
systems that store and operate with data; currently the relation model
\cite{date:introduction} is the referent.

In time series analysis there are some common operations that can be
generalised when treating time series.

The main attribute of time series is the time, therefore dealing with
time is a common operation, such as querying time intervals, finding
time correlations, or calculating distances between two time
series. TSMS must respect the temporal coherence of the time series.
In the context of statistics, aggregation of time series is also
common operation. Aggregation consists in summarising a time series
subset with an attribute such as the mean, the maximum, or the mode.

A particular feature of time series is representation. A time series
is discrete in the set sense, that is a set of value and time
pairs. Representation is the function model approximating the time
series to its continuous nature. TSMS operate on time series
respecting the representation coherence. Furthermore, the values of a
time series can be of any type.
% , preferably using piecewise operations in the set domain rather
% than solving numerical methods for the continuous domain.





%MTSMS
\subsection{Multiresolution features}

A MTSMS is a TSMS that has multiresolution capabilities.
In a MTSMS a schema has to be configured, which mainly consist in
defining different pairs of time resolutions and attribute summarising
functions. We show it with an example.

Figure~\ref{fig:mtsms:sequence} is a diagram showing a multiresolution
summary. It is for a regular time series, in
figure~\ref{fig:mtsms:sequence-irregular} there is an example for some
irregular cases. Both figures show a snapshot in time, suppose between
time 9 and 10.  

\begin{figure}[tp]
  \centering
  %\usetikzlibrary{positioning}
\begin{tikzpicture}[scale=0.77, every node/.style={transform shape}]

  %referencia
  \node (-6) {};

  \foreach \x in {-5,...,12}
  {
    \pgfkeys{/pgf/number format/.cd,int trunc}
    \pgfmathparse{abs(\x)}
    \let\absx=\pgfmathresult
    \pgfmathparse{\x-1}
    \let\antx=\pgfmathresult
    %time
    \node[node distance=1mm] (\x) [right=of \antx] 
    {\ifnum\x<11 \x \else \phantom{9} \fi};

    %graph values
    \node [above=\absx mm of \x] 
    {\ifnum\x=10 \color{gray} \fi \ifnum\x<11 $\bullet$ \fi};    

    %values
    % \node[rectangle,draw] (s\x) [below=of \x] 
    % {\ifnum\x<10 \pgfmathprintnumber{\absx} \else \phantom{9} \fi};
    \ifnum\x<10
    \node[rectangle,draw] (s\x) [below=of \x] 
    {\pgfmathprintnumber{\absx}};
    \else
    \node[rectangle,dotted,draw] (s\x) [below=of \x] 
    {\phantom{9}};
    \fi
  }

  \node [below=of 10] {\color{gray}10}; 
  

  
  %rd: 5s |inf| mean
  \node [circle,draw] (rd5-5) [below=3cm of s-5] {u};
  \node [circle,draw] (rd50) [below=3cm of s0] {u};
  \node [circle,draw] (rd55) [below=3cm of s5] {3};
  \node [circle,dotted,draw] (rd510) [below=3cm of s10] {\color{gray}u};
  \node [below=3.3cm of s10] {\color{gray}8};
 
  \draw[->,bend right] (s5) to (rd55);
  \draw[->,bend right] (s4) to (rd55);
  \draw[->,bend right] (s3) to (rd55);
  \draw[->,bend right] (s2) to (rd55);
  \draw[->,bend right] (s1) to (rd55);

  \draw[->,dotted,bend right] (s10) to (rd510);
  \draw[->,bend right] (s9) to (rd510);
  \draw[->,bend right] (s8) to (rd510);
  \draw[->,bend right] (s7) to (rd510);
  \draw[->,bend right] (s6) to (rd510);

  
  %rd: 3s |inf| mean
  \node [circle,draw] (rd3-3) [below=of s-3] {u};
  \node [circle,draw] (rd30) [below=of s0] {u};
  \node [circle,draw,fill=white] (rd33) [below=of s3] {2};
  \node [circle,draw,fill=white] (rd36) [below=of s6] {5};
  \node [circle,draw,fill=white] (rd39) [below=of s9] {8};
  \node [circle,dotted,draw] (rd312) [below=of s12] {\color{gray}u};

  \draw[->] (s3) to (rd33);
  \draw[->] (s2) to (rd33);
  \draw[->] (s1) to (rd33);

  \draw[->] (s6) to (rd36);
  \draw[->] (s5) to (rd36);
  \draw[->] (s4) to (rd36);

  \draw[->] (s9) to (rd39);
  \draw[->] (s8) to (rd39);
  \draw[->] (s7) to (rd39);

  \draw[->,dotted] (s12) to (rd312);
  \draw[->,dotted] (s11) to (rd312);
  \draw[->,dotted] (s10) to (rd312);



  %eixos
  \node (et0) [above=1mm of -5] {};
  \node (et12) [above=1mm of 11] {};
  \node [right=-2mm of et12] {time};
  \draw[->] (et0) to (et12);
  \node (y5) [above=5mm of 0] {--};
  \node [left=-1.5mm of y5] {5};
  \node (y10) [above=10mm of 0] {--};
  \node [left=-1.5mm of y10] {10};

  \node (inici) [above=4cm of s0] {init};
  \node (inici2) [below=4cm of s0] {};
  \draw[-,dotted] (inici) to (inici2);

  \node (fi) [above=4.4cm of s9.east] {now};
  \node (fi2) [below=4.4cm of s9.east] {};
  \draw[-,dotted] (fi) to (fi2);


  \node (fut) [below right=1mm and 1mm of fi] {future};
  \draw[->] (fut.south west) to (fut.south east);

  \node (pas) [below left=1mm and 1mm of fi] {past};
  \draw[->] (pas.south east) to (pas.south west);

  \node (unk) [below left=1mm and 1mm of inici] {unknown};
  \draw[->] (unk.south east) to (unk.south west);



  \node [above=0cm of s-5] {\makebox[0cm][l]{sample every 1 u.t.}};
  \node [below=0.5cm of s-5] {\makebox[0cm][l]{mean every 3 u.t.}};
  \node [below=2.5cm of s-5] {\makebox[0cm][l]{mean every 5 u.t.}};


\end{tikzpicture}



%%% Local Variables:
%%% TeX-master: "../main"
%%% ispell-local-dictionary: "british"
%%% End:

  \caption{Multiresolution diagram snapshot with regular sampling}
  \label{fig:mtsms:sequence}
\end{figure}


\begin{figure}[tp]
  \centering
  %\usetikzlibrary{positioning}
\begin{tikzpicture}

  \node[node distance=1mm] (0) {0};
  \node[node distance=1mm] (-1) [left=of 0]{\phantom{9}};
  \node[node distance=1mm] (1) [right=of 0] {\phantom{1}};
  \node[node distance=1mm] (2) [right=of 1] {2};
  \node[node distance=1mm] (3) [right=of 2] {\phantom{3}};
  \node[node distance=1mm] (4) [right=of 3] {4};
  \node[node distance=1mm] (5) [right=of 4] {\phantom{5}};
  \node[node distance=1mm] (6) [right=of 5] {6};
  \node[node distance=1mm] (7) [right=of 6] {\phantom{7}};
  \node[node distance=1mm] (8) [right=of 7] {8};
  \node[node distance=1mm] (9) [right=of 8] {\phantom{9}};
  \node[node distance=1mm] (10) [right=of 9] {10};
  \node[node distance=1mm] (11) [right=of 10] {\phantom{9}};
  \node[node distance=1mm] (12) [right=of 11] {\phantom{9}};


  \node [above=0 mm of 0] {$\bullet$}; 
  \node [above=2 mm of 2] (v2) {$\bullet$}; 
  \node [above=4 mm of 4] {?}; 
  \node [above=6 mm of 6] (v6) {$\bullet$}; 
  \node [above=7 mm of 7] {$\bullet$}; 
  \node [above=9 mm of 9] {$\bullet$}; 
  \node [above=10 mm of 10] (v10) {$\bullet$}; 


  \node[rectangle,draw] (s0) [below=of 0] {0};
  \node[rectangle,draw] (s2) [below=of 2] {2};
  \node[rectangle,draw] (s4) [below=of 4] {u};
  \node[rectangle,draw] (s6) [below=of 6] {6};
  \node[rectangle,draw] (s7) [below=of 7] {7};
  \node[rectangle,draw] (s9) [below=of 9] {9};
  \node[rectangle,draw] (s10) [below=of 10] {\color{gray}10};
  \node[rectangle,draw] (s11) [below=of 11] {\phantom{9}};
  \node[rectangle,draw] (s12) [below=of 12] {\phantom{9}};


  \draw[<->] (v2.north east) to (v6.north west)
  node [above,sloped,midway] {\small gap};

  \draw[<->] (v6.south east) to (v10.south west)
  node [below,sloped,midway] {\small irregular};

  
  %rd: 5s |inf| mean
  \node [circle,draw] (rd50) [below=4cm of 0] {u};
  \node [circle,draw] (rd55) [below=4cm of 5] {2};
  \node [circle,draw] (rd510) [below=4cm of 10] {u};
  \node [below=4.3cm of 10] {\color{gray}8};
 
  \draw[->,bend right] (s4) to (rd55);
  \draw[->,bend right] (s2) to (rd55);

  \draw[->,dotted,bend right] (s10) to (rd510);
  \draw[->,bend right] (s9) to (rd510);
  \draw[->,bend right] (s7) to (rd510);
  \draw[->,bend right] (s6) to (rd510);

  
  %rd: 3s |inf| mean
  \node [circle,draw] (rd30) [below=of s0] {u};
  \node [circle,draw,fill=white] (rd33) [below=2.5cm of 3] {2};
  \node [circle,draw,fill=white] (rd36) [below=2.5cm of 6] {5};
  \node [circle,draw,fill=white] (rd39) [below=2.5cm of 9] {8};
  \node [circle,draw] (rd312) [below=2.5cm of 12] {u};

  \draw[->] (s2) to (rd33);

  \draw[->] (s6) to (rd36);
  \draw[->] (s4) to (rd36);

  \draw[->] (s9) to (rd39);
  \draw[->] (s7) to (rd39);

  \draw[->,dotted] (s12) to (rd312);
  \draw[->,dotted] (s11) to (rd312);
  \draw[->,dotted] (s10) to (rd312);



  %eixos
  \node (et0) [above=1mm of -1] {};
  \node (et12) [above=1mm of 11] {};
  \node [right=-2mm of et12] {time};
  \draw[->] (et0) to (et12);
  \node (y5) [above=5mm of 0] {--};
  \node [left=-1.5mm of y5] {5};
  \node (y10) [above=10mm of 0] {--};
  \node [left=-1.5mm of y10] {10};

  \node (inici) [above=3.1cm of s0] {init};
  \node (inici2) [below=3.3cm of s0] {};
  \draw[-,dotted] (inici) to (inici2);

  \node (fi) [above=3.4cm of s9.east] {now};
  \node (fi2) [below=3.5cm of s9.east] {};
  \draw[-,dotted] (fi) to (fi2);


  % \node (fut) [below right=1mm and 1mm of fi] {future};
  % \draw[->] (fut.south west) to (fut.south east);

  % \node (pas) [below left=1mm and 1mm of fi] {past};
  % \draw[->] (pas.south east) to (pas.south west);

  \node [above=0cm of s0] {\makebox[0.5cm][l]{sample every 2 u.t.}};
  \node [below=0.5cm of s0] {\makebox[0.5cm][l]{mean/3}};
  \node [below=2cm of s0] {\makebox[0.5cm][l]{mean/5}};

\end{tikzpicture}



%%% Local Variables:
%%% TeX-master: "../main"
%%% ispell-local-dictionary: "british"
%%% End:

  \caption{Multiresolution diagram snapshot with irregular sampling}
  \label{fig:mtsms:sequence-irregular}
\end{figure}

At the top of the figures there is a plot of a time series with time
axis in general units of time (u.t.) and with value axis in
undetermined units. The 'now' point shows when the snapshot has been
taken, so the time before is the past and the time after is the
future, which is grey coloured. The 'init' point shows when the
database system has started sampling, so data in time before is
unknown; we indicate the starting point as being zero u.t.\ and unknown
time points with negative units.

At the bottom of the figures there is a digram showing the
multiresolution action. The first row shows the numerical time series'
values corresponding to the above plot; in
fig.~\ref{fig:mtsms:sequence} the time series is sampled every one
unit of time and in fig.~\ref{fig:mtsms:sequence-irregular} every
two. The second and the third row show a particular schema of a
multiresolution database consisting in two time resolutions for the
time series: one computes the mean of the sampled values every three
u.t.\ and the other computes the mean every five u.t. In this example,
computing the mean acts as the attribute summarising function of which
we have spoken earlier. All data stored before zero time, this
included, is unknown as sampling had not started. For the future
values we also simulate as having unknown stored values which will
change as time advances.

If we look at fig.~\ref{fig:mtsms:sequence} we see, drawn by arrows,
that every three sampled values a mean is stored and independently
every five values another mean is stored. For the future values we
show in gray that if we advance the time one u.t.\ then value 10 is
sampled an the mean for time 10 can be computed resulting 8 but not
yet the mean for time 12.

Fig.~\ref{fig:mtsms:sequence-irregular} is essentially the same but
showing two possible monitoring irregularities: a gap and a time
disruption. In other words, we want to sample the time series every 2
u.t.\ but first for some reason it can not be done in time 4 and
second the sampling clock is disrupted and samples are done in time 7
and 9 instead of 8. The resulting stored time schema is the same: on
time resolution every 3 u.t.\ and the other every 5 u.t.; that is,
without time disruptions. The resulting stored values are computed
from the known sampled values, some coincide with
fig.~\ref{fig:mtsms:sequence} whereas some differ specially in the
gap. A better function than mean would solve this, we extend this
further in section~\ref{sec:model:interpolador}.





MTSMS improve TSMS features in various aspects:
\begin{itemize}

\item Voluminous data. Monitoring systems capture a huge amount of
  data from sensors. In order to be able to process this information,
  data volume must be reduced. With the multiresolution approach only
  the most interesting segments of data are stored. This segments are
  seen as different resolutions for the same time series and the user
  configures how they are extracted and summarised by defining
  different consolidation steps and functions. Multiresolution can
  also be useful at visualisation time as the user is able to select
  the best time range and time step that fits into the screen; there
  is no need to process with more quantity of data than the one that
  can be shown. In figure~\ref{fig:mtsms:sequence} there is an example of
  extracting two resolutions: one every three units of time and
  another every five.

\item Data validation. Monitoring systems capture data but can occur
  some drawbacks that will affect later the process of time series
  analysis. Main problems are found when monitors can not capture
  data, known as gaps, or capture data erroneously, such as outlayers.
  The multiresolution attribute functions cope well with validating,
  filtering and calculating with this unknown data in order to keep a
  consistent historic. In figure~\ref{fig:mtsms:sequence-irregular} an
  example of a gap can be seen.

\item Data time regularising. Another monitoring side effect happens
  when the sampling rate is not constant, that is when the resulting
  data is not equi-time spaced. This no regularities can come from
  sampling jitters in periodic sampling or from no periodic
  event-based sampling. The multiresolution consolidation regularises
  the time interval when processes a time series, therefore each
  resulting time series segment has a regular time resolution. This
  regularising approach could also be used when the user wants to
  consult another resolution for a time series, such as changing
  periodic data from a month to a year step. In
  figure~\ref{fig:mtsms:sequence-irregular} an example of time
  regularising can be seen.

\item Information summaries. Time series analysis typically focuses on
  reconstructing the original signal. However, the user objective in a
  database system is to consult some information. The multiresolution
  approach is a lossy compression storage solution for data. Therefore
  it can be regarded as not only approximating to the original time
  series function but also extracting the interesting information. The
  selected information must be determined a priori assuming the
  context where the future queries will be done. In
  figure~\ref{fig:mtsms:sequence} there is an example of summarising by
  mean attribute.

\end{itemize}





\subsection{Related work}

There are some prior works concerning TSMS. 

RRDtool from Oetiker \cite{rrdtool} is a free software database
management system. It is designed to be used in monitoring
systems. Because of this, it is focused to a particular kind of data,
gauges and counters, and it lacks general time series
operations. RRDtool can store multiple time resolution data. The work
in this paper is partially inspired in RRDtool.

Cougar \cite{bonnet01} is a sensor database system. It has two
structures: one for sensor properties stored into relational tables
and another for time series stored into data sequences from
sensors. Time series have specific operations and can combine
relations and sequences. Cougar target field is sensor networks, where
sensor data is stored distributed in sensors. Queries are resolved
combining sensor data in a data stream orientation, which improves
processing performance. However, data streams imposes restriction on
operators so this TSMS can not be generalised to other time series
types. 
%Moreover, sequences and relations can collide in representing ambiguously time series.


SciDB \cite{stonebraker09:scidb} and SciQL \cite{zhang11} are array
database systems. These systems are intended for science applications,
in which time series play a principal role. They structure time series
into arrays in order to achieve multidimensional analysis and allow
tables to store other data.  Although SciDB is based on arrays, in
which it includes time series, it does not consider time series
special needs, that is not considering how continuously voluminous
data or temporal coherence are achieved.  In contrast, SciQL shows how
some time series properties are achieved by the DBMS such as time
series regularities, interpolation or correlation queries.  However,
difference between tables and arrays seems too physical approached and
consequently can collide in representing ambiguously time series.

Bitemporal DBMS is another database field related with
time. Bitemporal data is mainly targeted at keeping historical events
in the database by associating time intervals to data.  Bitemporal
data and time series data are not exactly the same and so can not be
treated interchangeably \cite{schmidt95}. However, there are some
similarities between time series and bitemporal data that can be
considered. First, extending a relational database model to manage
bitemporal data shows the way to extend relational DBMS with new types
and how to model them. Second, bitemporal data modelling settles some
time-related concepts that can be extended to time series.

The recent bitemporal data research in relational DBMS model terms
\cite{date02:_tempor_data_relat_model} marks a promising
foundation. It models bitemporal data as relations extended with time
intervals attributes and extends relational operations in order to
deal with related time aspects.










%%% Local Variables:
%%% TeX-master: "main"
%%% ispell-local-dictionary: "british"
%%% End:

% LocalWords:  multiresolution

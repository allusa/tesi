\section{Example}
\label{sec:example}

\todo{refer}


Next we show a real example database for a time series data. Actual
data comes from a temperature distributed sensor monitoring system,
\cite{alippi10}. We focus on one sensor data.

The Figure~\ref{fig:exemple:original} shows the original data for one
year and a half. The plot interpolates linearly the measures. In this
plot we can see that there is missing data and some outlying
observations. There are $146\,709$ stored values.

\begin{figure}[tp]
  \centering
  \tikzset{every picture/.style={scale=0.8}}
  \usetikzlibrary{dateplot}    
\begin{tikzpicture}
    \begin{axis}[
        date coordinates in=x,
%        xticklabel={\pgfcalendar{tickcal}{\tick}{\tick}{\pgfcalendarshorthand{m}{.}}},
        xticklabel={\pgfcalendarmonthshortname{\month} \year},
        xticklabel style= {rotate=15,anchor=east},
        xlabel=Time,
        ylabel=Temperature (K),
        ymax = 320,
        clip=false,
        y filter/.code = { \pgfmathparse{(#1>320)*322+(#1<320)*#1}},
        ]
       \addplot[blue] file {dades/matriu0.originalbyday.dat};

%      \node[right] at (axis cs:2011-10-12,330) {\footnotesize(2938)};
       \node (break) at (axis cs:2011-10-02,318)[inner sep=0pt,minimum width=0.75em, minimum height=0.5ex,fill=white] {};
    \draw [fill=red,color=blue] (break.north east) -- (break.north west) (break.south west) -- (break.south east);



  \end{axis}
\end{tikzpicture}



%%% Local Variables:
%%% TeX-master: "../main"
%%% ispell-local-dictionary: "british"
%%% End:

  \caption{Example of a temperature time series data}
  \label{fig:exemple:original}
\end{figure}

\emph{Schema}. We design a multiresolution time series database that
stores a time series with high resolution at recent times and with low
resolution at older times. The schema is illustrated in the
Figure~\ref{fig:exemple:window}. At the top there are four discs with
different number of measures and at the bottom there is a timeline
showing the time series chopped along time. Going from most to least
granularity disks are configures as follows: (i) a measure every 5 h
in the fourth disc which has a capacity of 24 measures and thus it
spans 5 days; (ii) a measure every 2 days in the third disc, with a
capacity of 20 thus spanning 40 days; (iii) a measure every 15 days in
the second disc, with a capacity of 12 thus spanning 180 days and;
(iv) a measure every 50 days in the first disc that, with a capacity
of 12 results in a span of 600 days.

\begin{figure}[tp]
  \centering
  \setlength{\unitlength}{1.3mm}
  %mrd.afegeix_disc(h5,24,mitjana,zero)
%mrd.afegeix_disc(d2,20,mitjana,zero)
%mrd.afegeix_disc(d15,12,mitjana,zero)
%mrd.afegeix_disc(d50,12,mitjana,zero)
\tiny
\begin{center}
%\begin{multicols}{4} 


    \begin{picture}(14,12)(-7,-6)
    \put(0,-1){\makebox(0,0)[c]{{\color{blue}50 days}}}
      \put(0,0){\circle{10}}
      \put(5,0){\circle{0.8}}
      \put(4.33,2.5){\circle{0.8}}
      \put(2.5,4.33){\circle{0.8}}
      \put(0,5){\circle{0.8}}
      \put(-2.5,4.33){\circle{0.8}}   
      \put(-4.33,2.5){\circle{0.8}}
      \put(-5,0){\circle{0.8}}
      \put(-4.33,-2.5){\circle{0.8}}
      \put(-2.5,-4.33){\circle{0.8}} 
      \put(0,-5){\circle{0.8}}
      \put(2.5,-4.33){\circle{0.8}} 
      \put(4.33,-2.5){\circle{0.8}}
      \put(0,0){\vector(0,1){5}}
      \put(0,0){\oval(5,5)[t]}
      \put(-2.5,0){\makebox(0,0)[c]{$\vee$}}
    \end{picture}
%
    \begin{picture}(14,12)(-7,-6)
    \put(0,-1){\makebox(0,0)[c]{{\color{brown}15 days}}}
      \put(0,0){\circle{10}}
      \put(5,0){\circle{0.8}}
      \put(4.33,2.5){\circle{0.8}}
      \put(2.5,4.33){\circle{0.8}}
      \put(0,5){\circle{0.8}}
      \put(-2.5,4.33){\circle{0.8}}   
      \put(-4.33,2.5){\circle{0.8}}
      \put(-5,0){\circle{0.8}}
      \put(-4.33,-2.5){\circle{0.8}}
      \put(-2.5,-4.33){\circle{0.8}} 
      \put(0,-5){\circle{0.8}}
      \put(2.5,-4.33){\circle{0.8}} 
      \put(4.33,-2.5){\circle{0.8}}
      \put(0,0){\vector(0,1){5}}
      \put(0,0){\oval(5,5)[t]}
      \put(-2.5,0){\makebox(0,0)[c]{$\vee$}}
    \end{picture}
%
    \begin{picture}(14,12)(-7,-6)
    \put(0,-1){\makebox(0,0)[c]{{\color{red}2 days}}}
      \put(0,0){\circle{10}}
      %\put(5,0){\circle{0.8}}
      \put(4.82,1.29){\circle{0.8}}
      \put(4.33,2.5){\circle{0.8}}
     \put(3.5,3.5){\circle{0.8}}
      \put(2.5,4.33){\circle{0.8}}
      \put(1.29,4.82){\circle{0.8}}
      %\put(0,5){\circle{0.8}}
      \put(-1.29,4.82){\circle{0.8}}
      \put(-2.5,4.33){\circle{0.8}}
       \put(-3.5,3.5){\circle{0.8}} 
      \put(-4.33,2.5){\circle{0.8}}
    \put(-4.82,1.29){\circle{0.8}}
      %\put(-5,0){\circle{0.8}}
    \put(-4.82,-1.29){\circle{0.8}}
      \put(-4.33,-2.5){\circle{0.8}}
      \put(-3.5,-3.5){\circle{0.8}} 
      \put(-2.5,-4.33){\circle{0.8 } } 
      \put(-1.29,-4.82){\circle{0.8 }}
      % \put(0,-5){\circle{0.8 }}
     \put(1.29,-4.82){\circle{0.8 }}
      \put(2.5,-4.33){\circle{0.8}}
      \put(3.5,-3.5){\circle{0.8}} 
      \put(4.33,-2.5){\circle{0.8}}
  \put(4.82,-1.29){\circle{0.8}}
      \put(0,0){\vector(0,1){5}}
      \put(0,0){\oval(5,5)[t]}
      \put(-2.5,0){\makebox(0,0)[c]{$\vee$}}
    \end{picture}
%
    \begin{picture}(14,12)(-7,-6)
    \put(0,-1){\makebox(0,0)[c]{{\color{cyan}5 hours}}}
      \put(0,0){\circle{10}}
      \put(5,0){\circle{0.8}}
      \put(4.82,1.29){\circle{0.8}}
      \put(4.33,2.5){\circle{0.8}}
     \put(3.5,3.5){\circle{0.8}}
      \put(2.5,4.33){\circle{0.8}}
      \put(1.29,4.82){\circle{0.8}}
      \put(0,5){\circle{0.8}}
      \put(-1.29,4.82){\circle{0.8}}
      \put(-2.5,4.33){\circle{0.8}}
       \put(-3.5,3.5){\circle{0.8}} 
      \put(-4.33,2.5){\circle{0.8}}
    \put(-4.82,1.29){\circle{0.8}}
      \put(-5,0){\circle{0.8}}
    \put(-4.82,-1.29){\circle{0.8}}
      \put(-4.33,-2.5){\circle{0.8}}
      \put(-3.5,-3.5){\circle{0.8}} 
      \put(-2.5,-4.33){\circle{0.8 } } 
      \put(-1.29,-4.82){\circle{0.8 }}
\put(0,-5){\circle{0.8 }}
     \put(1.29,-4.82){\circle{0.8 }}
      \put(2.5,-4.33){\circle{0.8}}
      \put(3.5,-3.5){\circle{0.8}} 
      \put(4.33,-2.5){\circle{0.8}}
  \put(4.82,-1.29){\circle{0.8}}
      \put(0,0){\vector(0,1){5}}
      \put(0,0){\oval(5,5)[t]}
      \put(-2.5,0){\makebox(0,0)[c]{$\vee$}}
    \end{picture}


%\end{multicols}

\vspace{-10pt}

\setlength{\unitlength}{900sp}
\begin{picture}(14460,5066)(7322,-7148)
\thinlines
{\color[rgb]{0,0,0}\put(7300,-6271){\line( 0,-1){386}}
}%
{\color[rgb]{0,0,0}\put(7782,-6271){\line( 0,-1){386}}
}%
{\color[rgb]{0,0,0}\put(8263,-6271){\line( 0,-1){386}}
}%
{\color[rgb]{0,0,0}\put(8745,-6271){\line( 0,-1){386}}
}%
{\color[rgb]{0,0,0}\put(9227,-6271){\line( 0,-1){386}}
}%
{\color[rgb]{0,0,0}\put(9709,-6271){\line( 0,-1){386}}
}%
{\color[rgb]{0,0,0}\put(10191,-6271){\line( 0,-1){386}}
}%
{\color[rgb]{0,0,0}\put(10673,-6271){\line( 0,-1){386}}
}%
{\color[rgb]{0,0,0}\put(11155,-6271){\line( 0,-1){386}}
}%
{\color[rgb]{0,0,0}\put(11637,-6271){\line( 0,-1){386}}
}%
{\color[rgb]{0,0,0}\put(12119,-6271){\line( 0,-1){386}}
}%
{\color[rgb]{0,0,0}\put(12600,-6271){\line( 0,-1){386}}
}%
{\color[rgb]{0,0,0}\put(13082,-6271){\line( 0,-1){386}}
}%
{\color[rgb]{0,0,0}\put(13564,-6271){\line( 0,-1){386}}
}%
{\color[rgb]{0,0,0}\put(14046,-6271){\line( 0,-1){386}}
}%
{\color[rgb]{0,0,0}\put(14528,-6271){\line( 0,-1){386}}
}%
{\color[rgb]{0,0,0}\put(15010,-6271){\line( 0,-1){386}}
}%
{\color[rgb]{0,0,0}\put(15492,-6271){\line( 0,-1){386}}
}%
{\color[rgb]{0,0,0}\put(15974,-6271){\line( 0,-1){386}}
}%
{\color[rgb]{0,0,0}\put(16456,-6271){\line( 0,-1){386}}
}%
{\color[rgb]{0,0,0}\put(16938,-6271){\line( 0,-1){386}}
}%
{\color[rgb]{0,0,0}\put(17419,-6271){\line( 0,-1){386}}
}%
{\color[rgb]{0,0,0}\put(17901,-6271){\line( 0,-1){386}}
}%
{\color[rgb]{0,0,0}\put(18383,-6271){\line( 0,-1){386}}
}%
{\color[rgb]{0,0,0}\put(18865,-6271){\line( 0,-1){386}}
}%
{\color[rgb]{0,0,0}\put(19347,-6271){\line( 0,-1){386}}
}%
{\color[rgb]{0,0,0}\put(19829,-6271){\line( 0,-1){386}}
}%
{\color[rgb]{0,0,0}\put(20311,-6271){\line( 0,-1){386}}
}%
{\color[rgb]{0,0,0}\put(20793,-6271){\line( 0,-1){386}}
}%
{\color[rgb]{0,0,0}\put(21275,-6271){\line( 0,-1){386}}
}%
{\color[rgb]{0,0,0}\put(7300,-6271){\line( 0,-1){1157}}
}%
{\color[rgb]{0,0,0}\put(9709,-6271){\line( 0,-1){1157}}
}%
{\color[rgb]{0,0,0}\put(12119,-6271){\line( 0,-1){1157}}
}%
{\color[rgb]{0,0,0}\put(14528,-6271){\line( 0,-1){1157}}
}%
{\color[rgb]{0,0,0}\put(16938,-6271){\line( 0,-1){1157}}
}%
{\color[rgb]{0,0,0}\put(19347,-6271){\line( 0,-1){1157}}
}%
{\color[rgb]{0,0,0}\put(21756,-6271){\line( 0,-1){1157}}
}%
{\color[rgb]{0,0,0}\put(7300,-6271){\line( 1, 0){14456}}
}%

\put(7322,-6271){\line( 0,1){3000}}
\put(21756,-7783){\makebox(0,0)[b]{now}}%
\put(7322,-7783){\makebox(0,0)[b]{600 days back}}%

\color{blue}
\put(21782,-5928){\line( -1,0){14460}}
\put(21782,-5928){\line( 0,1){779}}
\put(21782,-5149){\line( -1,0){14460}}
\put(7322,-5928){\line( 0,1){779}}
\put(14530,-5450){\makebox(0,0)[c]{600 days}}

\color{brown}
\put(21782,-5149){\line( 0,1){779}}
\put(21782,-4370){\line( -1,0){4438}}
\put(17344,-5149){\line( 0,1){779}}
\put(19563,-4800){\makebox(0,0)[c]{180 days}}

\color{red}
\put(21782,-4370){\line( 0,1){779}}
\put(21782,-3591){\line( -1,0){964}}
\put(20818,-4370){\line( 0,1){779}}
\put(21300,-3950){\makebox(0,0)[c]{40d}}

\color{cyan}
\put(21782,-3591){\line( 0,1){779}}
\put(21782,-2812){\line( -1,0){120}}
\put(21661,-3591){\line( 0,1){779}}
\put(21300,-3201){\makebox(0,0)[c]{5d}}
\end{picture}%


\normalsize

\end{center}
  \caption{Schema of resolutions in a \acro{MTSDB}}
  \label{fig:exemple:window}
\end{figure}

\emph{Attribute aggregate functions}.  In order to illustrate this
example we consolidate all the resolution discs using the zohe
arithmetic mean aggregate function and the highest resolution disc
using the zohe maximum aggregate function. Next, we show the process
in designing both aggregate functions.

In accordance to the zohe Equation~\ref{eq:zohe} defined using
left-continuous step functions, we define the zohe attribute aggregate
function family as the one interpreting the consolidation time
interval left-continuous $i=(T_0,T_f]$ and the one aggregating on the
subset $S'=S(T_0,T_f] \cup \{\min(S-S(-\infty,T_f))\}$:
\begin{itemize}
  \renewcommand{\labelitemi}{--}
\item maximum$^{zohe}$: $S \times i \mapsto m'$ where $V(m') =
  \max_{\forall m \in S'}(V(m))$ and $T(m')=T_f$.
\item arithmetic mean$^{zohe}$: $S \times i \mapsto m'$ where $V(m')
  = \frac{1}{|S'|} \sum\limits_{\forall m\in S'} V(m)$ and
  $T(m')=T_f$. 
\end{itemize}

%\subsection{Results}

The time series after consolidating the \acro{MTSDB} are shown in the
Figure~\ref{fig:exemple:4mrd}, where each graphic corresponds to a
resolution disc time series. Each title shows the disc resolution and
its cardinal, and each attribute aggregate function has different
colour.  Each time series is plotted with zohe continuous
representation. Time axis has \acro{UTC} units rounded to nearest time
points and temperature axis has Kelvin units. Outlayers are marked as
discontinuities, for instance see fourth plot's 2938 K maximum.

\begin{figure}[tp]
  \centering
  \tikzset{
    every picture/.style={scale=0.7},
  }
  
  \begin{tikzpicture}[scale=0.6, every node/.style={transform shape}]
    \begin{axis}[
        multiresoluciodate,
        title={$R_1$: 5h $|24|$},
        xticklabel={\day--\hour:\minute},
        clip=false,
        ]
       \addplot[const plot mark right, blue] table[col sep=comma] {imatges/exemple/dades-matriu0/R18000mean_zohe.csv};
     \node[left] at (axis cs:2011-10-19,274) {\footnotesize oct.~2011};
  \end{axis}
\end{tikzpicture}
%
  \begin{tikzpicture}[scale=0.6, every node/.style={transform shape}]
    \begin{axis}[
        multiresoluciodate,
        title={$R_2$: 2d $|20|$},
        xticklabel={\day~\pgfcalendarmonthshortname{\month}},
        clip=false,
        ]
       \addplot[const plot mark right, blue] table[col sep=comma] {imatges/exemple/dades-matriu0/R172800mean_zohe.csv};
     \node[left] at (axis cs:2011-10-21,279) {\footnotesize 2011};
  \end{axis}
\end{tikzpicture}
%
  \begin{tikzpicture}[scale=0.6, every node/.style={transform shape}]
    \begin{axis}[
        multiresoluciodate,
        title={$R_3$: 15d $|12|$},
        xticklabel={\day~\pgfcalendarmonthshortname{\month}},
        y filter/.code = { \pgfmathparse{(#1>320)*330+(#1<320)*#1}},
        ymax = 320,
        clip=false,
        ]

       \addplot[const plot mark right, blue] table[col sep=comma] {imatges/exemple/dades-matriu0/R1296000mean_zohe.csv};

      \addplot[const plot mark right, orange] table[col sep=comma] {imatges/exemple/dades-matriu0/R1296000maximum_zohe.csv};

      \node[right] at (axis cs:2011-10-07,330) {\footnotesize(2938)};
       \node (break) at (axis cs:2011-09-23,325)[inner sep=0pt,minimum width=0.75em, minimum height=0.5ex,fill=white] {};
    \draw [fill=red,color=orange] (break.north east) -- (break.north west) (break.south west) -- (break.south east);

     \node[left] at (axis cs:2011-10-27,273) {\footnotesize 2011};

  \end{axis}
\end{tikzpicture}
%
\begin{tikzpicture}[scale=0.6, every node/.style={transform shape}]
    \begin{axis}[
        multiresoluciodate,
        xticklabel={\pgfcalendarmonthshortname{\month}~\year},
        title={$R_4$: 50d $|12|$},
        xlabel={Temps (UTC)},
%        ylabel={Temperatura (K)},
        ymax = 320,
        clip=false,
%v1.6     restrict y to domain=0:320,
        y filter/.code = { \pgfmathparse{(#1>320)*330+(#1<320)*#1}},
        ]

       \addplot[const plot mark right, blue] table[col sep=comma] {imatges/exemple/dades-matriu0/R4320000mean_zohe.csv};
       \addlegendentry{mitjana};

       \addplot[const plot mark right, orange] table[col sep=comma] {imatges/exemple/dades-matriu0/R4320000maximum_zohe.csv};
       \addlegendentry{màxim};

       \node[right] at (axis cs:2011-10-12,330) {\footnotesize(2938)};
       \node (break) at (axis cs:2011-08-24,325)[inner sep=0pt,minimum width=0.75em, minimum height=0.5ex,fill=white] {};
    \draw [fill=red,color=orange] (break.north east) -- (break.north west) (break.south west) -- (break.south east);

  \end{axis}
\end{tikzpicture}




%%% Local Variables:
%%% TeX-master: "../../main"
%%% End:

  \caption{Resolution discs' time series in a \acro{MTSDB}}
  \label{fig:exemple:4mrd}
\end{figure}

In all the four plots, we can see that mean aggregate function has
filled missing data and filtered outlayer observations. This is due
to the aggregate function coming from a zohe interpretation.

Data can be queried.  For example, one query would be the union of the
four time subseries choosing the one with the highest resolution as
shown in the Figure~\ref{fig:exemple:4mrdtot}.  Each time series is
plotted interpolating linearly its measures, note that this linearly
visualisation seems right time displaced as time series comes from a
zohe aggregation.  Comparing this figure with the original series, see
Figure~\ref{fig:exemple:original}, we observe that it resembles an
incremental low-pass filter because we applied mean aggregation while
the maximum aggregation resembles an envelope function.

\begin{figure}[tp]
  \centering
  \tikzset{every picture/.style={scale=0.8}}
  \usetikzlibrary{dateplot}  
%\usetikzlibrary{pgfplots.groupplots}

\pgfplotsset{
   petit/.style={
        ylabel=Temperature (K),
%        width=\textwidth,
%        height=3.5cm,
        legend style={font=\footnotesize},
        tick label style={font=\footnotesize},
        label style={font=\tiny},
        title style={font=\small,below, anchor=north,fill=white},
        xticklabel style= {rotate=15,anchor=east},
%        every axis title shift=0pt,
%        max space between ticks=15,
        every mark/.append style={mark size=6},
        major tick length=0.1cm,
        minor tick length=0.066cm,
        very thin,
        every axis legend/.append style={
          at={(1,0.02)},
          anchor=south east,
          draw = none},
        legend columns = 4,
    }
}

\begin{tikzpicture}
    \begin{axis}[
        petit,
        date coordinates in=x,
        xticklabel={\pgfcalendarmonthshortname{\month} \year},
        xlabel=Time (UTC),
%        unbounded coords=jump, %v>1.4
%        unbounded coords=discard, %v>1.4
        ymax = 320,
        clip=false,
%v1.6     restrict y to domain=0:320,
        y filter/.code = { \pgfmathparse{(#1>320)*330+(#1<320)*#1}},
        ]

       \addplot[black!15] file {dades/matriu0.originalbyday.dat};
       \addlegendentry{original};

       \addplot[blue] table[col sep=comma] {dades/mrdb-matriu0/union1.csv};
       \addlegendentry{mean};

       \addplot[orange] table[col sep=comma] {dades/mrdb-matriu0/union0.csv};
       \addlegendentry{max};

%       \node[right] at (axis cs:2011-10-12,330) {\mbox{(2938)}};
       \node (break) at (axis cs:2011-09-25,325)[inner sep=0pt,minimum width=0.9em, minimum height=0.4ex,fill=white] {};
    \draw [fill=red,color=orange] (break.north east) -- (break.north west) (break.south west) -- (break.south east);


  \end{axis}
\end{tikzpicture}
%http://tex.stackexchange.com/questions/46422/axis-break-in-pgfplots

%http://tex.stackexchange.com/questions/52409/insert-a-separate-mark-inside-a-pgfplots-graph



%%% Local Variables:
%%% TeX-master: "../main"
%%% ispell-local-dictionary: "british"
%%% End:

  \caption{All time series united from the MTSDB}
  \label{fig:exemple:4mrdtot}
\end{figure}

Note that this \acro{MTSDB} example schema does not store the complete
original data but an approximation to the original function which
contains more information for recent times.


%%% Local Variables:
%%% TeX-master: "main"
%%% ispell-local-dictionary: "british"
%%% End:

% LocalWords:  multiresolution MTSDB

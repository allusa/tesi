

\documentclass{acm_proc_article-sp}

\usepackage[utf8]{inputenc}

\usepackage{todo}

%\usepackage{url}



\usepackage{tikz,pgfplots}
\usetikzlibrary{dateplot}
\usetikzlibrary{shapes,arrows,positioning}
%\usetikzlibrary{dateplot}  
%\usetikzlibrary{pgfplots.groupplots}






% \pgfplotsset{
%    rrbtimeseries/.style={
%         % date coordinates in=x,
%         ylabel=Temperature (K),
%         % legend style={font=\footnotesize},
%         % tick label style={font=\footnotesize},
%         % every axis x label/.style={
%         %   at={(1.3,0)},
%         %   anchor=north,
%         %   },
%         % label style={font=\footnotesize},
%         % xticklabel style= {rotate=17,anchor=north east},
% %        every axis title shift=0pt,
% %        max space between ticks=15,
%        %  every mark/.append style={mark size=6},
%        %  major tick length=0.1cm,
%        %  minor tick length=0.066cm,
%        %  very thin,
%        %  every axis legend/.append style={
%        %    at={(1.2,0)},
%        %    anchor=south east,
%        %    draw = none},
%        % legend columns = 4,
%        % unbounded coords=jump, %v>1.4
%     },

%   % rrbrs/.style={
%   %       rrbtimeseries,
%   %       width = \textwidth,
%   %       height = 0.25\textwidth,
%   %       every axis x label/.style={
%   %         at={(1.3,-1)},
%   %         anchor=north,
%   %         },
%   %       ylabel = {},  
%   %       max space between ticks=50,
%   %       every axis legend/.append style={
%   %         at={(1,-1.1)},
%   %         anchor=north east,
%   %         draw = none},
%   %       title style={font=\small,below,anchor=north,fill=white},
%   %   },
%  }




\pgfplotsset{
   timeseries/.style={
%        date coordinates in=x,
        ylabel=Temperature (K),
        legend style={font=\footnotesize},
        tick label style={font=\footnotesize},
        every axis x label/.style={
          at={(1.3,0)},
          anchor=north,
          },
        label style={font=\footnotesize},
        xticklabel style= {rotate=17,anchor=north east},
%        every axis title shift=0pt,
%        max space between ticks=15,
        every mark/.append style={mark size=6},
        major tick length=0.1cm,
        minor tick length=0.066cm,
        very thin,
        every axis legend/.append style={
          at={(1.2,0)},
          anchor=south east,
          draw = none},
       legend columns = 4,
    },
    rd/.style={
        timeseries,
        % every axis x label/.style={
        %   at={(1.3,-1)},
        %   anchor=north,
        %   },
        label style={font=\footnotesize},
        ylabel = {},
        width=\textwidth,
        height=3.5cm,   
        max space between ticks=50,
        every axis legend/.append style={
          at={(1,-1.1)},
          anchor=north east,
          draw = none},
%        title style={font=\small,below, at={(0.7,1.7)},anchor=north},
    }
}


%        unbounded coords=jump, %v>1.4
%        unbounded coords=discard, %v>1.4


%http://tex.stackexchange.com/questions/46422/axis-break-in-pgfplots

%http://tex.stackexchange.com/questions/52409/insert-a-separate-mark-inside-a-pgfplots-graph


\usepackage{amsmath}
\newtheorem{definition}{Definition}
\DeclareMathOperator*{\nex}{next}
\DeclareMathOperator*{\prev}{prev}
\newcommand{\acro}[1]{\textsc{\lowercase{#1}}}


\usepackage[
style=numeric-comp,
sortcites=true,
% backref=true,
]{biblatex}
\bibliography{bibliografia}
\ExecuteBibliographyOptions{
  isbn=false,url=false,doi=false,eprint=false,alldates=terse,firstinits=true,abbreviate=true,maxnames=2,
}
\bibitemsep 0mm \bibnamesep 0mm \bibinitsep 0mm \biblabelsep 1mm \bibhang 0mm 



\begin{document}



\title{Operating with a Multiresolution Model}
\subtitle{[Extended Abstract]}

\numberofauthors{3} 
\author{%
% The command \alignauthor (no curly braces needed) should
% precede each author name, affiliation/snail-mail address and
% e-mail address. Additionally, tag each line of
% affiliation/address with \affaddr, and tag the
% e-mail address with \email.
%
% 1st. author
\alignauthor
Escobet Canal, Teresa\\
       %\affaddr{Institute for Clarity in Documentation}\\
       %\affaddr{1932 Wallamaloo Lane}\\
       %\affaddr{Wallamaloo, New Zealand}\\
       \email{teresa.escobet@upc.edu}
% 2nd. author
\alignauthor
Llusà Serra, Aleix\\
       \affaddr{Department of Electronic System Design and Programming}\\
       \affaddr{Universitat Politècnica de Catalunya}\\
       %\affaddr{Av.~Bases de Manresa 61--73}
       \affaddr{08242 Manresa, ES-CT}\\
       \email{aleix@dipse.upc.edu}
% 3rd. author
\alignauthor Vila-Marta, Sebastià\\
       %\affaddr{The Th{\o}rv{\"a}ld Group}\\
       %\affaddr{1 Th{\o}rv{\"a}ld Circle}\\
       %\affaddr{Hekla, Iceland}\\
       \email{sebas@dipse.upc.edu}
}



\maketitle


\begin{abstract}
%\boldmath
The abstract goes here.
\end{abstract}

% A category with the (minimum) three required fields
\category{H.4}{Information Systems Applications}{Miscellaneous}
%A category including the fourth, optional field follows...
\category{D.2.8}{Software Engineering}{Metrics}[complexity measures, performance measures]

\terms{Theory}

\keywords{ACM proceedings, \LaTeX, text tagging} % NOT required for Proceedings







\section{Model}

\acro{MTSMS}
\acro{MTSM}


La definició del model s'estructura en dues parts:

\begin{itemize}
\item Un model pels (SGST)  que defineix mesura i sèrie temporals.
\item Un model pels (SGSTM) que defineix buffer, disc i subsèrie
  resolució, el qual treballa sobre el model de SGST.
\end{itemize}

\todo{sobre tres nivells}
A l'estat de l'art s'ha d'haver explicat els tres nivell de model de dades segons Date i deixar clar aquí que nosaltres definim un model pel segon nivell: nivell de model lògic. Els models lògics modelen les dades, en canvi els models conceptuals modelen la realitat, Fabian Pascal posa d'exemple conceptual el model E/RM.


Objectius:

En el model de SGST s'observen algunes patologies que poden presentar les sèries temporals. El model de SGSTM soluciona algunes d'aquestes patologies:

\begin{itemize}
\item Regularitza les sèries temporals
\item Tracta i validar les sèries temporals: gestiona els casos de dades errònies o desconegudes i marca quan hi ha valors erronis.
\item És una solució de compressió per a quantitats enormes de dades
\end{itemize}


Però el model de SGSTM també es pot fer servir per altres aplicacions:

* Regularitzar en línia (temps real) una sèrie temporal en diferents períodes de mostreig

* Tenir unes vistes (consultes) a punt (ja processades) amb diferents resolucions d'una sèrie temporal

* Comprimir per decimació (downsampling) o bé farcir forats (reconstrucció del senyal)




\section{Time series preliminaries}
\label{sec:model:preliminaries}

% In this section we introduce some background concepts and the
% nomenclature which we will use later.  First we define the main
% objects of a \acro{MTSMS} which are measures and time series.

% A \emph{measure} is a value measured in a time instant. More formally
% it is a tuple $(v,t)$ where $v$ is the value of the measure and $t \in
% \mathbb{R}$ is the time instant of measurement.  The values of a time
% series can be of any type. For simplicity examples are presented with
% integers or real numbers but can also be strings or vectors.  Let $m =
% (v,t)$ be a measure, $v$ is written as $V(m)$ and $t$ is written as
% $T(m)$.

% The time value defines the canonical order between measures.  Let $m =
% (v_m, t_m)$ and $n = (v_n, t_n)$ be two measures, then $m\geq n$ if
% and only if $t_m\geq t_n$.

% A \emph{time series} is sequence of measures of the same phenomena
% that are ordered in time.
% \begin{definition}[Time series]
%   A \emph{time series} $S$ is a a set of measures of the same
%   phenomena $S = \{m_0, \ldots, m_k\}$ without repeated time values
%   $\forall i,j: i\leq k, j\leq k, i\neq j : T(m_i)\neq T(m_j)$. Given
%   a time series $|S|$, we note its size by $|S|=k+1$. Observe that,
%   because measures in $S$ are of the same phenomena, the type of $S$
%   values is homogeneous.
% \end{definition}

% The order defined by measures implies a total order in a time
% series. As a time series is a finite set, if it is not empty it has a
% maximum and a minimum.  Let $S=\{m_0,\ldots,m_k\}$ be a time series
% and $n\in S$ be a measure. The time series' maximum is $n=\max(S)$ if
% and only if $\forall m \in S: n \geq m $.  Similarly, the time series'
% minimum is $n=\min(S)$ if and only if $\forall m \in S: n \leq m$.

% Given the order defined by time, in a time series we define the
% sequence interval following \cite{last:keogh,last:hetland}.  Let
% $S=\{m_0, \ldots, m_k\}$ be a time series. We define the subset
% $S(r,t] \subseteq S$ as the time series $S(r,t]=\{m\in S | r<T(m)\leq
% t\}$, where $r$ and $t$ are two instants in time.  We also define the
% subset $S(r,+\infty)\subseteq S$ as the time series $S(r,+\infty) =
% \{m\in S | r< T(m) \leq T(\max(S))\}$ and the subset
% $S(-\infty,t)\subseteq S$ as the time series $S(-\infty,t) = \{m\in S
% | T(\min(S))\leq T(m) < t\}$.

% The time order in time series also implies the sequence concept of
% next and previous measure.  Let $S=\{m_0, \ldots, m_k\}$ be a time
% series and $l\in S$ and $n$ be two measures. We define the next
% measure of $n$ in $S$ as $l=\nex_S(n)$ where $l =
% \min(S(T(n),+\infty))$. We define the previous measure of $n$ in $S$
% as $l=\prev_S(n)$ where $l = \max(S(-\infty,T(n)))$.

% Let $S$ be a time series, $t$ be a time instant and $\delta$ be a
% time duration, then the time series' measures can be located in the
% time interval $i_0=[t, t+\delta]$ and its multiples $i_j=[t+j\delta,
% t+(j+1)\delta]$ for $j=0,1,2,\ldots$. When time series' measures are
% equally spaced we say it to be regular.
% \begin{definition}[Regular time series]
%   Let $S=\{m_0,$ $ldots,$ $m_k\}$ be a time series and $\delta$ a time
%   duration. $S$ is regular if and only if $\forall m \in
%   S(T(\min(S),+\infty):T(m) - T(\prev_S(m)) = \delta$.
% \end{definition}









\section{Multiresolution model}
\label{sec:MTSMS}

The \acro{MTSMS} are \acro{TSMS} that store time series with a lossy
compression approach, that is some information is selected and spread in
different time resolutions. The \acro{MTSMS} model is based on the
concepts of measures and time series as defined in
Section~\ref{sec:model:preliminaries}.


The multiresolution concept comes from thoroughly analysis of the
RRDtool \cite{rrdtool} \acro{TSMS}. Our objective is to formalise its
essential parts into an abstract model, where what we call
multiresolution plays a main role, and to include more genericity in
order to describe \acro{MTSMS} as fully \acro{TSMS}. Then we will be
able to apply these systems to other applications.



A \acro{MTSMS} stores multiresolution time series where each has a
multiresolution schema as shown in Figure~\ref{fig:model:mtsdb}. A
multiresolution time series is a collection of resolution subseries
which temporarily accumulate measures in a buffer in order to select
some information and finally store it in a disc. The information
selection process changes the time intervals between measures to
compact information by aggregating the time series attributes. 

\begin{figure}[tp]
  \centering
  \input{imatges/mtsms-arquitectura_interna.tex}
  \smallskip
  \caption{Architecture of \acro{MTSMS} model}
  \label{fig:model:mtsdb}
\end{figure}


In this way, the original time series gets stored spread in the discs,
each with a different time resolution and attribute aggregation.
Discs are size bounded so they only contain a fixed amount of
measures. When a disc becomes full it discards a measure. Thus,
multiresolution database is bounded in size and the time series gets
stored in pieces, that is time subseries.

Regarding to operations, \acro{MTSMS} structure needs operators to
change the time intervals between measures and to select
attributes. Mainly, these operators are measure additions and time
series consolidations which some functionality is delegated to operators called 
attribute aggregate functions.
 Most of these operators
are attribute aggregate functions and consolidation actions.
the operations to
create a multiresolution database, to add measures, and to consolidate
time series.
 Attribute aggregate functions are required but not linked
to the model.\todo{}

Following we define the \acro{MTSMS} model by: (i) four basic
structure model elements ---buffer, disc, resolution subserie, and
multiresolution time series--- with its structure operators, (ii) the
operations to change and consult a multiresolution schema, and (iii)
the attribute aggregate functions.



\subsection{Structure}



\subsection{operations}




\section{The proposed data model}



Regarding to operations, \acro{MTSMS} structure needs operators to
change the time intervals between measures. Most of these operators
are attribute aggregate functions and consolidation actions.

In what follows we describe the basic \acro{MTSMS} model centered in:
(i) the four basic data model elements ---buffer, disc, resolution
disc, and multiresolution database---, and (ii) the operations to
create a multiresolution database, to add measures, and to consolidate
time series. Attribute aggregate functions are required but not linked
to the model. They are defined in the
Section~\ref{sec:model:interpolador}.

A \emph{buffer} is a container for a regular or a no-regular time
series. The buffer objective is to regularise the time series using a
predetermined step and an attribute function. We name
\emph{consolidation} to this action.
\begin{definition}[Buffer]
  A \emph{buffer} is defined as the tuple $(S,\tau,\delta,f)$ where
  $S$ is a time series, $\tau$ is the last consolidation time,
  $\delta$ is the duration of the consolidation step and $f$ is an
  attribute aggregate function.

  An empty buffer $B_{\emptyset} = (\emptyset,t_0, \delta, f)$ has an
  empty time series, an initial consolidation time $t_0$ and
  predetermined $\delta$ and $f$. From the $B_{\emptyset}$ all the
  consolidation time instants can be calculated as $t_0+i\delta,
  i\in\mathbb{N}$.
\end{definition}

Operator \emph{addBuffer} adds a measure to its time series:
$\text{addBuffer}: B = (S,\tau,\delta,f) \times m \mapsto
(S',\tau,\delta,f)$ where $S' = S \cup \{m\} $.

A buffer is ready to consolidate when the time of some measure is
bigger than the buffer's next consolidation time.  Let
$B=(S,\tau,\delta,f)$ be a buffer and $m=\max(S)$ the maximum measure,
$B$ is ready to consolidate if and only if $T(m) \geq \tau+\delta$.
The consolidation of $B$ in the time interval $i=[\tau,\tau+\delta]$
results in a measure $m'=(v,\tau+\delta)$ where $m'=f(S,i)$ and $f$ is
an attribute aggregate function $f$. Operator \emph{consolidateBuffer}
consolidates a set of measures and removes the consolidated part of
the time series from the buffer. Usually consolidateBuffer is only
applied to the present consolidation interval and it is defined as
follows: $\text{consolidateBuffer}: B=(S,\tau,\delta,f) \mapsto B'
\times m' $ where $ B'= (S',\tau+\delta,\delta,f)$, $S' = S$ and $m' =
f(S,[\tau,\tau+\delta])$. When historic data is not needed anymore the
consolidated buffer measures can be removed applying $S' =
S(\tau+\delta,\infty)$.

A \emph{disc} is a finite capacity measures container. A time series
stored in a disc has its cardinal bounded. When the cardinal of the
time series is to overcome the limit, some measures need to be
discarded.
\begin{definition}[Disc]
  A \emph{disc} is a tuple $(S,k)$ where $S$ is a time series and
  $k\in\mathbb{N}$ is the maximum allowed cardinal of $S$.  An empty
  disc $D_{\emptyset} = (\emptyset,k)$ has an empty time series and
  the $k$ maximum cardinal allowed.
\end{definition}

The cardinal of the times series is kept under control by the add
operator, $\text{addDisc}:D=(S,k)\times m\mapsto (S',k)$ where 
$$
S' = \begin{cases}
  S\cup\{m\}                 & \text{if } |S|<k  \\
  (S-\{\min(S)\}) \cup \{m\} & \text{otherwise}
\end{cases}  
$$

A \emph{resolution disc} is a disc which stores a regular time
series. It is composed of a buffer, that contains the partial time
series to be regularised, and a disc, that contains the regularised
time series.
\begin{definition}[Resolution disc]
  A \emph{resolution disc} is a tuple $(B,D)$ where $B$ is a buffer
  and $D$ is a disc.  An empty buffer and empty disc imply an empty
  resolution disc $R_{\emptyset} = (B_{\emptyset},D_{\emptyset})$.
\end{definition}
 
The operators of a resolution disc extend the buffer and disc ones:
(i) The addition of a measure to the buffer of the resolution disc,
$\text{addRD}:R=(B,D) \times m \mapsto R'$ where $R'= (B',D)$, and
$B'= \text{addBuffer}(B,m)$; (ii) The consolidation of the resolution
disc by consolidating its buffer and adding the consolidation measure
to its disc, $\text{consolidateRD}:R=(B,D) \mapsto R'$ where $R'=
(B',D')$ and $(B',m') = \text{consolidateBuffer}(B)$ and $D'=
\text{addDisc}(B,m')$.
% \]

A \emph{multiresolution database} is a set of resolution discs which
share the input of measures, that is they store the same time
series. A time series is stored regularised and distributed with
different resolutions in the various resolution discs, as it was shown
in the Figure~\ref{fig:model:mtsdb}.
\begin{definition}[Multiresolution Database]
  A \emph{Multi\-re\-solution Database} is a set of resolution discs
  $M=\{R_0, \dots, R_d\}$.  An empty multiresolution database has
  empty resolution discs $M_{\emptyset}=\{R_{0_\emptyset}, \dots,
  R_{d_\emptyset}\}$.
\end{definition}

We define the addition of a measure to every resolution disc as
$\text{addMD} : M=\{R_0, \dots, R_d\} \times m \mapsto \{R'_0, \dots,
R'_d\}$ where $R'_i=\text{addRD}(R_i,m)$.

The consolidation of all resolution discs can be defined as follows:
$\text{consolidateMD}: M=\{R_0, \dots, R_d\} \mapsto \{R'_0, \dots,
R'_d\}$ where
$$ 
R'_i = \begin{cases}
  \text{consolidateRD}(R_i) & \text{if } R_i \text{ ready to consolidate} \\
  R_i                       & \text{otherwise}
\end{cases}
$$.


\subsection{Attribute aggregate function}
\label{sec:model:interpolador}

When a buffer is consolidated we summarise the time series information
using an attribute aggregate function.  Let $S$ be a time series and
$t_0$ and $t_f$ two time instants, an attribute aggregate function $f$
calculates a measure that summarises the measures of $S$ included in
the time interval $i=[T_0,T_f]$:
\begin{align*}
f&:S=\{m_0,\ldots,m_k\} \times [T_0,T_f] \mapsto m'
\end{align*}

To summarise a time series we can use different attribute aggregate
functions.  For instance, we can calculate an statistic indicator of
the time series such as the average or we can apply a more complex
digital signal processing operation, \cite{zhang11}.

Below there are some examples. Let $S'=S(T_0,T_f]$. Then:
\begin{itemize}
\renewcommand{\labelitemi}{--}
\item maximum$^d$: $S \times i \mapsto m'$ where $V(m') =
  \max_{\forall m \in S'}(V(m))$. It summarises $S'$ with the maximum
  of the measure values.
\item last$^d$: $S \times i \mapsto m'$ where $V(m') = \max(S')$. It
  summarises $S'$ with the maximum measure.
\item arithmetic mean$^d$: $S \times i \mapsto m'$ where $V(m') =
  \frac{1}{|S'|} \sum\limits_{\forall m\in S'} V(m)$. It
  summarises $S'$ with the mean of the measure values.
\end{itemize}

% With reference to data validation, attribute aggregate functions
% can cope with this process. When data has not been captured or has
% been captured erroneously, it must be treated as unknown data.
% \begin{itemize}
% \item When data has not been captured it is unknown by nature. For
%   example, we try to capture data from a sensor and there is no
%   response.
% \item When data is erroneously it must be marked as unknown. For
%   example, we capture data from a sensor but it responses in a not
%   reasonable time or we capture data that is clearly outside a
%   reasonable limits.
% \end{itemize}
% As a consequence, attribute aggregate functions deals with these two
% subprocesses: treating unknown data and marking data as
% unknown. Following with real numbers example, we extend the
% domain with a value that means 'unknown', let this unknown value be
% represented by the improper element infinity ($\infty$).

% An attribute aggregate functions treating unknown
% data is a one that can calculate a result when there are unknown
% values in the original time series, $f^u: S \times i \mapsto m'$ where
% $\exists m \in S: V(m)=\infty$. Although from a strict point of view
% operating with unknown data makes unknown result, aggregate functions
% are free to calculate whatever is needed such as time series analysis
% does with data reconstruction.

% For example, arithmetic mean$^{d}$ aggregate function returns
% $V(m')=\infty$ if $\exists m \in S: V(m)=\infty$.  We can define a new
% mean function, based on the original arithmetic mean$^{d}$ aggregate,
% that naively treats unknown values by keeping the
% known mean; in other words, it ignores unknown values found in the time
% interval: arithmetic mean$^{du}$: $S \times i \mapsto m'$ where $m' =
% \text{arithmetic mean}^{d}(S'',i)$ and $S''= \{m''\in S':V(m'')\neq
% \infty\}$.
% % ignore$^{u}$: $S \mapsto S'$ where $S'= \{m''\in S':V(m'')\neq
% % \infty\}$,
% % arithmetic mean$^{du}$: $S \times i \mapsto m'$ where $m' =
% % \text{arithmetic mean}^{d}(\text{ignore}^u(S),i)$.

% An attribute aggregate functions marking data as unknown is a one
% that can give unknown value as the resulting measure's value, $f^{mu}:
% S \times i \mapsto m'$ where $V(m')\in \mathbb{R}\cup\{\infty\}$.

% For example, we can define a maximum aggregate, based on the
% maximum$^d$ aggregate, that returns unknown if there is a
% measure's value bigger than 2:  maximum$^{dmu2}$: $S \times i
% \mapsto m'$ where $V(m') = 
% \begin{cases}
%   \infty &\text{if }  m''>2\\
%   m'' & \text{else }
% \end{cases}$ and $m''=\text{maximum}^d(S,i)$.

% %Per exemple definim un termini, si les dades estan més espaiades que 2 es marca com a desconeguda

In the design of the attribute aggregate function we can interpret a
time series in different ways, that is what we call the representation
of a time series. \citeauthor{last:keogh}, \cite{last:keogh}, cite
some possible representations for time series such as Fourier
transforms, wavelets, symbolic mappings or piecewise linear
representation. The last one is very usual due to its simplicity,
\cite{keogh01}.

Time series representations can be taken into account when computing
with the measures of the time series.  For example, a maximum
attribute aggregate function may give different values if we consider
a linear or a constant piecewise representation.

Following we show a possible family of attribute aggregate functions
for time series represented by a staircase function, that is with a
piecewise constant representation.  We define a new representation for
time series named \emph{zero-order hold backwards} (zohe). This
representation holds back each value until the preceding value. 
RRDtool, \cite{lisa98:oetiker}, has a similar aggregate function.

Let $S=\{m_0,\ldots,m_k\}$ be a time series, we define
$S(t)^{\text{zohe}}$ as its continuous representation along time $t$:
$\forall t \in \mathbb{R} ,\forall m \in S:$
\begin{equation}
 S(t)^{\text{zohe}} =  
\begin{cases}
  \infty & \text{if } t > T(\max S) \\
  V(m)   & \text{if } t\in (T(\prev_S m),T(m)]
\end{cases}
\label{eq:zohe}
\end{equation}


In conclusion, we can define many attribute aggregate functions and
thus no global assumptions can be made about them. Each user has to
decide which combination of aggregation and representation fits better
with the measured phenomena.  Therefore, \acro{MTSMS} must allow to
define user aggregate functions.







%%% Local Variables:
%%% TeX-master: "main"
%%% ispell-local-dictionary: "british"
%%% End:

% LocalWords:  genericity





\section{ALTRES}



\section{Introduction}

This paper focuses on Data Base Management Systems (DBMS) that store
and treat data as time series. Traditional DBMS, as is ones derived
from relational model, are not adequate for these cases as they do not
have enough facilities to manage and retrieve time series
information.



* Temporal databases. Basades en esdeveniments. Data mining basat en sèries temporals definides per parelles temps-valor; calen TSMS

* Alta dimensió sèries temporals, cal reduir-la. Es conserven els segments de temps més interessants; multiresolució

* Multiresolució, diferents resolucions, es pot treballar amb més o menys dades segons convingui

*Cal saber canviar de resolució, exemple transformar dades periòdiques d'un mes a un any.

* Aggregates, una sèrie temporal pot estar mostrant diferent informació. ex: mitjana, màxim, valor al final del període, ...

* Les sèries temmporals tenen una metainformació que cal guardar en una base de dades relacional (localització, etiquetes de classificació, últim valor mesurat, unitats, etc.)

* Disseny del model de TSMS, aleshores veurem si una TSMS pot ser implementada com a camp d'una altra DBMS o si els DBMS no són capaços de manipular TS adequadament i cal implementar TSMS específics.

* Calendari, passa a segon terme. El temps es defineix com universal i constant (semblant a Unix Time Epoch). Aquests temps es pot convertir a calendari. Cal definir la interacció usuari/calendari amb temps universal.

* El temps és un nom donat al camp, qualsevol objecte que tingui la mateixa interfície que el temps pot funcionar. En el cas del valor pot ser qualsevol objecte, s'exemplifica amb reals per facilitar-ne la comprensió i per ser el més proper al time series analysis: statistical methods focused on sequences of values representing a single numeric variable [llibre-last].


* Representació: Entre dos punts de mesura, quin valor pren la sèrie temporal?.





 P.ex. a la Tesis Soriguera, APPENDIX A2 , 
Requiem for Freeway Travel Time Estimation Methods Based
on Blind Speed Interpolations between Point Measurements, parla d'interpolació entre mesures (compte, EN l'ESPAI NO EN EL TEMPS!! però seria un exemple de que els links compleixen amb la interfície temps?) amb diferents mètodes:  interpolacions constants (endavant, endarrere, optimistica), a trossos, lineal, quadràtica;    tot i que conclou que The present paper shows that all speed interpolation methods that omit traffic
dynamics and queue evolution do not contribute to better travel time estimations. 
%http://www.fundacioabertis.org/pdf/Editarium_FSoriguera.pdf
Unlike in fusion 1, in the fusion 2 process the information is not provided by the same
data source, and hence, the data will not be equally located in space and time. 
Therefore, a spatial and temporal alignment is needed before the data can be fused.


* Xarxa de sensors, tsms distribuïda. Sensor dades recents, màquina grossa històrics. Quan es llança una consulta, es llança distribuïdament: si es té prou resolució es respon sinó s'envia la consulta al sensor. (consultar si en deligiannakis en diu alguna cosa).

%Multi-Sensor Data Fusion: An Introduction. Harvey B. Mitchell


\newpage

\mbox{}

\newpage





\section{Conclusion}
The conclusion goes here.


\section*{Acknowledgment}


The authors would like to thank...






\end{document}




%%% Local Variables: 
%%% ispell-local-dictionary: "british"
%%% End: 
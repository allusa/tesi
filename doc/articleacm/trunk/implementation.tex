
\section{Implementation}

\todo{}

We implement the TSMS and MTSMS models into three different approaches:

\begin{itemize}
\item Pytsms+RoundRobinson, a Python implementation. This is our referent
  implementation, which we use for example data.
\item Reltsms, a Tutorial D implementation. This shows a TSMS
  implemented on a relational language.
\item Roundrobindoop, a MapReduce implementation. This is a specific
  implementation for offline computing multiresolution with
  parallelism approaches.
\end{itemize}

RRDtool can also be seen as a MTSMS implementation in a specific
field. We give more references for RRDtool in
Section~\ref{sec:related-work}.



\subsection{Python referent implementation}

We implement the two models of TSMS and MTSMS respectively as two
separated Python libraries: \emph{Pytsms} and \emph{RoundRobinson}.
RoundRobinson has a strong dependency on Pytsms following the MTSMS
being defined based on TSMS.  It is mainly a referent implementation
so it has fidelity to the algebraic model defined but has not extended
\acro{DBMS} capabilities, such would be query optimisation or
transaction management.

We design the implementation concepts with object orientation, so that
there is a clear mapping between model and implementation
objects. Next we use Unified Modeling Language (UML) diagrams in order
to define the classes structure, mainly to show the relationships
among objects.



\todo{}


Pytsms is the referent implementation for the model concepts of
measure, time series and temporal representation function.  Figure
\ref{fig:implementacio:pytsms-uml} shows the relationships among these
objects in a UML diagram. A \emph{TimeSeries} object is an aggregation
of \emph{Measure} objects. TimeSeries and \emph{Representation}
objects are associated, that is each TimeSeries has a default
representation and a Representation operates over a TimeSeries.

\todo{figura}



A TimeSeries object has a huge amount of methods, we classify them
based on their functionality. Firstly, a TimeSeries has methods that
manipulate its structural model, a TimeSeries is a subclass of the
predefined Set Python type. Secondly, there are methods for the
operational model which are classified into set, sequence and temporal
function operators.  Set operators are composed of the partial order, the
temporal order and the relational set operators.




\subsection{Reltsms}

\todo{}


\subsection{Roundrobindoop}

\todo{}



%%% Local Variables:
%%% TeX-master: "main"
%%% ispell-local-dictionary: "british"
%%% End:

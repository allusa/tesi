

\section{Related work}
\label{sec:related-work}


Bitemporal \acro{DBMS}, sometimes referred directly as temporal data,
is another database field related with time. Bitemporal data target is
to keep historical events in the database by associating time
intervals to data.  Bitemporal data and time series data are not
exactly the same and so can not be treated interchangeably
\cite{schmidt95}. However, there are some similarities that can be
considered.  The recent bitemporal data research in \acro{DBMS} model
terms \cite{jensen99:temporaldata,date02:_tempor_data_relat_model}
marks a promising foundation. It models bitemporal data as relations
extended with time intervals attributes and extends relational
operations in order to deal with related time aspects.  Therefore, we
formalise time series similarly as how bitemporal data is formalised
for relational DBMS. Moreover, some bitemporal time concepts might be
taken into account by TSMS, such as the discussions about time
granularities.


\acro{TSMS} are treated as a particular \acro{DBMS} field.  Segev and
Shoshani \cite{segev87:sigmod} propose an structured language for
querying \acro{TSMS}. Their time series structures include the notion
of regularity and representation in time series and their operations
are SQL-like.  Dreyer et al.\ \cite{dreyer94} propose the requirements
of special purpose TSMS and they base the model on four basic
structural elements: events, time series, groups, metadata and the
time series bases. They implement a special purpose \acro{TSMS} called
\emph{Calanda} which has calendar operations, can group time series
and make simple queries. They exemplify it with financial data. In
\cite{schmidt95} \emph{Calanda} is compared with temporal systems
designed for time series.




As \acro{TSMS} suffer from problematic properties of time
series, like the ones we have described in
Section~\ref{sec:model:properties} mainly the huge data volume,
compression techniques are used.  Next, we summarise some current work
in \acro{TSMS} with compression.



\emph{RRDtool} from Oetiker, \cite{rrdtool,lisa98:oetiker}, is a free
software database management system. It is designed to be used for
monitoring systems. Because of this, it is focused to a particular
kind of data, gauges and counters, and it lacks general time series
operations. \emph{RRDtool} can store multiple time resolution data,
however Plonka et al.\ \cite{lisa07:plonka} evaluated \emph{RRDtool}
performance and found a limitation for storing huge number of
different time series. They propose a caching system on top of
\emph{RRDtool} as a solution.  \emph{RRDtool} is extremely used by the
free software community so it inspired us to develop a model from its
main characteristics, that is now what we call multiresolution. A
similar approach is done by \cite{weigel10} in a system called
\emph{TSDS} that caches queries by aggregate parameters. They notice
the necessity to show the data over its full time range and not only
subsets of data as it is usually provided.  They develop the software
package \emph{TSDS} where time series are stored fully and then
requested by date ranges or by applying different filters and
operations to the time series data.  Our MTSMS model is a generic
approach to the multiresolution features, we define it open so that
users can define any attribute aggregate functions.


Deri et al.\ \cite{deri12:tsdb_compressed_database} present
\emph{Tsdb}, a lossless compression storage \acro{TSMS} for time
series that share the same time instants of acquisition. Different
time series are stored grouped by new measures addition instead of
each time series isolated.  They compare \emph{Tsdb} to \emph{RRDtool} and
with a relational product. As a consequence of \emph{Tsdb} structure,
they achieve a better measure addition time but a worse global
retrieval time as data has to be contiguously regrouped. However, when
measures have same time this is seen as the same time series in a
MTSMS, so it would be interesting to use this implementation
architecture of shared time arrays in MTSMS for resolution subseries
with same delta in order to achieve better performance requirements
when having much equal acquired time series.

% Therefore, as it is a very different approach from RRDtool we find it difficult to compare both in these performance requirements. 
% He only evaluates compression performance but we think that a global
% evaluation of compression plus descompression must be done. MTSMS
% are not aimed to be a replacement for offline long time storage
% systems that are rarely queried and then queries can be slow but
% have tot be exact. MTSMS are adequate to stream processing and
% resolving queries with less data than original and so they are
% quicker but give information previously selected.


There are other lossy compression techniques for time series devoted
to the optimal approximation representation, that is finding the
compromise between least data that can reconstruct the original signal
with least error. Keogh et al.\ \cite{keogh01} cite some possible
approximation representations for time series such as Fourier
transforms, wavelets, symbolic mappings or piecewise linear
representation. They remark this last one as very usual due to its
simplicity and develop a system called \emph{iSAX}
\cite{keogh08:isax,keogh10:isax} in order to analyse and index massive
collections of time series. They describe that the main problem is in
the indexing of time series and they propose methods that process
efficiently. The first method proposed is based on a constant
piecewise approximation. The time series representation obtained with
\emph{iSAX} allows to reduce the stored space and to index faster and
with the same quality as other more complex representation methods.
These techniches of compression are candidates for being used as
attribute aggregate functions in the MTSMS model, as instance it
would be interesting to define aggregations in the frequency domain of
time series.


% A Multiresolution Symbolic Representation of
% Time Series; Megalooikonomou, Faloutsos; 2005 proposes multiresolution by decomposing a signal in frequency subsequences and intended mainly to similarity searching. The objective is to reconstruct the original time series. 
 


% \paragraph{T-Time}  \textcite{assfalg08:thesis} shows a TSMS that can do similarity search, which is calculated as distances between time series. Mainly, two time series are marked as similar if they distance is less than a threshold in each interval. From this method efficient algorithms are developed and implemented in a program called T-Time, which is described in \cite{assfalg08:ttime}.

 

 
Others implement \acro{TSMS} with array database approaches.
\emph{SciDB} \cite{stonebraker09:scidb} and \emph{SciQL}
\cite{zhang11} are array database systems intended for science
applications, in which time series play a principal role. They
structure time series into arrays in order to achieve multidimensional
analysis and allow tables to store other data.  \emph{SciDB} is based
on arrays which, according to the authors, allow to represent time
series. However, it does not consider time series special needs: it
does not care for managing continuously voluminous data neither for
achieving temporal coherence.  In contrast, \emph{SciQL} defines time
series as a mixture of array, set, and sequence properties and
exhibits some time series managing characteristics that include time
series regularities, interpolation or correlation queries.  However,
difference between tables and arrays seems too physical and leads to
ambiguity when representing time series.  Our TSMS model proposes time
series as firmly based on relational algebra, clarifying this
ambiguity and describing them coherently in terms of information
systems theory.




There are other \acro{TSMS} specifically designed for a particular
field requirements.  \emph{Cougar} \cite{bonnet01} is a sensor
database system. It has two main structures: one for sensor properties
stored into relational tables and another for time series stored into
data sequences from sensors. Time series have specific operations and
can combine relations and sequences. \emph{Cougar} target field is
sensor networks, where data is stored distributed. Queries are
resolved combining sensor data in a data stream abstraction that
improves processing performance.  The MTSMS model can also be thought
as an stream processor if measures are added in time order and there
is no possibility of update operations, as instance RRDtool is stream
oriented and the consolidation process is done at the same time of
inserting new measures.





\section{Conclusions}
\label{sec:concl-future-work}


In this paper we have shown a \acro{MTSMS} model, including a
motivation example for multiresolution and an application together
with \acro{TSMS}. Our \acro{MTSMS} model is based on \acro{TSMS}
notation which we have described firmly rooted on set and relational
algebra. We have gone a bit further and proposed \acro{TSMS} including
set, sequence and temporal function behaviour.



The main objective of a \acro{MTSMS} is to store compactly a time
series and manage consistently its temporal dimension.  It stores
multiresolution time series, that is time series split into time
subseries called resolution subseries.  Each resolution subseries has
a different resolution and is compacted with an attribute aggregate
function. Therefore, each multiresolution time series is configured by
the quantity of resolution subseries and four parameters for each: the
consolidation step, the initial consolidation time, the attribute
aggregate function, and the capacity.  These configuration parameters
are degrees of freedom for each application. Giving different
values a multiresolution database is capable to keep the desired
information from a time series.



The queries over \acro{MTSMS} obtain time series from stored
multiresolution time series. In this way \acro{TSMS} operators can be
applied if needed.


Compared to other \acro{TSMS} we propose a compression solution that
stores only the information we will require by latter queries or by
human visualisation, instead of trying to reconstruct the original
signal.  Moreover, our multiresolution solution copes well with
typical problematic properties of time series: regularity, data
validation and data volume.  The decompression time is minimal as data
in discs get stored directly as a time series. As a consequence, the
queries or visualisation computing time is only due to the computation
itself. Moreover, if the query is an aggregation or resolution already
computed in \acro{MTSMS} consolidation, then the visualisation is
immediate.


\acro{MTSMS} imply a data information selection and so the information
not considered important is discarded. When this is not possible, we
have showed a dual structure of \acro{TSMS} and \acro{MTSMS}. Then a
\acro{TSMS} stores losslessly and a \acro{MTSMS} takes advantages of
manipulating data in time order in order to achieve pre-computed
queries in a stream-like orientation.  In future work, information
theory has to be evaluated for multiresolution schemes. Multimedia
lossy compression techniques are well founded on information theory
and similar approaches could be taken for multiresolution time series,
such as evaluating whether a human can visualise original qualities in
the multiresoluted time series or evaluating
whether given a query it has the same validity for a multiresoluted
one as it has for the original.




% Tenim una consulta sobre els TSMS M=multiresolution(S) que ens permetrà estudiar que passa amb la multiresolució aplicada a una sèrie temporal, per exemple quina diferència hi ha entre fer S1+S2 i multiresolution(S1)+multiresolution(S2).
% Tenim un model MTSMS que dóna pistes de com implementar la multiresolució, de l'estructura que ha de tenir.


%Ens podem considerar un DBMS NoSQL? 
% We are working with a DBMS that could be considered as a NoSQL product but we formalise and root it heavily with traditional algebra theory.

A \acro{MTSMS} could be implemented as a SQL \acro{DBMS} system or as
a NoSQL one. As a referent implementation we have developed a
\emph{Python} package centred on the basic algebra, that is without
extended \acro{DBMS} capabilities. Regarding other implementations,
\emph{RRDtool} can be seen as an specific case of \acro{MTSMS} and as
a NoSQL system, although Oetiker \cite{rrdtool} has not commented
whether it can be considered as being NoSQL. However, regardless of
the implementation backend, we have shown how a generic model for
\acro{MTSMS} can be defined firmly rooted on \acro{DBMS} algebra
theory. 




% Concluding, in this paper we show that using \acro{TSMS} facilitates
% substantially time series management. The current field interest makes
% us optimistic to expect soon an adequate management in \acro{DBMS}.







%%% Local Variables:
%%% TeX-master: "main"
%%% ispell-local-dictionary: "british"
%%% End:


%  LocalWords:  multiresolution



\section{Related work}
\label{sec:related-work}

Bitemporal \acro{DBMS}, sometimes referred directly as temporal data,
is another database field related with time. Bitemporal data target is
to keep historical events in the database by associating time
intervals to data.  Bitemporal data and time series data are not
exactly the same and so can not be treated interchangeably
\cite{schmidt95}. However, there are some similarities that can be
considered.  The recent bitemporal data research in \acro{DBMS} model
terms \cite{jensen99:temporaldata,date02:_tempor_data_relat_model}
marks a promising foundation. It models bitemporal data as relations
extended with time intervals attributes and extends relational
operations in order to deal with related time aspects.


Therefore \acro{TSMS} are treated as a particular \acro{DBMS}
field.  Segev and Shoshani \cite{segev87:sigmod} propose an structured
language for querying \acro{TSMS}. Their time series structures
include the notion of regularity and representation in time series and
their operations are SQL-like.  Dreyer et al.\ \cite{dreyer94} propose
the requirements of special purpose TSMS and they base the model on
four basic structural elements: events, time series, groups, metadata
and the time series bases. They implement a special purpose
\acro{TSMS} called \emph{Calanda} which has calendar operations, can
group time series and make simple queries. They exemplify it with
financial data. In \cite{schmidt95} \emph{Calanda} is compared with
temporal databases systems designed for time series.




As \acro{TSMS} suffer from problematic properties of time
series, like the ones we have described in
Section~\ref{sec:model:properties} mainly the huge data volume,
compression techniques are used.  Next, we summarise some current work
in \acro{TSMS} with compression.



\emph{RRDtool} from Oetiker, \cite{rrdtool,lisa98:oetiker}, is a free
software database management system. It is designed to be used for
monitoring systems. Because of this, it is focused to a particular
kind of data, gauges and counters, and it lacks general time series
operations. \emph{RRDtool} can store multiple time resolution data,
however Plonka et al.\ \cite{lisa07:plonka} evaluated \emph{RRDtool}
performance and found a limitation for storing huge number of
different time series. They propose a caching system on top of
\emph{RRDtool} as a solution.  \emph{RRDtool} is extremely used by the
free software community so it inspired us to develop a model from its
main characteristics, that is now what we call multiresolution. A
similar approach is done by \cite{weigel10} in a system called
\emph{TSDS} that caches queries by aggregate parameters. They notice
the necessity to show the data over its full time range and not only
subsets of data as it is usually provided.  They develop the software
package \emph{TSDS} where time series are stored fully and then
requested by date ranges or by applying different filters and
operations to the time series data.


Deri et al.\ present \emph{tsdb}, a lossless compression storage \acro{TSMS}
for time series that share the same time instants of
acquisition. Different time series are stored grouped by new measures
addition instead of each time series isolated.  They compare
\emph{tsdb} to RRDtool and with a relational product. As a consequence
of \emph{tsdb} structure, they achieve a better measure addition time
but a worse global retrieval time as data has to be contiguously
regrouped. However, when measures have same time this is seen as the
same time series in a MTSMS, so it would be interesting to use this
implementation architecture of shared time arrays in MTSMS for
resolution subseries with same delta in order to achieve better
performance requirements when having much equal acquired time series.

% Therefore, as it is a very different approach from RRDtool we find it difficult to compare both in these performance requirements. 
% He only evaluates compression performance but we think that a global
% evaluation of compression plus descompression must be done. MTSMS
% are not aimed to be a replacement for offline long time storage
% systems that are rarely queried and then queries can be slow but
% have tot be exact. MTSMS are adequate to stream processing and
% resolving queries with less data than original and so they are
% quicker but give information previously selected.


There are other lossy compression techniques for time series devoted
to the optimal approximation representation, that is finding the
compromise between least data that can reconstruct the original signal
with least error. Keogh et al.\ \cite{keogh01} cite some possible
approximation representations for time series such as Fourier
transforms, wavelets, symbolic mappings or piecewise linear
representation. They remark this last one as very usual due to its
simplicity and develop a system called \emph{iSAX}
\cite{keogh08:isax,keogh10:isax} in order to analyse and index massive
collections of time series. They describe that the main problem is in
the indexing of time series and they propose methods that process
efficiently. The first method proposed is based on the constant
piecewise approximation, the PAA. The time
series representation obtained with \emph{iSAX} allows to reduce the
stored space and to index faster and with the same quality as other
more complex representation methods.


% A Multiresolution Symbolic Representation of
% Time Series; Megalooikonomou, Faloutsos; 2005 proposes multiresolution by decomposing a signal in frequency subsequences and intended mainly to similarity searching. The objective is to reconstruct the original time series. 
 


% \paragraph{T-Time}  \textcite{assfalg08:thesis} shows a TSMS that can do similarity search, which is calculated as distances between time series. Mainly, two time series are marked as similar if they distance is less than a threshold in each interval. From this method efficient algorithms are developed and implemented in a program called T-Time, which is described in \cite{assfalg08:ttime}.

 

 
Others implement \acro{TSMS} with array database approaches.
\emph{SciDB} \cite{stonebraker09:scidb} and \emph{SciQL}
\cite{zhang11} are array database systems intended for science
applications, in which time series play a principal role. They
structure time series into arrays in order to achieve multidimensional
analysis and allow tables to store other data.  \emph{SciDB} is based
on arrays which, according to the authors, allow to represent time
series. However, it does not consider time series special needs: it
does not care for managing continuously voluminous data neither for
achieving temporal coherence.  In contrast, \emph{SciQL} defines time
series as a mixture of array, set, and sequence properties and
exhibits some time series managing characteristics that include time
series regularities, interpolation or correlation queries.  However,
difference between tables and arrays seems too physical and leads to
ambiguity when representing time series.




There are other \acro{TSMS} specifically designed for a particular
field requirements.  \emph{Cougar}
\cite{bonnet01} is a sensor database system. It has two main
structures: one for sensor properties stored into relational tables
and another for time series stored into data sequences from
sensors. Time series have specific operations and can combine
relations and sequences. \emph{Cougar} target field is sensor networks, where
data is stored distributed in sensors. Queries are resolved combining
sensor data using a data stream abstraction that improves processing
performance.






\section{Conclusions}
\label{sec:concl-future-work}

\todo{}


Other TSMS solves the problem for finding compression storage that
best approximates the original time series. We propose a solution that
stores only the information we will require by latter queries or by
human visualisation.  Moreover, our multiresolution solution copes
well with typical problematic properties we have found in time series:
regularity, data validation and data volume.

We then have showed a structure for manipulating in time order as then there are no updates in data and it can be managed more simpler. 

The decompression time is minimal as data in discs get stored directly
as a time series. Therefore, the queries or visualisation computing
time is only due to the computation itself. Moreover, if the query is
an aggregation and resolution already computed in MTSMS consolidation
then the visualisation is immediate.


Our \acro{MTSMS} is based on \acro{TSMS} notation which we have
described firmly rooted on set and relational algebra. We have gone a
bit further and proposed \acro{TSMS} including set, sequence and
temporal function behaviour.



% A multiresolution database is an storage system for one time series, that
% is a collection of data measured in different instants in time.  The
% time series is compactly stored in the database as has been shown in
% figure \ref{fig:model:mtsdb}. The principal part of a multiresolution
% database is the set of resolution discs where the time series is
% stored distributed by the different interpolation functions and
% sampling periods. Each resolution disc uses its buffer to interpolate
% the measures and uses its disc to consolidate the result. 

% This system allows to reduce the storage space and the analysis time
% needed for large collections of times series.  With the compactness of
% the time series we have in mind the scalability for large monitoring
% systems.

% Interpolation functions and sampling periods are degrees
% of freedom for each application. Giving different values a multiresolution
% database is capable to keep the desired information from a time series.


% \section{Future work}\label{sec:future}

% In future work the retrieval operations will be defined. Mainly, how a
% time series can be restored from joining the information stored
% distributed in the resolution discs. And it also will be important
% how data fusion can be obtained from two time series databases.

% Then it will be possible to check the database management system with
% experimental data such as the proposed by \cite{keogh02}.

% The MTSMS imply a data information selection and the information not considered important is discarded.  Therefore, this systems are not adequate when all the monitored data must be kept as acquired. This happens either when it is not known a priori which interpolation functions will work better with the future data monitored or when detailed questions must be retrieved such as at what hour exactly an event triggered. This may be overcome with dual DBMS: one a TSMS for the common information retrieval and the other a large size database (VLDB) that is only consulted in occasional cases. 
% However, the applications when a TSMS is not appropriate are to be better understood. 



% In this paper we have shown a \acro{MTSMS} model, including the
% requirements for these special systems and how they can be applied to
% an example time series. The main objective is to store compactly a
% time series and manage consistently its temporal dimension.

% Our \acro{MTSMS} model proposes to store a time series split into time
% subseries, which we call resolution discs.  Each resolution disc has a
% different resolution and is compacted with an attribute aggregate
% function. Therefore, in a multiresolution database the configuration
% parameters are the quantity of resolution discs and each of their
% three parameters: the consolidation step, the attribute aggregate
% function and the capacity.

% The data model shown is the first step to develop a complete model for
% a \acro{MTSMS}. In the future the operations will be defined. In this
% context, there is a need for a model collecting generic properties for
% the \acro{TSMS}, as it can be the time series union operation or the
% time interval operations. Then, the multiresolution model would be
% build upon the generic \acro{TSMS} model.

% Concluding, in this paper we show that using \acro{TSMS} facilitates
% substantially time series management. The current field interest makes
% us optimistic to expect soon an adequate management in \acro{DBMS}.


% Tenim una consulta sobre els TSMS M=multiresolution(S) que ens permetrà estudiar que passa amb la multiresolució aplicada a una sèrie temporal, per exemple quina diferència hi ha entre fer S1+S2 i multiresolution(S1)+multiresolution(S2).
% Tenim un model MTSMS que dóna pistes de com implementar la multiresolució, de l'estructura que ha de tenir.



%Ens podem considerar un DBMS NoSQL? 

% We are working with a DBMS that could be considered as a NoSQL product but we formalise and root it heavily with traditional algebra theory.



\section{Acknowledgements}

The research presented in this paper has been supported by Spanish
research project \textsc{nasp (TEC2012-35571)}, \textsc{sherecs
  (dpi2011-26243)}, the UE project i-Sense ({\small FP7-ICT-270428}),
and Universitat Polit\`{e}cnica de Catalunya predoctoral grant.






%%% Local Variables:
%%% TeX-master: "main"
%%% ispell-local-dictionary: "british"
%%% End:


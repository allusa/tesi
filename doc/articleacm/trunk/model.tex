

\section{Time series preliminaries}
\label{sec:model:preliminaries}


In this section we introduce some background concepts and the
nomenclature which we will use later concerning TSMS. Measures and
time series are the main objects of TSMS and their main characteristic
is that they have a time attribute that requires a coherent
treatment. We describe the TSMS model in three parts:

\begin{itemize}
\item The data structure model of measures and time series.
\item The manipulation and operation of time series.
\item Specific properties of time series.
\end{itemize}



\subsection{Structure}

A \emph{time series} is a relation of times with values. Each pair of
time-value is called \emph{measure}. Then, a time series is a set of
measures. 

Firstly we address time and value concepts and secondly we define formally
measures and time series.


\subsubsection{Time}

Time is an attribute that indicates order between measures. We define
$T$ as the \emph{time domain}. $T$ can be either a finite or an infinite set
and normally it will be a closed set. As an example for the time
domain $T$ we use the affinely extended real numbers $\Rb \in \R \cup
\{+\infty,-\infty\}$ \cite{cantrell:extendedreal}.

The real numbers set is a metric space as it has a distance function
(or metric). Consequently, we can differentiate between time instants
(the elements of the set) and durations (the metric). Time can be
defined as a coordinate system
\cite{iep:time-supplement,kopetz11:realtime} by noting time instants
as points in the real line, durations as segments of the real line,
and specifying a time instant as a reference system. Next, we define
time in order we can (i) order events, (ii) measure events durations,
and (iii) position when events happen. This is a naive approximation
without complex details of the time concept.

\begin{definition}(Time)
  \label{def:model:temps}
  Let $t^i_i$ and $t^i_j$ be two \emph{time instants} with same $t^R$
  as \emph{reference time}, we define the \emph{quantity} or
  \emph{duration of time} $t^d$ as a value $t^d \in\Rb$ which measures
  the distance in time units between two time instants $t^d =
  d(t^i_i,t^i_j)$ where $d$ is the metric of the set $T$. When time
  instants are also real numbers $t^i_i , t^i_j \in \Rb$, then $t^d =
  t^i_i - t^i_j$.

  Let $T$ be the domain for time, we define a \emph{time instant} $I$
  as an element of the set $I \in T$. Following with the definition of
  a coordinate system and real numbers as time domain, let $t^{R}$ be
  a \emph{reference time instant}, then we define \emph{time instants}
  as a value $t^i \in\Rb$ that measures signed time distance from the
  \emph{reference time instant} $t^i= d(t^{R},I)$ where $d$ is the
  metric of the set $T$. $\square$
\end{definition}

Summarising, time instants are a sequence of real values which
position and order events, moreover between two time instants there is
a duration.  There are several time standards \cite{allen:timescales}
which specify how time duration must be measured and how time instants
positions must be noted. 

Time standards deal with a similar concept of
time coordinate system which we have illustrated with real
numbers. There is also a calendar approach which define the time
domain with names for the points in the time line and with rules for
time durations. Usually the aim of calendars is to give a relation
between time and Earth rotation. However, we do not see calendars as
essential part of TSMS \cite{dreyer94} as maps can be established
between a time coordinate system and a calendar system.


\subsubsection{Value}

Value is an attribute that indicates the magnitude of measures. The
domain for values can be any data type, that is an object that belongs
to a determined set of values and that has operations associated.
This way valid examples for values are integers, real numbers,
strings, and more elaborated data structures such as arrays, lists, or
even other time series.

%For simplicity
In accordance to the time examples, for the value domain we use
projectively extended real numbers $\R* \in \R \cup \{\infty\}$.  This
example is for scalar values but can be extended to the concept of
multivalues $\R*^n$, which represent a collection of values measured
at the same time instant \cite{assfalg08:thesis}.




\subsubsection{Measure}


\begin{definition}
  A \emph{measure} is defined as a tuple $(t,v)$ where $v$ is the
  value of the measure and $t$ is the time instant when the measure
  has been acquired.
\end{definition}

Let $m = (t,v)$ be a measure, $v$ is written as $V(m)$ and $t$ is
written as $T(m)$.

The time attribute defines the canonical order relation between
measures. If only time is concerned then measures have a \emph{total
  order}, on the contrary they have a \emph{partial order}. Let $m =
(t_m, v_m)$ and $n = (t_n, v_n)$ be two measures, there is a total
order $m\geq n$ if and only if $t_m\geq t_n$ and there is a partial
order $m > n$ if and only if $t_m > t_n$ and $m_n = n_n$ only if
$(t_m, v_m) = (t_n, v_n)$.  We can call \emph{temporal order} to the
total order as it only considers the time attribute in measures.





\subsubsection{Time series}

Time series are sets of measures of the same phenomena that are
ordered in time.  Sometimes they are also called time sequences
\cite{last:hetland}.


\begin{definition}
  A \emph{time series} $S$ is a a set of measures $S = \{m_0, \ldots,
  m_k\}$ without repeated time values $\forall i,j: i\leq k, j\leq k,
  i\neq j : T(m_i)\neq T(m_j)$.
\end{definition}

As there are no repeated time values, measures in a time series have a
total order.

Given a time series $S$, we note its cardinality $|S|=k+1$.  A time
series without measures is the empty time series
$S_\emptyset=\emptyset=\{\}$ and has zero cardinality
$|S_\emptyset|=0$.  Observe that, because measures in $S$ are of the
same phenomena, the type of $S$ values is homogeneous.



A time series can record more than one phenomena if they share the
time instants of acquisition. Then we have a multivalued time series,
which can be expressed by two forms. Let $S = \{m_0, \ldots, m_k\}$ be
a time series, in one form we write measures as
$m=(t,(v_1,v_2,\ldots,v_n))$, that is the value has domain $\R*^n$; in
the other form we write measures directly as
$m=(t,v_1,v_2,\ldots,v_n)$.




\subsection{Operations}

Time series have a time attribute that must be manipulated
coherently. Owing to this time attribute, the behaviour of a time
series can be of three types:
\begin{itemize}
\item Set operations that manipulate the set structure, including
  relational operations.
\item Sequence operations that consider the set with order.
\item Temporal function operations that manipulate a time series
  considering it is a representation for a continuous function.
\end{itemize}


\todo{dir que per raons d'espai describim els més importants i de forma breu?}


\subsubsection{Set operations}

We describe how common set operators can be applied to time series. We
rely on how the relational model of DBMS describes operations based on
set algebra \cite{date:introduction}.

The two possible order relations of measures induces two operation
definitions for some set operators. Mainly it induces two membership
operations and so to all operations based on it. As we have a partial
order and a total order called temporal order, respectively we call
these operators with the normal name and with a \emph{temporal}
prefix. In Figure~\ref{fig:model:venn} we show typical Venn diagrams
for both cases, where $t_A$ and $t_b$ are subsets with measures that
only share the time attribute between sets $A$ and $B$. The normal
intersecting area indicates measures that share time and value
attributes.

\begin{figure}
  \centering
  \def\escala{0.7}

\def\nodeA{node [anchor=east] {$A$}}
\def\nodeB{node [anchor=west] {$B$}}
\def\nodeT{node [left=0.4cm] {\tiny $t_A$} node [right=0.4cm] {\tiny $t_B$}}
% Definition of circles
\def\firstcircle{(0,0) circle (1.5cm)}
\def\secondcircle{(0:2cm) circle (1.5cm)}
\def\thirdcircle{(0:1cm) circle (1.11cm)}

\colorlet{circle edge}{blue!50}
\colorlet{circle area}{blue!20}

\tikzset{
  filled/.style={fill=circle area, draw=circle edge, thick},
  outline/.style={draw=circle edge, thick},
  every node/.style={transform shape}
}

%\setlength{\parskip}{5mm}



%Set A in B
\begin{tikzpicture}[scale=\escala]
    \begin{scope}
        \clip \secondcircle;
        \draw[even odd rule,blue] \firstcircle \nodeA
                                 \secondcircle ;
                                 %\thirdcircle;
            \fill[filled] \firstcircle;
   \end{scope}
      \draw[outline] \secondcircle \nodeB;

   \node[anchor=south] at (current bounding box.north) {$A \subset B$};
   \node[anchor=west] at (current bounding box.west) {$A$};
\end{tikzpicture}
%Set temporal A in B
\begin{tikzpicture}[scale=\escala]
    \begin{scope}
        \clip \firstcircle;
        \fill[filled] \thirdcircle;
      \draw[outline] \thirdcircle \nodeT;
    \end{scope}
    \begin{scope}
        \clip \secondcircle;
        \draw \thirdcircle \nodeT;
    \end{scope}
      \draw[circle edge] \thirdcircle;
    \begin{scope}
        \clip \secondcircle;
        \draw[even odd rule,blue] \firstcircle \nodeA
                                 \secondcircle ;
                                 %\thirdcircle;
            \draw[outline] \firstcircle;
   \end{scope}
      \draw[outline] \secondcircle \nodeB;

   \node[anchor=south] at (current bounding box.north) {$A \subset^t B$};
   \node[anchor=east] at (current bounding box.center) {$A$};
\end{tikzpicture}







%Set A or B
\begin{tikzpicture}[scale=\escala]
  \draw[filled] \firstcircle \nodeA;
    \begin{scope}
        \clip \secondcircle;
        \draw[filled, even odd rule] \firstcircle \nodeA
                                 \secondcircle 
                                 \thirdcircle;
   \end{scope}
    \draw[outline] \firstcircle
                   \secondcircle \nodeB
                   \thirdcircle \nodeT;

   \node[anchor=south] at (current bounding box.north) {$A \cup B$};
\end{tikzpicture}
%Set temporal A or B
\begin{tikzpicture}[scale=\escala]
    \draw[filled, even odd rule] \firstcircle \nodeA
                                 \secondcircle \nodeB
                                 \thirdcircle \nodeT;
    \node[anchor=south] at (current bounding box.north) {$A \cup^t B$};
\end{tikzpicture}




% Set A but not B
\begin{tikzpicture}[scale=\escala]
    \begin{scope}
        \clip \firstcircle;
        \draw[filled, even odd rule] \firstcircle \nodeA
                                     \secondcircle;

    \end{scope}
    \draw[outline] \firstcircle
                   \secondcircle \nodeB
                   \thirdcircle \nodeT;
    \node[anchor=south] at (current bounding box.north) {$A - B$};
\end{tikzpicture}
% Set temporal A but not B
\begin{tikzpicture}[scale=\escala]
    \begin{scope}
        \clip \firstcircle;
        \draw[filled, even odd rule] \firstcircle \nodeA
                                     \thirdcircle;

    \end{scope}
    \draw[outline] \firstcircle
                   \secondcircle \nodeB
                   \thirdcircle \nodeT;
    \node[anchor=south] at (current bounding box.north) {$A -^t B$};
\end{tikzpicture}





% % Set A and B
% \begin{tikzpicture}
%     \begin{scope}
%         \clip \firstcircle;
%         \fill[filled] \secondcircle;
%     \end{scope}
%     \draw[outline] \firstcircle \nodeA;
%     \draw[outline] \secondcircle \nodeB;
%     \draw[outline] \thirdcircle \nodeT;
%     \node[anchor=south] at (current bounding box.north) {$A \cap B$};
% \end{tikzpicture}
% % Set temporal A and B
% \begin{tikzpicture}
%     \begin{scope}
%         \clip \firstcircle;
%         \fill[filled] \thirdcircle;
%     \end{scope}
%     \draw[outline] \firstcircle \nodeA;
%     \draw[outline] \secondcircle \nodeB;
%     \draw[outline] \thirdcircle \nodeT;
%     \node[anchor=south] at (current bounding box.north) {$A \cap^t B$};
% \end{tikzpicture}



% %Set A or B but not (A and B) also known a A xor B
% \begin{tikzpicture}
%     \begin{scope}
%         \clip \firstcircle;
%         \draw[filled, even odd rule] \firstcircle
%                                      \secondcircle;
%     \end{scope}
%     \begin{scope}
%         \clip \secondcircle;
%         \draw[filled, even odd rule] \secondcircle 
%                                      \thirdcircle;

%     \end{scope}
%     \draw[outline] \firstcircle \nodeA;
%     \draw[outline] \secondcircle \nodeB;
%     \draw[outline] \thirdcircle \nodeT;
%     \node[anchor=south] at (current bounding box.north) {$A \ominus B$};
% \end{tikzpicture}
% %Set temporal A or B but not (A and B) also known a A xor B
% \begin{tikzpicture}
%     \begin{scope}
%         \clip \firstcircle;
%         \draw[filled, even odd rule] \firstcircle
%                                      \thirdcircle;
%     \end{scope}
%     \begin{scope}
%         \clip \secondcircle;
%         \draw[filled, even odd rule] \secondcircle 
%                                      \thirdcircle;

%     \end{scope}
%     \draw[outline] \firstcircle \nodeA;
%     \draw[outline] \secondcircle \nodeB;
%     \draw[outline] \thirdcircle \nodeT;
%     \node[anchor=south] at (current bounding box.north) {$A \ominus^t B$};
% \end{tikzpicture}
  \caption{Venn diagrams for set and temporal set operations of TSMS}
  \label{fig:model:venn}
\end{figure}


The \emph{membership} defines when one measure belongs to a time
series. Considering the partial order of measures, the normal set
membership operation applies. Let $S$ be a time series and $m$ a
measure, the \emph{membership} is denoted $m \in S$. Considering the
temporal order of measures, we define the \emph{temporal membership}
as $m \inst S$ when $\exists m_a \in S : T(m) = T(m_a)$.  Let $S_1$
and $S_2$ be two time series, from membership operations we could
define \emph{inclusion} $S_1\subseteq S_2$ and \emph{temporal
  inclusion} $S_1\subseteqt S_2$.


In a time series the measures have a total order.  As a time series is
a finite set, if it is not empty it has a maximum and a minimum.  Let
$S$ be a time series and $n\in S$ be a measure, the time series'
\emph{maximum} is $n=\max(S)$ if and only if $\forall m \in S: n \geq
m $.  For extended real numbers the supremum is defined even when a
set is empty $\sup(\emptyset)$ \cite{cantrell:extendedreal}, so
applying this property we can say $n=\sup(S)$ is the \emph{supremum}
of the time series where $n=\max(S)$ when defined and
$n=(-\infty,\infty)$ otherwise.
Dually, we can define \emph{minimum} $\min(S)$ and \emph{infimum} $\inf(S)$.



The \emph{union} of two sets is a set containing elements from both
sets. For time series it must consider when the result would have
repeated measures so we define a non commutative operation and a
commutative temporal operation. The union requires both time series to
have the same structure and type as happens with union operation in
relational DBMS \cite{date:introduction}. %
Let $S_1$ and $S_2$ be two time series. The \emph{union} of both is a
time series $S_1 \cup S_2$ with all measures of $S_1$ and those not
repeated at $S_2$: $S_1 \cup S_2 = \{m^1 \in S_1 \vee m^2 \in S_2 |
m^2 \notinst S_1 \}$. The \emph{temporal union} of both is a time
series $S_1 \cupt S_2$ with all measures from $S_1$ and $S_2$
excluding those that only share the time attribute: $S_1 \cupt S_2 =
\{ m^1 \in S_1 \vee m^2 \in S_2 | m^1 \notinst S_2 \vee m^1 \in S_2,
m^2 \notinst S_1 \}$.



The \emph{difference} of two sets is a set containing all elements
from first not contained in second. Like union, the difference
requires both time series to have the same structure. %
Let $S_1$ and $S_2$ be two time series. The \emph{difference} of first
and second is a time series $S_1 - S_2$ with all measures of $S_1$ not
belonging to $S_2$: $S_1 \cup S_2 = \{ m \in S_1 | m \notin S_2
\}$. The \emph{temporal difference} of first and second is a time
series $S_1 - S_2$ with all measures of $S_1$ not temporal belonging
to $S_2$: $S_1 -^t S_2 = \{ m \in S_1 | m \notinst S_2 \}$.


Based on union and difference we can define \emph{intersection} $S_1\cap
S_2 \equiv S_1 - (S_1 - S_2)$ and \emph{symmetric difference} $S_1 \ominus
S_2 \equiv (S_1 - S_2) \cup (S_2 - S_1)$ operations as well as its
temporal dual operations.


Relational DBMS extend these set operators with specific ones like
projection, selection, rename, product and join. They can be also used
for time series. We show a definition for join operation.


The \emph{join} of two time series is the grouping of pairs that share
the same time attribute in both.  Let $S_1$ i $S_2$ be two time series,
the \emph{join} of both time series $S_1 \join S_2$ is a multivalued
time series $S_1 \join S_2 = \{ (t,v_1,v_2) | (t_1,v_1) \in S_1 \wedge
(t_2,v_2) \in S_2 \wedge t=t_1=t_2 \}$.


We need computational operations in addition to operations defined so
far to be able to compute with the elements contained in time
series. In relational DBMS these operators are extend, aggregate and
summarise \cite{date:introduction}. We define a more general
equivalent computational operators map and fold.


Map applies a function to each measure of a time series.  Let $S$ be a
time series and $f^m$ a function over a measure where $f^m:m_a\mapsto
m_r$, the \emph{map} of $f^m$ to $S$ is a time series $\map(S,f^m) =
\{\forall m_i\in S : f^m(m_i) \}$.

Fold combines recursively every measures of a time series.  Let
$S=\{m_0, \dotsc, m_k\}$ and $S_i$ be two time series, and $f^f$ a
function over a measure and a time series where $f^f: S_a \times m_b
\mapsto S_r$. The fold of $S$ by $f^f$ with initial value $S_i$ is a
time series $\fold(S,S_i,f^f) = f^f(\dots(
f^f(f^f(f^f(S_i,m_0),\allowbreak m_1),\allowbreak m_2
)\dots),\allowbreak m_k)$.


A simplified fold operation is \emph{aggregate} which
is used for combining a time series into one measure, normally it is
used to compute aggregate statistics.  Let $S$ be a time series, $m_i$
a measure, and $f^a$ a function over two measures where $f^a: m_a
\times m_b \mapsto m_r$.  \emph{Aggregate} can be defined based on
fold operation: $\agg(S,m_i,f^a) \equiv \fold(S,\{m_i\},f')$ where $f':
\{m_i\} \times m \mapsto \{f^a(m_i,m)\}$.

A more generic version of fold is a fold considering order. In the
previous fold the measures are computed in random order, however in
some computational operations it is necessary to define the order,
especially when $f^f$ is not commutative. We define a \emph{fold with
  order} as an extension of fold with a function $o$ that selects
measures in order where $o: S_a \mapsto m_r$
\[
 \orderfold(S,S_i,f^f,o) =
  \begin{cases}
    S_i  \text{ if } |S|=0, \\
    \orderfold(S_o,f^f(S_i,m_o),f^f,o)  \text{ oth.}
  \end{cases}
\]
 where $m_o = o(S)$ and $S_o = S - \{m_o\}$.




\todo{revisat fins aquí} 

\todo{binary computacional}

Finally, we describe how binary computational operation can be defined
for two time series in order to operate with their value attributes.
First it is required to join the two time series and then apply
computational operations. 

 
El producte i la junció són els operadors que permeten crear parelles
de mesures de dues sèries temporals. Per a operar amb els valors de
dues sèries temporals la junció és més adequada ja que permet ajuntar
el valors que tenen temps comuns. Així doncs, per a aplicar un
operador binari $\operatorname{op}$ que calculi amb els valors de
dues sèries temporals:

\[
\operatorname{op}: S_1 \times S_2 \longrightarrow S'
\]
\[
\text{a on } S' = \map(\join(S_1,S_2),(t,v^1,v^2)\mapsto(t,v^1
\operatorname{op} v^2))
\]

Cal tenir en compte que la junció de  només
sap operar amb dues sèries temporals que tinguin el mateix vector de
temps; és a dir regulars entre elles (v.\
def.~\ref{def:st:regular}). En el cas que no tinguin el mateix vector
de temps, es pot aplicar la junció temporal


Exemples de l'aplicació d'operacions computacionals per a dues sèries
temporals
\begin{itemize}
\item $S' = S_1 + S_2$
\item $S' = S_1 / S_2$
\end{itemize}


We use join, but when time series do not temporal intersect then we should use temporal functions operations which are defined below.




\subsubsection{Sequence operations}

Sequence operations manipulate time series considering measures as
being total ordered. Then we can define three basic operations:
interval, successor and concatenation.


The \emph{interval} $(r,t)$, where $r$ and $t$ are two time instants, over a
time series $S$ is a time subseries between the two time instants
$S(r,t) \subseteq S$. Similarly as \cite{last:hetland}, we define the
open interval as $S(r,t)=\{m\in S | r<T(m)<t\}$. Other intervals can
be defined with different continuities in the limits: closed $S[r,t]$,
left-open $S(r,t]$, and right-open $S[r,t)$.

The time order in time series also implies the sequence concept of
\emph{successor} and \emph{predecessor}.  Let $S=\{m_0, \ldots, m_k\}$
be a time series and $l\in S$ and $n$ be two measures. We say
$l=\nex_S(n)$ is the \emph{next} measure to $n$ in $S$ if and only if
$l=\inf(S(T(n),+\infty])$.  We say $l=\prev_S(n)$ is the
\emph{previous} measure to $n$ in $S$ if and only if
$l=\sup(S[-\infty,T(n)))$. %
Infinite measures are obtained when next and previous are applied to
supremum and infimum measure respectively: $\nex_S(\sup
S)=(+\infty,\infty)$ and $\prev_S(\inf S)=(-\infty,\infty)$.



\emph{Concatenation} comprises the measures of the first time series followed
in time order by the measures of the second. It is similar to union
for sets but considering the sequence interval. The concatenation
requires both time series to have the same structure as seen with
union operation.  Let $S_1$ and $S_2$ be two time series, the
concatenation of both $S_1 || S_2$ is a time series $S$ containing all
measures of $S_1$ and those of $S_2$ that not intersect in the
interval of $S_1$.  $S_1 || S_2 = S_1 \cup ( S_2 - S_2[t_1,t_2] )$ a
on $t_1=T(\inf S_1)$ i $t_2=T(\sup S_1)$.




\subsubsection{Temporal function operations}

\todo{interval temporal}

\todo{interval temporal ZOHE}

\todo{interval temporal delta}

From temporal interval other operators can be defined such as temporal
selection, temporal concatenation, or temporal join.




\subsection{Properties}


\todo{}





% Let $S$ be a time series, $t$ be a time instant and $\delta$ be a
% time duration, then the time series' measures can be located in the
% time interval $i_0=[t, t+\delta]$ and its multiples $i_j=[t+j\delta,
% t+(j+1)\delta]$ for $j=0,1,2,\ldots$. When time series' measures are
% equally spaced we say it to be regular.
% \begin{definition}[Regular time series]
%   Let $S=\{m_0,$ $ldots,$ $m_k\}$ be a time series and $\delta$ a time
%   duration. $S$ is regular if and only if $\forall m \in
%   S(T(\min(S),+\infty):T(m) - T(\prev_S(m)) = \delta$.
% \end{definition}



%Representation:
% In the design of the attribute aggregate function we can interpret a
% time series in different ways, that is what we call the representation
% of a time series. Keogh et al.\ \cite{last:keogh} cite
% some possible representations for time series such as Fourier
% transforms, wavelets, symbolic mappings or piecewise linear
% representation. The last one is very usual due to its simplicity,
% \cite{keogh01}.

% Time series representations can be taken into account when computing
% with the measures of the time series.  For example, a maximum
% attribute aggregate function may give different values if we consider
% a linear or a constant piecewise representation.

% Following we show a possible family of attribute aggregate functions
% for time series represented by a staircase function, that is with a
% piecewise constant representation.  We define a new representation for
% time series named \emph{zero-order hold backwards} (zohe). This
% representation holds back each value until the preceding value. 
% RRDtool, \cite{lisa98:oetiker}, has a similar aggregate function.

% Let $S=\{m_0,\ldots,m_k\}$ be a time series, we define
% $S(t)^{\text{zohe}}$ as its continuous representation along time $t$:
% $\forall t \in \mathbb{R} ,\forall m \in S:$
% \begin{equation}
%  S(t)^{\text{zohe}} =  
% \begin{cases}
%   \infty & \text{if } t > T(\max S) \\
%   V(m)   & \text{if } t\in (T(\prev_S m),T(m)]
% \end{cases}
% \label{eq:zohe}
% \end{equation}


\section{Multiresolution model}
\label{sec:MTSMS}


The \acro{MTSMS} are \acro{TSMS} that store time series with a lossy
compression approach, that is some information is selected and spread in
different time resolutions. The \acro{MTSMS} model is based on the
concepts of measures and time series as defined in
Section~\ref{sec:model:preliminaries}.


The multiresolution concept comes from thoroughly analysis of the
RRDtool \cite{rrdtool} \acro{TSMS}. Our objective is to formalise its
essential parts into an abstract model, where what we call
multiresolution plays a main role, and to include more genericity in
order to describe \acro{MTSMS} as fully \acro{TSMS}. Then we will be
able to apply these systems to other applications.
\todo{repassar paràgraf}


A \acro{MTSMS} stores multiresolution time series where each has a
multiresolution schema as shown in Figure~\ref{fig:model:mtsdb}. A
multiresolution time series is a collection of resolution subseries
which temporarily accumulate measures in a buffer in order to select
some information and finally store it in a disc. The information
selection process changes the time intervals between measures to
compact information by aggregating the time series attributes. 

\begin{figure}
  \centering
  \begin{tikzpicture}
 \tikzset{
        myarrow/.style={->, >=latex',  thick},
      }
      

  \node[rectangle,draw,minimum height=6cm,minimum width=9cm] (m) {};
  \draw[shift=( m.south west)]   
  node[above right] {base de dades multiresolució};


  %discmig
  \node (m.center) (discr1) {...};

  %discr
  
  \node[ellipse,draw,minimum height=3.5cm,minimum width=2.5cm,alias=discr0] [left=of discr1] {};
  \node[above=0cm of discr0.north] {R0};
  \node[below=0cm of discr0] {disc resolució};

  \node[cylinder, draw, shape border rotate=90, aspect=0.25,alias=buffer0] [below=3mm of discr0.north] {buffer};
  \node[circle, draw,alias=disc0]  [above=3mm of discr0.south] {disc} ;
  \draw [->] (disc0.center)++(.4:.4cm) arc(0:180:.4cm);
  \draw[myarrow] (buffer0.bottom) -- (disc0.north);


  %discrd

  \node[ellipse,draw,minimum height=3.5cm,minimum width=2.5cm,alias=discrd] [right=of discr1] {};
  \node[above=0cm of discrd] {Rd};
  \node[below=0cm of discrd] {disc resolució};

  \node[cylinder, draw, shape border rotate=90, aspect=0.25,alias=bufferd] [below=3mm of discrd.north] {buffer};
  \node[circle, draw,alias=discd]  [above=3mm of discrd.south] {disc} ;
  \draw [->] (discd.center)++(.4:.4cm) arc(0:180:.4cm);
  \draw[myarrow] (bufferd.bottom) -- (discd.north);



  %mesura 
  \node[above=1cm of m.north] (m0) {};

  \draw[myarrow] (m0) -- (m.north) 
  node[right,midway] {mesura};

  \draw[myarrow] (m.north) -- (buffer0);
  \draw[myarrow] (m.north) -- (bufferd);
  \draw[myarrow] (m.north) -- (discr1);

\end{tikzpicture}
  %\smallskip
  \caption{Architecture of MTSMS model}
  \label{fig:model:mtsdb}
\end{figure}


In this way, the original time series gets stored spread in the discs,
each with a different time resolution and attribute aggregation.
Discs are size bounded so they only contain a fixed amount of
measures. When a disc becomes full it discards a measure. Thus,
multiresolution database is bounded in size and the time series gets
stored in pieces, that is time subseries.

Regarding operations, \acro{MTSMS} structure needs operators to change
the time intervals between measures and to select attributes. Mainly,
these operators are measure additions and time series consolidations,
which some functionality is delegated to operators called attribute
aggregate functions. Secondarily, there are operators to query the
multiresolution schema and extract time series data.


Following we define the \acro{MTSMS} model by: (i) four basic
structure model elements ---buffer, disc, resolution subserie, and
multiresolution time series--- with its structure operators, (ii) the
operations to change and consult a multiresolution schema, and (iii)
the attribute aggregate functions.



\subsection{Structure}

A \emph{buffer} is a container for a regular or a no-regular time
series. The buffer objective is to regularise the time series using a
predetermined step and an attribute function. We name
\emph{consolidation} to this action.
\begin{definition}%(Buffer)
  A \emph{buffer} is defined as the tuple $(S_B,\tau,\delta,f)$ where
  $S_B$ is a time series, $\tau$ is the last consolidation time,
  $\delta$ is the duration of the consolidation step and $f$ is an
  attribute aggregate function.

  An empty buffer $B_{\emptyset} = (\emptyset,t_0, \delta, f)$ has an
  empty time series, an initial consolidation time $t_0$ and
  predetermined $\delta$ and $f$.
\end{definition}

Operator \emph{addBuffer} adds a measure to its time series:
$\addB: B = (S_B,\tau,\delta,f) \times m \mapsto
(S'_B,\tau,\delta,f)$ where $S'_B = S \cup \{m\} $.

From the $B_{\emptyset}$ all the consolidation time instants can be
calculated as $t_0+i\delta, i\in\N$. The consolidation of $B$ in a
time interval $i=[\tau,\tau+\delta]$ results in a measure
$m'=f(S_B,i)$ where $f$ is an attribute aggregate function
$f$. Operator \emph{consolidateBuffer} consolidates a set of measures
and removes the consolidated part of the time series from the buffer:
$\consB : B=(S_B,\tau,\delta,f) \mapsto B' \times m'$ where $ B'=
(S'_B,\tau+\delta,\delta,f)$, $m' = f(S,[\tau,\tau+\delta])$, and
$S'_B$ is the discarding of historic data not needed anymore, for example
$S'_B = S[\tau+\delta,+\infty]$.

On a simplified way, the $\consB$ is only applied to the present
consolidation interval and the total consolidation is obtained by
successive application of the operator. This requires measures to be
added by time order and to consolidate the buffer when the time of
some measure is bigger than the buffer's next consolidation time.  Let
$B=(S_B,\tau,\delta,f)$ be a buffer and $m=\sup(S_B)$ the maximum
measure, $B$ is consolidable if and only if $T(m) \geq
\tau+\delta$.


A \emph{disc} is a finite capacity measures container. A time series
stored in a disc has its cardinal bounded. When the cardinal of the
time series is to overcome the limit, some measures need to be
discarded.
\begin{definition}%(Disc)
  A \emph{disc} is a tuple $(S_D,k)$ where $S$ is a time series and
  $k\in\N$ is the maximum allowed cardinal of $S_D$.  An empty
  disc $D_{\emptyset} = (\emptyset,k)$ has an empty time series and
  the $k$ maximum cardinal allowed.
\end{definition}

The cardinal of the times series is kept under control by the add
operator, $\addD : D=(S_D,k)\times m\mapsto (S'_D,k)$ where %
$
 S_D' = \begin{cases}
  S_D\cup\{m\}                 & \text{if } |S_D|<k  \\
  (S_D-\{\min(S_D)\}) \cup \{m\} & \text{otherwise}
\end{cases}  
$.


A \emph{resolution subseries} is a structure that regularises and
aggregates a time series. It is composed of a buffer, that contains
the partial time series to be regularised, and a disc, that contains
the regularised time series.
\begin{definition}%(Resolution subseries)
  A \emph{resolution subseries} is a tuple $(B,D)$ where $B$ is a
  buffer and $D$ is a disc.  An empty buffer and empty disc imply an
  empty resolution subseries $R_{\emptyset} =
  (B_{\emptyset},D_{\emptyset})$.
\end{definition}
 
The operators of a resolution subseries extend the buffer and disc
ones: (i) The addition of a measure to the buffer of the resolution
subseries: $\addR : R=(B,D) \times m \mapsto R'$ where $R'= (B',D)$,
and $B'= \addB(B,m)$; (ii) The consolidation of the resolution
subseries by consolidating its buffer and adding the consolidation
measure to its disc: $\consR : R=(B,D) \mapsto R'$ where $R'=
(B',D')$, $(B',m') = \consB(B)$, and $D'= \addD(D,m')$.  A resolution
subseries is consolidable only when its buffer is consolidable.




A \emph{multiresolution time series} is a set of resolution subseries
which share the input of measures, that is they buffer the same time
series. A time series is stored regularised and distributed with
different resolutions in the various resolution subseries, as
previously shown in Figure~\ref{fig:model:mtsdb}.
\begin{definition}%(Multiresolution time series)
  A \emph{Mul\-ti\-re\-solution time series} is a set of resolution
  subseries $\{R_0, \dots, R_d\}$.  An empty multiresolution series
  has empty resolution subseries $M_{\emptyset}=\{R_{0_\emptyset},
  \dots, R_{d_\emptyset}\}$. Usually there are no repeated pairs of
  ($\delta_i$,$f_i$) among a multiresolution series, so they act as a
  key attributes.
\end{definition}

The operators of a multiresolution time series apply to every
resolution subseries contained: (i) The addition of a measure to every
resolution subseries: $\addM : M=\{R_0,\allowbreak \dots,\allowbreak
R_d\} \times m \mapsto \{R'_0, \dots,\allowbreak R'_d\}$ where
$R'_i=\addR(R_i,m)$; (ii) The consolidation of all resolution
subseries: $\consM : M=\{R_0,\allowbreak \dots,\allowbreak R_d\}
\mapsto \{R'_0,\allowbreak \dots,\allowbreak R'_d\}$ where $R'_i =
\consR(R_i)$ if $R_i$ $\text{ consolidable}$ and $R'_i=R_i$
$\text{otherwise}$.


The multiresolution consolidation operation should be applied
regularly based on a consolidation clock. When the measure ordered
addition approach is taken as explained in the buffer's consolidation,
then there is no need for a clock in a MTSMS. The consolidation clock
is induced by the measure's addition and then it is only necessary to
check the multiresolution consolidation operation on new
additions. However, there could be other approaches where the
consolidation clock was given by an external clock or external
events. Then the consolidable definitions would depend on this
external clock.


\todo{s'hauria de dir que cada R en una M es configura amb els parametres (delta,tau,f,k) si més no a data manipulation}


\subsection{Data manipulation}


\todo{tenim espai per a les operacions d'esquema? o potser deixar-ho per una altre article}



\subsection{Queries}


There are two basic time series queries for a MTSMS: (i) extract a
time subseries from a resolution subseries' disc or (ii) query for a
total time series from all consolidated information.

The first is a selection of a disc over a multiresolution time series,
being $(\delta,f)$ the key attributes: $\seriedisc: M=\{R_0, \dots,
R_d\} \times \delta \times f \mapsto S'_D \in D' | (B',D') \in R',R' \in
M$.

The second is a concatenation of all discs' time subseries trying to
obtain the most resolution as possible, which is to say by $\delta$
order: $\totalseries: M*=\{R_0, \dots, R_d\} \mapsto S'$ where $S' =
S_{D0} || S_{D1} || \cdots || S_{Dd}$ and $\delta_0 < \delta_1 <
\cdots < \delta_d$. This states that $M*$ is a multiresolution time
subseries where $R_i$ have a total order by its attribute
$\delta_i$. Being $(\delta,f)$ the key attributes, the $M*$ can
be obtained from $M$ by selecting resolution subseries with same $f$. If we
operated a $\totalseries$ to a general $M$ then it could be ambiguous
as it could contain repeated $\delta_i$.


From these two basic time series queries, more elaborated queries can
be applied to MTSMS by using TSMS operations. For example, let $M_1$
and $M_2$ be two multiresolution time series, we can compute the sum
of both with $\totalseries(M_1) + \totalseries(M_2)$. 
% This is the general algebraic expressions that describes the model,
% but an implementation of the model could accomplish this operation
% in a more efficient way.





\subsection{Attribute aggregate function}
\label{sec:model:interpolador}

Attribute aggregate functions are used when consolidating a buffer in
order to summarise the time series information. Let $S$ be a time
series and $t_0$ and $t_f$ two time instants, an attribute aggregate
function $f$ calculates a measure that summarises the measures of $S$
included in the time interval $i=[t_0,t_f]$:
\[
f : S=\{m_0,\ldots,m_k\} \times [t_0,t_f] \mapsto m'
\]
where, generally, $m'$ results from two operations on the time series:
(i) a time subseries selection $S'$ depending on the consolidating
interval, for example $S' = S[t_0,t_f]$, and (ii) an aggregation over
this time subseries $m' = \agg(S',m_i a)$ with $a: m_i \times
m\rightarrow m''$.  \todo{mirar be sintaxi agg}

To summarise a time series we can use different attribute aggregate
functions.  For instance, we can calculate an statistic indicator of
the time series such as the average or we can apply a more complex
digital signal processing operation, \cite{zhang11}. Furthermore, the
possible representation for a time series and some of its pathologies
can be considered during the aggregation process.


In conclusion, we can define many attribute aggregate functions and
thus no global assumptions can be made about them. Each user has to
decide which combination of aggregation and representation fits better
with the measured phenomena.  Therefore, \acro{MTSMS} must allow users to
define aggregate functions.

We can classify attribute aggregate functions based on a discrete or
continuous approach. For each approach we can define patterns that
explain how the resulting value $V(m')$ is calculated but not the
resulting time instant $T(m')$, which is subject to
interpretation.  


Regarding the resulting consolidation time, normally it will be
$T(m')=t_f$ to be consistent with the consolidation operation of a
buffer where $\tau' = \tau + \delta \equiv t_f$. However, as will be
shown below on some examples, $T(m')$ can have an offset with buffer
consolidating times. A trivial example of an offset is for an
aggregate function that returns the first measure
$m'=\min(S[t_0,t_f))$ as then $t_0 \leq T(m') < t_f$.


First, discrete aggregation functions are based on set and sequence
time series operators. The patterns of these functions leave $T(m')$
undefined as well as the time subseries consolidating interval, that
is the resulting measure can be aggregated from a time subseries $S'$
with open interval $S'=S(t_0,t_f)$, closed interval $S'=S[t_0,t_f]$,
or other combinations like $S'=S(t_0-d,t_f-d]$ where $d$ is a time
duration. Next there are some attribute patterns examples, let the
time be continuous on all the time domain $t\in T$:
\begin{itemize}
%\renewcommand{\labelitemi}{--}
\item maximum$^d$: $S \times i \mapsto m'$ where $V(m') =
  \max\limits_{\forall m \in S'}(V(m))$. It summarises $S'$ with the maximum
  of the measure values.
\item last$^d$: $S \times i \mapsto m'$ where $V(m') = \max(S')$. It
  summarises $S'$ with the maximum measure.
\item arithmetic mean$^d$: $S \times i \mapsto m'$ where $V(m') = 
  \frac{1}{|S'|} \allowbreak \sum\limits_{\forall m\in S'} V(m)$. It
  summarises $S'$ with the mean of the measure values.
\end{itemize}


Second, continuous aggregation functions are based on temporal
function time series operators, that is the time series aggregated
corresponds to a continuous function $S(t)^r$ where $r$ is a
representation. The patterns of these functions leave $T(m')$
undefined as well as the representation $r$ of the time series.  Next
there are some attribute patterns examples, let the time be continuous
on all the time domain $t\in T$:

\begin{itemize}
%\renewcommand{\labelitemi}{--}
\item maximum$^c$: $S \times i \mapsto m'$ where $V(m') =
  \max\limits_{\forall t \in i}(S(t)^r)$. It summarises $S$ with the maximum
  of the measure values in the interval $i$.
\item last$^c$: $S \times i \mapsto m'$ where $V(m') = S(t_f)^r$. It
  summarises $S$ with the value at $t_f$ time instant.
\item mean$^c$: $S \times i \mapsto m'$ where $V(m') =
  \frac{1}{t_f-t_0} \int\limits_{t_0}^{t_f} S(t)^r dt$. It summarises $S$
  with the mean of the function in the interval $i$.
\end{itemize}


The continuous aggregation function patterns can be expressed with
discrete mathematics for each particular representation, that is based
on the temporal interval defined in TSMS. Next we exemplify it by
defining the previous general continuous patterns for two particular
representations: delta and \zohe{}.


Delta attribute aggregation functions $f^\delta$ have a general form
$f^\delta : S \times [t_0,t_f]\mapsto m'$ where $m'=(t',v')$, the
resulting time is interpreted as centred on the interval
$t'=\frac{t_f+t_0}{2}$ and the resulting value depends on the
attribute, let $S'=S[t_0,t_f]^\delta$ be the selection of measures by
delta temporal interval:
\begin{itemize}
%\renewcommand{\labelitemi}{--}
\item maximum$^\delta$: $v' = \max\big(0,\max_{\forall m \in S'}(V(m))\big)$. 
\item last$^\delta$: $v' = \max(S')$.
\item mean$^\delta$: $v' = \frac{1}{t_f-t_0} \sum\limits_{\forall m
    \in S'} V(m)$, as delta function has property $\int\delta(t)dt=1$.
\end{itemize}


\zohe{} attribute aggregation functions $f^\zohe{}$ have a general
form $f^\zohe{} : S \times [t_0,t_f]\mapsto m'$ where $m'=(t',v')$,
the resulting time is interpreted as right limit of the interval
$t'=t_f$ and the resulting value depends on the attribute, let
$S'=S[t_0,t_f]^\zohe{}$ be the selection of measures by \zohe{} temporal
interval:
\begin{itemize}
%\renewcommand{\labelitemi}{--}
\item maximum$^\zohe{}$: $v' = \max_{\forall m \in S'}(V(m))$. 
\item last$^\zohe{}$: $v' = \max(S')$.
\item mean$^\zohe{}$: $v' = \frac{1}{t_f-t_0} \big[ (T(o)-t_0)V(o) +
  \sum\limits_{\forall m \in S''}( T(m)- T(\prev_S
  m) )V(m) \big]$ where $o=\min(S')$ and $S''= S' - \{o\}$.
\end{itemize}

A similar aggregation function to mean$^\zohe{}$ is used by RRDtool
\cite{rrdtool} in order to summarise information for velocity counter
data by keeping the total counting information, as mean aggregation
can be seen as one keeping the area below the original signal.


In conclusion, some continuous patterns are very similar to discrete
ones. As instance maximum and last attributes differ basically on the
interval selection operation. However, other patterns have a more
elaborated interpretation in a particular representation. As instance
mean$^\zohe{}$ and mean$^\delta$ is an elaborated interpretation for
the general integral definition.




\subsubsection{Data validation}

\todo{potser encara es verd aixo}

% With reference to data validation, attribute aggregate functions
% can cope with this process. When data has not been captured or has
% been captured erroneously, it must be treated as unknown data.
% \begin{itemize}
% \item When data has not been captured it is unknown by nature. For
%   example, we try to capture data from a sensor and there is no
%   response.
% \item When data is erroneously it must be marked as unknown. For
%   example, we capture data from a sensor but it responses in a not
%   reasonable time or we capture data that is clearly outside a
%   reasonable limits.
% \end{itemize}
% As a consequence, attribute aggregate functions deals with these two
% subprocesses: treating unknown data and marking data as
% unknown. Following with real numbers example, we extend the
% domain with a value that means 'unknown', let this unknown value be
% represented by the improper element infinity ($\infty$).

% An attribute aggregate functions treating unknown
% data is a one that can calculate a result when there are unknown
% values in the original time series, $f^u: S \times i \mapsto m'$ where
% $\exists m \in S: V(m)=\infty$. Although from a strict point of view
% operating with unknown data makes unknown result, aggregate functions
% are free to calculate whatever is needed such as time series analysis
% does with data reconstruction.

% For example, arithmetic mean$^{d}$ aggregate function returns
% $V(m')=\infty$ if $\exists m \in S: V(m)=\infty$.  We can define a new
% mean function, based on the original arithmetic mean$^{d}$ aggregate,
% that naively treats unknown values by keeping the
% known mean; in other words, it ignores unknown values found in the time
% interval: arithmetic mean$^{du}$: $S \times i \mapsto m'$ where $m' =
% \text{arithmetic mean}^{d}(S'',i)$ and $S''= \{m''\in S':V(m'')\neq
% \infty\}$.
% % ignore$^{u}$: $S \mapsto S'$ where $S'= \{m''\in S':V(m'')\neq
% % \infty\}$,
% % arithmetic mean$^{du}$: $S \times i \mapsto m'$ where $m' =
% % \text{arithmetic mean}^{d}(\text{ignore}^u(S),i)$.

% An attribute aggregate functions marking data as unknown is a one
% that can give unknown value as the resulting measure's value, $f^{mu}:
% S \times i \mapsto m'$ where $V(m')\in \mathbb{R}\cup\{\infty\}$.

% For example, we can define a maximum aggregate, based on the
% maximum$^d$ aggregate, that returns unknown if there is a
% measure's value bigger than 2:  maximum$^{dmu2}$: $S \times i
% \mapsto m'$ where $V(m') = 
% \begin{cases}
%   \infty &\text{if }  m''>2\\
%   m'' & \text{else }
% \end{cases}$ and $m''=\text{maximum}^d(S,i)$.

% %Per exemple definim un termini, si les dades estan més espaiades que 2 es marca com a desconeguda









\section{Multiresolution over TSMS}

We have defined the MTSMS model as an structure based on time
series. Now we express the multiresolution as a query on a time series
using TSMS operations. 

Given a time series $S$ and a multiresolution time series $M$, a MTSMS
consolidates a multiresolution schema for $S$ by applying successively
$\forall m \in S: M=\addM(M,m)$ and then $M=\consM(M')$ until $M$ is
no more consolidable. Then over this consolidated $M$ we have defined
two basic queries $\seriedisc(M,\delta,f)$ and $\totalseries(M)$.

The time series $S_D=\seriedisc(M,\delta,f)$ stored in a resolution
subseries' disc $R=(B,D)$, where $B=(S_D,\tau,\delta,f)$ and
$D=(S_D,k)$, can be expressed with a map operation over the original
time series $S$:
\[
\seriedisc(M,\delta,f) \equiv \dmap(S,\delta,\tau,f,k)
\]
\[
\dmap(S,\delta,\tau,f,k) = \map(S_I, m_i\mapsto m') \text{ where }
\]
\[
 m' = (T(m_i), f(S, i)),\;  i = [T(\prev_{S_I}(m_i)),T(m_i)],
\]
\[
 S_I = \{ (t,\infty) | t\in T_I  \},\;  t_M = T(\max(S)),
\]
\[
T_I = \{ t_I = \tau+n\delta | n\in\N, t_M - k\delta < t_I \leq t_M \}
\]


The total time series $S'=\totalseries(M)$ is a concatenation of all
$\seriedisc$ without repeated $\delta$ can be expressed with a fold
operation over the original time series $S$. Let $S_\delta = \{
(\delta_0,\tau_0,f_0,k_0), (\delta_1,\tau_1,f_1,k_1), \ldots,
(\delta_d,\tau_d,f_d,k_d)\}$ be the parameters of $M$ expressed as a
multivalued time series 
\[
\totalseries(M) \equiv \multiresolution(S)
\]
\[
\multiresolution(S) = \orderfold(S_\delta,\{\},f_c,\min) \text{ where }
\]
\[
f_c: S_i \times (\delta_c,\tau_c,f_c,k_c) \mapsto S_i ||
\dmap(S,\delta_c,\tau_c,f_c,k_c)
\]



Summarising, a MTSMS and the multiresolution query are equivalent on
functionality.



\subsection{Two database structures}


\todo{citar stonebraker05 The 8 Requirements of Real-Time Stream Processing}


\begin{figure}
  \centering
  %\usetikzlibrary{shapes,arrows,positioning}
\begin{tikzpicture}[scale=0.8, every node/.style={transform shape}]

 \node (m) {measure};

 \node[rectangle,draw,above right=1cm of m] (tsms) {TSMS};
 \node[rectangle,draw,below right=1cm of m] (mtsms) {MTSMS};

 \node[right=4.3cm of m] (ts) {
   \begin{tabular}[h]{|c|c|}
    \multicolumn{2}{c}{$S'$} \\ \hline
     t & v \\ \hline
       &   \\
       &   \\ \hline
   \end{tabular}
 };

 \draw[->] (m.east) -- (tsms.west);
 \draw[->] (m.east) -- (mtsms.west);

 \draw[->] (tsms.east) -- (ts) node[above,midway,sloped] {multiresolution(S)};
 \draw[->] (mtsms.east) -- (ts) node[above,midway,sloped] {TotalSeries(M)};

\end{tikzpicture}



%%% Local Variables:
%%% TeX-master: "../main"
%%% ispell-local-dictionary: "british"
%%% End:

  \caption{Two forked TSMS+MTSMS}
  \label{fig:model:mtsms-tsms}
\end{figure}


Figure~\ref{fig:model:mtsms-tsms}


* Important! dir que si seguim l'addició amb ordre de les mesures, aleshores es pot fer l'stream en els MTSMS ja que no hi ha possibilitat d'operacions d'UPDATE.

\todo{}
* Esquema de treball amb TSMS i MTSMS:


        +----+  consultaMultiresolution
        |TSMS| --------------> 
     /  +----+
addM                              fan el mateix les dues vies però
     \  +-----+  total            a sota es pre-calcula com a stream
        |MTSMS| -----------> 
        +-----+





\subsection{Stream orientation}

\todo{}
* Dues variacions possibles interessants pels MTSMS:

 
  - Buffers com a streams, sempre de mida fitada 
  - Discos enllaçats










%%% Local Variables:
%%% TeX-master: "main"
%%% ispell-local-dictionary: "british"
%%% End:

% LocalWords:  genericity multiresolution subseries consolidable MTSM

% LocalWords:  pathologies MTSMS TSMS cardinality multivalued infimum
% LocalWords:  multivalues supremum

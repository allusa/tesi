


\section{Introduction}

TSMS vs temporal data: While most databases tend to model reality at a point in time (at the
“current” time), temporal databases model the states of the real world
across time.
 The transaction time for a fact is the time interval during which the fact
is current within the database system.
 In a temporal relation, each tuple has an associated time when it is true;
the time may be either valid time or transaction time.
 A bi-temporal relation stores both valid and transaction time.




% SSDBM:

% \cite{dreyer94b}


% Jensen:

% jensen99:temporaldata

% jensen00:thesis

% jensen98:temporal_database_glossary

% atzeni13:relational_model_dead

% Kersten:

% sciql

% zhang11

% kersten11



\todo{paper structure}


\section{Features of multiresolution}


\todo{Definicions que cal tenir:}

* \acro{MTSMS}
* \acro{TSMS}
* \acro{DBMS}
* time series representation
* time series pathologies


La definició del model s'estructura en dues parts:

\begin{itemize}
\item Un model pels (SGST)  que defineix mesura i sèrie temporals.
\item Un model pels (SGSTM) que defineix buffer, disc i subsèrie
  resolució, el qual treballa sobre el model de SGST.
\end{itemize}

% \todo{sobre tres nivells}
% A l'estat de l'art s'ha d'haver explicat els tres nivell de model de dades segons Date i deixar clar aquí que nosaltres definim un model pel segon nivell: nivell de model lògic. Els models lògics modelen les dades, en canvi els models conceptuals modelen la realitat, Fabian Pascal posa d'exemple conceptual el model E/RM.


Objectius:

En el model de SGST s'observen algunes patologies que poden presentar les sèries temporals. El model de SGSTM soluciona algunes d'aquestes patologies:

\begin{itemize}
\item Regularitza les sèries temporals
\item Tracta i validar les sèries temporals: gestiona els casos de dades errònies o desconegudes i marca quan hi ha valors erronis.
\item És una solució de compressió per a quantitats enormes de dades
\end{itemize}


Però el model de SGSTM també es pot fer servir per altres aplicacions:

* Regularitzar en línia (temps real) una sèrie temporal en diferents períodes de mostreig

* Tenir unes vistes (consultes) a punt (ja processades) amb diferents resolucions d'una sèrie temporal

* Comprimir per decimació (downsampling) o bé farcir forats (reconstrucció del senyal)


Dir que és interessant complementar els TSMS i MTSMS amb altra informació (localització del sensor, unitats del valors, etc.) però que aquesta informació és millor guardar-la en un SGBD relacional.




%%% Local Variables:
%%% TeX-master: "main"
%%% ispell-local-dictionary: "british"
%%% End:

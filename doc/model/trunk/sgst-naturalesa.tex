\section{Naturalesa de les sèries temporals}


Perquè RRDtool diferencia entre comptadors i magnituds?

[segev87] diferencia entre step-wise constant, discret (potser aquest tal com se'l defineix són intervals temporals), continu. Ho anomena (semantic behavior, property, interpret)
 tipus de la sèrie temporal i diu que es poden definir interpolacions per cada una.




[John G. Proakis, Dimitris G. Manolakis 2007 Tratamiento digital de señales/Digital signal processing 4a ed pp11-12(segons wikipedia)] Acquisition: Discrete signals may have several origins, but can usually be classified into one of two groups:[1]
*By acquiring values of an analog signal at constant or variable rate. This process is called sampling.[2]
*By recording the number of events of a given kind over finite time periods. For example, this could be the number of people taking a certain elevator every day.



\subsubsection{Regularitat de les sèries temporals} 

Sigui $S=\{m_0,\ldots,m_k\}$ una sèrie temporal, $t$ un instant de
temps i $\delta$ una durada de temps, les mesures de la sèrie temporal
es poden localitzar en l'interval de temps $i_0=[t,t+\delta]$ i els
seus múltiples $i_j=[t+j\delta \,,\, t+(j+1)\delta]$ per $j=0,1,2,\ldots$.
En processat de senyal aquests intervals de temps s'anomenen intervals
de mostreig, $\delta$ s'anomena període de mostreig i $t$ s'anomena
temps inicial del mostreig.  La sèrie temporal $S$ és de naturalesa
diferent segons la situació dels temps $T(m_i)$ en els intervals de
temps $i_j$.

Una sèrie temporal és regular quan les mesures són equidistants en el
temps, tal com ho anomenen a \cite{last:hetland}.

\begin{definition}[Sèrie temporal regular]
  \label{def:st:regular}
  Sigui $S=\{m_0,\ldots,m_k\}$ una sèrie temporal, $t$ un instant de
  temps i $\delta$ una durada de temps. Direm que $S$ és regular si i
  només si $\forall m \in S(T(\min(S),\infty):T(m) - T(\ant(m)) =
  \delta$ i $T(\min(S))=t$.
\end{definition}

Si una sèrie temporal és regular, l'anomenem sèrie temporal mostrejada
regularment amb període de mostreig $\delta$ iniciada a $t$. Si el $t$
pot ser qualsevol llavors simplement l'anomenem sèrie temporal
mostrejada regularment amb període de mostreig $\delta$.

Noteu que si es complís
la definició excepte que no s'iniciés en el temps que exigim
$T(\min(S))=t$, aleshores la sèrie temporal seria equidistant però a
efectes de mostreig no la podríem anomenar regular; sí que seria una
sèrie temporal de temps real (v.\ def.~\ref{def:st:tempsreal}).


Una sèrie temporal és no regular quan no és regular. 
En les sèries temporals no regulars es poden distingir tres casos: temps real, ultramostreig i inframostreig.

Una sèrie temporal és de temps real quan a cada interval de mostreig hi ha una i només una mesura. L'interval de mostreig pot estar acotat per una durada anomenada termini.

\begin{definition}[Sèrie temporal de temps real]\label{def:st:tempsreal}
  Sigui $S=\{m_0,\dotsc,m_k\}$ una sèrie temporal, $t$ un instant de
  temps, $\delta$ una durada de temps i $D$ una durada que indica
  termini. Direm que $S$ és de temps real si i només si $D\leq\delta$
  i $\forall n\in\{0,\ldots,|S|-1\}: \exists!m \in
  S(t+n\delta,t+n\delta+D]$.  Aleshores la sèrie temporal està
  mostrejada en temps real per al temps de mostreig $\delta$ amb
  compliment del termini $D$.
\end{definition}

Si una sèrie temporal és de temps real, l'anomenem  sèrie temporal mostrejada
en temps real amb període de mostreig $\delta$ i compliment del termini $D$.
Si $D=\delta$, es pot anomenar que $S$ és una sèrie temporal de temps real sense termini.


% \paragraph{Ultramostreig} Una sèrie temporal està ultramostrejada (\emph{upsampling}) quan a cada interval de mostreig hi ha una mesura o més d'una. 
% \[
% \text{Ultramostrejada?}: \text{Sèrie temporal} \times T_0 \times \delta \longrightarrow \text{Booleà}
% \]

% Una sèrie temporal $S$ està ultramostrejada ssi $S$ no és de temps real i $\exists m_i=(v_i,t_i)\in S:T_0+(n-1)\delta \leq t_i < T_0+n\delta:\forall n\in\{1,\ldots,|S|\}$.

% \paragraph{Inframostreig} Una sèrie temporal està inframostrejada (\emph{downsampling}) quan en algun interval de mostreig no hi ha cap mesura. 
% \[
% \text{Inframostrejada?}: \text{Sèrie temporal} \times T_0 \times \delta \longrightarrow \text{Booleà}
% \]

% Una sèrie temporal $S$ està inframostrejada ssi $\nexists m_i=(v_i,t_i)\in S:T_0+(n-1)\delta \leq t_i < T_0+n\delta:\forall n\in\{1,\ldots,|S|\}$.








\subsubsection{Representació de les sèries temporals}



La naturalesa indueix representacions?
Jo puc utilitzar qualsevol representació donada una sèrie temporal, però això em pot causa perjudici si no s'adiu amb la naturalesa.


La representació serveix per interpolar:

zoh, zoh cap enrere, lineal, etc.


Una sèrie temporal és la representació discreta d'una funció contínua. A partir de la sèrie temporal es pot definir una funció contínua. 

A teoria de senyal s'estudia com fer que aquesta s'aproximi a la real. Estudiant com a senyal fan: donada una sèrie temporal dir quina funció s'hi 'ajusta' més. 

Però jo puc preguntar donada una sèrie temporal quina funció representa i puc dir per representar a zohe és tal, per representar a lineal és qual. 

Potser millor dir-li interpretació?



\paragraph{Representació de sèries temporals}

\textcite{last:keogh}, cita vàries representacions per les sèries temporals com per exemple \emph{Fourier Transforms}, \emph{Wavelets}, \emph{Symbolic Mappings} o \emph{Piecewise Linear Representation} (PLR), però assenyala aquesta última com la representació més utilitzada. 
La PLR, funció definida a trossos lineal, és l'aproximació d'una sèrie temporal $S$, de llargada $n$, amb $K$ segments rectes. Els segments podrien ser polinomis de qualsevol grau, però la manera més comuna de representar sèries temporals és amb funcions lineals, segons Keogh, \cite{keogh02}.
Per aproximar el segment $S(t_a:t_b]$ d'una sèrie $S$, Keogh defineix dues tècniques: interpolació lineal, la recta que connecta $t_a$ i $t_b$, i regressió lineal, la millor recta que aproxima per mínims quadrats el segment entre $t_a$ i $t_b$.

Però també es pot representar una sèrie temporal amb una funció esglaó (\emph{step} o \emph{staircase function}); és a dir, amb una funció definida a trossos constant (\emph{piecewise constant representation}).
La representació a trossos constant és utilitzada en electrònica als convertidors digital-analògic (DAC, \emph{digital-to-analog converter}). En aquest cas, un senyal discret es considera una sèrie temporal i per reconstruir el senyal continu típicament s'aplica el model de \emph{zero-order hold}, equivalent a la representació a trossos constant,  o el de \emph{first-order hold},  equivalent a la representació a trossos lineal.
El model de \emph{zero-order hold} consisteix en mantenir constant cada valor fins al proper. S'obté una representació a trossos constant que en electrònica s'anomena seqüència de pulsos rectangulars (\emph{rectangular pulses}).

%http://en.wikipedia.org/wiki/Piecewise

%http://ca.wikipedia.org/wiki/Funció_definida_a_trossos

%http://en.wikipedia.org/wiki/Rectangular_function

%http://en.wikipedia.org/wiki/Step_function

% Piecewise Aggregate Approximation (PAA) \cite{keogh00}: aproxima una sèrie temporal partint-la en segments de la mateixa mida i emmagatzemant la mitjana dels punts que cauen dins del segment. Redueix de dimensió $n$ a dimensió $N$

% Adaptive Piecewise Constant Approximation (APCA) \cite{keogh01}: com el PAA però amb segments de mida variable.

A continuació,  la representació  d'una sèrie temporal segons el model de \emph{zero-order hold} s'estén per diferents continuïtats en els intervals de temps de representació.

Sigui $S$ una sèrie temporal, es defineix $S(t)$ com la representació
de la sèrie temporal contínuament al llarg del temps $t$.  En primer
lloc, es representa amb \emph{zero-order hold} a partir de funcions
graó contínues per la dreta (\emph{right-continuous}).

\begin{definition}[Representació amb \emph{zero-order hold}]
Sigui $S=\{m_0,\ldots,m_k\}$ una sèrie temporal, la representació  $S(t)$ amb \emph{zero-order hold} es defineix
\[
\forall t \in \mathbb{R} ,\forall m \in S: S(t) =
\begin{cases}
  V(\min S) & \text{si } t < T(\min S) \\
  V(m) & \text{si }  t\in [T(m),T(\seg m))
\end{cases}
\]
\end{definition}

En segon lloc, es representa $S(t)$ amb \emph{zero-order hold} centrada en
l'interval, definit també a partir de funcions graó contínues per la
dreta.

\begin{definition}[Representació amb \emph{zero-order hold} centrada en l'interval]
  Sigui $S=\{m_0,\ldots,m_k\}$ una sèrie temporal, la representació
  $S(t)$ amb \emph{zero-order hold} centrada en l'interval es defineix
\[
\forall t \in \mathbb{R}  ,\forall m \in S:
S(t) =  
\begin{cases}
  V(m) & \text{si } t = \frac{T(\ant m)+T(m)}{2} \\
  V(m) & \text{si } t\in \left( \frac{T(\ant m)+T(m)}{2},\frac{T(m)+T(\seg m)}{2} \right) \
\end{cases}
\]
\end{definition}

En tercer lloc, es representa $S(t)$ amb \emph{zero-order hold} cap enrere, ara definit a partir de funcions graó contínues per l'esquerra.
\begin{definition}[Representació en \emph{zero-order hold} cap enrere]
  Sigui $S=\{m_0,\ldots,m_k\}$ una sèrie temporal, la representació
  $S(t)$ amb \emph{zero-order hold} cap enrere es defineix
\[
\forall t \in \mathbb{R}  ,\forall m \in S:
S(t) =  
\begin{cases}
  V(\max S) & \text{si } t > T(\max S) \\
  V(m) & \text{si }  t\in (T(\ant m),T(m)]
\end{cases}
\]
\end{definition}

Sigui $S$ una sèrie temporal regular i $\delta$ una durada de temps, aleshores la representació de $S(t)$ amb \emph{zero-order hold} és la mateixa que la de $S(t-\delta)$ amb \emph{zero-order hold} cap enrere i és la mateixa que la de $S(t-\frac{\delta}{2})$ centrada en l'interval. 




\subsubsection{Equivalència en representació}


$S_1=\{(1,1),(3,0),(5,1)\}$ i $S_2=\{(1,1),(2,0),(3,0)(4,1),(5,1)\}$ són equivalents amb representació zohe però no són equivalents amb representació lineal ja que hauria de ser $S_2'=\{(1,1),(2, 0{,}5 ),(3,0),(4, 0{,}5),(5,1)\}$. 




%%% Local Variables:
%%% TeX-master: "main"
%%% End:







% LocalWords:  SGST

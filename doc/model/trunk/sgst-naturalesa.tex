\section{Propietats (Naturalesa/tipologia/patologies) de les sèries temporals}


\todo{}



\subsection{Naturalesa de les sèries temporals}

\todo{}


segons el que es diu a la secció d'operacions (però això potser és més de model?):

* conjunt
* seqüència
* funció temporal

segons adquisició:

* contínua
* discreta


segons tipus de dades mesurades:

* comptadors -> conservar totals

* soroll d'un senyal -> conservar mitjana

* temperatura -> aplicar a priori la DFT i determinar deltes bons per a després poder interpolar mitjanes (processat digital del senyal)


El valor d’una magnitud física es pot calcular com el producte d’un valor nu-
mèric per una unitat:
magnitud física = valor numèric x unitat [llibreverd IEC-IUPAC]



Perquè RRDtool diferencia entre comptadors i magnituds?

[segev87] diferencia entre step-wise constant, discret (potser aquest tal com se'l defineix són intervals temporals), continu. Ho anomena (semantic behavior, property, interpret)
 tipus de la sèrie temporal i diu que es poden definir interpolacions per cada una.




[John G. Proakis, Dimitris G. Manolakis 2007 Tratamiento digital de señales/Digital signal processing 4a ed pp11-12(segons wikipedia)] Acquisition: Discrete signals may have several origins, but can usually be classified into one of two groups:[1]
*By acquiring values of an analog signal at constant or variable rate. This process is called sampling.[2]
*By recording the number of events of a given kind over finite time periods. For example, this could be the number of people taking a certain elevator every day.










\subsection{Graf i funció temporal de representació}
\label{sec:model:repr}

Una sèrie temporal és la representació discreta d'una funció contínua;
la qual és una funció temporal atès que depèn del temps. El model de
SGST definit anteriorment formalitza la sèrie temporal com aquesta
representació discreta. Però a partir d'una sèrie temporal es pot
voler interpretar quina era la funció contínua original; és a dir
obtenir nous valors segons una funció a la que s'anomena graf d'una
funció (\emph{graph of a function}), en el sentit de gràfic
(\emph{plot}) que cal no confondre amb els grafs d'arestes i vèrtexs
(\emph{vertex-edge graph}).

\begin{definition}[Graf d'una sèrie temporal]%an. graph
  Sigui $S=\{m_0,\ldots,m_k\}$ una sèrie temporal i $T$ un domini del
  temps, es defineix el graf de la sèrie temporal
  $\glssymboldef{not:sgst:graf} S(t)$ com un conjunt de parells
  ordenats $(t,S(t)$ : $\glssymboldef{not:sgst:graf} S(t) = \{ (t,S(t)) |
  t\in T \}$ a on $S(t)$ és una funció de representació de la sèrie
  temporal.
\end{definition}

Per a calcular el graf d'una sèrie temporal es necessita una funció de
representació. Mentre que la funció que permet canviar d'una funció
contínua a una sèrie temporal s'anomena procés d'adquisició o
mostreig, la funció que permet canviar d'una sèrie temporal a una
funció contínua l'anomenem funció de representació.  Així doncs,
donada una sèrie temporal es poden definir funcions amb el temps com a
variable que calculin nous valors a partir de les mesures
emmagatzemades.
\begin{definition}[Funció de representació]
  Sigui $S=\{m_0,\ldots,m_k\}$ una sèrie temporal, es defineix $S(t)$
  com la funció de representació de la sèrie temporal contínuament al
  llarg del temps $t$; és a dir que per cada instant de temps la
  funció pren un valor: $\forall t\in T: v(t) = S(t)$. 

  Atenent a les operacions de càlcul que es facin per a obtenir $S(t)$
  diem que hi ha diverses funcions de representació. Així per a cada
  funció de representació indicarem a quina ens referim amb un
  superíndex \glsdispdef{not:sgst:frepr}{$S^r(t)$} a on
  \glssymboldef{not:sgst:repr} és el nom d'una funció de representació
  de la sèrie temporal.
\end{definition}

La utilitat de les funcions de representació és diversa i per això les
operacions de càlcul poden ser qualssevol. Una funció de
representació es pot utilitzar per a interpolar valors d'una sèrie
temporal però també per a extrapolar-los o per a canviar la resolució
de la sèrie temporal. També es pot utilitzar per al problema aproximar
la sèrie temporal a la funció original; és a dir donar una funció de
representació que sigui la funció contínua que més s'aproxima a la
funció temporal original.


El lligam entre una sèrie temporal i la seva representació no és fix;
és a dir que donada una sèrie temporal es pot representar amb una
funció o amb una altra segons convingui.  Encara que en alguns àmbits,
per exemple a teoria del senyal o en algunes aplicacions de l'anàlisi
de sèries temporals, l'objectiu és cercar la parella de sèrie temporal
i representació que més s'aproxima a la funció contínua original; en
altres àmbits, com per exemple el del model multiresolució que definim
posteriorment, l'objectiu està més orientat a utilitzar
representacions de les sèries temporals segons els càlculs que es
volen fer o segons la naturalesa que s'assumeixi de la sèrie temporal.


No obstant, les funcions de representació s'han d'utilitzar amb
criteri. Per exemple la naturalesa de la sèrie temporal indueix a unes
possibles operacions de càlcul que es poden realitzar, per tant
l'aplicació de qualsevol funció de representació a la sèrie temporal
pot donar resultats incoherents. O bé un altre exemple és aplicar
càlculs successius a una sèrie temporal seguint diferents funcions de
representació.  Així i tot, en la formalització del model de funcions
de representacions no hi definim cap criteri en concret per a, així,
donar llibertat en les possibilitats de càlcul amb les sèries
temporals.


A continuació, definim diverses funcions de representació per a
exemplificar-ne l'ús. Les agrupem per algunes de les seves
característiques, a les qual podem veure com famílies de funcions de
representació. Per a cada definició una oferim, quan es pugui, dues
expressions equivalents: una en matemàtica contínua, que ajuda a
comprendre el significat, i una en matemàtica discreta, que utilitza
l'àlgebra del model de SGST.

\todo{fer exemple de cada família: posar una sèrie temporal i dibuixar el graf segons cada representació}



\subsubsection{Funcions parcials}
\todo{no sé si aporten res les parcials? cal acabar la secció d'agregadors dels SGSTM i repensar això}

Primerament, definim una funció de representació anomenada discreta
pura que no és totalment contínua en el temps sinó que és una funció
parcial.
\begin{definition}[Funció de representació discreta pura]
  Sigui $S=\{m_0,\ldots,m_k\}$ una sèrie temporal, es defineix
  $S^d(t)$ com la funció de representació discreta pura de la sèrie
  temporal $\forall m \in S: S^d(t) =
  \begin{cases}
    V(m) & \text{si }  t=T(m) \\
    \text{no definit} & \text{altrament}
  \end{cases}$.
\end{definition}

Aquest és un cas especial de funció de representació perquè permet que
el graf de la sèrie temporal sigui equivalent a les mesures de la
sèrie temporal: $\glssymbol{not:sgst:graf} S^d(t) \equiv
\{m_0,\ldots,m_k\}$.

Es poden definir altres funcions de representació de la família
parcial, però presenten el problema que el domini queda restringit a
un subconjunt $T'$ del domini temps $T$; el domini queda restringit
als instants de temps elegits $T'$ ja que per qualsevol altre instant de
$T$ no hi ha imatge definida.

Així doncs, és millor definir funcions totals que sempre seran
funcions ben-definides per al domini temps; per això a la definició de
funcions de representació hem preferit parlar només de funcions
contínues al llarg del temps. 
\todo{sinó, es poden trobar equivalents en les contínues?}




\subsubsection{A impulsos}

Una família de funcions contínues que recorda a la funció discreta són
les funcions d'impulsos (\emph{impulse train function}).  A
continuació ho exemplifiquem amb una representació que anomenem delta
de Dirac ($\delta$) perquè es basa en la funció homònima, la qual val
zero a tot arreu excepte en el punt zero.

\begin{definition}[Funció de representació delta de Dirac]
  Sigui $S=\{m_0,\ldots,m_k\}$ una sèrie temporal i $T$ el domini del
  temps, es defineix $S^\delta(t)$ com la funció de representació
  delta de Dirac al llarg del temps, $\forall m \in S:$
  \begin{align*}
    S^\delta(t) = &  \\
    = & \sum_{t\in T} V(m) \delta(t-T(m))  \\
    = & \begin{cases}
      V(m) & \text{si }  t=T(m) \\
      0 & \text{altrament}
    \end{cases}
         \end{align*}.
\end{definition}





\subsubsection{A trossos constants}

Una altra família de funcions són les que es basen en funcions
definides a trossos constants (\emph{piecewise constant functions}).
A continuació ho exemplifiquem amb quatre representacions basades en la
funció graó (\emph{step function} o \emph{staircase function}) atenent
a quatre de les possibles continuïtats en els intervals de temps. 



%(\emph{right-continuous})
En primer lloc, definim una representació en base a funcions graó
contínues per la dreta. L'anomenem representació \emph{zero-order
  hold}(ZOH) a causa de la semblança que té amb el model utilitzat en
electrònica per a reconstruir senyals, el qual consisteix en mantenir
constant cada valor fins al proper.
\begin{definition}[Funció de representació \emph{zero-order hold}]
  Sigui $S=\{m_0,\ldots,m_k\}$ una sèrie temporal i $T$ el domini del
  temps, es defineix $S^\text{ZOH}(t)$ com la funció de representació
  \emph{zero-order hold} al llarg del temps, $\forall m \in S:$
  \begin{align*}
    S^\text{ZOH}(t) = &  \\
    = & \sum_{t\in T} V(m) X_d(t): X_d(t) = 
    \begin{cases}
      1 & \text{si } t\in \big[T(m),T(\glssymbol{not:sgst:next}(m))\big) \\
      0 & \text{altrament}
    \end{cases} \\
    = & \begin{cases}
      0 & \text{si }  t<T(\min(S)) \\
      V(m) & \text{si } t\in \big[T(m),T(\glssymbol{not:sgst:next}(m))\big)
    \end{cases}
         \end{align*}.
\end{definition}



En segon lloc, definim una representació en base a funcions graó
contínues per l'esquerra. L'anomenem representació \emph{zero-order
  hold} cap enrere (ZOHE) perquè consisteix en mantenir constant cada
valor fins al predecessor. Una representació similar s'utilitza a
\gls{RRDtool} \parencite{lisa98:oetiker}.
\begin{definition}[Funció de representació \emph{zero-order hold} cap
  enrere]%\emph{zero-order hold backwards}(zohe%from \emph{zero-order
         %hold everted}
  Sigui $S=\{m_0,\ldots,m_k\}$ una sèrie temporal i $T$ el domini del
  temps, es defineix $S^\text{ZOHE}(t)$ com la funció de representació
  \emph{zero-order hold} cap enrere al llarg del temps, $\forall m \in S:$
  \begin{align*}
    S^\text{ZOHE}(t) = &  \\
    = & \sum_{t\in T} V(m) X_e(t): X_e(t) = 
    \begin{cases}
      1 & \text{si } t\in \big(T(\glssymbol{not:sgst:prev}(m)),T(m)\big] \\
      0 & \text{altrament}
    \end{cases} \\
    = & \begin{cases}
      0 & \text{si }  t > T(\max(S)) \\
      V(m) & \text{si } t\in \big(T(\glssymbol{not:sgst:prev}(m)),T(m)\big]
    \end{cases}
         \end{align*}.
\end{definition}



En tercer lloc, definim una representació en base a funcions graó
contínues per la dreta centrades en l'interval. L'anomenem
representació \emph{zero-order hold} centrada en l'interval (ZOHC).




\begin{definition}[Representació amb \emph{zero-order hold} centrada en l'interval]
  Sigui $S=\{m_0,\ldots,m_k\}$ una sèrie temporal, la representació
  $S(t)$ amb \emph{zero-order hold} centrada en l'interval es defineix
\[
\forall t \in \mathbb{R}  ,\forall m \in S:
S(t) =  
\begin{cases}
  V(m) & \text{si } t = \frac{T(\ant m)+T(m)}{2} \\
  V(m) & \text{si } t\in \left( \frac{T(\ant m)+T(m)}{2},\frac{T(m)+T(\seg m)}{2} \right) \
\end{cases}
\]
\end{definition}



\begin{definition}[Funció de representació \emph{zero-order hold}
  centrada en l'interval]
  Sigui $S=\{m_0,\ldots,m_k\}$ una sèrie temporal i $T$ el domini del
  temps, es defineix $S^\text{ZOHC}(t)$ com la funció de representació
  \emph{zero-order hold} centrada en l'interval al llarg del temps,
  $\forall m \in S:$
  \begin{align*}
    S^\text{ZOHC}(t) = &  \\
   = & \sum_{t\in T} V(m) X_c(t):
    X_c(t) = 
    \begin{cases}
      1 & \text{si } t\in \left[
        \frac{T(\glssymbol{not:sgst:prev}(m))+T(m)}{2},
        \frac{T(m)+T(\glssymbol{not:sgst:next}(m))}{2}
      \right) \\
      0 & \text{altrament}
    \end{cases}  \\
    = & V(m): t\in \left[
        \frac{T(\glssymbol{not:sgst:prev}(m))+T(m)}{2},
        \frac{T(m)+T(\glssymbol{not:sgst:next}(m))}{2} \right)
         \end{align*}.
\end{definition}




En quart lloc, representem la sèrie temporal en base a la funció
rectangular. La funció rectangular és un cas especial de les funcions
graó a on s'especifiquen valors simètrics per als punts de
discontinuïtat.
\begin{definition}[Funció de representació rectangular]
  Sigui $S=\{m_0,\ldots,m_k\}$ una sèrie temporal i $T$ el domini del
  temps, es defineix $S^\text{rect}(t)$ com la funció de representació
  rectangular al llarg del temps, $\forall m \in S:$
  \begin{align*}
    S^\text{rect}(t) = &  \\
    = & \sum_{t\in T} V(m) \operatorname{rect}(t):  \operatorname{rect}(t) = 
    \begin{cases}
      1 & \text{si } t\in \big(T(m),T(\glssymbol{not:sgst:next}(m))\big) \\
      \frac{1}{2}& \text{si } t = T(m) \vee t=T(\glssymbol{not:sgst:next}(m)) \\
      0 & \text{altrament}
    \end{cases} \\
    = & \begin{cases}
      0 & \text{si }  t<T(\min(S)) \\
      V(m) & \text{si } t\in \big(T(m),T(\glssymbol{not:sgst:next}(m))\big) \\
      \frac{V(m)+V(\ant(m))}{2} & \text{si } t = T(m) \wedge t > T(\min(S)) \\
      \frac{V(m)}{2} & \text{si } t = T(\min(S)) \\
    \end{cases}
         \end{align*}.
\end{definition}




\todo{descriure la propietat de translació de les tres representacions}
i dir que la rect només n'hem definit una però que també té les tres translacions possibles.

Sigui $S$ una sèrie temporal regular i $\delta$ una durada de temps, aleshores la representació de $S(t)$ amb \emph{zero-order hold} és la mateixa que la de $S(t-\delta)$ amb \emph{zero-order hold} cap enrere i és la mateixa que la de $S(t-\frac{\delta}{2})$ centrada en l'interval. 






\begin{example}[Sèrie temporal amb representació ZOHE]
  Sigui la sèrie temporal $S=\{ (1,1), (3,3), (4,2), (8,1) \}$, la seva representació ZOHE és  $S^\text{ZOHE}(t) = 1X_{(-\infty,1}+3X_{1,3}+2X_{3,4}+1X_{4,8}+3X_{8,+\infty}$


 el
  graf de la seva representació ZOHE, $\glssymbol{not:sgst:graf}
  S^\text{ZOHE}(t)$ es mostra a la figura\todo{}.

\end{example}




\subsubsection{A trossos lineals}

\todo{}

%http://en.wikipedia.org/wiki/First-order_hold





\subsubsection{Altres}

\todo{}


\todo{potser posar-ho a estat de l'art tot això?}


In the design of the attribute interpolation function we can interpret
a time series in different ways, that is what we call the
representation of a time series. Keogh et al.\ \cite{last:keogh} cite
some possible representations for time series such as \emph{Fourier
  Transforms}, \emph{Wavelets}, \emph{Symbolic Mappings} or
\emph{Piecewise Linear Representation} (PLR). This last is remarked as
the most used owing to the most common representation is with linear
functions \cite{keogh01}.

\paragraph{Representació de sèries temporals}

\textcite{last:keogh}, cita vàries representacions per les sèries temporals com per exemple \emph{Fourier Transforms}, \emph{Wavelets}, \emph{Symbolic Mappings} o \emph{Piecewise Linear Representation} (PLR), però assenyala aquesta última com la representació més utilitzada. 
La PLR, funció definida a trossos lineal, és l'aproximació d'una sèrie temporal $S$, de llargada $n$, amb $K$ segments rectes. Els segments podrien ser polinomis de qualsevol grau, però la manera més comuna de representar sèries temporals és amb funcions lineals, segons Keogh, \cite{keogh02}.
Per aproximar el segment $S(t_a:t_b]$ d'una sèrie $S$, Keogh defineix dues tècniques: interpolació lineal, la recta que connecta $t_a$ i $t_b$, i regressió lineal, la millor recta que aproxima per mínims quadrats el segment entre $t_a$ i $t_b$.

Però també es pot representar una sèrie temporal amb una funció graó (\emph{step} o \emph{staircase function}); és a dir, amb una funció definida a trossos constant (\emph{piecewise constant representation}).
La representació a trossos constant és utilitzada en electrònica als convertidors digital-analògic (DAC, \emph{digital-to-analog converter}). En aquest cas, un senyal discret es considera una sèrie temporal i per reconstruir el senyal continu típicament s'aplica el model de \emph{zero-order hold}, equivalent a la representació a trossos constant,  o el de \emph{first-order hold},  equivalent a la representació a trossos lineal.
El model de \emph{zero-order hold} consisteix en mantenir constant cada valor fins al proper. S'obté una representació a trossos constant que en electrònica s'anomena seqüència de pulsos rectangulars (\emph{rectangular pulses}).

%http://en.wikipedia.org/wiki/Piecewise

%http://ca.wikipedia.org/wiki/Funció_definida_a_trossos

%http://en.wikipedia.org/wiki/Rectangular_function

%http://en.wikipedia.org/wiki/Step_function

% Piecewise Aggregate Approximation (PAA) \cite{keogh00}: aproxima una sèrie temporal partint-la en segments de la mateixa mida i emmagatzemant la mitjana dels punts que cauen dins del segment. Redueix de dimensió $n$ a dimensió $N$

% Adaptive Piecewise Constant Approximation (APCA) \cite{keogh01}: com el PAA però amb segments de mida variable.



\subsubsection{Ús de la representació en els operadors}

\todo{}

Com lliguen els operadors de sèries temporals amb les funcions de
representació? De fet no poden utilitzar-se mútuament: les funcions de
representació són una funció contínua i voler que això aparegués al
model voldria dir aplicar mètodes numèrics. És a dir és incompatible
amb l'àlgebra? Així el que fem és definir les funcions de
representació per a estudiar exactament què estem fent i després
agafant-ne el significat s'implementen les operacions que siguin
necessàries (de fet només és necessari implementar l'operació
d'interval temporal per a cada representació (com a exemple la def d'interval temporal zohe): les altres es defineixen
a partir d'aquesta.

Per  altra banda,  les representacions  també es  tindran en  compte a
l'hora de definir les interpretacions pels agregadors d'atributs en el
model de SGSTM \todo{ref capítol}

We now define the \emph{zero-order hold backwards} attribute
interpolation function family as the one interpreting the
consolidation time interval left-continuous $i=(T_0,T_f]$ and the
resulting interpolated measure's time always being $T_f$, in
accordance to the \emph{zero-order hold backwards} representation
being defined using left-continuous step functions.  


\subsubsection{Equivalència en representació}
representació o graf?

$S_1=\{(1,1),(3,0),(5,1)\}$ i $S_2=\{(1,1),(2,0),(3,0)(4,1),(5,1)\}$ són equivalents amb representació zohe però no són equivalents amb representació lineal ja que hauria de ser $S_2'=\{(1,1),(2, 0{,}5 ),(3,0),(4, 0{,}5),(5,1)\}$. 















\subsection{Regularitat de les sèries temporals} 
\todo{fer-ho subsecció de les patologies}


Sigui $S=\{m_0,\ldots,m_k\}$ una sèrie temporal, $t$ un instant de
temps i $\delta$ una durada de temps, les mesures de la sèrie temporal
es poden localitzar en l'interval de temps $i_0=[t,t+\delta]$ i els
seus múltiples $i_j=[t+j\delta \,,\, t+(j+1)\delta]$ per $j=0,1,2,\ldots$.
En processat de senyal aquests intervals de temps s'anomenen intervals
de mostreig, $\delta$ s'anomena període de mostreig i $t$ s'anomena
temps inicial del mostreig.  La sèrie temporal $S$ és de naturalesa
diferent segons la situació dels temps $T(m_i)$ en els intervals de
temps $i_j$.

Una sèrie temporal és regular quan les mesures són equidistants en el
temps, tal com ho anomenen a \cite{last:hetland}.

\begin{definition}[Sèrie temporal regular]
  \label{def:st:regular}
  Sigui $S=\{m_0,\ldots,m_k\}$ una sèrie temporal, $t$ un instant de
  temps i $\delta$ una durada de temps. Direm que $S$ és regular si i
  només si $\forall m \in S(T(\min(S),\infty):T(m) - T(\ant(m)) =
  \delta$ i $T(\min(S))=t$.
\end{definition}

Si una sèrie temporal és regular, l'anomenem sèrie temporal mostrejada
regularment amb període de mostreig $\delta$ iniciada a $t$. Si el $t$
pot ser qualsevol llavors simplement l'anomenem sèrie temporal
mostrejada regularment amb període de mostreig $\delta$.

Noteu que si es complís
la definició excepte que no s'iniciés en el temps que exigim
$T(\min(S))=t$, aleshores la sèrie temporal seria equidistant però a
efectes de mostreig no la podríem anomenar regular; sí que seria una
sèrie temporal de temps real (v.\ def.~\ref{def:st:tempsreal}).


Una sèrie temporal és no regular quan no és regular. 
En les sèries temporals no regulars es poden distingir tres casos: temps real, ultramostreig i inframostreig.

Una sèrie temporal és de temps real quan a cada interval de mostreig hi ha una i només una mesura. L'interval de mostreig pot estar acotat per una durada anomenada termini.

\begin{definition}[Sèrie temporal de temps real]\label{def:st:tempsreal}
  Sigui $S=\{m_0,\dotsc,m_k\}$ una sèrie temporal, $t$ un instant de
  temps, $\delta$ una durada de temps i $D$ una durada que indica
  termini. Direm que $S$ és de temps real si i només si $D\leq\delta$
  i $\forall n\in\{0,\ldots,|S|-1\}: \exists!m \in
  S(t+n\delta,t+n\delta+D]$.  Aleshores la sèrie temporal està
  mostrejada en temps real per al temps de mostreig $\delta$ amb
  compliment del termini $D$.
\end{definition}

Si una sèrie temporal és de temps real, l'anomenem  sèrie temporal mostrejada
en temps real amb període de mostreig $\delta$ i compliment del termini $D$.
Si $D=\delta$, es pot anomenar que $S$ és una sèrie temporal de temps real sense termini.


% \paragraph{Ultramostreig} Una sèrie temporal està ultramostrejada (\emph{upsampling}) quan a cada interval de mostreig hi ha una mesura o més d'una. 
% \[
% \text{Ultramostrejada?}: \text{Sèrie temporal} \times T_0 \times \delta \longrightarrow \text{Booleà}
% \]

% Una sèrie temporal $S$ està ultramostrejada ssi $S$ no és de temps real i $\exists m_i=(v_i,t_i)\in S:T_0+(n-1)\delta \leq t_i < T_0+n\delta:\forall n\in\{1,\ldots,|S|\}$.

% \paragraph{Inframostreig} Una sèrie temporal està inframostrejada (\emph{downsampling}) quan en algun interval de mostreig no hi ha cap mesura. 
% \[
% \text{Inframostrejada?}: \text{Sèrie temporal} \times T_0 \times \delta \longrightarrow \text{Booleà}
% \]

% Una sèrie temporal $S$ està inframostrejada ssi $\nexists m_i=(v_i,t_i)\in S:T_0+(n-1)\delta \leq t_i < T_0+n\delta:\forall n\in\{1,\ldots,|S|\}$.
















\subsection{Patologies de les sèries temporals}
\todo{?}

* Problemes en l'adquisició

* Problemes en el rellotge



* Tractament/validació de dades

With reference to data validation, attribute interpolation functions
can cope with this process. When data has not been captured or has
been captured erroneously, it must be treated as unknown data.
\begin{itemize}
\item When data has not been captured it is unknown by nature. For
  example, we try to capture data from a sensor and there is no
  response.
\item When data is erroneously it must be marked as unknown. For
  example, we capture data from a sensor but it responses in a not
  reasonable time or we capture data that is clearly outside a
  reasonable limits.
\end{itemize}
As a consequence, attribute interpolation functions deals with these
two subprocesses: treating unknown data and marking data as
unknown. Following with real numbers, let unknown value be represented
by the improper element infinity ($\infty$).  





%%% Local Variables:
%%% TeX-master: "main"
%%% End:







% LocalWords:  SGST Dirac

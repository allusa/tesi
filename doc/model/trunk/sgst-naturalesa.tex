\section{Propietats (Naturalesa/tipologia/patologies) de les sèries temporals}


\todo{}



\subsection{Naturalesa de les sèries temporals}

\todo{}


segons el que es diu a la secció d'operacions (però això potser és més de model?):

* conjunt
* seqüència
* funció temporal

segons adquisició:

* contínua
* discreta


segons tipus de dades mesurades:

* comptadors -> conservar totals

* soroll d'un senyal -> conservar mitjana

* temperatura -> aplicar a priori la DFT i determinar deltes bons per a després poder interpolar mitjanes (processat digital del senyal)


El valor d’una magnitud física es pot calcular com el producte d’un valor nu-
mèric per una unitat:
magnitud física = valor numèric x unitat [llibreverd IEC-IUPAC]



Perquè RRDtool diferencia entre comptadors i magnituds?

[segev87] diferencia entre step-wise constant, discret (potser aquest tal com se'l defineix són intervals temporals), continu. Ho anomena (semantic behavior, property, interpret)
 tipus de la sèrie temporal i diu que es poden definir interpolacions per cada una.




[John G. Proakis, Dimitris G. Manolakis 2007 Tratamiento digital de señales/Digital signal processing 4a ed pp11-12(segons wikipedia)] Acquisition: Discrete signals may have several origins, but can usually be classified into one of two groups:[1]
*By acquiring values of an analog signal at constant or variable rate. This process is called sampling.[2]
*By recording the number of events of a given kind over finite time periods. For example, this could be the number of people taking a certain elevator every day.










\subsection{Regularitat de les sèries temporals} 
\todo{fer-ho subsecció de les patologies}


Sigui $S=\{m_0,\ldots,m_k\}$ una sèrie temporal, $t$ un instant de
temps i $\delta$ una durada de temps, les mesures de la sèrie temporal
es poden localitzar en l'interval de temps $i_0=[t,t+\delta]$ i els
seus múltiples $i_j=[t+j\delta \,,\, t+(j+1)\delta]$ per $j=0,1,2,\ldots$.
En processat de senyal aquests intervals de temps s'anomenen intervals
de mostreig, $\delta$ s'anomena període de mostreig i $t$ s'anomena
temps inicial del mostreig.  La sèrie temporal $S$ és de naturalesa
diferent segons la situació dels temps $T(m_i)$ en els intervals de
temps $i_j$.

Una sèrie temporal és regular quan les mesures són equidistants en el
temps, tal com ho anomenen a \cite{last:hetland}.

\begin{definition}[Sèrie temporal regular]
  \label{def:st:regular}
  Sigui $S=\{m_0,\ldots,m_k\}$ una sèrie temporal, $t$ un instant de
  temps i $\delta$ una durada de temps. Direm que $S$ és regular si i
  només si $\forall m \in S(T(\min(S),\infty):T(m) - T(\ant(m)) =
  \delta$ i $T(\min(S))=t$.
\end{definition}

Si una sèrie temporal és regular, l'anomenem sèrie temporal mostrejada
regularment amb període de mostreig $\delta$ iniciada a $t$. Si el $t$
pot ser qualsevol llavors simplement l'anomenem sèrie temporal
mostrejada regularment amb període de mostreig $\delta$.

Noteu que si es complís
la definició excepte que no s'iniciés en el temps que exigim
$T(\min(S))=t$, aleshores la sèrie temporal seria equidistant però a
efectes de mostreig no la podríem anomenar regular; sí que seria una
sèrie temporal de temps real (v.\ def.~\ref{def:st:tempsreal}).


Una sèrie temporal és no regular quan no és regular. 
En les sèries temporals no regulars es poden distingir tres casos: temps real, ultramostreig i inframostreig.

Una sèrie temporal és de temps real quan a cada interval de mostreig hi ha una i només una mesura. L'interval de mostreig pot estar acotat per una durada anomenada termini.

\begin{definition}[Sèrie temporal de temps real]\label{def:st:tempsreal}
  Sigui $S=\{m_0,\dotsc,m_k\}$ una sèrie temporal, $t$ un instant de
  temps, $\delta$ una durada de temps i $D$ una durada que indica
  termini. Direm que $S$ és de temps real si i només si $D\leq\delta$
  i $\forall n\in\{0,\ldots,|S|-1\}: \exists!m \in
  S(t+n\delta,t+n\delta+D]$.  Aleshores la sèrie temporal està
  mostrejada en temps real per al temps de mostreig $\delta$ amb
  compliment del termini $D$.
\end{definition}

Si una sèrie temporal és de temps real, l'anomenem  sèrie temporal mostrejada
en temps real amb període de mostreig $\delta$ i compliment del termini $D$.
Si $D=\delta$, es pot anomenar que $S$ és una sèrie temporal de temps real sense termini.


% \paragraph{Ultramostreig} Una sèrie temporal està ultramostrejada (\emph{upsampling}) quan a cada interval de mostreig hi ha una mesura o més d'una. 
% \[
% \text{Ultramostrejada?}: \text{Sèrie temporal} \times T_0 \times \delta \longrightarrow \text{Booleà}
% \]

% Una sèrie temporal $S$ està ultramostrejada ssi $S$ no és de temps real i $\exists m_i=(v_i,t_i)\in S:T_0+(n-1)\delta \leq t_i < T_0+n\delta:\forall n\in\{1,\ldots,|S|\}$.

% \paragraph{Inframostreig} Una sèrie temporal està inframostrejada (\emph{downsampling}) quan en algun interval de mostreig no hi ha cap mesura. 
% \[
% \text{Inframostrejada?}: \text{Sèrie temporal} \times T_0 \times \delta \longrightarrow \text{Booleà}
% \]

% Una sèrie temporal $S$ està inframostrejada ssi $\nexists m_i=(v_i,t_i)\in S:T_0+(n-1)\delta \leq t_i < T_0+n\delta:\forall n\in\{1,\ldots,|S|\}$.











\subsection{Patologies de les sèries temporals}
\todo{?}

* Problemes en l'adquisició

* Problemes en el rellotge



* Tractament/validació de dades

With reference to data validation, attribute interpolation functions
can cope with this process. When data has not been captured or has
been captured erroneously, it must be treated as unknown data.
\begin{itemize}
\item When data has not been captured it is unknown by nature. For
  example, we try to capture data from a sensor and there is no
  response.
\item When data is erroneously it must be marked as unknown. For
  example, we capture data from a sensor but it responses in a not
  reasonable time or we capture data that is clearly outside a
  reasonable limits.
\end{itemize}
As a consequence, attribute interpolation functions deals with these
two subprocesses: treating unknown data and marking data as
unknown. Following with real numbers, let unknown value be represented
by the improper element infinity ($\infty$).  








\subsection{Graf i funció temporal de representació}
\label{sec:model:repr}

Una sèrie temporal és la representació discreta d'una funció contínua;
la qual és una funció temporal atès que depèn del temps. El model de
SGST definit anteriorment formalitza la sèrie temporal com aquesta
representació discreta. Però a partir d'una sèrie temporal es pot
voler interpretar quina era la funció contínua original; és a dir
obtenir nous valors segons una funció a la que s'anomena graf d'una
funció (\emph{graph of a function}), en el sentit de gràfic
(\emph{plot}) que cal no confondre amb els grafs d'arestes i vèrtexs
(\emph{vertex-edge graph}).

\begin{definition}[Graf d'una sèrie temporal]%an. graph
  Sigui $S=\{m_0,\ldots,m_k\}$ una sèrie temporal i $T$ un domini del
  temps, es defineix el graf de la sèrie temporal
  $\glssymboldef{not:sgst:graf} S(t)$ com un conjunt de parells
  ordenats $(t,S(t)$ : $\glssymboldef{not:sgst:graf} S(t) = \{ (t,S(t)) |
  t\in T \}$ a on $S(t)$ és una funció de representació de la sèrie
  temporal.
\end{definition}

Per a calcular el graf d'una sèrie temporal es necessita una funció de
representació. Mentre que la funció que permet canviar d'una funció
contínua a una sèrie temporal s'anomena procés d'adquisició o
mostreig, la funció que permet canviar d'una sèrie temporal a una
funció contínua l'anomenem funció de representació.  Així doncs,
donada una sèrie temporal es poden definir funcions amb el temps com a
variable que calculin nous valors a partir de les mesures
emmagatzemades.
\begin{definition}[Funció de representació]
  Sigui $S=\{m_0,\ldots,m_k\}$ una sèrie temporal, es defineix $S(t)$
  com la funció de representació de la sèrie temporal contínuament al
  llarg del temps $t$; és a dir que per cada instant de temps la
  funció pren un valor: $\forall t\in T: v(t) = S(t)$. 

  Atenent a les operacions de càlcul que es facin per a obtenir $S(t)$
  diem que hi ha diverses funcions de representació. Així per a cada
  funció de representació indicarem a quina ens referim amb un
  superíndex \glsdispdef{not:sgst:frepr}{$S^r(t)$} a on
  \glssymboldef{not:sgst:repr} és el nom d'una funció de representació
  de la sèrie temporal.
\end{definition}

La utilitat de les funcions de representació és diversa i per això les
operacions de càlcul poden ser qualssevol. Una funció de
representació es pot utilitzar per a interpolar valors d'una sèrie
temporal però també per a extrapolar-los o per a canviar la resolució
de la sèrie temporal. També es pot utilitzar per al problema aproximar
la sèrie temporal a la funció original; és a dir donar una funció de
representació que sigui la funció contínua que més s'aproxima a la
funció temporal original.


El lligam entre una sèrie temporal i la seva representació no és fix;
és a dir que donada una sèrie temporal es pot representar amb una
funció o amb una altra segons convingui.  Encara que en alguns àmbits,
per exemple a teoria del senyal o en algunes aplicacions de l'anàlisi
de sèries temporals, l'objectiu és cercar la parella de sèrie temporal
i representació que més s'aproxima a la funció contínua original; en
altres àmbits, com per exemple el del model multiresolució que definim
posteriorment, l'objectiu està més orientat a utilitzar
representacions de les sèries temporals segons els càlculs que es
volen fer o segons la naturalesa que s'assumeixi de la sèrie temporal.


No obstant, les funcions de representació s'han d'utilitzar amb
criteri. Per exemple la naturalesa de la sèrie temporal indueix a unes
possibles operacions de càlcul que es poden realitzar, per tant
l'aplicació de qualsevol funció de representació a la sèrie temporal
pot donar resultats incoherents. O bé un altre exemple és aplicar
càlculs successius a una sèrie temporal seguint diferents funcions de
representació.  

Així i tot, en la formalització del model de funcions de
representacions no hi definim cap criteri en concret per a, així,
donar llibertat en les possibilitats de càlcul amb les sèries
temporals. Al capítol \todo{ref} d'estat actual ja s'ha vist que en
l'anàlisi de sèries temporals hi ha diversitat en els algoritmes de
representació que s'utilitzen.
% entre els quals destaquen els que es basen en aproximació als valors
% de la sèrie temporal. Aquí ens centrem en els algoritmes
% d'interpolació exactes, els quals són més senzills de comprendre.

,
A continuació, definim diverses funcions de representació per a
exemplificar-ne l'ús. Les agrupem per algunes de les seves
característiques, a les qual podem veure com famílies de funcions de
representació, i de cada una en definim les més representatives. Per a
cada definició una oferim, quan es pugui, dues expressions
equivalents: una en matemàtica contínua, que ajuda a comprendre el
significat, i una en matemàtica discreta, que utilitza l'àlgebra del
model de SGST.



\subsubsection{Funcions parcials}
\todo{no sé si aporten res les parcials? cal acabar la secció d'agregadors dels SGSTM i repensar això}

Primerament, definim una funció de representació anomenada discreta
pura que no és totalment contínua en el temps sinó que és una funció
parcial.
\begin{definition}[Funció de representació discreta pura]
  Sigui $S=\{m_0,\ldots,m_k\}$ una sèrie temporal, es defineix
  $S^d(t)$ com la funció de representació discreta pura de la sèrie
  temporal $\forall m \in S: S^d(t) =
  \begin{cases}
    V(m) & \text{si }  t=T(m) \\
    \text{no definit} & \text{altrament}
  \end{cases}$.
\end{definition}

Aquest és un cas especial de funció de representació perquè permet que
el graf de la sèrie temporal sigui equivalent a les mesures de la
sèrie temporal: $\glssymbol{not:sgst:graf} S^d(t) \equiv
\{m_0,\ldots,m_k\}$.

Es poden definir altres funcions de representació de la família
parcial, però presenten el problema que el domini queda restringit a
un subconjunt $T'$ del domini temps $T$; el domini queda restringit
als instants de temps elegits $T'$ ja que per qualsevol altre instant de
$T$ no hi ha imatge definida.

Així doncs, és millor definir funcions totals que sempre seran
funcions ben-definides per al domini temps.
\todo{sinó, es poden trobar equivalents en les contínues?}



\begin{example}[Sèrie temporal amb representació discreta pura]
  Sigui la sèrie temporal $S=\{ (3,1), (4,3), (6,2), (9,1) \}$, el
  graf de la representació discreta pura és $\glssymbol{not:sgst:graf}
  S^d(t)=\{ (3,1), (4,3), (6,2), (9,1) \}$, el qual es mostra a la
  \autoref{fig:model:repr:d}.


  \begin{figure}[tp]
  \centering
  \begin{tabular}[c]{|c|c|}
    \multicolumn{2}{c}{$S$} \\ \hline
    $t$  & $v$ \\ \hline
    3  & 1 \\
    4  & 3 \\
    6  & 2 \\
    9  & 1 \\ \hline
  \end{tabular} \qquad
  \begin{tikzpicture}[baseline=(current bounding box.center)]
    \begin{axis}[
        timeseriesrel,
        title=$S^d$,
        ]
    \addplot[mark=*,blue,only marks] coordinates { 
        (3,1) 
        (4,3)
        (6,2)
        (9,1)
    };




    \end{axis}
   \end{tikzpicture}
   \caption{Taula d'una sèrie temporal $S$ i
     $\glssymbol{not:sgst:graf} S^d(t)$}
  \label{fig:model:repr:d}
  \end{figure}
\end{example}




\subsubsection{A impulsos}

Una família de funcions contínues que recorda a la funció discreta són
les funcions d'impulsos (\emph{impulse train function}).  A
continuació ho exemplifiquem amb una representació que anomenem delta
($\delta$) perquè es basa en la funció delta de Dirac, la qual val
zero a tot arreu excepte en el punt zero.

\begin{definition}[Funció de representació delta]
  Sigui $S=\{m_0,\ldots,m_k\}$ una sèrie temporal i $T$ el domini del
  temps, es defineix $S^\delta(t)$ com la funció de representació
  delta al llarg del temps, $\forall m \in S:$
  \begin{align*}
    S^\delta(t) = &  \\
    = & \sum_{t\in T} V(m) \delta(t-T(m)): \delta(t)= 
      \begin{cases}
        1 & \text{si }  t=0 \\
        0 & \text{altrament}
      \end{cases} \\
    = & \begin{cases}
      V(m) & \text{si }  t=T(m) \\
      0 & \text{altrament}
    \end{cases}
         \end{align*}.
\end{definition}



\begin{example}[Sèrie temporal amb representació delta]
  Sigui la sèrie temporal $S=\{ (3,1), (4,3), (6,2), (9,1) \}$, la
  seva representació delta és $S^\delta(t) = 0 +1\delta(t-3)
  +3\delta(t-4) +2\delta(t-6) +1\delta(t-9)$. El graf d'aquesta
  representació, $\glssymbol{not:sgst:graf} S^\delta(t)$, es mostra a
  la \autoref{fig:model:repr:delta}.


  \begin{figure}[tp]
  \centering
  \begin{tabular}[c]{|c|c|}
    \multicolumn{2}{c}{$S$} \\ \hline
    $t$  & $v$ \\ \hline
    3  & 1 \\
    4  & 3 \\
    6  & 2 \\
    9  & 1 \\ \hline
  \end{tabular} \qquad
  \begin{tikzpicture}[baseline=(current bounding box.center)]
    \begin{axis}[
        timeseriesrel,
        title=$S^\delta$,
        xmin=0,
        xmax=11,
        xtickmin=0,
        xtickmax=10,
        try min ticks=6,
        ]
    \addplot[mark=*,blue,ycomb] coordinates { 
        (3,1) 
        (4,3)
        (6,2)
        (9,1)
    };

    \addplot[mark=o,blue,only marks] coordinates { 
        (3,0)
        (4,0)
        (6,0)
        (9,0)
    };


    \pgfplotsextra{%
      \pgfpathmoveto{\pgfplotspointaxisxy{0.5}{0}}%
      \pgfpathlineto{\pgfplotspointaxisxy{12}{0}}%
      \pgfsetarrowsstart{latex}
      \pgfsetarrowsend{latex}
      \pgfsetcolor{blue}
      \pgfusepath{stroke}%
    }


    \end{axis}
   \end{tikzpicture}
   \caption{Taula d'una sèrie temporal $S$ i
     $\glssymbol{not:sgst:graf} S^\delta(t)$}
  \label{fig:model:repr:delta}
  \end{figure}
\end{example}


\subsubsection{A trossos constants}

Una altra família de funcions són les que es basen en funcions
definides a trossos constants (\emph{piecewise constant functions}).
A continuació ho exemplifiquem amb quatre representacions basades en la
funció graó (\emph{step function} o \emph{staircase function}) atenent
a quatre de les possibles continuïtats en els intervals de temps. 


A les definicions següents s'utilitza la notació de funció
característica $\glssymbol{not:Ia}_A(t)$ per a indicar quan un instant
de temps pertany a un determinat interval de temps:
\[
\glssymbol{not:Ia}_A(t) = 
   \begin{cases}
      1 & \text{si } t\in A \\
      0 & \text{altrament}
    \end{cases}
\]



%(\emph{right-continuous})
En primer lloc, definim una representació en base a funcions graó
contínues per la dreta. L'anomenem representació \emph{zero-order
  hold}(ZOH) a causa de la semblança que té amb el model utilitzat en
electrònica per a reconstruir senyals, el qual consisteix en mantenir
constant cada valor fins al proper.
\begin{definition}[Funció de representació \emph{zero-order hold}]
  Sigui $S=\{m_0,\ldots,m_k\}$ una sèrie temporal i $T$ el domini del
  temps, es defineix $S^\text{ZOH}(t)$ com la funció de representació
  \emph{zero-order hold} al llarg del temps, $\forall m \in S:$
  \begin{align*}
    S^\text{ZOH}(t) = &  \\
    = & \sum_{t\in T} V(m) \glssymbol{not:Ia}_{\big[T(m), T(\glssymbol{not:sgst:next}(m)) \big)}(t)\\
    = & \begin{cases}
      0 & \text{si }  t<T(\min(S)) \\
      V(m) & \text{si } t\in
      \big[T(m),T(\glssymbol{not:sgst:next}(m))\big)
    \end{cases}
         \end{align*}.
\end{definition}



En segon lloc, definim una representació en base a funcions graó
contínues per l'esquerra. L'anomenem representació \emph{zero-order
  hold} cap enrere (ZOHE) perquè consisteix en mantenir constant cada
valor fins al predecessor. Una representació similar s'utilitza a
\gls{RRDtool} \parencite{lisa98:oetiker}.
\begin{definition}[Funció de representació \emph{zero-order hold} cap
  enrere]%\emph{zero-order hold backwards}(zohe%from \emph{zero-order
         %hold everted}
  Sigui $S=\{m_0,\ldots,m_k\}$ una sèrie temporal i $T$ el domini del
  temps, es defineix $S^\text{ZOHE}(t)$ com la funció de representació
  \emph{zero-order hold} cap enrere al llarg del temps, $\forall m \in S:$
  \begin{align*}
    S^\text{ZOHE}(t) = &  \\
    = & \sum_{t\in T} V(m) \glssymbol{not:Ia}_{\big(T(\glssymbol{not:sgst:prev}(m)),T(m)\big]}(t) \\
    = & \begin{cases}
      0 & \text{si }  t > T(\max(S)) \\
      V(m) & \text{si } t\in \big(T(\glssymbol{not:sgst:prev}(m)),T(m)\big]
    \end{cases}
         \end{align*}.
\end{definition}



En tercer lloc, definim una representació en base a funcions graó
contínues per la dreta centrades en l'interval. L'anomenem
representació \emph{zero-order hold} centrada en l'interval (ZOHC).
\begin{definition}[Funció de representació \emph{zero-order hold}
  centrada en l'interval]
  Sigui $S=\{m_0,\ldots,m_k\}$ una sèrie temporal i $T$ el domini del
  temps, es defineix $S^\text{ZOHC}(t)$ com la funció de representació
  \emph{zero-order hold} centrada en l'interval al llarg del temps,
  $\forall m \in S:$
  \begin{align*}
    S^\text{ZOHC}(t) = &  \\
   = & \sum_{t\in T} V(m) \glssymbol{not:Ia}_{\left[
        \frac{T(\glssymbol{not:sgst:prev}(m))+T(m)}{2},
        \frac{T(m)+T(\glssymbol{not:sgst:next}(m))}{2}
      \right)}(t) \\
    = & V(m): t\in \left[
        \frac{T(\glssymbol{not:sgst:prev}(m))+T(m)}{2},
        \frac{T(m)+T(\glssymbol{not:sgst:next}(m))}{2} \right)
         \end{align*}.
\end{definition}




En quart lloc, representem la sèrie temporal en base a la funció
rectangular. La funció rectangular és un cas especial de les funcions
graó a on s'especifiquen valors simètrics per als punts de
discontinuïtat.
\begin{definition}[Funció de representació rectangular]
  Sigui $S=\{m_0,\ldots,m_k\}$ una sèrie temporal i $T$ el domini del
  temps, es defineix $S^\text{rect}(t)$ com la funció de representació
  rectangular al llarg del temps, $\forall m \in S:$
  \begin{align*}
    S^\text{rect}(t) = &  \\
    = & \sum_{t\in T} V(m) \operatorname{rect}(t):  \operatorname{rect}(t) = 
    \begin{cases}
      1 & \text{si } t\in \big(T(m),T(\glssymbol{not:sgst:next}(m))\big) \\
      \frac{1}{2}& \text{si } t = T(m) \vee t=T(\glssymbol{not:sgst:next}(m)) \\
      0 & \text{altrament}
    \end{cases} \\
    = & \begin{cases}
      0 & \text{si }  t<T(\min(S)) \\
      V(m) & \text{si } t\in \big(T(m),T(\glssymbol{not:sgst:next}(m))\big) \\
      \frac{V(m)+V(\ant(m))}{2} & \text{si } t = T(m) \wedge t > T(\min(S)) \\
      \frac{V(m)}{2} & \text{si } t = T(\min(S)) \\
    \end{cases}
         \end{align*}.
\end{definition}


% En la representació rectangular només hem definit un cas semblant al
% del la representació ZOH. També podríem definir variacions de la
% rectangular com s'ha fet per la ZOH: la ZOHE i la ZOHC. 
% La variació entre la ZOH i la ZOHC o la ZOHE no és només una translació. 



\begin{example}[Sèrie temporal amb representació ZOHE]
  Sigui la sèrie temporal $S=\{ (3,1), (4,3), (6,2), (9,1) \}$, la
  seva representació ZOHE és $S^\text{ZOHE}(t) =
  1\glssymbol{not:Ia}_{(-\infty,3]} +3\glssymbol{not:Ia}_{(3,4]}
  +2\glssymbol{not:Ia}_{(4,6]} +1\glssymbol{not:Ia}_{(6,9]}
  +0\glssymbol{not:Ia}_{(9,+\infty)}$. El graf d'aquesta
  representació, $\glssymbol{not:sgst:graf} S^\text{ZOHE}(t)$, es
  mostra a la \autoref{fig:model:repr:zohe}.


  \begin{figure}[tp]
  \centering
  \begin{tabular}[c]{|c|c|}
    \multicolumn{2}{c}{$S$} \\ \hline
    $t$  & $v$ \\ \hline
    3  & 1 \\
    4  & 3 \\
    6  & 2 \\
    9  & 1 \\ \hline
  \end{tabular} \qquad
  \begin{tikzpicture}[baseline=(current bounding box.center)]
    \begin{axis}[
        timeseriesrel,
        title=$S^\text{ZOHE}$,
        xmin=0,
        xmax=11,
        xtickmin=0,
        xtickmax=10,
        try min ticks=6,
        ]
    \addplot[mark=*,blue,const plot mark right] coordinates { 
        (3,1) 
        (4,3)
        (6,2)
        (9,1)
    };

    \addplot[mark=o,blue,only marks] coordinates { 
        (3,3)
        (4,2)
        (6,1)
        (9,0)
    };

    \pgfplotsextra{%
      \pgfpathmoveto{\pgfplotspointaxisxy{9}{1}}%
      \pgfpathlineto{\pgfplotspointaxisxy{9}{0}}%
      \pgfsetcolor{blue}
      \pgfusepath{stroke}%
    }

    \pgfplotsextra{%
      \pgfpathmoveto{\pgfplotspointaxisxy{9}{0}}%
      \pgfpathlineto{\pgfplotspointaxisxy{12}{0}}%
      \pgfsetarrowsend{latex}
      \pgfsetcolor{blue}
      \pgfusepath{stroke}%
    }

    \pgfplotsextra{%
      \pgfpathmoveto{\pgfplotspointaxisxy{3}{1}}%
      \pgfpathlineto{\pgfplotspointaxisxy{0.5}{1}}%
      \pgfsetarrowsend{latex}
      \pgfsetcolor{blue}
      \pgfusepath{stroke}%
    }

    \end{axis}
   \end{tikzpicture}
   \caption{Taula d'una sèrie temporal $S$ i
     $\glssymbol{not:sgst:graf} S^\text{ZOHE}(t)$}
  \label{fig:model:repr:zohe}
  \end{figure}
\end{example}



\begin{example}[Sèrie temporal amb representació rectangular]
  Sigui la sèrie temporal $S=\{ (3,1), (4,3), (6,2), (9,1) \}$, la
  seva representació rectangular és $S^\text{rect}(t) =
  0\glssymbol{not:Ia}_{(-\infty,3)} +0{,}5\glssymbol{not:Ia}_{[3,3]}
  +1\glssymbol{not:Ia}_{(3,4)} +2\glssymbol{not:Ia}_{[4,4]}
  +3\glssymbol{not:Ia}_{(4,6)} +2{,}5\glssymbol{not:Ia}_{[6,6]}
  +2\glssymbol{not:Ia}_{(6,9)} +1{,}5\glssymbol{not:Ia}_{[9,9]}
  +1\glssymbol{not:Ia}_{(9,+\infty)}$. El graf d'aquesta
  representació, $\glssymbol{not:sgst:graf} S^\text{rect}(t)$, es
  mostra a la \autoref{fig:model:repr:rect}.


  \begin{figure}[tp]
  \centering
  \begin{tabular}[c]{|c|c|}
    \multicolumn{2}{c}{$S$} \\ \hline
    $t$  & $v$ \\ \hline
    3  & 1 \\
    4  & 3 \\
    6  & 2 \\
    9  & 1 \\ \hline
  \end{tabular} \qquad
  \begin{tikzpicture}[baseline=(current bounding box.center)]
    \begin{axis}[
        timeseriesrel,
        title=$S^\text{rect}$,
        xmin=0,
        xmax=11,
        xtickmin=0,
        xtickmax=10,
        try min ticks=6,
        ]
    \addplot[mark=o,blue,const plot mark left] coordinates { 
        (3,1) 
        (4,3)
        (6,2)
        (9,1)
    };

    \addplot[mark=o,blue,only marks] coordinates { 
        (3,0)
        (4,1)
        (6,3)
        (9,2)
    };

    \addplot[mark=*,blue,only marks] coordinates { 
        (3,0.5)
        (4,2)
        (6,2.5)
        (9,1.5)
    };

    \pgfplotsextra{%
      \pgfpathmoveto{\pgfplotspointaxisxy{3}{1}}%
      \pgfpathlineto{\pgfplotspointaxisxy{3}{0}}%
      \pgfsetcolor{blue}
      \pgfusepath{stroke}%
    }

    \pgfplotsextra{%
      \pgfpathmoveto{\pgfplotspointaxisxy{9}{1}}%
      \pgfpathlineto{\pgfplotspointaxisxy{12}{1}}%
      \pgfsetarrowsend{latex}
      \pgfsetcolor{blue}
      \pgfusepath{stroke}%
    }

    \pgfplotsextra{%
      \pgfpathmoveto{\pgfplotspointaxisxy{3}{0}}%
      \pgfpathlineto{\pgfplotspointaxisxy{0.5}{0}}%
      \pgfsetarrowsend{latex}
      \pgfsetcolor{blue}
      \pgfusepath{stroke}%
    }

    \end{axis}
   \end{tikzpicture}
   \caption{Taula d'una sèrie temporal $S$ i
     $\glssymbol{not:sgst:graf} S^\text{rect}(t)$}
  \label{fig:model:repr:rect}
  \end{figure}

\end{example}






\subsubsection{A trossos lineals}

Una família de funcions d'un ordre superior a les de trossos constants
són les que es basen en funcions definides a trossos lineals
(\emph{piecewise linear functions}).  A continuació ho exemplifiquem
amb una representació basada en la funció triangular (\emph{triangular
  function}). Seguint amb l'analogia electrònica l'anomenem
representació \emph{first-order hold}(FOH), el qual consisteix en
interpolar linealment cada valor fins al proper.


La definició del FOH amb funcions matemàtiques contínues es construeix
a partir de la funció triangular $\operatorname{tri}(t)$
\[
\operatorname{tri}(t) = 
\begin{cases}
  1-|t| & \text{si } |t| < 1\\
  0 & \text{altrament}
\end{cases}
\]

Per al cas particular de sèries temporals regulars, es pot
utilitzar directament la funció triangular general.  Sigui
$S_R=\{m_0,\ldots,m_k\}$ una sèrie temporal regular amb període $P$ i
sigui $T$ el domini del temps, es defineix $S^\text{FOH}_ R(t)$ com la
funció de representació \emph{first-order hold} al llarg del temps,
$\forall m \in S:$
\[
S_ R^\text{FOH}(t) = \sum_{t\in T} V(m)
\operatorname{tri}\left(\frac{t-T(m)}{P}\right)
\]



Per al cas general, tant per a sèries temporals regulars com per a no
regulars, s'ha de construir a partir de funcions triangulars no
simètriques:
\[
\operatorname{tri}^{-1}(t) = 
\begin{cases}
  1-|t| & \text{si } -1 < t < 0\\
  0 & \text{altrament}
\end{cases}
\quad
\operatorname{tri}^1(t) = 
\begin{cases}
  1-|t| & \text{si } 0 \leq t < 1\\
  0 & \text{altrament}
\end{cases}
\]


\begin{definition}[Funció de representació \emph{first-order hold}]
  Sigui $S=\{m_0,\ldots,m_k\}$ una sèrie temporal i $T$ el domini del
  temps, es defineix $S^\text{FOH}(t)$ com la funció de representació
  \emph{first-order hold} al llarg del temps, $\forall m \in S:$
  \begin{align*}
    :\; & x_1=T(m), y_1=V(m), \\
    & m_2=\glssymbol{not:sgst:next}(m), x_2=T(m_s), y_2=V(m_s),\\
    & m_0=\glssymbol{not:sgst:prev}(m), x_0=T(m_a), y_0=V(m_a) :\\
    S^\text{FOH}(t) = & \\
    = & \sum_{t\in T} y_1
    \left(
      \operatorname{tri}^{-1}\left(\frac{t-x_1}{x_1-x_0}\right)
      +\operatorname{tri}^1\left(\frac{t-x_1}{x_2-x_1}\right)
    \right)  \\
    = & \begin{cases}
      V(\min(S)) & \text{si } t<T(\min(S) )\\
      V(\max(S)) & \text{si } t> T(\max(S))  \\
      \frac{y_2-y_1}{x_2-x_1}(t-x_1)+y_1 & \text{si } t\in
      \big[x_1,x_2\big) \wedge t \leq T(\max(S))
    \end{cases}
  \end{align*}.
\end{definition}




\begin{example}[Sèrie temporal amb representació FOH]
  Sigui la sèrie temporal $S=\{ (3,1), (4,3), (6,2), (9,1) \}$, la
  seva representació amb FOH és $S^\text{FOH}(t) =
  1\operatorname{tri}^{-1}\left(\frac{t-3}{3-(-\infty)}\right) +
  1\operatorname{tri}^1\left(\frac{t-3}{4-3}\right)
  +3\operatorname{tri}^{-1}\left(\frac{t-4}{4-3}\right) +
  3\operatorname{tri}^1\left(\frac{t-4}{6-4}\right)
  +2\operatorname{tri}^{-1}\left(\frac{t-6}{6-4}\right) +
  2\operatorname{tri}^1\left(\frac{t-6}{9-6}\right)
  +1\operatorname{tri}^{-1}\left(\frac{t-9}{9-6}\right) +
  1\operatorname{tri}^1\left(\frac{t-9}{+\infty-9}\right)$. El graf
  d'aquesta representació, $\glssymbol{not:sgst:graf}
  S^\text{FOH}(t)$, es mostra a la
  \autoref{fig:model:repr:foh}.


  \begin{figure}[tp]
  \centering
  \begin{tabular}[c]{|c|c|}
    \multicolumn{2}{c}{$S$} \\ \hline
    $t$  & $v$ \\ \hline
    3  & 1 \\
    4  & 3 \\
    6  & 2 \\
    9  & 1 \\ \hline
  \end{tabular} \qquad
  \begin{tikzpicture}[baseline=(current bounding box.center)]
    \begin{axis}[
        timeseriesrel,
        title=$S^\text{FOH}$,
        xmin=0,
        xmax=11,
        ymin=0,
        xtickmin=0,
        xtickmax=10,
        try min ticks=6,
        ]
    \addplot[mark=*,blue] coordinates { 
        (3,1) 
        (4,3)
        (6,2)
        (9,1)
    };

    \addplot[lightgray] coordinates { 
        (3,1) 
        (4,0)
    };
     \addplot[lightgray] coordinates { 
        (3,0)
        (4,3)
        (6,0)
    };
    \addplot[lightgray] coordinates { 
        (4,0)
        (6,2)
        (9,0)
    };
    \addplot[lightgray] coordinates { 
        (6,0)
        (9,1)
    };



    \pgfplotsextra{%
      \pgfpathmoveto{\pgfplotspointaxisxy{9}{1}}%
      \pgfpathlineto{\pgfplotspointaxisxy{12}{1}}%
      \pgfsetarrowsend{latex}
      \pgfsetcolor{blue}
      \pgfusepath{stroke}%
    }

    \pgfplotsextra{%
      \pgfpathmoveto{\pgfplotspointaxisxy{3}{1}}%
      \pgfpathlineto{\pgfplotspointaxisxy{0.5}{1}}%
      \pgfsetarrowsend{latex}
      \pgfsetcolor{blue}
      \pgfusepath{stroke}%
    }

    \end{axis}
   \end{tikzpicture}
   \caption{Taula d'una sèrie temporal $S$ i
     $\glssymbol{not:sgst:graf} S^\text{FOH}(t)$}
  \label{fig:model:repr:foh}
  \end{figure}

\end{example}





\subsubsection{Ús en els operadors de funció temporal}

Alguns operadors dels SGST presenten particularitats a causa del
fenomen de representació, aleshores els anomenem operadors de funció
temporal (v.\ l'apartat~\ref{sec:sgst:operadors-temporals}).


Els operadors de funció temporal han d'operar tenint en compte les
funcions de representació; així aquestes esdevenen paràmetres
d'aquests operadors. Això no obstant, no podem definir un lligam
directe entre els operadors de funció temporal i les funcions de
representació: aquestes darreres són una funció contínua al llarg del
temps i la interpolació genèrica com a $S(t)$ implica mètodes
numèrics, cosa que no encaixa amb el model definit basat en l'àlgebra.

Així doncs, el lligam que modelem entre els operadors de funció
temporal i les funcions de representació és simbòlic: en els operadors
de funció temporal, el paràmetre de representació és un nom que indica
el concepte de representació en el que es basen els càlculs de
l'operació. Per tant, per a cada parella d'operació de funció temporal
i nom de representació cal definir quins càlculs s'han de dur a terme
tot interpretant el significat de la funció de representació
associada. Si bé, generalment, només és necessari implementar
l'operació d'interval temporal per a cada representació atès que les
altres operacions es defineixen a partir d'aquesta. Un exemple d'això
és la definició d'interval temporal amb representació ZOHE
(v.\ def.~\ref{def:sgst:interval-temporal-zohe}).

 
D'altra banda, les funcions de representació també es tindran en
compte a l'hora de definir interpretacions pels agregadors d'atributs
en el model de SGSTM (v.\ l'apartat~\ref{sec:model:agregador}).



En conclusió, es defineixen les funcions de representació per a
estudiar exactament els grafs que es poden obtenir de la sèrie
temporal i posteriorment, agafant-ne el significat, s'implementen les
operacions que siguin necessàries



% \subsubsection{Equivalència en grafs}
% \todo{}

% El graf de $S_1=\{(1,1),(3,0),(5,1)\}$ i $S_2=\{(1,1),(2,0),(3,0)(4,1),(5,1)\}$ són equivalents amb representació zohe però no són equivalents amb representació lineal ja que hauria de ser $S_2'=\{(1,1),(2, 0{,}5 ),(3,0),(4, 0{,}5),(5,1)\}$. 















%%% Local Variables:
%%% TeX-master: "main"
%%% End:







% LocalWords:  SGST Dirac

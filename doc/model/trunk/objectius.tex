Potser la gran diferència que marca el model de multiresoució respecte altres treballs és:

* Si bé els streams agreguen estadístics de les dades, la multiresolució té en compte l'evolució d'aquests estadístics al llarg del temps, cosa que lliga més amb l'àmbit del monitoratge. És a dir, tenim en compte el temps, altres projectes només avaluen el moment present.  However, as a lossy storage solution, the
  multiresolution schema has to be decided for each application
  planning what approximate queries will be needed to resolve.

* Si bé l'orientació a agregadors amb stream és molt interessant, el model es manté genèric per a poder calcular amb qualsevol funció d'agregació.

* Es tenen en compte les irregularitats de mostreig de les series temporals

* Es modelitza fortament el concepte de funció de representació de les sèries temporals i així es pot tenir en compte la semàntica de cada sèrie temporal, ja que en cada agregació té molta afectació, per exemple quan les dades monitorades tenen naturalesa de comptadors als quals RRDtool s'hi ha especialitzat.
 We formalise the representation function concept of time series
  in order the user can define different operators considering the
  semantics of time series in different contexts; especially they
  behave differently in aggregation operations, i.e. RRDtool specific counter
  time series aggregations. Furthermore, we formalise representation
  as an independent object of the main model.

* Els altres treballs se centren en el problema de recuperar el senyal original, ho demostra el fet que avaluin el resultat amb SSE (per mínims quadrats). Nosaltres enfoquem el problema en un domini més genèric de les consultes aproximades on volem avaluar el resultat per si respon correctament a les consultes, encara que el senyal recuperat no s'assembli gens a l'original.

\item Our model considers the time irregularities sampling of time
  series. Moreover, it operates coherently with the time dimension of
  time series.

\item We describe the model firmly rooted on relational algebra as a formal
  theory for information systems.
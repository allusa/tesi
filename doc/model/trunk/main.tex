%--------------------
% document principal
%--------------------
% cal compilar amb `pdflatex -shell-escape main.tex`
% makeglossaries main
%--------------------
\documentclass[paper=a4,parskip=half,
twoside,fontsize=11pt,BCOR12mm,
%oneside,fontsize=11pt, %%format web
% twoside,openany,fontsize=10pt,DIV = 10,BCOR12mm, %%format en paper
]{scrbook}
%%%%BCOR12mm  factor de correcció per enquadernació
%%%%BCOR??mm  factor de correcció per enquadernació amb espiral -0.5mm??
%\usepackage[T1]{fontenc}
%------------- capçalera ----------------------
%--------------------
% capçalera del document 
%--------------------
\usepackage[catalan]{babel}
\usepackage[utf8]{inputenc}
%\usepackage[T1]{fontenc} 
%\usepackage{parskip}

%---------- Bibliografia -------------------
\usepackage[
style=numeric-comp,%style=authoryear
sortcites=true,
%backref=true,
]{biblatex}
%\cite{} \textcite{} \parencite{}
\bibliography{bibliografia}
\newcommand{\bibendash}{--}
%\bibparsep 0.2cm
%\bibhang 0.25cm
%---------- Gràfics ------------------------
\usepackage[final]{graphicx}
\usepackage{epstopdf}
\usepackage{tikz}
\usepackage{pgfplots}
\usepackage{epic,xcolor,multicol}
%---------- Símbols ------------------------
\usepackage{cclicenses}%\usepackage{ccicons}
\usepackage{url} %\url i \path
\usepackage{eurosym}
\usepackage{amsmath,amssymb,amsthm}
\numberwithin{equation}{chapter}
\newtheorem{definition}{Definició}
%---------- Codi ---------------------------
%\usepackage{longtable}
\usepackage{upquote} %perquè en verbatim surtin les cometes `
\usepackage{listings}
\lstloadlanguages{bash,C,HTML,Python,XML}
\lstset{escapechar=@,numberstyle=\tiny,frame=single,frameround=tttt,
        breaklines=true,breakindent=0pt,
        prebreak=\mbox{{\color{blue}\tiny$\searrow$}},
        postbreak=\mbox{{\color{blue}\tiny$\hookrightarrow$}},
        columns=[l]fullflexible,
        xleftmargin=1em,
        extendedchars=true,
        literate={à}{{\`a}}1 {è}{{\`e}}1 {é}{{\'e}}1 {í}{{\'\i}}1 {ï}{{\"\i}}1
                 {ò}{{\`o}}1 {ó}{{\'o}}1 {ú}{{\'u}}1 {ü}{{\"u}}1
                 {ç}{{\c{c}}}1 {l·l}{{\l.l}}1
                 {À}{{\`A}}1 {È}{{\`E}}1 {É}{{\'E}}1 {Í}{{\'I}}1 {Ï}{{\"I}}1
                 {Ò}{{\`O}}1 {Ó}{{\'O}}1 {Ú}{{\'U}}1 {Ü}{{\"U}}1
                 {Ç}{{\c{C}}}1 {L·L}{{\L.L}}1, 
        }

\lstdefinestyle{py}{
  style=pynocolor
}

\lstdefinestyle{pynocolor}{
  language=python,
  frame=none,
  inputencoding=utf8,
  backgroundcolor=\color[gray]{0.95},
}

\lstdefinestyle{pycolor}{
  style=py,
        basicstyle=\sffamily\footnotesize,
        stringstyle=\color{green},
        showstringspaces=false,
        alsoletter={1234567890},
        otherkeywords={\ , \}, \{},
        keywordstyle=\color{blue},
        emph={access,and,as,break,class,continue,def,del,elif,else,%
          except,exec,finally,for,from,global,if,import,in,is,%
          lambda,not,or,pass,print,raise,return,try,while,assert},
        emphstyle=\color{orange}\bfseries,
        emph={[2]self},
        emphstyle=[2]\color{gray},
        emph={[4]ArithmeticError,AssertionError,AttributeError,BaseException,%
          DeprecationWarning,EOFError,Ellipsis,EnvironmentError,Exception,%
          False,FloatingPointError,FutureWarning,GeneratorExit,IOError,%
          ImportError,ImportWarning,IndentationError,IndexError,KeyError,%
          KeyboardInterrupt,LookupError,MemoryError,NameError,None,%
          NotImplemented,NotImplementedError,OSError,OverflowError,%
          PendingDeprecationWarning,ReferenceError,RuntimeError,RuntimeWarning,%
          StandardError,StopIteration,SyntaxError,SyntaxWarning,SystemError,%
          SystemExit,TabError,True,TypeError,UnboundLocalError,UnicodeDecodeError,%
          UnicodeEncodeError,UnicodeError,UnicodeTranslateError,UnicodeWarning,%
          UserWarning,ValueError,Warning,ZeroDivisionError,abs,all,any,apply,%
          basestring,bool,buffer,callable,chr,classmethod,cmp,coerce,compile,%
          complex,copyright,credits,delattr,dict,dir,divmod,enumerate,eval,%
          execfile,exit,file,filter,float,frozenset,getattr,globals,hasattr,%
          hash,help,hex,id,input,int,intern,isinstance,issubclass,iter,len,%
          license,list,locals,long,map,max,min,object,oct,open,ord,pow,property,%
          quit,range,raw_input,reduce,reload,repr,reversed,round,set,setattr,%
          slice,sorted,staticmethod,str,sum,super,tuple,type,unichr,unicode,%
          vars,xrange,zip},
        emphstyle=[4]\color{purple}\bfseries,
        morecomment=[s][\color{lightgreen}]{"""}{"""},
        commentstyle=\color{red}\slshape,
        literate=
          {>>>}{\textbf{\textcolor{red}{$>$\kern-.5ex$>$\kern-.5ex$>$}~}}3%
          {...}{{\textcolor{gray}{...}}}3%
          {à}{{\`a}}1 {è}{{\`e}}1 {é}{{\'e}}1 {í}{{\'\i}}1 {ï}{{\"\i}}1%
          {ò}{{\`o}}1 {ó}{{\'o}}1 {ú}{{\'u}}1 {ü}{{\"u}}1 {ç}{{\c{c}}}1%
          {l·l}{{\l.l}}1 {À}{{\`A}}1 {È}{{\`E}}1 {É}{{\'E}}1 {Í}{{\'\I}}1%
          {Ï}{{\"\I}}1 {Ò}{{\`O}}1 {Ó}{{\'O}}1 {Ú}{{\'U}}1 {Ü}{{\"U}}1%
          {Ç}{{\c{C}}}1 {L·L}{{\L.L}}1, 
        rulesepcolor=\color{blue},
} 

\lstdefinestyle{sh}{
  language=bash,
  frame=none,
  prebreak =\textbackslash,
  postbreak ={},
  basicstyle=\ttfamily,
  showspaces=false,
  keepspaces=true,
}
\lstdefinestyle{file}{
  frame=none,
  showspaces=false,
  keepspaces=true,
  backgroundcolor=\color{yellow!20!white}
}

\lstdefinestyle{stdout}{
}

%% ÚS del lstlisting
%%\begin{lstlisting}[language=C,caption=Plantilla de NagiosGrapher pels missatges,label=NGmis,numbers=left]
%%\lstinline[style=sh]!for i:integer;!
%--------------------------------------------




% \makeatletter
% \lst@CCPutMacro
%     \lst@ProcessOther {"C0}{\`{A}}
%     \lst@ProcessOther {"C1}{\'{A}}
%     \lst@ProcessOther {"C2}{\^{A}}
%     \lst@ProcessOther {"C4}{\"{A}}
%     \lst@ProcessOther {"C7}{\c{C}}
%     \lst@ProcessOther {"C8}{\`{E}}
%     \lst@ProcessOther {"C9}{\'{E}}
%     \lst@ProcessOther {"CA}{\^{E}}
%     \lst@ProcessOther {"CB}{\"{E}}
%     \lst@ProcessOther {"CE}{\^{I}}
%     \lst@ProcessOther {"CF}{\"{I}}
%     \lst@ProcessOther {"D4}{\^{O}}
%     \lst@ProcessOther {"D6}{\"{O}}
%     \lst@ProcessOther {"D9}{\`{U}}
%     \lst@ProcessOther {"DB}{\^{U}}
%     \lst@ProcessOther {"E0}{\`{a}}
%     \lst@ProcessOther {"E1}{\'{a}}
%     \lst@ProcessOther {"E2}{\^{a}}
%     \lst@ProcessOther {"E4}{\"{a}}
%     \lst@ProcessOther {"E7}{\c{c}}
%     \lst@ProcessOther {"E8}{\`{e}}
%     \lst@ProcessOther {"E9}{\'{e}}
%     \lst@ProcessOther {"EA}{\^{e}}
%     \lst@ProcessOther {"EB}{\"{e}}
%     \lst@ProcessOther {"EE}{\^{\i}}
%     \lst@ProcessOther {"EF}{\"{\i}}
%     \lst@ProcessOther {"F4}{\^{o}}
%     \lst@ProcessOther {"F6}{\"{o}}
%     \lst@ProcessOther {"F9}{\`{u}}
%     \lst@ProcessOther {"FB}{\^{u}}
%     \@empty\z@\@empty
% \makeatother




%%% Local Variables: 
%%% mode: latex
%%% TeX-master: "main.default"
%%% End: 

\bibliography{bibliografia}
\ExecuteBibliographyOptions{annotation=true,backref=true,}
%backref=true, urldate=long, abbreviate=false,%%format web 
% isbn=false,url=false,doi=false,alldates=terse,firstinits=true,abbreviate=true %%format en paper
%---------- Mode esborrany --------------------
\includeonly{sgstm-operacions}
\usepackage[catalan]{todonotes} %%ús: \todo{text} \missingfigure{text}
\usepackage{fancyhdr}\pagestyle{fancyplain}\chead{\fancyplain{--- esborrany \today\ ---}{\footnotesize\today}}
%%\renewcommand{\headrulewidth}{0pt}
%----------------------------------------------
%\usetikzlibrary{shapes,arrows,positioning}
%\usetikzlibrary{calc}
%------------- format -------------------------
%%ús coma decimal sense espais:  2{,}5
\newtheorem{definition}{Definició}
\def\figureautorefname{figura} %ús: \autoref{}
\def\tableautorefname{taula} %ús: \autoref{}
\numberwithin{equation}{chapter}
\DeclareMathOperator*{\seg}{seg}
\DeclareMathOperator*{\ant}{ant}
\DeclareMathOperator*{\map}{map}
\DeclareMathOperator*{\fold}{fold}
\usepackage{longtable}
\usepackage{multirow}

%glossaris
\usepackage[
          acronym,
          %nonumberlist,
          %toc,
          section,
          numberedsection=autolabel,
          sanitize=none, %pels accents en el vegeu
          ]{glossaries}
%\renewcommand*{\glspostdescription}{}%anul·la el punt final
\renewcommand*{\acronymname}{Sigles}%{Índex de sigles}??si té refs pàgines 
%Índex d'abreviacions?? si conté abreviatures o símbols
% \short<type>name,
\newglossary{notation}{not}{ntn}{Notació}
\newglossarystyle{estil-notation}{%
  \renewcommand{\glsgroupskip}{}% make nothing happen between groups
  \renewenvironment{theglossary}
  {\begin{longtable}{ll}
      % \caption{Notació dels SGSTM \label{tab:sgstm-simbols}}
      % \endfirsthead
      % \caption[]{Notació dels SGSTM (continuació)}
      % \endhead
          % \endfoot
          % \endlastfoot
    }{\end{longtable}}
  \renewcommand*{\glossarysubentryfield}[6]{%
    \glstarget{##2}{##3}% the entry name
    &
    % \space (##4)% the symbol in brackets
    \space ##4% the description
    % \space [##6]% the number list in square brackets
    \\
  }%
  \renewcommand*{\glossaryentryfield}[5]{%
    \\\pagebreak[3]\hline
    \glossarysubentryfield{##2}{##1}{##2}{##3}{##4}{##5}
    \hline
  }
}


\renewcommand{\seename}{vegeu}
\renewcommand{\entryname}{Notació}
\renewcommand{\descriptionname}{Descripció}

\makeglossaries


%\renewcommand{\glossarypreamble}{Text com a préambul}



%TERMES


\newglossaryentry{SistemaGestioBaseDades}{name={sistema de gesti{ó} de base de dades}, description={(\emph{Data Base Management System})} }




%terme:SGBDR

\newglossaryentry{terme:SGBDR}{name={sistema de gestió de base de dades relacional}, description={(\emph{Relational Data Base Management System}). Totes les definicions són coherents amb \textcite{date} } }


%tipus,valor,variable,operador

\newglossaryentry{terme:SGBDR:domini}{see={terme:SGBDR},name={domini}, description = {(\emph{domain}), equivalent a tipus de dades.
Conjunt de valors. Cada domini té associat un conjunt d'operadors, en alguns casos fins i tot s'entén que el domini inclou els operadors (concepte de classe a orientació a objecte). Els tipus tenen una representació (estructura) o més d'una, és a dir els seus valors poden estar denotats per més d'un literal} }
\newglossaryentry{terme:SGBDR:tipus}{see={terme:SGBDR:domini}, name={tipus de dades}, description = {(\emph{data type}), a vegades solament 'tipus' (\emph{type}) o bé 'tipus de dades abstracte' (\emph{abstract data type}). Segons \textcite{date} en el context de model tots els tipus de dades han de ser abstractes} }

\newglossaryentry{terme:SGBDR:escalar}{parent={terme:SGBDR:domini}, name={escalar}, description = {Un tipus és escalar (\emph{scalar}) quan no té components visibles a l'usuari i és no escalar (\emph{nonscalar}) en cas contrari; no obstant, tant els escalars com els no escalars tenen representació, la qual pot contenir components} }


\newglossaryentry{terme:SGBDR:valor}{see={terme:SGBDR},name={valor}, description = {(\emph{value}), equivalent a objecte i instància.
'Constant individual' que és d'un tipus de dades. A vegades s'utilitza 'constant' per designar una  variable que mai canvia de valor, però aquest no és el cas d'aquesta definició} }
\newglossaryentry{terme:SGBDR:objecte}{see={terme:SGBDR:valor}, name={objecte}, description = {(\emph{object})} }
\newglossaryentry{terme:SGBDR:instancia}{see={terme:SGBDR:valor}, name={instància}, description = {(\emph{instance})} }

\newglossaryentry{terme:SGBDR:literal}{see={terme:SGBDR},name={literal}, description = {(\emph{literal}).
Símbol que denota un valor. Un valor pot estar denotat per més d'un literal. Segons aquesta definició literal no és equivalent a valor} }


\newglossaryentry{terme:SGBDR:variable}{see={terme:SGBDR},name={variable}, description = {(\emph{variable}).
Contenidor d'una aparició d'un valor. El valor que conté la variable pot ser canviat mitjançant l'operador d'assignació. En canvi els valors, per si mateixos, no poden ser actualitzats} } %A l'esquerra de l'operador d'assignació sempre hi ha variables, tot i que s'admeten simplificacions mitjançant expressions que són pseudovariables (p.ex. s[1] := 3 és equivalent a s := [s[0],3,s[2],..]).
%Les variables tenen adreces (\emph{addresses}) i per tant es pot apuntar (\emph{point to}) a les variables mitjançant els operadors de referència (\emph{referencing}), el qual retorna l'adreça d'una variable, i de desreferència (\emph{dereferencing}), el qual retorna la variable a partir de l'adreça. Els valors adreces pertanyen al tipus apuntador, però el model relacional prohibeix els valors de tipus apuntador i per tant no té REF ni DEREF; les relvar s'identifiquen pel seu nom i no cal que tinguin adreça. (Compte que en orientació a objectes una variable és el contenidor d'un valor que és un ID d'objecte, és a dir és el contenidor d'una referència).




%pendent: falta posar el name

% \newglossaryentry{SGBD-model}{ description = {Un model és}, name={Model de SGBD} }


% \newglossaryentry{SBDR-cap}{ description = {La capçalera d'un SGBDR}, name={Capçalera}, parent={SGBD-model} }



% \newglossaryentry{heading}{ description = {Equivalent to intension and relation schema} }
% \newglossaryentry{intension}{ description = {}, see=heading }
% \newglossaryentry{relation schema}{ description = {}, see=heading }

% \newglossaryentry{body}{ description = {Equivalent to extension} }
% \newglossaryentry{extension}{ description = {buit}, see=body}


% \newglossaryentry{DBMS data model}{ description = {A data model (first sense) is an abstract, self-contained, logical definition of the
% objects, operators, and so forth, that together constitute the abstract machine with which
% users interact. The objects allow us to model the structure of data. The operators allow us
% to model its behavior.\cite{date}. Sometimes it is referred as architecture.
% } }

% \newglossaryentry{data model}{ description = { A data model (second sense) is a model of the persistent data of some particular
% enterprise. [date06]}}


% \newglossaryentry{DBMS implementation}{ description = {An implementation of a given data model is a physical realization on a real
% machine of the components of the abstract machine that together constitute that model.\cite{date}} }


% \newglossaryentry{data independence}{ description = {model and implementation kept separated}}




% \newglossaryentry{relationships}{
% description={relationships are semantic. relationships are entities.}}








%%% Local Variables: 
%%% mode: latex
%%% TeX-master: "main"
%%% End: 

\loadglsentries{vocabulari/abreviacions.tex}
\loadglsentries[notation]{vocabulari/notacio.tex}

%-------------- dades --------------------------
\hypersetup{
    pdftitle={Disseny i modelització d'un sistema de gestió multiresolució per a sèries temporals},
    pdfauthor={Aleix Llusà Serra},
    pdfcreator={DiPSE--UPC},
    pdfsubject={Tesi 2011--2014},
    pdfkeywords={sèries temporals; model de dades; sistemes de bases de dades; sistemes de monitoratge},
    pdflang={ca},
}

\title{Disseny i modelització d'un sistema de gestió multiresolució per a sèries temporals}
\author{Aleix Llusà Serra}
%----------------------------------------------


\begin{document}


%------------- Pàgina de títol ------------
%\maketitle
%%------------- pàgina de portada -----------
\begin{titlepage}
  \begin{center} 

   

    {\Large \scshape Universitat Politècnica de Catalunya} \vskip 1cm 

    {Programa de Doctorat:} \vskip 0.5cm 
    
    {\scshape Automàtica, Robòtica i Visió} \vfill%\vskip 4cm 

    {Tesi Doctoral} \vskip 1cm 
    
    {\scshape \bfseries \Large Disseny i modelització d'un sistema de gestió\\
 multiresolució per a sèries temporals} \vskip 2cm

    {\bfseries Aleix Llusà Serra} \vfill%\vskip 4cm 

    {Direcció:}
       
    {Teresa Escobet Canal i
    Sebastià Vila-Marta}  \vskip 1cm 
    %\vfill 

    {Juny de 2015}

\end{center}
\end{titlepage}

%----------------------------------------------

%------------- Abstract ------------
%\begin{abstract}
%\chapter*{Resum}

\todo{repassar la taula de notació que a vegades fa un salt de pàgina on no toca}

Actualment és possible d'adquirir una gran quantitat de dades,
principalment gràcies a la facilitat de disposar de sistemes de
monitoratge amb grans xarxes de sensors. Això no obstant, no és tan
senzill de gestionar posteriorment totes aquestes dades.  A més, també
cal tenir en compte com s'emmagatzemen aquestes dades.


D'una banda, l'adquisició de valors d'una variable al llarg del temps
es formalitza com a sèrie temporal. Així, hi ha multitud d'algoritmes
i metodologies d'anàlisi de sèries temporals que descriuen com
extreure informació de les dades. D'altra banda, l'emmagatzematge i la
gestió de les dades es formalitza com a \glspl{SGBD}. Així, hi ha
sistemes informàtics dedicats a inferir la informació que un usuari
vol consultar. Aquests sistemes són descrits per models lògics
formals, entre els quals el model relacional n'és la referència
principal.


En aquesta tesi dissertem sobre el fet d'emmagatzemar només aquella
part de les dades originals que conté una certa informació
seleccionada. Aquesta selecció de la informació es duu a terme
mitjançant el resum de diferents resolucions de les dades, cadascuna
de les quals bàsicament són agregacions de les dades a intervals de
temps periòdics. A aquesta tècnica l'anomenem multiresolució.



La multiresolució s'aplica a les sèries temporals. Com a resultat,
s'obtenen subsèries temporals de mida finita i amb la informació
resumida. Per tal de gestionar les sèries temporals, s'utilitzen
\gls{SGBD} específics anomenats \glspl{SGST}. Així doncs, proposem
\gls{SGST} amb capacitats de multiresolució i els anomenem
\glspl{SGSTM}. De la mateixa manera que en els \gls{SGBD}, formalitzem
un model pels \gls{SGST} i pels \gls{SGSTM}.



A causa de la naturalesa de variable capturada al llarg del temps, en
l'adquisició de les sèries temporals apareixen propietats
problemàtiques. Els \gls{SGSTM} tenen en compte algunes d'aquestes
propietats com:
\begin{itemize}
\item La sincronització dels rellotges en els diferents sistemes
  d'adquisició.
\item L'aparició de dades desconegudes perquè no s'han pogut adquirir
  o perquè són errònies.
\item La gestió d'una quantitat enorme de dades, i que a més segueix
  creixent al llarg del temps.
\item Les consultes amb dades que no s'han recollit de manera uniforme
  en el temps.
\end{itemize}


Ara bé, els \gls{SGSTM} són uns sistemes que emmagatzemen unes dades
segons una selecció d'informació i descarten les que no es consideren
importants. Per tant, prèviament a l'emmagatzematge, cal decidir els
paràmetres de selecció de la informació. Per tal d'avaluar la qualitat
d'aquests sistemes, depenent dels paràmetres que s'escullin, es pot
utilitzar la teoria de la informació. En aquest sentit, la
multiresolució es pot considerar com una tècnica de compressió amb
pèrdua. Així doncs, introduïm una reflexió sobre com avaluar l'error
que es comet amb la multiresolució en comparació amb disposar de totes
les dades originals.


Com es diu actualment en l'àmbit dels \gls{SGBD}, un mateix sistema no
pot ser adequat per a tots els contextos. A més, els sistemes han de
tenir en compte un bon rendiment en altres recursos a part del temps
de computació, com per exemple la capacitat finita, el consum
d'energia o la transmissió per la xarxa. Així doncs, dissenyem
diverses implementacions del model dels \gls{SGSTM}. Aquestes
implementacions exploren diverses tècniques de computació: computació
incremental seguint el flux de dades, computació para\l.lela i
computació de bases de dades relacional.


En resum, en aquesta tesi dissenyem els \gls{SGSTM} i en formalitzem
un model.  Els \gls{SGSTM} són útils per a emmagatzemar sèries
temporals en sistemes amb capacitat finita i per a precomputar la
multiresolució. D'aquesta manera, permeten disposar de consultes i
visualitzacions immediates de les sèries temporals de forma
resumida. Això no obstant, impliquen una selecció de la informació que
cal decidir prèviament. En aquesta tesi proposem consideracions i
reflexions sobre els límits de la multiresolució.





\chapter*{Abstract}






\glsreset{SGBD}
\glsreset{SGST}
\glsreset{SGSTM}


%%% Local Variables: 
%%% mode: latex
%%% TeX-master: "main"
%%% End: 

%  LocalWords:  multiresolució

%\end{abstract}
%----------------------------------------------

%------------- Índexs ------------
%\listoftodos
\cleardoublepage\pdfbookmark{\contentsname}{bookmark:index}\tableofcontents{}
%\addtocontents{toc}{\protect\enlargethispage{1cm}}
%\listoffigures
%\listoftables
%\lstlistoflistings
%----------------------------------------------



%------------- Cos ------------
%\frontmatter
%\mainmatter


\chapter{Model SGST}

En aquest capítol es defineixen els objectes que ens permeten modelar l'estructura de les dades.

Dos models estructurals:

* Hi ha un model pels SGST (TSMS) que inclou mesura i sèries temporals.

* Hi ha un model pels SGSTM (MTSMS) que té buffer, discs i mtsdb, els quals inclouen el model de sèrie temporal del SGST.




\section{Model estructural de dades}

Una sèrie temporal és una relació de temps i valors. A cada parella
temps-valor l'anomenem mesura. Així doncs, una sèrie temporal és un
conjunt de mesures. Una mesura és un tuple temps-valor.



Una mesura és un valor mesurat en un instant de temps i una sèrie
temporal és un co\l.lecció de mesures.





\subsection{Temps}

Utilitzem el temps com un valor que ens permet ordenar les mesures.  A
tal efecte, el domini del temps es defineix com un conjunt tancat
(compactificat) i amb ordre total. No obstant, pot ser tant un conjunt
finit com infinit.

Per facilitar la comprensió, en el document utilitzarem el conjunt de
reals com a conjunt pels temps. Concretament, per a complir que sigui
un conjunt tancat usarem el conjunt estès de nombres reals
$\bar{\mathbb{R}} \in \mathbb{R} \cup
\{+\infty,-\infty\}$, \parencite{wiki:extendedreal,cantrell:extendedreal},
també anomenat recta real acabada.


El conjunt estès de nombres reals té dos punts límits corresponents al
valor impropi infinit, aleshores en notació d'interval el conjunt es
pot escriure com $\bar{\mathbb{R}} \in [-\infty,+\infty]$.  Més
endavant a la definició~\ref{def:model:mesura_indefinida} es detallen
algunes propietats induïdes a les mesures com a resultat d'aquesta
extensió.

Les relacions d'ordre i algunes operacions aritmètiques s'estenen al
conjunt $\bar{\mathbb{R}}$, \cite{cantrell:extendedreal}.  Algunes
expressions esdevenen indefinides (p.ex.\ $0/0$) i altres depenen del
context, com és el cas de l'expressió indeterminada $0 \times \infty$ que
per exemple en la teoria de la mesura habitualment es defineix com $0 \times
\infty = 0$, \cite{wiki:extendedreal}.


El conjunt dels reals és un espai mètric ja que té definida una funció
distància (o mètrica), com per exemple la distància euclidiana. Com a
conseqüència, ens permet distingir entre instants de temps (els
elements del conjunt) i durades (la mètrica). Observant els instants
de temps com a punts en la recta real i les durades com a segments de
la recta real, es pot definir el temps com a sistema de coordenades
especificant un instant com a marc de
referència, \parencite{iep:time-supplement,wiki:coordinate}.


\begin{definition}[Temps]
  \label{def:model:temps}
  Siguin $t^i_i$ i $t^i_j$ dos instants de temps, observem la quantitat
  de temps o la durada $t^d$ com un valor $t^d \in\bar{\mathbb{R}}$
  que mesura la distància en unitats de temps entre dos temps
  absoluts $t^d = t^i_i - t^i_j$.
  
  Sigui $t^d$ una durada i $t^{R}$ un temps absolut de referència,
  observem un instant de temps $t^i$ com un valor $t^i
  \in\bar{\mathbb{R}}$ que mesura la quantitat de temps respecte al
  temps de referència $t^i= t^{R} + t^d$ . Aquest valor de referència
  $t^{R}\in\mathbb{R}$ és també un instant de temps però que permet
  definir unívocament la posició de qualsevol altre instant de temps.


\end{definition}

En resum, els instants de temps es poden veure com una seqüència de
valors reals que indiquen esdeveniments amb ordre clarament definit i
entre dos instants de temps sempre hi ha una durada. Tant els instants
de temps com les durades s'expressen amb un real que té unitats de
temps. Aquestes unitats són 'segons' en sistema internacional.



\subsubsection{Calendari}
\textcite{dreyer94} situen els calendaris i les seves operacions com a
essencials en els SGST. Tanmateix, pot no ser necessari modelar les
dates de calendari en el model de temps. El temps és la línia contínua
de temps, el calendari són nom especials a certs punts de la línia de
temps. Només cal una eina que sigui capaç de convertir de noms a
instants de temps.

Per una banda, no afecta al model SGST que els calendaris siguin més o
menys complicats, en aquest cas només es veuen complicades les
funcions de conversió de temps a calendari i viceversa.  Per altra
banda, tampoc afecta que els calendaris siguin ambigus (p.ex.\ dos
noms per al mateix instant o instants sense nom) o que continguin
propietats impredictibles (p.ex.\ cas dels segons addicionals
(intercalats) en UTC) ja que la responsabilitat d'aquests problemes
correspon a la bona definició dels sistemes de calendari.


Unix time (posix) no incorpora els leap seconds.
Millor TAI, unix time (right), ja que és una mesura totalment lineal del temps. 
Unix time, UTC i TAI: http://lwn.net/Articles/504744/


\subsection{Valor}

El \gls{terme:SGBDR:valor} és qualsevol element que és d'un
\gls{terme:SGBDR:tipus}; és a dir, un objecte que
pertany a un determinat conjunt de valors i que té associat les
operacions que s'hi poden aplicar. Exemples de tipus de dades són els
enters, els reals, les cadenes de text i les estructures de dades com
vectors, llistes o \glspl{terme:SGBDR:relacio}.  \todo{vigilar amb
  date, ell en diu escalars i no escalars (amb components visibles) i
  per exemple considera que un punt és escalar}

El model de dades dels valors ha d'incloure una dada que defineixi el
valor indefinit. Més endavant a la
definició~\ref{def:model:mesura_valor_indefinit} es detallen les
propietats de les mesures amb valor indefinit. Seguint l'exemple amb
els reals, el valor indefinit es defineix amb el valor impropi infinit
del conjunt dels reals estès
projectivament, \parencite{cantrell:projectivelyextendedreal},
$\mathbb{R}^*\in\mathbb{R} \cup \{\infty\}$.  En aquest cas el valor
és un escalar però fàcilment es pot estendre el concepte a valors
multivaluats ${\mathbb{R}^*}^n$ que representin una co\l.lecció de
valors mesurats en el mateix instant de temps, tal i com fa per
exemple \textcite{assfalg08:thesis}.





\subsection{Mesura}\label{sec:model:mesura} 

Una mesura és una parella de temps i valor.

\begin{definition}[Mesura]
  \label{def:model:mesura}
  Definim \emph{mesura} com el tuple $(t,v)$, en el que $v$ és el
  valor de la mesura i $t$ és l'instant de temps en que s'ha pres
  aquesta mesura.
\end{definition}


Donada una mesura $m=(t,v)$ escriurem $V(m)$ per referir-nos a $v$ i
$T(m)$ per referir-nos a $t$.

Donades dues mesures és fàcil establir la relació d'ordre induïda pel
temps.

\begin{definition}[Relació d'ordre]
  \label{def:model:mesura-relacio-ordre}
  Sigui $m=(t_m,v_m)$ i $n=(t_n,v_n)$. Direm que $m\geq n$ si i solament
  si $t_m\geq t_n$.
\end{definition}


En les definicions de temps i valor s'han estès els conjunts amb
valors impropis, concretament s'ha exemplificat amb el conjunt estès
de nombres reals afí $\bar{\mathbb{R}} \in \mathbb{R} \cup
\{+\infty,-\infty\}$ i amb el projectiu $\mathbb{R}^*\in\mathbb{R}
\cup\{\infty\}$,
\parencite{cantrell:extendedreal,cantrell:projectivelyextendedreal}. Aquesta
extensió amb l'element impropi infinit ($\infty$) dóna com a resultat
unes mesures impròpies que anomenarem mesura de valor indefinit i
mesura indefinida.

\begin{definition}[Mesura de valor indefinit]
  \label{def:model:mesura_valor_indefinit}
  Definim \emph{mesura de valor indefinit} com el tuple $(t,v)$, en el
  que el valor és $v=\infty$ i l'instant de temps és
  $t\in\bar{\mathbb{R}}$.
\end{definition}

\begin{definition}[Mesura indefinida]
  \label{def:model:mesura_indefinida}
  Definim \emph{mesura indefinida} com el tuple $(t,v)$, en el que el
  valor és $v\in\mathbb{R}^*$ i l'instant de temps és
  $t\in\{+\infty,-\infty\}$.
\end{definition}

Així doncs, sigui $m$ una mesura, es podrà notar la mesura de valor
indefinit com $m=(t,\infty)$ i les mesures indefinides com
$m=(+\infty,v)$ per la positiva i $m=(-\infty,v)$ per la negativa, les
quals normalment s'anotaran també amb valor indefinit:
$m=(+\infty,\infty)$ i $m=(-\infty,\infty)$ respectivament.


Les mesures de valor indefinit es podran utilitzar en aquells casos en
els que el valor de la mesura és desconegut. Els valors desconeguts
són aquells valors que no existeixen (es desconeixen, \emph{missing
  data} ) o que s'ignoren (es descarten, \emph{censoring} o
\emph{truncation}). Els valors que no existeixen prenen el valor
desconegut en el moment de la mesura, en canvi els valors descartats
són marcats com a desconeguts després d'un processament de les dades.

Nota: en alguns sistemes es distingeix entre valors infinits
($\infty$) i valors indefinits (NaN, \emph{not a number}),
\cite{wiki:ieee754}. Aquest no és el cas de les definicions de mesures
indefinides presents.



\subsection{Sèrie temporal}
\label{sec:model:serietemporal}

Les sèries temporals són seqüències de mesures ordenades en el temps.
Tradicionalment s'anomenen sèries temporals tot i que també s'anomenen
seqüències temporals, per exemple a \cite{last:hetland}.

\begin{definition}[Sèrie temporal]
  \label{def:serie_temporal}
  Una sèrie temporal $S$ és un conjunt de mesures
  $S=\{m_0,\ldots,m_k\}$ sense temps repetits
  $\forall i,j: i\leq k, j\leq k, i\neq j : T(m_i)\neq T(m_j)$.
\end{definition}

Per ser un conjunt, les sèries temporals tenen mesura de cardinalitat.
\begin{definition}[Cardinal]
  Sigui $S=\{m_0,\ldots,m_k\}$ una sèrie temporal, definim el nombre
  de mesures que conté la sèrie temporal com el cardinal del conjunt
  $|S|=k+1$. Una sèrie temporal sense mesures és la sèrie temporal
  buida $S_\emptyset= \emptyset$, és a dir que no té cap element
  $|S_\emptyset|=0$.
\end{definition}


 





\subsection{Relació sèrie temporal}

Una sèrie temporal és una relació de temps i valors. A cada parella temps-valor l'anomenem mesura. Així doncs, una sèrie temporal és un conjunt de mesures.

Una sèrie temporal és un conjunt de mesures, així doncs s'observa com una relació de grau dos (relació binària)  a on la capçalera conté els atributs temps i valor, ambdós amb els dominis de temps i valor ja vistos com per exemple el tipus de dades 'reals estesos'. Inclou algunes restriccions més que les relacions:

* Els temps no poden estar repetits

* Els valors han de contenir el mateix tipus d'objecte.

Els temps no repetits indueixen un ordre temporal a les sèries temporals. Tot i així, les relacions, per ser conjunts, conserven la no definció d'un ordre dels elements. 


En el model relacional no hi ha ordre en els atributs a diferència de les relacions matemàtiques que tenen un ordre d'esquerra a dreta \parencite[sec.\ 5.3]{date:introduction}.

\subsection{Exemples}

\paragraph{Exemple 1}
Sèrie temporal $S_1$ on el temps i els valors pertanyen a $\bar{\mathbb{R}}$. Conté la mesura de valor 1 en el temps 5, la mesura de valor 3 en el temps 7 i la mesura de valor 1 en el temps 10. Modelada com a relació, és a dir com a parella capçalera i conjunt de valors certs, s'escriu com 
$S_1 = ( \{temps: \bar{\mathbb{R}}, valor: \bar{\mathbb{R}}\}, \{ \{temps:5,valor:1\}, \{temps:7,valor:3\}, \{temps:10,valor:1\} \} )$.

Degut al format esquemàtic, simplifiquem l'escriptura de les sèries temporals com a conjunt de tuples $(t,v)$ a on $t$ és el temps i $v$ és el valor. Així doncs la sèrie temporal $S$ es pot escriure de manera simplificada com a 
$S = \{ (5,1), (7,3), (10,1) \}$.

Tal com s'utilitza en les relacions, les sèries temporals es poden visualitzar com a taules. La sèrie temporal $S_1$ es visualitza com a taula a la \autoref{fig:model:serietemporal:real}.

\begin{figure}[tp]
  \centering
  \begin{tabular}{|c|c|}
    \multicolumn{2}{c}{$S_1$} \\ \hline
    $t$  & $v$ \\ \hline
    5  & 1 \\
    7  & 3 \\
    10 & 1 \\ \hline
  \end{tabular}
  \caption{Taula d'una sèrie temporal amb valors reals}
  \label{fig:model:serietemporal:real}
\end{figure}


\paragraph{Exemple 2}
Sèrie temporal $S_2$ on el temps pertany a $\bar{\mathbb{R}}$ i el valor pertany a  $\bar{\mathbb{R}}^3$; és a dir és un vector. Conté el valor (1,2,3) en el temps 5, el valor (3,4,5) en el temps 7 i el valor (1,2,3) en el temps 10.

De manera simplificada s'escriu com 
$S_2 = \{ (5,(1,2,3)), (7,(3,4,5)), (10,(1,2,3)) \}$ i es visualitza com a taula a la \autoref{fig:model:serietemporal:vector}. No obstant, es pot visualitzar de forma més còmode com a $S_2^b = \{ (5,1,2,3), (7,3,4,5), (10,1,2,3) \}$

\begin{figure}[tp]
  \centering
  \begin{tabular}{|c|c|}
    \multicolumn{2}{c}{$S_2$} \\ \hline
    $t$  & $v$ \\ \hline
    5  & (1,2,3) \\
    7  & (3,4,5) \\
    10 & (1,2,3) \\ \hline
  \end{tabular} \qquad
  \begin{tabular}[tp]{|c|c|c|c|}
   \multicolumn{4}{c}{$S_2^b$} \\ \hline
    $t$  & $v_1$ & $v_2$ & $v_3$ \\ \hline
    5  & 1 & 2 & 3 \\
    7  & 3 & 4 & 5 \\
    10 & 1 & 2 & 3 \\ \hline
  \end{tabular}

  \caption{Taula d'una sèrie temporal amb valors vectors}
  \label{fig:model:serietemporal:vector}
\end{figure}


\paragraph{Exemple 3} \emph{Valors relació}. \label{par:model:exemple-relvalues}
Sèrie temporal $S_3$ on el temps pertany a $\bar{\mathbb{R}}$ i el valor és una sèrie temporal del mateix format que en l'exemple 1. Conté les tuples de $S_1$ com a valors en el temps 1 i 2. 

De manera simplificada s'escriu com
$S_3 =  \{ (1,\{ (5,1), (7,3), (10,1) \}), 
(2,\{ (5,1),$ $(7,3),$ $(10,1) \}) \}$
i es visualitza com a taula a la \autoref{fig:model:serietemporal:serietemporal}.


\begin{figure}[tp]
  \centering
  \begin{tabular}{|c|c|}
    \multicolumn{2}{c}{$S_3$} \\ \hline
    $t$  & $v$ \\ \hline
    1 &   
       \begin{tabular}{|c|c|}
         \hline
         $t$  & $v$ \\ \hline
         5  & 1 \\
         7  & 3 \\
         10 & 1 \\ \hline
       \end{tabular} \\ \hline
    2 & 
       \begin{tabular}{|c|c|}
         \hline
         $t$  & $v$ \\ \hline
         5  & 1 \\
         7  & 3 \\
         10 & 1 \\ \hline
       \end{tabular} \\ \hline
  \end{tabular}
  \caption{Taula d'una sèrie temporal amb valors sèrie temporal}
  \label{fig:model:serietemporal:serietemporal}
\end{figure}


S'observa que la capçalera de $S3$ és $\{temps:\bar{\mathbb{R}},valor:
relacio\{temps:\bar{\mathbb{R}},valor:\bar{\mathbb{R}}\}\}$ \parencite[sec.\ 5.3]{date:introduction}. És a dir, el valor és de tipus relació que es defineix amb la capçalera de la relació on el temps i el valor pertanyen a $\bar{\mathbb{R}}$. Per tant, el valor de $S3$ és de tipus sèrie temporal amb valors reals. Cal insistir que \emph{tots} el valors de $S3$ han de pertànyer al mateix domini \parencite[sec.\ 5.4]{date:introduction}, el qual és $relacio\{temps:\bar{\mathbb{R}},valor:\bar{\mathbb{R}}\}$.



\paragraph{Exemple 4} \emph{Variable relació}.\todo{això no es pot fer, perquè no existeix el tipus relvar?? però les tuples poden contenir expressions?? No existeixen les tuplevar (Date rebutja ferotjament els apuntadors a dins dels DBMS) [Date on database :writings 2000-2006 / C.J. Date]}
Sèrie temporal $S_4$ on el temps pertany a $\bar{\mathbb{R}}$ i el valor és una referència a una sèrie temporal. Conté $S_1$ com a valors en el temps 1 i 2. 

De manera simplificada s'escriu com
$S_4 =  \{ (1,S_1) , (2,S_1) \}$ 
i es visualitza com a taula a la \autoref{fig:model:serietemporal:relvar}.

S'aplica el concepte de variable relació (\emph{relvar}) dels SGBDR \parencite[sec.\ 3.3]{date:introduction}.
Així doncs, cal notar que $S_4$  no és el mateix que $S_3$.
\begin{figure}[tp]
  \centering
  \begin{tabular}{|c|c|}
    \multicolumn{2}{c}{$S_4$} \\ \hline
    $t$  & $v$ \\ \hline
    1 & $S_1$ \\
    2 & $S_1$ \\ \hline
  \end{tabular}
  \caption{Taula d'una sèrie temporal amb valors \emph{relvar}}
  \label{fig:model:serietemporal:relvar}
\end{figure}


Relació de noms i sèries temporals $R =  ((nom:string,serie:relacio\{temps:\bar{\mathbb{R}},valor:\bar{\mathbb{R}}),\{ ('S_1',S_1),('S_2',S_2)  \})$

Sèrie temporal amb strings com a valors:
$N= ( (temps:\bar{\mathbb{R}},valor:string) ,\{ (1,'S_1') , (2,'S_1') \})$

Sèrie temporal com a variable relació de vista (relvar view)
$S_4 =  (N RENAME valor as nom) JOIN R$
\todo{cal definir una view}



\subsection{Naturalesa de les sèries temporals}


Perquè RRDtool diferencia entre comptadors i magnituds?

[segev87] diferencia entre step-wise constant, discret (potser aquest tal com se'l defineix són intervals temporals), continu. Ho anomena tipus de la sèrie temporal i diu que es poden definir interpolacions per cada una.


[John G. Proakis, Dimitris G. Manolakis 2007 Tratamiento digital de señales/Digital signal processing 4a ed pp11-12(segons wikipedia)] Acquisition: Discrete signals may have several origins, but can usually be classified into one of two groups:[1]
*By acquiring values of an analog signal at constant or variable rate. This process is called sampling.[2]
*By recording the number of events of a given kind over finite time periods. For example, this could be the number of people taking a certain elevator every day.



\subsubsection{Regularitat de les sèries temporals} 

Sigui $S=\{m_0,\ldots,m_k\}$ una sèrie temporal, $t$ un instant de
temps i $\delta$ una durada de temps, les mesures de la sèrie temporal
es poden localitzar en l'interval de temps $i_0=[t,t+\delta]$ i els
seus múltiples $i_j=[t+j\delta \,,\, t+(j+1)\delta]: j=0,1,2,\ldots$.
En processat de senyal aquests intervals de temps s'anomenen intervals
de mostreig, $\delta$ s'anomena període de mostreig i $t$ s'anomena
temps inicial del mostreig.  La sèrie temporal $S$ és de naturalesa
diferent segons la situació dels temps $T(m_i)$ en els intervals de
temps $i_j$.

Una sèrie temporal és regular quan les mesures són equidistants en el
temps, tal com ho anomenen a \cite{last:hetland}.

\begin{definition}[Sèrie temporal regular]
  Sigui $S=\{m_0,\ldots,m_k\}$ una sèrie temporal, $t$ un instant de
  temps i $\delta$ una durada de temps. Direm que $S$ és regular si i
  només si $\forall m \in S(T(\min(S),\infty):T(m) - T(\ant(m)) =
  \delta$ i $T(\min(S))=t$.
\end{definition}

Si una sèrie temporal és regular, l'anomenem sèrie temporal mostrejada
regularment amb període de mostreig $\delta$. Noteu que si es complís
la definició excepte que s'iniciés en el temps que exigim
$T(\min(S))=t$, aleshores la sèrie temporal seria equidistant però a
efectes de mostreig no la podríem anomenar regular; sí que seria una sèrie temporal de temps real (v.\ def.~\ref{def:st:tempsreal}).


Una sèrie temporal és no regular quan no és regular. 
En les sèries temporals no regulars es poden distingir tres casos: temps real, ultramostreig i inframostreig.

Una sèrie temporal és de temps real quan a cada interval de mostreig hi ha una i només una mesura. L'interval de mostreig pot estar acotat per una durada anomenada termini.

\begin{definition}[Sèrie temporal de temps real]\label{def:st:tempsreal}
  Sigui $S=\{m_0,\dotsc,m_k\}$ una sèrie temporal, $t$ un instant de
  temps, $\delta$ una durada de temps i $D$ una durada que indica
  termini. Direm que $S$ és de temps real si i només si $D\leq\delta$
  i $\forall n\in\{0,\ldots,|S|-1\}: \exists!m \in
  S(t+n\delta,t+n\delta+D]$.  Aleshores la sèrie temporal està
  mostrejada en temps real per al temps de mostreig $\delta$ amb
  compliment del termini $D$.
\end{definition}

Si una sèrie temporal és de temps real, l'anomenem  sèrie temporal mostrejada
en temps real amb període de mostreig $\delta$ i compliment del termini $D$.
Si $D=\delta$, es pot anomenar que $S$ és una sèrie temporal de temps real sense termini.


% \paragraph{Ultramostreig} Una sèrie temporal està ultramostrejada (\emph{upsampling}) quan a cada interval de mostreig hi ha una mesura o més d'una. 
% \[
% \text{Ultramostrejada?}: \text{Sèrie temporal} \times T_0 \times \delta \longrightarrow \text{Booleà}
% \]

% Una sèrie temporal $S$ està ultramostrejada ssi $S$ no és de temps real i $\exists m_i=(v_i,t_i)\in S:T_0+(n-1)\delta \leq t_i < T_0+n\delta:\forall n\in\{1,\ldots,|S|\}$.

% \paragraph{Inframostreig} Una sèrie temporal està inframostrejada (\emph{downsampling}) quan en algun interval de mostreig no hi ha cap mesura. 
% \[
% \text{Inframostrejada?}: \text{Sèrie temporal} \times T_0 \times \delta \longrightarrow \text{Booleà}
% \]

% Una sèrie temporal $S$ està inframostrejada ssi $\nexists m_i=(v_i,t_i)\in S:T_0+(n-1)\delta \leq t_i < T_0+n\delta:\forall n\in\{1,\ldots,|S|\}$.








\subsubsection{Representació de les sèries temporals}



La naturalesa indueix representacions?
Jo puc utilitzar qualsevol representació donada una sèrie temporal, però això em pot causa perjudici si no s'adiu amb la naturalesa.


La representació serveix per interpolar:

zoh, zoh cap enrere, lineal, etc.


Una sèrie temporal és la representació discreta d'una funció contínua. A partir de la sèrie temporal es pot definir una funció contínua. 

A teoria de senyal s'estudia com fer que aquesta s'aproximi a la real. Estudiant com a senyal fan: donada una sèrie temporal dir quina funció s'hi 'ajusta' més. 

Però jo puc preguntar donada una sèrie temporal quina funció representa i puc dir per representar a zohe és tal, per representar a lineal és qual. 

Potser millor dir-li interpretació?



\paragraph{Representació de sèries temporals}

\textcite{last:keogh}, cita vàries representacions per les sèries temporals com per exemple \emph{Fourier Transforms}, \emph{Wavelets}, \emph{Symbolic Mappings} o \emph{Piecewise Linear Representation} (PLR), però assenyala aquesta última com la representació més utilitzada. 
La PLR, funció definida a trossos lineal, és l'aproximació d'una sèrie temporal $S$, de llargada $n$, amb $K$ segments rectes. Els segments podrien ser polinomis de qualsevol grau, però la manera més comuna de representar sèries temporals és amb funcions lineals, segons Keogh, \cite{keogh02}.
Per aproximar el segment $S(t_a:t_b]$ d'una sèrie $S$, Keogh defineix dues tècniques: interpolació lineal, la recta que connecta $t_a$ i $t_b$, i regressió lineal, la millor recta que aproxima per mínims quadrats el segment entre $t_a$ i $t_b$.

Però també es pot representar una sèrie temporal amb una funció esglaó (\emph{step} o \emph{staircase function}); és a dir, amb una funció definida a trossos constant (\emph{piecewise constant representation}).
La representació a trossos constant és utilitzada en electrònica als convertidors digital-analògic (DAC, \emph{digital-to-analog converter}). En aquest cas, un senyal discret es considera una sèrie temporal i per reconstruir el senyal continu típicament s'aplica el model de \emph{zero-order hold}, equivalent a la representació a trossos constant,  o el de \emph{first-order hold},  equivalent a la representació a trossos lineal.
El model de \emph{zero-order hold} consisteix en mantenir constant cada valor fins al proper. S'obté una representació a trossos constant que en electrònica s'anomena seqüència de pulsos rectangulars (\emph{rectangular pulses}).

%http://en.wikipedia.org/wiki/Piecewise

%http://ca.wikipedia.org/wiki/Funció_definida_a_trossos

%http://en.wikipedia.org/wiki/Rectangular_function

%http://en.wikipedia.org/wiki/Step_function

% Piecewise Aggregate Approximation (PAA) \cite{keogh00}: aproxima una sèrie temporal partint-la en segments de la mateixa mida i emmagatzemant la mitjana dels punts que cauen dins del segment. Redueix de dimensió $n$ a dimensió $N$

% Adaptive Piecewise Constant Approximation (APCA) \cite{keogh01}: com el PAA però amb segments de mida variable.

A continuació,  la representació  d'una sèrie temporal segons el model de \emph{zero-order hold} s'estén per diferents continuïtats en els intervals de temps de representació.

Sigui $S$ una sèrie temporal, es defineix $S(t)$ com la representació
de la sèrie temporal contínuament al llarg del temps $t$.  En primer
lloc, es representa amb \emph{zero-order hold} a partir de funcions
graó contínues per la dreta (\emph{right-continuous}).

\begin{definition}[Representació amb \emph{zero-order hold}]
Sigui $S=\{m_0,\ldots,m_k\}$ una sèrie temporal, la representació  $S(t)$ amb \emph{zero-order hold} es defineix
\[
\forall t \in \mathbb{R} ,\forall m \in S: S(t) =
\begin{cases}
  V(\min S) & \text{si } t < T(\min S) \\
  V(m) & \text{si }  t\in [T(m),T(\seg m))
\end{cases}
\]
\end{definition}

En segon lloc, es representa $S(t)$ amb \emph{zero-order hold} centrada en
l'interval, definit també a partir de funcions graó contínues per la
dreta.

\begin{definition}[Representació amb \emph{zero-order hold} centrada en l'interval]
  Sigui $S=\{m_0,\ldots,m_k\}$ una sèrie temporal, la representació
  $S(t)$ amb \emph{zero-order hold} centrada en l'interval es defineix
\[
\forall t \in \mathbb{R}  ,\forall m \in S:
S(t) =  
\begin{cases}
  V(m) & \text{si } t = \frac{T(\ant m)+T(m)}{2} \\
  V(m) & \text{si } t\in \left( \frac{T(\ant m)+T(m)}{2},\frac{T(m)+T(\seg m)}{2} \right) \
\end{cases}
\]
\end{definition}

En tercer lloc, es representa $S(t)$ amb \emph{zero-order hold} cap enrere, ara definit a partir de funcions graó contínues per l'esquerra.
\begin{definition}[Representació en \emph{zero-order hold} cap enrere]
  Sigui $S=\{m_0,\ldots,m_k\}$ una sèrie temporal, la representació
  $S(t)$ amb \emph{zero-order hold} cap enrere es defineix
\[
\forall t \in \mathbb{R}  ,\forall m \in S:
S(t) =  
\begin{cases}
  V(\max S) & \text{si } t > T(\max S) \\
  V(m) & \text{si }  t\in (T(\ant m),T(m)]
\end{cases}
\]
\end{definition}

Sigui $S$ una sèrie temporal regular i $\delta$ una durada de temps, aleshores la representació de $S(t)$ amb \emph{zero-order hold} és la mateixa que la de $S(t-\delta)$ amb \emph{zero-order hold} cap enrere i és la mateixa que la de $S(t-\frac{\delta}{2})$ centrada en l'interval. 














%%% Local Variables:
%%% TeX-master: "main"
%%% End:







% LocalWords:  SGST

\section{Model d'operacions}

Una sèrie temporal té un atribut de temps que ha de ser tingut en
compte pels operadors que la manipulin.  Així, atenent a aquest
atribut de temps, el comportament d'una sèrie temporal pot tenir
naturaleses diferents:
\begin{itemize}
\item Conjunt, és a dir els operadors només atenen a la forma
  estructural bàsica.
\item Seqüència, en la qual els operadors la tracten com a conjunts
  amb ordre.
\item Funció temporal, en la qual els operadors treballen tenint en
  compte que una sèrie temporal és la representació d'un funció
  contínua.
\end{itemize}


En el disseny del model d'operacions següent es distingeix el
comportament per als tres casos anteriors.  Es dissenyen les
operacions bàsiques que permeten que posteriorment es combinin per
elaborar-ne de més complexes.




\subsection{Bàsiques de conjunts}

En el model estructural de SGST hem definit les sèries temporals
utilitzant conjunts. En aquest apartat definim operadors per a les
sèries temporals recollint els operadors habituals que tenen els
conjunts.

El model relacional de SGBD defineix els seus operadors bàsics a
partir de l'àlgebra de
conjunts \parencite[cap.~6]{date:introduction}. En aquest apartat
apliquem el mateix estudi per al model de SGST. Tot i així de manera
simplificada, a les definicions no es descriuen les sèries temporals
com a relacions amb capçaleres sinó que se n'escriuen només els
conjunts de valors. Seguint el model de \citeauthor{date:introduction}
es poden estendre les definicions i introduir el model complet de
relacions.



\subsubsection{Pertinença i inclusió}

La pertinença determina si un element pertany a un conjunt.  Sigui
$S=\{m_0,\ldots,m_k\}$ una sèrie temporal i $m$ una mesura, es
defineix de la mateixa manera que en els conjunts la pertinença de la
mesura a la sèrie temporal $m \in S$. Atenent a l'atribut de temps, es
defineix la pertinença temporal d'una mesura a una sèrie temporal.
\begin{definition}[Pertinença temporal]
  Sigui $S=\{m_0,\ldots,m_k\}$ una sèrie temporal i $m=(t,v)$ una
  mesura.  Direm que la mesura pertany temporalment a la sèrie
  temporal $m \inst S$ si i només si $\exists m_a \in S : T(m) =
  T(m_a)$.
\end{definition}


La inclusió determina si tots els elements d'un conjunt pertanyen a un
altre conjunt. Siguin $S_1=\{m_0^1,\ldots,m_k^1\}$ i
$S_2=\{m_0^2,\ldots,m_l^2\}$ dues sèries temporals, la primera sèrie
temporal està inclosa en la segona $S_1 \subseteq S_2$ si $\forall m
\in S_1: m \in S_2$. Aleshores, $S_1$ és una subsèrie temporal de
$S_2$. Aquesta definició d'inclusió determina un ordre parcial entre
les sèries temporals.



\subsubsection{Màxim i suprem}

La relació definida a~\ref{def:model:mesura-relacio-ordre} indueix
sobre una sèrie temporal una relació d'ordre total. Com que la sèrie
temporal s'ha considerat finita i sense elements repetits, quan la
sèrie temporal no és buida això comporta l'existència d'un màxim i
d'un mínim.  Si $S$ és una sèrie temporal, $\max(S)$ i $\min(S)$ són
respectivament la mesura màxima i mínima d'$S$.

\begin{definition}[Màxim i mínim]
  Sigui $S=\{m_0,\ldots,m_k\}$ una sèrie temporal i $n\in S$ una
  mesura.  Direm que $n=\max(S)$ és el màxim de la sèrie temporal si i
  només si $\forall m \in S: n \geq m $.  Direm que $n=\min(S)$ és el
  mínim de la sèrie temporal si i només si $\forall m \in S: n \leq
  m$.
\end{definition}

El $\max(S)$ i el $\min(S)$ no estan definits quan la sèrie temporal
és buida: $S= \emptyset$. En
canvi, el suprem i l'ínfim estan definits per qualsevol
sèrie temporal tal com passa amb el conjunt estès de nombres reals,
\cite{cantrell:extendedreal}.  

\begin{definition}[Suprem i ínfim]\label{def:sgst:sup}\label{def:sgst:inf}
  Sigui $S=\{m_0,\ldots,m_k\}$ una sèrie temporal i $n\in S$ una
  mesura.  Direm que $n=\sup(S)$ és el suprem de la sèrie temporal si
  $n=\max(S)$ en cas que el màxim estigui definit o
  $n=(-\infty,\infty)$ en cas contrari.  Direm que $n=\inf(S)$ és
  l'ínfim de la sèrie temporal si $n=\min(S)$ en cas que el mínim
  estigui definit o $n=(+\infty,\infty)$ en cas contrari.
\end{definition}

Quan la sèrie temporal no és buida, per
ser un conjunt finit i d'ordre total, sempre hi ha un i només un màxim
i un mínim i per tant es corresponen amb el suprem i l'ínfim
respectivament.




\subsubsection{Unió}

La unió de dos conjunts és un conjunt que conté tots els elements
d'ambdós conjunts.  Per a poder unir dos conjunts amb estructura de
relació, $A \cup B$, cal que tots dos tinguin la mateixa estructura;
és a dir, en termes de SGBDR cal que $A$ i $B$ tinguin la mateixa
capçalera.

Per tal que l'operació d'unió de conjunts sigui vàlida per a les
sèries temporals cal, a més, tenir en compte quan dues sèries
temporals tenen mesures en el mateix instant de temps. En cas
d'utilitzar l'operació d'unió de conjunts la sèrie temporal resultant
no compliria amb la definició \ref{def:serie_temporal} ja que
contindria mesures amb temps repetits. Com a conseqüència, es
defineixen dues operacions d'unió per a les sèries temporals que
resolen la restricció del temps de forma diferent.

En primer lloc, es defineix la unió de dues sèries temporals que
escull les mesures del primer operand en cas de mesures repetides.
\begin{definition}[unió]
  Sigui $S_1=\{m_0^1, \dotsc, m_{k_1}^1\}$ i $S_2=\{m_0^2, \dotsc,
  m_{k_2}^2\}$ dues sèries temporals, la unió de les dues
  sèries temporals $S_1 \cup S_2$ és una sèrie temporal $S=\{m_0,
  \dotsc, m_k\}$ que conté totes les mesures de $S_1$ i les mesures de
  $S_2$ no repetides: $S_1 \cup S_2 = \{m^1 \in S_1 \vee m^2 \in S_2
  | m^2 \notinst S_1 \}$.
\end{definition}

Propietats de la unió de sèries temporals:
\begin{itemize}
\item La dimensió $k$ de la sèrie temporal resultant està fitada a
  $k_1 \leq k \leq k_1 + k_2$. 
\item No commutativa. En general
  $S_1\cup S_2 \neq S_2\cup S_1$ tot i que sí que es compleix
  l'equivalència respecte al cardinal $|S_1 \cup S_2| = |S_2\cup S_1|$.
\end{itemize}

En segon lloc, es defineix la unió temporal de dues sèries temporals
que és la unió sense tenir en compte les mesures que tenen el mateix
instant de temps i diferent valor.
\begin{definition}[unió temporal]
  Sigui $S_1=\{m_0^1, \dotsc, m_{k_1}^1\}$ i $S_2=\{m_0^2, \dotsc,
  m_{k_2}^2\}$ dues sèries temporals, la unió temporal de les dues
  sèries temporals $S_1 \cupt S_2$ és una sèrie temporal $S=\{m_0,
  \dotsc, m_k\}$ que conté les mesures de $S_1$ i de $S_2$ excloent
  les que només comparteixen el temps: $S_1 \cupt S_2 = \{ m^1 \in S_1
  \vee m^2 \in S_2 | m^1 \notinst S_2 \vee m^1 \in S_2, m^2 \notinst
  S_1 \}$.
\end{definition}


Propietats de la unió temporal:
\begin{itemize}
\item Commutativa
\end{itemize}



\subsubsection{Diferència}

La diferència de dos conjunts és un conjunt que conté tots els
elements del primer conjunt que no pertanyen al segon.  Per a poder
restar dos conjunts amb estructura de relació, $A - B$, cal que
tots dos tinguin la mateixa estructura; és a dir, en termes de SGBDR
cal que $A$ i $B$ tinguin la mateixa capçalera.

En la definició de l'operació de diferència cal tenir en compte les
dues pertinences possibles.

En primer lloc, es defineix la diferència atenent a la pertinença
estricta de conjunts. És a dir s'aplica la diferència de
conjunts a les sèries temporals.
\begin{definition}[diferència]
  Sigui $S_1=\{m_0^1, \dotsc, m_{k_1}^1\}$ i $S_2=\{m_0^2, \dotsc,
  m_{k_2}^2\}$ dues sèries temporals, la diferència de les dues
  sèries temporals $S_1 - S_2$ és una sèrie temporal $S=\{m_0,
  \dotsc, m_k\}$ que conté totes les mesures de $S_1$ que no pertanyen a
  $S_2$: $S_1 - S_2 = \{ m \in S_1 | m \notin S_2  \}$.
\end{definition}

En segon lloc, es defineix la diferència atenent a la pertinença
temporal.
\begin{definition}[diferència temporal]
  Sigui $S_1=\{m_0^1, \dotsc, m_{k_1}^1\}$ i $S_2=\{m_0^2, \dotsc,
  m_{k_2}^2\}$ dues sèries temporals, la diferència temporal de les
  dues sèries temporals $S_1 -^t S_2$ és una sèrie temporal
  $S=\{m_0, \dotsc, m_k\}$ que conté totes les mesures de $S_1$ que no
  pertanyen temporalment a $S_2$: $S_1 -^t S_2 = \{ m \in S_1 | m
  \notinst S_2 \}$.
\end{definition}




\subsubsection{Intersecció}

La intersecció de dos conjunts és un conjunt que conté els elements
comuns als dos conjunts.  Per a poder intersecar dos conjunts amb estructura
de relació, $A \cap B$, cal que tots dos tinguin la mateixa
estructura; és a dir, en termes de SGBDR cal que $A$ i $B$ tinguin la
mateixa capçalera.

En la definició de l'operació d'intersecció cal tenir en compte les
dues pertinences possibles.

En primer lloc, es defineix la diferència atenent a la pertinença
estricta de conjunts. És a dir s'aplica l'operació d'intersecció de
conjunts.
\begin{definition}[intersecció]
  Sigui $S_1=\{m_0^1, \dotsc, m_{k_1}^1\}$ i $S_2=\{m_0^2, \dotsc,
  m_{k_2}^2\}$ dues sèries temporals, la intersecció de les dues
  sèries temporals $S_1 \cap S_2$ és una sèrie temporal $S=\{m_0,
  \dotsc, m_k\}$ que conté les mesures de $S_1$ repetides a $S_2$:
  $S_1 \cap S_2 = \{ m \in S_1 | m \in S_2 \}$.
\end{definition}

En segon lloc, es defineix la intersecció atenent a la pertinença
temporal tenint en compte quan dues sèries temporals tenen mesures en
el mateix instant de temps però de valor diferent.
\begin{definition}[intersecció temporal]
  Sigui $S_1=\{m_0^1, \dotsc, m_{k_1}^1\}$ i $S_2=\{m_0^2, \dotsc,
  m_{k_2}^2\}$ dues sèries temporals, la intersecció temporal de les
  dues sèries temporals $S_1 \capt S_2$ és una sèrie temporal
  $S=\{m_0, \dotsc, m_k\}$ que conté les mesures de $S_1$ repetides
  temporalment a $S_2$: $S_1 \capt S_2 = \{ m \in S_1 | m \inst S_2
  \}$.
\end{definition}

Propietats de la intersecció:
\begin{itemize}
\item La intersecció és commutativa però la intersecció temporal no és
  commutativa.
\item A partir de la diferència es pot definir la intersecció: $S_1
  \cap S_2 \equiv S_1 - (S_1 - S_2)$.
\end{itemize}


\subsubsection{Diferència simètrica}

La diferència simètrica de dos conjunts és un conjunt que conté els
elements no comuns dels dos conjunts. La diferència simètrica de dos
conjunts $A \ominus B$ es defineix a partir de la diferència i la
unió:
\begin{align*}
A \ominus B  & \equiv (A-B)\cup(B-A)\\
             & \equiv (A\cup B)-(A\cap B)  \\
A \ominus B  & \subseteq A\cup B
\end{align*}

Seguint aquestes propietats es defineixen dues diferències
simètriques: una a partir de la diferència i la unió de sèries
temporals i una altra a partir de la diferència temporal i la unió
temporal.  Per tal que l'operació de diferència simètrica sigui vàlida
per a les sèries temporals cal tenir en compte quan dues sèries
temporals tenen mesures en el mateix instant de temps.

En primer lloc, es defineix la diferència simètrica excloent les
mesures amb el mateix temps però de valor diferent.
\begin{definition}[diferència simètrica]
  Sigui $S_1=\{m_0^1, \dotsc, m_{k_1}^1\}$ i $S_2=\{m_0^2, \dotsc,
  m_{k_2}^2\}$ dues sèries temporals, la diferència simètrica de les
  dues sèries temporals $S_1 \ominus S_2$ és una sèrie temporal
  $S=\{m_0, \dotsc, m_k\}$ que conté les mesures de $S_1$ o
  exclusivament les de $S_2$: $S_1 \ominus S_2 = \{ m^1 \in S_1 \vee
  m^2 \in S_2 | m^1 \notinst S_2, m^2 \notin S_1 \}$.
\end{definition}

En segon lloc, es defineix la diferència simètrica temporal excloent les
mesures amb el mateix temps.
\begin{definition}[diferència simètrica temporal]
  Sigui $S_1=\{m_0^1, \dotsc, m_{k_1}^1\}$ i $S_2=\{m_0^2, \dotsc,
  m_{k_2}^2\}$ dues sèries temporals, la diferència simètrica de les
  dues sèries temporals $S_1 \ominus S_2$ és una sèrie temporal
  $S=\{m_0, \dotsc, m_k\}$ que conté les mesures de $S_1$ o
  exclusivament les de $S_2$: $S_1 \ominus S_2 = \{ m^1 \in S_1 \vee
  m^2 \in S_2 | m^1 \notinst S_2, m^2 \notinst S_1 \}$.
\end{definition}



\subsubsection{Projecció}

La projecció és una operació dels SGBDR que selecciona uns atributs
determinats d'un conjunt. Es pot aplicar la projecció a les sèries
temporals de la mateixa manera que als
SGBDR \parencite{date:introduction}. 

Sigui $S =\{m_0, \dotsc, m_k\}$ una sèrie temporal i $A=\{a_0, \dotsc,
a_n\}$ un conjunt de noms d'atributs, la projecció de $S$ en $A$
s'escriu com $S\{a_0, \dotsc, a_n\}$.




\subsubsection{Selecció}

La selecció és una operació dels SGBDR que selecciona uns tuples
determinats d'un conjunt. Es pot aplicar la selecció a les sèries
temporals de la mateixa manera que als
SGBDR \parencite{date:introduction}. 

Sigui $S =\{m_0, \dotsc, m_k\}$ una sèrie temporal, $a_1$ i $a_2$ dos
noms d'atributs que pertanyen a $S$, i $a_1 \Theta a_2$ una expressió
booleana sobre $a_1$ i $a_2$, la selecció de $S$ per l'expressió
booleana s'escriu com $S \where a_1 \Theta a_2$.


\subsubsection{Reanomena}

El reanomena és una operació dels SGBDR que canvia el nom dels
atributs. Es pot aplicar el reanomena les sèries temporals de la
mateixa manera que als SGBDR \parencite{date:introduction}.

Sigui $S =\{m_0, \dotsc, m_k\}$ una sèrie temporal, $a$ un nom
d'atribut que pertany a $S$ i $b$ un que no hi pertany, el reanomenat
de $a$ per $b$ s'escriu com $S \rename a \as b$.





\subsubsection{Producte i junció}

El producte cartesià de dos conjunts és un conjunt que conté totes les
parelles possibles d'elements d'ambdós conjunts.  Per a poder
multiplicar dos conjunts amb estructura de relació, $A \times B$, en
termes de SGBDR cal que $A$ i $B$ no tinguin en comú noms d'atributs.
En els SGBDR, a diferència del producte de conjunts, el conjunt
resultant no és un conjunt de parells de tuples sinó un conjunt de
tuples.

\todo{} Definim el producte de dues sèries temporals, les qual en
forma canònica tinguin els atributs $t$ i $v$, com una sèrie temporal
amb atributs $t_1$, $v_1$, $t_2$ i $v_2$. Així doncs, per a sèries
temporals el producte resulta en una sèrie temporals amb dos atributs
de temps. Per aquest fet l'anomenem sèrie temporal doble
(v.\ \autoref{def:sgst:st-doble}).
\begin{definition}[producte]
  Sigui $S_1=\{m_0^1, \dotsc, m_{k_1}^1\}$ i $S_2=\{m_0^2, \dotsc,
  m_{k_2}^2\}$ dues sèries temporals en forma canònica, el producte de
  les dues sèries temporals $S_1 \times S_2$ és una sèrie temporal
  doble $S=\{m_0, \dotsc, m_k\}$ que conté la unió de totes les
  parelles de mesures de $S_1$ i $S_2$: $S_1 \times S_2 = \{
  (t_1,v_1,t_2,v_2) | (t_1,v_1) \in S_1 \wedge (t_2,v_2) \in S_2 \}$
\end{definition}

Propietats del producte:
\begin{itemize}
\item El cardinal resultant és $|S|=k_1k_2$
\item El grau resultant és $4$
\end{itemize}



La junció (\emph{join}) de dos conjunts és un conjunt que conté les
parelles d'elements d'ambdós conjunts que tenen el mateix valor per
als atributs comuns.  La junció de dos conjunts amb estructura de
relació, $A \join B$, es defineix com una selecció sobre el
producte \parencite{date:introduction}.


Per a les sèries temporals, definim la junció com l'ajuntament de les
parelles que tenen el mateix atribut de temps en ambdues sèries
temporals . El resultat de la junció és una sèrie temporal
multivaluada.
\begin{definition}[junció]\label{def:sgst:join}
  Sigui $S_1=\{m_0^1, \dotsc, m_{k_1}^1\}$ i $S_2=\{m_0^2, \dotsc,
  m_{k_2}^2\}$ dues sèries temporals en forma canònica, la junció de
  les dues sèries temporals $S_1 \join S_2$ és una sèrie temporal
  multivaluada $S=\{m_0, \dotsc, m_k\}$ que selecciona del producte de
  $S_1$ amb $S_2$ les mesures dobles amb temps iguals: $S_1 \join S_2
  = \{ (t,v_1,v_2) | (t_1,v_1,t_2,v_2) \in S_1\times S_2 \wedge
  t=t_1=t_2 \}$.
\end{definition}


Propietats de la junció:
\begin{itemize}
\item El cardinal resultant és $|S|\leq\min(k_1,k_2)$
\item És commutativa; tenint en compte que els atributs tenen nom i
  per tant l'ordre no importa.
\end{itemize}







\subsubsection{Computacionals: mapa, agregat i plec}

Per a poder operar amb els conjunts, a més de l'àlgebra definida fins
ara, es necessiten operadors amb funcionalitats computacionals; és a
dir, operadors que calculin amb els valors continguts en els conjunts. 

En els SGBDR els operadors computacionals bàsics són \emph{extend},
\emph{aggregate} i \emph{summarize} \parencite{date:introduction}.
Per a les sèries temporals definim operacions equivalents a les dues
primeres de la manera amb que habitualment s'utilitzen per als
conjunts. La tercera, el \emph{summarize}, és una operació creada a
partir de les altres dues que s'utilitza per a sintetitzar per grups i
per tant en les sèries temporals no té sentit perquè l'atribut temps
d'una sèrie temporal no pot tenir instants repetits i per tant no se'n
poden fer grups d'instants compartits.
% No obstant, es pot aplicar el \emph{summarize} per a l'atribut de
% valors: summarize S per S {v} add ...  però això ja no mapa a una
% sèrie temporal.

% De l'operador \emph{aggregate} dels SGBDR definit per
% \textcite{date:introduction} cal tenir en compte que en defineix
% dues vessants. Per una banda, defineix els \emph{aggregate operator
% invocation} que retornen valors escalars. Per altra banda, defineix
% els \emph{aggregate operator invocation} que serveixen per a
% treballar amb el \emph{summarize}.

Així doncs, a continuació es defineix l'operador mapa (\emph{map}) com
a equivalent a l'\emph{extend}, l'operador agregat (\emph{aggregate})
com a equivalent a l'\emph{aggregate} i l'operador plec (\emph{fold})
com una forma més general de calcular recursivament amb les mesures
que l'\emph{aggregate}.




L'operació de mapatge aplica una funció a cada element del conjunt.
\begin{definition}[mapa]
  Sigui $S=\{m_0, \dotsc, m_k\}$ una sèrie temporal i $f$ una funció
  sobre una mesura a on $f:m\mapsto m'$, el mapa de $f$ a $S$ és una
  sèrie temporal $S'=\{m_0', \dotsc, m_k'\}$ amb la funció aplicada a
  cada mesura: $\map(S,f) = \{\forall m\in S : f(m) \}$.
\end{definition}


L'operació d'agregació agrupa en una mesura els elements del conjunt
segons un criteri, per exemple estadístics.
\begin{definition}[agregat]
  Sigui $S=\{m_0, \dotsc, m_k\}$ una sèrie temporal, $m_i$ una mesura
  i $f$ una funció de dues mesures a on $f: m_a \times m_b \mapsto
  m_r$, l'agregat de $S$ segons $f$ amb valor inicial $m_i$ és una
  mesura $m' = (t',v')$ amb l'agrupament de les mesures seguint el
  criteri de la funció: $\agg(S,m_i,f) = f(\dots(
  f(f(f(m_i,m_0),m_1),\allowbreak m_2 )\dots),\allowbreak m_k)$.
\end{definition}

% Més compactament descrit amb
% \begin{align*}
%   \text{fold}: & S=\{m_0,\dotsc,m_k\} \times m_i \times f \longrightarrow m'= \\
%   & \begin{cases}
%     m_i & \text{si} |S|=0, \\
%     \text{fold}(S_1,f(m_i,m_1),f) & \text{altrament}
%   \end{cases}\\
%   \text{ a on } & m_1 \in S, S_1 = S - \{m_1\}
% \end{align*}


L'operació de plegament combina recursivament els elements del conjunt
segons un criteri.
\begin{definition}[plec]
  Sigui $S=\{m_0, \dotsc, m_k\}$ i $S_i=\{m_{i0}, \dotsc, m_{ik}\}$
  dues sèries temporals i $f$ una funció d'una mesura amb una sèrie
  temporal a on $f: S_a \times m_b \mapsto S_r$, el plec de $S$ segons
  $f$ amb valor inicial $S_i$ és una sèrie temporal $S'= \{m_0',
  \dotsc, m_k'\}$ amb l'agrupament de les mesures seguint el criteri
  de la funció: $\fold(S,S_i,f) = f(\dots(
  f(f(f(S_i,m_0),m_1),\allowbreak m_2 )\dots),\allowbreak m_k)$.
\end{definition}


Les operacions d'agregació i plegament tal com s'han definit es
realitzen en ordre aleatori de mesures. Segons el criteri que
s'utilitzi, l'ordre és important i per tant cal una operació que
computi tenint-lo en compte. A tal efecte, a continuació s'amplia la
funció de plegament. Per a la funció d'agregació es pot aplicar el
mateix concepte d'ordre.
\begin{definition}[plec amb ordre]
  Sigui $S=\{m_0, \dotsc, m_k\}$ i $S_i=\{m_{i0}, \dotsc, m_{ik}\}$
  dues sèries temporals, $f$ una funció d'una mesura amb una sèrie
  temporal a on $f: S_a \times m_b \mapsto S_r$ i $o$ una funció que
  treu una mesura d'una sèrie temporal a on $o: S_c \mapsto m_c$, el
  plec de $S$ segons $f$ amb valor inicial $S_i$ i ordre $o$ és una
  sèrie temporal $S'= \{m_0', \dotsc, m_k'\}$ amb l'agrupament de les
  mesures seguint el criteri i l'ordre de les funcions:
  $\fold(S,S_i,f,o) =
  \begin{cases}
    S_i & \text{si } |S|=0, \\
    \fold(S_o,f(S_i,m_o),f,o) & \text{altrament}
  \end{cases}$ a on $m_o = o(S)$ i $S_o = S - \{m_o\}$.
\end{definition}

El plec amb ordre és necessari quan la funció $f$ no és associativa ni
commutativa perquè llavors l'ordre dels càlculs importa. Es pot
observar que el plec sense ordre és un plec amb ordre aleatori:
$\fold(S,S_i,f)\equiv \fold(S,S_i,f,o)$ a on $o=\text{aleatori}(S)$.
De manera semblant $\agg(S,m_i,f)\equiv \agg(S,m_i,f,o)$ a on
$o=\text{aleatori}(S)$.



Propietats d'operacions de plegament:
\begin{itemize}
\item El plec d'una sèrie temporal buida és la sèrie inicial; $\fold:
  \{\} \times f \times S_i \mapsto S_i$.

\item El plec per una funció que sempre retorni la sèrie inicial és la
  sèrie inicial; $\fold: S \times S_i \times f \mapsto S_i$ a on
  $f: S_i \times m \mapsto S_i$.

\item El plec per una funció que només retorni la mesura original és
  una sèrie amb una sola mesura; $\fold: S \times S_i \times f \mapsto
  S'$ a on $f:S_i\times m \mapsto \{m\}$ i $|S'|=1$.


\item La funció d'unió en el plegament permet fer la identitat, $S
  \equiv \fold(S,\{\},(S_i,m_i) \mapsto S_i \cup \{m_i\}$.


\item Els mapes es poden implementar com a plecs; $\map(S,f) \equiv
  \fold(S,\{\},f')$ a on $f': S_i \times m_a \mapsto \{f(m_a)\}
  \cup S_i$.

\item Els agregats es poden implementar com a plecs; $\agg(S,m_i,f) \
  equiv \fold(S,\{m_i\},f')$ a on $f': \{m_i\} \times m \mapsto
  \{f(m_i,m)\}$.

\end{itemize}


\paragraph{Exemples} Definicions de funcions d'exemple a partir de les
operacions computacionals.

Exemple d'operacions de mapatge:
\begin{itemize}
\item $\operatorname{identitat}: S \mapsto S'$ a on $S'=
  \map(S,(t,v)\mapsto(t,v))$
\item $\operatorname{intercanvi}: S \mapsto S'$ a on $S'=
  \map(S,(t,v)\mapsto(v,t))$
\item $\operatorname{translaci\acute{o}}: S \times \delta \mapsto S'$ a on $S'=
  \map(S,(t,v)\mapsto(t+\delta,v))$
\item $\operatorname{t\times v}: S \mapsto S'$ a on $S'=
  \map(S,(t,v)\mapsto(t,t\cdot v))$
\item $\operatorname{tpredecessors_{v1}}: S \mapsto S'$ a on $S'= \map(S,(t,v)
  \mapsto (t,T(\ant_S(m)))$, usant l'operació predecessor de la
  \autoref{def:sgst:ant}
\item $\operatorname{vpredecessors}: S \mapsto S'$ a on $S'= \map(S,(t,v)
  \mapsto (t,V(\ant_S(m)))$, usant l'operació predecessor de la
  \autoref{def:sgst:ant}
\end{itemize}

Exemple d'operacions d'agregació:
\begin{itemize}
\item $\operatorname{cardinal}: S \mapsto m'$ a on $m'=
  \agg(S,(0,0),(t^i,v^i)\times(t,v)\mapsto(t^i,v^i+1)$
\item $\operatorname{sumaV}: S \mapsto m'$ a on $m'=
  \agg(S,(0,0),(t^i,v^i)\times(t,v)\mapsto(t,v+v^i))$
\item $\operatorname{sup}: S \mapsto m'$ a on $m'=
  \agg(S,(-\infty,\infty),(t^i,v^i)\times(t,v)\mapsto [(t^i,v^i)
  \text{ if } t < t^i \text{ else } (t,v) ])$, implementació de
  l'operació suprem de la \autoref{def:sgst:sup} a partir de
  l'agregació
\item $\operatorname{ant}: S \times m \mapsto m'$ a on $m'=
  \agg(S,(-\infty,\infty),(t^i,v^i)\times(t,v)\mapsto [(t,v)
  \text{ if } t^i < t < T(m) \text{ else } (t^i,v^i) ])$,
  implementació de l'operació predecessor de la \autoref{def:sgst:ant} a
  partir de l'agregació 
\end{itemize}


Exemple d'operació de plegament:
\begin{itemize}
\item $\operatorname{tpredecessors}_{v2}: S \mapsto S'$ a on $S'=
  \fold(S,S^b,S_i\times (t,v) \mapsto f)$ a on $S^b =
  \map(S,(t^b,v^b)\mapsto(t^b,-\infty))$, $f= \{(t,tp)\} \cup S_i$ i
  $t_p=T(\sup(S^i \where t^i < t))$, sense usar l'operació predecessor
  a diferència de l'exemple $\operatorname{tpredecessors_{v1}}$
\end{itemize}




\subsubsection{Computacionals per a dues sèries temporals}

Una operació en els conjunts és la d'aplicar un operador binari a
totes les parelles possibles dels elements de dos conjunts. Per
exemple la suma aplicada a dos conjunts A i B és un conjunt $A + B =
\{ e_a+e_b : (e_a,e_b) \in A\times B \}$.

Per a les sèries temporals també calen operacions computacionals amb
els valors de dues sèries temporals. En el cas d'operar amb dues
sèries temporals primer cal ajuntar les dues sèries temporals amb les
que es vol operar i després aplicar les operacions computacionals a la
sèrie temporal resultant.


El producte i la junció són els operadors que permeten crear parelles
de mesures de dues sèries temporals. Per a operar amb els valors de
dues sèries temporals la junció és més adequada ja que permet ajuntar
el valors que tenen temps comuns. Així doncs, per a aplicar un
operador binari $\operatorname{op}$ que calculi amb els valors de
dues sèries temporals:

\[
\operatorname{op}: S_1 \times S_2 \longrightarrow S'
\]
\[
\text{a on } S' = \map(\join(S_1,S_2),(t,v^1,v^2)\mapsto(t^1,v^1
\operatorname{op} v^2))
\]

Cal tenir en compte que la junció de la \autoref{def:sgst:join} només
sap operar amb dues sèries temporals que tinguin el mateix vector de
temps; és a dir regulars entre elles (v.\
def.~\ref{def:st:regular}). En el cas que no tinguin el mateix vector
de temps, es pot aplicar la junció temporal de la
\autoref{def:sgst:joint}.


Exemples de l'aplicació d'operacions computacionals per a dues sèries
temporals
\begin{itemize}
\item $S' = S_1 + S_2$
\item $\operatorname{gradient}: S \mapsto S'$ a on $S'= S -
  \operatorname{vpredecessors}(S)$
\end{itemize}









\subsection{Bàsiques de seqüències}

Atesa la relació d'ordre induïda pel temps en una sèrie temporal
(def.\ \ref{def:model:mesura-relacio-ordre}), les sèries temporals es
poden tractar com a seqüències.  En aquest apartat definim operadors
per a les sèries temporals recollint els operadors habituals que tenen
les seqüències.

Els operadors que treballen amb seqüències tenen en compte l'atribut
que marca un ordre total en el conjunt. En el cas de les sèries
temporals aquest atribut és el temps.



\subsubsection{Interval}

L'interval sobre una seqüència és la subseqüència compresa entre dos
elements.  Per a les sèries temporals és possible definir el concepte
d'interval sobre la seqüència com la subsèrie entre dos instants de
temps, semblant a com es fa a \cite{last:keogh,last:hetland}.

\begin{definition}[Interval]
  \label{def:model:st-interval}
  Sigui $S=\{m_0, \ldots, m_k\}$ una sèrie temporal. Definirem el subconjunt
  $S(r,t) \subseteq S$ com la sèrie temporal $S(r,t)=\{m\in S
  | r<T(m)<t\}$, a on $r$ i $t$ són dos instants de temps.

  Tal com es fa en les seqüències, es defineix una notació de
  parèntesis i claudàtors per indicar si l'interval és obert, tancat o
  semiobert:

  $S[r,t)=\{m\in S  | r\leq T(m)< t\}$

  $S(r,t]=\{m\in S  | r<T(m)\leq t\}$

  $S[r,t]=\{m\in S  | r\leq T(m)\leq t\}$
\end{definition}


Propietats:
\begin{itemize}
\item La subsèrie $S[-\infty,t)\subseteq S$ és equivalent a la sèrie
  temporal $S[-\infty,t) \equiv S[T(\inf(S)),t)$. De la mateixa manera
  $S(r,+\infty] \equiv S(r,T(\sup(S))]$.

\item L'interval degenerat $S[t,t]\subseteq S$ és equivalent a la
  sèrie temporal $S[t,t] \equiv \{m\in S | T(m)=t \}$. El intervals
  $S(t,t]\subseteq S$ i $S[t,t)\subseteq S$ són equivalents a la sèrie
  temporal buida $S(t,t] \equiv S[t,t) \equiv \emptyset$ ja que per
  ser els temps d'ordre total $\nexists T(m): t < T(m) \leq t$ o
  $\nexists T(m): t \leq T(m) < t$, respectivament. 

\item La subsèrie $S[-\infty,+\infty] \subseteq S$ és equivalent a la
  sèrie temporal original $S[-\infty,+\infty] = S$. La subsèrie
  $S(-\infty,+\infty) \subseteq S$ només és equivalent a la sèrie
  temporal original quan aquesta no conté mesures indefinides
  $S(-\infty,+\infty) \equiv S: (-\infty,v_a)\notin S \wedge
  (+\infty,v_b)\notin S$.
\end{itemize}




\subsubsection{Successió}

També atenent a la relació d'ordre induïda pel temps en una sèrie temporal, es
defineix el concepte de mesura següent i mesura anterior en una
seqüència.


\begin{definition}[Successor i
  predecessor]\label{def:sgst:seg}\label{def:sgst:ant}
  Sigui $S=\{m_0, \ldots, m_k\}$ una sèrie temporal i $l\in S$ i $n$ dues
  mesures. Direm que $l$ és el successor de $n$ en $S$ i ho notarem
  com $l=\seg\limits_S(n)$ si i només si $l=\inf(S(T(n),+\infty])$.
  Direm que $l$ és el predecessor de $n$ en $S$ i ho notarem com
  $l=\ant\limits_S(n)$ si i només si $l=\sup(S[-\infty,T(n)))$.

Quan no hi hagi dubte de la sèrie temporal que marca l'ordre, per
exemple quan $n\in S$, podrem escriure $\seg(n)$ i $\ant(n)$.
\end{definition}

S'observa que s'obtenen mesures indefinides en els casos que la
mesura següent o anterior es calcula respectivament per la mesura
suprema o ínfima de la sèrie temporal: $\seg\limits_S(\sup
S)=(+\infty,\infty)$ i $\ant\limits_S(\inf S)=(-\infty,\infty)$.

De la definició anterior es dedueix que donada una sèrie temporal $S$
que no conté mesures indefinides i donada la mesura indefinida
$o=(+\infty,\infty)$, el predecessor de $o$ sempre és el suprem de la
sèrie temporal $\ant\limits_S( (+\infty,\infty) ) = \sup(S): \forall
m\in S: T(m)\in\mathbb{R}$.  % S\equiv S(-\infty,+\infty)
\emph{Demostració: Sigui $S$ una sèrie temporal i $o=(+\infty,\infty)$
  una mesura indefinida, el predecessor de $o$ en $S$ és una mesura
  $l=\ant\limits_S(o)$ que compleix
  $l=\sup(S[-\infty,T(o)))$. Substituint, s'obté que
  $l=\sup(S[-\infty,+\infty))=\sup(S-m):m\in S:T(m)=+\infty \notin
  \mathbb{R}$, i per tant com que $S$ no té mesures indefinides es
  demostra que $l=\sup(S)$.  } De manera semblant es pot demostrar que
$\seg\limits_S( (-\infty,\infty) ) = \inf(S): \forall m\in S:
T(m)\in\mathbb{R}$.


\subsubsection{Concatenació}

La concatenació és una operació que uneix dues seqüències amb els
elements de la primera seqüència seguits pels de la segona. Així
doncs, la concatenació de les seqüències té un sentit semblant al que
la unió té en els conjunts. 

Per a les sèries temporals, per tal que l'operació de concatenació
uneixi amb ordre els operands, cal tenir en compte l'interval que
ocupa cada sèrie temporal segons el seu atribut de temps.  És a dir,
la concatenació de dues sèries temporals consisteix a unir la part de
la segona sèrie temporal que no està inclosa en el rang temporal de la
primera.

Per a poder concatenar dues sèries temporals cal que ambdues tinguin
la mateixa estructura, de la mateixa manera que ja s'ha vist amb
l'operació d'unió.


\begin{definition}[concatenació]
  Sigui $S_1=\{m_0^1, \dotsc, m_{k_1}^1\}$ i $S_2=\{m_0^2, \dotsc,
  m_{k_2}^2\}$ dues sèries temporals, la concatenació de les dues
  sèries temporals $S_1 || S_2$ és una sèrie temporal $S=\{m_0,
  \dotsc, m_k\}$ que conté totes les mesures de $S_1$ i les mesures de
  $S_2$ que no intersequen en l'interval de $S_1$; $S_1 || S_2 = S_1
  \cup ( S_2 - S_2[t_1,t_2] )$ a on $t_1=T(\inf S_1)$ i $t_2=T(\sup
  S_1)$.
\end{definition}

Propietats
\begin{itemize}
\item La concatenació no és commutativa
\end{itemize}







\subsection{Funció temporal}

Atenent a que una sèrie temporal és la representació d'un funció
contínua (v.\ cap.~\todo{Ref!Falta fer el capítol de representació})
cal definir operacions per a tractar convenientment aquesta
naturalesa.

En aquest apartat definim aquestes operacions com una redefinició de
les bàsiques anteriors per a aplicar-les tenint en compte la sèrie
temporal com una funció contínua.



\subsubsection{Interval temporal}

Sigui $S$ una sèrie temporal i $i=[t_0,t_f]$ un interval de temps, per
una banda s'ha definit l'interval sobre la seqüència d'una sèrie
temporal $S(t_0,t_f]$ (def.~\ref{def:model:st-interval}) i per altra
banda s'ha definit la representació contínua $r$ d'una sèrie temporal
$S(t)^r$ \todo{referenciar la definició de repr}.  Per seleccionar un
interval temporal cal tenir en compte tant l'interval sobre la
seqüència com la representació contínua, Aquest interval temporal
s'anota com interval temporal de $S$ en $i$ amb representació $r$ o bé
$S[t_o,t_f]^r$.



\begin{definition}[Interval temporal]
  Sigui $S=\{m_0, \ldots, m_k\}$ una sèrie temporal, $i=[t_0,t_f]$ un
  interval de temps i $r$ una funció de representació, l'interval
  temporal de $S$ en $i$ amb representació $r$, $S[t_o,t_f]^r$, és una
  sèrie temporal $S'=\{m_0, \ldots, m_{k'}\}$ amb les mesures que són
  dins del rang temporal $i$ segons marca la funció de representació:
  $S[t_o,t_f]^r= \forall t \in [t_0,t_f] : S' = S(t)^r $
\end{definition}


A continuació s'exemplifica utilitzant la representació \emph{zohe}
\todo{ref}.
\begin{definition}[Interval temporal \emph{zohe}]
  Sigui $S$ una sèrie temporal, $i=[t_0,t_f]$ un interval de temps i
  \emph{zohe} la representació $S(t)$ amb \emph{zero-order-hold} cap
  enrere, es defineix la subsèrie $S[t_0,t_f]^{\text{zohe}}$ com la
  sèrie temporal $S[t_0,t_f]^{\text{zohe}} = S(t_0,t_f] \cup \{m\}$ a
  on $m=(t_f,v)$ i $v= V(\inf( S[t_f,+\infty] ))$.
\end{definition}
  %Atenció S(t_0,t_f] \cup \{m\} no és equivalent a  (S \cup \{m\})(t_0,t_f] ni sabent que m=(t_f,v); comprovar-ho pel cas t_0=t_f



Propietats de l'interval temporal:

\begin{itemize}
\item Sigui $t_a$ un instant de temps, la selecció de
  $S$ en $[t_a,t_a]^r$ és equivalent a la representació contínua
  $S(t_a)^r$.
\end{itemize}




\subsubsection{Selecció temporal}


La selecció  temporal d'una sèrie temporal permet canviar, en el
context d'una representació, la resolució a una de marcada per un
conjunt d'instants de temps. 

Sigui $S$ una sèrie temporal, $i= \{t_0,t_1,\dotsc,t_n\}$ un conjunt
d'instants de temps i la representació contínua $r$ de la sèrie
temporal $S(t)$, la selecció de resolució s'anota com resolució de $S$
en $i$ amb representació $r$ o bé $S[i]^r$.


\begin{definition}[Selecció temporal]
  Sigui $S=\{m_0, \ldots, m_k\}$ una sèrie temporal,
  $i=\{t_0,t_1,\dotsc,t_n\}$ un conjunt d'instants de temps i $r$ una
  funció de representació, la selecció temporal de $S$ en $i$ amb
  representació $r$, $S[i]^r$, és una sèrie temporal $S'=\{m_0, \ldots, m_n\}$
  amb les mesures amb els temps d'$i$ segons marca la funció de
  representació: $S[i]^r= S[t_0,t_0]^r \cup S[t_1,t_1]^r \cup \dotsb
  \cup S[t_n,t_n]^r$.
\end{definition}

Propietats de la selecció temporal:
\begin{itemize}

\item El cardinal de la sèrie temporal resultant és el mateix que el
  del conjunt d'instants de temps $|S[i]^r| = |i|$

\item La selecció temporal d'una sèrie temporal en un conjunt de temps
  equi-espaiat $i = \{\tau+n\delta | n\in\mathbb{N}, n\leq s \}$ és una
  sèrie temporal regular $S[i]^r \equiv \{ (\tau, v_0),
  (\tau+\delta,v_1), \dotsc , (\tau+s\delta,v_1)\}$
\end{itemize}




\subsubsection{Concatenació temporal}

La concatenació temporal és l'operació de concatenació que té en
compte la representació de les sèries temporals.  És a dir, la
concatenació temporal de dues sèries temporals uneix la part de la
segona sèrie temporal que no està inclosa en l'interval temporal de la
primera.


\begin{definition}[concatenació temporal]
  Sigui $S_1=\{m_0^1, \dotsc, m_{k_1}^1\}$ i $S_2=\{m_0^2, \dotsc,
  m_{k_2}^2\}$ dues sèries temporals i $r$ una funció de
  representació, la concatenació temporal de les dues sèries temporals
  amb representació $r$, $S_1 ||^r S_2$, és una sèrie temporal $S=\{m_0,
  \dotsc, m_k\}$ que conté totes les mesures de $S_1$ i les mesures de
  $S_2$ que no intersequen en l'interval temporal de $S_1$; $S_1 ||^r
  S_2 = S_1 \cup ( S_2 - S_2[t_1,t_2]^r )$ a on $t_1=T(\inf S_1)$ i
  $t_2=T(\sup S_1)$.
\end{definition}

Propietats de la concatenació temporal:
\begin{itemize}
\item No commutativa
\end{itemize}




\subsubsection{Junció temporal}

La junció temporal de dues sèries temporals és la junció que té en
compte la representació de les sèries temporals. És a dir, la junció
temporal de dues sèries temporals ajunta parelles de mesures
seleccionant el mateix atribut de temps en ambdues sèries temporals.


\begin{definition}[junció temporal]
  Sigui $S_1=\{m_0^1, \dotsc, m_{k_1}^1\}$ i $S_2=\{m_0^2, \dotsc,
  m_{k_2}^2\}$ dues sèries temporals en forma canònica i $r$ una
  funció de representació, la junció temporal de les dues sèries
  temporals amb representació $r$, $S_1 \join^r S_2$, és una sèrie
  temporal multivaluada $S=\{m_0, \dotsc, m_k\}$ que ajunta les
  mesures seleccionant els mateixos temps a cada sèrie temporal segons
  la funció de representació; $S_1 \join^r S_2 = \{m=(t',v_1,v_2) |
  (t',v_1) \in S_1[t']^r \wedge (t',v_2) \in S_2[t']^r \}$ a on $t'\in
  S_1\{t\} \cup S_2\{t\}$.
\end{definition}


Propietats de la junció temporal:
\begin{itemize}
\item El cardinal resultant és $|S'| \leq k_1 + k_2$
\item És commutativa; tenint en compte que els atributs tenen nom i
  per tant l'ordre no importa.
\end{itemize}



\todo{}
També es defineix l'operació de semijunció temporal que és una junció
no commutativa a on la primera sèrie temporal marca el vector de temps
de fusió,

\todo{}
Semifusió de dues sèries temporals $S_1 \text{ semifusió } S_2$, a on la primera sèrie temporal marca el vector de temps de fusió, 
\[
S_1 \text{ semifusió } S_2 = S_1 \text{ fusió } S_2[S_1\{t\}]^r
\]











%%% Local Variables:
%%% TeX-master: "main"
%%% End:







% LocalWords:  SGST


\chapter{Model SGSTM}


En aquest capítol es defineixen els operadors que permeten modelar el comportament i la manipulació de les dades.



\section{Model estructural de dades}

Una MTSDB és una relació de buffers amb discs. 


\begin{figure}[tp]
\centering
\input{imatges/model/mtsms-arquitectura_interna.tex}
\caption{Arquitectura del model SGSTM}
\label{fig:model:bdstm}
\end{figure}


\subsection{Buffer}\label{sec:model:buffer}\todo{falta parlar de regularitat de ST}\todo{falta parlar de representació de ST}

Un buffer és un contenidor d'una sèrie temporal, regular o no regular, que mitjançant una funció permet regularitzar aquesta sèrie temporal amb un període de mostreig constant. A l'acció de regularitzar un interval d'una sèrie temporal l'anomenarem consolidació, al període de mostreig contant l'anomenarem pas de consolidació i a la funció de regularització l'anomenarem agregador d'atributs.

\begin{definition}[Buffer]
  Definim \emph{buffer} com el tuple $(S,\tau,\delta,f)$, en el que
  $S$ és una sèrie temporal, $\tau$ és el darrer instant de temps de
  consolidació, $\delta$ és la durada del pas de consolidació i $f$ és
  un agregador d'atributs.
\end{definition}

La consolidació d'una sèrie temporal s'inicia en un instant de temps concret i té lloc a cada pas de consolidació. Amb la finalitat d'establir els intervals de consolidació de la sèrie temporal, es defineix un buffer inicial.

\begin{definition}\label{def:model:buffer_buit}
  Definim buffer inicial o buffer buit com el buffer $B_{\emptyset} =
  (\emptyset,t_0, \delta_0, f)$, el qual
  conté una sèrie temporal buida, l'instant de temps inicial de
  consolidació, una durada que indica el pas de consolidació i un
  agregador d'atributs.
\end{definition}

A partir del buffer buit es poden conèixer tots els instants de temps de consolidació del buffer, els quals seran $t_0+k\delta, k\in\mathbb{N}$. 



\subsection{Disc}\label{sec:model:disc}

Un disc és un contenidor d'una sèrie temporal regular amb un nombre acotat de mesures. En arribar al nombre màxim de mesures permeses, cada cop que s'afegeix una mesura nova s'elimina la mesura mínima de la sèrie temporal.
Així doncs, un disc és semblant a una cua \emph{First In First Out} (FIFO), a on el primer d'arribar és el primer de sortir.  

\begin{definition}[Disc]
  Definim \emph{disc} com el tuple $(S,k)$, en el que $S$
  és una sèrie temporal i $k\in\mathbb{N}$ és el cardinal màxim de $S$.
\end{definition}

A l'inici, un disc no conté mesures però cal que estigui caracteritzat pel cardinal màxim. Amb aquesta finalitat es defineix un disc inicial.

\begin{definition}\label{def:model:disc_buit}
  Definim disc inicial o disc buit com el disc $D_{\emptyset} =
  (\emptyset,k)$, el qual conté una sèrie temporal buida i el cardinal
  màxim que podrà prendre $S$.
\end{definition}




\subsection{Disc resolució}\label{sec:model:disc_multiresolucio}

Un disc resolució és un disc amb buffer. En el buffer hi ha la part d'una sèrie temporal a regularitzar i en el disc hi ha l'altra part ja regularitzada, amb un nombre acotat de mesures. 

\begin{definition}[Disc resolució]
  Definim \emph{disc resolució} com el tuple $(B,D)$, en el que $B$
  és un buffer i $D$ és un disc.
\end{definition}
 
La definició de buffer buit (def.~\ref{def:model:buffer_buit}) i de disc buit (def.~\ref{def:model:disc_buit}) indueixen a una definició de disc resolució buit. 

\begin{definition}\label{def:model:disc_resolucio_buit}
  Definim disc resolució buit com el disc resolució $R_{\emptyset}
  = (B_{\emptyset},D_{\emptyset})$, el qual conté un buffer buit i un
  disc buit.
\end{definition}




\subsection{Base de dades multiresolució}\label{sec:model:bdstm}

Una base de dades multiresolució és un conjunt de discs resolució que comparteixen l'entrada de mesures, les quals provenen d'una mateixa sèrie temporal. La sèrie temporal queda regularitzada i distribuïda  en els diferents discs resolució amb resolucions diferents, tal com s'ha vist a la \autoref{fig:model:bdstm}


\begin{definition}[Base de dades multiresolució]
  Definim \emph{base de dades multiresolució} com el conjunt de discs resolució
  $M=\{R_0,\dotsc,R_d\}$.
\end{definition}

A partir de la definició de disc resolució buit (def.~\ref{def:model:disc_resolucio_buit}) és defineix la base de dades multiresolució buida. 
 
\begin{definition}\label{def:model:bd_multiresolucio_buit}
  Definim base de dades multiresolució buida com el conjunt de discs
  resolució buits
  $M_{\emptyset}=\{R_{0_{\emptyset}},\dotsc,R_{d_{\emptyset}\}}$.
\end{definition}

Normalment, en una base de dades multiresolució no hi ha dos discs
resolució amb la mateixa informació. És a dir, donats dos discs
resolució $R_a = (B_a, D_a)$ i $R_b = (B_b, D_b)$, 
els seus respectius buffers 
$B_a=(S_a,\tau_a,\delta_a,f_a)$ i
$B_b=(S_b,\tau_b,\delta_b,f_b)$ no tenen el mateix interval de
consolidació i agregador d'atributs: 
$\delta_a \neq \delta_b \wedge f_a \neq f_b$.









\subsection{Exemples}

\paragraph{Exemple 1}


S'observa que per tal de complir amb les propietats de les relacions, totes les sèries temporals dels buffers han de ser del mateix tipus, és a dir tenir la mateixa capçalera. El mateix succeeix amb les sèries temporals dels discs. (Vegeu els exemples de la secció \ref{par:model:exemple-relvalues} s'obre valors relació).

\begin{figure}[tp]
  \centering
  \begin{tabular}{|c|c|c|c|c|c|}
    \multicolumn{2}{c}{$M_1$} \\ \hline
    $S_B$  & $S_D$ & $\tau$ & $\delta$ & $k$ & $f$ \\ \hline
    $S_{B1}$ & $S_{D1}$ & 0 & 5  & 2 & mitjana  \\
    $S_{B2}$ & $S_{D2}$ & 0 & 10 & 4 & mitjana  \\ \hline
  \end{tabular}
  \caption{Taula d'una mtsdb independent}
  \label{fig:model:mtsdb:independent}
\end{figure}



\paragraph{Exemple 2}\todo{Compte! que no existeix el tipus relvar i potser no es pot definir una relació que contingui relvars (apuntadors). Cal pensar amb l'exemple 4 suprimit del model dels SGST}

\begin{figure}[tp]
  \centering
  \begin{tabular}{|c|c|c|c|c|c|}
    \multicolumn{2}{c}{$M_2$} \\ \hline
    $S_B$  & $S_D$ & $\tau$ & $\delta$ & $k$ & $f$ \\ \hline
    $S_{B1}$ & $S_{D1}$ & 0 & 5  & 2 & mitjana  \\
    $S_{D1}$ & $S_{D2}$ & 0 & 10 & 4 & mitjana  \\ \hline
  \end{tabular}
  \caption{Taula d'una mtsdb en cadena}
  \label{fig:model:mtsdb:cadena}
\end{figure}












\section{Model d'operacions}


\subsection{Estructurals}

\subsubsection{Buffer}


Abans de consolidar, però, cal que la sèrie temporal contingui mesures. L'operació \emph{afegeix} permet afegir una mesura a un buffer.

\begin{definition}
  L'operació \emph{afegeix} afegeix una mesura a la sèrie temporal del buffer:
  \[
  \text{afegeix}: \text{Buffer} \times \text{Mesura} \longrightarrow \text{Buffer}
  \]
  \[
   B \times m \longrightarrow B'= B \cup \{m\}
   \]
\end{definition}

Cada cop que s'afegeix una mesura a un buffer es pot comprovar si el buffer ja és consolidable mitjançant un predicat que ens retorna un booleà: cert o fals. 

\begin{definition}
  Un buffer és consolidable quan el temps d'una mesura de la sèrie temporal és més gran que el proper instant de temps de consolidació:
  \[
  \text{consolidable?}: \text{Buffer} \longrightarrow \text{Booleà}
  \]
  Sigui $B=(S,\tau,\delta,f)$ un buffer i $m=\max(S)$ la mesura màxima, $B$ és consolidable si i només si $T(m) \geq \tau+\delta$
\end{definition}



Propietats:

\begin{itemize}
\item Les mesures habitualment s'insereixen ordenades en el temps,
  sinó un cop duta a terme la consolidació les mesures inserides
  desordenades poden no ser tingudes en compte.
\end{itemize}



\subsubsection{Consolidació}

Quan un buffer és consolidable, es pot calcular una mesura de consolidació de la sèrie temporal per cada interval de temps consolidable. De manera simplificada, a cada consolidació només es té en compte l'interval que comença al darrer temps de consolidació del buffer. 

Sigui $B=(S,\tau,\delta,f)$ un buffer consolidable, la mesura de consolidació de $B$ en l'interval de temps $i=[\tau,\tau+\delta]$ és $m'=(v,\tau+\delta)$ on $m'=f(S,i)$ i $f$ és un agregador d'atributs. L'operació \emph{consolida} permet consolidar la sèrie temporal del buffer calculant-ne la mesura de consolidació.

\begin{definition}
  L'operació \emph{consolida} calcula la mesura de consolidació i treu
  les mesures consolidades de la sèrie temporal del buffer, en
  l'interval de consolidació actual:
  \[
  \text{consolida}: \text{Buffer} \longrightarrow \text{Buffer} \times \text{Mesura}
  \]
  \[
  B=(S,\tau,\delta,f) \longrightarrow B' \times m'
  \]
  \[
  B'= (S',\tau+\delta,\delta,f)
  \]
  \[
  S' = S(\tau+\delta,\infty)
  \]
  \[
  m' = f(S,[\tau,\tau+\delta]): f \text{ és un agregador d'atributs}
  \]
\end{definition}
\todo{$S'$ pot ser $S$ en el model, en tot cas fer una nota que en la implementació normalment es reduirà per no ocupar espai}





\subsubsection{Disc}

L'operació \emph{afegeix} permet afegir una mesura a un disc, controlant-ne el cardinal màxim.

\begin{definition}
  L'operació \emph{afegeix} afegeix una mesura a la sèrie temporal del disc:
  \[
  \text{afegeix}: \text{Disc} \times \text{Mesura} \longrightarrow \text{Disc}
  \]
  \[
  D=(S,k) \times m \longrightarrow D'= (S',k)
  \]
  \[
  S' =  
  \begin{cases}
      S\cup\{m\} &\text{si }  |S|<k\\
      (S-\{\min(S)\}) \cup \{m\} 
    \end{cases}  \
  \]
\end{definition}





\subsubsection{Disc resolució}


Per altra banda, les operacions dels buffers i dels discs estan relacionades amb les operacions dels discs Round Robin. 

L'operació \emph{afegeix} permet afegir una mesura a un disc Round Robin.

\begin{definition}
  L'operació \emph{afegeix} afegeix una mesura al buffer del disc Round Robin:
  \[
  \text{afegir}: \text{Disc Round Robin} \times \text{Mesura} \longrightarrow \text{Disc Round Robin}
  \]
  \[
  R=(B,D) \times m \longrightarrow R'= (B',D)
  \]
  \[
  B'= B \text{ afegeix } m
  \]
\end{definition}

Cada cop que s'afegeix una mesura a un disc Round Robin es pot comprovar si ja és consolidable. 

\begin{definition}
  Un disc Round Robin és consolidable quan el seu buffer és consolidable:
  \[
  \text{consolidable?}: \text{Disc Round Robin} \longrightarrow \text{Booleà}
  \]
  Sigui $R=(B,D)$ un disc Round Robin, $R$ és consolidable si i només
  si $B$ és consolidable.
\end{definition}


Quan un disc Round Robin és consolidable, es pot consolidar amb l'operació \emph{consolida}. 

\begin{definition}
  L'operació \emph{consolida} calcula una  mesura de consolidació del buffer, en
  l'interval de consolidació actual, i la desa al disc. 
  \[
  \text{consolida}: \text{Disc Round Robin} \longrightarrow \text{Disc Round Robin}
  \]
  \[
  R=(B,D) \longrightarrow R'= (B',D')
  \]
  \[
  B' \times m'= \text{ consolida } B 
  \]
  \[
  D'= D \text{ afegeix } m'
  \]
\end{definition}





\subsubsection{Base de dades multiresolució}



With reference to the operators, the add and consolidate in a multiresolution database are applied to every resolution disc it contains.


\begin{definition}
  Operator \emph{add} adds a measure to every resolution disc:
  \[
  \text{add}: \text{multiresolution database} \times \text{Measure}
  \longrightarrow \text{multiresolution database}
  \]
  \[
  M=\{R_0,\dotsc,R_d\} \times m \mapsto M' 
  \]
  \[  
  M'= \{ \forall R_i\in M: R_i \text{ add } m \}
  \]
\end{definition}


\begin{definition}
  Operator \emph{consolidate} consolidates the resolution discs that
  are ready to consolidate.
  \[
  \text{consolidate}: \text{multiresolution database} \longrightarrow
  \text{multiresolution database}
  \]
  \[
  M=\{R_0,\dotsc,R_d\} \mapsto M'
  \]
  \[
  M'= \big\{
  \forall R_i\in M: 
  \begin{cases}
    \text{ consolidate } R_i & \text{if } R_i \text{ ready to consolidate} \\
    R_i & \text{else }
  \end{cases}\big\}
  \]
\end{definition}





\subsection{Consultes}


Abstracció d'una BDSTM com a sèrie temporal

És possible treballar amb una BDSTM com si fos una sèrie temporal?

Com a consulta total: $\text{SerieTotal}(M)$
Com a consulta amb informació multiresolució: $\text{DiscSelecció}(M,\delta,f)$


\subsubsection{Selecció de disc}


Consulta la subsèrie de la BDSTM que té una resolució i atribut
determinat. 


\begin{definition}[DiscSelecció]
  \begin{gather*}
    \text{DiscSelecció}: M \times \delta \times f \longrightarrow S' = S_D: \\
    (S_B,S_D,\delta,\tau,k,f) \in M
\end{gather*}
\end{definition}



\subsubsection{Sèrie temporal total}



\begin{definition}[Sèrie temporal total]
  Sigui $M^*$ una base de dades multiresolució a on no hi ha $\delta$ repetits
  \begin{gather*}
    \text{SerieTotal}: M^* \longrightarrow S': \\
    \forall (S_{Bi},S_{Di},\delta_i,\tau_i,k_i,f_i) \in M : \\
    \delta_0 < \delta_1 < \delta_2 < \dots < \delta_d : \\
    S' = S_{D0} || S_{D1} || S_{D2} || \dotsb || S_{Dd}
\end{gather*}
\end{definition}

Prèviament es pot fer una selecció dels discs resolució que
comparteixin un determinat agregador d'atributs. \todo{També hi podria
  haver una operació estructural que sabés fusionar dos discs
  resolució}



L'operació de consulta de la sèrie temporal total també es pot aplicar
tenint en compte la representació.
\begin{definition}[Sèrie total amb representació]
  Sigui $M^*$ una base de dades multiresolució a on no hi ha $\delta$
  repetits i $r$ una representació
  \begin{gather*}
    \text{SerieTotal}: M^* \times r \longrightarrow S': \\
    \forall (S_{Bi},S_{Di},\delta_i,\tau_i,k_i,f_i) \in M : \\
    \delta_0 < \delta_1 < \delta_2 < \dots < \delta_d : \\
    S' = S_{D0} \cup^r S_{D1} \cup^r  S_{D2}  \cup^r \dotsb \cup^r  S_{Dd}
\end{gather*}
\end{definition}



\paragraph{Selecció de resolució}


Per a extreure una resolució determinada de la sèrie temporal
emmagatzemada a la base de dades multiresolució, es consulta la sèrie
temporal total i s'aplica una selecció de resolució
$\text{SerieTotal}(M)[i]^r$ a on $i$ és el conjunt d'instants de
temps.






\subsection{Operacions sobre l'estructura}

* Fusió d'esquemes de BDM
* Canvis d'esquemes de BDM (afegir multivaluat, canvi de delta d'un disc,canvi de la k,...)
* Estudiar Push o pull?




\subsubsection{Estudis en l'esquema}

Quin és el període de la sèrie temporal d'un disc?
  \begin{gather*}
    \text{periodeR}: R \longrightarrow \delta':\\
    \delta'=
    \begin{cases}
      \delta_r &\text{si } S_D \text{ regular o temps real amb } \delta_r\\
      \delta &\text{altrament}
    \end{cases}
  \end{gather*}
  
Quin és el l'interval temporal de la sèrie temporal d'un disc?
  \begin{gather*}
    \text{intervalR}: R \longrightarrow [T_0,T_f] :\\
    T_0 = T(\min(S_D)),     T_f = T(\max(S_D))
  \end{gather*}

Quin és el lapse temporal d'un disc?
  \begin{gather*}
    \text{lapseR}: R \longrightarrow [T_0,T_f] :\\
    T_0 = \tau - k\delta,  T_f = \tau
  \end{gather*}




Quin disc conté més resolució?
  \begin{gather*}
    \text{maxR}: R_1 \times R_2 \longrightarrow R_i' | d_i = \max(d_1,d_2) : \\
    d_1 = periodeR(R_1), d_2 = periodeR(R_1)
  \end{gather*}
  





\subsubsection{Canvis en l'esquema}


Redueix o augmenta la mida d'un disc
  \begin{gather*}
    \text{CanviaK}: R \times k' \longrightarrow R': \\
    R' = (S_B,S'_D,\delta,\tau,k',f) : \\
    k_d = |S_D|:\\
    S'_D = \begin{cases}
      S_D         & \text{si } k' \geq k_d   \\
      treuN(S_D,k_d-k')    & \text{altrament}
    \end{cases}, \\
    treuN: S \times n \mapsto S'=  
    \begin{cases}
      S                & \text{si } n=0   \\
      treuN(S - \{\min(S)\},n-1)  & \text{altrament}
    \end{cases}
\end{gather*}


Redueix o augmenta el pas de consolidació d'un disc (sense canviar la sèrie temporal emmagatzemada; ja s'anirà canviant quan es consolidin noves mesures)
  \begin{gather*}
    \text{Canvia}\delta: R \times \delta' \longrightarrow R': \\
    R' = (S_B,S_D,\delta',\tau,k,f)
  \end{gather*}


Redueix o augmenta alhora el pas de consolidació i la mida d'un disc
  \begin{gather*}
    \text{CanviaK}\delta: R \times k' \times \delta' \longrightarrow R': \\
    R' = (S_B,S_D',\delta',\tau,k',f): \\    
    t = \{ \tau-n\delta' | n\in\mathbb{N},n<k' \} \\
    S_D' = \text{seleccioResolucio}(S_D,t)
  \end{gather*}




Afegeix un multivalor per a emmagatzemar sèries temporals multivaluades
  \begin{gather*}
    \text{afegeixMultivalor}: R \longrightarrow R': \\
    R' = (S'_{B},S'_{D},\delta,\tau,k,f): \\
    S'_{B} = \text{map}(S_B,(t,v)\mapsto(t,v,\infty)), \\
    S'_{D} = \text{map}(S_D,(t,v)\mapsto(t,v,\infty))
  \end{gather*}





\subsubsection{Unió de multiresolució}

Cas típic:
Mesuro una sèrie temporal. Durant un temps emmagatzemo valors a una
base de dades i després els emmagatzemo a una altra base de dades. Al final vull unir les dues bases de dades.


Unió de dos discs resolució que tenen el mateix $\delta$ i $f$ és un
disc resolució que conté la unió de les sèries de cada un.  Sigui
$R_1^*=(S_{B1},S_{D1},\delta,\tau_1,k_1,f)$ i
$R_2^*=(S_{B2},S_{D2},\delta,\tau_2,k_2,f)$
  \begin{gather*}
    \text{unioR}: R_1^* \times R_2^* \longrightarrow R': \\
    R' = (S'_B,S'_D,\delta,\max(\tau_1,\tau_2),k_1+k_2,f), \\
    S_{Di}, S_{Dj} | R_i = \text{maxR}(R_1,R_2), j \neq i:  \\
    S'_B = \text{unio}(S_{B1},S_{B2})\\
    S'_D = \text{unio}^r(S_{Di},S_{Dj})
\end{gather*}

També es pot unir dos discs resolució amb diferent $\delta$ i $f$,
però llavors s'ha de determinar quins són els $\delta'$ i $f'$
resultants.


Com a relacions multiresolució, dues bases de dades multiresolució es
poden unir si no intersecten en les claus $(\delta,f)$.  En cas que
intersectin, podem definir la unió multiresolució com la unió que sap unir els discs resolució repetits.

\begin{gather*}
    \text{UnioM}: M_1 \times M_2 \longrightarrow M': \\
    K_1 = \{(delta_1,f_1) \in M_1\},K_2 = \{(delta_2,f_2) \in M_2\}, \\
    K_a = K_1 \cap K_2, K_u =  (K_1 \cup K_2) - K_f : \\
    M_{u1}'= seleccio(M_1, (delta,f) \in K_u)\\
    M_{u2}'= seleccio(M_2, (delta,f) \in K_u)\\
    M_1 = seleccio(M_1, (delta,f) \in K_f) \\
    M_2 = seleccio(M_2, (delta,f) \in K_f) \\
    M_u = \{\forall R_1\in M_1,R_2\in M_2: unioR(R_1,R_2) |
       (delta_1,f_1) = (delta_2,f_2) \} \\
    M' =  M_{a} \cup  M'_{1}  \cup  M'_{2}     
\end{gather*}






\subsubsection{Fusió de multiresolució}

Tinc una sèrie temporal en una base de dades, i una altra sèrie temporal en una altra base de dades. Vull emmagatzemar-les totes dues en una mateixa base de dades amb una sèrie temporal multivaluada.


Fusió de dos discs resolució que tenen el mateix $\delta$ i $f$.
Sigui $R_1^*=(S_{B1},S_{D1},\delta,\tau_1,k_1,f)$ i
$R_2^*=(S_{B2},S_{D2},\delta,\tau_2,k_2,f)$
  \begin{gather*}
    \text{FusioR}: R_1^* \times R_2^* \longrightarrow R': \\
    k^M=\max(k_1,k_2), R' = (S'_B,S'_D,\delta,\max(\tau_1,\tau_2),k^M,f), \\
    S^M_D = S_{Di} | k_i = k_M,  S^m_D = S_{Di} | k_i \neq k^M   : \\
    S'_B = \text{fusio}^r(S_{B1},S_{B2})\\
    S'_D = \text{semifusio}^r(S^M_{D},S^m_{D})
\end{gather*}


Fusió de dues bases de dades multiresolució

\begin{gather*}
    \text{FusioM}: M_1 \times M_2 \longrightarrow M': \\
    K_1 = \{(delta_1,f_1) \in M_1\},K_2 = \{(delta_2,f_2) \in M_2\}, \\
    K_f = K_1 \cap K_2, K_u =  (K_1 \cup K_2) - K_f : \\
    M_1'=\text{afegeixMultivalor}(seleccio(M_1, (delta,f) \in K_u))\\
    M_2'=\text{afegeixMultivalor*} (seleccio(M_2, (delta,f) \in K_u))\\
    \text{afegeixMultivalor*: com a }v_1\\
    M_{f1} = seleccio(M_1, (delta,f) \in K_f) \\
    M_{f2} = seleccio(M_2, (delta,f) \in K_f) \\
    M_f = \{\forall R_1\in M_1,R_2\in M_2: fusioR(R_1,R_2) |
       (delta_1,f_1) = (delta_2,f_2) \in K_u  \} \\
    M' =  M_{f} \cup  M'_{1}  \cup  M'_{2}     
\end{gather*}\todo{repensar perquè $|S_{D1} \cup S_{D2}| = k$?} 




\subsection{Com treure profit de les operacions dels SGSTM}

Temes que després es poden aprofitar a les implementacions

* No hi ha updates --> les sèries temporals no s'han de canviar

* Per exemple, vull calcular la mitjana de  BDSTM(a,b] si tinc un disc resolució amb $\delta=b-a$ i $f=$mitjana aquest seria l'adequat en comptes de calcular mitjana(SerieTotal(M)(a,b])

%??
% No obstant, la base de dades multiresolució conté informació sobre la
% resolució de les subsèries i per tant aquesta operació és susceptible
% d'implementar-se aprofitant aquesta informació.  A tall d'exemple es
% defineix una operació per extreure de la base de dades multiresolució
% una sèrie temporal regular amb període $T$:


% \begin{definition}[Selecció de resolució regular]
%   \begin{gather*}
%     \text{ResolucióRegular}: M^* \times T \times r \longrightarrow S'\\
%     \forall (S_{Bi},S_{Di},\delta_i,\tau_i,k_i,f_i) \in M : \\
%     d_i = T - \delta_i , \\
%     0 \geq d_0 > d_1 \dots > d_a, 0 < d_{a+1} < \dots < d_d: \\
%     S'' = S_{D0} || S_{D1} || \dotsb || S_{Da}  ||  S_{Da+1} || \dotsb || S_{Dd}, \\
%     S' = S''[i]^r: i = {t|0+nT,n\in\mathbb{N}}
%   \end{gather*}
% \end{definition}

% Nota: les operacions no són equivalents, l'operació $\text{SerieTotal}(M)[i]^r$ és molt més potent que la $\text{ResolucióRegular}(M,T)$.







%%% Local Variables:
%%% TeX-master: "main"
%%% End:
% LocalWords:  SGSTM


\section{Funcions d'agregació d'atributs}
\label{sec:model:interpolador}
\label{sec:model:agregador}
\glsaddsection{not:sgstm:fdef} %%%%secció de model
\glsaddsection{not:sgstm:f} %%%%secció de model


Les funcions d'agregació d'atributs s'utilitzen en la consolidació
dels buffers per tal de compactar certa informació de la sèrie
temporals. Sigui $S$ una sèrie temporal i $t_a$ i $t_b$ dos instants
de temps, una funció d'agregació d'atributs $f$ calcula una mesura que
resumeix la informació de $S$ en un interval de temps $i=[t_a,t_b]$:
\[
f: \text{sèrie temporal} \glsdisp{not:times}{\times}
\text{interval de temps} \longrightarrow \text{mesura}
\]
\[
f: S=\{m_0,\dotsc,m_k\} \times i=[t_a,t_b] \longrightarrow  m'
\]


Generalment, $m'$ resulta d'aplicar dues operacions a $S$: 
\begin{enumerate}
\item una selecció d'una subsèrie $S'$ segons l'interval de temps $i$,
  per exemple $S' = S[t_a,t_b]$
\item i una agregació en aquesta subsèrie $m' =
  \glssymbol{not:sgst:aggregate}(S',m_i,\glssymbol{not:sgst:fagg})$ on
  $\glssymbol{not:sgst:fagg}$ i $m_i$ són els atributs d'aquesta agregació.
\end{enumerate}



Atès que hi ha maneres diferents de resumir la informació d'una sèrie
temporal, cal plantejar diferents funcions d'agregació d'atributs. Per
exemple, es poden calcular estadístics de la sèrie temporal, com el
valor màxim o la mitjana, o aplicar operacions de processament digital
del senyal, com fan \textcite{zhang11}. A més a més, la representació
de les sèries temporals (v.~\autoref{sec:model:repr}) pot afectar els
càlculs que es fan en l'agregació o bé es pot aprofitar l'agregació
per a tractar algunes de les patologies de les sèries temporals
(v.~\autoref{sec:sgst:patologies}).  Així doncs, es poden definir una
enorme varietat de funcions d'agregació d'atributs i no hi ha cap
assumpció global que es pugui fer, cada usuari ha d'interpretar quina
combinació d'agregació i representació s'adiu més amb el fenomen
mesurat. Com a conseqüència, els \gls{SGSTM} han de donar llibertat
als usuaris per a definir funcions d'agregació d'atributs
personalitzades.


Com a mostra de com dissenyar funcions d'agregació d'atributs, a
continuació descrivim algunes interpretacions possibles que se'n poden
fer, tant pel que fa al càlcul de l'instant de temps resultant de la
consolidació com pel que fa al càlcul amb representació de sèries
temporals, i descrivim com utilitzar-les per a tractar i validar dades
desconegudes en les sèries temporals.



\subsection{Interpretació de l'agregació}


L'agregació d'una sèrie temporal en un interval resulta en una mesura
$m'=(t',v')$. Així per a definir les operacions d'agregació cal
interpretar quin ha de ser el temps resultant $t'=T(m')$ i el valor
resultant $v'=V(m')$.


Podem definir patrons generals de funcions d'agregació d'atributs que
indiquin quina informació o estadístic es resumeix de la sèrie
temporal, és a dir patrons generals que indiquin com s'ha de calcular
el valor resultant $V(m')$ independentment del mètode de representació
que es vulgui associar a la sèrie temporal.  Tot i així, el temps
resultant $T(m')$ no queda definit sinó que s'ha interpretar
coherentment per a cada cas particular de representació.


A continuació mostrem alguns exemples de patrons generals per a
calcular el valor resultant $V(m')$ que resumeix atributs d'una sèrie
temporal $S$ en un interval $i=[t_a,t_B]$. Sigui $S^r(t)$ la funció de
representació de la sèrie temporal i $t\in T$ els instants de temps
en el domini de temps:
\begin{itemize}
\item màxim: $S \times i \mapsto m'$ on $V(m') = \max_{\forall t \in
    [t_a,t_b]}(S^r(t))$. Resumeix $S$ amb el màxim dels valors de les
  mesures a l'interval $i$.
\item darrer: $S \times i \mapsto m'$ on $V(m') = S^r(t_b)$. Resumeix
  $S$ amb el valor del darrer instant de temps de l'interval $i$.

\item mitjana: $S \times i \mapsto m'$ on $V(m') = \frac{1}{t_b-t_a}
  \int_{t_a}^{t_b} S^r(t)dt$. Resumeix $S$ amb la \emph{mitjana de la
    funció} a l'interval $i$. \emph{Nota:} La mitjana d'una
  funció \parencite{weisstein:averagefunction}, $\bar f=f(x^*)$,
  utilitza la propietat $\int_a^b f(x)dx = f(x^*)(b-a)$ quan $f$ és
  contínua a $[a,b]$.
  % Explicació:
  % If $f$ is continuous on a closed interval $[a,b]$, then there is at least one number $x^*$ in $[a,b]$ such that
  % $$
  % \int_a^b f(x)dx = f(x^*)(b-a)
  % $$

  % The average value of the function ($\bar f$)  on this interval is then given by  $f(x^*)$.
  % $S(t)$ ha de ser contínua en l'interval $i$.
\end{itemize}




En aquests patrons d'atributs es treballa sobre una funció $S^r(t)$,
que a cada cas serà una funció de representació concreta i el temps
resultant $T(m')$ serà interpretat coherentment.  A més, per a cada
representació concreta també cal interpretar amb matemàtica discreta
el càlcul del valor resultant $V(m')$, atès que aquests patrons estan
definits com a problemes d'anàlisi numèric però a cada cas $S^r(t)$ és
una funció que prové d'un conjunt de mesures i podem expressar els
operadors segons el model de \gls{SGST} descrit amb àlgebra discreta
matemàtica.  A continuació s'exemplifiquen algunes interpretacions
possibles per al càlcul de $T(m')$ i de $V(m')$.





\subsubsection{Temps d'agregació resultant}


L'objectiu de les funcions d'agregació d'atributs és determinar un
instant de temps $T(m')$ i un valor $V(m')$. Aquest càlcul del temps i
del valor es pot realitzar al mateix temps però també pot ser
independent. Així, en principi el temps resultant serà independent i
valdrà $T(m')=t_b$ per estar d'acord amb l'operació de consolidació
del buffer i no causar desfasament de la subsèrie resolució
(v.~\autoref{def:sgstm:desdsamentR}), però en alguns casos aquest
$T(m')$ serà dependent del valor calculat o estarà subjecte a una
interpretació adient com és el cas per les representacions a l'apartat
següent.


Un exemple de funció d'agregació on temps i valor són dependents és
una que retorni la primera mesura que troba, $\operatorname{primera}:
S \times i \mapsto m'$ on $m' = \min(S[t_a,t_b))$ i llavors el temps
resultant pot ser $t_a \leq T(m') < t_b$. En aquest cas la sèrie
temporal consolidada resultant no és regular.


Un exemple de funció d'agregació on temps i valor són independents i
on la subsèrie resolució resultant és regular però amb desfasament, és
una funció que fa la mitjana amb un desfasament de 5 unitats de temps.
La funció d'agregació $\operatorname{mitjanad5}$ s'ha utilitzat
anteriorment a l'\autoref{ex:model:bdm-desfasaments}, ara podem
definir-la contextualitzada en les funcions d'agregació d'atributs,
$\operatorname{mitjanad5}: S \times i \mapsto m'$ on $V(m')=
\glssymbol{not:sgst:mitjanav}(S[t_a-5,t_b-5))$ i $T(m')=t_b-5$.

%De què pot servir la mitjanad5? per calcular mitjanes centrades? estem fent una interpolació sobre la representació centrada en l'interval de la sèrie temporal?


%mitjana mòbil, MM
%moving average, MA




\subsubsection{Agregació amb representació}

La varietat de funcions de representació per les sèries temporals
indueix a una varietat de funcions d'agregació per a un mateix patró
d'atributs. Per exemple, la funció d'agregació per l'atribut de màxim
dóna com a resultat valors diferents si es considera una representació
lineal o una representació a trossos constant. A continuació mostrem
la interpretació dels patrons definits anteriorment per a tres mètodes de
representació: \gls{pd}, \gls{dd} i \gls{zohe}.


\paragraph{Parcial discreta.}
En els casos parcials, $S^r(t)$ no és totalment contínua en el temps,
però es pot resoldre l'agregació del valor resultant assumint que el
domini de temps $T$ es correspon als instants de temps que hi ha a la
sèrie temporal, és a dir $T=\glssymbol{not:sgst:project}_{t}(S)$.  El
temps resultant es pot interpretar segon descrit a l'apartat anterior,
per exemple $T(m')=t_b$, i a més també es pot interpretar l'interval
de temps d'agregació $i=[t_a,t_b]$. Així sigui $S$ la sèrie original,
el resultat es pot calcular sobre una subsèrie amb interval obert
$S'=S(t_a,t_b)$, tancat $S'=S(t_a,t_b]$, semiobert $S'=S(t_a,t_b]$ o
$S'=S[t_a,t_b)$, o altres combinacions com per exemple tenir
desfasaments $S'=S[t_a-d,t_b-d]$ on $d$ és una durada.  Així de forma
general podem definir les funcions d'agregació d'atributs amb
representació \gls{pd}, $f^{\gls{pd}}\in f$, com $f^{\gls{pd}}: S
\times [t_a,t_b] \mapsto m'$ on $m'=(t_b,v')$ i el valor resultant
depèn del l'atribut que es vulgui resumir calculat en l'interval
$S'=S[t_a,t_b]$, a continuació es mostren els patrons d'exemple
interpretats segons aquest criteri.

\begin{definition}[Agregació parcial discreta]
  Sigui $S=\{m_0,\dotsc,m_k\}$ una sèrie temporal, $i=[t_a,t_b]$ un
  interval de temps i $S'=S[t_a,t_b]$ un interval de la sèrie
  temporal, les funcions d'agregació \gls{pd} per als atributs màxim,
  darrer i mitjana són:
  \begin{itemize}

  \item $\operatorname{m\grave{a}xim}^{\gls{pd}}$: $S \times i \mapsto
    m'$ on $V(m') = \max_{\forall m \in S'}(V(m))$ i
    $T(m')=t_b$. Aquest càlcul de $V(m')$ es correspon amb l'operació
    $\glssymbol{not:sgst:maxv}(S')$ dels \gls{SGST}.

\item $\operatorname{darrer}^{\gls{pd}}$: $S \times i \mapsto m'$ on $V(m') =
  V(\max(S'))$ i $T(m')=t_b$.

\item $\operatorname{mitjana}^{\gls{pd}}$: $S \times i \mapsto m'$ on $V(m') =
  \frac{1}{|S'|} \sum\limits_{\forall m\in S'} V(m)$ i $T(m')=t_b$. Aquest càlcul de
  $V(m')$ es correspon amb l'operació $\glssymbol{not:sgst:mitjanav}(S')$
  dels \gls{SGST}, és a dir amb calcular la mitjana aritmètica dels
  valors de les mesures.
\end{itemize}

\end{definition}



\paragraph{Delta de Dirac.} 
Per a les funcions d'agregació delta de Dirac interpretem el temps
d'agregació resultant centrat en l'interval $T(m')=\frac{t_b+t_a}{2}$,
tot i que també es podrien considerar altres interpretacions com per
exemple $T(m')=t_b$. Així de forma general podem definir les funcions
d'agregació d'atributs amb representació \gls{dd}, $f^{\gls{dd}}\in
f$, com $f^{\gls{dd}}: S \times [t_a,t_b] \mapsto m'$ on
$m'=(\frac{t_b+t_a}{2},v')$ i el valor resultant depèn del l'atribut
que es vulgui resumir calculat en l'interval temporal \gls{dd}
$S'=S[t_a,t_b]^{\gls{dd}}$.


\begin{definition}[Agregació delta de Dirac]
  Sigui $S=\{m_0,\dotsc,m_k\}$ una sèrie temporal, $i=[t_a,t_b]$ un
  interval de temps i $S'=S[t_a,t_b]^{\gls{dd}}$ un interval temporal
  de la sèrie temporal, les funcions d'agregació \gls{dd} per als
  atributs màxim, darrer i mitjana són:
\begin{itemize}
\item \glssymboldef{not:sgstm:maxdd}: $S
  \times i \mapsto m'$ on $V(m') = \max\big(0,\max_{\forall m \in
    S'}(V(m))\big)$ i $T(m')=\frac{t_b+t_a}{2}$. 

\item $\operatorname{darrer}^{\gls{dd}}$: $S \times i \mapsto m'$ on $V(m') =
  V(\max(S'))$ i $T(m')=\frac{t_b+t_a}{2}$.

\item \glssymboldef{not:sgstm:mitjanadd}: $S \times i \mapsto m'$ on
  $V(m') = \frac{1}{t_b-t_a}\sum\limits_{\forall m \in S'} V(m)$ i
  $T(m')=\frac{t_b+t_a}{2}$. Nota: la funció delta de Dirac té la
  propietat fonamental $\int \delta(t)dt = 1$. 
\end{itemize}
\end{definition}



\paragraph{Zero-order hold enrere.}
Per a les funcions d'agregació \gls{zohe} interpretem sempre el temps
d'agregació resultant com $T(m')=t_b$, atès que la representació
\gls{zohe} es defineix amb funcions graó contínues per
l'esquerra. Així de forma general podem definir les funcions
d'agregació d'atributs amb representació \gls{zohe},
$f^{\gls{zohe}}\in f$, com $f^{\gls{zohe}}: S \times [t_a,t_b] \mapsto
m'$ on $m'=(t_b,v')$ i el valor resultant depèn de l'atribut que es
vulgui resumir calculat en l'interval temporal \gls{zohe}
$S'=S[t_a,t_b]^{\gls{zohe}}$.
\begin{definition}[Agregació zero-order hold enrere]
  Sigui $S=\{m_0,\dotsc,m_k\}$ una sèrie temporal, $i=[t_a,t_b]$ un
  interval de temps i $S'=S[t_a,t_b]^{\gls{zohe}}$ un interval
  temporal de la sèrie temporal, les funcions d'agregació \gls{zohe}
  per als atributs màxim, darrer i mitjana són:
  \begin{itemize}
  \item \glssymboldef{not:sgstm:maxzohe}: $S \times i \mapsto m'$ on
    $V(m') = \max_{\forall m \in S'}(V(m))$ i $T(m')=t_b$.

  \item $\operatorname{darrer}^{\gls{zohe}}$: $S \times i \mapsto m'$
    on $V(m') = V(\max(S'))$ i $T(m')=t_b$.

  \item \glssymboldef{not:sgstm:meanzohe}: $S \times i \mapsto m'$ on
    $V(m') = \frac{1}{t_b-t_a} \big[ (T(o)-t_a)V(o) +
    \sum\limits_{\forall m \in S''}( T(m)-
    T(\glssymbol{not:sgst:prev}_S (m)) )V(m) \big]$; $o=\min(S')$;
    $S''= S' - \{o\}$; i $T(m')=t_b$.
% \[
%   \begin{split}
%   V(m')  = & \frac{1}{t_b-T_0} 
%   \big[ (T(o)-T_0)V(o) -( T(n)-T_f)V(n) \\
%     & {}+\sum\limits_{\forall m \in S''}( T(m)- T(\prev_S m) )V(m) \big]   
%    \end{split}
%   \]
% Nota: s'aplica la definició $0 \times \infty = 0$ tal com es fa habitualment a la teoria de mesura, \cite{wiki:extendedreal}.
  \end{itemize}
\end{definition}




Un cop definits els tres exemples de famílies d'agregacions, podem
comparar-les en funció de com resumeixen la informació de la sèrie
temporal. Reprenent la consolidació dels buffers
(v.~\autoref{sec:model:buffer}), l'interval de consolidació es
correspon a $t_a=\tau$ i $t_b=\tau+\delta$ i és consolidable quan
existeix una mesura $T(m)\geq\tau+\delta$. A la
\autoref{fig:sgstm:agg} dibuixem les mesures d'una sèrie temporal en
vermell, un interval de consolidació del buffer en línies blaves i la
mesura resultant de consolidació en verd.  Així, sigui
$S=\{\dotsc,m_{a-1},m_{a+1},\dotsc,m_{b-1},m_{b+1}, \ldots\}$ una
sèrie temporal on $ T(m_{a-1}) < t_a < T(m_{a+1}) < \dotsc <
T(m_{b-1}) < t_b < T(m_{b+1})$ i la consolidació del buffer que
calcula la mesura resultant $m'=f(S,[t_a,t_b])$ amb la funció
d'agregació d'atributs $f$.  Assumim $T(m')=t_b$ per simplificar el
dibuix, de manera general el càlcul del valor resultant és una
agregació a partir de les mesures:
\begin{itemize}
\item $\{m_{a+1},\dotsc,m_{b-1}\}$ en el cas de les agregacions \gls{pd}
\item $\{(t_a,0),(\ldots,0),m_{a+1},\dotsc,(\ldots,0),\dotsc,m_{b-1},(\ldots,0),(t_b,0)\}$ en el cas de les
  agregacions \gls{dd}
\item $\{m_{a+1},\dotsc,m_{b-1},m_{b+1}\}$ en el cas de les
  agregacions \gls{zohe}
\end{itemize}






\begin{figure}[tp]
  \centering
 
    \begin{tikzpicture}
        \begin{axis}[
          % width=10cm,
%          scale only axis, height=3cm,
          ymin = 0,
          xmax = 50,
          xmin = 20,
          yticklabels= {},
          xticklabels={,,,$t_a$,,$t_b$},
          ]
          \addplot[ycomb,blue] coordinates {
            (30,10)
            (40,10)
          }; 
          
          \addplot[only marks,mark=*,red] coordinates {
            (25,5)
            (32,2)
            (35,4)
            (38,6)
            (45,8)
          };
          
          \addplot[only marks,mark=*,green] coordinates {
            (40,4)
          };
          
          \node[above] at (axis cs:26,5) {$m_{a-1}$};
          \node[below] at (axis cs:32,2) {$m_{a+1}$};
          \node[below] at (axis cs:35,4) {$\ldots$};
          \node[above] at (axis cs:38,6) {$m_{b-1}$};
          \node[above] at (axis cs:45,8) {$m_{b+1}$};
          \node[right] at (axis cs:40,4) {$m'$};
        \end{axis}
      \end{tikzpicture}

    
  \caption{Agregació d'un interval de la sèrie temporal}
  \label{fig:sgstm:agg}
\end{figure}






En conclusió, per una banda alguns exemples mostrats de patrons tenen
una interpretació semblant per a les representacions particulars, en
certa manera només es diferencien en la interpretació de l'interval on
s'ha de resumir la sèrie temporal. Per exemple la diferència principal
en els atributs de màxim i darrer per a les tres representacions rau
en la $S'$, tot i que en el cas del $\glssymbol{not:sgstm:maxdd}$
l'agregació a més ha de tenir en compte que en la funció de
representació hi ha valors intermitjos que valen zero.

Per altra banda, altres exemples són molt diferents, com és el cas de
l'atribut mitjana. En aquest cas, per a la \gls{pd} i la \gls{dd} és
el càlcul de la suma dels valors tot i que dividit per $|S'|$ en la
primera i per $t_b-t_a$ en la segona, i és una mitjana ponderada per
les durades de temps en la \gls{zohe}.  En general, es pot dissenyar
qualsevol operació d'agregació, com per exemple calcular la mitjana
aritmètica de l'interval \gls{zohe} amb
$\glssymbol{not:sgst:mitjanav}(S[t_a,t_b]^{\gls{zohe}})$, tot i que
llavors cal interpretar quin patró d'atribut li correspon o altrament
aquesta operació d'agregació pot no tenir sentit real.


\textcite{rrdtool} utilitza a RRDtool una funció d'agregació semblant
a la $\glssymboldef{not:sgstm:meanzohe}$ per a resumir la informació
conservant el comptatge total si les sèries temporals mesurades tenen
trets semàntics de comptador i són en forma de velocitat; així aquesta
agregació es pot veure com una consolidació que conserva l'àrea del
senyal original. 





% Notes:

% * Quan una sèrie temporal és regular, l'intepolador mitjana aritmètica i l'interpolador àrea valen el mateix en l'interval $(T_o,n\delta]$.




\subsection{Tractament i validació de dades}


En les patologies de les sèries temporals
(v.~\autoref{sec:sgst:patologies}) s'ha descrit el problema de les
dades desconegudes, les funcions d'agregació d'atributs poden cooperar
en els processos de validació i tractament de dades. Així, les
funcions d'agregació poden marcar o tractar dades desconegudes:
\begin{itemize}
\item Marcar dades com a desconegudes. És a dir determinar quan el
  resultat d'una agregació ha de ser desconegut perquè la sèrie
  temporal avaluada pateix una de les causes descrites: valors fora de
  rang, temps de termini excedit, etc.

\item Tractar dades que són desconegudes, ja sigui perquè d'origen són
  desconegudes o perquè les hem marcat abans com a desconegudes.
  Si una funció d'agregació rep valors que són desconeguts, des d'un
  punt de vista estricte el resultat de l'agregació ha de ser
  desconegut. No obstant això, es poden aplicar operacions que tractin
  aquest valors desconeguts: reconstrucció del senyal, ignorar els
  valors desconeguts, etc.
\end{itemize}

 
A continuació definim el procés que fan les funcions d'agregació per a
ambdós casos. Com a exemple de domini pels valors utilitzem els
nombres reals projectius \glssymbol{not:R*}, en els quals representem
el valor desconegut mitjançant l'element infinit ($\infty$), segons la
\autoref{def:model:mesura_valor_indefinit} de mesura de valor
indefinit. Això no obstant, el domini de valors podria tenir diversos
valors per a marcar diferents casos de dades desconegudes.

\paragraph{Tractament de dades desconegudes.}
Una funció d'agregació d'atributs $f^u \in f$ que tracti dades
desconegudes és aquella que pot calcular un resultat quan la sèrie
temporal original conté valors desconeguts
\[
f^u: S \times i \mapsto m' \text{ on } \exists m \in S: V(m)=\infty
\]

Per exemple, podem redefinir el patró de la funció d'agregació mitjana
en una $\operatorname{mitjana}^{u}$ que sigui capaç de tractar valors
desconeguts conservant l'àrea coneguda, és a dir, l'àrea total
coneguda quedarà escampada en l'interval de consolidació.
\begin{gather*}
  \operatorname{mitjana}^{cu}: S \times i \mapsto m' \text{ on }\\
  V(m') = \frac{1}{t_b-t_a}\int_{t_a}^{t_b} S^u(t)dt \text{ i }
  S^u(t)=
  \begin{cases}
    0 &\text{si }  S^r(t)=\infty\\
    S^r(t) & \text{altrament }
  \end{cases}
\end{gather*}


\paragraph{Marcatge de dades desconegudes.}
Una funció d'agregació d'atributs $f^{mu} \in f$ que marqui
dades desconegudes és aquella que pot retornar una mesura de valor
indefinit com a resultat
\[
f^{mu}: S \times i \mapsto m' \text{ on } V(m')\in \glssymbol{not:R*}
\]


Per exemple, podem definir un patró de funció d'agregació d'atribut
màxim que retorni valor desconegut
si hi ha un a mesura amb el valor més gran que 2; és a dir establim un
límit superior de 2 (L2). 
\begin{gather*}
  \operatorname{m\grave{a}xim}^{L2}: S \times i \mapsto m' \text{ on }\\
  m' = \begin{cases}
    (T(m''),\infty) &\text{si }  \exists m\in S[t_a,t_b]: V(m)>2\\
    m'' & \text{altrament }
  \end{cases} \text{ i } m''= \operatorname{m\grave{a}xim}(S,i)
\end{gather*}






%\todo{}
%hauria d'aparèixer algun exemple on es resolgués inframostreig. Potser també algun exemple on es veiés on els agregadors solucionen el problema de l'ultramostreig.



%Per exemple definim un termini, si les dades estan més espaiades que 2 es marca com a desconeguda
% Sigui $S=\{m_0,\ldots,m_k\}$ una sèrie temporal i $H$ un termini de temps, una mesura $m_i=(v_i,t_i)\in S$ és desconeguda si, donada la mesura anterior $m_{i-1}=(v_{i-1},t_{i-1})$, $t_i - t_{i-1} > H$.    





% Sigui $S=\{m_0,\ldots,m_k\}$ una sèrie temporal, $f$ un interpolador, $i=[T_0,T_f]$ un interval de temps i $\alpha$ un llindar, la mesura de consolidació calculada per l'interpolador $f$ és desconeguda ssi  
% \[
% \frac{t_d }{T_f - T_0} > \alpha :
% \]
% \[
% :t_d = t_{d0} + t_{df} + \sum\limits_{i=1}^{k-1}(t_i-t_{i-1}) : v_k = 'desconegut':
% \]
% \[
% : t_{d0} = \left\{\begin{array}{l} t_0-T_0 \text{ si } v_0 = 'desconegut' \\ 0\end{array}\right. ,
% t_{df} = \left\{\begin{array}{l} T_f-t_{k-1} \text{ si } v_k = 'desconegut' \\ 0\end{array}\right. :
% \]
% \[
% :k=|S|-1,(v_k,t_k)=m_k\in S' :S'= S_{T_0:T_f} \cup \{min(S_{T_f:\infty})\}
% \]



%operacions amb nan de octave i matlab
%http://biosig-consulting.com/matlab/NaN/
% The NaN-toolbox v2.0: A statistics and machine learning toolbox for Octave and Matlab
% for data with and w/o MISSING VALUES encoded as NaN's.











%%% Local Variables:
%%% TeX-master: "main"
%%% End:
% LocalWords: buffer buffers ZOHE








\section{Model}

\acro{MTSMS}
\acro{MTSM}


La definició del model s'estructura en dues parts:

\begin{itemize}
\item Un model pels (SGST)  que defineix mesura i sèrie temporals.
\item Un model pels (SGSTM) que defineix buffer, disc i subsèrie
  resolució, el qual treballa sobre el model de SGST.
\end{itemize}

\todo{sobre tres nivells}
A l'estat de l'art s'ha d'haver explicat els tres nivell de model de dades segons Date i deixar clar aquí que nosaltres definim un model pel segon nivell: nivell de model lògic. Els models lògics modelen les dades, en canvi els models conceptuals modelen la realitat, Fabian Pascal posa d'exemple conceptual el model E/RM.


Objectius:

En el model de SGST s'observen algunes patologies que poden presentar les sèries temporals. El model de SGSTM soluciona algunes d'aquestes patologies:

\begin{itemize}
\item Regularitza les sèries temporals
\item Tracta i validar les sèries temporals: gestiona els casos de dades errònies o desconegudes i marca quan hi ha valors erronis.
\item És una solució de compressió per a quantitats enormes de dades
\end{itemize}


Però el model de SGSTM també es pot fer servir per altres aplicacions:

* Regularitzar en línia (temps real) una sèrie temporal en diferents períodes de mostreig

* Tenir unes vistes (consultes) a punt (ja processades) amb diferents resolucions d'una sèrie temporal

* Comprimir per decimació (downsampling) o bé farcir forats (reconstrucció del senyal)




\section{Time series preliminaries}
\label{sec:model:preliminaries}

% In this section we introduce some background concepts and the
% nomenclature which we will use later.  First we define the main
% objects of a \acro{MTSMS} which are measures and time series.

% A \emph{measure} is a value measured in a time instant. More formally
% it is a tuple $(v,t)$ where $v$ is the value of the measure and $t \in
% \mathbb{R}$ is the time instant of measurement.  The values of a time
% series can be of any type. For simplicity examples are presented with
% integers or real numbers but can also be strings or vectors.  Let $m =
% (v,t)$ be a measure, $v$ is written as $V(m)$ and $t$ is written as
% $T(m)$.

% The time value defines the canonical order between measures.  Let $m =
% (v_m, t_m)$ and $n = (v_n, t_n)$ be two measures, then $m\geq n$ if
% and only if $t_m\geq t_n$.

% A \emph{time series} is sequence of measures of the same phenomena
% that are ordered in time.
% \begin{definition}[Time series]
%   A \emph{time series} $S$ is a a set of measures of the same
%   phenomena $S = \{m_0, \ldots, m_k\}$ without repeated time values
%   $\forall i,j: i\leq k, j\leq k, i\neq j : T(m_i)\neq T(m_j)$. Given
%   a time series $|S|$, we note its size by $|S|=k+1$. Observe that,
%   because measures in $S$ are of the same phenomena, the type of $S$
%   values is homogeneous.
% \end{definition}

% The order defined by measures implies a total order in a time
% series. As a time series is a finite set, if it is not empty it has a
% maximum and a minimum.  Let $S=\{m_0,\ldots,m_k\}$ be a time series
% and $n\in S$ be a measure. The time series' maximum is $n=\max(S)$ if
% and only if $\forall m \in S: n \geq m $.  Similarly, the time series'
% minimum is $n=\min(S)$ if and only if $\forall m \in S: n \leq m$.

% Given the order defined by time, in a time series we define the
% sequence interval following \cite{last:keogh,last:hetland}.  Let
% $S=\{m_0, \ldots, m_k\}$ be a time series. We define the subset
% $S(r,t] \subseteq S$ as the time series $S(r,t]=\{m\in S | r<T(m)\leq
% t\}$, where $r$ and $t$ are two instants in time.  We also define the
% subset $S(r,+\infty)\subseteq S$ as the time series $S(r,+\infty) =
% \{m\in S | r< T(m) \leq T(\max(S))\}$ and the subset
% $S(-\infty,t)\subseteq S$ as the time series $S(-\infty,t) = \{m\in S
% | T(\min(S))\leq T(m) < t\}$.

% The time order in time series also implies the sequence concept of
% next and previous measure.  Let $S=\{m_0, \ldots, m_k\}$ be a time
% series and $l\in S$ and $n$ be two measures. We define the next
% measure of $n$ in $S$ as $l=\nex_S(n)$ where $l =
% \min(S(T(n),+\infty))$. We define the previous measure of $n$ in $S$
% as $l=\prev_S(n)$ where $l = \max(S(-\infty,T(n)))$.

% Let $S$ be a time series, $t$ be a time instant and $\delta$ be a
% time duration, then the time series' measures can be located in the
% time interval $i_0=[t, t+\delta]$ and its multiples $i_j=[t+j\delta,
% t+(j+1)\delta]$ for $j=0,1,2,\ldots$. When time series' measures are
% equally spaced we say it to be regular.
% \begin{definition}[Regular time series]
%   Let $S=\{m_0,$ $ldots,$ $m_k\}$ be a time series and $\delta$ a time
%   duration. $S$ is regular if and only if $\forall m \in
%   S(T(\min(S),+\infty):T(m) - T(\prev_S(m)) = \delta$.
% \end{definition}









\section{Multiresolution model}
\label{sec:MTSMS}

The \acro{MTSMS} are \acro{TSMS} that store time series with a lossy
compression approach, that is some information is selected and spread in
different time resolutions. The \acro{MTSMS} model is based on the
concepts of measures and time series as defined in
Section~\ref{sec:model:preliminaries}.


The multiresolution concept comes from thoroughly analysis of the
RRDtool \cite{rrdtool} \acro{TSMS}. Our objective is to formalise its
essential parts into an abstract model, where what we call
multiresolution plays a main role, and to include more genericity in
order to describe \acro{MTSMS} as fully \acro{TSMS}. Then we will be
able to apply these systems to other applications.



A \acro{MTSMS} stores multiresolution time series where each has a
multiresolution schema as shown in Figure~\ref{fig:model:mtsdb}. A
multiresolution time series is a collection of resolution subseries
which temporarily accumulate measures in a buffer in order to select
some information and finally store it in a disc. The information
selection process changes the time intervals between measures to
compact information by aggregating the time series attributes. 

\begin{figure}[tp]
  \centering
  \input{imatges/mtsms-arquitectura_interna.tex}
  \smallskip
  \caption{Architecture of \acro{MTSMS} model}
  \label{fig:model:mtsdb}
\end{figure}


In this way, the original time series gets stored spread in the discs,
each with a different time resolution and attribute aggregation.
Discs are size bounded so they only contain a fixed amount of
measures. When a disc becomes full it discards a measure. Thus,
multiresolution database is bounded in size and the time series gets
stored in pieces, that is time subseries.

Regarding to operations, \acro{MTSMS} structure needs operators to
change the time intervals between measures and to select
attributes. Mainly, these operators are measure additions and time
series consolidations which some functionality is delegated to operators called 
attribute aggregate functions.
 Most of these operators
are attribute aggregate functions and consolidation actions.
the operations to
create a multiresolution database, to add measures, and to consolidate
time series.
 Attribute aggregate functions are required but not linked
to the model.\todo{}

Following we define the \acro{MTSMS} model by: (i) four basic
structure model elements ---buffer, disc, resolution subserie, and
multiresolution time series--- with its structure operators, (ii) the
operations to change and consult a multiresolution schema, and (iii)
the attribute aggregate functions.



\subsection{Structure}



\subsection{operations}




\section{The proposed data model}



Regarding to operations, \acro{MTSMS} structure needs operators to
change the time intervals between measures. Most of these operators
are attribute aggregate functions and consolidation actions.

In what follows we describe the basic \acro{MTSMS} model centered in:
(i) the four basic data model elements ---buffer, disc, resolution
disc, and multiresolution database---, and (ii) the operations to
create a multiresolution database, to add measures, and to consolidate
time series. Attribute aggregate functions are required but not linked
to the model. They are defined in the
Section~\ref{sec:model:interpolador}.

A \emph{buffer} is a container for a regular or a no-regular time
series. The buffer objective is to regularise the time series using a
predetermined step and an attribute function. We name
\emph{consolidation} to this action.
\begin{definition}[Buffer]
  A \emph{buffer} is defined as the tuple $(S,\tau,\delta,f)$ where
  $S$ is a time series, $\tau$ is the last consolidation time,
  $\delta$ is the duration of the consolidation step and $f$ is an
  attribute aggregate function.

  An empty buffer $B_{\emptyset} = (\emptyset,t_0, \delta, f)$ has an
  empty time series, an initial consolidation time $t_0$ and
  predetermined $\delta$ and $f$. From the $B_{\emptyset}$ all the
  consolidation time instants can be calculated as $t_0+i\delta,
  i\in\mathbb{N}$.
\end{definition}

Operator \emph{addBuffer} adds a measure to its time series:
$\text{addBuffer}: B = (S,\tau,\delta,f) \times m \mapsto
(S',\tau,\delta,f)$ where $S' = S \cup \{m\} $.

A buffer is ready to consolidate when the time of some measure is
bigger than the buffer's next consolidation time.  Let
$B=(S,\tau,\delta,f)$ be a buffer and $m=\max(S)$ the maximum measure,
$B$ is ready to consolidate if and only if $T(m) \geq \tau+\delta$.
The consolidation of $B$ in the time interval $i=[\tau,\tau+\delta]$
results in a measure $m'=(v,\tau+\delta)$ where $m'=f(S,i)$ and $f$ is
an attribute aggregate function $f$. Operator \emph{consolidateBuffer}
consolidates a set of measures and removes the consolidated part of
the time series from the buffer. Usually consolidateBuffer is only
applied to the present consolidation interval and it is defined as
follows: $\text{consolidateBuffer}: B=(S,\tau,\delta,f) \mapsto B'
\times m' $ where $ B'= (S',\tau+\delta,\delta,f)$, $S' = S$ and $m' =
f(S,[\tau,\tau+\delta])$. When historic data is not needed anymore the
consolidated buffer measures can be removed applying $S' =
S(\tau+\delta,\infty)$.

A \emph{disc} is a finite capacity measures container. A time series
stored in a disc has its cardinal bounded. When the cardinal of the
time series is to overcome the limit, some measures need to be
discarded.
\begin{definition}[Disc]
  A \emph{disc} is a tuple $(S,k)$ where $S$ is a time series and
  $k\in\mathbb{N}$ is the maximum allowed cardinal of $S$.  An empty
  disc $D_{\emptyset} = (\emptyset,k)$ has an empty time series and
  the $k$ maximum cardinal allowed.
\end{definition}

The cardinal of the times series is kept under control by the add
operator, $\text{addDisc}:D=(S,k)\times m\mapsto (S',k)$ where 
$$
S' = \begin{cases}
  S\cup\{m\}                 & \text{if } |S|<k  \\
  (S-\{\min(S)\}) \cup \{m\} & \text{otherwise}
\end{cases}  
$$

A \emph{resolution disc} is a disc which stores a regular time
series. It is composed of a buffer, that contains the partial time
series to be regularised, and a disc, that contains the regularised
time series.
\begin{definition}[Resolution disc]
  A \emph{resolution disc} is a tuple $(B,D)$ where $B$ is a buffer
  and $D$ is a disc.  An empty buffer and empty disc imply an empty
  resolution disc $R_{\emptyset} = (B_{\emptyset},D_{\emptyset})$.
\end{definition}
 
The operators of a resolution disc extend the buffer and disc ones:
(i) The addition of a measure to the buffer of the resolution disc,
$\text{addRD}:R=(B,D) \times m \mapsto R'$ where $R'= (B',D)$, and
$B'= \text{addBuffer}(B,m)$; (ii) The consolidation of the resolution
disc by consolidating its buffer and adding the consolidation measure
to its disc, $\text{consolidateRD}:R=(B,D) \mapsto R'$ where $R'=
(B',D')$ and $(B',m') = \text{consolidateBuffer}(B)$ and $D'=
\text{addDisc}(B,m')$.
% \]

A \emph{multiresolution database} is a set of resolution discs which
share the input of measures, that is they store the same time
series. A time series is stored regularised and distributed with
different resolutions in the various resolution discs, as it was shown
in the Figure~\ref{fig:model:mtsdb}.
\begin{definition}[Multiresolution Database]
  A \emph{Multi\-re\-solution Database} is a set of resolution discs
  $M=\{R_0, \dots, R_d\}$.  An empty multiresolution database has
  empty resolution discs $M_{\emptyset}=\{R_{0_\emptyset}, \dots,
  R_{d_\emptyset}\}$.
\end{definition}

We define the addition of a measure to every resolution disc as
$\text{addMD} : M=\{R_0, \dots, R_d\} \times m \mapsto \{R'_0, \dots,
R'_d\}$ where $R'_i=\text{addRD}(R_i,m)$.

The consolidation of all resolution discs can be defined as follows:
$\text{consolidateMD}: M=\{R_0, \dots, R_d\} \mapsto \{R'_0, \dots,
R'_d\}$ where
$$ 
R'_i = \begin{cases}
  \text{consolidateRD}(R_i) & \text{if } R_i \text{ ready to consolidate} \\
  R_i                       & \text{otherwise}
\end{cases}
$$.


\subsection{Attribute aggregate function}
\label{sec:model:interpolador}

When a buffer is consolidated we summarise the time series information
using an attribute aggregate function.  Let $S$ be a time series and
$t_0$ and $t_f$ two time instants, an attribute aggregate function $f$
calculates a measure that summarises the measures of $S$ included in
the time interval $i=[T_0,T_f]$:
\begin{align*}
f&:S=\{m_0,\ldots,m_k\} \times [T_0,T_f] \mapsto m'
\end{align*}

To summarise a time series we can use different attribute aggregate
functions.  For instance, we can calculate an statistic indicator of
the time series such as the average or we can apply a more complex
digital signal processing operation, \cite{zhang11}.

Below there are some examples. Let $S'=S(T_0,T_f]$. Then:
\begin{itemize}
\renewcommand{\labelitemi}{--}
\item maximum$^d$: $S \times i \mapsto m'$ where $V(m') =
  \max_{\forall m \in S'}(V(m))$. It summarises $S'$ with the maximum
  of the measure values.
\item last$^d$: $S \times i \mapsto m'$ where $V(m') = \max(S')$. It
  summarises $S'$ with the maximum measure.
\item arithmetic mean$^d$: $S \times i \mapsto m'$ where $V(m') =
  \frac{1}{|S'|} \sum\limits_{\forall m\in S'} V(m)$. It
  summarises $S'$ with the mean of the measure values.
\end{itemize}

% With reference to data validation, attribute aggregate functions
% can cope with this process. When data has not been captured or has
% been captured erroneously, it must be treated as unknown data.
% \begin{itemize}
% \item When data has not been captured it is unknown by nature. For
%   example, we try to capture data from a sensor and there is no
%   response.
% \item When data is erroneously it must be marked as unknown. For
%   example, we capture data from a sensor but it responses in a not
%   reasonable time or we capture data that is clearly outside a
%   reasonable limits.
% \end{itemize}
% As a consequence, attribute aggregate functions deals with these two
% subprocesses: treating unknown data and marking data as
% unknown. Following with real numbers example, we extend the
% domain with a value that means 'unknown', let this unknown value be
% represented by the improper element infinity ($\infty$).

% An attribute aggregate functions treating unknown
% data is a one that can calculate a result when there are unknown
% values in the original time series, $f^u: S \times i \mapsto m'$ where
% $\exists m \in S: V(m)=\infty$. Although from a strict point of view
% operating with unknown data makes unknown result, aggregate functions
% are free to calculate whatever is needed such as time series analysis
% does with data reconstruction.

% For example, arithmetic mean$^{d}$ aggregate function returns
% $V(m')=\infty$ if $\exists m \in S: V(m)=\infty$.  We can define a new
% mean function, based on the original arithmetic mean$^{d}$ aggregate,
% that naively treats unknown values by keeping the
% known mean; in other words, it ignores unknown values found in the time
% interval: arithmetic mean$^{du}$: $S \times i \mapsto m'$ where $m' =
% \text{arithmetic mean}^{d}(S'',i)$ and $S''= \{m''\in S':V(m'')\neq
% \infty\}$.
% % ignore$^{u}$: $S \mapsto S'$ where $S'= \{m''\in S':V(m'')\neq
% % \infty\}$,
% % arithmetic mean$^{du}$: $S \times i \mapsto m'$ where $m' =
% % \text{arithmetic mean}^{d}(\text{ignore}^u(S),i)$.

% An attribute aggregate functions marking data as unknown is a one
% that can give unknown value as the resulting measure's value, $f^{mu}:
% S \times i \mapsto m'$ where $V(m')\in \mathbb{R}\cup\{\infty\}$.

% For example, we can define a maximum aggregate, based on the
% maximum$^d$ aggregate, that returns unknown if there is a
% measure's value bigger than 2:  maximum$^{dmu2}$: $S \times i
% \mapsto m'$ where $V(m') = 
% \begin{cases}
%   \infty &\text{if }  m''>2\\
%   m'' & \text{else }
% \end{cases}$ and $m''=\text{maximum}^d(S,i)$.

% %Per exemple definim un termini, si les dades estan més espaiades que 2 es marca com a desconeguda

In the design of the attribute aggregate function we can interpret a
time series in different ways, that is what we call the representation
of a time series. \citeauthor{last:keogh}, \cite{last:keogh}, cite
some possible representations for time series such as Fourier
transforms, wavelets, symbolic mappings or piecewise linear
representation. The last one is very usual due to its simplicity,
\cite{keogh01}.

Time series representations can be taken into account when computing
with the measures of the time series.  For example, a maximum
attribute aggregate function may give different values if we consider
a linear or a constant piecewise representation.

Following we show a possible family of attribute aggregate functions
for time series represented by a staircase function, that is with a
piecewise constant representation.  We define a new representation for
time series named \emph{zero-order hold backwards} (zohe). This
representation holds back each value until the preceding value. 
RRDtool, \cite{lisa98:oetiker}, has a similar aggregate function.

Let $S=\{m_0,\ldots,m_k\}$ be a time series, we define
$S(t)^{\text{zohe}}$ as its continuous representation along time $t$:
$\forall t \in \mathbb{R} ,\forall m \in S:$
\begin{equation}
 S(t)^{\text{zohe}} =  
\begin{cases}
  \infty & \text{if } t > T(\max S) \\
  V(m)   & \text{if } t\in (T(\prev_S m),T(m)]
\end{cases}
\label{eq:zohe}
\end{equation}


In conclusion, we can define many attribute aggregate functions and
thus no global assumptions can be made about them. Each user has to
decide which combination of aggregation and representation fits better
with the measured phenomena.  Therefore, \acro{MTSMS} must allow to
define user aggregate functions.







%%% Local Variables:
%%% TeX-master: "main"
%%% ispell-local-dictionary: "british"
%%% End:

% LocalWords:  genericity



\include{implementacio-relacional}


%------- Annexos ------
\appendix

%------- Glossari ------
% \cleardoublepage
% %\phantomsection\addcontentsline{toc}{chapter}{\glossaryname}
% %\pdfbookmark{\glossaryname}{bookmark:glossari}
% \chapter{Nomenclatura i abreviacions}
% \glsaddall
% \printglossary
% \printglossary[type=\acronymtype]
% \printglossary[type=notation,nonumberlist,style=estil-notation]
%------- Bibliografia ------
\cleardoublepage
%\phantomsection\addcontentsline{toc}{chapter}{\bibname}
\pdfbookmark{\bibname}{bookmark:bibliografia}
\printbibliography
%----------------------------------------------

%\backmatter

\end{document}



%%%%%%%%%%%%%%%%%%%%%%%%%%%%%%%%%%%%%%%%%%%%%%%%%%%%%%%%%%%%%%%%%%%%%%%%%%  
% Memòria Tesi Doctoral. Model d'un sistema de gestió multiresolució per a sèries temporals.
%
% Copyright (C) 2011-2013 Aleix Llusà Serra.
% 
% This LaTeX document is free software: you can redistribute it and/or
% modify it under the terms of the GNU General Public License as
% published by the Free Software Foundation, either version 3 of the
% License, or (at your option) any later version.
%
% This document is distributed in the hope that it will be useful, but
% WITHOUT ANY WARRANTY; without even the implied warranty of
% MERCHANTABILITY or FITNESS FOR A PARTICULAR PURPOSE. See the GNU
% General Public License for more details.
%
% You should have received a copy of the GNU General Public License
% along with this document. If not, see <http://www.gnu.org/licenses/>.
%
%
% Aleix Llusà Serra
% Departament de Disseny i Programació de Sistemes Electrònics de la Universitat Politècnica de Catalunya (DiPSE-UPC)
% Escola Politècnica Superior d'Enginyeria de Manresa (EPSEM)
% Av. de les Bases de Manresa, 61-73
% 08242 Manresa (Barcelona)
% PAÏSOS CATALANS 
%
% aleix (a) dipse.upc.edu
% 
% El codi font LaTeX del document es troba a 
% <http://escriny.epsem.upc.edu/projects/rrb/>
%%%%%%%%%%%%%%%%%%%%%%%%%%%%%%%%%%%%%%%%%%%%%%%%%%%%%%%%%%%%%%%%%%%%%%%%%% 
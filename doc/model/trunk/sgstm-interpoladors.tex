\newcommand{\prev}{\ant}


\section{Funcions d'agregació d'atributs}




\subsection{Interpolació (refer)}
\todo{refer}

Sigui $S$ una sèrie temporal i $T_0$ i $T_f$ dos instants de temps, un agregador d'atributs $f$ calcula la mesura que resumeix a $S$ en un interval de temps $i=[T_0,T_f]$. 
\[
f: \text{Sèrie temporal} \times \text{interval de temps} \longrightarrow \text{Mesura}
\]

Hi poden haver diferents agregadors d'atributs depenent de com es vol resumir la sèrie temporal. A més, la representació de la sèrie temporal influeix en la manera d'interpolar. A continuació es defineixen alguns agregador d'atributs per a sèries temporals representades amb  \emph{zero-order hold} cap enrere, a on l'interval de temps d'interpolació $i$ s'interpreta continu per l'esquerra $(T_0,T_f]$. 


\begin{definition}[Agregador mitjana aritmètica]
  Sigui $S=\{m_0,\ldots,m_k\}$ una sèrie temporal, $S(t)$ la
  representació de la sèrie temporal amb \emph{zero-order hold} cap
  enrere i $i=[T_0,T_f]$ un interval de temps, l'agregador mitjana
  aritmètica (MA) resumeix $S(t)$ amb una mesura que és la mitjana dels
  valors de les mesures al conjunt $S(T_0,T_f]$.
\[
MA: \text{Sèrie temporal} \times \text{interval de temps}
\longrightarrow \text{Mesura}
\]
\[
S=\{m_0,\ldots,m_k\} \times i=[T_0,T_f] \longrightarrow m'=(v',T_f)
\]
\[
S'=S(T_0:T_f]:
v' = \frac{1}{|S'|} \sum\limits_{\forall m\in S'} V(m)
\]
\end{definition}

\begin{definition}[Agregador D'atributs màxim]
  Sigui $S=\{m_0,\ldots,m_k\}$ una sèrie temporal, $S(t)$ la
  representació de la sèrie temporal amb \emph{zero-order hold} cap
  enrere i $i=(T_0,T_f]$ un interval de temps, l'agregador d'atributs màxim
  (MAX) resumeix $S(t)$ amb una mesura que és el màxim dels valors
  de les mesures al conjunt $S(T_0,T_f]$.
\[
MAX: \text{Sèrie temporal} \times \text{interval de temps} \longrightarrow \text{Mesura}
\]
\[
S=\{m_0,\ldots,m_k\} \times i=[T_0,T_f]  \longrightarrow m'=(v',T_f)
\]
\[
S'=S(T_0:T_f]:
v' = \max_{\forall m \in S'}(V(m))
\]
\end{definition}

De manera dual es pot definir l'agregador d'atributs mínim (MIN) com el
que resumeix $S(t)$ amb una mesura que és el mínim dels valors de les
mesures al conjunt $S(T_0,T_f]$.


\begin{definition}[Agregador D'atributs últim]
  Sigui $S=\{m_0,\ldots,m_k\}$ una sèrie temporal, $S(t)$ la
  representació de la sèrie temporal amb \emph{zero-order hold} cap
  enrere i $i=(T_0,T_f]$ un interval de temps, l'agregador d'atributs últim
  (LAST) resumeix $S(t)$ amb una mesura que és la mesura màxima del
  conjunt $S(T_0,T_f]$.
\[
LAST: \text{Sèrie temporal} \times \text{interval de temps} \longrightarrow \text{Mesura}
\]
\[
S=\{m_0,\ldots,m_k\} \times i=[T_0,T_f]  \longrightarrow m'=(\max(S(T_0:T_f]),T_f)
\]
\end{definition}



\begin{definition}[Agregador D'atributs àrea]
  Sigui $S=\{m_0,\ldots,m_k\}$ una sèrie temporal, $S(t)$ la
  representació de la sèrie temporal amb \emph{zero-order hold} cap
  enrere i $i=(T_0,T_f]$ un interval de temps, l'agregador d'atributs àrea
  (AREA) resumeix $S(t)$ amb una mesura que conserva l'àrea sota la
  corba de la representació \emph{zero-order hold} cap enrere del
  conjunt $S(T_0,T_f]$.
\[
AREA: \text{Sèrie temporal} \times \text{interval de temps} \longrightarrow \text{Mesura}
\]
\[
S=\{m_0,\ldots,m_k\} \times i=[T_0,T_f]  \longrightarrow m'=(v',T_f)
\]
\[
v' = 
\frac{\int_{T_0}^{T_f} S(t) dt}{T_f - T_0}
\]
per $S(t)$ estar definida a trossos, $v'$ es pot expressar com
\[
o=\min(S(T_0:T_f]),
S'= S(T_0:T_f] - \{o\},
n=\seg\limits_S \max(S'): %equivalent a n=\min(S(T_f,\infty)) quan existeix S(T_f,\infty)
\]
\[
:v'= \frac{(T(o)-T_0)V(o) 
+( T_f- T(\ant\limits_S n) )V(n) 
+\sum\limits_{\forall m \in S'}( T(m)- T(\ant\limits_S m) )V(m)}
{T_f-T_0} 
\]

Nota: s'aplica la definició $0 \times \infty = 0$ tal com es fa habitualment a la teoria de mesura, \cite{wiki:extendedreal}.
\end{definition}

%Compte amb AREA S [t<T_0,T_0] que no està definida en la definició discreta

%Quan una sèrie temporal és regular, l'intepolador mitjana aritmètica i l'interpolador àrea valen el mateix en l'interval $(T_o,n\delta]$.

If $f$ is continuous on a closed interval $[a,b]$, then there is at least one number $x^*$ in $[a,b]$ such that
$$
\int_a^b f(x)dx = f(x^*)(b-a)
$$

The average value of the function ($\bar f$)  on this interval is then given by  $f(x^*)$.

Weisstein, Eric W. "Average Function." From MathWorld--A Wolfram Web Resource. http://mathworld.wolfram.com/AverageFunction.html






\subsection{Agregador d'atributs (nova)}
\label{sec:model:interpolador}
\label{sec:model:agregador}



An attribute interpolation function is used when a buffer is
consolidated in order to summarise information from the time series.
Let $S$ be a time series and $t_0$ and $t_f$ two time instants, an
attribute interpolation function $f$ calculates a measure that
summarises an attribute of $S$ in the time interval $i=[T_0,T_f]$:
\[
f: \text{Time series} \times \text{time interval} \longrightarrow
\text{Measure}
\]
\[
f: S=\{m_0,\ldots,m_k\} \times i=[T_0,T_f] \mapsto m'
\]

There can be different attribute interpolation functions in order to
summarise a time series. As instance, we may want to calculate
statistics from a time series such as the maximum value, the average
or apply digital signal processing operations as is done in
\cite{zhang11}.  Furthermore, attribute interpolation functions copes
with data validation, which we will discuss latter in this section.

We can globally define the behaviour of attribute interpolation
functions in the discrete (set) and continuous forms. However, $T(m')$
rest undefined as the resulting consolidation time is subject to
interpretation, which we discuss below. Next, we exemplify some
operators for both cases. For simplicity, we present examples with
real numbers but they could also be integers, strings or more
elaborated structures such as arrays.


First, purely discrete interpolation (d) which resembles typical set
operators. We must define the consolidation interval depending on $i$,
let $S'=S(T_0,T_f]$:
\begin{itemize}
\item maximum$^d$: $S \times i \mapsto m'$ where $V(m') = \max_{\forall m
    \in S'}(V(m))$. It summarises $S'$ with the maximum of
  the measure values.
\item last$^d$: $S \times i \mapsto m'$ where $V(m') = \max(S')$. It
  summarises $S'$ with the maximum measure.
\item arithmetic mean$^d$: $S \times i \mapsto m'$ where $V(m') =
  \frac{1}{|S'|} \sum\limits_{\forall m\in S'} V(m)$. It
  summarises $S'$ with the mean of the measure values.
\end{itemize}


Second, continuous interpolation (c). Note that we need a continuous
function $S(t)$ which is discussed below. Let $t\in\mathbb{R}$:
\begin{itemize}
\item maximum$^c$: $S \times i \mapsto m'$ where $V(m') = \max_{\forall t
    \in [T_0,T_f]}(S(t))$. It summarises $S(t)$ with the maximum in
  interval $i$.
\item last$^c$: $S \times i \mapsto m'$ where $V(m') = S(T_f)$. It
  summarises $S(t)$ with the last value in interval $i$.
\item average$^c$: $S \times i \mapsto m'$ where $V(m') =
  \frac{\int_{T_0}^{T_f} S(t)dt}{T_f-T_0}$. It summarises $S(t)$ with
  the average value, that is the area, in interval $i$. $S(t)$ must be
  continuous on the closed interval.
\end{itemize}


In the design of the attribute interpolation function we can interpret
a time series in different ways, that is what we call the
representation of a time series. Keogh et al.\ \cite{last:keogh} cite
some possible representations for time series such as \emph{Fourier
  Transforms}, \emph{Wavelets}, \emph{Symbolic Mappings} or
\emph{Piecewise Linear Representation} (PLR). This last is remarked as
the most used owing to the most common representation is with linear
functions \cite{keogh01}.

The variety of time series representations results in a variety of the
same attribute interpolation functions. As instance, a maximum
attribute interpolation function may give different values if we
consider a linear or a constant piecewise representation. Therefore,
the time series representation results in attribute interpolation
functions families.


Next, we show two possible families for attribute interpolation
functions: one for time series represented by a staircase function and
another by a impulse train function.

First, we show some attribute interpolation functions for time series
represented by a staircase function, that is with a piecewise constant
representation.  We define a new representation for time series called
\emph{zero-order hold backwards}
(zohe%from \emph{zero-order hold everted}
) consisting in holding each value until the preceding value, which a
similar representation is used by RRDtool \cite{lisa98:oetiker}.

Let $S=\{m_0,\ldots,m_k\}$ be a time series, we define
$S(t)^{\text{zohe}}$ as its \emph{zero-order hold backwards}
continuous representation along time $t$:
%continuous definition of the time series using left-continuous step functions.
\[
\forall t \in \mathbb{R}  ,\forall m \in S:
S(t)^{\text{zohe}} =  
\begin{cases}
  \infty & \text{if } t > T(\max S) \\%\text{not defined}
  V(m) & \text{if }  t\in (T(\prev_S m),T(m)]
\end{cases}
\]


We now define the \emph{zero-order hold backwards} attribute
interpolation function family as the one interpreting the
consolidation time interval left-continuous $i=(T_0,T_f]$ and the
resulting interpolated measure's time always being $T_f$, in
accordance to the \emph{zero-order hold backwards} representation
being defined using left-continuous step functions.  Let
$S=\{m_0,\ldots,m_k\}$ be a time series and $i=[T_0,T_f]$ be a time
interval, the attribute interpolation function $f^{\text{zohe}}\in f$
summarises $S$ with a measure, $f^{zohes}: S=\{m_0,\ldots,m_k\} \times
i=[T_0,T_f] \mapsto m'$ where $m'=(v',T_f)$ and the resulting value
$v'$ depends on the attribute interpolation function calculated from
the subset $S^{zohe}=(S(T_0,T_f] \cup \{\min(S-S(-\infty,Tf))\}$.
% hauria de ser seltemp:$S[t_0,t_f]^{zohe}$.

On the one hand, purely discrete zohe interpolation functions (zohed)
have $T(m')=T_f$ and remain with $V(m')$ as defined above except for
$S'$, let now $S'=S^{zohe}$:
\begin{itemize}
\item maximum$^{zohed}$: $S \times i \mapsto m'$ where $V(m') =
  \max_{\forall m \in S'}(V(m))$ and $T(m')=T_f$.
\item last$^{zohed}$: $S \times i \mapsto m'$ where $V(m') = \max(S')$ and
  $T(m')=T_f$.
\item arithmetic mean$^{zohed}$: $S \times i \mapsto m'$ where $V(m')
  = \frac{1}{|S'|} \sum\limits_{\forall m\in S'} V(m)$ and
  $T(m')=T_f$. 
\end{itemize}
On the other hand, continuous zohe interpolation functions (zohec)
have also $T(m')=T_f$ and we must define $V(m')$ in the piecewise
approach, let now $S'=S^{zohe}$:
\begin{itemize}
\item maximum$^{zohec}$: $S \times i \mapsto m'$ where $V(m') =
  \max_{\forall m \in S'}(V(m))$ and $T(m')=T_f$.
%Note: \emph{Minimum interpolation} can be defined dually.
\item last$^{zohec}$: $S \times i \mapsto m'$ where $V(m') = \max(S')$
  and $T(m')=T_f$.
\item average$^{zohec}$: $S \times i \mapsto m'$ where $V(m') =
  \frac{1}{T_f-T_0} \big[ (T(o)-T_0)V(o) -( T(n)-T_f)V(n) +
  \sum\limits_{\forall m \in S''}( T(m)- T(\prev_S m) )V(m) \big]$,
  $T(m')=T_f$, $o=\min(S')$, $n=\max(S')$, and $S''= S' - \{o\}$.
% \[
%   \begin{split}
%   T(m')  = & \frac{1}{T_f-T_0} 
%   \big[ (T(o)-T_0)V(o) -( T(n)-T_f)V(n) \\
%     & {}+\sum\limits_{\forall m \in S''}( T(m)- T(\prev_S m) )V(m) \big]   
%    \end{split}
%   \]
\end{itemize}



Second, we show attribute interpolation functions for time series
represented by a impulse train function, that is by Dirac delta
functions.  We define a new representation for time series called
\emph{Dirac delta} ($\delta$) based on the Dirac delta, which is a
function that is zero everywhere except at zero.

Let $S=\{m_0,\ldots,m_k\}$ be a time series, we define $S(t)^{\delta}$
as its \emph{Dirac delta} continuous representation along time $t$:
\[
\forall t \in \mathbb{R}  ,\forall m \in S:
S(t)^{\delta} =  \sum V(m)\delta(t-T(m)) =
\begin{cases}
  V(m) & \text{if }  t=T(m) \\
  0 & \text{else} 
\end{cases}
\]

We now define the \emph{Dirac delta} attribute interpolation function
family as the one interpreting the consolidation time interval closed
$i=[T_0,T_f]$ and the resulting interpolated measure's time always
being $\frac{T_f+T_0}{2}$.  Let $S=\{m_0,\ldots,m_k\}$ be a time
series and $i=[T_0,T_f]$ be a time interval, the attribute
interpolation function $f^{\delta}\in f$ summarises $S$ with a
measure, $f^{\delta}: S=\{m_0,\ldots,m_k\} \times i=[T_0,T_f] \mapsto
m'$ where $m'=(v',\frac{T_f+T_0}{2})$ and the resulting value
$v'$ depends on the attribute interpolation function calculated from
the subset $S^{\delta}=S-(S(-\infty,T_0)+S(T_f,+\infty))$.
% equivalent a l'interval tancat S[T_0,T_f]

On the one hand, purely discrete $\delta$ interpolation functions
($\delta d$) have $T(m')=\frac{T_f+T_0}{2}$ and remain with $V(m')$ as
defined above, let now $S'=S^{\delta}$:
\begin{itemize}
\item maximum$^{\delta d}$: $S \times i \mapsto m'$ where $V(m') =
  \max_{\forall m \in S'}(V(m))$ and $T(m')=\frac{T_f+T_0}{2}$.
\item last$^{\delta d}$: $S \times i \mapsto m'$ where $V(m') =
  \max(S')$ and $T(m')=\frac{T_f+T_0}{2}$.
\item arithmetic mean$^{\delta d}$: $S \times i \mapsto m'$ where $V(m')
  = \frac{1}{|S'|} \sum\limits_{\forall m\in S'} V(m)$ and
  $T(m')=\frac{T_f+T_0}{2}$. 
\end{itemize}
On the other hand, continuous $\delta$ interpolation functions
($\delta c$) have also $T(m')=\frac{T_f+T_0}{2}$ and we must define
$V(m')$ in the piecewise approach, let now $S'=S^{\delta}$:
\begin{itemize}
\item maximum$^{\delta c}$: $S \times i \mapsto m'$ where $V(m') = \max(0,
  \max_{\forall m \in S'}(V(m)))$ and $T(m')=\frac{T_f+T_0}{2}$.
\item last$^{\delta c}$: $S \times i \mapsto m'$ where $V(m') = 
\begin{cases}
V(m) & \text{if } m\in S':T(m)=T_f \\
0 & \text{else}
\end{cases}$ and $T(m')=\frac{T_f+T_0}{2}$.
\item average$^{\delta c}$: $S \times i \mapsto m'$ where $V(m') =
  \sum\limits_{\forall m \in S'} V(m)$ and
  $T(m')=\frac{T_f+T_0}{2}$. Note that delta function has the
  fundamental property $\int \delta(t)dt = 1$.
\end{itemize}

In these two families shown, the continuous and the discrete form for
maximum and last functions coincide in the zohe interpretation but do
not coincide in the $\delta$ interpretation. 


\paragraph{Tractament/validació de dades}\todo{}

With reference to data validation, attribute interpolation functions
can cope with this process. When data has not been captured or has
been captured erroneously, it must be treated as unknown data.
\begin{itemize}
\item When data has not been captured it is unknown by nature. For
  example, we try to capture data from a sensor and there is no
  response.
\item When data is erroneously it must be marked as unknown. For
  example, we capture data from a sensor but it responses in a not
  reasonable time or we capture data that is clearly outside a
  reasonable limits.
\end{itemize}
As a consequence, attribute interpolation functions deals with these
two subprocesses: treating unknown data and marking data as
unknown. Following with real numbers, let unknown value be represented
by the improper element infinity ($\infty$).  

An attribute interpolation functions treating unknown data is a one
that can calculate a result when there are unknown values in the
original time series, $f^u: S \times i \mapsto m'$ where $\exists m \in
S: V(m)=\infty$. Although from a strict point of view
operating with unknown data makes unknown result, interpolation
functions are free to calculate whatever is needed such as time series
analysis does with data reconstruction.

For example, we can redefine the average$^{c}$ interpolation function
into one capable of treating unknown values by keeping the known area;
that is, total known area is spread on the time interval.
average$^{cu}$: $S \times i \mapsto m'$ where $V(m') =
\frac{\int_{T_0}^{T_f} S^u(t)dt}{T_f-T_0}$ and $S^u(t)=
\begin{cases}
  0 &\text{if }  S(t)=\infty\\
  S(t) & \text{else }
\end{cases}$.


An attribute interpolation functions marking data as unknown is a one
that can give unknown value as the resulting measure's value, $f^{mu}:
S \times i \mapsto m'$ where $V(m')\in \mathbb{R}\cup\{\infty\}$.

For example, we can define a maximum interpolation, based on the
maximum$^d$ interpolation, that returns unknown if there is a
measure's value bigger than 2.  maximum$^{dmu2}$: $S \times i
\mapsto m'$ where $V(m') = 
\begin{cases}
  \infty &\text{if }  m''>2\\
  m'' & \text{else }
\end{cases}$ and $m''=\text{maximum}^d(S,i)$.

%Per exemple definim un termini, si les dades estan més espaiades que 2 es marca com a desconeguda
    
In conclusion, a huge amount of attribute interpolation functions can
be defined and no global assumptions can be made. Therefore, MTSMS
must give freedom to the users to define their own functions.




% \section{Tractament dels desconeguts}


%operacions amb nan de octave i matlab
%http://biosig-consulting.com/matlab/NaN/
% The NaN-toolbox v2.0: A statistics and machine learning toolbox for Octave and Matlab®
% for data with and w/o MISSING VALUES encoded as NaN's.


% La censura per interval és una verificació possible. En aquesta, un valor $v$ es considera desconegut quan $v<L_{m}$ o $v>L_{M}$.


% Sigui $m=(v,t)$ una mesura, $m$ és desconeguda si $v$ és desconegut.

% Sigui $S=\{m_0,\ldots,m_k\}$ una sèrie temporal i $H$ un termini de temps, una mesura $m_i=(v_i,t_i)\in S$ és desconeguda si, donada la mesura anterior $m_{i-1}=(v_{i-1},t_{i-1})$, $t_i - t_{i-1} > H$.


% Sigui $f$ un interpolador, en el moment de calcular la mesura de consolidació $f$ decideix si és desconeguda
% \[
% desconeguda?: \text{Sèrie temporal} \times \text{interval} \longrightarrow \text{Mesura}
% \]

% Sigui $S=\{m_0,\ldots,m_k\}$ una sèrie temporal, $f$ un interpolador, $i=[T_0,T_f]$ un interval de temps i $\alpha$ un llindar, la mesura de consolidació calculada per l'interpolador $f$ és desconeguda ssi  
% \[
% \frac{t_d }{T_f - T_0} > \alpha :
% \]
% \[
% :t_d = t_{d0} + t_{df} + \sum\limits_{i=1}^{k-1}(t_i-t_{i-1}) : v_k = 'desconegut':
% \]
% \[
% : t_{d0} = \left\{\begin{array}{l} t_0-T_0 \text{ si } v_0 = 'desconegut' \\ 0\end{array}\right. ,
% t_{df} = \left\{\begin{array}{l} T_f-t_{k-1} \text{ si } v_k = 'desconegut' \\ 0\end{array}\right. :
% \]
% \[
% :k=|S|-1,(v_k,t_k)=m_k\in S' :S'= S_{T_0:T_f} \cup \{min(S_{T_f:\infty})\}
% \]










%%% Local Variables:
%%% TeX-master: "main"
%%% ispell-local-dictionary: "british"
%%% End:
% LocalWords: buffer buffers  MTSMS multiresolution DBMS zohe






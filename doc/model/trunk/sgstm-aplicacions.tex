\chapter{Usos de SGSTM}



\section{Estructures interessants}



\subsection{Discos enllaçats}



\paragraph{Exemple 2}

Les taules es poden veure a la \autoref{fig:model:mtsdb:cadena} a on la base de dades multiresolució és la vista-relació 
\begin{verbatim}
M_2 = ( ((M_2' RENAME S'_B AS S') JOIN (M^{series}_2 RENAME S AS S_B)) RENAME S'_D AS S') JOIN (M^{series}_2 RENAME S AS S_D)
\end{verbatim}



\begin{figure}[tp]
  \centering
  \begin{tabular}{|c|c|c|c|c|c|}
    \multicolumn{2}{c}{$M'_2$} \\ \hline
    $S'_B$  & $S'_D$ & $\tau$ & $\delta$ & $k$ & $f$ \\ \hline
    $S_{B1}$ & $S_{D1}$ & 45 & 5  & 2 & mitjana  \\
    $S_{D1}$ & $S_{D2}$ & 40 & 10 & 4 & mitjana  \\ \hline
  \end{tabular}\qquad
  \begin{tabular}{|c|c|c|}
    \multicolumn{3}{c}{$M^{series}_{2}$} \\ \hline
    \multirow{2}{*}{$S'$}  &  \multicolumn{2}{c|}{$S$} \\ \cline{2-3}
    & $t$      & $v$  \\ \hline
    \multirow{3}{*}{$S_{B1}$} & 46 & 0 \\ 
    & 48 & 0 \\ 
    & 49 & 0 \\ \hline
    \multirow{2}{*}{$S_{D1}$} & 40 & 0 \\ 
    & 45 & 0 \\ \hline
    \multirow{4}{*}{$S_{D2}$} & 10 & 0 \\ 
    & 20 & 0 \\ 
    & 30 & 0 \\ 
    & 40 & 0 \\ \hline
  \end{tabular}
  \caption{Taula d'una mtsdb en cadena}
  \label{fig:model:mtsdb:cadena}
\end{figure}

\todo{per a fer l'exemple falta conèixer els operadors estructurals}

\todo{falta definir qui són els buffer d'entrada de mesures}
definir una $M^{in}_2$.





Respecte a l'estructura general, l'estructura enllaçada restringeix
els períodes de consolidació de les sèries temporals: aquests són
múltiples dels discs anteriors.



\subsection{Data stream}



Base de dades multiresolució a on les sèries temporals dels buffers
només tenen una mesura; és a dir tenen cardinal afitat a 1.


Per a orientar a streams els buffers s'han de canviar els operadors
d'afegir i consolidar:

Es canvia l'operador d'afegir per tal que incorpori el càlcul orientat
a stream cada cop:
\[
\text{addB}^{\text{stream}}: B \times m \longrightarrow B' =
(streamB(S,m),\tau,\delta,f)
\]

Es canvia l'operador de consolidar per tal que reconegui la sèrie
temporal del buffer com a consolidada amb stream.

  \[
  \text{consolidaB}^{\text{stream}}: B \longrightarrow B' \times m'
  \]
  \[
  B'= (S',\tau+\delta,\delta,f)
  \]
  \[
  S' = S(\tau+\delta,\infty)
  \]
  \[
  m' \in S(\tau,\tau+\delta] 
  \]


Per a orientar a streams els buffers es defineix un nou operador
\[
\text{streamB}: S \times m \longrightarrow S' = \{f^{\text{stream}}(m_o,m)\}
\]
\[
m_o \in S
\]
\[
f^{\text{stream}} \text{ és un agregador d'atributs orientat a streams}
\]
 

Aleshores els agregadors d'atributs funcionen orientats a stream;
nota: no tots els agregadors d'atributs es poden definir com a
streams.


Per exemple l'interpolador mitjana orientat a stream:

\[
\text{mitjana}^{\text{stream}}: m_o \times m_n \longrightarrow m' = (T(m_n),v')
\]
\[
\text{a on } v' = (V(m_0) + V^1(m_n), V^2(m_n) + 1 )
\]


\subsection{Compartició de buffers}


Les diferents $f$ amb mateix $\delta$ poden compartir buffer.





\subsection{Arquitectura RRDtool}


RRDtool té una estructura multiresolució amb un buffer únic d'entrada
i buffers orientats a stream; segons es va avaluar anteriorment per
\textcite{llusa11:tfm}.






\section{Operacions habituals en les sèries temporals}


\paragraph{Semblança de dues sèries temporals}


Similarity Measures for Time Series

Hi ha varis mètodes, [keogh08:vldb] n'avalua uns quants i els generalitza amb:

Given two
time series T1 and T2 , a similarity function Dist calcu-
lates the distance between the two time series, denoted by
Dist(T1 , T2 ).

Exemplifiquem amb la distància euclídia, [keogh08:vldb] nota que és
competitiva amb les altres.

Distancia euclídia segons [faloutsous94-sigmod]


\[
D(S,Q) = \left( \sum_{i=1}^{l} (S[i]-Q[i])^2  \right)^{1/2}
\]

\begin{gather*}
  D: S \times Q \longrightarrow v: \\
  S' = map(fusio(S,Q),(t,v_1,v_2)\mapsto(t,(v_1-v_2)^2)), \\
  S'' = fold(quad,(0,0),(t^1,v^1,t^2,v^2)\mapsto(t^1,v^1+v^2)), \\
  v = \sqrt{V(m)}:m\in S''
\end{gather*}


S i Q haurien de ser regulars entre elles, sinó cal aplicar una fusió amb representació/interpretació.

Amb la multiresolució la fusió es pot fer de forma eficient. Per altra banda, es podria crear un disc resolució amb agregador de semblança.


\paragraph{Semblança de dues sèries temporals amb offset}

Aquí es descriu la solució general del problema (SequentialScan),
[faloutsous94-sigmod] n'estudia implementacions amb certes
heurístiques que aconsegueixen més eficiència.





\paragraph{Filtratge senzill per mitjana mòbil}

Sigui $p$ la mida de la finestra mòbil
\begin{gather*}
  \text{MitMobil}: S \times \text{p} \longrightarrow S':\\
  \text{map}(S,(t,v)\mapsto \text{mitjanaV}(S[t,t+p]))
\end{gather*}


Mitjana mòbil sobre la multiresolució



\paragraph{Farciment de forats}

Jo tinc una sèrie temporal i vull que entre dues mesures no hi hagi més d'un cert temps. Si no es compleix dic que té forats. 

Sigui $S$ una sèrie temporal, aquesta té forats de més durada que $d$
si alguna mesura compleix $\text{forats}(S,d) = \text{selecciona}(difT(S),v>d \bigwedge v\neq\infty)$ a on $difT(S) = \text{map}(\text{tpredecessors}(S),(t,v)\mapsto(t,t-v))$.

Amb la multiresolució el farciment de forats és natural a l'estructura i és controlat per la funció agregadora d'atributs.


* Com farciria els forats manualment a una sèrie temporal?

1. Passar-ho per un esquema de multiresolució

2. Treballar sobre la sèrie temporal:

a partir del càlcul de forats anterior $\text{forats}(S,d)$ per
exemple apliquem un farciment amb representació
zohe. $\text{farciment}(S,d) = \text{unio}(S,S')$ a on fem la selecció
de resolució $S' = S[T]^{\text{zohe}}$, $\forall (t,v) \in
\text{forats}(S,d): T = \{ \tau = t - dn |
\tau\in(t-v,t),n\in\mathbb{N} \}$.







%%% Local Variables:
%%% TeX-master: "main"
%%% End:
% LocalWords:  SGSTM

\chapter{Model SGSTM}
\label{cap:model:sgstm}


\todo{Proposta de noms}
\begin{itemize}
\item Buffer
\item Disc
\item Subsèrie temporal resolució-atribut
\item Sèrie temporal multiresolució
\item Base de dades de sèries temporals multiresolució
\end{itemize}



En aquest capítol es defineix un model dels sistemes de gestió de
bases de dades per a sèries temporals multiresolució (SGSTM). Aquest
model s'estructura un objecte principal que són les sèries temporals
multiresolució, les quals són conjunts de subsèries temporals
resolució formades per discs i buffers. El model es dissenya en dues
parts.

\begin{itemize}
\item Primer, es defineix el model d'estructura de les dades, és a
  dir, la forma que prenen els buffers, els discs, les subsèries
  resolució, les sèries temporals multiresolució i les bases de dades
  de sèries temporals multiresolució.
\item Segon, es defineix el model d'operacions sobre les dades, és a
  dir, els operadors bàsics que permeten modelar el comportament i la
  manipulació de bases de dades multiresolució.
\end{itemize}





\section{Model estructural de dades}

Una base de dades de sèries temporals multiresolució (BDSTM) és un
conjunt de sèries temporals multiresolució a on cada una és una
relació de buffers amb discs.

\todo{descriure l'arquitectura. Dir que l'arquitectura té tot de paràmetres variables i que configurar-los és el que anomenem definir l'esquema multiresolució?}
\begin{figure}[tp]
\centering
\begin{tikzpicture}
 \tikzset{
        myarrow/.style={->, >=latex',  thick},
      }
      

  \node[rectangle,draw,minimum height=6cm,minimum width=9cm] (m) {};
  \draw[shift=( m.south west)]   
  node[above right] {base de dades multiresolució};


  %discmig
  \node (m.center) (discr1) {...};

  %discr
  
  \node[ellipse,draw,minimum height=3.5cm,minimum width=2.5cm,alias=discr0] [left=of discr1] {};
  \node[above=0cm of discr0.north] {R0};
  \node[below=0cm of discr0] {disc resolució};

  \node[cylinder, draw, shape border rotate=90, aspect=0.25,alias=buffer0] [below=3mm of discr0.north] {buffer};
  \node[circle, draw,alias=disc0]  [above=3mm of discr0.south] {disc} ;
  \draw [->] (disc0.center)++(.4:.4cm) arc(0:180:.4cm);
  \draw[myarrow] (buffer0.bottom) -- (disc0.north);


  %discrd

  \node[ellipse,draw,minimum height=3.5cm,minimum width=2.5cm,alias=discrd] [right=of discr1] {};
  \node[above=0cm of discrd] {Rd};
  \node[below=0cm of discrd] {disc resolució};

  \node[cylinder, draw, shape border rotate=90, aspect=0.25,alias=bufferd] [below=3mm of discrd.north] {buffer};
  \node[circle, draw,alias=discd]  [above=3mm of discrd.south] {disc} ;
  \draw [->] (discd.center)++(.4:.4cm) arc(0:180:.4cm);
  \draw[myarrow] (bufferd.bottom) -- (discd.north);



  %mesura 
  \node[above=1cm of m.north] (m0) {};

  \draw[myarrow] (m0) -- (m.north) 
  node[right,midway] {mesura};

  \draw[myarrow] (m.north) -- (buffer0);
  \draw[myarrow] (m.north) -- (bufferd);
  \draw[myarrow] (m.north) -- (discr1);

\end{tikzpicture}
\caption{Arquitectura del model SGSTM}
\label{fig:model:bdstm}
\end{figure}


Aquí només es defineixen els conceptes referents a l'estructura del
model. Aquesta estructura, però, requereix uns operadors específics
per a emmagatzemar-hi les mesures, els quals es defineixen a
l'apartat~\ref{sec:model:sgstm-estructurals}.




\subsection{Buffer}\label{sec:model:buffer}\todo{falta parlar de regularitat de ST}\todo{falta parlar de representació de ST}

Un buffer és un contenidor d'una sèrie temporal, regular o no regular, que mitjançant una funció permet regularitzar aquesta sèrie temporal amb un període de mostreig constant. A l'acció de regularitzar un interval d'una sèrie temporal l'anomenarem consolidació, al període de mostreig contant l'anomenarem pas de consolidació i a la funció de regularització l'anomenarem agregador d'atributs.

\begin{definition}[Buffer]
  Definim \emph{buffer} com el tuple $(S,\tau,\delta,f)$, en el que
  $S$ és una sèrie temporal, $\tau$ és el darrer instant de temps de
  consolidació, $\delta$ és la durada del pas de consolidació i $f$ és
  un agregador d'atributs.
\end{definition}

La consolidació d'una sèrie temporal s'inicia en un instant de temps
concret i té lloc a cada pas de consolidació. Amb la finalitat
d'establir els intervals de consolidació de la sèrie temporal, es
defineix un buffer inicial.

\begin{definition}\label{def:model:buffer_buit}
  Definim buffer inicial o buffer buit com el buffer $B_{\emptyset} =
  (\emptyset,t_0, \delta_0, f)$, el qual
  conté una sèrie temporal buida, l'instant de temps inicial de
  consolidació, una durada que indica el pas de consolidació i un
  agregador d'atributs.
\end{definition}

A partir del buffer buit es poden conèixer tots els instants de temps
de consolidació del buffer, els quals seran $t_0+k\delta,
k\in\mathbb{N}$. Aquests instants de temps de consolidació també
defineixen els intervals de temps de consolidació del buffer de la
forma $i=[\tau,\tau+\delta]$. La consolidació de la sèrie temporal $S$
d'un buffer en un interval de temps $i$ dóna com a resultat una mesura
$m=(t,v)$ calculada a partir de l'agregador d'atributs $m = f (S,
i)$. Més endavant a la \autoref{sec:model:agregador} detallem el
concepte d'agregador d'atributs.








\subsection{Disc}\label{sec:model:disc}

Un disc és un contenidor d'una sèrie temporal regular amb un nombre
acotat de mesures. En arribar al nombre màxim de mesures permeses,
cada cop que s'afegeix una mesura nova s'elimina la mesura mínima de
la sèrie temporal.  Així doncs, un disc és semblant a una cua
\emph{First In First Out} (FIFO), a on el primer d'arribar és el
primer de sortir.

\begin{definition}[Disc]
  Definim \emph{disc} com el tuple $(S,k)$, en el que $S$
  és una sèrie temporal i $k\in\N{}$ és el cardinal màxim de $S$.
\end{definition}

A l'inici, un disc no conté mesures però cal que estigui caracteritzat
pel cardinal màxim. Amb aquesta finalitat es defineix un disc inicial.

\begin{definition}\label{def:model:disc_buit}
  Definim disc inicial o disc buit com el disc $D_{\emptyset} =
  (\emptyset,k)$, el qual conté una sèrie temporal buida i el cardinal
  màxim que podrà prendre $S$.
\end{definition}




\subsection{Subsèrie resolució}\label{sec:model:subserie-resolucio}

Una subsèrie temporal resolució és una parella de disc i buffer. En el
buffer hi ha la part d'una sèrie temporal a regularitzar i en el disc
hi ha l'altra part ja regularitzada, amb un nombre acotat de
mesures. A l'acció de regularitzar l'anomenem consolidar en coherència
amb el concepte descrit pels buffers.


\begin{definition}[Subsèrie resolució]
  Definim \emph{subsèrie resolució} com el tuple $(B,D)$, en el que
  $B$ és un buffer i $D$ és un disc.
\end{definition}
 
La definició de buffer buit (def.~\ref{def:model:buffer_buit}) i de
disc buit (def.~\ref{def:model:disc_buit}) indueixen a una definició
de subsèrie resolució buida.
\begin{definition}\label{def:model:subserie_resolucio_buida}
  Definim subsèrie resolució buida com la subsèrie resolució $R_{\emptyset}
  = (B_{\emptyset},D_{\emptyset})$, la qual conté un buffer buit i un
  disc buit.
\end{definition}


Les subsèries resolució es consoliden seguint els criteris del seu
buffer i emmagatzemant la mesura de consolidació al seu disc.




\subsection{Sèrie temporal multiresolució}

Una sèrie temporal multiresolució és un conjunt de subsèries resolució que
comparteixen l'entrada de mesures, les quals provenen d'una mateixa
sèrie temporal. La sèrie temporal queda regularitzada i distribuïda en
les diferents subsèries resolució amb resolucions diferents, tal com s'ha
vist a la \autoref{fig:model:bdstm}


\begin{definition}[Sèrie temporal multiresolució]
  Definim \emph{sèrie temporal multiresolució} com el conjunt de subsèries
  resolució $M=\{R_0,\dotsc,R_d\}$.
\end{definition}

A partir de la definició de subsèrie resolució buida
(def.~\ref{def:model:subserie_resolucio_buida}) és defineix la sèrie
temporal multiresolució buida.
 
\begin{definition}\label{def:model:st_multiresolucio_buit}
  Definim sèrie temporal multiresolució buida com el conjunt de subsèries
  resolució buides
  $M_{\emptyset}=\{R_{0_{\emptyset}},\dotsc,R_{d_{\emptyset}\}}$.
\end{definition}

Normalment, en una sèrie temporal multiresolució no hi ha dues subsèries
resolució amb la mateixa informació. És a dir, donats dues subsèries
resolució $R_a = (B_a, D_a)$ i $R_b = (B_b, D_b)$, els seus respectius
buffers $B_a=(S_a,\tau_a,\delta_a,f_a)$ i
$B_b=(S_b,\tau_b,\delta_b,f_b)$ no tenen el mateix interval de
consolidació i agregador d'atributs: $\delta_a \neq \delta_b \wedge
f_a \neq f_b$.




\subsection{Base de dades de sèries temporals
  multiresolució}\label{sec:model:bdstm}

Una base de dades de sèries temporals multiresolució (BDSTM) és un
conjunt de sèries temporal multiresolució.

Les sèries temporals es poden emmagatzemar exclusivament en una BDSTM
o poden conviure amb altres tipus de dades en bases de dades que tinguin
capacitats de BDSTM.



\subsubsection{Relació sèrie temporal multiresolució}



Una sèrie temporal multiresolució és una relació de buffers i discs. A
cada parella buffer-disc l'anomenem subsèrie resolució. Així doncs, una
sèrie temporal multiresolució és un conjunt de subsèries resolució.

Com a conjunt de subsèries resolució, una sèrie temporal multiresolució
s'observa com una relació de grau sis a on la capçalera conté els
atributs
\begin{itemize}
\item sèrie temporal del buffer ($S_B$),
\item sèrie temporal del disc ($S_D$),
\item darrer instant de consolidació ($\tau$),
\item pas de consolidació ($\delta$),
\item màxim cardinal del disc ($k$),
\item i funció d'agregació d'atributs ($f$).
\end{itemize}

Una restricció habitual és que $\delta$ i $f$ no estiguin repetits; és
a dir que són els atributs clau de la relació sèrie temporal multiresolució.

Així doncs observada com a relació, tal com s'ha fet en el model de
SGST, podem escriure una sèrie temporal multiresolució
\begin{definition}[Representació sèrie temporal multiresolució]
  Sigui $M=\{R_0,\dotsc,R_d\}$ una sèrie temporal multiresolució a on
  $R_i =(B_i,D_i)$ són subsèries resolució amb domini de sèrie
  temporal per les sèries temporals, de $\bar{\R{}}$ pels temps, de
  $\N{}$ pel pas de consolidació i de funció per a la funció
  d'agregació d'atributs; representada com a relació s'escriu com $ M
  = ( S_B: \text{Sèrie Temporal}, S_D: \text{Sèrie Temporal}, \tau:
  \R{}, \delta: \R{}, k: \N{}, f: \text{funció}\}, \{ \{ S_B: S_{B_0}
  , S_D : S_{D_0} , \tau : \tau_{B_0}, \delta : \delta_{B_0}, k:
  k_{D_0}, f : f_{B_0} \} , \dotsc, \{ S_B: S_{B_d} , S_D : S_{D_d} ,
  \tau : \tau_{B_d}, \delta : \delta_{B_d}, k: k_{D_d}, f : f_{B_d} \}
  \} )$.
\end{definition}

De la mateixa manera que per a les sèries temporals en el model de
SGST, una sèrie temporal multiresolució es pot escriure de manera
simplificada com al conjunt de tuples $M = \{ (S_{B_0}, S_{D_0} ,
\tau_{B_0}, \delta_{B_0}, k_{D_0}, f_{B_0} ), \dotsc, (S_{B_d},
S_{D_d} , \tau_{B_d}, \delta_{B_d}, k_{D_d}, f_{B_d} ) \}$.




\subsection{Exemples}

\begin{example} [Sèrie temporal multiresolució]
\label{ex:model:bdm1}%ATENCIÓ als canvis: les dades d'aquest exemple s'utilitzen en altres apartat.


\todo{cal explicar en que consisteix l'exemple}

$M^1 = \{ ( \{(26,0),(27,0)\}  , \{(10,0),(15,0),(20,0),(25,0)\} , 25 , 5 ,4 , \text{mitjana} ) , ( \{(22,0),(26,0),(27,0)\} , \{(0,0),(10,0),(20,0)\} , 20 , 10 ,3 , \text{mitjana} ) \}$


Una sèrie temporal multiresolució fotografiada en un instant de temps que podria
ser entre el 27 i el 30. Tots els valors valen zero per tal de facilitar la comprensió de l'exemple.


\begin{figure}[tp]
\centering
%\usetikzlibrary{shapes,arrows,positioning}
\begin{tikzpicture}
 \tikzset{
        myarrow/.style={->, >=latex',  thick},
      }
      

  \node[rectangle,draw,minimum height=6cm,minimum width=9cm] (m) {};
  \draw[shift=( m.south west)]   
  node[above right] {base de dades multiresolució};


  %discmig
  \node (m.center) (discr1) {};

  %discr
  
  \node[ellipse,draw,minimum height=3.5cm,minimum width=2.5cm,alias=discr0] [left=of discr1] {};
  \node[above=0cm of discr0.north] {$R_0$};
  \node[below=0cm of discr0] {subsèrie resolució};

  \node[cylinder, draw, shape border rotate=90, aspect=0.25,alias=buffer0] [below=3mm of discr0.north] {mitjana};
  \node[circle, draw,alias=disc0,minimum width=1.2cm]  [above=3mm of discr0.south] {5} ;
  \draw [->] (disc0.center)++(.2:.2cm) arc(0:180:.2cm);
  \draw[myarrow] (buffer0.bottom) -- (disc0.north);

  \node[circle,minimum width=9mm] (d0boles) [below=0mm of disc0.center,anchor=center] {};
  \node[below=0mm of d0boles.north,anchor=center] {$\circ$};
  \node[below=0mm of d0boles.east,anchor=center] {$\circ$};
  \node[below=0mm of d0boles.south,anchor=center] {$\circ$};
  \node[below=0mm of d0boles.west,anchor=center] {$\circ$};


  %discrd

  \node[ellipse,draw,minimum height=3.5cm,minimum width=2.5cm,alias=discrd] [right=of discr1] {};
  \node[above=0cm of discrd] {$R_1$};
  \node[below=0cm of discrd] {subsèrie resolució};

  \node[cylinder, draw, shape border rotate=90, aspect=0.25,alias=bufferd] [below=3mm of discrd.north] {mitjana};
  \node[circle, draw,alias=discd,minimum width=1.2cm]  [above=3mm of discrd.south] {10} ;
  \draw [->] (discd.center)++(.3:.3cm) arc(0:180:.3cm);
  \draw[myarrow] (bufferd.bottom) -- (discd.north);

  \node[circle,minimum width=9mm] (d1boles) [below=0mm of discd.center,anchor=center] {};
  \node[below=0mm of d1boles.north,anchor=center] {$\circ$};
  \node[below=0mm of d1boles.south east,anchor=center] {$\circ$};
  \node[below=0mm of d1boles.south west,anchor=center] {$\circ$};


  %mesura 
  \node[above=1cm of m.north] (m0) {};

  \draw[myarrow] (m0) -- (m.north) 
  node[right,midway] {mesura};

  \draw[myarrow] (m.north) -- (buffer0);
  \draw[myarrow] (m.north) -- (bufferd);


\end{tikzpicture}
\caption{Arquitectura de SGSTM particular per l'\autoref{ex:model:bdm1}}
\label{fig:model:ex1}
\end{figure}\todo{cal canviar buffer, disc, etc. pels paràmetres addients}

L'arquitectura de la base de dades que conté aquesta sèrie temporal multiresolució es pot veure a la \autoref{fig:model:ex1}.



%S'observa que per tal de complir amb les propietats de les relacions, totes les sèries temporals dels buffers han de ser del mateix tipus, és a dir tenir la mateixa capçalera. El mateix succeeix amb les sèries temporals dels discs. (Vegeu els exemples de la secció \ref{par:model:exemple-relvalues} sobre valors relació).

\begin{figure}[tp]
  \centering
  \begin{tabular}{|c|c|c|c|c|c|}
    \multicolumn{2}{c}{$M_1$} \\ \hline
    $S_B$  & $S_D$ & $\tau$ & $\delta$ & $k$ & $f$ \\ \hline
      \begin{tabular}{|c|c|}
         \hline
         $t$  & $v$ \\ \hline
         26  & 0 \\
         27 & 0 \\\hline
       \end{tabular} & 
      \begin{tabular}{|c|c|}
         \hline
         $t$  & $v$ \\ \hline
         10  & 0 \\
         15  & 0 \\
         20 & 0 \\ 
         25 & 0 \\\hline
       \end{tabular} 
       & 25 & 5  & 4 & mitjana  \\ \hline
       \begin{tabular}{|c|c|}
         \hline
         $t$  & $v$ \\ \hline
         22  & 0 \\
         26  & 0 \\
         27 & 0 \\\hline
       \end{tabular} & 
      \begin{tabular}{|c|c|}
         \hline
         $t$  & $v$ \\ \hline
          0  & 0 \\
         10  & 0 \\
         20  & 0 \\\hline
       \end{tabular}  
       & 20 & 10 & 3 & mitjana  \\ \hline
  \end{tabular}
  \caption{Taula d'una sèrie temporal multiresolució}
  \label{fig:model:stm}
\end{figure}

\end{example}


\begin{example} [Sèrie temporal multiresolució amb vistes]

\todo{}

És el mateix que la sèrie temporal multiresolució de
l'\autoref{ex:model:bdm1} però organitzat de forma més còmode amb
vistes relacionals.  \todo{escriure l'expressió relacional de les
  vistes}


\begin{figure}[tp]
  \centering
  \begin{tabular}{|c|c|c|c|c|c|}
    \multicolumn{2}{c}{$M'_2$} \\ \hline
    $S'_B$  & $S'_D$ & $\tau$ & $\delta$ & $k$ & $f$ \\ \hline
    $S_{B1}$ & $S_{D1}$ & 45 & 5  & 4 & mitjana  \\
    $S_{B2}$ & $S_{D2}$ & 40 & 10 & 3 & mitjana  \\ \hline
  \end{tabular}\qquad
  \begin{tabular}{|c|c|c|}
    \multicolumn{3}{c}{$M^{\text{series}}_{2}$} \\ \hline
    \multirow{2}{*}{$S'$}  &  \multicolumn{2}{c|}{$S$} \\ \cline{2-3}
    & $t$      & $v$  \\ \hline
    \multirow{2}{*}{$S_{B1}$} 
    & 26 & 0 \\ 
    & 27 & 0 \\ \hline
    \multirow{3}{*}{$S_{B2}$} 
    & 22 & 0 \\ 
    & 26 & 0 \\ 
    & 27 & 0 \\ \hline
    \multirow{4}{*}{$S_{D1}$} 
    & 10 & 0 \\ 
    & 15 & 0 \\ 
    & 20 & 0 \\ 
    & 25 & 0 \\ \hline
    \multirow{2}{*}{$S_{D2}$} 
    &  0 & 0 \\ 
    & 10 & 0 \\ 
    & 20 & 0 \\ \hline
  \end{tabular}
  \caption{Taula d'una sèrie temporal multiresolució amb vistes relacionals}
  \label{fig:model:stm:vistes}
\end{figure}

\end{example}



\begin{example} [Sèrie temporal multiresolució amb desfasaments]
\label{ex:model:bdm-desfasaments}

\todo{}

El mateix que l'\autoref{ex:model:bdm1} però ara canviem la funció de
la segona subsèrie resolució per un agregador amb desfasament; és a
dir que cada cop que consolida retorna una mesura amb un retard d'una certa durada. Aquest nou agregador també fa la mitjana però amb un desfasament de $5$ unitat de temps. \todo{potser dir que quan parlem dels esquemes multiresolució més endavant definirem amb precisió aquests conceptes?}

De què pot servir la mitjanad1? per calcular mitjanes centrades? estem fent una interpolació sobre la representació centrada en l'interval de la sèrie temporal? \todo{}


$M^d = \{ ( \{(26,0),(27,0)\}  , \{(10,0),(15,0),(20,0),(25,0)\} , 25 , 5 ,4 , \text{mitjana} ) , ( \{(19,0),(22,0),(26,0),(27,0)\} , \{(5,0),(15,0)\} , 20 , 10 ,3 , \text{mitjanad1} ) \}$




Nota sobre la $S_{D2}$: encara no ha arribat al cardinal màxim de
$k=3$ degut a que suposem que la base de dades s'inicia a l'instant de
temps $0$. Nota sobre la $S_{B2}$: el buffer ara és 5 unitats més gran i emmagatzema mesures de l'interval [15,30].


\begin{figure}[tp]
  \centering
  \begin{tabular}{|c|c|c|c|c|c|}
    \multicolumn{2}{c}{$M_1$} \\ \hline
    $S_B$  & $S_D$ & $\tau$ & $\delta$ & $k$ & $f$ \\ \hline
      \begin{tabular}{|c|c|}
         \hline
         $t$  & $v$ \\ \hline
         26  & 0 \\
         27 & 0 \\\hline
       \end{tabular} & 
      \begin{tabular}{|c|c|}
         \hline
         $t$  & $v$ \\ \hline
         10  & 0 \\
         15  & 0 \\
         20 & 0 \\ 
         25 & 0 \\\hline
       \end{tabular} 
       & 25 & 5  & 4 & mitjana  \\ \hline
       \begin{tabular}{|c|c|}
         \hline
         $t$  & $v$ \\ \hline
         19  & 0 \\
         22  & 0 \\
         26  & 0 \\
         27 & 0 \\\hline
       \end{tabular} & 
      \begin{tabular}{|c|c|}
         \hline
         $t$  & $v$ \\ \hline
          5  & 0 \\
         15  & 0 \\\hline
       \end{tabular}  
       & 20 & 10 & 3 & mitjanad1  \\ \hline
  \end{tabular}
  \caption{Taula d'una sèrie temporal multiresolució amb desfasaments}
  \label{fig:model:stm}
\end{figure}





\end{example}


%%% Local Variables: 
%%% mode: latex
%%% TeX-master: "main"
%%% End: 

% LocalWords:  buffers multiresolució agregador l'agregador subsèries
% LocalWords:  d'agregador

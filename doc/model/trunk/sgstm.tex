\chapter{Model SGSTM}
\label{cap:model:sgstm}

En aquest capítol es defineix un \glsdispsec{not:sgstm}{model dels
  sistemes de gestió de bases de dades per a sèries temporals
  multiresolució (SGSTM)}. Aquest model s'estructura en un objecte
principal que són les sèries temporals multiresolució, les quals es
defineixen com a conjunts de subsèries temporals resolució formades
per discs i buffers. El model es dissenya en tres parts.

\begin{itemize}
\item Primer, es defineix el model d'estructura de les dades, és a
  dir, la forma que prenen els buffers, els discs, les subsèries
  resolució, les sèries temporals multiresolució i les bases de dades
  de sèries temporals multiresolució.

\item Segon, es defineix el model d'operacions sobre les dades, és a
  dir, els operadors bàsics que permeten modelar el comportament i la
  manipulació de bases de dades multiresolució.

\item Tercer, s'expliquen amb més detall les funcions específiques del
  model que permeten agregar diferents atributs de les sèries
  temporals per a obtenir les multiresolucions. Les funcions
  d'agregació d'atributs són requerides pel model però no hi estan
  lligades.
\end{itemize}




\section{Model estructural de dades}

Els objectes estructurals principals d'un SGSTM, els quals es
defineixen en aquesta secció, són els següents:
\begin{itemize}
\item Buffer
\item Disc
\item Subsèrie temporal resolució
\item Sèrie temporal multiresolució
\item Base de dades de sèries temporals multiresolució
\end{itemize}

Amb aquest elements definim que en una base de dades per a sèries
temporals multiresolució (BDSTM) hi ha sèries temporals multiresolució
a on cada una és una subsèrie temporal resolució formada per una
relació d'un buffer amb un disc. Aquests objectes tenen molta
correspondència amb les sèries temporals i per tant amb el model de
SGST.


\begin{figure}[tp]
\centering
\begin{tikzpicture}
 \tikzset{
        myarrow/.style={->, >=latex',  thick},
      }
      

  \node[rectangle,draw,minimum height=6cm,minimum width=9cm] (m) {};
  \draw[shift=( m.south west)]   
  node[above right] {base de dades multiresolució};


  %discmig
  \node (m.center) (discr1) {...};

  %discr
  
  \node[ellipse,draw,minimum height=3.5cm,minimum width=2.5cm,alias=discr0] [left=of discr1] {};
  \node[above=0cm of discr0.north] {R0};
  \node[below=0cm of discr0] {disc resolució};

  \node[cylinder, draw, shape border rotate=90, aspect=0.25,alias=buffer0] [below=3mm of discr0.north] {buffer};
  \node[circle, draw,alias=disc0]  [above=3mm of discr0.south] {disc} ;
  \draw [->] (disc0.center)++(.4:.4cm) arc(0:180:.4cm);
  \draw[myarrow] (buffer0.bottom) -- (disc0.north);


  %discrd

  \node[ellipse,draw,minimum height=3.5cm,minimum width=2.5cm,alias=discrd] [right=of discr1] {};
  \node[above=0cm of discrd] {Rd};
  \node[below=0cm of discrd] {disc resolució};

  \node[cylinder, draw, shape border rotate=90, aspect=0.25,alias=bufferd] [below=3mm of discrd.north] {buffer};
  \node[circle, draw,alias=discd]  [above=3mm of discrd.south] {disc} ;
  \draw [->] (discd.center)++(.4:.4cm) arc(0:180:.4cm);
  \draw[myarrow] (bufferd.bottom) -- (discd.north);



  %mesura 
  \node[above=1cm of m.north] (m0) {};

  \draw[myarrow] (m0) -- (m.north) 
  node[right,midway] {mesura};

  \draw[myarrow] (m.north) -- (buffer0);
  \draw[myarrow] (m.north) -- (bufferd);
  \draw[myarrow] (m.north) -- (discr1);

\end{tikzpicture}
\caption{Arquitectura d'una BDSTM}
\label{fig:model:bdstm}
\end{figure}


Aquests objectes també permeten definir l'arquitectura d'una BDSTM com
es pot veure a la \autoref{fig:model:bdstm}.  Un SGSTM és una solució
d'emmagatzematge per a sèries temporals a on, resumint, la informació
es distribueix mitjançant diferents resolucions temporals.  Una sèrie
temporal multiresolució és una co\l.lecció de subsèries resolució, les
quals acumulen temporalment mesures en un buffer a on són processades
i finalment emmagatzemades en un disc. El processament de les dades té
per objectiu canviar els intervals de temps entre les mesures per tal
de compactar la informació de les sèries temporals. D'aquesta manera,
les sèries temporals queden emmagatzemades en diferents resolucions
temporals distribuïdes en els discs.

Els discs tenen la mida limitada i només poden contenir un nombre
fixat de mesures. Quan un disc no té més capacitat ha d'eliminar una
mesura. Com a conseqüència una BDSTM té la mida fixada i les sèries
temporals hi queden emmagatzemades a trossos; és a dir com a subsèries
temporals. 


En aquesta secció només es defineixen els conceptes referents a
l'estructura del model. Aquesta estructura, però, requereix uns
operadors específics per a emmagatzemar-hi i consolidar-hi les
mesures, els quals es defineixen a
l'apartat~\ref{sec:model:sgstm-estructurals}. També requereix unes
funcions per a agregar els atributs de les sèries temporals, els quals
es defineixen a l'apartat~\ref{sec:model:agregador}.




\subsection{Buffer}\label{sec:model:buffer}\todo{falta parlar de regularitat de ST}\todo{falta parlar de representació de ST}

Un buffer és un contenidor d'una sèrie temporal, regular o no regular, que mitjançant una funció permet regularitzar aquesta sèrie temporal amb un període de mostreig constant. A l'acció de regularitzar un interval d'una sèrie temporal l'anomenarem consolidació, al període de mostreig contant l'anomenarem pas de consolidació i a la funció de regularització l'anomenarem agregador d'atributs.

\begin{definition}[Buffer]
  Definim \emph{buffer} com el tuple
  \glsdispdef{not:buffer}{$(S,\tau,\delta,f)$}, en el que
  \glsdispdef{not:sgstm:sb}{$S$} és una sèrie temporal,
  \glsdispdef{not:sgstm:tau}{$\tau$} és el darrer instant de temps de
  consolidació, \glsdispdef{not:sgstm:delta}{$\delta$} és la durada
  del pas de consolidació i \glsdispdef{not:sgstm:f}{$f$} és un
  agregador d'atributs.
\end{definition}

La consolidació d'una sèrie temporal s'inicia en un instant de temps
concret i té lloc a cada pas de consolidació. Amb la finalitat
d'establir els intervals de consolidació de la sèrie temporal, es
defineix un buffer inicial.

\begin{definition}\label{def:model:buffer_buit}
  Definim buffer inicial o buffer buit com el buffer $B_{\emptyset} =
  (\emptyset,t_0, \delta_0, f)$, el qual
  conté una sèrie temporal buida, l'instant de temps inicial de
  consolidació, una durada que indica el pas de consolidació i un
  agregador d'atributs.
\end{definition}

A partir del buffer buit es poden conèixer tots els instants de temps
de consolidació del buffer, els quals seran $t_0+k\delta,
k\in\mathbb{N}$. Aquests instants de temps de consolidació també
defineixen els intervals de temps de consolidació del buffer de la
forma $i=[\tau,\tau+\delta]$. La consolidació de la sèrie temporal $S$
d'un buffer en un interval de temps $i$ dóna com a resultat una mesura
$m=(t,v)$ calculada a partir de l'agregador d'atributs $m = f (S,
i)$. Més endavant a la \autoref{sec:model:agregador} detallem el
concepte d'agregador d'atributs.








\subsection{Disc}\label{sec:model:disc}

Un disc és un contenidor d'una sèrie temporal regular amb un nombre
acotat de mesures. En arribar al nombre màxim de mesures permeses,
cada cop que s'afegeix una mesura nova s'elimina la mesura mínima de
la sèrie temporal.  Així doncs, un disc és semblant a una cua
\emph{First In First Out} (FIFO), a on el primer d'arribar és el
primer de sortir.

\begin{definition}[Disc]
  Definim \emph{disc} com el tuple \glsdispdef{not:disc}{$(S,k)$}, en
  el que \glsdispdef{not:sgstm:sd}{$S$} és una sèrie temporal i
  \glsdispdef{not:sgstm:k}{$k\in\N{}$} és el cardinal màxim de $S$.
\end{definition}

A l'inici, un disc no conté mesures però cal que estigui caracteritzat
pel cardinal màxim. Amb aquesta finalitat es defineix un disc inicial.

\begin{definition}\label{def:model:disc_buit}
  Definim disc inicial o disc buit com el disc $D_{\emptyset} =
  (\emptyset,k)$, el qual conté una sèrie temporal buida i el cardinal
  màxim que podrà prendre $S$.
\end{definition}




\subsection{Subsèrie resolució}\label{sec:model:subserie-resolucio}

Una subsèrie temporal resolució és una parella de disc i buffer. En el
buffer hi ha la part d'una sèrie temporal a regularitzar i en el disc
hi ha l'altra part ja regularitzada, amb un nombre acotat de
mesures. A l'acció de regularitzar l'anomenem consolidar en coherència
amb el concepte descrit pels buffers.


\begin{definition}[Subsèrie resolució]
  Definim \emph{subsèrie resolució} com el tuple
  \glsdispdef{not:subserieresolucio}{$(B,D)$}, en el que $B$ és un buffer i $D$
  és un disc.
\end{definition}
 
La definició de buffer buit (def.~\ref{def:model:buffer_buit}) i de
disc buit (def.~\ref{def:model:disc_buit}) indueixen a una definició
de subsèrie resolució buida.
\begin{definition}\label{def:model:subserie_resolucio_buida}
  Definim subsèrie resolució buida com la subsèrie resolució $R_{\emptyset}
  = (B_{\emptyset},D_{\emptyset})$, la qual conté un buffer buit i un
  disc buit.
\end{definition}


Les subsèries resolució es consoliden seguint els criteris del seu
buffer i emmagatzemant la mesura de consolidació al seu disc.




\subsection{Sèrie temporal multiresolució}

Una sèrie temporal multiresolució és un conjunt de subsèries resolució que
comparteixen l'entrada de mesures, les quals provenen d'una mateixa
sèrie temporal. La sèrie temporal queda regularitzada i distribuïda en
les diferents subsèries resolució amb resolucions diferents, tal com s'ha
vist a la \autoref{fig:model:bdstm}


\begin{definition}[Sèrie temporal multiresolució]
  Definim \emph{sèrie temporal multiresolució} com el conjunt de subsèries
  resolució \glsdispdef{not:seriemultiresolucio}{$M=\{R_0,\dotsc,R_d\}$}.
\end{definition}

A partir de la definició de subsèrie resolució buida
(def.~\ref{def:model:subserie_resolucio_buida}) és defineix la sèrie
temporal multiresolució buida.
 
\begin{definition}\label{def:model:st_multiresolucio_buit}
  Definim sèrie temporal multiresolució buida com el conjunt de subsèries
  resolució buides
  $M_{\emptyset}=\{R_{0_{\emptyset}},\dotsc,R_{d_{\emptyset}\}}$.
\end{definition}

Normalment, en una sèrie temporal multiresolució no hi ha dues subsèries
resolució amb la mateixa informació. És a dir, donats dues subsèries
resolució $R_a = (B_a, D_a)$ i $R_b = (B_b, D_b)$, els seus respectius
buffers $B_a=(S_a,\tau_a,\delta_a,f_a)$ i
$B_b=(S_b,\tau_b,\delta_b,f_b)$ no tenen el mateix interval de
consolidació i agregador d'atributs: $\delta_a \neq \delta_b \wedge
f_a \neq f_b$.




\subsection{Base de dades de sèries temporals
  multiresolució}\label{sec:model:bdstm}

Una base de dades de sèries temporals multiresolució (BDSTM) és una
co\l.lecció de sèries temporal multiresolució.  Les sèries temporals
es poden emmagatzemar exclusivament en una BDSTM o poden conviure amb
altres tipus de dades en bases de dades que tinguin capacitats de
BDSTM.


Una BDSTM té uns paràmetres que cal configurar per a cada sèrie
temporal multiresolució: pas de consolidació, funció d'agregació,
etc. Anomenem esquema de multiresolució a l'efecte que produeixen les
configuracions possibles d'aquests paràmetres. Així doncs, a cada
BDSTM li podem observar i manipular els seus esquemes de
multiresolució, el qual mostrem amb més detall a
l'apartat~\ref{sec:model:sgstm-manipulacio-esquema}.



\subsubsection{Relació sèrie temporal multiresolució}


Una sèrie temporal multiresolució és una relació de buffers i discs. A
cada parella buffer-disc l'anomenem subsèrie resolució. Així doncs, una
sèrie temporal multiresolució és un conjunt de subsèries resolució.

Com a conjunt de subsèries resolució, una sèrie temporal multiresolució
s'observa com una relació de grau sis a on la capçalera conté els
atributs
\begin{itemize}
\item sèrie temporal del buffer ($S_B$),
\item sèrie temporal del disc ($S_D$),
\item darrer instant de consolidació ($\tau$),
\item pas de consolidació ($\delta$),
\item màxim cardinal del disc ($k$),
\item i funció d'agregació d'atributs ($f$).
\end{itemize}

Una restricció habitual és que $\delta$ i $f$ no estiguin repetits; és
a dir que ($\delta$,$f$) són els atributs clau de la relació sèrie
temporal multiresolució.

Així doncs observada com a relació, tal com s'ha fet en el model de
SGST, podem escriure una sèrie temporal multiresolució
\begin{definition}[Representació sèrie temporal multiresolució]
  Sigui $M=\{R_0,\dotsc,R_d\}$ una sèrie temporal multiresolució a on
  $R_i =(B_i,D_i)$ són subsèries resolució amb domini de sèrie
  temporal per les sèries temporals, de $\bar{\R{}}$ pels temps, de
  $\N{}$ pel pas de consolidació i de funció per a la funció
  d'agregació d'atributs;
  \glsdispdef{not:sgstm:relaciomultiresolucio}{representada com a relació}
  s'escriu com $ M = ( S_B: \text{Sèrie Temporal}, S_D: \text{Sèrie
    Temporal}, \tau: \R{}, \delta: \R{}, k: \N{}, f: \text{funció}\},
  \{ \{ S_B: S_{B_0} , S_D : S_{D_0} , \tau : \tau_{B_0}, \delta :
  \delta_{B_0}, k: k_{D_0}, f : f_{B_0} \} , \dotsc, \{ S_B: S_{B_d} ,
  S_D : S_{D_d} , \tau : \tau_{B_d}, \delta : \delta_{B_d}, k:
  k_{D_d}, f : f_{B_d} \} \} )$.
\end{definition}

De la mateixa manera que per a les sèries temporals en el model de
SGST, una sèrie temporal multiresolució es pot escriure de manera
simplificada com al conjunt de tuples $M = \{ (S_{B_0}, S_{D_0} ,
\tau_{B_0}, \delta_{B_0}, k_{D_0}, f_{B_0} ), \dotsc, (S_{B_d},
S_{D_d} , \tau_{B_d}, \delta_{B_d}, k_{D_d}, f_{B_d} ) \}$.




\subsection{Exemples}

\begin{example} [Sèrie temporal multiresolució]
\label{ex:model:bdm1}%ATENCIÓ als canvis: les dades d'aquest exemple s'utilitzen en altres apartat.


Sèrie temporal multiresolució $M_1=\{R_0,R_1\}$ que té dues subsèries
resolució amb els paràmetres següents:
\begin{itemize}
\item La subsèrie resolució $R_0$ té un pas de consolidació de 5
  unitats de temps, una mida màxima de 4 mesures i una funció de
  consolidació de 'mitjana' de les mesures.
\item La subsèrie resolució $R_1$ té un pas de consolidació de 10
  unitats de temps, una mida màxima de 3 mesures i una funció de
  consolidació de 'mitjana' de les mesures.
\end{itemize}

\begin{figure}[tp]
\centering
%\usetikzlibrary{shapes,arrows,positioning}
\begin{tikzpicture}
 \tikzset{
        myarrow/.style={->, >=latex',  thick},
      }
      

  \node[rectangle,draw,minimum height=6cm,minimum width=9cm] (m) {};
  \draw[shift=( m.south west)]   
  node[above right] {base de dades multiresolució};


  %discmig
  \node (m.center) (discr1) {};

  %discr
  
  \node[ellipse,draw,minimum height=3.5cm,minimum width=2.5cm,alias=discr0] [left=of discr1] {};
  \node[above=0cm of discr0.north] {$R_0$};
  \node[below=0cm of discr0] {subsèrie resolució};

  \node[cylinder, draw, shape border rotate=90, aspect=0.25,alias=buffer0] [below=3mm of discr0.north] {mitjana};
  \node[circle, draw,alias=disc0,minimum width=1.2cm]  [above=3mm of discr0.south] {5} ;
  \draw [->] (disc0.center)++(.2:.2cm) arc(0:180:.2cm);
  \draw[myarrow] (buffer0.bottom) -- (disc0.north);

  \node[circle,minimum width=9mm] (d0boles) [below=0mm of disc0.center,anchor=center] {};
  \node[below=0mm of d0boles.north,anchor=center] {$\circ$};
  \node[below=0mm of d0boles.east,anchor=center] {$\circ$};
  \node[below=0mm of d0boles.south,anchor=center] {$\circ$};
  \node[below=0mm of d0boles.west,anchor=center] {$\circ$};


  %discrd

  \node[ellipse,draw,minimum height=3.5cm,minimum width=2.5cm,alias=discrd] [right=of discr1] {};
  \node[above=0cm of discrd] {$R_1$};
  \node[below=0cm of discrd] {subsèrie resolució};

  \node[cylinder, draw, shape border rotate=90, aspect=0.25,alias=bufferd] [below=3mm of discrd.north] {mitjana};
  \node[circle, draw,alias=discd,minimum width=1.2cm]  [above=3mm of discrd.south] {10} ;
  \draw [->] (discd.center)++(.3:.3cm) arc(0:180:.3cm);
  \draw[myarrow] (bufferd.bottom) -- (discd.north);

  \node[circle,minimum width=9mm] (d1boles) [below=0mm of discd.center,anchor=center] {};
  \node[below=0mm of d1boles.north,anchor=center] {$\circ$};
  \node[below=0mm of d1boles.south east,anchor=center] {$\circ$};
  \node[below=0mm of d1boles.south west,anchor=center] {$\circ$};


  %mesura 
  \node[above=1cm of m.north] (m0) {};

  \draw[myarrow] (m0) -- (m.north) 
  node[right,midway] {mesura};

  \draw[myarrow] (m.north) -- (buffer0);
  \draw[myarrow] (m.north) -- (bufferd);


\end{tikzpicture}
\caption{Arquitectura de SGSTM particular per l'\autoref{ex:model:bdm1}}
\label{fig:model:ex1}
\end{figure}

L'arquitectura de la base de dades que conté aquesta sèrie temporal
multiresolució es pot veure a la \autoref{fig:model:ex1}. 
 L'esquema
de multiresolució que correspon als instants de consolidació, des de 0
fins a 30, és el següent:
\begin{itemize}
\item La subsèrie resolució $R_0$ serà consolidada en els instants 5,
  10, 15, 20, 25 i 30.
\item La subsèrie resolució $R_1$ serà consolidada en els instants 10,
  20 i 30.
\end{itemize}


Iniciem la base de dades a l'instant de temps 0, instant en el qual la
sèrie temporal multiresolució és $M_1^0 = \{ ( \{\} , \{\} , 0 , 5 ,4
, \text{mitjana} ) , ( \{\} , \{\} , 0 , 10 ,3 , \text{mitjana} ) \}$;
és a dir amb les sèries temporals buides i els darrers instants de
consolidació iniciats a 0.




A continuació, afegim a la sèrie temporal multiresolució les mesures
de la sèrie temporal $S_1=\{
(1,0),(5,0),(8,0),(10,0),(14,0),(19,0),(22,0),(26,0),(29,0) \}$. Tots
els valors valen zero per tal de centrar la comprensió de l'exemple en
l'estructura de temps de consolidació; pel que fa a exemples
d'agregació de valors es poden veure amb més detall a la
secció~\ref{sec:model:agregador}.


Si consolidem la sèrie temporal multiresolució cada cop que sigui
consolidable, és a dir en els instants que marca l'esquema de
multiresolució, a l'instant 29 després d'haver inserit la darrera
mesura la sèrie temporal multiresolució és $M_1^{29} = \{ (
\{(26,0),(29,0)\},\{(10,0),(15,0),(20,0),(25,0)\}, 25 , 5 ,4 ,
\text{mitjana} ), ( \{(22,0),(26,0),(29,0)\}, \{(10,0),(20,0)\},
20 , 10 ,3 , \text{mitjana} ) \}$.  Aquesta sèrie temporal
multiresolució es mostra a la \autoref{fig:model:stm} en forma de
taula.


Es pot observar que als buffers hi ha emmagatzemades les mesures
pendents de consolidar per a cada subsèrie i als discs les darreres
mesures consolidades:
\begin{itemize}
\item Per a la subsèrie resolució $R_0$ hi ha pendent de consolidar
  l'interval de temps $[25,30]$ i al disc hi ha emmagatzemades les 4
  mesures màximes permeses; és a dir que la que s'havia consolidat a
  l'instant $5$ ja s'ha perdut.
\item Per a la subsèrie resolució $R_1$ hi ha pendent de consolidar
  l'interval de temps $[20,30]$ i al disc hi ha emmagatzemades 2
  mesures. El disc encara no ha arribat al cardinal màxim $k=3$ degut
  a que la base de dades s'ha iniciat a l'instant $0$ i la primera
  consolidació d'aquesta subsèrie ha estat a l'instant $10$.
\end{itemize}




\begin{figure}[tp]
  \centering
  \begin{tabular}{|c|c|c|c|c|c|}
    \multicolumn{2}{c}{$M_1^{29}$} \\ \hline
    $S_B$  & $S_D$ & $\tau$ & $\delta$ & $k$ & $f$ \\ \hline
      \begin{tabular}{|c|c|}
         \hline
         $t$  & $v$ \\ \hline
         26  & 0 \\
         29 & 0 \\\hline
       \end{tabular} & 
      \begin{tabular}{|c|c|}
         \hline
         $t$  & $v$ \\ \hline
         10  & 0 \\
         15  & 0 \\
         20 & 0 \\ 
         25 & 0 \\\hline
       \end{tabular} 
       & 25 & 5  & 4 & mitjana  \\ \hline
       \begin{tabular}{|c|c|}
         \hline
         $t$  & $v$ \\ \hline
         22  & 0 \\
         26  & 0 \\
         29 & 0 \\\hline
       \end{tabular} & 
      \begin{tabular}{|c|c|}
         \hline
         $t$  & $v$ \\ \hline
         10  & 0 \\
         20  & 0 \\\hline
       \end{tabular}  
       & 20 & 10 & 3 & mitjana  \\ \hline
  \end{tabular}
  \caption{Taula d'una sèrie temporal multiresolució}
  \label{fig:model:stm}
\end{figure}

\end{example}


\begin{example} [Sèrie temporal multiresolució amb vistes]

  En el model relacional de SGBD molt sovint s'utilitzen vistes per a
  agrupar informació de vàries relacions, per a mostrar-ne una part,
  etc. Una vista és una variable relació virtual derivada d'una
  expressió relacional \parencite{date13}. En aquest exemple mostrem
  la mateixa sèrie temporal multiresolució de
  l'\autoref{ex:model:bdm1} però organitzada amb forma de vistes
  relacionals.


  Sigui $M_2^{\text{series}}= ((S':\text{nom},S:\text{sèrie
    temporal}),\{ (S_{B1},\{(26,0),(29,0)\}),
  (S_{B2},\{(22,0),(26,0),(29,0)\}),
  (S_{D1},\{(10,0),(15,0),(20,0),(25,0)\}),
  (S_{D2},\{(10,0),(20,0)\} )\})$ una relació de sèries
  temporals i noms, i $M_2'=
  ((S'_B:\text{nom},S'_D:\text{nom},\tau:\R,\delta:\R,k:\N,f:\text{funció}
  ),\{ (S_{B1},S_{D1},25 ,5 ,4 ,\text{mitjana} ), ( S_{B2},S_{D2},20 ,
  10 ,3 , \text{mitjana} ) \})$ una sèrie temporal multiresolució amb
  noms als atributs de sèries temporals; la vista de la sèrie temporal
  multiresolució es mostra a la \autoref{fig:model:stm:vistes} i es
  defineix com a
  \begin{align*}
    \text{vista } M_2 =& (M_2' \join ( M_2^{\text{series}} \rename S' \as S_B', S \as S_B )) \\
    &\join ( M_2^{\text{series}} \rename S' \as S_D', S \as S_D )
    \{\text{all but } S_B',S_D' \}
  \end{align*}




  \begin{figure}[tp]
    \centering
    \begin{tabular}{|c|c|c|c|c|c|}
      \multicolumn{2}{c}{$M'_2$} \\ \hline
      $S'_B$  & $S'_D$ & $\tau$ & $\delta$ & $k$ & $f$ \\ \hline
      $S_{B1}$ & $S_{D1}$ & 25 & 5  & 4 & mitjana  \\
      $S_{B2}$ & $S_{D2}$ & 20 & 10 & 3 & mitjana  \\ \hline
    \end{tabular}\qquad
    \begin{tabular}{|c|c|c|}
      \multicolumn{3}{c}{$M^{\text{series}}_{2}$} \\ \hline
      \multirow{2}{*}{$S'$}  &  \multicolumn{2}{c|}{$S$} \\ \cline{2-3}
      & $t$      & $v$  \\ \hline
      \multirow{2}{*}{$S_{B1}$} 
      & 26 & 0 \\ 
      & 29 & 0 \\ \hline
      \multirow{3}{*}{$S_{B2}$} 
      & 22 & 0 \\ 
      & 26 & 0 \\ 
      & 29 & 0 \\ \hline
      \multirow{4}{*}{$S_{D1}$} 
      & 10 & 0 \\ 
      & 15 & 0 \\ 
      & 20 & 0 \\ 
      & 25 & 0 \\ \hline
      \multirow{2}{*}{$S_{D2}$} 
      & 10 & 0 \\ 
      & 20 & 0 \\ \hline
    \end{tabular}
    \caption{Taula d'una sèrie temporal multiresolució amb vistes
      relacionals}
    \label{fig:model:stm:vistes}
  \end{figure}


  D'aquesta manera $M_2$ té els mateixos valors que la $M_1$ definida
  a l'exemple anterior; observant només el resultat de $M_2$ no es pot
  distingir que és una vista. Així doncs, les vistes ens permeten
  organitzar una sèrie temporal multiresolució de forma més còmoda i,
  a més tal com es descriu a \textcite{date13}, mantenint que totes
  les operacions i propietats que són d'aplicació a les relacions ho
  són també a les seves vistes.


%S'observa que per tal de complir amb les propietats de les relacions, totes les sèries temporals dels buffers han de ser del mateix tipus, és a dir tenir la mateixa capçalera. El mateix succeeix amb les sèries temporals dels discs. (Vegeu els exemples de la secció \ref{par:model:exemple-relvalues} sobre valors relació). No obstant les mesures que entren a la base de dades provenen de la mateixa sèrie temporal i per tant les sèries temporals emmagatzemades sempre seran del mateix tipus.



\end{example}



\begin{example} [Sèrie temporal multiresolució amb desfasaments]
\label{ex:model:bdm-desfasaments}

A l'\autoref{ex:model:bdm1} s'ha mostrat una sèrie temporal
multiresolució en la que la consolidació de les dues subsèries obeeix
a la mateixa funció d'agregador d'atributs $f$. En aquest exemple
treballem amb els mateixos valors que a l'\autoref{ex:model:bdm1} però
ara canviem la funció de la segona subsèrie resolució per un agregador
amb desfasament; és a dir que cada cop que consolida retorna una
mesura amb un retard d'una certa durada. Aquest nou agregador que
anomenem \emph{mitjanad5} també fa la mitjana però amb un desfasament
de $5$ unitat de temps, a l'apartat
\ref{sec:model:sgstm-manipulacio-esquema} es defineix amb més precisió
aquest concepte de desfasament.


Seguint el mateix procediment que a l'\autoref{ex:model:bdm1}, a
l'instant 29 després d'haver inserit la darrera mesura la sèrie
temporal multiresolució és $M_3^{29} = \{ ( \{(26,0),(29,0)\} ,
\{(10,0),(15,0),(20,0),(25,0)\} , 25 , 5 ,4 , \text{mitjana} ) , (
\{(19,0),(22,0),(26,0),(29,0)\} , \{(5,0),(15,0)\} , 20 , 10 ,3 ,
\text{mitjanad5} ) \}$.  Aquesta sèrie temporal multiresolució es
mostra a la \autoref{fig:model:stm:desfasaments} en forma de taula.


\begin{figure}[tp]
  \centering
  \begin{tabular}{|c|c|c|c|c|c|}
    \multicolumn{2}{c}{$M_3^{29}$} \\ \hline
    $S_B$  & $S_D$ & $\tau$ & $\delta$ & $k$ & $f$ \\ \hline
      \begin{tabular}{|c|c|}
         \hline
         $t$  & $v$ \\ \hline
         26  & 0 \\
         29 & 0 \\\hline
       \end{tabular} & 
      \begin{tabular}{|c|c|}
         \hline
         $t$  & $v$ \\ \hline
         10  & 0 \\
         15  & 0 \\
         20 & 0 \\ 
         25 & 0 \\\hline
       \end{tabular} 
       & 25 & 5  & 4 & mitjana  \\ \hline
       \begin{tabular}{|c|c|}
         \hline
         $t$  & $v$ \\ \hline
         19  & 0 \\
         22  & 0 \\
         26  & 0 \\
         29 & 0 \\\hline
       \end{tabular} & 
      \begin{tabular}{|c|c|}
         \hline
         $t$  & $v$ \\ \hline
          5  & 0 \\
         15  & 0 \\\hline
       \end{tabular}  
       & 20 & 10 & 3 & mitjanad5  \\ \hline
  \end{tabular}
  \caption{Taula d'una sèrie temporal multiresolució amb desfasaments}
  \label{fig:model:stm:desfasaments}
\end{figure}



Així doncs, mentre que l'esquema de multiresolució segueix sent el
mateix pel que fa als instants de consolidació, els instants de temps
de la sèrie temporal emmagatzemada a la subsèrie resolució $R_1$ tenen
un retard de $5$ unitats de temps.  Per una banda, es pot observar a
la $S_{B2}$ que el buffer ara és 5 unitats més gran i emmagatzema
mesures de l'interval $[15,30]$. Per altra banda, es pot observar a la
$S_{D2}$ que els instants emmagatzemats són $5$ i $15$ corresponents
als instants de consolidació $10$ i $20$. Pel que fa a la resta de
valors, no han variat respecte de l'\autoref{ex:model:bdm1}.



\end{example}


%%% Local Variables: 
%%% mode: latex
%%% TeX-master: "main"
%%% End: 

% LocalWords:  buffers multiresolució agregador l'agregador subsèries
% LocalWords:  d'agregador

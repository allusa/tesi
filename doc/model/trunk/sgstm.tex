\chapter{Model SGSTM}


En aquest capítol es defineixen els operadors que permeten modelar el comportament i la manipulació de les dades.



\section{Model estructural de dades}

Una MTSDB és una relació de buffers amb discs. 


\begin{figure}[tp]
\centering
\begin{tikzpicture}
 \tikzset{
        myarrow/.style={->, >=latex',  thick},
      }
      

  \node[rectangle,draw,minimum height=6cm,minimum width=9cm] (m) {};
  \draw[shift=( m.south west)]   
  node[above right] {base de dades multiresolució};


  %discmig
  \node (m.center) (discr1) {...};

  %discr
  
  \node[ellipse,draw,minimum height=3.5cm,minimum width=2.5cm,alias=discr0] [left=of discr1] {};
  \node[above=0cm of discr0.north] {R0};
  \node[below=0cm of discr0] {disc resolució};

  \node[cylinder, draw, shape border rotate=90, aspect=0.25,alias=buffer0] [below=3mm of discr0.north] {buffer};
  \node[circle, draw,alias=disc0]  [above=3mm of discr0.south] {disc} ;
  \draw [->] (disc0.center)++(.4:.4cm) arc(0:180:.4cm);
  \draw[myarrow] (buffer0.bottom) -- (disc0.north);


  %discrd

  \node[ellipse,draw,minimum height=3.5cm,minimum width=2.5cm,alias=discrd] [right=of discr1] {};
  \node[above=0cm of discrd] {Rd};
  \node[below=0cm of discrd] {disc resolució};

  \node[cylinder, draw, shape border rotate=90, aspect=0.25,alias=bufferd] [below=3mm of discrd.north] {buffer};
  \node[circle, draw,alias=discd]  [above=3mm of discrd.south] {disc} ;
  \draw [->] (discd.center)++(.4:.4cm) arc(0:180:.4cm);
  \draw[myarrow] (bufferd.bottom) -- (discd.north);



  %mesura 
  \node[above=1cm of m.north] (m0) {};

  \draw[myarrow] (m0) -- (m.north) 
  node[right,midway] {mesura};

  \draw[myarrow] (m.north) -- (buffer0);
  \draw[myarrow] (m.north) -- (bufferd);
  \draw[myarrow] (m.north) -- (discr1);

\end{tikzpicture}
\caption{Arquitectura del model SGSTM}
\label{fig:model:bdstm}
\end{figure}


\subsection{Buffer}\label{sec:model:buffer}\todo{falta parlar de regularitat de ST}\todo{falta parlar de representació de ST}

Un buffer és un contenidor d'una sèrie temporal, regular o no regular, que mitjançant una funció permet regularitzar aquesta sèrie temporal amb un període de mostreig constant. A l'acció de regularitzar un interval d'una sèrie temporal l'anomenarem consolidació, al període de mostreig contant l'anomenarem pas de consolidació i a la funció de regularització l'anomenarem agregador d'atributs.

\begin{definition}[Buffer]
  Definim \emph{buffer} com el tuple $(S,\tau,\delta,f)$, en el que
  $S$ és una sèrie temporal, $\tau$ és el darrer instant de temps de
  consolidació, $\delta$ és la durada del pas de consolidació i $f$ és
  un agregador d'atributs.
\end{definition}

La consolidació d'una sèrie temporal s'inicia en un instant de temps concret i té lloc a cada pas de consolidació. Amb la finalitat d'establir els intervals de consolidació de la sèrie temporal, es defineix un buffer inicial.

\begin{definition}\label{def:model:buffer_buit}
  Definim buffer inicial o buffer buit com el buffer $B_{\emptyset} =
  (\emptyset,t_0, \delta_0, f)$, el qual
  conté una sèrie temporal buida, l'instant de temps inicial de
  consolidació, una durada que indica el pas de consolidació i un
  agregador d'atributs.
\end{definition}

A partir del buffer buit es poden conèixer tots els instants de temps de consolidació del buffer, els quals seran $t_0+k\delta, k\in\mathbb{N}$. 



\subsection{Disc}\label{sec:model:disc}

Un disc és un contenidor d'una sèrie temporal regular amb un nombre acotat de mesures. En arribar al nombre màxim de mesures permeses, cada cop que s'afegeix una mesura nova s'elimina la mesura mínima de la sèrie temporal.
Així doncs, un disc és semblant a una cua \emph{First In First Out} (FIFO), a on el primer d'arribar és el primer de sortir.  

\begin{definition}[Disc]
  Definim \emph{disc} com el tuple $(S,k)$, en el que $S$
  és una sèrie temporal i $k\in\mathbb{N}$ és el cardinal màxim de $S$.
\end{definition}

A l'inici, un disc no conté mesures però cal que estigui caracteritzat pel cardinal màxim. Amb aquesta finalitat es defineix un disc inicial.

\begin{definition}\label{def:model:disc_buit}
  Definim disc inicial o disc buit com el disc $D_{\emptyset} =
  (\emptyset,k)$, el qual conté una sèrie temporal buida i el cardinal
  màxim que podrà prendre $S$.
\end{definition}




\subsection{Disc resolució}\label{sec:model:disc_multiresolucio}

Un disc resolució és un disc amb buffer. En el buffer hi ha la part d'una sèrie temporal a regularitzar i en el disc hi ha l'altra part ja regularitzada, amb un nombre acotat de mesures. 

\begin{definition}[Disc resolució]
  Definim \emph{disc resolució} com el tuple $(B,D)$, en el que $B$
  és un buffer i $D$ és un disc.
\end{definition}
 
La definició de buffer buit (def.~\ref{def:model:buffer_buit}) i de disc buit (def.~\ref{def:model:disc_buit}) indueixen a una definició de disc resolució buit. 

\begin{definition}\label{def:model:disc_resolucio_buit}
  Definim disc resolució buit com el disc resolució $R_{\emptyset}
  = (B_{\emptyset},D_{\emptyset})$, el qual conté un buffer buit i un
  disc buit.
\end{definition}




\subsection{Base de dades multiresolució}\label{sec:model:bdstm}

Una base de dades multiresolució és un conjunt de discs resolució que comparteixen l'entrada de mesures, les quals provenen d'una mateixa sèrie temporal. La sèrie temporal queda regularitzada i distribuïda  en els diferents discs resolució amb resolucions diferents, tal com s'ha vist a la \autoref{fig:model:bdstm}


\begin{definition}[Base de dades multiresolució]
  Definim \emph{base de dades multiresolució} com el conjunt de discs resolució
  $M=\{R_0,\dotsc,R_d\}$.
\end{definition}

A partir de la definició de disc resolució buit (def.~\ref{def:model:disc_resolucio_buit}) és defineix la base de dades multiresolució buida. 
 
\begin{definition}\label{def:model:bd_multiresolucio_buit}
  Definim base de dades multiresolució buida com el conjunt de discs
  resolució buits
  $M_{\emptyset}=\{R_{0_{\emptyset}},\dotsc,R_{d_{\emptyset}\}}$.
\end{definition}

Normalment, en una base de dades multiresolució no hi ha dos discs
resolució amb la mateixa informació. És a dir, donats dos discs
resolució $R_a = (B_a, D_a)$ i $R_b = (B_b, D_b)$, 
els seus respectius buffers 
$B_a=(S_a,\tau_a,\delta_a,f_a)$ i
$B_b=(S_b,\tau_b,\delta_b,f_b)$ no tenen el mateix interval de
consolidació i agregador d'atributs: 
$\delta_a \neq \delta_b \wedge f_a \neq f_b$.









\subsection{Exemples}

\paragraph{Exemple 1}


S'observa que per tal de complir amb les propietats de les relacions, totes les sèries temporals dels buffers han de ser del mateix tipus, és a dir tenir la mateixa capçalera. El mateix succeeix amb les sèries temporals dels discs. (Vegeu els exemples de la secció \ref{par:model:exemple-relvalues} s'obre valors relació).

\begin{figure}[tp]
  \centering
  \begin{tabular}{|c|c|c|c|c|c|}
    \multicolumn{2}{c}{$M_1$} \\ \hline
    $S_B$  & $S_D$ & $\tau$ & $\delta$ & $k$ & $f$ \\ \hline
    $S_{B1}$ & $S_{D1}$ & 0 & 5  & 2 & mitjana  \\
    $S_{B2}$ & $S_{D2}$ & 0 & 10 & 4 & mitjana  \\ \hline
  \end{tabular}
  \caption{Taula d'una mtsdb independent}
  \label{fig:model:mtsdb:independent}
\end{figure}



\paragraph{Exemple 2}\todo{Compte! que no existeix el tipus relvar i potser no es pot definir una relació que contingui relvars (apuntadors). Cal pensar amb l'exemple 4 suprimit del model dels SGST}

\begin{figure}[tp]
  \centering
  \begin{tabular}{|c|c|c|c|c|c|}
    \multicolumn{2}{c}{$M_2$} \\ \hline
    $S_B$  & $S_D$ & $\tau$ & $\delta$ & $k$ & $f$ \\ \hline
    $S_{B1}$ & $S_{D1}$ & 0 & 5  & 2 & mitjana  \\
    $S_{D1}$ & $S_{D2}$ & 0 & 10 & 4 & mitjana  \\ \hline
  \end{tabular}
  \caption{Taula d'una mtsdb en cadena}
  \label{fig:model:mtsdb:cadena}
\end{figure}










\section{Model d'operacions}

En aquesta secció es defineixen els operadors que permeten modelar el
comportament i la manipulació de les dades en el model de SGSTM.

Per a treballar amb les sèries temporals multiresolució s'utilitzen
els conceptes descrits al model d'operacions de SGST. El model de
SGSTM es defineix a partir del model de SGST i per tant les operacions
dels SGSTM també hi estan basades. Tot i així cal tenir en compte dues
particularitats.

Per una banda, el model de SGSTM treballa amb sèries temporals
multiresolució. Així, es defineixen operadors que permeten extreure
les sèries temporals emmagatzemades en aquestes bases de dades amb
l'objectiu d'aplicar-hi posteriorment els operadors dels SGST.

Per altra banda, el model de SGSTM té una estructura específica que
requereix ser manipulada coherentment. Així, es defineixen operadors
que saben treballar amb aquesta estructura agrupats en dos grups.  El
primer grup són els operadors requerits pel model estructural;
operadors que són inseparables de l'estructura i són utilitzats en el
procés d'emmagatzemar les mesures. El segon grup són els operadors
necessaris per a manipular l'estructura; és a dir operadors que
permeten fer canvis en l'esquema de la base de dades o consultar
paràmetres de l'esquema actual.


En el disseny del model d'operacions següent es distingeixen tres
grups d'operadors segons els casos anteriors:

\begin{itemize}
\item Estructurals: operadors requerits pel model estructural.
\item Manipulació de l'esquema: operadors per a manipular l'esquema de
  multiresolució.
\item Consultes: operadors per a extreure les sèries temporals
  emmagatzemades.
\end{itemize}





\subsection{Estructurals}
\label{sec:model:sgstm-estructurals}

En el model estructural de SGSTM hem definit les sèries temporals
multiresolució com un conjunt de subsèries resolució a les quals es
van afegint mesures compactant-les i consolidant-les. En aquest
apartat definim els operadors que permeten inserir mesures noves i
consolidar-les al seu lloc corresponent en l'estructura.

A continuació es descriuen els operadors associats a cada objecte del
model de SGSTM.


\subsubsection{Buffer}

Els buffers reben les noves mesures i les consoliden a cada instant de
consolidació. Així, tenen dos operadors associats: un per afegir noves
mesures al buffer i un altre per consolidar-les.


L'operació d'afegir una mesura al buffer consisteix en afegir-la a la
sèrie temporal pendent de consolidar.
\begin{definition}[Afegeix mesura al buffer]
  Sigui $B=(S,\tau,\delta,f)$ un buffer i $m=(t,v)$ una mesura, la
  inserció de la mesura al buffer $\addB(B,m)$ és un buffer
  $B'=(S',\tau,\delta,f)$ amb la mesura afegida a la sèrie temporal
  del buffer: $\addB(B,m) = (S',\tau,\delta,f)$ a on $S'=S\cup \{m\}$.
\end{definition}


L'operació de consolidació d'un buffer consisteix en compactar les
mesures segons els intervals de consolidació i la funció d'agregació i
a suprimir la part ja consolidada de la sèrie temporal.  Així doncs,
la consolidació d'un buffer per cada interval de temps
$i=[\tau,\tau+\delta]$ dóna com a resultat una mesura calculada en
funció de l'agregador d'atributs i un nou buffer amb la sèrie temporal
reduïda.
\begin{definition}[Consolida el buffer]
  Sigui $B=(S,\tau,\delta,f)$ un buffer, la consolidació del buffer
  $\consB(B)$ en l'interval de temps $[\tau,\tau+\delta]$ és un
  buffer $B'=(S',\tau',\delta,f)$, amb el nou instant de consolidació,
  i la mesura $m'=(t',v')$ resultant de la consolidació: $\consB(B) =
  (S',\tau+\delta',\delta,f) \times m'$ a on
  $m'=f(S,[\tau,\tau+\delta])$ i $S'$ és el resultat d'eliminar les
  dades històriques que no es necessiten més. 

  \emph{Nota}: En el model teòric es pot donar $S'=S$ tot i que a les
  implementacions normalment caldrà eliminar les dades ja no
  necessàries per no ocupar espai amb per exemple $S'=
  S[\tau+\delta,+\infty]$.
\end{definition}

De manera simplificada, hem definit que cada consolidació només
s'aplica a l'interval de consolidació actual; així la consolidació
total del buffer és l'aplicació successiva de l'operació de
consolidació.

Aquesta consolidació successiva requereix que les mesures s'insereixen
al buffer ordenades en el temps, sinó un cop duta a terme la
consolidació les mesures inserides desordenades poden no ser tingudes
en compte. Si es duu a terme aquesta inserció ordenada, aleshores un
buffer té l'estat de consolidable quan el temps d'una mesura de la
sèrie temporal és més gran que el següent instant de temps de
consolidació del buffer.
\begin{definition}[Buffer consolidable]\label{def:model:buffer_consolidable}
  Sigui $B=(S,\tau,\delta,f)$ un buffer, definim que $B$ és
  consolidable si i només si $T(m) \geq \tau+\delta$ a on $m=\sup(S)$
  és la mesura suprema de la sèrie temporal del buffer.
\end{definition}



\subsubsection{Disc}

Els discs reben les mesures consolidades per a emmagatzemar-les de
forma acotada. Així, tenen un operador associat que afegeix noves
mesures al disc mantenint sota control el seu cardinal.


L'operació d'afegir una mesura al disc consisteix en afegir-la a la
sèrie temporal i a eliminar la mesura mínima d'aquesta si se supera el
cardinal permès.
\begin{definition}[Afegeix mesura al disc]
  Sigui $D=(S,k)$ un disc i $m=(t,v)$ una mesura, la inserció de la
  mesura al disc $\addD(D,m)$ és un disc $D'=(S',k)$ amb la mesura
  afegida a la sèrie temporal del disc mantenint el cardinal màxim:
  $\addD(S,m) = (S',k)$ a on $S'=
  \begin{cases}
      S\cup\{m\} &\text{si }  |S|<k\\
      (S-\{\min(S)\}) \cup \{m\} &\text{altrament}
    \end{cases}$.
\end{definition}




\subsubsection{Subsèrie resolució}

Les subsèries resolució són l'aparellament d'un buffer amb un disc.
Així tenen dos operadors associats, els quals treballen amb els
operadors del buffer i del disc : un per afegir una mesura al buffer i
un altre per consolidar el buffer i afegir la mesura resultant al disc


L'operació d'afegir una mesura a la subsèrie resolució consisteix en
afegir-la al buffer.
\begin{definition}[Afegeix mesura a la subsèrie resolució]
  Sigui $R=(B,D)$ una subsèrie resolució i $m=(t,v)$ una mesura, la
  inserció de la mesura a la subsèrie resolució $\addR(R,m)$ és una
  subsèrie resolució $R'=(B',D)$ amb la mesura afegida al buffer:
  $\addR(R,m) = (B',D)$ a on $B'=\addB=(B,m)$.
\end{definition}


L'operació de consolidar una subsèrie resolució consisteix en calcular
una mesura de consolidació del buffer, en l'interval de consolidació
actual, i desar-la al disc. Una subsèrie resolució és consolidable
quan ho és el seu buffer.
\begin{definition}[Consolida la subsèrie resolució]
  Sigui $R=(B,D)$ una subsèrie resolució, la consolidació de la
  subsèrie resolució $\consR(R)$ és una subsèrie resolució
  $R'=(B',D')$ a on $(B',m') = \consB(B)$ i $D'=\addD(D,m')$.
\end{definition}





\subsubsection{Sèrie temporal multiresolució}

Les sèries temporals multiresolució són un conjunt de subsèries
resolució. Així tenen dos operadors per a treballar globalment amb
totes les subèries que contingui: un per a afegir una mesura a cada
subsèrie i un altre per a consolidar cadascuna de les subsèries.


L'operació d'afegir una mesura a la sèrie temporal multiresolució
consisteix en afegir-la a cadascuna de les subsèries resolució.
\begin{definition}[Afegeix mesura a la sèrie temporal multiresolució]
  Sigui $M=\{R_0,\dotsc,R_d\}$ una sèrie temporal multiresolució i
  $m=(t,v)$ una mesura, la inserció de la mesura a la sèrie temporal
  multiresolució $\addM(M,m)$ és una sèrie temporal multiresolució
  $M'=\{R_0',\dotsc,R_d'\}$ amb la mesura afegida a cada subsèrie
  resolució: $\addM(M,m) = \{ \forall R_i\in M: \addR(R_i,m) \}$.
\end{definition}


L'operació de consolidar una sèrie temporal multiresolució consisteix
en consolidar cadascuna de les subsèries resolució que siguin
consolidables.

\begin{definition}[Consolida la sèrie temporal multiresolució]
  Sigui $M=\{R_0,\dotsc,R_d\}$ una sèrie temporal multiresolució, la
  consolidació de la sèrie temporal multiresolució $\consM(M)$ és una
  sèrie temporal multiresolució $M'=\{R_0',\dotsc,R_d'\}$ que
  consolida les subsèries resolució consolidables:
  \[
  \consM(M) = \big\{ \forall R_i\in M:
  \begin{cases}
    \consR(R_i) & \text{si } R_i \text{ és consolidable} \\
    R_i & \text{altrament}
  \end{cases}\big\}
  \].
\end{definition}





\subsection{Manipulació de l'esquema}

El model de SGSTM associa a cada sèrie temporal un esquema de
multiresolució. En aquest apartat definim els operadors que permeten
consultar i manipular aquest esquema de multiresolució de forma
coherent amb el model de SGSTM.

L'esquema de multiresolució de cada sèrie temporal multiresolució
consisteix en quatre paràmetres variables: el darrer instant de
consolidació ($\tau$), el pas de consolidació ($\delta$), el cardinal
màxim ($k$) i la funció d'agregació d'atributs ($f$).

Així doncs, quan es manipula una base de dades multiresolució cal
conservar o tractar adequadament aquest esquema de multiresolució.  A
continuació es descriuen operadors per a poder estudiar aquest
esquema, operadors per a canviar-lo i operadors per a unir o ajuntar
dos esquemes.



\subsubsection{Propietats de l'esquema}

La configuració dels quatre paràmetres variables de l'esquema de
multiresolució confereix una sèrie de propietats als objectes d'una
base de dades multiresolució. A continuació definim algunes propietats
que es poden estudiar a partir d'un esquema de multiresolució.




Una propietat de l'esquema de multiresolució és el cronograma que se'n deriva. Així, donat un esquema de multiresolució podem dibuixar la situació relativa en el temps que prendran les mesures.
\todo{fer dibuix aclaridor}


\todo{}

  
Els paràmetres $k$, $\delta$ i $f$ d'una subsèrie resolució són fixats
per l'esquema, mentre que el paràmetre $\tau$ és fixat a un valor
inicial i va sent canviat per l'operació de consolidar. Així doncs,
les propietats que impliquin a $\tau$ dependran de l'instant temporal
en que es faci la consulta i les que no l'impliquin seran fixes per a
cada esquema.



La primera propietat que observem és el lapse temporal d'una subsèrie
resolució; és a dir una mesura de la mida temporal que ocupa la
subsèrie temporal emmagatzemada en el seu disc.
\begin{definition}[Lapse de la subsèrie resolució] %angl. span
  Sigui $R=(S_B,S_D,\tau,\delta,k,f)$ una subsèrie resolució, el seu
  lapse $\text{lapseR}(R)$ és una durada de temps $t^d$ que mesura la
  mida de l'interval que ocupa el disc: $\text{lapseR}(R) = k\delta$.
\end{definition}

\todo{} Si definim l'interval del lapse com a $[\tau - k\delta, \tau]$
i l'interval temporal real de la sèrie temporal del disc com a
$[\min(S_D),\max(S_D)]$, aleshores normalment en el cas que $\max(S_D)=\tau$ es compleix que $\min(S_D)=\tau - (k-1)\delta $: fixem-nos que això pot no complir-se (perquè la sèrie temporal no sigui regular, perquè $\tau$ i $\max(S_D)$ no coincideixin, etc.)
 fixem-nos que el lapse pot no coincidir amb
l'interval temporal real de la sèrie temporal del disc,



Com en diem a la certa variació que es correspon a $t_{actual}-\tau$?





Així doncs podem dir que un disc resolució ens cobreix un span temporal de  $\text{duradaR}(R)$. Es posiciona absolutament des de $\tau$ enrere (si no hi ha iffset). Això vol dir que tindrem un temps recent no cobert de $t_{actual}-\tau$ (anomenar-lo desfasament o retard, en angl. offset), el qual habitualment com a màxim serà de $\delta$ ($t_{actual}-\tau \leq \delta$) i com a mínim de zero si consolidem la multiresolució periòdicament.






Quin és el període de la sèrie temporal d'un disc?
  \begin{gather*}
    \text{periodeR}: R \longrightarrow \delta':\\
    \delta'=
    \begin{cases}
      \delta_r &\text{si } S_D \text{ regular o temps real amb } \delta_r\\
      \delta &\text{altrament}
    \end{cases}
  \end{gather*}


Quin disc conté més resolució?
  \begin{gather*}
    \text{maxR}: R_1 \times R_2 \longrightarrow R_i' | d_i = \max(d_1,d_2) : \\
    d_1 = periodeR(R_1), d_2 = periodeR(R_1)
  \end{gather*}



\subsubsection{Canvis en l'esquema d'un disc resolució}


Redueix o augmenta la mida d'un disc
  \begin{gather*}
    \text{CanviaK}: R \times k' \longrightarrow R': \\
    R' = (S_B,S'_D,\delta,\tau,k',f) : \\
    k_d = |S_D|:\\
    S'_D = \begin{cases}
      S_D         & \text{si } k' \geq k_d   \\
      treuN(S_D,k_d-k')    & \text{altrament}
    \end{cases}, \\
    treuN: S \times n \mapsto S'=  
    \begin{cases}
      S                & \text{si } n=0   \\
      treuN(S - \{\min(S)\},n-1)  & \text{altrament}
    \end{cases}
\end{gather*}


Redueix o augmenta el pas de consolidació d'un disc (sense canviar la sèrie temporal emmagatzemada; ja s'anirà canviant quan es consolidin noves mesures)
  \begin{gather*}
    \text{Canvia}\delta: R \times \delta' \longrightarrow R': \\
    R' = (S_B,S_D,\delta',\tau,k,f)
  \end{gather*}


Redueix o augmenta alhora el pas de consolidació i la mida d'un disc
  \begin{gather*}
    \text{CanviaK}\delta: R \times k' \times \delta' \longrightarrow R': \\
    R' = (S_B,S_D',\delta',\tau,k',f): \\    
    t = \{ \tau-n\delta' | n\in\mathbb{N},n<k' \} \\
    S_D' = \text{seleccioResolucio}(S_D,t)
  \end{gather*}




Afegeix un multivalor per a emmagatzemar sèries temporals multivaluades
  \begin{gather*}
    \text{afegeixMultivalor}: R \longrightarrow R': \\
    R' = (S'_{B},S'_{D},\delta,\tau,k,f): \\
    S'_{B} = \text{map}(S_B,(t,v)\mapsto(t,v,\infty)), \\
    S'_{D} = \text{map}(S_D,(t,v)\mapsto(t,v,\infty))
  \end{gather*}\todo{s'ha de fer amb extends i les v han de poder tenir nom}




\subsubsection{Canvis en l'esquema d'una sèrie temporal multiresolució}

Aplicació de map a una BDM.
\begin{gather*}
  \text{map}: M \times f \longrightarrow M' = \{ f(R_0), \dotsc, f(R_k) \} :\\
   \text{a on } f: R_a \mapsto R'
 \end{gather*}
 
Aplicació de fold a una BDM
\begin{gather*}
  \text{fold}: M \times M_i \times f \longrightarrow M' :\\
  = f(\dots(f(f(f(M_i,R_0),R_1),R_2)\dots),R_k), \\
   \text{a on } f: M_a \times R_b \mapsto M'
\end{gather*}





\subsubsection{Treball amb dos esquemes}


\paragraph{Unió de multiresolució}

Cas típic:
Mesuro una sèrie temporal. Durant un temps emmagatzemo valors a una
base de dades i després els emmagatzemo a una altra base de dades. Al final vull unir les dues bases de dades.


Unió de dos discs resolució que tenen el mateix $\delta$ i $f$ és un
disc resolució que conté la unió de les sèries de cada un.  Sigui
$R_1^*=(S_{B1},S_{D1},\delta,\tau_1,k_1,f)$ i
$R_2^*=(S_{B2},S_{D2},\delta,\tau_2,k_2,f)$
  \begin{gather*}
    \text{unioR}: R_1^* \times R_2^* \longrightarrow R': \\
    R' = (S'_B,S'_D,\delta,\max(\tau_1,\tau_2),k_1+k_2,f), \\
    S_{Di}, S_{Dj} | R_i = \text{maxR}(R_1,R_2), j \neq i:  \\
    S'_B = \text{unio}(S_{B1},S_{B2})\\
    S'_D = \text{unio}^r(S_{Di},S_{Dj})
\end{gather*}

També es pot unir dos discs resolució amb diferent $\delta$ i $f$,
però llavors s'ha de determinar quins són els $\delta'$ i $f'$
resultants.


Com a relacions multiresolució, dues bases de dades multiresolució es
poden unir si no intersecten en les claus $(\delta,f)$.  En cas que
intersectin, podem definir la unió multiresolució com la unió que sap unir els discs resolució repetits.

\begin{gather*}
    \text{UnioM}: M_1 \times M_2 \longrightarrow M': \\
    K_1 = \{(delta_1,f_1) \in M_1\},K_2 = \{(delta_2,f_2) \in M_2\}, \\
    K_a = K_1 \cap K_2, K_u =  (K_1 \cup K_2) - K_a : \\
    M_{u1}'= seleccio(M_1, (delta,f) \in K_u)\\
    M_{u2}'= seleccio(M_2, (delta,f) \in K_u)\\
    M_a = \{\forall R_1\in M_1,R_2\in M_2: unioR(R_1,R_2) |
       (delta_1,f_1) = (delta_2,f_2) \} \\
    M' =  M_{a} \cup  M'_{1}  \cup  M'_{2}     
\end{gather*}






\paragraph{Junció de multiresolució}

\todo{canviar fusió per junció}

Tinc una sèrie temporal en una base de dades, i una altra sèrie temporal en una altra base de dades. Vull emmagatzemar-les totes dues en una mateixa base de dades amb una sèrie temporal multivaluada.


Fusió de dos discs resolució que tenen el mateix $\delta$ i $f$.
Sigui $R_1^*=(S_{B1},S_{D1},\delta,\tau_1,k_1,f)$ i
$R_2^*=(S_{B2},S_{D2},\delta,\tau_2,k_2,f)$
  \begin{gather*}
    \text{FusioR}: R_1^* \times R_2^* \longrightarrow R': \\
    R' = (S'_B,S'_D,\delta,\max(\tau_1,\tau_2),k_1+k_2,f), \\
    S'_B = \text{fusio}^r(S_{B1},S_{B2})\\
    S'_D = \text{fusio}^r(S_{D1},S_{D2})
\end{gather*}


Fusió de dues bases de dades multiresolució
\begin{gather*}
    \text{FusioM}: M_1 \times M_2 \longrightarrow M': \\
    K_1 = \{(delta_1,f_1) \in M_1\},K_2 = \{(delta_2,f_2) \in M_2\}, \\
    K_a = K_1 \cap K_2, K_u =  (K_1 \cup K_2) - K_a : \\
    M_{u1} =\text{afegeixMultivalor}(seleccio(M_1, (delta,f) \in K_u))\\
    M_{u2} =\text{afegeixMultivalor}^{v0} (seleccio(M_2, (delta,f) \in K_u))\\
    M_1 = seleccio(M_1, (delta,f) \in K_f) \\
    M_2 = seleccio(M_2, (delta,f) \in K_f) \\
    M_a = \{\forall R_1\in M_1,R_2\in M_2: fusioR(R_1,R_2) |
       (delta_1,f_1) = (delta_2,f_2) \} \\
    M' =  M_{a} \cup  M_{u1}  \cup  M_{u2}     
\end{gather*}






\subsection{Consultes}



Operadors per fer consultes sobre una base de dades multiresolució. 

Mostrar com s'utilitzen els operadors dels SGST quan tenim multiresolució.


* Abans de fer una consulta temporal pot fer falta fer una selecció dels discs resolució amb el mateix interpolador.

* Fer una unió temporal de tots els discs de la BDM i treballar sobre aquest senyal $\longrightarrow$ Abstracció d'una BDSTM com a sèrie temporal; és possible treballar amb una BDSTM com si fos una sèrie temporal?

  - Com a consulta total: $\text{SerieTotal}(M)$

  - Com a consulta amb informació multiresolució: $\text{DiscSelecció}(M,\delta,f)$

* Pensar com operar (per exemple sumar) amb sèries temporals de diferents BDM. 




\subsubsection{Selecció de disc}


Consulta la subsèrie de la BDSTM que té una resolució i atribut
determinat. 


\begin{definition}[DiscSelecció]
  \begin{gather*}
    \text{DiscSelecció}: M \times \delta \times f \longrightarrow S' = S_D: \\
    (S_B,S_D,\delta,\tau,k,f) \in M
\end{gather*}
\end{definition}



\subsubsection{Sèrie temporal total}



\begin{definition}[Sèrie temporal total]
  Sigui $M^*$ una base de dades multiresolució a on no hi ha $\delta$ repetits
  \begin{gather*}
    \text{SerieTotal}: M^* \longrightarrow S': \\
    \forall (S_{Bi},S_{Di},\delta_i,\tau_i,k_i,f_i) \in M : \\
    \delta_0 < \delta_1 < \delta_2 < \dots < \delta_d : \\
    S' = S_{D0} || S_{D1} || S_{D2} || \dotsb || S_{Dd}
\end{gather*}
\end{definition}

Prèviament es pot fer una selecció dels discs resolució que
comparteixin un determinat agregador d'atributs. \todo{També hi podria
  haver una operació estructural que sabés fusionar dos discs
  resolució}



L'operació de consulta de la sèrie temporal total també es pot aplicar
tenint en compte la representació.
\begin{definition}[Sèrie total amb representació]
  Sigui $M^*$ una base de dades multiresolució a on no hi ha $\delta$
  repetits i $r$ una representació
  \begin{gather*}
    \text{SerieTotal}: M^* \times r \longrightarrow S': \\
    \forall (S_{Bi},S_{Di},\delta_i,\tau_i,k_i,f_i) \in M : \\
    \delta_0 < \delta_1 < \delta_2 < \dots < \delta_d : \\
    S' = S_{D0} \cup^r S_{D1} \cup^r  S_{D2}  \cup^r \dotsb \cup^r  S_{Dd}
\end{gather*}
\end{definition}



\paragraph{Selecció de resolució}


Per a extreure una resolució determinada de la sèrie temporal
emmagatzemada a la base de dades multiresolució, es consulta la sèrie
temporal total i s'aplica una selecció de resolució
$\text{SerieTotal}(M)[i]^r$ a on $i$ és el conjunt d'instants de
temps.






%%% Local Variables:
%%% TeX-master: "main"
%%% End:
% LocalWords:  SGSTM l'agregador buffer multiresolució subsèries
% LocalWords:  subsèrie




\section{Funcions d'agregació d'atributs}
\label{sec:model:interpolador}
\label{sec:model:agregador}
\glsaddsection{not:sgstm:fdef} %%%%secció de model
\glsaddsection{not:sgstm:f} %%%%secció de model


Les funcions d'agregació d'atributs s'utilitzen en la consolidació
dels buffers per tal de compactar certa informació de la sèrie
temporals. Sigui $S$ una sèrie temporal i $t_a$ i $t_b$ dos instants
de temps, una funció d'agregació d'atributs $f$ calcula una mesura que
resumeix la informació de $S$ en un interval de temps $i=[t_a,t_b]$:
\[
f: \text{sèrie temporal} \glsdisp{not:times}{\times}
\text{interval de temps} \longrightarrow \text{mesura}
\]
\[
f: S=\{m_0,\dotsc,m_k\} \times i=[t_a,t_b] \longrightarrow  m'
\]


Generalment, $m'$ resulta d'aplicar dues operacions a $S$: 
\begin{enumerate}
\item una selecció d'una subsèrie $S'$ segons l'interval de temps $i$,
  per exemple $S' = S[t_a,t_b]$
\item i una agregació en aquesta subsèrie $m' =
  \glssymbol{not:sgst:aggregate}(S',m_i,\glssymbol{not:sgst:fagg})$ on
  $\glssymbol{not:sgst:fagg}$ i $m_i$ són els atributs d'aquesta agregació.
\end{enumerate}



Atès que hi ha maneres diferents de resumir la informació d'una sèrie
temporal, cal plantejar diferents funcions d'agregació d'atributs. Per
exemple, es poden calcular estadístics de la sèrie temporal, com el
valor màxim o la mitjana; aplicar operacions de processament digital
del senyal, com fan \textcite{zhang11}, o algoritmes per a detectar
comportaments aberrants, com fa \textcite{lisa00:brutlag}. A més a
més, la representació de les sèries temporals
(v.~\autoref{sec:model:repr}) pot afectar els càlculs que es fan en
l'agregació o bé es pot aprofitar l'agregació per a tractar algunes de
les patologies de les sèries temporals
(v.~\autoref{sec:sgst:patologies}).  Així doncs, es poden definir una
enorme varietat de funcions d'agregació d'atributs i no hi ha cap
assumpció global que es pugui fer, cada usuari ha d'interpretar quina
combinació d'agregació i representació s'adiu més amb el fenomen
mesurat. Com a conseqüència, els \gls{SGSTM} han de donar llibertat
als usuaris per a definir funcions d'agregació d'atributs
personalitzades.


Com a mostra de com dissenyar funcions d'agregació d'atributs, a
continuació descrivim algunes interpretacions possibles que se'n poden
fer, tant pel que fa al càlcul de l'instant de temps resultant de la
consolidació com pel que fa al càlcul amb representació de sèries
temporals, i descrivim com utilitzar-les per a tractar i validar dades
desconegudes en les sèries temporals.



\subsection{Interpretació de l'agregació}


L'agregació d'una sèrie temporal en un interval resulta en una mesura
$m'=(t',v')$. Així per a definir les operacions d'agregació cal
interpretar quin ha de ser el temps resultant $t'=T(m')$ i el valor
resultant $v'=V(m')$.


Podem definir patrons generals de funcions d'agregació d'atributs que
indiquin quina informació o estadístic es resumeix de la sèrie
temporal, és a dir patrons generals que indiquin com s'ha de calcular
el valor resultant $V(m')$ independentment del mètode de representació
que es vulgui associar a la sèrie temporal.  Tot i així, el temps
resultant $T(m')$ no queda definit sinó que s'ha interpretar
coherentment per a cada cas particular de representació.


A continuació mostrem alguns exemples de patrons generals per a
calcular el valor resultant $V(m')$ que resumeix atributs d'una sèrie
temporal $S$ en un interval $i=[t_a,t_B]$. Sigui $S^r(t)$ la funció de
representació de la sèrie temporal i $t\in T$ els instants de temps
en el domini de temps:
\begin{itemize}
\item màxim: $S \times i \mapsto m'$ on $V(m') = \max_{\forall t \in
    [t_a,t_b]}(S^r(t))$. Resumeix $S$ amb el màxim dels valors de les
  mesures a l'interval $i$.
\item darrer: $S \times i \mapsto m'$ on $V(m') = S^r(t_b)$. Resumeix
  $S$ amb el valor del darrer instant de temps de l'interval $i$.

\item mitjana: $S \times i \mapsto m'$ on $V(m') = \frac{1}{t_b-t_a}
  \int_{t_a}^{t_b} S^r(t)dt$. Resumeix $S$ amb la \emph{mitjana de la
    funció} a l'interval $i$. \emph{Nota:} La mitjana d'una
  funció \parencite{weisstein:averagefunction}, $\bar f=f(x^*)$,
  utilitza la propietat $\int_a^b f(x)dx = f(x^*)(b-a)$ quan $f$ és
  contínua a $[a,b]$. \label{sec:sgstm:mitjanafuncio}
  % Explicació:
  % If $f$ is continuous on a closed interval $[a,b]$, then there is at least one number $x^*$ in $[a,b]$ such that
  % $$
  % \int_a^b f(x)dx = f(x^*)(b-a)
  % $$

  % The average value of the function ($\bar f$)  on this interval is then given by  $f(x^*)$.
  % $S(t)$ ha de ser contínua en l'interval $i$.
\end{itemize}




En aquests patrons d'atributs es treballa sobre una funció $S^r(t)$,
que a cada cas serà una funció de representació concreta i el temps
resultant $T(m')$ serà interpretat coherentment.  A més, per a cada
representació concreta també cal interpretar amb matemàtica discreta
el càlcul del valor resultant $V(m')$, atès que aquests patrons estan
definits com a problemes d'anàlisi numèric però a cada cas $S^r(t)$ és
una funció que prové d'un conjunt de mesures i podem expressar els
operadors segons el model de \gls{SGST} descrit amb àlgebra discreta
matemàtica.  A continuació s'exemplifiquen algunes interpretacions
possibles per al càlcul de $T(m')$ i de $V(m')$.





\subsubsection{Temps d'agregació resultant}


L'objectiu de les funcions d'agregació d'atributs és determinar un
instant de temps $T(m')$ i un valor $V(m')$. Aquest càlcul del temps i
del valor es pot realitzar al mateix temps però també pot ser
independent. Així, en principi el temps resultant serà independent i
valdrà $T(m')=t_b$ per estar d'acord amb l'operació de consolidació
del buffer i no causar desfasament de la subsèrie resolució
(v.~\autoref{def:sgstm:desdsamentR}), però en alguns casos aquest
$T(m')$ serà dependent del valor calculat o estarà subjecte a una
interpretació adient com és el cas per les representacions a l'apartat
següent.


Un exemple de funció d'agregació on temps i valor són dependents és
una que retorni la primera mesura que troba, $\operatorname{primera}:
S \times i \mapsto m'$ on $m' = \min(S[t_a,t_b))$ i llavors el temps
resultant pot ser $t_a \leq T(m') < t_b$. En aquest cas la sèrie
temporal consolidada resultant no és regular.


Un exemple de funció d'agregació on temps i valor són independents i
on la subsèrie resolució resultant és regular però amb desfasament, és
una funció que fa la mitjana amb un desfasament de 5 unitats de temps.
La funció d'agregació $\operatorname{mitjanad5}$ s'ha utilitzat
anteriorment a l'\autoref{ex:model:bdm-desfasaments}, ara podem
definir-la contextualitzada en les funcions d'agregació d'atributs,
$\operatorname{mitjanad5}: S \times i \mapsto m'$ on $V(m')=
\glssymbol{not:sgst:mitjanav}(S[t_a-5,t_b-5))$ i $T(m')=t_b-5$.

%De què pot servir la mitjanad5? per calcular mitjanes centrades? estem fent una interpolació sobre la representació centrada en l'interval de la sèrie temporal?


%mitjana mòbil, MM
%moving average, MA




\subsubsection{Agregació amb representació}

La varietat de funcions de representació per les sèries temporals
indueix a una varietat de funcions d'agregació per a un mateix patró
d'atributs. Per exemple, la funció d'agregació per l'atribut de màxim
dóna com a resultat valors diferents si es considera una representació
lineal o una representació a trossos constant. A continuació mostrem
la interpretació dels patrons definits anteriorment per a tres mètodes de
representació: \gls{pd}, \gls{dd} i \gls{zohe}.


\paragraph{Parcial discreta.}
En els casos parcials, $S^r(t)$ no és totalment contínua en el temps,
però es pot resoldre l'agregació del valor resultant assumint que el
domini de temps $T$ es correspon als instants de temps que hi ha a la
sèrie temporal, és a dir $T=\glssymbol{not:sgst:project}_{t}(S)$.  El
temps resultant es pot interpretar segon descrit a l'apartat anterior,
per exemple $T(m')=t_b$, i a més també es pot interpretar l'interval
de temps d'agregació $i=[t_a,t_b]$. Així sigui $S$ la sèrie original,
el resultat es pot calcular sobre una subsèrie amb interval obert
$S'=S(t_a,t_b)$, tancat $S'=S(t_a,t_b]$, semiobert $S'=S(t_a,t_b]$ o
$S'=S[t_a,t_b)$, o altres combinacions com per exemple tenir
desfasaments $S'=S[t_a-d,t_b-d]$ on $d$ és una durada.  Així de forma
general podem definir les funcions d'agregació d'atributs amb
representació \gls{pd}, $f^{\gls{pd}}\in f$, com $f^{\gls{pd}}: S
\times [t_a,t_b] \mapsto m'$ on $m'=(t_b,v')$ i el valor resultant
depèn del l'atribut que es vulgui resumir calculat en l'interval
$S'=S[t_a,t_b]$, a continuació es mostren els patrons d'exemple
interpretats segons aquest criteri.

\begin{definition}[Agregació parcial discreta]
  Sigui $S=\{m_0,\dotsc,m_k\}$ una sèrie temporal, $i=[t_a,t_b]$ un
  interval de temps i $S'=S[t_a,t_b]$ un interval de la sèrie
  temporal, les funcions d'agregació \gls{pd} per als atributs màxim,
  darrer i mitjana són:
  \begin{itemize}

  \item $\glssymboldef{not:sgstm:maxpd}: S \times i \mapsto
    m'$ on $V(m') = \max_{\forall m \in S'}(V(m))$ i
    $T(m')=t_b$. Aquest càlcul de $V(m')$ es correspon amb l'operació
    $\glssymbol{not:sgst:maxv}(S')$ dels \gls{SGST}.  \label{def:sgstm:maxpd}

\item $\operatorname{darrer}^{\gls{pd}}$: $S \times i \mapsto m'$ on $V(m') =
  V(\max(S'))$ i $T(m')=t_b$.

\item $\glssymboldef{not:sgstm:mitjanapd}: S \times i \mapsto m'$ on $V(m') =
  \frac{1}{|S'|} \sum\limits_{\forall m\in S'} V(m)$ i $T(m')=t_b$. Aquest càlcul de
  $V(m')$ es correspon amb l'operació $\glssymbol{not:sgst:mitjanav}(S')$
  dels \gls{SGST}, és a dir amb calcular la mitjana aritmètica dels
  valors de les mesures. \label{def:sgstm:mitjanapd}
\end{itemize}

\end{definition}



\paragraph{Delta de Dirac.} 
Per a les funcions d'agregació delta de Dirac interpretem el temps
d'agregació resultant centrat en l'interval $T(m')=\frac{t_b+t_a}{2}$,
tot i que també es podrien considerar altres interpretacions com per
exemple $T(m')=t_b$. Així de forma general podem definir les funcions
d'agregació d'atributs amb representació \gls{dd}, $f^{\gls{dd}}\in
f$, com $f^{\gls{dd}}: S \times [t_a,t_b] \mapsto m'$ on
$m'=(\frac{t_b+t_a}{2},v')$ i el valor resultant depèn del l'atribut
que es vulgui resumir calculat en l'interval temporal \gls{dd}
$S'=S[t_a,t_b]^{\gls{dd}}$.


\begin{definition}[Agregació delta de Dirac]
  \label{def:sgstm:maxdd}
  Sigui $S=\{m_0,\dotsc,m_k\}$ una sèrie temporal, $i=[t_a,t_b]$ un
  interval de temps i $S'=S[t_a,t_b]^{\gls{dd}}$ un interval temporal
  de la sèrie temporal, les funcions d'agregació \gls{dd} per als
  atributs màxim, darrer i mitjana són:
\begin{itemize}
\item \glssymboldef{not:sgstm:maxdd}: $S
  \times i \mapsto m'$ on $V(m') = \max\big(0,\max_{\forall m \in
    S'}(V(m))\big)$ i $T(m')=\frac{t_b+t_a}{2}$. 

\item $\operatorname{darrer}^{\gls{dd}}$: $S \times i \mapsto m'$ on $V(m') =
  V(\max(S'))$ i $T(m')=\frac{t_b+t_a}{2}$.

\item \glssymboldef{not:sgstm:mitjanadd}: $S \times i \mapsto m'$ on
  $V(m') = \frac{1}{t_b-t_a}\sum\limits_{\forall m \in S'} V(m)$ i
  $T(m')=\frac{t_b+t_a}{2}$. Nota: la funció delta de Dirac té la
  propietat fonamental $\int \delta(t)dt = 1$. 
\end{itemize}
\end{definition}



\paragraph{Zero-order hold enrere.}
Per a les funcions d'agregació \gls{zohe} interpretem sempre el temps
d'agregació resultant com $T(m')=t_b$, atès que la representació
\gls{zohe} es defineix amb funcions graó contínues per
l'esquerra. Així de forma general podem definir les funcions
d'agregació d'atributs amb representació \gls{zohe},
$f^{\gls{zohe}}\in f$, com $f^{\gls{zohe}}: S \times [t_a,t_b] \mapsto
m'$ on $m'=(t_b,v')$ i el valor resultant depèn de l'atribut que es
vulgui resumir calculat en l'interval temporal \gls{zohe}
$S'=S[t_a,t_b]^{\gls{zohe}}$.
\begin{definition}[Agregació zero-order hold enrere]
  Sigui $S=\{m_0,\dotsc,m_k\}$ una sèrie temporal, $i=[t_a,t_b]$ un
  interval de temps i $S'=S[t_a,t_b]^{\gls{zohe}}$ un interval
  temporal de la sèrie temporal, les funcions d'agregació \gls{zohe}
  per als atributs màxim, darrer i mitjana són:
  \begin{itemize}
  \item \glssymboldef{not:sgstm:maxzohe}: $S \times i \mapsto m'$ on
    $V(m') = \max_{\forall m \in S'}(V(m))$ i $T(m')=t_b$.

  \item $\operatorname{darrer}^{\gls{zohe}}$: $S \times i \mapsto m'$
    on $V(m') = V(\max(S'))$ i $T(m')=t_b$.

  \item \glssymboldef{not:sgstm:meanzohe}: $S \times i \mapsto m'$ on
    $V(m') = \frac{1}{t_b-t_a} \big[ (T(o)-t_a)V(o) +
    \sum\limits_{\forall m \in S''}( T(m)-
    T(\glssymbol{not:sgst:prev}_S (m)) )V(m) \big]$; $o=\min(S')$;
    $S''= S' - \{o\}$; i $T(m')=t_b$.  \label{def:sgstm:meanzohe}
% \[
%   \begin{split}
%   V(m')  = & \frac{1}{t_b-T_0} 
%   \big[ (T(o)-T_0)V(o) -( T(n)-T_f)V(n) \\
%     & {}+\sum\limits_{\forall m \in S''}( T(m)- T(\prev_S m) )V(m) \big]   
%    \end{split}
%   \]
% Nota: s'aplica la definició $0 \times \infty = 0$ tal com es fa habitualment a la teoria de mesura, \cite{wiki:extendedreal}.
  \end{itemize}
\end{definition}




Un cop definits els tres exemples de famílies d'agregacions, podem
comparar-les en funció de com resumeixen la informació de la sèrie
temporal. Reprenent la consolidació dels buffers
(v.~\autoref{sec:model:buffer}), l'interval de consolidació es
correspon a $t_a=\tau$ i $t_b=\tau+\delta$ i és consolidable quan
existeix una mesura $T(m)\geq\tau+\delta$. A la
\autoref{fig:sgstm:agg} dibuixem les mesures d'una sèrie temporal en
vermell, un interval de consolidació del buffer en línies blaves i la
mesura resultant de consolidació en verd.  Així, sigui
$S=\{\dotsc,m_{a-1},m_{a+1},\dotsc,m_{b-1},m_{b+1}, \ldots\}$ una
sèrie temporal on $ T(m_{a-1}) < t_a < T(m_{a+1}) < \dotsc <
T(m_{b-1}) < t_b < T(m_{b+1})$ i la consolidació del buffer que
calcula la mesura resultant $m'=f(S,[t_a,t_b])$ amb la funció
d'agregació d'atributs $f$.  Assumim $T(m')=t_b$ per simplificar el
dibuix, de manera general el càlcul del valor resultant és una
agregació a partir de les mesures:
\begin{itemize}
\item $\{m_{a+1},\dotsc,m_{b-1}\}$ en el cas de les agregacions \gls{pd}
\item $\{(t_a,0),(\ldots,0),m_{a+1},\dotsc,(\ldots,0),\dotsc,m_{b-1},(\ldots,0),(t_b,0)\}$ en el cas de les
  agregacions \gls{dd}
\item $\{m_{a+1},\dotsc,m_{b-1},m_{b+1}\}$ en el cas de les
  agregacions \gls{zohe}
\end{itemize}






\begin{figure}[tp]
  \centering
 
    \begin{tikzpicture}
        \begin{axis}[
          % width=10cm,
%          scale only axis, height=3cm,
          ymin = 0,
          xmax = 50,
          xmin = 20,
          yticklabels= {},
          xticklabels={,,,$t_a$,,$t_b$},
          ]
          \addplot[ycomb,blue] coordinates {
            (30,10)
            (40,10)
          }; 
          
          \addplot[only marks,mark=*,red] coordinates {
            (25,5)
            (32,2)
            (35,4)
            (38,6)
            (45,8)
          };
          
          \addplot[only marks,mark=*,green] coordinates {
            (40,4)
          };
          
          \node[above] at (axis cs:26,5) {$m_{a-1}$};
          \node[below] at (axis cs:32,2) {$m_{a+1}$};
          \node[below] at (axis cs:35,4) {$\ldots$};
          \node[above] at (axis cs:38,6) {$m_{b-1}$};
          \node[above] at (axis cs:45,8) {$m_{b+1}$};
          \node[right] at (axis cs:40,4) {$m'$};
        \end{axis}
      \end{tikzpicture}

    
  \caption{Agregació d'un interval de la sèrie temporal}
  \label{fig:sgstm:agg}
\end{figure}






En conclusió, per una banda alguns exemples mostrats de patrons tenen
una interpretació semblant per a les representacions particulars, en
certa manera només es diferencien en la interpretació de l'interval on
s'ha de resumir la sèrie temporal. Per exemple la diferència principal
en els atributs de màxim i darrer per a les tres representacions rau
en la $S'$, tot i que en el cas del $\glssymbol{not:sgstm:maxdd}$
l'agregació a més ha de tenir en compte que en la funció de
representació hi ha valors intermitjos que valen zero.

Per altra banda, altres exemples són molt diferents, com és el cas de
l'atribut mitjana. En aquest cas, per a la \gls{pd} i la \gls{dd} és
el càlcul de la suma dels valors tot i que dividit per $|S'|$ en la
primera i per $t_b-t_a$ en la segona, i és una mitjana ponderada per
les durades de temps en la \gls{zohe}.  En general, es pot dissenyar
qualsevol operació d'agregació, com per exemple calcular la mitjana
aritmètica de l'interval \gls{zohe} amb
$\glssymbol{not:sgst:mitjanav}(S[t_a,t_b]^{\gls{zohe}})$, tot i que
llavors cal interpretar quin patró d'atribut li correspon o altrament
aquesta operació d'agregació pot no tenir sentit real.


\textcite{rrdtool} utilitza a RRDtool una funció d'agregació semblant
a la $\glssymboldef{not:sgstm:meanzohe}$ per a resumir la informació
conservant el comptatge total si les sèries temporals mesurades tenen
trets semàntics de comptador i són en forma de velocitat; així aquesta
agregació es pot veure com una consolidació que conserva l'àrea del
senyal original. 





% Notes:

% * Quan una sèrie temporal és regular, l'intepolador mitjana aritmètica i l'interpolador àrea valen el mateix en l'interval $(T_o,n\delta]$.




\subsection{Tractament i validació de dades}


En les patologies de les sèries temporals
(v.~\autoref{sec:sgst:patologies}) s'ha descrit el problema de les
dades desconegudes, les funcions d'agregació d'atributs poden cooperar
en els processos de validació i tractament de dades. Així, les
funcions d'agregació poden marcar o tractar dades desconegudes:
\begin{itemize}
\item Marcar dades com a desconegudes. És a dir determinar quan el
  resultat d'una agregació ha de ser desconegut perquè la sèrie
  temporal avaluada pateix una de les causes descrites: valors fora de
  rang, temps de termini excedit, etc.

\item Tractar dades que són desconegudes, ja sigui perquè d'origen són
  desconegudes o perquè les hem marcat abans com a desconegudes.
  Si una funció d'agregació rep valors que són desconeguts, des d'un
  punt de vista estricte el resultat de l'agregació ha de ser
  desconegut. No obstant això, es poden aplicar operacions que tractin
  aquest valors desconeguts: reconstrucció del senyal, ignorar els
  valors desconeguts, etc.
\end{itemize}

 
A continuació definim el procés que fan les funcions d'agregació per a
ambdós casos. Com a exemple de domini pels valors utilitzem els
nombres reals projectius \glssymbol{not:R*}, en els quals representem
el valor desconegut mitjançant l'element infinit ($\infty$), segons la
\autoref{def:model:mesura_valor_indefinit} de mesura de valor
indefinit. Això no obstant, el domini de valors podria tenir diversos
valors per a marcar diferents casos de dades desconegudes.

\paragraph{Tractament de dades desconegudes.}
Una funció d'agregació d'atributs $f^u \in f$ que tracti dades
desconegudes és aquella que pot calcular un resultat quan la sèrie
temporal original conté valors desconeguts
\[
f^u: S \times i \mapsto m' \text{ on } \exists m \in S: V(m)=\infty
\]

Com ja hem comentat, treballar amb valors desconeguts estrictament
hauria de resultar en valors desconeguts. Això no obstant, les dades
desconegudes es poden tractar mitjançant tècniques de reconstrucció,
d'interpolació, d'aproximació, etc. L'usuari, però, s'ha d'assegurar i
estudiar en cada context que la tècnica que apliqui per a tractar
dades desconegudes sigui vàlida. Altrament, només podrà considerar el
resultat com a desconegut.


Per exemple, podem redefinir el patró de la funció d'agregació mitjana
en una $\operatorname{mitjana}^{u}$ que sigui capaç de tractar valors
desconeguts conservant l'àrea coneguda, és a dir, l'àrea total
coneguda quedarà escampada en l'interval de consolidació.
\begin{gather*}
  \operatorname{mitjana}^{cu}: S \times i \mapsto m' \text{ on }\\
  V(m') = \frac{1}{t_b-t_a}\int_{t_a}^{t_b} S^u(t)dt \text{ i }
  S^u(t)=
  \begin{cases}
    0 &\text{si }  S^r(t)=\infty\\
    S^r(t) & \text{altrament }
  \end{cases}
\end{gather*}


\paragraph{Marcatge de dades desconegudes.}
Una funció d'agregació d'atributs $f^{mu} \in f$ que marqui
dades desconegudes és aquella que pot retornar una mesura de valor
indefinit com a resultat
\[
f^{mu}: S \times i \mapsto m' \text{ on } V(m')\in \glssymbol{not:R*}
\]


Per exemple, podem definir un patró de funció d'agregació d'atribut
màxim que retorni valor desconegut
si hi ha un a mesura amb el valor més gran que 2; és a dir establim un
límit superior de 2 (L2). 
\begin{gather*}
  \operatorname{m\grave{a}xim}^{L2}: S \times i \mapsto m' \text{ on }\\
  m' = \begin{cases}
    (T(m''),\infty) &\text{si }  \exists m\in S[t_a,t_b]: V(m)>2\\
    m'' & \text{altrament }
  \end{cases} \text{ i } m''= \operatorname{m\grave{a}xim}(S,i)
\end{gather*}






%\todo{}
%hauria d'aparèixer algun exemple on es resolgués inframostreig. Potser també algun exemple on es veiés on els agregadors solucionen el problema de l'ultramostreig.



%Per exemple definim un termini, si les dades estan més espaiades que 2 es marca com a desconeguda
% Sigui $S=\{m_0,\ldots,m_k\}$ una sèrie temporal i $H$ un termini de temps, una mesura $m_i=(v_i,t_i)\in S$ és desconeguda si, donada la mesura anterior $m_{i-1}=(v_{i-1},t_{i-1})$, $t_i - t_{i-1} > H$.    





% Sigui $S=\{m_0,\ldots,m_k\}$ una sèrie temporal, $f$ un interpolador, $i=[T_0,T_f]$ un interval de temps i $\alpha$ un llindar, la mesura de consolidació calculada per l'interpolador $f$ és desconeguda ssi  
% \[
% \frac{t_d }{T_f - T_0} > \alpha :
% \]
% \[
% :t_d = t_{d0} + t_{df} + \sum\limits_{i=1}^{k-1}(t_i-t_{i-1}) : v_k = 'desconegut':
% \]
% \[
% : t_{d0} = \left\{\begin{array}{l} t_0-T_0 \text{ si } v_0 = 'desconegut' \\ 0\end{array}\right. ,
% t_{df} = \left\{\begin{array}{l} T_f-t_{k-1} \text{ si } v_k = 'desconegut' \\ 0\end{array}\right. :
% \]
% \[
% :k=|S|-1,(v_k,t_k)=m_k\in S' :S'= S_{T_0:T_f} \cup \{min(S_{T_f:\infty})\}
% \]



%operacions amb nan de octave i matlab
%http://biosig-consulting.com/matlab/NaN/
% The NaN-toolbox v2.0: A statistics and machine learning toolbox for Octave and Matlab
% for data with and w/o MISSING VALUES encoded as NaN's.











%%% Local Variables:
%%% TeX-master: "main"
%%% End:
% LocalWords: buffer buffers ZOHE










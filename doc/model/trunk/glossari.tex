\newglossary{notation}{not}{ntn}{Notació}
\newglossarystyle{estil-notation}{%
  \renewcommand{\glsgroupskip}{}% make nothing happen between groups
  \renewenvironment{theglossary}
  {\begin{longtable}{ll}
      % \caption{Notació dels SGSTM \label{tab:sgstm-simbols}}
      % \endfirsthead
      % \caption[]{Notació dels SGSTM (continuació)}
      % \endhead
          % \endfoot
          % \endlastfoot
    }{\end{longtable}}
  \renewcommand*{\glossarysubentryfield}[6]{%
    \glstarget{##2}{##3}% the entry name
    &
    % \space (##4)% the symbol in brackets
    \space ##4% the description
    % \space [##6]% the number list in square brackets
    \\
  }%
  \renewcommand*{\glossaryentryfield}[5]{%
    \\\pagebreak[3]\hline
    \glossarysubentryfield{##2}{##1}{##2}{##3}{##4}{##5}
    \hline
  }
}


\renewcommand{\seename}{vegeu}
\renewcommand{\entryname}{Notació}
\renewcommand{\descriptionname}{Descripció}

\makeglossaries


%\renewcommand{\glossarypreamble}{Text com a préambul}



%TERMES


\newglossaryentry{SistemaGestioBaseDades}{name={sistema de gesti{ó} de base de dades}, description={(\emph{Data Base Management System})} }




%terme:SGBDR

\newglossaryentry{terme:SGBDR}{name={sistema de gestió de base de dades relacional}, description={(\emph{Relational Data Base Management System}). Totes les definicions són coherents amb \textcite{date} } }


%tipus,valor,variable,operador

\newglossaryentry{terme:SGBDR:domini}{see={terme:SGBDR},name={domini}, description = {(\emph{domain}), equivalent a tipus de dades.
Conjunt de valors. Cada domini té associat un conjunt d'operadors, en alguns casos fins i tot s'entén que el domini inclou els operadors (concepte de classe a orientació a objecte). Els tipus tenen una representació (estructura) o més d'una, és a dir els seus valors poden estar denotats per més d'un literal} }
\newglossaryentry{terme:SGBDR:tipus}{see={terme:SGBDR:domini}, name={tipus de dades}, description = {(\emph{data type}), a vegades solament 'tipus' (\emph{type}) o bé 'tipus de dades abstracte' (\emph{abstract data type}). Segons \textcite{date} en el context de model tots els tipus de dades han de ser abstractes} }

\newglossaryentry{terme:SGBDR:escalar}{parent={terme:SGBDR:domini}, name={escalar}, description = {Un tipus és escalar (\emph{scalar}) quan no té components visibles a l'usuari i és no escalar (\emph{nonscalar}) en cas contrari; no obstant, tant els escalars com els no escalars tenen representació, la qual pot contenir components} }


\newglossaryentry{terme:SGBDR:valor}{see={terme:SGBDR},name={valor}, description = {(\emph{value}), equivalent a objecte i instància.
'Constant individual' que és d'un tipus de dades. A vegades s'utilitza 'constant' per designar una  variable que mai canvia de valor, però aquest no és el cas d'aquesta definició} }
\newglossaryentry{terme:SGBDR:objecte}{see={terme:SGBDR:valor}, name={objecte}, description = {(\emph{object})} }
\newglossaryentry{terme:SGBDR:instancia}{see={terme:SGBDR:valor}, name={instància}, description = {(\emph{instance})} }

\newglossaryentry{terme:SGBDR:literal}{see={terme:SGBDR},name={literal}, description = {(\emph{literal}).
Símbol que denota un valor. Un valor pot estar denotat per més d'un literal. Segons aquesta definició literal no és equivalent a valor} }


\newglossaryentry{terme:SGBDR:variable}{see={terme:SGBDR},name={variable}, description = {(\emph{variable}).
Contenidor d'una aparició d'un valor. El valor que conté la variable pot ser canviat mitjançant l'operador d'assignació. En canvi els valors, per si mateixos, no poden ser actualitzats} } %A l'esquerra de l'operador d'assignació sempre hi ha variables, tot i que s'admeten simplificacions mitjançant expressions que són pseudovariables (p.ex. s[1] := 3 és equivalent a s := [s[0],3,s[2],..]).
%Les variables tenen adreces (\emph{addresses}) i per tant es pot apuntar (\emph{point to}) a les variables mitjançant els operadors de referència (\emph{referencing}), el qual retorna l'adreça d'una variable, i de desreferència (\emph{dereferencing}), el qual retorna la variable a partir de l'adreça. Els valors adreces pertanyen al tipus apuntador, però el model relacional prohibeix els valors de tipus apuntador i per tant no té REF ni DEREF; les relvar s'identifiquen pel seu nom i no cal que tinguin adreça. (Compte que en orientació a objectes una variable és el contenidor d'un valor que és un ID d'objecte, és a dir és el contenidor d'una referència).





% [date2005]
% The original version of the model also omitted a few things I now consider vital. For example, it excluded any
% mention—at least, any explicit mention—of all of the following: predicates, constraints (other than candidate
% and foreign key constraints), relation variables, relational comparisons, relation type inference and associated
% features, certain algebraic operators (especially rename, extend, summarize, semijoin, and semidifference),
% and the important relations TABLE_DUM and TABLE_DEE.




%pendent: falta posar el name

% \newglossaryentry{SGBD-model}{ description = {Un model és}, name={Model de SGBD} }


% \newglossaryentry{SBDR-cap}{ description = {La capçalera d'un SGBDR}, name={Capçalera}, parent={SGBD-model} }



% \newglossaryentry{heading}{ description = {Equivalent to intension and relation schema} }
% \newglossaryentry{intension}{ description = {}, see=heading }
% \newglossaryentry{relation schema}{ description = {}, see=heading }

% \newglossaryentry{body}{ description = {Equivalent to extension} }
% \newglossaryentry{extension}{ description = {buit}, see=body}


% \newglossaryentry{DBMS data model}{ description = {A data model (first sense) is an abstract, self-contained, logical definition of the
% objects, operators, and so forth, that together constitute the abstract machine with which
% users interact. The objects allow us to model the structure of data. The operators allow us
% to model its behavior.\cite{date}. Sometimes it is referred as architecture.
% } }

% \newglossaryentry{data model}{ description = { A data model (second sense) is a model of the persistent data of some particular
% enterprise. [date06]}}


% \newglossaryentry{DBMS implementation}{ description = {An implementation of a given data model is a physical realization on a real
% machine of the components of the abstract machine that together constitute that model.\cite{date}} }


% \newglossaryentry{data independence}{ description = {model and implementation kept separated}}




% \newglossaryentry{relationships}{
% description={relationships are semantic. relationships are entities.}}








%%% Local Variables: 
%%% mode: latex
%%% TeX-master: "main"
%%% End: 

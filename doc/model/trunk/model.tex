\chapter{Introducció als models}


En els capítols següents es dissenya un model matemàtic pels sistemes
de gestió de bases de dades multiresolució per a sèries
temporals. 


En aquest capítol, s'introdueix el concepte de model matemàtic
pels sistemes de gestió de bases de dades.  


En els següents capítols es defineixen els objectes que ens permeten
modelar l'estructura de les dades i els operadors que s'hi poden
aplicar.

La definició del model s'estructura en dues parts:

\begin{itemize}
\item Un model pels (SGST)  que defineix mesura i sèrie temporals.
\item Un model pels (SGSTM) que defineix buffer, disc i subsèrie
  resolució, el qual treballa sobre el model de SGST.
\end{itemize}


  

\section{Introducció}


\todo{Sobre model}
Kopetz en el llibre de temps real (capítol 2 o 3?) diu que un model té l'objectiu d'estudiar una realitat simplificada per a facilitar la comprensió d'una determinada característica. Potser la definició que fa i la intencionalitat que té no és ben bé la mateixa que el tipus de model que parlem aquí? Si és el cas potser estaria bé fer notar que hi ha diferències en el concepte de model segon l'àmbit i que aquí s'utilitza en tal sentit.

Per exemple, Fabian Pascal parla de representar la realitat de manera simple (i no tant de simplificar la realitat):
``For the informational purpose that RM satisfies--inferencing facts that are logical implications of facts represented in databases--the RM is superior, because it is the simplest way to guarantee logically correct results with respect to the real world and it has the highest scope-to-simplicity ratio: it can represent any reality with the least and simplest of constructs''




\todo{sobre tres nivells}
A l'estat de l'art s'ha d'haver explicat els tres nivell de model de dades segons Date i deixar clar aquí que nosaltres definim un model pel segon nivell: nivell de model lògic. Els models lògics modelen les dades, en canvi els models conceptuals modelen la realitat, Fabian Pascal posa d'exemple conceptual el model E/RM.



Segons Date,~\cite{date:introduction}, ``una base de dades és un contenidor informàtic per a una co\l.lecció de dades''. El sistemes informàtics que tracten amb bases de dades s'anomenen sistemes de gestió de bases de dades (SGBD) i tenen l'objectiu d'emmagatzemar informació i permetre consultar i afegir aquesta informació  per part dels usuaris.
Per complir aquests objectius, els SGBD ofereixen a l'usuari diferents operacions a fer amb la base de dades, com per exemple crear-la, afegir dades o consultar informació a partir de les dades emmagatzemades.

Els SGBD es basen en teories matemàtiques que reben el nom de model de dades, un SGBD és una implementació d'un model de dades.
Segons Date, ``un model de dades és una definició abstracta, auto continguda i lògica dels objectes, de les operacions i  de la resta que conjuntament constitueixen la màquina abstracta amb la que els usuaris interactuen. Els objectes permeten modelar l'estructura de les dades. Les operacions permeten modelar el comportament''.

Les bases de dades multiresolució per a sèries temporals són bases de dades que contenen sèries temporals. Les sèries temporals són una co\l.lecció de dades mesurades en diferents instants de temps i necessiten un tractament adequat per part de la base de dades. 
El model de dades multiresolució és una solució d'emmagatzematge per a les sèries temporals que, resumint, consisteix a repartir la informació d'una sèrie temporal en intervals de temps diferents.


El model de dades per a sèries temporals es dissenya en el
capítol~\ref{cap:model:sgst}. El disseny d'aquest model és necessari
per a comprendre i construir el model multiresolució.  El model de
dades multiresolució es dissenya en el capítol~\ref{cap:model:sgstm}.
El concepte de multiresolució prové d'estudis anteriors
\parencite{llusa12:ptd}, el qual es va formalitzar com a abstracció
d'una característica essencial de l'SGBD
RRDtool \parencite{rrdtool}. Aquesta abstracció es va realitzar a
partir d'una anàlisi profunda dels conceptes de RRDtool duta a terme
en una tesi de màster \parencite{llusa11:tfm}.



Així doncs, a continuació es presenten els model de dades.

En el model per a sèries temporals es defineixen els objectes principals d'estudi: les \emph{mesures} i les \emph{sèries temporals}. Les mesures són dades mesurades en un instant de temps i les sèries temporals són co\l.leccions de mesures.

El model de dades multiresolució s'estructura a partir de \emph{sèries temporals multiresolució} com a conjunt de \emph{subsèries resolució}, les quals  acumulen temporalment les mesures en un \emph{buffer} per tal de tractar-les abans d'emmagatzemar-les  a un \emph{disc}. El tractament principal consisteix en canviar els intervals de temps entre mesures amb l'objectiu de compactar la informació de la sèrie temporal.
Així doncs, la sèrie temporal queda emmagatzemada com una sèrie temporal multiresolució en intervals de temps diferents, repartits en les subsèries resolució. 

Pel que fa a les operacions, és indispensable que el model multiresolució pugui fer aquests canvis d'intervals de temps, els quals s'aconsegueixen amb les operacions d'\emph{agregació} i \emph{consolidació}. En el model de dades multiresolució es defineixen els operadors específics per a aquestes tasques anomenats \emph{agregadors d'atributs}


A banda de les estructures, en el model també es defineixen els
operadors que permeten tractar les dades; ja sigui per operar amb les
estructures o bé per a fer consultes.



\todo{sobre àlgebra relacional i càlcul relacional}
Els models que definim es basen en l'àlgebra relacional. Es podria definir el mateix des del càlcul relacional.










\section{Resum dels objectius models}


En el model de SGST s'observen algunes patologies que poden presentar les sèries temporals. El model de SGSTM soluciona algunes d'aquestes patologies:

\begin{itemize}
\item Regularitza les sèries temporals
\item Tracta i validar les sèries temporals: gestiona els casos de dades errònies o desconegudes i marca quan hi ha valors erronis.
\item És una solució de compressió per a quantitats enormes de dades
\end{itemize}


Però el model de SGSTM també es pot fer servir per altres aplicacions:

* Regularitzar en línia (temps real) una sèrie temporal en diferents períodes de mostreig

* Tenir unes vistes (consultes) a punt (ja processades) amb diferents resolucions d'una sèrie temporal

* Comprimir per decimació (downsampling) o bé farcir forats (reconstrucció del senyal)




% \dots
% \ldots
% \cdots
% \vdots
% \ddots diagonal dots
% \iddots  inverse diagonal dots (requires the mathdots package)
% \hdotsfor{n} to be used in matrices, it creates a row of dots spanning n columns

% A_1,A_2,\dotsc,   with commas
% A_1+\dotsb+A_N    with binary operators/relations
% A_1 \dotsm A_N    multiplication dots
% \int_a^b \dotsi   with integrals
% A_1\dotso A_N     other dots




%%% Local Variables:
%%% TeX-master: "main"
%%% End:
% LocalWords: buffer buffers






\chapter{Introducció als models}


En els capítols següents es dissenya un model matemàtic per a la
gestió en bases de dades de la multiresolució de sèries temporals.

En els següents capítols es defineixen els objectes que ens permeten
modelar l'estructura de les dades i els operadors que s'hi poden
aplicar.

La definició del model s'estructura en dues parts:

\begin{itemize}
\item Un model pels \gls{SGST}  que defineix mesura i sèrie temporals.
\item Un model pels \gls{SGSTM} que defineix buffer, disc, subsèrie
  resolució, sèrie temporal multiresolució i esquema de
  multiresolució. Aquest model es defineix a partir del model de \gls{SGST}.
\end{itemize}


  
En aquest capítol, es relaciona i s'explica més bé el concepte de
model matemàtic pels \gls{SGBD} del capítol d'estat de l'art
\todo{ref} amb el model que proposem. Sobretot s'aclareixen alguns termes i conceptes que altrament podrien ser confusos. 






\section{Introducció}


Una definició més particular de base de dades que l'exposada en el
capítol d'estat de l'art \todo{ref} és ``conjunt de dades organitzades
segons una estructura coherent i accessibles des de més d'un programa
o aplicació, de manera que qualsevol d'aquestes dades pot ésser
extreta del conjunt i actualitzada, sense que això afecti ni
l'estructura del conjunt ni les altres dades'' \parencite[s.~v.~base
de dades]{termcat} amb la corresponent definició per als sistemes que
les gestionen de ``sistema informàtic que permet la gestió automàtica
d'una base de dades, generalment la creació, l'emmagatzematge, la
modificació i la protecció de les dades que s'hi
contenen'' \parencite[s.~v.~sistema de gestió de bases de
dades]{termcat}.  En aquest cas, hem de precisar que en els capítols
següents ens centrem en la teoria dels sistemes d'informació,
basant-nos en el model relacional \parencite{date04:introduction8},
per a proposar el model lògic i deixem de banda els aspectes més
d'implementació informàtica, com per exemple accedir o manipular
físicament les dades.

D'aquestes definicions cal destacar la precisió en els termes
d'estructura, organitzada i coherent, és a dir que s'enumeren els
requisits per al correcte emmagatzematge de les dades en concordança
amb l'expressat en el mode relacional.  En aquestes definicions, a
més, caldria afegir que els \gls{SGBD} han de ser capaços d'inferir
informació, és a dir a partir de les dades emmagatzemades deduir-ne de
noves mitjançant les consultes.




\todo{Sobre model}
%Sobre model aproximat i model exacte


Aquí es tracta d'establir un model de SGBD, per tant la realitat directa que volem modelar és un SGBD: ho podem fer de forma exacta? Ara bé, aquest SGBD serà un model per a aproximar la realitat. 


 1 m. [QU] Representació ideal d’un aspecte concret de la realitat física emprada amb finalitats d’interpretació i de quantificació dels fenòmens i dels comportaments. 
4 2 m. [MT] Teoria o descripció matemàtica d’un objecte o fenomen real. 
4 3 m. [LC] [ECT] Simplificació de la realitat que intenta detectar els elements fonamentals d’un problema concret, eliminant-ne aspectes secundaris. 


Kopetz en el llibre de temps real (capítol 2 o 3?) diu que un model té l'objectiu d'estudiar una realitat simplificada per a facilitar la comprensió d'una determinada característica. Potser la definició que fa i la intencionalitat que té no és ben bé la mateixa que el tipus de model que parlem aquí? Si és el cas potser estaria bé fer notar que hi ha diferències en el concepte de model segon l'àmbit i que aquí s'utilitza en tal sentit.

Per exemple, Fabian Pascal parla de representar la realitat de manera simple (i no tant de simplificar la realitat):
``For the informational purpose that RM satisfies--inferencing facts that are logical implications of facts represented in databases--the RM is superior, because it is the simplest way to guarantee logically correct results with respect to the real world and it has the highest scope-to-simplicity ratio: it can represent any reality with the least and simplest of constructs''



* Sobre model aproximat i model exacte

El model de SGST i SGSTM que proposem són models lògics
matemàtics. Per tant, són models abstractes i exactes: defineixen un
discurs que és exacte i alhora independent de la realitat.  Ara bé,
aquest model ha de servir per a descriure la realitat; és a dir que
també s'ha d'interpretar el significat que té el model en la
realitat. Aleshores aquest procés d'interpretació construeix un model
aproximat; és a dir que consisteix en definir quines variables hi
haurà, de quin tipus seran els valors, quina forma tenen, com
s'interpreten, etc. i per tant és un model aproximat i simplificat de
la realitat. La mateixa diferència entre el model aproximat i model
exacte es pot aplicar pels SGBD en general: és a dir cal no confondre
la definició del model matemàtic amb la interpretació del model en una
realitat concreta. D'alguna manera, els SGBD descriuen exactament els
fets que es coneixen i mitjançant altres teories es dedueix en quina
mesura aquests fets coneguts s'aproximen a la realitat (particularment
un fet descrit com a cert en un SGBD pot ser totalment esbiaixat de la
realitat però això no se n'ocupen els SGBD; és a dir el predicat exacte és el d'allò que és observat tot i que mitjançant altres teories es pot assegurar un predicat d'allò que és la realitat).  En l'àmbit dels nostres
models de sèries temporals, això vol dir que els SGST i SGSTM
descriuen exactament el fet que coneixem que hem mesurat un valor però
cal una teoria, com per exemple la teoria de la mesura, que avaluï
fins a quin punt el valor mesurat s'aproxima a la realitat (particularment si els fets descriuen allò que s'ha mesurat aleshores ens podem plantejar la mesura de valors desconeguts mentre que si descriuen allò que realment valia la variable no podem parlar sobre aspectes de les mesures); i per tant
la interpretació del significat dels models matemàtics a la realitat,
i com a conseqüència el modelatge aproximat de la realitat, depèn del
conjunt del model lògic de SGBD més altres teories. Cal aclarir que
not té sentit pràctic només el model lògic abstracte sense tenir en
compte la vessant d'interpretació de model aproximat, encara que sí
que té tot el sentit matemàtic.










\todo{sobre tres nivells}
A l'estat de l'art s'ha d'haver explicat els tres nivell de model de dades segons Date i deixar clar aquí que nosaltres definim un model pel segon nivell: nivell de model lògic. Els models lògics modelen les dades, en canvi els models conceptuals modelen la realitat, Fabian Pascal posa d'exemple conceptual el model E/RM.



Segons Date,~\cite{date:introduction}, ``una base de dades és un contenidor informàtic per a una co\l.lecció de dades''. El sistemes informàtics que tracten amb bases de dades s'anomenen sistemes de gestió de bases de dades (SGBD) i tenen l'objectiu d'emmagatzemar informació i permetre consultar i afegir aquesta informació  per part dels usuaris.
Per complir aquests objectius, els SGBD ofereixen a l'usuari diferents operacions a fer amb la base de dades, com per exemple crear-la, afegir dades o consultar informació a partir de les dades emmagatzemades.

Els SGBD es basen en teories matemàtiques que reben el nom de model de dades, un SGBD és una implementació d'un model de dades.
Segons Date, ``un model de dades és una definició abstracta, auto continguda i lògica dels objectes, de les operacions i  de la resta que conjuntament constitueixen la màquina abstracta amb la que els usuaris interactuen. Els objectes permeten modelar l'estructura de les dades. Les operacions permeten modelar el comportament''.

Les bases de dades multiresolució per a sèries temporals són bases de dades que contenen sèries temporals. Les sèries temporals són una co\l.lecció de dades mesurades en diferents instants de temps i necessiten un tractament adequat per part de la base de dades. 
El model de dades multiresolució és una solució d'emmagatzematge per a les sèries temporals que, resumint, consisteix a repartir la informació d'una sèrie temporal en intervals de temps diferents.


El model de dades per a sèries temporals es dissenya en el
capítol~\ref{cap:model:sgst}. El disseny d'aquest model és necessari
per a comprendre i construir el model multiresolució.  El model de
dades multiresolució es dissenya en el capítol~\ref{cap:model:sgstm}.
El concepte de multiresolució prové d'estudis anteriors
\parencite{llusa12:ptd}, el qual es va formalitzar com a abstracció
d'una característica essencial de l'SGBD
RRDtool \parencite{rrdtool}. Aquesta abstracció es va realitzar a
partir d'una anàlisi profunda dels conceptes de RRDtool duta a terme
en una tesi de màster \parencite{llusa11:tfm}.



Així doncs, a continuació es presenten els model de dades.

En el model per a sèries temporals es defineixen els objectes principals d'estudi: les \emph{mesures} i les \emph{sèries temporals}. Les mesures són dades mesurades en un instant de temps i les sèries temporals són co\l.leccions de mesures.

El model de dades multiresolució s'estructura a partir de \emph{sèries temporals multiresolució} com a conjunt de \emph{subsèries resolució}, les quals  acumulen temporalment les mesures en un \emph{buffer} per tal de tractar-les abans d'emmagatzemar-les  a un \emph{disc}. El tractament principal consisteix en canviar els intervals de temps entre mesures amb l'objectiu de compactar la informació de la sèrie temporal.
Així doncs, la sèrie temporal queda emmagatzemada com una sèrie temporal multiresolució en intervals de temps diferents, repartits en les subsèries resolució. 

Pel que fa a les operacions, és indispensable que el model multiresolució pugui fer aquests canvis d'intervals de temps, els quals s'aconsegueixen amb les operacions d'\emph{agregació} i \emph{consolidació}. En el model de dades multiresolució es defineixen els operadors específics per a aquestes tasques anomenats \emph{agregadors d'atributs}


A banda de les estructures, en el model també es defineixen els
operadors que permeten tractar les dades; ja sigui per operar amb les
estructures o bé per a fer consultes.



\todo{sobre àlgebra relacional i càlcul relacional}
Els models que definim es basen en l'àlgebra relacional. Es podria definir el mateix des del càlcul relacional.


Perquè l'àlgebra?


http://cacm.acm.org/magazines/2014/2/171675-a-new-type-of-mathematics/fulltext

for a century, mathematicians have considered set theory to be an adequate basis for formalizing all of mathematics. Starting with concepts like the null set (corresponding to zero) and the set containing only the null set (corresponding to one), one can, in principle, systematically construct all the objects of mathematics. In practice, however, the process is clunky and time-consuming—and therefore, rare. The proponents of HoTT hope it will provide easier and more intuitive tools that will allow rigorously formalized mathematics to become standard practice. Tot i que aquests proposen HoTT: HoTT is based on a mathematical framework called type theory. Unlike sets, which are like bags that can contain various kinds of object, objects of a particular type have specific rules about how they can be manipulated. They are reminiscent of the data types that help enforce rigor in high-level programming languages, but the mathematical version of types can be more elaborate; 



\section{Resum dels objectius models}


En el model de SGST s'observen algunes patologies que poden presentar les sèries temporals. El model de SGSTM soluciona algunes d'aquestes patologies:

\begin{itemize}
\item Regularitza les sèries temporals
\item Tracta i validar les sèries temporals: gestiona els casos de dades errònies o desconegudes i marca quan hi ha valors erronis.
\item És una solució de compressió per a quantitats enormes de dades
\end{itemize}


Però el model de SGSTM també es pot fer servir per altres aplicacions:

* Regularitzar en línia (temps real) una sèrie temporal en diferents períodes de mostreig

* Tenir unes vistes (consultes) a punt (ja processades) amb diferents resolucions d'una sèrie temporal

* Comprimir per decimació (downsampling) o bé farcir forats (reconstrucció del senyal)






% This paper focuses on Data Base Management Systems (DBMS) that store
% and treat data as time series. Traditional DBMS, as is ones derived
% from relational model, are not adequate for these cases as they do not
% have enough facilities to manage and retrieve time series
% information.



% * Temporal databases. Basades en esdeveniments. Data mining basat en sèries temporals definides per parelles temps-valor; calen TSMS

% * Alta dimensió sèries temporals, cal reduir-la. Es conserven els segments de temps més interessants; multiresolució

% * Multiresolució, diferents resolucions, es pot treballar amb més o menys dades segons convingui

% *Cal saber canviar de resolució, exemple transformar dades periòdiques d'un mes a un any.

% * Aggregates, una sèrie temporal pot estar mostrant diferent informació. ex: mitjana, màxim, valor al final del període, ...

% * Les sèries temmporals tenen una metainformació que cal guardar en una base de dades relacional (localització, etiquetes de classificació, últim valor mesurat, unitats, etc.)

% * Disseny del model de TSMS, aleshores veurem si una TSMS pot ser implementada com a camp d'una altra DBMS o si els DBMS no són capaços de manipular TS adequadament i cal implementar TSMS específics.

% * Calendari, passa a segon terme. El temps es defineix com universal i constant (semblant a Unix Time Epoch). Aquests temps es pot convertir a calendari. Cal definir la interacció usuari/calendari amb temps universal.

% * El temps és un nom donat al camp, qualsevol objecte que tingui la mateixa interfície que el temps pot funcionar. En el cas del valor pot ser qualsevol objecte, s'exemplifica amb reals per facilitar-ne la comprensió i per ser el més proper al time series analysis: statistical methods focused on sequences of values representing a single numeric variable [llibre-last].


% * Representació: Entre dos punts de mesura, quin valor pren la sèrie temporal?.

% Si el volum de dades és gran no hi ha cap altre manera d'abordar-les directament que amb computació intensiva paral·lela [tenim alguna citació d'això?]. Per altra banda es pot intentar estudiar el problema per tal de veure si es pot trencar en trossos, mirar-lo des d'una altra perspectiva, etc. que en simplifiqui els càlculs. Això és el que fem amb el model SGSTM, dir que les sèries temporals en podem seleccionar prèviament informació.










% \dots
% \ldots
% \cdots
% \vdots
% \ddots diagonal dots
% \iddots  inverse diagonal dots (requires the mathdots package)
% \hdotsfor{n} to be used in matrices, it creates a row of dots spanning n columns

% A_1,A_2,\dotsc,   with commas
% A_1+\dotsb+A_N    with binary operators/relations
% A_1 \dotsm A_N    multiplication dots
% \int_a^b \dotsi   with integrals
% A_1\dotso A_N     other dots




%%% Local Variables:
%%% TeX-master: "main"
%%% End:
% LocalWords: buffer buffers






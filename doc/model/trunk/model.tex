\chapter[Model RRD]{Model dels SGBD Round Robin}\label{cap:model-rrd}

En aquest capítol es dissenya un model matemàtic per a les bases de dades Round Robin (RRD). Aquest model està inspirat en el sistema de gestió de bases de dades RRDtool i és el resultat de l'abstracció dels conceptes que s'han exposat als capítols %\ref{cap:rrdtool} i \ref{cap:rrdtool-etapes}. 

A l'inici del capítol, s'introdueix el concepte de model matemàtic per a les bases de dades. 
A continuació, es defineixen els conceptes necessaris per a finalment definir una base de dades Round Robin. 
Al final del capítol, es resumeix breument la informació exposada.

  

\section{Introducció}

Segons Date,~\cite{date}, ``una base de dades és un contenidor informàtic per a una co\l.lecció de dades''. El sistemes informàtics que tracten amb bases de dades s'anomenen sistemes de gestió de bases de dades (SGBD) i tenen l'objectiu d'emmagatzemar informació i permetre consultar i afegir aquesta informació  per part dels usuaris.
Per complir aquests objectius, els SGBD ofereixen a l'usuari diferents operacions a fer amb la base de dades, com per exemple crear-la, afegir dades o consultar informació a partir de les dades emmagatzemades.

Els SGBD es basen en teories matemàtiques que reben el nom de model de dades, un SGBD és una implementació d'un model de dades.
Segons Date, ``un model de dades és una definició abstracta, auto continguda i lògica dels objectes, de les operacions i  de la resta que conjuntament constitueixen la màquina abstracta amb la que els usuaris interactuen. Els objectes permeten modelar l'estructura de les dades. Les operacions permeten modelar el comportament''.

Les bases de dades Round Robin són bases de dades que contenen sèries temporals. Les sèries temporals són una co\l.lecció de dades mesurades en diferents instants de temps i necessiten un tractament adequat per part de la base de dades. 
El model de dades Round Robin és una solució d'emmagatzematge per a les sèries temporals que, resumint, consisteix a repartir la informació d'una sèrie temporal en intervals de temps diferents.

El model de dades Round Robin es dissenya per primera vegada en aquest capítol. 
Actualment existeix un SGBD, RRDtool, que implementa conceptes susceptibles d'esdevenir un model de dades però enlloc s'ha recollit com a tal. A partir de l'anàlisi i l'abstracció profunda dels conceptes de RRDtool s'ha dissenyat el model de dades Round Robin, batejat segons el nom que té a RRDtool (\emph{Round Robin Database tool}). Els conceptes de RRDtool s'han exposat detalladament als capítols %\ref{cap:rrdtool} i \ref{cap:rrdtool-etapes}. 


Així doncs, a continuació es presenta el model de dades Round Robin. 
Es defineixen els objectes principals d'estudi: les \emph{mesures} i les \emph{sèries temporals}. Les \emph{mesures} són dades mesurades en un instant de temps i les \emph{sèries temporals} són co\l.leccions de mesures, a cada base de dades Round Robin hi ha emmagatzemada una sèrie temporal. 

El model de dades Round Robin s'estructura a partir de \emph{discs Round Robin}, els quals  acumulen temporalment les \emph{mesures} en un \emph{buffer} per tal de tractar-les abans d'emmagatzemar-les  a un \emph{disc}. El tractament principal consisteix en canviar els intervals de temps entre \emph{mesures} amb l'objectiu de compactar la informació de la sèrie temporal.
Així doncs, la sèrie temporal queda emmagatzemada en intervals de temps diferents, repartits en els \emph{discs Round Robin}. 

Pel que fa a les operacions, és indispensable que el model Round Robin pugui fer aquests canvis d'intervals de temps, els quals s'aconsegueixen amb les operacions d'\emph{interpolació} i \emph{consolidació}. En el model de dades Round Robin d'aquest capítol també es defineixen operacions per crear una base de dades Round Robin, per inserir-hi mesures i per representar sèries temporals.



\section{Sèrie temporal}

\subsection{Regularitat de les sèries temporals} 

Sigui $S=\{m_0,\ldots,m_k\}$ una sèrie temporal, $t$ un instant de temps i $\delta$ una durada de temps, en l'interval de temps $i_0=[t,t+\delta]$ i els seus múltiples $i_j=[t+j\delta \,,\, t+(j+1)\delta]: j=0,1,2,\ldots$,  
la sèrie temporal $S$ és de naturalesa diferent segons la situació dels temps $T(m_i)$ en els intervals de temps $i_j$.
En aquest context, aquests intervals de temps s'anomenen intervals de mostreig, $\delta$ s'anomena període de mostreig i $t$ s'anomena temps d'inici del mostreig.

Una sèrie temporal és regular quan les mesures són equidistants en el temps, tal com ho anomenen a \cite{last:hetland}.

\begin{definition}[Sèrie temporal regular]
  Sigui $S=\{m_0,\ldots,m_k\}$ una sèrie temporal, $t$ un instant de
  temps i $\delta$ una durada de temps. Direm que $S$ és regular si i
  només si $\forall m \in S(T(\min(S),\infty):T(m) - T(\ant(m)) = \delta$ i
  $T(\min(S))=t$. 
\end{definition}

Si una sèrie temporal és regular, l'anomenem  sèrie temporal mostrejada
regularment amb període de mostreig $\delta$.


Una sèrie temporal és no regular quan no és regular. 
En les sèries temporals no regulars es poden distingir tres casos: temps real, ultramostreig i inframostreig.

Una sèrie temporal és de temps real quan a cada interval de mostreig hi ha una i només una mesura. L'interval de mostreig pot estar acotat per una durada anomenada termini.

\begin{definition}[Sèrie temporal de temps real]
  Sigui $S=\{m_0,\dotsc,m_k\}$ una sèrie temporal, $t$ un instant de
  temps, $\delta$ una durada de temps i $D$ una durada que indica
  termini. Direm que $S$ és de temps real si i només si $D\leq\delta$
  i $\forall n\in\{0,\ldots,|S|-1\}: \exists!m \in
  S(t+n\delta,t+n\delta+D]$.  Aleshores la sèrie temporal està
  mostrejada en temps real per al temps de mostreig $\delta$ amb
  compliment del termini $D$.
\end{definition}

Si una sèrie temporal és de temps real, l'anomenem  sèrie temporal mostrejada
en temps real amb període de mostreig $\delta$ i compliment del termini $D$.
Si $D=\delta$, es pot anomenar que $S$ és una sèrie temporal de temps real sense termini.


% \paragraph{Ultramostreig} Una sèrie temporal està ultramostrejada (\emph{upsampling}) quan a cada interval de mostreig hi ha una mesura o més d'una. 
% \[
% \text{Ultramostrejada?}: \text{Sèrie temporal} \times T_0 \times \delta \longrightarrow \text{Booleà}
% \]

% Una sèrie temporal $S$ està ultramostrejada ssi $S$ no és de temps real i $\exists m_i=(v_i,t_i)\in S:T_0+(n-1)\delta \leq t_i < T_0+n\delta:\forall n\in\{1,\ldots,|S|\}$.

% \paragraph{Inframostreig} Una sèrie temporal està inframostrejada (\emph{downsampling}) quan en algun interval de mostreig no hi ha cap mesura. 
% \[
% \text{Inframostrejada?}: \text{Sèrie temporal} \times T_0 \times \delta \longrightarrow \text{Booleà}
% \]

% Una sèrie temporal $S$ està inframostrejada ssi $\nexists m_i=(v_i,t_i)\in S:T_0+(n-1)\delta \leq t_i < T_0+n\delta:\forall n\in\{1,\ldots,|S|\}$.

\subsection{Representació de sèries temporals}

\textcite{last:keogh}, cita vàries representacions per les sèries temporals com per exemple \emph{Fourier Transforms}, \emph{Wavelets}, \emph{Symbolic Mappings} o \emph{Piecewise Linear Representation} (PLR), però assenyala aquesta última com la representació més utilitzada. 
La PLR, funció definida a trossos lineal, és l'aproximació d'una sèrie temporal $S$, de llargada $n$, amb $K$ segments rectes. Els segments podrien ser polinomis de qualsevol grau, però la manera més comuna de representar sèries temporals és amb funcions lineals, segons Keogh, \cite{keogh02}.
Per aproximar el segment $S(t_a:t_b]$ d'una sèrie $S$, Keogh defineix dues tècniques: interpolació lineal, la recta que connecta $t_a$ i $t_b$, i regressió lineal, la millor recta que aproxima per mínims quadrats el segment entre $t_a$ i $t_b$.

Però també es pot representar una sèrie temporal amb una funció esglaó (\emph{step} o \emph{staircase function}); és a dir, amb una funció definida a trossos constant (\emph{piecewise constant representation}).
La representació a trossos constant és utilitzada en electrònica als convertidors digital-analògic (DAC, \emph{digital-to-analog converter}). En aquest cas, un senyal discret es considera una sèrie temporal i per reconstruir el senyal continu típicament s'aplica el model de \emph{zero-order hold}, equivalent a la representació a trossos constant,  o el de \emph{first-order hold},  equivalent a la representació a trossos lineal.
El model de \emph{zero-order hold} consisteix en mantenir constant cada valor fins al proper. S'obté una representació a trossos constant que en electrònica s'anomena seqüència de pulsos rectangulars (\emph{rectangular pulses}).

%http://en.wikipedia.org/wiki/Piecewise

%http://ca.wikipedia.org/wiki/Funció_definida_a_trossos

%http://en.wikipedia.org/wiki/Rectangular_function

%http://en.wikipedia.org/wiki/Step_function

% Piecewise Aggregate Approximation (PAA) \cite{keogh00}: aproxima una sèrie temporal partint-la en segments de la mateixa mida i emmagatzemant la mitjana dels punts que cauen dins del segment. Redueix de dimensió $n$ a dimensió $N$

% Adaptive Piecewise Constant Approximation (APCA) \cite{keogh01}: com el PAA però amb segments de mida variable.

A continuació,  la representació  d'una sèrie temporal segons el model de \emph{zero-order hold} s'estén per diferents continuïtats en els intervals de temps de representació.

Sigui $S$ una sèrie temporal, es defineix $S(t)$ com la representació
de la sèrie temporal contínuament al llarg del temps $t$.  En primer
lloc, es representa amb \emph{zero-order hold} a partir de funcions
graó contínues per la dreta (\emph{right-continuous}).

\begin{definition}[Representació amb \emph{zero-order hold}]
Sigui $S=\{m_0,\ldots,m_k\}$ una sèrie temporal, la representació  $S(t)$ amb \emph{zero-order hold} es defineix
\[
\forall t \in \mathbb{R} ,\forall m \in S: S(t) =
\begin{cases}
  V(\min S) & \text{si } t < T(\min S) \\
  V(m) & \text{si }  t\in [T(m),T(\seg m))
\end{cases}
\]
\end{definition}

En segon lloc, es representa $S(t)$ amb \emph{zero-order hold} centrada en
l'interval, definit també a partir de funcions graó contínues per la
dreta.

\begin{definition}[Representació amb \emph{zero-order hold} centrada en l'interval]
  Sigui $S=\{m_0,\ldots,m_k\}$ una sèrie temporal, la representació
  $S(t)$ amb \emph{zero-order hold} centrada en l'interval es defineix
\[
\forall t \in \mathbb{R}  ,\forall m \in S:
S(t) =  
\begin{cases}
  V(m) & \text{si } t = \frac{T(\ant m)+T(m)}{2} \\
  V(m) & \text{si } t\in \left( \frac{T(\ant m)+T(m)}{2},\frac{T(m)+T(\seg m)}{2} \right) \
\end{cases}
\]
\end{definition}

En tercer lloc, es representa $S(t)$ amb \emph{zero-order hold} cap enrere, ara definit a partir de funcions graó contínues per l'esquerra.
\begin{definition}[Representació en \emph{zero-order hold} cap enrere]
  Sigui $S=\{m_0,\ldots,m_k\}$ una sèrie temporal, la representació
  $S(t)$ amb \emph{zero-order hold} cap enrere es defineix
\[
\forall t \in \mathbb{R}  ,\forall m \in S:
S(t) =  
\begin{cases}
  V(\max S) & \text{si } t > T(\max S) \\
  V(m) & \text{si }  t\in (T(\ant m),T(m)]
\end{cases}
\]
\end{definition}

Sigui $S$ una sèrie temporal regular i $\delta$ una durada de temps, aleshores la representació de $S(t)$ amb \emph{zero-order hold} és la mateixa que la de $S(t-\delta)$ amb \emph{zero-order hold} cap enrere i és la mateixa que la de $S(t-\frac{\delta}{2})$ centrada en l'interval. 





\section{Resum}

Aquest capítol s'acaba amb un resum dels conceptes exposats en el model de dades Round Robin. Una base de dades Round Robin és un sistema informàtic d'emmagatzematge d'una sèrie temporal entesa com una  una co\l.lecció de dades mesurades en diferents instants de temps.

A la base de dades, la sèrie temporal queda estructurada com s'ha esquematitzat a  la figura~\ref{fig:model:bdstm}. És una forma compacta d'emmagatzemar la sèrie temporal de manera que queda repartida segons diferents funcions d'interpolació i períodes de mostreig. Aquest repartiment té lloc en els diferents discs Round Robin, els quals fan ús del seu buffer per interpolar les mesures i fan ús del seu disc per consolidar-les. 

El conjunt de discs Round Robin constitueixen la part principal d'una base de dades Round Robin tot i que, com a l'esquema, hi pot haver un buffer d'entrada de mesures comú per tal de regularitzar la sèrie temporal des d'un principi, ja que el pas de no regular a regular requereix interpoladors més complicats.


En el capítol%~\ref{cap:roundrobinson},
 utilitzant el llenguatge de programació Python es dissenya, a nivell acadèmic, un sistema de gestió de bases de dades que implementa el model de dades Round Robin tal com s'ha definit en aquest capítol.


En resum, a partir del model de dades Round Robin descrit en aquest capítol per una banda es poden estudiar quin efecte té una configuració determinada de paràmetres i per altra banda es poden dissenyar sistemes de gestió de bases de dades assegurant que implementen el model i per tant que tenen el funcionament desitjat.




% \section{Tractament dels desconeguts}


%operacions amb nan de octave i matlab
%http://biosig-consulting.com/matlab/NaN/
% The NaN-toolbox v2.0: A statistics and machine learning toolbox for Octave and Matlab®
% for data with and w/o MISSING VALUES encoded as NaN's.


% La censura per interval és una verificació possible. En aquesta, un valor $v$ es considera desconegut quan $v<L_{m}$ o $v>L_{M}$.


% Sigui $m=(v,t)$ una mesura, $m$ és desconeguda si $v$ és desconegut.

% Sigui $S=\{m_0,\ldots,m_k\}$ una sèrie temporal i $H$ un termini de temps, una mesura $m_i=(v_i,t_i)\in S$ és desconeguda si, donada la mesura anterior $m_{i-1}=(v_{i-1},t_{i-1})$, $t_i - t_{i-1} > H$.


% Sigui $f$ un interpolador, en el moment de calcular la mesura de consolidació $f$ decideix si és desconeguda
% \[
% desconeguda?: \text{Sèrie temporal} \times \text{interval} \longrightarrow \text{Mesura}
% \]

% Sigui $S=\{m_0,\ldots,m_k\}$ una sèrie temporal, $f$ un interpolador, $i=[T_0,T_f]$ un interval de temps i $\alpha$ un llindar, la mesura de consolidació calculada per l'interpolador $f$ és desconeguda ssi  
% \[
% \frac{t_d }{T_f - T_0} > \alpha :
% \]
% \[
% :t_d = t_{d0} + t_{df} + \sum\limits_{i=1}^{k-1}(t_i-t_{i-1}) : v_k = 'desconegut':
% \]
% \[
% : t_{d0} = \left\{\begin{array}{l} t_0-T_0 \text{ si } v_0 = 'desconegut' \\ 0\end{array}\right. ,
% t_{df} = \left\{\begin{array}{l} T_f-t_{k-1} \text{ si } v_k = 'desconegut' \\ 0\end{array}\right. :
% \]
% \[
% :k=|S|-1,(v_k,t_k)=m_k\in S' :S'= S_{T_0:T_f} \cup \{min(S_{T_f:\infty})\}
% \]


\section{Consultes}

\subsection{Selecció temporal d'una sèrie temporal}

Sigui $S$ una sèrie temporal i $i=[t_0,t_f]$ un interval de temps,
per una banda s'ha definit l'interval sobre la seqüència d'una sèrie temporal $S(t_0,t_f]$  i per altra banda s'ha definit la representació contínua $r$ d'una sèrie temporal $S(t)$. Per seleccionar un interval temporal cal tenir en compte tant l'interval sobre la seqüència com la representació contínua; aquesta selecció temporal s'anota com selecció de $S$ en $i$ amb representació $r$ o bé $S[t_o,t_f]^r$. 

\begin{definition}[Selecció temporal de $S$ en $i$ amb representació
  $r$]
  \[
  \text{selecció}: \text{Sèrie Temporal} \times \text{interval de
    temps} \times \text{representació} \longrightarrow \text{Sèrie
    Temporal}
  \]
  \[
  S = \{m_0 , \ldots , m_k\}  \times i = [t_0,t_f] \times r \longrightarrow S'
  \]
  \[
  \forall  t \in i: S' = S(t)^r 
  \] 
\end{definition}

A continuació s'exemplifica utilitzant la representació \emph{zoh} enrere.


\begin{definition}[Selecció temporal de $S$ en $i$ amb representació
  \emph{zohe}]
  Sigui $S$ una sèrie temporal, $i=[t_0,t_f]$ un interval de temps i
  \emph{zohe} la representació $S(t)$ amb \emph{zero-order-hold} cap
  enrere, es defineix la subsèrie $S[t_0,t_f]^{\text{zohe}}\subseteq
  S$ com la sèrie temporal 
  \[
  S[t_0,t_f]^{\text{zohe}} = (S \cup \{m\})(t_0,t_f] : m=(v,t_f) \in S:
  \]
  \[
  v=  V(\inf(S-S[-\infty,t_f)))
  \]
  % $S-S[-\infty,t_f)$ seria equivalent a $S[t_f,+\infty]$
\end{definition}

Observeu que, sigui $t$ un instant de temps, la selecció de $S$ en $[t,t]$ és equivalent a la representació contínua $S(t)$. 



\subsection{Selecció de la resolució d'una sèrie temporal}

La selecció de la resolució d'una sèrie temporal permet canviar, en el
context d'una representació, la resolució a una de marcada per un
conjunt d'instants de temps. A diferència d'un buffer, la selecció de
resolució no permet aplicar interpoladors ni obliga a que la sèrie
temporal resultant sigui regular.

Sigui $S$ una sèrie temporal, $i= \{t_0,t_1,\dotsc,t_n\}$ un conjunt
d'instants de temps i la representació contínua $r$ de la sèrie
temporal $S(t)$, la selecció de resolució s'anota com resolució de $S$
en $i$ amb representació $r$ o bé $S[i]^r$.

\begin{definition}[Selecció de la resolució de $S$ en $i$ amb representació
  $r$]
  \[
  \text{resolució}: \text{Sèrie Temporal} \times \text{instants de
    temps} \times \text{representació} \longrightarrow \text{Sèrie
    Temporal}
  \]
  \[
  S = \{m_0 , \ldots , m_k\} \times i = \{t_0,t_1,\dotsc,t_n\} \times r
  \longrightarrow S'
  \]
  \[
  t_0 < t_1 < \dotsb < t_n:
  \]
  \[
  S' = S[-\infty,t_0]^r \cup  S[t_0,t_1]^r \cup \dotsb \cup S[t_{n-1},t_n]^r
  \] 
  % Es podria fer recursiu
  % \[
  % t_f = \sup(i): i_n = i - t_f: t_a = \sup(i_n):
  % S' = \left\{\begin{array}{ll}
  %     \{\} & \text{si } |i| = 0 \\
  %     S[i_n]^r \cup S[t_a,t_f]^r 
  %   \end{array}\right.
  % \] 
\end{definition}




\subsection{Unió temporal de sèries temporals}

$S_1 \cup^r S_2$. En principi per la unió les sèries temporals han de tenir la mateixa dimensió pel que fa a mesures multivaluades.

\begin{definition}[Unió temporal de $S_1$ i $S_2$ amb representació
  $r$]
  \[
  \text{unió}: \text{Sèrie Temporal} \times \text{Sèrie temporal}
  \times \text{representació} \longrightarrow \text{Sèrie Temporal}
  \]
  \[
  S_1 = \{m_0^1 , \ldots , m_{k1}^1\}  \times S_2 = \{m_0^2 , \ldots , m_{k2}^2\} \times r \longrightarrow S'
  \]
  \[
  t_1=T(\inf S_1), t_2=T(\sup S_1):
  \]
  \[
  S' = S_1 \cup  ( S_2 - S_2[t_1,t_2]^r )
  \] 

  Nota: la unió temporal no és commutativa. 
  % Però, $(S_1 \cup^r S_2) \cup $(S_2 \cup^r S_1) = S_1 \cup S_2$?
\end{definition}


\subsection{Consulta temporal d'una RRD}

Abans de fer una consulta temporal pot fer falta fer una selecció dels discs Round Robin amb el mateix interpolador.


\begin{definition}[Consulta temporal de $RRD$ amb representació $r$]
  \[
  \text{consulta}: \text{Base de dades RR} \times
  \text{interpolador} \times \text{representació} \longrightarrow
  \text{Sèrie Temporal}
  \]
  \[
  M = \{R_0 , \ldots , R_d\} \times r \longrightarrow S'
  \]
  \[
  \forall R_j=(B_j,D_j) \in M : B_j=(S_{Bj},\tau_j,\delta_j,f_j), D_j=(S_{Dj},k_j):
  \]
  \[
  \delta_0 < \delta_1 < \delta_2 < \dots < \delta_d : 
  \footnote{S'assumeix que en una RRD no hi ha discs Round Robin amb la mateixa informació.}
  \]
  \[
  S' = S_{D0} \cup^r  S_{D1} \cup^r  S_{D2} \cup^r \dots \cup^r S_{Dd}
  \]
  % Es podria fer recursiu
  % \[
  % \delta_0 = min (\{\delta_j \in B_j  \}): \forall R_j=(B_j,D_j) : 
  % R_0=(B_0,D_0)  \in M : B_0=(S_{B0},\tau_0,\delta_0,f_0), D_0=(S_{D0},k_0):
  % S' = \left\{\begin{array}{ll}
  %     \{\} & \text{si } |M| = 0 \\
  %     S_{D0} \cup^r (\text{ consulta } (M-\{R_0\})) 
  %   \end{array}\right.
  % \] 
  
\end{definition}


\subsection{Fusió temporal de dues sèries temporals}

Fusió (join) de dues sèries temporals.


\begin{definition}[Fusió temporal de $S1$ i $S2$ amb representació $r$]
  \[
  \text{fusió}: \text{Sèrie Temporal} \times
  \text{Sèrie Temporal} \times \text{representació} \longrightarrow
  \text{Sèrie Temporal}
  \]
  \[
  S_1 = \{m_0^1 , \ldots , m_{k_1}^1\} \times S_2 = \{m_0^2 , \ldots ,
  m_{k_2}^2\} \times r \longrightarrow S'
  \]
  \[
  |S'| <= k_1 + k_2 : m' = (v',t'):
  \]
  \[
  t = \{t^1 \, | \, \forall m^1=(v^1,t^1) \in S_1\} \cup \{t^2 | \forall
  m^2=(v^2,t^2) \in S_2\}:
  \]
  \[
  S' = \{m'=(t',v_1',v_2') \, | \, (t',v_1') \in S_1[t]^r \wedge (t',v_2') \in S_2[t]^r \}
  \]

\end{definition}




% \dots
% \ldots
% \cdots
% \vdots
% \ddots diagonal dots
% \iddots  inverse diagonal dots (requires the mathdots package)
% \hdotsfor{n} to be used in matrices, it creates a row of dots spanning n columns

% A_1,A_2,\dotsc,   with commas
% A_1+\dotsb+A_N    with binary operators/relations
% A_1 \dotsm A_N    multiplication dots
% \int_a^b \dotsi   with integrals
% A_1\dotso A_N     other dots

%%% Local Variables:
%%% TeX-master: "main"
%%% End:
% LocalWords:  Round  buffer buffers






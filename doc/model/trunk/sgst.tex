
\chapter{Model SGST}

En aquest capítol es defineixen els objectes que ens permeten modelar l'estructura de les dades.

Dos models estructurals:

* Hi ha un model pels SGST (TSMS) que inclou mesura i sèries temporals.

* Hi ha un model pels SGSTM (MTSMS) que té buffer, discs i mtsdb, els quals inclouen el model de sèrie temporal del SGST.




\section{Model estructural de dades}

Una sèrie temporal és una relació de temps i valors. A cada parella
temps-valor l'anomenem mesura. Així doncs, una sèrie temporal és un
conjunt de mesures. Una mesura és un tuple temps-valor.



Una mesura és un valor mesurat en un instant de temps i una sèrie
temporal és un co\l.lecció de mesures.





\subsection{Temps}

Utilitzem el temps com un valor que ens permet ordenar les mesures.  A
tal efecte, el domini del temps es defineix com un conjunt tancat
(compactificat) i amb ordre total. No obstant, pot ser tant un conjunt
finit com infinit.

Per facilitar la comprensió, en el document utilitzarem el conjunt de
reals com a conjunt pels temps. Concretament, per a complir que sigui
un conjunt tancat usarem el conjunt estès de nombres reals
$\bar{\mathbb{R}} \in \mathbb{R} \cup
\{+\infty,-\infty\}$, \parencite{wiki:extendedreal,cantrell:extendedreal},
també anomenat recta real acabada.


El conjunt estès de nombres reals té dos punts límits corresponents al
valor impropi infinit, aleshores en notació d'interval el conjunt es
pot escriure com $\bar{\mathbb{R}} \in [-\infty,+\infty]$.  Més
endavant a la definició~\ref{def:model:mesura_indefinida} es detallen
algunes propietats induïdes a les mesures com a resultat d'aquesta
extensió.

Les relacions d'ordre i algunes operacions aritmètiques s'estenen al
conjunt $\bar{\mathbb{R}}$, \cite{cantrell:extendedreal}.  Algunes
expressions esdevenen indefinides (p.ex.\ $0/0$) i altres depenen del
context, com és el cas de l'expressió indeterminada $0 \times \infty$ que
per exemple en la teoria de la mesura habitualment es defineix com $0 \times
\infty = 0$, \cite{wiki:extendedreal}.


El conjunt dels reals és un espai mètric ja que té definida una funció
distància (o mètrica), com per exemple la distància euclidiana. Com a
conseqüència, ens permet distingir entre instants de temps (els
elements del conjunt) i durades (la mètrica). Observant els instants
de temps com a punts en la recta real i les durades com a segments de
la recta real, es pot definir el temps com a sistema de coordenades
especificant un instant com a marc de
referència, \parencite{iep:time-supplement,wiki:coordinate}.


\begin{definition}[Temps]
  \label{def:model:temps}
  Siguin $t^i_i$ i $t^i_j$ dos instants de temps, observem la quantitat
  de temps o la durada $t^d$ com un valor $t^d \in\bar{\mathbb{R}}$
  que mesura la distància en unitats de temps entre dos temps
  absoluts $t^d = t^i_i - t^i_j$.
  
  Sigui $t^d$ una durada i $t^{R}$ un temps absolut de referència,
  observem un instant de temps $t^i$ com un valor $t^i
  \in\bar{\mathbb{R}}$ que mesura la quantitat de temps respecte al
  temps de referència $t^i= t^{R} + t^d$ . Aquest valor de referència
  $t^{R}\in\mathbb{R}$ és també un instant de temps però que permet
  definir unívocament la posició de qualsevol altre instant de temps.


\end{definition}

En resum, els instants de temps es poden veure com una seqüència de
valors reals que indiquen esdeveniments amb ordre clarament definit i
entre dos instants de temps sempre hi ha una durada. Tant els instants
de temps com les durades s'expressen amb un real que té unitats de
temps. Aquestes unitats són 'segons' en sistema internacional.



\subsubsection{Calendari}
\textcite{dreyer94} situen els calendaris i les seves operacions com a
essencials en els SGST. Tanmateix, pot no ser necessari modelar les
dates de calendari en el model de temps. El temps és la línia contínua
de temps, el calendari són nom especials a certs punts de la línia de
temps. Només cal una eina que sigui capaç de convertir de noms a
instants de temps.

Per una banda, no afecta al model SGST que els calendaris siguin més o
menys complicats, en aquest cas només es veuen complicades les
funcions de conversió de temps a calendari i viceversa.  Per altra
banda, tampoc afecta que els calendaris siguin ambigus (p.ex.\ dos
noms per al mateix instant o instants sense nom) o que continguin
propietats impredictibles (p.ex.\ cas dels segons addicionals
(intercalats) en UTC) ja que la responsabilitat d'aquests problemes
correspon a la bona definició dels sistemes de calendari.


Unix time (posix) no incorpora els leap seconds.
Millor TAI, unix time (right), ja que és una mesura totalment lineal del temps. 
Unix time, UTC i TAI: http://lwn.net/Articles/504744/


\subsection{Valor}

El \gls{terme:SGBDR:valor} és qualsevol element que és d'un
\gls{terme:SGBDR:tipus}; és a dir, un objecte que
pertany a un determinat conjunt de valors i que té associat les
operacions que s'hi poden aplicar. Exemples de tipus de dades són els
enters, els reals, les cadenes de text i les estructures de dades com
vectors, llistes o \glspl{terme:SGBDR:relacio}.  \todo{vigilar amb
  date, ell en diu escalars i no escalars (amb components visibles) i
  per exemple considera que un punt és escalar}

El model de dades dels valors ha d'incloure una dada que defineixi el
valor indefinit. Més endavant a la
definició~\ref{def:model:mesura_valor_indefinit} es detallen les
propietats de les mesures amb valor indefinit. Seguint l'exemple amb
els reals, el valor indefinit es defineix amb el valor impropi infinit
del conjunt dels reals estès
projectivament, \parencite{cantrell:projectivelyextendedreal},
$\mathbb{R}^*\in\mathbb{R} \cup \{\infty\}$.  En aquest cas el valor
és un escalar però fàcilment es pot estendre el concepte a valors
multivaluats ${\mathbb{R}^*}^n$ que representin una co\l.lecció de
valors mesurats en el mateix instant de temps, tal i com fa per
exemple \textcite{assfalg08:thesis}.





\subsection{Mesura}\label{sec:model:mesura} 

Una mesura és una parella de temps i valor.

\begin{definition}[Mesura]
  \label{def:model:mesura}
  Definim \emph{mesura} com el tuple $(t,v)$, en el que $v$ és el
  valor de la mesura i $t$ és l'instant de temps en que s'ha pres
  aquesta mesura.
\end{definition}


Donada una mesura $m=(t,v)$ escriurem $V(m)$ per referir-nos a $v$ i
$T(m)$ per referir-nos a $t$.

Donades dues mesures és fàcil establir la relació d'ordre induïda pel
temps.

\begin{definition}[Relació d'ordre]
  \label{def:model:mesura-relacio-ordre}
  Sigui $m=(t_m,v_m)$ i $n=(t_n,v_n)$. Direm que $m\geq n$ si i solament
  si $t_m\geq t_n$.
\end{definition}


En les definicions de temps i valor s'han estès els conjunts amb
valors impropis, concretament s'ha exemplificat amb el conjunt estès
de nombres reals afí $\bar{\mathbb{R}} \in \mathbb{R} \cup
\{+\infty,-\infty\}$ i amb el projectiu $\mathbb{R}^*\in\mathbb{R}
\cup\{\infty\}$,
\parencite{cantrell:extendedreal,cantrell:projectivelyextendedreal}. Aquesta
extensió amb l'element impropi infinit ($\infty$) dóna com a resultat
unes mesures impròpies que anomenarem mesura de valor indefinit i
mesura indefinida.

\begin{definition}[Mesura de valor indefinit]
  \label{def:model:mesura_valor_indefinit}
  Definim \emph{mesura de valor indefinit} com el tuple $(t,v)$, en el
  que el valor és $v=\infty$ i l'instant de temps és
  $t\in\bar{\mathbb{R}}$.
\end{definition}

\begin{definition}[Mesura indefinida]
  \label{def:model:mesura_indefinida}
  Definim \emph{mesura indefinida} com el tuple $(t,v)$, en el que el
  valor és $v\in\mathbb{R}^*$ i l'instant de temps és
  $t\in\{+\infty,-\infty\}$.
\end{definition}

Així doncs, sigui $m$ una mesura, es podrà notar la mesura de valor
indefinit com $m=(t,\infty)$ i les mesures indefinides com
$m=(+\infty,v)$ per la positiva i $m=(-\infty,v)$ per la negativa, les
quals normalment s'anotaran també amb valor indefinit:
$m=(+\infty,\infty)$ i $m=(-\infty,\infty)$ respectivament.


Les mesures de valor indefinit es podran utilitzar en aquells casos en
els que el valor de la mesura és desconegut. Els valors desconeguts
són aquells valors que no existeixen (es desconeixen, \emph{missing
  data} ) o que s'ignoren (es descarten, \emph{censoring} o
\emph{truncation}). Els valors que no existeixen prenen el valor
desconegut en el moment de la mesura, en canvi els valors descartats
són marcats com a desconeguts després d'un processament de les dades.

Nota: en alguns sistemes es distingeix entre valors infinits
($\infty$) i valors indefinits (NaN, \emph{not a number}),
\cite{wiki:ieee754}. Aquest no és el cas de les definicions de mesures
indefinides presents.



\subsection{Sèrie temporal}
\label{sec:model:serietemporal}

Les sèries temporals són seqüències de mesures ordenades en el temps.
Tradicionalment s'anomenen sèries temporals tot i que també s'anomenen
seqüències temporals, per exemple a \cite{last:hetland}.

\begin{definition}[Sèrie temporal]
  \label{def:serie_temporal}
  Una sèrie temporal $S$ és un conjunt de mesures
  $S=\{m_0,\ldots,m_k\}$ sense temps repetits
  $\forall i,j: i\leq k, j\leq k, i\neq j : T(m_i)\neq T(m_j)$.
\end{definition}

Per ser un conjunt, les sèries temporals tenen mesura de cardinalitat.
\begin{definition}[Cardinal]
  Sigui $S=\{m_0,\ldots,m_k\}$ una sèrie temporal, definim el nombre
  de mesures que conté la sèrie temporal com el cardinal del conjunt
  $|S|=k+1$. Una sèrie temporal sense mesures és la sèrie temporal
  buida $S_\emptyset= \emptyset$, és a dir que no té cap element
  $|S_\emptyset|=0$.
\end{definition}

La relació definida a~\ref{def:model:mesura-relacio-ordre} indueix
sobre una sèrie temporal una relació d'ordre total. Com que la sèrie
temporal s'ha considerat finita i sense elements repetits, quan la
sèrie temporal no és buida això comporta l'existència d'un màxim i
d'un mínim.  Si $S$ és una sèrie temporal, $\max(S)$ i $\min(S)$ són
respectivament la mesura màxima i mínima d'$S$.

\begin{definition}[Màxim i mínim]
  Sigui $S=\{m_0,\ldots,m_k\}$ una sèrie temporal i $n\in S$ una
  mesura.  Direm que $n=\max(S)$ és el màxim de la sèrie temporal si i
  només si $\forall m \in S: n \geq m $.  Direm que $n=\min(S)$ és el
  mínim de la sèrie temporal si i només si $\forall m \in S: n \leq
  m$.
\end{definition}

El $\max(S)$ i el $\min(S)$ no estan definits quan la sèrie temporal
és buida: $S= \emptyset$. En
canvi, el suprem i l'ínfim estan definits per qualsevol
sèrie temporal tal com passa amb el conjunt estès de nombres reals,
\cite{cantrell:extendedreal}.  

\begin{definition}[Suprem i ínfim]
  Sigui $S=\{m_0,\ldots,m_k\}$ una sèrie temporal i $n\in S$ una
  mesura.  Direm que $n=\sup(S)$ és el suprem de la sèrie temporal si
  $n=\max(S)$ en cas que el màxim estigui definit o
  $n=(-\infty,\infty)$ en cas contrari.  Direm que $n=\inf(S)$ és
  l'ínfim de la sèrie temporal si $n=\min(S)$ en cas que el mínim
  estigui definit o $n=(+\infty,\infty)$ en cas contrari.
\end{definition}
Quan la sèrie temporal no és buida, per
ser un conjunt finit i d'ordre total, sempre hi ha un i només un màxim
i un mínim i per tant es corresponen amb el suprem i l'ínfim
respectivament.


Atesa la relació d'ordre induïda pel temps en una sèrie temporal
(def.\ \ref{def:model:mesura-relacio-ordre}) és possible definir el
concepte d'interval sobre la seqüència, semblant a com es fa a \cite{last:keogh,last:hetland}.

\begin{definition}[Interval]
  \label{def:model:st-interval}
  Sigui $S=\{m_0, \ldots, m_k\}$ una sèrie temporal. Definirem el subconjunt
  $S(r,t] \subseteq S$ com la sèrie temporal $S(r,t]=\{m\in S
  | r<T(m)\leq t\}$, a on $r$ i $t$ són dos instants de temps.

  També es defineix la subsèrie $S[-\infty,t)\subseteq S$ com la sèrie
  temporal $S[-\infty,t) = \{m\in S | T(\inf(S))\leq T(m) < t\}$.
\end{definition}
S'observa que la subsèrie $S(r,+\infty]\subseteq S$ és
equivalent a la sèrie temporal $S(r,+\infty] \equiv S(r,T(\sup(S))]$ i
anàlogament $S(-\infty,t] \equiv S(T(\inf(S)),t]$. També s'observa que les subsèries $S(t,t]\subseteq S$ i $S[t,t)\subseteq S$ són equivalents a la sèrie temporal buida $S(t,t] \equiv S[t,t) \equiv \emptyset$ ja que per ser els temps d'ordre total $\nexists T(m): t < T(m) \leq t$ o $\nexists T(m): t \leq T(m) < t$, respectivament. 
%Finalment, s'observa que la subsèrie $S(-\infty,+\infty]\subseteq S$ només és equivalent a la sèrie temporal original quan aquesta no conté la mesura indefinida negativa $S(-\infty,+\infty]\equiv S: (-\infty,v)\notin S$
 
També atenent a la relació d'ordre induïda pel temps en una sèrie temporal, es
defineix el concepte de mesura següent i mesura anterior en una
seqüència.

\begin{definition}[Successor i predecessor]
  Sigui $S=\{m_0, \ldots, m_k\}$ una sèrie temporal i $l\in S$ i $n$ dues
  mesures. Direm que $l$ és el successor de $n$ en $S$ i ho notarem
  com $l=\seg\limits_S(n)$ si i només si $l=\inf(S(T(n),+\infty])$.
  Direm que $l$ és el predecessor de $n$ en $S$ i ho notarem com
  $l=\ant\limits_S(n)$ si i només si $l=\sup(S[-\infty,T(n)))$.

Quan no hi hagi dubte de la sèrie temporal que marca l'ordre, per
exemple quan $n\in S$, podrem escriure $\seg(n)$ i $\ant(n)$.
\end{definition}
S'observa que s'obtenen mesures indefinides en els casos que la
mesura següent o anterior es calcula respectivament per la mesura
suprema o ínfima de la sèrie temporal: $\seg\limits_S(\sup
S)=(+\infty,\infty)$ i $\ant\limits_S(\inf S)=(-\infty,\infty)$.

De la definició anterior es dedueix que donada una sèrie temporal $S$
que no conté mesures indefinides i donada la mesura indefinida
$o=(+\infty,\infty)$, el predecessor de $o$ sempre és el suprem de la
sèrie temporal $\ant\limits_S( (+\infty,\infty) ) = \sup(S): \forall
m\in S: T(m)\in\mathbb{R}$.  % S\equiv S(-\infty,+\infty)
\emph{Demostració: Sigui $S$ una sèrie temporal i $o=(+\infty,\infty)$
  una mesura indefinida, el predecessor de $o$ en $S$ és una mesura
  $l=\ant\limits_S(o)$ que compleix
  $l=\sup(S[-\infty,T(o)))$. Substituint, s'obté que
  $l=\sup(S[-\infty,+\infty))=\sup(S-m):m\in S:T(m)=+\infty \notin
  \mathbb{R}$, i per tant com que $S$ no té mesures indefinides es
  demostra que $l=\sup(S)$.  } De manera semblant es pot demostrar que
$\seg\limits_S( (-\infty,\infty) ) = \inf(S): \forall m\in S:
T(m)\in\mathbb{R}$.




\subsection{Relació sèrie temporal}

Una sèrie temporal és una relació de temps i valors. A cada parella temps-valor l'anomenem mesura. Així doncs, una sèrie temporal és un conjunt de mesures.

Una sèrie temporal és un conjunt de mesures, així doncs s'observa com una relació de grau dos (relació binària)  a on la capçalera conté els atributs temps i valor, ambdós amb els dominis de temps i valor ja vistos com per exemple el tipus de dades 'reals estesos'. Inclou algunes restriccions més que les relacions:

* Els temps no poden estar repetits

* Els valors han de contenir el mateix tipus d'objecte.

Els temps no repetits indueixen un ordre temporal a les sèries temporals. Tot i així, les relacions, per ser conjunts, conserven la no definció d'un ordre dels elements. 


En el model relacional no hi ha ordre en els atributs a diferència de les relacions matemàtiques que tenen un ordre d'esquerra a dreta \parencite[sec.\ 5.3]{date:introduction}.

\subsection{Exemples}

\paragraph{Exemple 1}
Sèrie temporal $S_1$ on el temps i els valors pertanyen a $\bar{\mathbb{R}}$. Conté la mesura de valor 1 en el temps 5, la mesura de valor 3 en el temps 7 i la mesura de valor 1 en el temps 10. Modelada com a relació, és a dir com a parella capçalera i conjunt de valors certs, s'escriu com 
$S_1 = ( \{temps: \bar{\mathbb{R}}, valor: \bar{\mathbb{R}}\}, \{ \{temps:5,valor:1\}, \{temps:7,valor:3\}, \{temps:10,valor:1\} \} )$.

Degut al format esquemàtic, simplifiquem l'escriptura de les sèries temporals com a conjunt de tuples $(t,v)$ a on $t$ és el temps i $v$ és el valor. Així doncs la sèrie temporal $S$ es pot escriure de manera simplificada com a 
$S = \{ (5,1), (7,3), (10,1) \}$.

Tal com s'utilitza en les relacions, les sèries temporals es poden visualitzar com a taules. La sèrie temporal $S_1$ es visualitza com a taula a la \autoref{fig:model:serietemporal:real}.

\begin{figure}[tp]
  \centering
  \begin{tabular}{|c|c|}
    \multicolumn{2}{c}{$S_1$} \\ \hline
    $t$  & $v$ \\ \hline
    5  & 1 \\
    7  & 3 \\
    10 & 1 \\ \hline
  \end{tabular}
  \caption{Taula d'una sèrie temporal amb valors reals}
  \label{fig:model:serietemporal:real}
\end{figure}


\paragraph{Exemple 2}
Sèrie temporal $S_2$ on el temps pertany a $\bar{\mathbb{R}}$ i el valor pertany a  $\bar{\mathbb{R}}^3$; és a dir és un vector. Conté el valor (1,2,3) en el temps 5, el valor (3,4,5) en el temps 7 i el valor (1,2,3) en el temps 10.

De manera simplificada s'escriu com 
$S_2 = \{ (5,(1,2,3)), (7,(3,4,5)), (10,(1,2,3)) \}$ i es visualitza com a taula a la \autoref{fig:model:serietemporal:vector}. No obstant, es pot visualitzar de forma més còmode com a $S_2^b = \{ (5,1,2,3), (7,3,4,5), (10,1,2,3) \}$

\begin{figure}[tp]
  \centering
  \begin{tabular}{|c|c|}
    \multicolumn{2}{c}{$S_2$} \\ \hline
    $t$  & $v$ \\ \hline
    5  & (1,2,3) \\
    7  & (3,4,5) \\
    10 & (1,2,3) \\ \hline
  \end{tabular} \qquad
  \begin{tabular}[tp]{|c|c|c|c|}
   \multicolumn{4}{c}{$S_2^b$} \\ \hline
    $t$  & $v_1$ & $v_2$ & $v_3$ \\ \hline
    5  & 1 & 2 & 3 \\
    7  & 3 & 4 & 5 \\
    10 & 1 & 2 & 3 \\ \hline
  \end{tabular}

  \caption{Taula d'una sèrie temporal amb valors vectors}
  \label{fig:model:serietemporal:vector}
\end{figure}


\paragraph{Exemple 3} \emph{Valors relació}. \label{par:model:exemple-relvalues}
Sèrie temporal $S_3$ on el temps pertany a $\bar{\mathbb{R}}$ i el valor és una sèrie temporal del mateix format que en l'exemple 1. Conté les tuples de $S_1$ com a valors en el temps 1 i 2. 

De manera simplificada s'escriu com
$S_3 =  \{ (1,\{ (5,1), (7,3), (10,1) \}), 
(2,\{ (5,1),$ $(7,3),$ $(10,1) \}) \}$
i es visualitza com a taula a la \autoref{fig:model:serietemporal:serietemporal}.


\begin{figure}[tp]
  \centering
  \begin{tabular}{|c|c|}
    \multicolumn{2}{c}{$S_3$} \\ \hline
    $t$  & $v$ \\ \hline
    1 &   
       \begin{tabular}{|c|c|}
         \hline
         $t$  & $v$ \\ \hline
         5  & 1 \\
         7  & 3 \\
         10 & 1 \\ \hline
       \end{tabular} \\ \hline
    2 & 
       \begin{tabular}{|c|c|}
         \hline
         $t$  & $v$ \\ \hline
         5  & 1 \\
         7  & 3 \\
         10 & 1 \\ \hline
       \end{tabular} \\ \hline
  \end{tabular}
  \caption{Taula d'una sèrie temporal amb valors sèrie temporal}
  \label{fig:model:serietemporal:serietemporal}
\end{figure}


S'observa que la capçalera de $S3$ és $\{temps:\bar{\mathbb{R}},valor:
relacio\{temps:\bar{\mathbb{R}},valor:\bar{\mathbb{R}}\}\}$ \parencite[sec.\ 5.3]{date:introduction}. És a dir, el valor és de tipus relació que es defineix amb la capçalera de la relació on el temps i el valor pertanyen a $\bar{\mathbb{R}}$. Per tant, el valor de $S3$ és de tipus sèrie temporal amb valors reals. Cal insistir que \emph{tots} el valors de $S3$ han de pertànyer al mateix domini \parencite[sec.\ 5.4]{date:introduction}, el qual és $relacio\{temps:\bar{\mathbb{R}},valor:\bar{\mathbb{R}}\}$.



\paragraph{Exemple 4} \emph{Variable relació}.\todo{això no es pot fer, perquè no existeix el tipus relvar?? però les tuples poden contenir expressions?? No existeixen les tuplevar (Date rebutja ferotjament els apuntadors a dins dels DBMS) [Date on database :writings 2000-2006 / C.J. Date]}
Sèrie temporal $S_4$ on el temps pertany a $\bar{\mathbb{R}}$ i el valor és una referència a una sèrie temporal. Conté $S_1$ com a valors en el temps 1 i 2. 

De manera simplificada s'escriu com
$S_4 =  \{ (1,S_1) , (2,S_1) \}$ 
i es visualitza com a taula a la \autoref{fig:model:serietemporal:relvar}.

S'aplica el concepte de variable relació (\emph{relvar}) dels SGBDR \parencite[sec.\ 3.3]{date:introduction}.
Així doncs, cal notar que $S_4$  no és el mateix que $S_3$.
\begin{figure}[tp]
  \centering
  \begin{tabular}{|c|c|}
    \multicolumn{2}{c}{$S_4$} \\ \hline
    $t$  & $v$ \\ \hline
    1 & $S_1$ \\
    2 & $S_1$ \\ \hline
  \end{tabular}
  \caption{Taula d'una sèrie temporal amb valors \emph{relvar}}
  \label{fig:model:serietemporal:relvar}
\end{figure}


Relació de noms i sèries temporals $R =  ((nom:string,serie:relacio\{temps:\bar{\mathbb{R}},valor:\bar{\mathbb{R}}),\{ ('S_1',S_1),('S_2',S_2)  \})$

Sèrie temporal amb strings com a valors:
$N= ( (temps:\bar{\mathbb{R}},valor:string) ,\{ (1,'S_1') , (2,'S_1') \})$

Sèrie temporal com a variable relació de vista (relvar view)
$S_4 =  (N RENAME valor as nom) JOIN R$
\todo{cal definir una view}



\subsection{Naturalesa de les sèries temporals}


Perquè RRDtool diferencia entre comptadors i magnituds?

[segev87] diferencia entre step-wise constant, discret (potser aquest tal com se'l defineix són intervals temporals), continu. Ho anomena tipus de la sèrie temporal i diu que es poden definir interpolacions per cada una.


[John G. Proakis, Dimitris G. Manolakis 2007 Tratamiento digital de señales/Digital signal processing 4a ed pp11-12(segons wikipedia)] Acquisition: Discrete signals may have several origins, but can usually be classified into one of two groups:[1]
*By acquiring values of an analog signal at constant or variable rate. This process is called sampling.[2]
*By recording the number of events of a given kind over finite time periods. For example, this could be the number of people taking a certain elevator every day.



\subsubsection{Regularitat de les sèries temporals} 

Sigui $S=\{m_0,\ldots,m_k\}$ una sèrie temporal, $t$ un instant de
temps i $\delta$ una durada de temps, les mesures de la sèrie temporal
es poden localitzar en l'interval de temps $i_0=[t,t+\delta]$ i els
seus múltiples $i_j=[t+j\delta \,,\, t+(j+1)\delta]: j=0,1,2,\ldots$.
En processat de senyal aquests intervals de temps s'anomenen intervals
de mostreig, $\delta$ s'anomena període de mostreig i $t$ s'anomena
temps inicial del mostreig.  La sèrie temporal $S$ és de naturalesa
diferent segons la situació dels temps $T(m_i)$ en els intervals de
temps $i_j$.

Una sèrie temporal és regular quan les mesures són equidistants en el
temps, tal com ho anomenen a \cite{last:hetland}.

\begin{definition}[Sèrie temporal regular]
  Sigui $S=\{m_0,\ldots,m_k\}$ una sèrie temporal, $t$ un instant de
  temps i $\delta$ una durada de temps. Direm que $S$ és regular si i
  només si $\forall m \in S(T(\min(S),\infty):T(m) - T(\ant(m)) =
  \delta$ i $T(\min(S))=t$.
\end{definition}

Si una sèrie temporal és regular, l'anomenem sèrie temporal mostrejada
regularment amb període de mostreig $\delta$. Noteu que si es complís
la definició excepte que s'iniciés en el temps que exigim
$T(\min(S))=t$, aleshores la sèrie temporal seria equidistant però a
efectes de mostreig no la podríem anomenar regular; sí que seria una sèrie temporal de temps real (v.\ def.~\ref{def:st:tempsreal}).


Una sèrie temporal és no regular quan no és regular. 
En les sèries temporals no regulars es poden distingir tres casos: temps real, ultramostreig i inframostreig.

Una sèrie temporal és de temps real quan a cada interval de mostreig hi ha una i només una mesura. L'interval de mostreig pot estar acotat per una durada anomenada termini.

\begin{definition}[Sèrie temporal de temps real]\label{def:st:tempsreal}
  Sigui $S=\{m_0,\dotsc,m_k\}$ una sèrie temporal, $t$ un instant de
  temps, $\delta$ una durada de temps i $D$ una durada que indica
  termini. Direm que $S$ és de temps real si i només si $D\leq\delta$
  i $\forall n\in\{0,\ldots,|S|-1\}: \exists!m \in
  S(t+n\delta,t+n\delta+D]$.  Aleshores la sèrie temporal està
  mostrejada en temps real per al temps de mostreig $\delta$ amb
  compliment del termini $D$.
\end{definition}

Si una sèrie temporal és de temps real, l'anomenem  sèrie temporal mostrejada
en temps real amb període de mostreig $\delta$ i compliment del termini $D$.
Si $D=\delta$, es pot anomenar que $S$ és una sèrie temporal de temps real sense termini.


% \paragraph{Ultramostreig} Una sèrie temporal està ultramostrejada (\emph{upsampling}) quan a cada interval de mostreig hi ha una mesura o més d'una. 
% \[
% \text{Ultramostrejada?}: \text{Sèrie temporal} \times T_0 \times \delta \longrightarrow \text{Booleà}
% \]

% Una sèrie temporal $S$ està ultramostrejada ssi $S$ no és de temps real i $\exists m_i=(v_i,t_i)\in S:T_0+(n-1)\delta \leq t_i < T_0+n\delta:\forall n\in\{1,\ldots,|S|\}$.

% \paragraph{Inframostreig} Una sèrie temporal està inframostrejada (\emph{downsampling}) quan en algun interval de mostreig no hi ha cap mesura. 
% \[
% \text{Inframostrejada?}: \text{Sèrie temporal} \times T_0 \times \delta \longrightarrow \text{Booleà}
% \]

% Una sèrie temporal $S$ està inframostrejada ssi $\nexists m_i=(v_i,t_i)\in S:T_0+(n-1)\delta \leq t_i < T_0+n\delta:\forall n\in\{1,\ldots,|S|\}$.








\subsubsection{Representació de les sèries temporals}



La naturalesa indueix representacions?
Jo puc utilitzar qualsevol representació donada una sèrie temporal, però això em pot causa perjudici si no s'adiu amb la naturalesa.


La representació serveix per interpolar:

zoh, zoh cap enrere, lineal, etc.


Una sèrie temporal és la representació discreta d'una funció contínua. A partir de la sèrie temporal es pot definir una funció contínua. 

A teoria de senyal s'estudia com fer que aquesta s'aproximi a la real. Estudiant com a senyal fan: donada una sèrie temporal dir quina funció s'hi 'ajusta' més. 

Però jo puc preguntar donada una sèrie temporal quina funció representa i puc dir per representar a zohe és tal, per representar a lineal és qual. 

Potser millor dir-li interpretació?



\paragraph{Representació de sèries temporals}

\textcite{last:keogh}, cita vàries representacions per les sèries temporals com per exemple \emph{Fourier Transforms}, \emph{Wavelets}, \emph{Symbolic Mappings} o \emph{Piecewise Linear Representation} (PLR), però assenyala aquesta última com la representació més utilitzada. 
La PLR, funció definida a trossos lineal, és l'aproximació d'una sèrie temporal $S$, de llargada $n$, amb $K$ segments rectes. Els segments podrien ser polinomis de qualsevol grau, però la manera més comuna de representar sèries temporals és amb funcions lineals, segons Keogh, \cite{keogh02}.
Per aproximar el segment $S(t_a:t_b]$ d'una sèrie $S$, Keogh defineix dues tècniques: interpolació lineal, la recta que connecta $t_a$ i $t_b$, i regressió lineal, la millor recta que aproxima per mínims quadrats el segment entre $t_a$ i $t_b$.

Però també es pot representar una sèrie temporal amb una funció esglaó (\emph{step} o \emph{staircase function}); és a dir, amb una funció definida a trossos constant (\emph{piecewise constant representation}).
La representació a trossos constant és utilitzada en electrònica als convertidors digital-analògic (DAC, \emph{digital-to-analog converter}). En aquest cas, un senyal discret es considera una sèrie temporal i per reconstruir el senyal continu típicament s'aplica el model de \emph{zero-order hold}, equivalent a la representació a trossos constant,  o el de \emph{first-order hold},  equivalent a la representació a trossos lineal.
El model de \emph{zero-order hold} consisteix en mantenir constant cada valor fins al proper. S'obté una representació a trossos constant que en electrònica s'anomena seqüència de pulsos rectangulars (\emph{rectangular pulses}).

%http://en.wikipedia.org/wiki/Piecewise

%http://ca.wikipedia.org/wiki/Funció_definida_a_trossos

%http://en.wikipedia.org/wiki/Rectangular_function

%http://en.wikipedia.org/wiki/Step_function

% Piecewise Aggregate Approximation (PAA) \cite{keogh00}: aproxima una sèrie temporal partint-la en segments de la mateixa mida i emmagatzemant la mitjana dels punts que cauen dins del segment. Redueix de dimensió $n$ a dimensió $N$

% Adaptive Piecewise Constant Approximation (APCA) \cite{keogh01}: com el PAA però amb segments de mida variable.

A continuació,  la representació  d'una sèrie temporal segons el model de \emph{zero-order hold} s'estén per diferents continuïtats en els intervals de temps de representació.

Sigui $S$ una sèrie temporal, es defineix $S(t)$ com la representació
de la sèrie temporal contínuament al llarg del temps $t$.  En primer
lloc, es representa amb \emph{zero-order hold} a partir de funcions
graó contínues per la dreta (\emph{right-continuous}).

\begin{definition}[Representació amb \emph{zero-order hold}]
Sigui $S=\{m_0,\ldots,m_k\}$ una sèrie temporal, la representació  $S(t)$ amb \emph{zero-order hold} es defineix
\[
\forall t \in \mathbb{R} ,\forall m \in S: S(t) =
\begin{cases}
  V(\min S) & \text{si } t < T(\min S) \\
  V(m) & \text{si }  t\in [T(m),T(\seg m))
\end{cases}
\]
\end{definition}

En segon lloc, es representa $S(t)$ amb \emph{zero-order hold} centrada en
l'interval, definit també a partir de funcions graó contínues per la
dreta.

\begin{definition}[Representació amb \emph{zero-order hold} centrada en l'interval]
  Sigui $S=\{m_0,\ldots,m_k\}$ una sèrie temporal, la representació
  $S(t)$ amb \emph{zero-order hold} centrada en l'interval es defineix
\[
\forall t \in \mathbb{R}  ,\forall m \in S:
S(t) =  
\begin{cases}
  V(m) & \text{si } t = \frac{T(\ant m)+T(m)}{2} \\
  V(m) & \text{si } t\in \left( \frac{T(\ant m)+T(m)}{2},\frac{T(m)+T(\seg m)}{2} \right) \
\end{cases}
\]
\end{definition}

En tercer lloc, es representa $S(t)$ amb \emph{zero-order hold} cap enrere, ara definit a partir de funcions graó contínues per l'esquerra.
\begin{definition}[Representació en \emph{zero-order hold} cap enrere]
  Sigui $S=\{m_0,\ldots,m_k\}$ una sèrie temporal, la representació
  $S(t)$ amb \emph{zero-order hold} cap enrere es defineix
\[
\forall t \in \mathbb{R}  ,\forall m \in S:
S(t) =  
\begin{cases}
  V(\max S) & \text{si } t > T(\max S) \\
  V(m) & \text{si }  t\in (T(\ant m),T(m)]
\end{cases}
\]
\end{definition}

Sigui $S$ una sèrie temporal regular i $\delta$ una durada de temps, aleshores la representació de $S(t)$ amb \emph{zero-order hold} és la mateixa que la de $S(t-\delta)$ amb \emph{zero-order hold} cap enrere i és la mateixa que la de $S(t-\frac{\delta}{2})$ centrada en l'interval. 







\section{Model d'operacions}






\subsubsection{Unió}

Per tal que l'operació d'unió de conjunts sigui vàlida per les sèries
temporals cal tenir en compte quan dues sèries temporals tenen mesures
en el mateix instant de temps. En cas d'utilitzar l'operació d'unió de
conjunts la sèrie temporal resultant no compliria amb la definició
\ref{def:serie_temporal} ja que contindria mesures amb temps
repetits. Com a conseqüència, es defineix l'operació d'unió per les
sèries temporals.

\begin{definition}[unió]
  Sigui $S_1=\{m_0^1, \dotsc, m_{k_1}^1\}$ i $S_2=\{m_0^2, \dotsc,
  m_{k_2}^2\}$ dues sèries temporals, la unió de les dues sèries
  temporals $S_1 \cup S_2$ és una sèrie temporal $S=\{m_0, \dotsc,
  m_k\}$ que conté totes les mesures de $S_1$ i les mesures de $S_2$
  no repetides: $S_1 \cup S_2 = \{ m \in S_1 \} \cup \{ m^2 =
  (t^2,v^2) \in S_2 | \forall (t^1,v^1)\in S_1 : t_1 \neq t_2
  \}$. 

  Tal com succeeix amb les relacions, per a poder unir dues sèries
  temporals cal que totes dues tinguin la mateixa estructura; és a dir,
  en termes de SGBDR cal que tinguin la mateixa capçalera.
\end{definition}

Propietats de la unió:

\begin{itemize}
\item La dimensió $k$ de la sèrie temporal resultant està fitada a
  $k_1 \leq k \leq k_1 + k_2$. Nota: a la definició, la dimensió $k$ és
  proporcional al cardinal $|S_1\cup S_2| = k+1$.
\item La unió de sèries temporals no és commutativa. En general
  $S_1\cup S_2 \neq S_2\cup S_1$ tot i que sí que es compleix
  l'equivalència respecte al cardinal $|S_1\cup S_2| = |S_2\cup S_1|$.
\end{itemize}



\subsection{Temporals}

Operacions a on el temps i la representació de la sèrie temporals
juguen un paper important.


\subsubsection{Pertinença temporal}

\begin{definition}[pertinença temporal]
  Sigui $S=\{m_0, \dotsc, m_{k}\}$ una sèrie temporal i $m=(t,v)$ una
  mesura, direm que la mesura pertany temporalment a la sèrie 
  $m\in^t S$ si i només si $\exists m_i=(t_i,v_i)\in S: t_i=t$.
\end{definition}



\subsubsection{Selecció temporal}

Sigui $S$ una sèrie temporal i $i=[t_0,t_f]$ un interval de temps,
per una banda s'ha definit l'interval sobre la seqüència d'una sèrie temporal $S(t_0,t_f]$ (def.~\ref{def:model:st-interval})  i per altra banda s'ha definit la representació contínua $r$ d'una sèrie temporal $S(t)$ \todo{referenciar la definició de repr}.
Per seleccionar un interval temporal cal tenir en compte tant l'interval sobre la seqüència com la representació contínua, Aquesta selecció temporal s'anota com selecció de $S$ en $i$ amb representació $r$ o bé $S[t_o,t_f]^r$. 

\begin{definition}[Selecció temporal de $S$ en $i$ amb representació
  $r$]
  \[
  \text{selecció}: \text{Sèrie temporal} \times \text{interval de
    temps} \times \text{representació} \longrightarrow \text{Sèrie
    temporal}
  \]
  \[
  S = \{m_0 , \ldots , m_k\}  \times i = [t_0,t_f] \times r \longrightarrow S'
  \]
  \[
  \forall  t \in i: S' = S(t)^r 
  \] 
\end{definition}

A continuació s'exemplifica utilitzant la representació \emph{zoh} enrere.


\begin{definition}[Selecció temporal de $S$ en $i$ amb representació
  \emph{zohe}]
  Sigui $S$ una sèrie temporal, $i=[t_0,t_f]$ un interval de temps i
  \emph{zohe} la representació $S(t)$ amb \emph{zero-order-hold} cap
  enrere, es defineix la subsèrie $S[t_0,t_f]^{\text{zohe}}\subseteq
  S$ com la sèrie temporal 
  \[
  S[t_0,t_f]^{\text{zohe}} = (S \cup \{m\})(t_0,t_f] : m=(t_f,v):
  \]
  \[
  v=  V(\inf(S-S[-\infty,t_f)))
  \]
  
  Nota: $S-S[-\infty,t_f)$ seria equivalent a l'interval tancat
  $S[t_f,+\infty]$ si aquest últim estigués definit.
\end{definition}

Propietats de la selecció temporal:

\begin{itemize}
\item Observeu que, sigui $t_a$ un instant de temps, la selecció de $S$ en $[t_a,t_a]$ és equivalent a la representació contínua $S(t_a)$. 
\end{itemize}




\subsubsection{Selecció de la resolució}

La selecció de la resolució d'una sèrie temporal permet canviar, en el
context d'una representació, la resolució a una de marcada per un
conjunt d'instants de temps. A diferència d'un buffer, la selecció de
resolució no permet aplicar interpoladors ni obliga a que la sèrie
temporal resultant sigui regular.

Sigui $S$ una sèrie temporal, $i= \{t_0,t_1,\dotsc,t_n\}$ un conjunt
d'instants de temps i la representació contínua $r$ de la sèrie
temporal $S(t)$, la selecció de resolució s'anota com resolució de $S$
en $i$ amb representació $r$ o bé $S[i]^r$.

\begin{definition}[Selecció de la resolució de $S$ en $i$ amb representació
  $r$]
  \[
  \text{resolució}: \text{Sèrie temporal} \times \text{instants de
    temps} \times \text{representació} \longrightarrow \text{Sèrie
    temporal}
  \]
  \[
  S = \{m_0 , \ldots , m_k\} \times i = \{t_0,t_1,\dotsc,t_n\} \times r
  \longrightarrow S'
  \]
  \[
  t_0 < t_1 < \dotsb < t_n:
  \]
  \[
  S' = S[t_0,t_0]^r \cup  S[t_1,t_1]^r \cup \dotsb \cup S[t_{n},t_n]^r
  \] 
  % Es podria fer recursiu
  % \[
  % t_f = \sup(i): i_n = i - t_f: t_a = \sup(i_n):
  % S' = \left\{\begin{array}{ll}
  %     \{\} & \text{si } |i| = 0 \\
  %     S[i_n]^r \cup S[t_a,t_f]^r 
  %   \end{array}\right.
  % \] 
\end{definition}



Propietats de la selecció de resolució:
\begin{itemize}
\item El cardinal de la sèrie temporal resultant és el mateix que el del conjunt d'instants de temps $|S[i]^r| = |i|$
\end{itemize}





\subsubsection{Unió temporal}

Unió temporal de dues sèries temporals $S_1 \cup^r S_2$. Per la unió temporal les sèries temporals han de tenir la mateixa estructura, tal com s'ha notat per a la unió de sèries temporals.

\begin{definition}[Unió temporal de $S_1$ i $S_2$ amb representació
  $r$]
  \[
  \text{unió}: \text{Sèrie temporal} \times \text{Sèrie temporal}
  \times \text{representació} \longrightarrow \text{Sèrie temporal}
  \]
  \[
  S_1 = \{m_0^1 , \ldots , m_{k1}^1\}  \times S_2 = \{m_0^2 , \ldots , m_{k2}^2\} \times r \longrightarrow S'
  \]
  \[
  t_1=T(\inf S_1), t_2=T(\sup S_1):
  \]
  \[
  S' = S_1 \cup  ( S_2 - S_2[t_1,t_2]^r )
  \] 
\end{definition}



Propietats de la unió temporal:
\begin{itemize}
\item No commutativa
\item Però, $(S_1 \cup^r S_2) \cup (S_2 \cup^r S_1) = S_1 \cup S_2$\todo{és cert?}
\end{itemize}



\subsubsection{Fusió temporal}

Fusió (join) de dues sèries temporals $S_1 \text{ fusió } S_2$.


\begin{definition}[Fusió temporal de $S1$ i $S2$ amb representació $r$]
  \[
  \text{fusió}: \text{Sèrie temporal} \times
  \text{Sèrie temporal} \times \text{representació} \longrightarrow
  \text{Sèrie temporal}
  \]
  \[
  S_1 = \{m_0^1 , \ldots , m_{k_1}^1\} \times S_2 = \{m_0^2 , \ldots ,
  m_{k_2}^2\} \times r \longrightarrow S'
  \]
  \[
  t = \{t^1 \, | \, \forall m^1=(t^1,v^1) \in S_1\} \cup \{t^2 | \forall
  m^2=(t^2,v^2) \in S_2\}:
  \]
  \[
  S' = \{m'=(t',v_1',v_2') \, | \, (t',v_1') \in S_1[t]^r \wedge (t',v_2') \in S_2[t]^r \} 
  \]\todo{els valors s'haurien de saber fusionar}
\end{definition}



Propietats de la fusió temporal:
\begin{itemize}
\item $|S'| <= k_1 + k_2$
\end{itemize}







%%% Local Variables:
%%% TeX-master: "main"
%%% End:







% LocalWords:  SGST

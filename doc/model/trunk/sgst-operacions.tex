\section{Model d'operacions}



\begin{verbatim}
VAR timeseries BASE RELATION
    { t RATIONAL, v RATIONAL }  KEY { t } ;

OPERATOR ts.t(m SAME_TYPE_AS  (timeseries)) RETURNS RATIONAL;
return t FROM TUPLE FROM m;
END OPERATOR;

OPERATOR ts.v(m SAME_TYPE_AS  (timeseries)) RETURNS RATIONAL;
return v FROM TUPLE FROM m;
END OPERATOR;
\end{verbatim}




\subsection{Màxim i suprem}


La relació definida a~\ref{def:model:mesura-relacio-ordre} indueix
sobre una sèrie temporal una relació d'ordre total. Com que la sèrie
temporal s'ha considerat finita i sense elements repetits, quan la
sèrie temporal no és buida això comporta l'existència d'un màxim i
d'un mínim.  Si $S$ és una sèrie temporal, $\max(S)$ i $\min(S)$ són
respectivament la mesura màxima i mínima d'$S$.

\begin{definition}[Màxim i mínim]
  Sigui $S=\{m_0,\ldots,m_k\}$ una sèrie temporal i $n\in S$ una
  mesura.  Direm que $n=\max(S)$ és el màxim de la sèrie temporal si i
  només si $\forall m \in S: n \geq m $.  Direm que $n=\min(S)$ és el
  mínim de la sèrie temporal si i només si $\forall m \in S: n \leq
  m$.
\end{definition}

El $\max(S)$ i el $\min(S)$ no estan definits quan la sèrie temporal
és buida: $S= \emptyset$. En
canvi, el suprem i l'ínfim estan definits per qualsevol
sèrie temporal tal com passa amb el conjunt estès de nombres reals,
\cite{cantrell:extendedreal}.  

\begin{definition}[Suprem i ínfim]
  Sigui $S=\{m_0,\ldots,m_k\}$ una sèrie temporal i $n\in S$ una
  mesura.  Direm que $n=\sup(S)$ és el suprem de la sèrie temporal si
  $n=\max(S)$ en cas que el màxim estigui definit o
  $n=(-\infty,\infty)$ en cas contrari.  Direm que $n=\inf(S)$ és
  l'ínfim de la sèrie temporal si $n=\min(S)$ en cas que el mínim
  estigui definit o $n=(+\infty,\infty)$ en cas contrari.
\end{definition}
Quan la sèrie temporal no és buida, per
ser un conjunt finit i d'ordre total, sempre hi ha un i només un màxim
i un mínim i per tant es corresponen amb el suprem i l'ínfim
respectivament.



Tutorial D:
\begin{verbatim}
OPERATOR ts.max(s1 SAME_TYPE_AS  (timeseries)) RETURNS RELATION SAME_HEADING_AS  (timeseries);
return s1 JOIN ( SUMMARIZE s1 {t} PER (s1 {}) ADD (MAX (t) AS t));
END OPERATOR;

OPERATOR ts.sup(s1 SAME_TYPE_AS  (timeseries)) RETURNS RELATION SAME_HEADING_AS  (timeseries);
return ts.max(ts.union(s1,(RELATION { TUPLE {t -1.0/0.0, v 1.0/0.0} })));
END OPERATOR;
\end{verbatim}



Exemple:
\begin{verbatim}
WITH RELATION {
TUPLE { t 2.0, v 3.0 },
TUPLE { t 4.0, v 2.0 },
TUPLE { t 6.0, v 4.0 }
 } AS ts1: 
ts.max(ts1)
\end{verbatim}
\begin{verbatim}
RELATION {
TUPLE { t 1.0, v 2.0 },
TUPLE { t 5.0, v 3.0 },
TUPLE { t 6.0, v 5.0 },
TUPLE { t 1.0/0.0, v 1.0 }  //1.0/0.0 infinit
 } AS ts2: 
ts.max(ts2)
\end{verbatim}
\begin{verbatim}
WITH RELATION {
TUPLE { t 2.0, v 3.0 },
TUPLE { t 4.0, v 2.0 },
TUPLE { t 6.0, v 4.0 }
 } AS ts1: 
ts.sup(ts1)
\end{verbatim}
\begin{verbatim}
WITH RELATION {
TUPLE { t 2.0, v 3.0 },
TUPLE { t 4.0, v 2.0 },
TUPLE { t 6.0, v 4.0 }
 } AS ts1: 
ts.sup(timeseries)
\end{verbatim}



\subsection{Interval}

Atesa la relació d'ordre induïda pel temps en una sèrie temporal
(def.\ \ref{def:model:mesura-relacio-ordre}) és possible definir el
concepte d'interval sobre la seqüència, semblant a com es fa a \cite{last:keogh,last:hetland}.

\begin{definition}[Interval]
  \label{def:model:st-interval}
  Sigui $S=\{m_0, \ldots, m_k\}$ una sèrie temporal. Definirem el subconjunt
  $S(r,t] \subseteq S$ com la sèrie temporal $S(r,t]=\{m\in S
  | r<T(m)\leq t\}$, a on $r$ i $t$ són dos instants de temps.

  També es defineix la subsèrie $S[-\infty,t)\subseteq S$ com la sèrie
  temporal $S[-\infty,t) = \{m\in S | T(\inf(S))\leq T(m) < t\}$.
\end{definition}
S'observa que la subsèrie $S(r,+\infty]\subseteq S$ és
equivalent a la sèrie temporal $S(r,+\infty] \equiv S(r,T(\sup(S))]$ i
anàlogament $S(-\infty,t] \equiv S(T(\inf(S)),t]$. També s'observa que les subsèries $S(t,t]\subseteq S$ i $S[t,t)\subseteq S$ són equivalents a la sèrie temporal buida $S(t,t] \equiv S[t,t) \equiv \emptyset$ ja que per ser els temps d'ordre total $\nexists T(m): t < T(m) \leq t$ o $\nexists T(m): t \leq T(m) < t$, respectivament. 
%Finalment, s'observa que la subsèrie $S(-\infty,+\infty]\subseteq S$ només és equivalent a la sèrie temporal original quan aquesta no conté la mesura indefinida negativa $S(-\infty,+\infty]\equiv S: (-\infty,v)\notin S$


\begin{verbatim}
OPERATOR ts.interval(s1 SAME_TYPE_AS  (timeseries), l RATIONAL, h RATIONAL) RETURNS RELATION SAME_HEADING_AS  (timeseries);
return s1 WHERE t>l AND t<=h;
END OPERATOR;
OPERATOR ts.interval.ni(s1 SAME_TYPE_AS  (timeseries), h RATIONAL) RETURNS RELATION SAME_HEADING_AS  (timeseries);
return s1 WHERE t<h;
END OPERATOR;
\end{verbatim}



\subsection{Successor}


També atenent a la relació d'ordre induïda pel temps en una sèrie temporal, es
defineix el concepte de mesura següent i mesura anterior en una
seqüència.

\begin{definition}[Successor i predecessor]
  Sigui $S=\{m_0, \ldots, m_k\}$ una sèrie temporal i $l\in S$ i $n$ dues
  mesures. Direm que $l$ és el successor de $n$ en $S$ i ho notarem
  com $l=\seg\limits_S(n)$ si i només si $l=\inf(S(T(n),+\infty])$.
  Direm que $l$ és el predecessor de $n$ en $S$ i ho notarem com
  $l=\ant\limits_S(n)$ si i només si $l=\sup(S[-\infty,T(n)))$.

Quan no hi hagi dubte de la sèrie temporal que marca l'ordre, per
exemple quan $n\in S$, podrem escriure $\seg(n)$ i $\ant(n)$.
\end{definition}
S'observa que s'obtenen mesures indefinides en els casos que la
mesura següent o anterior es calcula respectivament per la mesura
suprema o ínfima de la sèrie temporal: $\seg\limits_S(\sup
S)=(+\infty,\infty)$ i $\ant\limits_S(\inf S)=(-\infty,\infty)$.

De la definició anterior es dedueix que donada una sèrie temporal $S$
que no conté mesures indefinides i donada la mesura indefinida
$o=(+\infty,\infty)$, el predecessor de $o$ sempre és el suprem de la
sèrie temporal $\ant\limits_S( (+\infty,\infty) ) = \sup(S): \forall
m\in S: T(m)\in\mathbb{R}$.  % S\equiv S(-\infty,+\infty)
\emph{Demostració: Sigui $S$ una sèrie temporal i $o=(+\infty,\infty)$
  una mesura indefinida, el predecessor de $o$ en $S$ és una mesura
  $l=\ant\limits_S(o)$ que compleix
  $l=\sup(S[-\infty,T(o)))$. Substituint, s'obté que
  $l=\sup(S[-\infty,+\infty))=\sup(S-m):m\in S:T(m)=+\infty \notin
  \mathbb{R}$, i per tant com que $S$ no té mesures indefinides es
  demostra que $l=\sup(S)$.  } De manera semblant es pot demostrar que
$\seg\limits_S( (-\infty,\infty) ) = \inf(S): \forall m\in S:
T(m)\in\mathbb{R}$.





\begin{verbatim}
OPERATOR ts.next( m SAME_TYPE_AS  (timeseries), s1 SAME_TYPE_AS  (timeseries)) RETURNS RELATION SAME_HEADING_AS  (timeseries);
return ts.inf(ts.interval(s1,ts.t(m),1.0/0.0));
END OPERATOR;
OPERATOR ts.prev( m SAME_TYPE_AS  (timeseries), s1 SAME_TYPE_AS  (timeseries)) RETURNS RELATION SAME_HEADING_AS  (timeseries);
return ts.sup(ts.interval.ni(s1,ts.t(m)));
END OPERATOR;
\end{verbatim}





\subsubsection{Unió}

Per tal que l'operació d'unió de conjunts sigui vàlida per les sèries
temporals cal tenir en compte quan dues sèries temporals tenen mesures
en el mateix instant de temps. En cas d'utilitzar l'operació d'unió de
conjunts la sèrie temporal resultant no compliria amb la definició
\ref{def:serie_temporal} ja que contindria mesures amb temps
repetits. Com a conseqüència, es defineix l'operació d'unió per les
sèries temporals.

\begin{definition}[unió]
  Sigui $S_1=\{m_0^1, \dotsc, m_{k_1}^1\}$ i $S_2=\{m_0^2, \dotsc,
  m_{k_2}^2\}$ dues sèries temporals, la unió de les dues sèries
  temporals $S_1 \cup S_2$ és una sèrie temporal $S=\{m_0, \dotsc,
  m_k\}$ que conté totes les mesures de $S_1$ i les mesures de $S_2$
  no repetides: $S_1 \cup S_2 = \{ m \in S_1 \} \cup \{ m^2 =
  (t^2,v^2) \in S_2 | \forall (t^1,v^1)\in S_1 : t_1 \neq t_2
  \}$. 

  Tal com succeeix amb les relacions, per a poder unir dues sèries
  temporals cal que totes dues tinguin la mateixa estructura; és a dir,
  en termes de SGBDR cal que tinguin la mateixa capçalera.
\end{definition}

Propietats de la unió:

\begin{itemize}
\item La dimensió $k$ de la sèrie temporal resultant està fitada a
  $k_1 \leq k \leq k_1 + k_2$. Nota: a la definició, la dimensió $k$ és
  proporcional al cardinal $|S_1\cup S_2| = k+1$.
\item La unió de sèries temporals no és commutativa. En general
  $S_1\cup S_2 \neq S_2\cup S_1$ tot i que sí que es compleix
  l'equivalència respecte al cardinal $|S_1\cup S_2| = |S_2\cup S_1|$.
\end{itemize}



TutorialD:
\begin{verbatim}
OPERATOR ts.union(s1 SAME_TYPE_AS  (timeseries), s2 SAME_TYPE_AS  (timeseries)) RETURNS RELATION SAME_HEADING_AS  (timeseries);
return s1 UNION (s2 JOIN (s2 {t} MINUS s1 {t}));
END OPERATOR;
\end{verbatim}


Exemple:
\begin{verbatim}
WITH RELATION {
TUPLE { t 2.0, v 3.0 },
TUPLE { t 4.0, v 2.0 },
TUPLE { t 6.0, v 4.0 }
 } AS ts1,
RELATION {
TUPLE { t 1.0, v 2.0 },
TUPLE { t 5.0, v 3.0 },
TUPLE { t 6.0, v 5.0 },
TUPLE { t 10.0, v 1.0 }
 } AS ts2: 
ts.union(ts1,ts2)
\end{verbatim}



\subsection{Temporals}

Operacions a on el temps i la representació de la sèrie temporals
juguen un paper important.


\subsubsection{Pertinença temporal}

\begin{definition}[pertinença temporal]
  Sigui $S=\{m_0, \dotsc, m_{k}\}$ una sèrie temporal i $m=(t,v)$ una
  mesura, direm que la mesura pertany temporalment a la sèrie 
  $m\in^t S$ si i només si $\exists m_i=(t_i,v_i)\in S: t_i=t$.
\end{definition}



\subsubsection{Selecció temporal}

Sigui $S$ una sèrie temporal i $i=[t_0,t_f]$ un interval de temps,
per una banda s'ha definit l'interval sobre la seqüència d'una sèrie temporal $S(t_0,t_f]$ (def.~\ref{def:model:st-interval})  i per altra banda s'ha definit la representació contínua $r$ d'una sèrie temporal $S(t)$ \todo{referenciar la definició de repr}.
Per seleccionar un interval temporal cal tenir en compte tant l'interval sobre la seqüència com la representació contínua, Aquesta selecció temporal s'anota com selecció de $S$ en $i$ amb representació $r$ o bé $S[t_o,t_f]^r$. 

\begin{definition}[Selecció temporal de $S$ en $i$ amb representació
  $r$]
  \[
  \text{selecció}: \text{Sèrie temporal} \times \text{interval de
    temps} \times \text{representació} \longrightarrow \text{Sèrie
    temporal}
  \]
  \[
  S = \{m_0 , \ldots , m_k\}  \times i = [t_0,t_f] \times r \longrightarrow S'
  \]
  \[
  \forall  t \in i: S' = S(t)^r 
  \] 
\end{definition}

A continuació s'exemplifica utilitzant la representació \emph{zoh} enrere.


\begin{definition}[Selecció temporal de $S$ en $i$ amb representació
  \emph{zohe}]
  Sigui $S$ una sèrie temporal, $i=[t_0,t_f]$ un interval de temps i
  \emph{zohe} la representació $S(t)$ amb \emph{zero-order-hold} cap
  enrere, es defineix la subsèrie $S[t_0,t_f]^{\text{zohe}}\subseteq
  S$ com la sèrie temporal 
  \[
  S[t_0,t_f]^{\text{zohe}} = (S \cup \{m\})(t_0,t_f] : m=(t_f,v):
  \]
  \[
  v=  V(\inf(S-S[-\infty,t_f)))
  \]
  
  Nota: $S-S[-\infty,t_f)$ seria equivalent a l'interval tancat
  $S[t_f,+\infty]$ si aquest últim estigués definit.
\end{definition}

Propietats de la selecció temporal:

\begin{itemize}
\item Observeu que, sigui $t_a$ un instant de temps, la selecció de $S$ en $[t_a,t_a]$ és equivalent a la representació contínua $S(t_a)$. 
\end{itemize}




\subsubsection{Selecció de la resolució}

La selecció de la resolució d'una sèrie temporal permet canviar, en el
context d'una representació, la resolució a una de marcada per un
conjunt d'instants de temps. A diferència d'un buffer, la selecció de
resolució no permet aplicar interpoladors ni obliga a que la sèrie
temporal resultant sigui regular.

Sigui $S$ una sèrie temporal, $i= \{t_0,t_1,\dotsc,t_n\}$ un conjunt
d'instants de temps i la representació contínua $r$ de la sèrie
temporal $S(t)$, la selecció de resolució s'anota com resolució de $S$
en $i$ amb representació $r$ o bé $S[i]^r$.

\begin{definition}[Selecció de la resolució de $S$ en $i$ amb representació
  $r$]
  \[
  \text{resolució}: \text{Sèrie temporal} \times \text{instants de
    temps} \times \text{representació} \longrightarrow \text{Sèrie
    temporal}
  \]
  \[
  S = \{m_0 , \ldots , m_k\} \times i = \{t_0,t_1,\dotsc,t_n\} \times r
  \longrightarrow S'
  \]
  \[
  t_0 < t_1 < \dotsb < t_n:
  \]
  \[
  S' = S[t_0,t_0]^r \cup  S[t_1,t_1]^r \cup \dotsb \cup S[t_{n},t_n]^r
  \] 
  % Es podria fer recursiu
  % \[
  % t_f = \sup(i): i_n = i - t_f: t_a = \sup(i_n):
  % S' = \left\{\begin{array}{ll}
  %     \{\} & \text{si } |i| = 0 \\
  %     S[i_n]^r \cup S[t_a,t_f]^r 
  %   \end{array}\right.
  % \] 
\end{definition}



Propietats de la selecció de resolució:
\begin{itemize}
\item El cardinal de la sèrie temporal resultant és el mateix que el del conjunt d'instants de temps $|S[i]^r| = |i|$
\end{itemize}





\subsubsection{Unió temporal}

Unió temporal de dues sèries temporals $S_1 \cup^r S_2$. Per la unió temporal les sèries temporals han de tenir la mateixa estructura, tal com s'ha notat per a la unió de sèries temporals.

\begin{definition}[Unió temporal de $S_1$ i $S_2$ amb representació
  $r$]
  \[
  \text{unió}: \text{Sèrie temporal} \times \text{Sèrie temporal}
  \times \text{representació} \longrightarrow \text{Sèrie temporal}
  \]
  \[
  S_1 = \{m_0^1 , \ldots , m_{k1}^1\}  \times S_2 = \{m_0^2 , \ldots , m_{k2}^2\} \times r \longrightarrow S'
  \]
  \[
  t_1=T(\inf S_1), t_2=T(\sup S_1):
  \]
  \[
  S' = S_1 \cup  ( S_2 - S_2[t_1,t_2]^r )
  \] 
\end{definition}



Propietats de la unió temporal:
\begin{itemize}
\item No commutativa
\item Però, $(S_1 \cup^r S_2) \cup (S_2 \cup^r S_1) = S_1 \cup S_2$\todo{és cert?}
\end{itemize}



\subsubsection{Fusió temporal}

Fusió (join) de dues sèries temporals $S_1 \text{ fusió } S_2$.


\begin{definition}[Fusió temporal de $S1$ i $S2$ amb representació $r$]
  \[
  \text{fusió}: \text{Sèrie temporal} \times
  \text{Sèrie temporal} \times \text{representació} \longrightarrow
  \text{Sèrie temporal}
  \]
  \[
  S_1 = \{m_0^1 , \ldots , m_{k_1}^1\} \times S_2 = \{m_0^2 , \ldots ,
  m_{k_2}^2\} \times r \longrightarrow S'
  \]
  \[
  t = \{t^1 \, | \, \forall m^1=(t^1,v^1) \in S_1\} \cup \{t^2 | \forall
  m^2=(t^2,v^2) \in S_2\}:
  \]
  \[
  S' = \{m'=(t',v_1',v_2') \, | \, (t',v_1') \in S_1[t]^r \wedge (t',v_2') \in S_2[t]^r \} 
  \]\todo{els valors s'haurien de saber fusionar}
\end{definition}



Propietats de la fusió temporal:
\begin{itemize}
\item $|S'| <= k_1 + k_2$
\end{itemize}











\section{Map i fold}

\[
\text{map}: S \times f \longrightarrow S' = \{\forall m\in S : f(m) \}
\]


\[
\text{fold}: S \times m_i \times f(m,m_i) \longrightarrow m' =
\begin{cases}
  m_i & si |S|=0 \\
  f_m & altrament
\end{cases},
\text{ a on}
\]
\[
f_m=m_a\in S : \text{fold}( S-\{m_a\},f(m_a,m_i),f)
\]









%%% Local Variables:
%%% TeX-master: "main"
%%% End:







% LocalWords:  SGST

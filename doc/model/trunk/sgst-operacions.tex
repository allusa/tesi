\section{Model d'operacions}
\label{sec:model:sgst-operacions}

En aquesta secció definim les operacions d'un \gls{SGST} que permeten
manipular les sèries temporals.  Una sèrie temporal té un atribut de
temps que ha de ser tingut en compte pels operadors que la manipulin.
Així, atenent a aquest atribut de temps, el comportament d'una sèrie
temporal pot tenir naturaleses diferents:
\begin{itemize}
\item Conjunt, és a dir els operadors només atenent a la forma
  estructural bàsica.
\item Seqüència, en la qual els operadors la tracten com a conjunts
  amb ordre.
\item Funció temporal, en la qual els operadors treballen tenint en
  compte que una sèrie temporal és la representació d'un funció
  contínua.
\end{itemize}



En el disseny del model d'operacions següent es distingeix el
comportament per als tres casos anteriors.  Es dissenyen les
operacions bàsiques que permeten que posteriorment es combinin per
a elaborar-ne de més complexes.


Les manipulacions de les sèries temporals es defineixen abstractament
per a qualsevol sèrie temporal que tingui l'estructura de \gls{SGST}.
Les definicions dels operadors avaluen els conceptes algebraics i
lògics de les dades però no avaluen la semàntica en un context
particular, com també ocorre per exemple en el model d'operacions del
model relacional. És a dir, en cada context particular de manipulació
d'una sèrie temporal s'ha de decidir si aquella àlgebra té significat
o, al contrari, no pot ser aplicada. Per exemple una suma de valors de
diferents unitats podria ser semànticament errònia. A
l'\autoref{sec:sgst:tretssemantics} estudiem el significat d'algunes
propietats de les sèries temporals.


% es treballa amb la forma canònica de les sèries temporals llevat que
% s'indiqui el contrari. 



\subsection{Bàsiques de conjunts}
\glsaddsec{not:op-conjunts} %%%%secció d'operacions

En el model estructural de \gls{SGST} hem definit les sèries temporals
utilitzant conjunts. En aquest apartat definim operadors per a les
sèries temporals recollint els operadors habituals que tenen els
conjunts.   

El model relacional de \gls{SGBDR} defineix els seus operadors bàsics
a partir de l'àlgebra de
conjunts \parencite[cap.~7]{date04:introduction8}. En aquest apartat
apliquem el mateix estudi per al model de \gls{SGST}. Tot i així de
manera simplificada, a les definicions no es descriuen les sèries
temporals com a relacions amb capçaleres sinó que se n'escriuen només
els conjunts de valors. Seguint el model relacional es poden estendre
les definicions i introduir el model complet de relacions.



En les operacions binàries de sèries temporals, entre les mesures
d'ambdós conjunts hi ha les dues possibles relacions d'ordre de la
\autoref{def:model:mesura-relacio-ordre}, és a dir poden tenir ordre o
parcial. Com a conseqüència, aquests dos ordres possibles indueixen
dues definicions per a alguns operadors de conjunts. Als operadors amb
ordre total els anomenarem temporals i hi afegirem un superíndex $t$
per a indicar-ho.


Les operacions que agrupen els elements dels conjunts habitualment se
solen representar gràficament mitjançant diagrames de Venn. A la
\autoref{fig:model:sgst:venn} es mostren els diagrames de Venn per a
cinc de les operacions dels \gls{SGSTM} que descrivim a continuació:
inclusió, unió, diferència, intersecció i diferència simètrica; tant
en la seva vessant parcial com en la seva vessant temporal.  

Per a dibuixar aquest diagrames de Venn, a banda del conjunt
corresponent a cada sèrie temporal $A$ i $B$ i la seva intersecció,
cal dibuixar uns subconjunts $t_A$ i $t_B$ que indiquen les mesures
que comparteixen el mateix temps amb una altra mesura de l'altre
conjunt però no el mateix valor. És a dir $t_A = \{ m_A \in A |
\exists m_B \in B: m_A =^t m_B \}$ i $t_B = \{ m_B \in B | \exists m_A
\in A: m_B =^t m_A \}$ segons la relació d'ordre total definida en de
les mesures.  Aquests dos subconjunts $t_A$ i $t_B$ són importants per
a les operacions dels \gls{SGST} perquè no hi pot haver cap sèrie
temporal resultant que els inclogui a tots dos, ja que significaria
que conté temps repetits. Per exemple, una operació que tingui la
sèrie resultant $t_A \cup t_B$, com es mostra a la
\autoref{fig:model:sgst:venn-impossible}, és impossible


\begin{figure}[tp]
  \centering
  \def\escala{0.7}

\def\nodeA{node [anchor=east] {$A$}}
\def\nodeB{node [anchor=west] {$B$}}
\def\nodeT{node [left=0.4cm] {\tiny $t_A$} node [right=0.4cm] {\tiny $t_B$}}
% Definition of circles
\def\firstcircle{(0,0) circle (1.5cm)}
\def\secondcircle{(0:2cm) circle (1.5cm)}
\def\thirdcircle{(0:1cm) circle (1.11cm)}

\colorlet{circle edge}{blue!50}
\colorlet{circle area}{blue!20}

\tikzset{
  filled/.style={fill=circle area, draw=circle edge, thick},
  outline/.style={draw=circle edge, thick},
  every node/.style={transform shape}
}

%\setlength{\parskip}{5mm}



%Set A in B
\begin{tikzpicture}[scale=\escala]
    \begin{scope}
        \clip \secondcircle;
        \draw[even odd rule,blue] \firstcircle \nodeA
                                 \secondcircle ;
                                 %\thirdcircle;
            \fill[filled] \firstcircle;
   \end{scope}
      \draw[outline] \secondcircle \nodeB;

   \node[anchor=south] at (current bounding box.north) {$A \subset B$};
   \node[anchor=west] at (current bounding box.west) {$A$};
\end{tikzpicture}
%Set temporal A in B
\begin{tikzpicture}[scale=\escala]
    \begin{scope}
        \clip \firstcircle;
        \fill[filled] \thirdcircle;
      \draw[outline] \thirdcircle \nodeT;
    \end{scope}
    \begin{scope}
        \clip \secondcircle;
        \draw \thirdcircle \nodeT;
    \end{scope}
      \draw[circle edge] \thirdcircle;
    \begin{scope}
        \clip \secondcircle;
        \draw[even odd rule,blue] \firstcircle \nodeA
                                 \secondcircle ;
                                 %\thirdcircle;
            \draw[outline] \firstcircle;
   \end{scope}
      \draw[outline] \secondcircle \nodeB;

   \node[anchor=south] at (current bounding box.north) {$A \subset^t B$};
   \node[anchor=east] at (current bounding box.center) {$A$};
\end{tikzpicture}







%Set A or B
\begin{tikzpicture}[scale=\escala]
  \draw[filled] \firstcircle \nodeA;
    \begin{scope}
        \clip \secondcircle;
        \draw[filled, even odd rule] \firstcircle \nodeA
                                 \secondcircle 
                                 \thirdcircle;
   \end{scope}
    \draw[outline] \firstcircle
                   \secondcircle \nodeB
                   \thirdcircle \nodeT;

   \node[anchor=south] at (current bounding box.north) {$A \cup B$};
\end{tikzpicture}
%Set temporal A or B
\begin{tikzpicture}[scale=\escala]
    \draw[filled, even odd rule] \firstcircle \nodeA
                                 \secondcircle \nodeB
                                 \thirdcircle \nodeT;
    \node[anchor=south] at (current bounding box.north) {$A \cup^t B$};
\end{tikzpicture}




% Set A but not B
\begin{tikzpicture}[scale=\escala]
    \begin{scope}
        \clip \firstcircle;
        \draw[filled, even odd rule] \firstcircle \nodeA
                                     \secondcircle;

    \end{scope}
    \draw[outline] \firstcircle
                   \secondcircle \nodeB
                   \thirdcircle \nodeT;
    \node[anchor=south] at (current bounding box.north) {$A - B$};
\end{tikzpicture}
% Set temporal A but not B
\begin{tikzpicture}[scale=\escala]
    \begin{scope}
        \clip \firstcircle;
        \draw[filled, even odd rule] \firstcircle \nodeA
                                     \thirdcircle;

    \end{scope}
    \draw[outline] \firstcircle
                   \secondcircle \nodeB
                   \thirdcircle \nodeT;
    \node[anchor=south] at (current bounding box.north) {$A -^t B$};
\end{tikzpicture}





% % Set A and B
% \begin{tikzpicture}
%     \begin{scope}
%         \clip \firstcircle;
%         \fill[filled] \secondcircle;
%     \end{scope}
%     \draw[outline] \firstcircle \nodeA;
%     \draw[outline] \secondcircle \nodeB;
%     \draw[outline] \thirdcircle \nodeT;
%     \node[anchor=south] at (current bounding box.north) {$A \cap B$};
% \end{tikzpicture}
% % Set temporal A and B
% \begin{tikzpicture}
%     \begin{scope}
%         \clip \firstcircle;
%         \fill[filled] \thirdcircle;
%     \end{scope}
%     \draw[outline] \firstcircle \nodeA;
%     \draw[outline] \secondcircle \nodeB;
%     \draw[outline] \thirdcircle \nodeT;
%     \node[anchor=south] at (current bounding box.north) {$A \cap^t B$};
% \end{tikzpicture}



% %Set A or B but not (A and B) also known a A xor B
% \begin{tikzpicture}
%     \begin{scope}
%         \clip \firstcircle;
%         \draw[filled, even odd rule] \firstcircle
%                                      \secondcircle;
%     \end{scope}
%     \begin{scope}
%         \clip \secondcircle;
%         \draw[filled, even odd rule] \secondcircle 
%                                      \thirdcircle;

%     \end{scope}
%     \draw[outline] \firstcircle \nodeA;
%     \draw[outline] \secondcircle \nodeB;
%     \draw[outline] \thirdcircle \nodeT;
%     \node[anchor=south] at (current bounding box.north) {$A \ominus B$};
% \end{tikzpicture}
% %Set temporal A or B but not (A and B) also known a A xor B
% \begin{tikzpicture}
%     \begin{scope}
%         \clip \firstcircle;
%         \draw[filled, even odd rule] \firstcircle
%                                      \thirdcircle;
%     \end{scope}
%     \begin{scope}
%         \clip \secondcircle;
%         \draw[filled, even odd rule] \secondcircle 
%                                      \thirdcircle;

%     \end{scope}
%     \draw[outline] \firstcircle \nodeA;
%     \draw[outline] \secondcircle \nodeB;
%     \draw[outline] \thirdcircle \nodeT;
%     \node[anchor=south] at (current bounding box.north) {$A \ominus^t B$};
% \end{tikzpicture}
  \caption{Diagrames de Venn per a les operacions dels \gls{SGSTM}. Els
    subconjunts $t_A$ i $t_B$ indiquen les mesures $m_A =^t m_B$.}
  \label{fig:model:sgst:venn}
\end{figure}


\begin{figure}[tp]
  \centering
  \def\nodeA{node [anchor=east] {$S_1$}}
\def\nodeB{node [anchor=west] {$S_2$}}
\def\nodeT{node [left=0.4cm] {\tiny $T_1$} node [right=0.4cm] {\tiny $T_2$}}
% Definition of circles
\def\firstcircle{(0,0) circle (1.5cm)}
\def\secondcircle{(0:2cm) circle (1.5cm)}
\def\thirdcircle{(0:1cm) circle (1.11cm)}

\colorlet{circle edge}{blue!50}
\colorlet{circle area}{blue!20}

\tikzset{filled/.style={fill=circle area, draw=circle edge, thick,color=lightgray},
    outline/.style={draw=circle edge, thick}}


% Set  A impossible B
\begin{tikzpicture}
    \begin{scope}
        \clip \firstcircle;
        \draw[filled, even odd rule] \secondcircle
                                     \thirdcircle;
    \end{scope}
    \begin{scope}
        \clip \secondcircle;
        \draw[filled, even odd rule] \firstcircle
                                     \thirdcircle;
    \end{scope}

    \draw[outline] \firstcircle \nodeA;
    \draw[outline] \secondcircle \nodeB;
    \draw[outline] \thirdcircle \nodeT;
    \node[anchor=south] at (current bounding box.north) {\color{gray}$T_1 \cup T_2$};
\end{tikzpicture}
  \caption{Diagrama de Venn impossible per a les operacions dels
    \gls{SGSTM}. Els subconjunts $t_A$ i $t_B$ indiquen les mesures
    $m_A =^t m_B$.}
  \label{fig:model:sgst:venn-impossible}
\end{figure}


\subsubsection{Pertinença i inclusió}


La pertinença determina si un element pertany a un conjunt.  Sigui
$S=\{m_0,\ldots,m_k\}$ una sèrie temporal i $m$ una mesura, es
defineix de la mateixa manera que en els conjunts la pertinença de la
mesura a la sèrie temporal $m \in S$. Aquesta pertinença es defineix a
partir de l'ordre parcial, és a dir que dues mesures són iguals quan
ho són els seus temps i valor.
\begin{definition}[Pertinença]
  Sigui $S=\{m_0,\ldots,m_k\}$ una sèrie temporal i $m=(t,v)$ una
  mesura, direm que la mesura pertany a la sèrie temporal
  $m \glssymboldef{not:sgst:in} S$ si i només si $\exists m_a \in S :
  T(m) = T(m_a) \wedge V(m) = V(m_a)$.
\end{definition}




A partir de l'ordre total, és a dir que dues mesures són iguals quan
ho són els seus temps, es defineix la pertinença temporal d'una mesura
a una sèrie temporal.
\begin{definition}[Pertinença temporal]
  Sigui $S=\{m_0,\ldots,m_k\}$ una sèrie temporal i $m=(t,v)$ una
  mesura, direm que la mesura pertany temporalment a la sèrie temporal
  $m \glssymboldef{not:sgst:int} S$ si i només si $\exists m_a \in S :
  T(m) = T(m_a)$.
\end{definition}


Si una mesura pertany a una sèrie temporal, $m\in S$, aleshores també
hi pertany temporalment, $m \glssymbol{not:sgst:int} S$.



La inclusió determina si tots els elements d'un conjunt pertanyen a un
altre conjunt. La inclusió de conjunts en el cas de les sèries
temporals està relacionada amb l'ordre parcial.
\begin{definition}[Inclusió]
  Siguin $S_1=\{m_0^1,\ldots,m_k^1\}$ i $S_2=\{m_0^2,\ldots,m_l^2\}$
  dues sèries temporals, la primera sèrie temporal està inclosa en la
  segona $S_1 \glssymboldef{not:sgst:sub} S_2$ si i només si $\forall
  m \in S_1: m \in S_2$. Aleshores, $S_1$ és una subsèrie temporal de
  $S_2$.
\end{definition}



Atenent a la pertinença temporal, es defineix la inclusió temporal
d'una mesura a una sèrie temporal.
\begin{definition}[Inclusió temporal]
  Siguin $S_1=\{m_0^1,\ldots,m_k^1\}$ i $S_2=\{m_0^2,\ldots,m_l^2\}$
  dues sèries temporals, la primera sèrie temporal està inclosa
  temporalment en la segona $S_1 \glssymboldef{not:sgst:subt} S_2$ si
  i només si $\forall m \in S_1: m \glssymbol{not:sgst:int} S_2$.
\end{definition}




\begin{example}
  Siguin les mesures $m_1=(1,1)$, $m_2=(3,1)$, $m_3=(3,2)$ i
  $m_4=(4,0)$ i les sèries temporals $S_1=\{m_1,m_2\}$ i
  $S_2=\{m_1,m_3,m_4\}$ aleshores les operacions següents són certes:
  $m_2 \in S_1$, $m_2 \notin S_2$, $m_2 \glssymbol{not:sgst:int} S_1$,
  $m_2 \glssymbol{not:sgst:int} S_2$, $S_1\not\subseteq S_2$ i
  $S_1\glssymbol{not:sgst:subt} S_2$.
\end{example}

\subsubsection{Màxim i suprem}


En una sèrie temporal les mesures tenen relació d'ordre total. Com que
la sèrie temporal s'ha considerat finita i sense elements repetits,
quan la sèrie temporal no és buida això comporta l'existència d'un
màxim i d'un mínim.  Si $S$ és una sèrie temporal, $\max(S)$ i
$\min(S)$ són respectivament la mesura màxima i mínima d'$S$.

\begin{definition}[Màxim i mínim]
  Sigui $S=\{m_0,\ldots,m_k\}$ una sèrie temporal i $n\in S$ una
  mesura.  Direm que $n=\glssymboldef{not:sgst:max}(S)$ és el màxim de
  la sèrie temporal si i només si $\forall m \in S: n \geq m $.  Direm
  que $n=\glssymboldef{not:sgst:min}(S)$ és el mínim de la sèrie
  temporal si i només si $\forall m \in S: n \leq m$.
\end{definition}

El $\max(S)$ i el $\min(S)$ no estan definits quan la sèrie temporal
és buida: $S= \emptyset$. En canvi si considerem el domini de temps
com a conjunt tancat, el suprem i l'ínfim estan definits per qualsevol
sèrie temporal, tal com passa amb el conjunt estès de nombres reals
\cite{cantrell:extendedreal}. A continuació seguim identificant amb
infinit ($\infty$) els valors impropis que tanquen el domini, tal com
hem exemplificat en el model estructural.
\begin{definition}[Suprem i
  ínfim]\label{def:sgst:sup}\label{def:sgst:inf}
  Sigui $S=\{m_0,\ldots,m_k\}$ una sèrie temporal i $n\in S$ una
  mesura.  Direm que $n=\glssymboldef{not:sgst:sup}(S)$ és el suprem
  de la sèrie temporal si $n=\max(S)$ en cas que el màxim estigui
  definit o $n=(-\infty,\infty)$ en cas contrari.  Direm que
  $n=\glssymboldef{not:sgst:inf}(S)$ és l'ínfim de la sèrie temporal
  si $n=\min(S)$ en cas que el mínim estigui definit o
  $n=(+\infty,\infty)$ en cas contrari.
\end{definition}

Quan la sèrie temporal no és buida, per
ser un conjunt finit i d'ordre total, sempre hi ha un i només un màxim
i un mínim i per tant es corresponen amb el suprem i l'ínfim
respectivament.


\begin{example}
  Siguin les sèries temporals $S_1=\{(1,1),m_2=(3,1)\}$ i $S_2=\{\}$
  aleshores les operacions següents són certes:
  $\min(S_1)=\inf(S_1)=(1,1)$ i $\sup(S_2)=(-\infty,\infty)$.
\end{example}





\subsubsection{Unió}


La unió de dos conjunts és un conjunt que conté tots els elements
d'ambdós conjunts.  Per a poder unir dos conjunts amb estructura de
relació, $A \cup B$, cal que tots dos tinguin la mateixa estructura;
és a dir, en termes de \gls{SGBDR} cal que $A$ i $B$ tinguin la
mateixa capçalera.

Per tal que l'operació d'unió de conjunts sigui vàlida per a les
sèries temporals cal, a més, tenir en compte quan dues sèries
temporals tenen mesures en el mateix instant de temps. En cas
d'utilitzar l'operació d'unió de conjunts la sèrie temporal resultant
no compliria amb la \autoref{def:serie_temporal} ja que contindria
mesures amb temps repetits. Com a conseqüència, es defineixen dues
operacions d'unió per a les sèries temporals que resolen la restricció
del temps de forma diferent.  Per a definir ambdues unions cal usar la
pertinença temporal ja que cal treballar amb els conjunts $t_A$ i
$t_B$, per tant és difícil establir la referència de pertinença per a
cada una.  Definim la primera unió com la més propera possible a la
unió de conjunts.



En primer lloc, es defineix la unió de dues sèries temporals que
escull les mesures del primer operand en cas de mesures amb el mateix
temps però diferent valor.
\begin{definition}[unió]
  Sigui $S_1=\{m_0^1, \dotsc, m_{k_1}^1\}$ i $S_2=\{m_0^2, \dotsc,
  m_{k_2}^2\}$ dues sèries temporals, la unió de les dues sèries
  temporals $S_1 \glssymboldef{not:sgst:cup} S_2$ és una sèrie
  temporal $S=\{m_0, \dotsc, m_k\}$ que conté totes les mesures de
  $S_1$ i les mesures de $S_2$ que no tenen temps repetits: $S_1 \cup
  S_2 = \{m^1 \in S_1 \vee m^2 \in S_2 | m^2
  \not\glssymbol{not:sgst:int} S_1 \}$.
\end{definition}

Propietats de la unió de sèries temporals:
\begin{itemize}
\item El cardinal de la sèrie temporal resultant està fitat a
  $|S_1| \leq |S| \leq |S_1| + |S_2|$. 
\item No commutativa. En general
  $S_1\cup S_2 \neq S_2\cup S_1$ tot i que sí que es compleix
  l'equivalència respecte al cardinal $|S_1 \cup S_2| = |S_2\cup S_1|$.
\end{itemize}

En segon lloc, es defineix la unió temporal de dues sèries temporals
que és la unió sense tenir en compte les mesures que tenen el mateix
instant de temps i diferent valor.
\begin{definition}[unió temporal]
  Sigui $S_1=\{m_0^1, \dotsc, m_{k_1}^1\}$ i $S_2=\{m_0^2, \dotsc,
  m_{k_2}^2\}$ dues sèries temporals, la unió temporal de les dues
  sèries temporals $S_1 \glssymboldef{not:sgst:cupt} S_2$ és una sèrie
  temporal $S=\{m_0, \dotsc, m_k\}$ que conté les mesures de $S_1$ i
  de $S_2$ excloent les que només comparteixen el temps: $S_1
  \glssymbol{not:sgst:cupt} S_2 = \{ m^1 \in S_1 \vee m^2 \in S_2 |
  m^1 \not\glssymbol{not:sgst:int} S_2 \vee m^1 \in S_2, m^2
  \not\glssymbol{not:sgst:int} S_1 \}$.
\end{definition}


Propietats de la unió temporal:
\begin{itemize}
\item Commutativa
\end{itemize}

%DEfinir quina de les dues és la unió i quina la unió temporal és difícil. Fem l'anterior definició perquè per la unió és la que més s'acosta a la unió de conjunts i compleix que $A \subseteq A \cup B$. El definim amb la pertinença temporal per comoditat, si no la definició és més rebuscada: $\{ m_1 \in S_1 \vee (m_2 \in S_2 \wedge (m_2\leq m_1 \vee m_1 \geq m_2) )$



\begin{example}\label{ex:model:s1s2}
  Siguin les dues sèries temporals $S_1=\{(1,1),(3,1),(4,0),(5,1)\}$
  and $S_2=\{(2,2),(3,2),(4,0),(6,2)\}$.  La unió de la primera amb la
  segona és $S_1 \cup S_2 = \{(1,1),(2,2), (3,1),
  (4,0),(5,1),(6,2)\}$. La unió temporal de totes dues és $S_1
  \glssymbol{not:sgst:cupt} S_2 = S_2 \glssymbol{not:sgst:cupt} S_1 =
  \{(1,1),(2,2),(4,0),(5,1),(6,2)\}$. %

  A la \autoref{fig:model:venn-unio} es mostren els diagrames Venn per
  a les dues operacions, on l'àrea pintada és la sèrie temporal
  resultant. L'àrea central d'intersecció dels dos conjunts són les
  mesures que comparteixen temps i valor, en aquest cas la mesura
  $(2,0)$. L'àrea central esquerra són les mesures de $S_1$ que només
  comparteixen temps amb una mesura de $S_2$, és a dir la $(3,1)$, i
  dualment a l'àrea central dreta hi ha la $(3,2)$. Les àrees més
  externes es corresponen amb la resta de mesures.  A la
  \autoref{fig:model:venn-unio} també es mostren les mateixes
  operacions amb la visualització en taula de les sèries temporals.

  \begin{figure}
    \centering 
    \def\escala{0.9}

\def\nodeA{node [above left=0.5cm and 0.1cm] {$(1,1)$} node [below left=0.5cm and 0.1cm] {$(5,1)$}}
\def\nodeB{node [above right=0.5cm and 0.1cm] {$(2,2)$} node [below right=0.5cm and 0.1cm] {$(6,2)$}}
\def\nodeT{node [above=0.1cm] {$(4,0)$} node [left=0.4cm] {$(3,1)$} node [right=0.4cm] {$(3,2)$}}
% Definition of circles
\def\firstcircle{(0,0) circle (1.5cm)}
\def\secondcircle{(0:2cm) circle (1.5cm)}
\def\thirdcircle{(0:1cm) circle (1.11cm)}

\colorlet{circle edge}{blue!50}
\colorlet{circle area}{blue!20}

\tikzset{
  filled/.style={fill=circle area, draw=circle edge, thick},
  outline/.style={draw=circle edge, thick},
  every node/.style={transform shape}
}

%\setlength{\parskip}{5mm}






%Set A or B
\begin{tikzpicture}[scale=\escala]
  \draw[filled] \firstcircle \nodeA;
    \begin{scope}
        \clip \secondcircle;
        \draw[filled, even odd rule] \firstcircle \nodeA
                                 \secondcircle 
                                 \thirdcircle;
   \end{scope}
    \draw[outline] \firstcircle
                   \secondcircle \nodeB
                   \thirdcircle \nodeT;

   \node[anchor=south] at (current bounding box.north) {$S_1 \cup S_2$};
\end{tikzpicture}
%Set temporal A or B
\begin{tikzpicture}[scale=\escala]
    \draw[filled, even odd rule] \firstcircle \nodeA
                                 \secondcircle \nodeB
                                 \thirdcircle \nodeT;
    \node[anchor=south] at (current bounding box.north) {$S_1 \cup^t S_2$};
\end{tikzpicture}






%%% Local Variables:
%%% TeX-master: "../main"
%%% ispell-local-dictionary: "british"
%%% End:


  \begin{tabular}[c]{|c|c|}
    \multicolumn{2}{c}{$S_1$} \\ \hline
    $t$  & $v$ \\ \hline
    1  & 1 \\
    3  & 1 \\
    4  & 0 \\
    5  & 1 \\ \hline
  \end{tabular} \qquad
  \begin{tabular}[c]{|c|c|}
    \multicolumn{2}{c}{$S_2$} \\ \hline
    $t$  & $v$ \\ \hline
    2  & 2 \\
    3  & 2 \\
    4  & 0 \\
    6  & 2 \\ \hline
  \end{tabular} \qquad
  \begin{tabular}[c]{|c|c|}
    \multicolumn{2}{c}{$S_1 \cup S_2$} \\ \hline
    $t$  & $v$ \\ \hline
    1  & 1 \\
    2  & 2 \\
    3  & 1 \\
    4  & 0 \\
    5  & 1 \\
    6  & 2 \\ \hline
  \end{tabular} \qquad
  \begin{tabular}[c]{|c|c|}
    \multicolumn{2}{c}{$S_1 \glssymbol{not:sgst:cupt} S_2$} \\ \hline
    $t$  & $v$ \\ \hline
    1  & 1 \\
    2  & 2 \\
    4  & 0 \\
    5  & 1 \\
    6  & 2 \\ \hline
  \end{tabular} 

    \caption{Diagrames Venn i taules per als exemples d'unió i
      d'unió temporal}
    \label{fig:model:venn-unio}
  \end{figure}

 \end{example}





\subsubsection{Diferència}

La diferència de dos conjunts és un conjunt que conté tots els
elements del primer conjunt que no pertanyen al segon.  Per a poder
restar dos conjunts amb estructura de relació, $A - B$, cal que tots
dos tinguin la mateixa estructura; és a dir, en termes de \gls{SGBDR}
cal que $A$ i $B$ tinguin la mateixa capçalera.
En la definició de l'operació de diferència cal tenir en compte les
dues pertinences possibles.

En primer lloc, es defineix la diferència atenent a la pertinença
estricta de conjunts. És a dir s'aplica la diferència de
conjunts a les sèries temporals.
\begin{definition}[diferència]
  Sigui $S_1=\{m_0^1, \dotsc, m_{k_1}^1\}$ i $S_2=\{m_0^2, \dotsc,
  m_{k_2}^2\}$ dues sèries temporals, la diferència de les dues sèries
  temporals $S_1 \glssymboldef{not:sgst:minus} S_2$ és una sèrie
  temporal $S=\{m_0, \dotsc, m_k\}$ que conté totes les mesures de
  $S_1$ que no pertanyen a $S_2$: $S_1 - S_2 = \{ m \in S_1 | m \notin
  S_2 \}$.
\end{definition}

En segon lloc, es defineix la diferència atenent a la pertinença
temporal.
\begin{definition}[diferència temporal]
  Sigui $S_1=\{m_0^1, \dotsc, m_{k_1}^1\}$ i $S_2=\{m_0^2, \dotsc,
  m_{k_2}^2\}$ dues sèries temporals, la diferència temporal de les
  dues sèries temporals $S_1 \glssymboldef{not:sgst:minust} S_2$ és
  una sèrie temporal $S=\{m_0, \dotsc, m_k\}$ que conté totes les
  mesures de $S_1$ que no pertanyen temporalment a $S_2$: $S_1
  \glssymbol{not:sgst:minust} S_2 = \{ m \glssymbol{not:sgst:int} S_1 | m
  \not\glssymbol{not:sgst:int} S_2 \}$.
\end{definition}




\subsubsection{Intersecció}

La intersecció de dos conjunts és un conjunt que conté els elements
comuns als dos conjunts.  Per a poder intersecar dos conjunts amb estructura
de relació, $A \cap B$, cal que tots dos tinguin la mateixa
estructura; és a dir, en termes de \gls{SGBDR} cal que $A$ i $B$ tinguin la
mateixa capçalera.

En la definició de l'operació d'intersecció cal tenir en compte les
dues pertinences possibles.

En primer lloc, es defineix la diferència atenent a la pertinença
estricta de conjunts. És a dir s'aplica l'operació d'intersecció de
conjunts.
\begin{definition}[intersecció]
  Sigui $S_1=\{m_0^1, \dotsc, m_{k_1}^1\}$ i $S_2=\{m_0^2, \dotsc,
  m_{k_2}^2\}$ dues sèries temporals, la intersecció de les dues
  sèries temporals $S_1 \glssymboldef{not:sgst:cap} S_2$ és una sèrie
  temporal $S=\{m_0, \dotsc, m_k\}$ que conté les mesures de $S_1$
  repetides a $S_2$: $S_1 \cap S_2 = \{ m \in S_1 | m \in S_2 \}$.
\end{definition}

En segon lloc, es defineix la intersecció atenent a la pertinença
temporal tenint en compte quan dues sèries temporals tenen mesures en
el mateix instant de temps però de valor diferent.
\begin{definition}[intersecció temporal]
  Sigui $S_1=\{m_0^1, \dotsc, m_{k_1}^1\}$ i $S_2=\{m_0^2, \dotsc,
  m_{k_2}^2\}$ dues sèries temporals, la intersecció temporal de les
  dues sèries temporals $S_1 \glssymboldef{not:sgst:capt} S_2$ és una
  sèrie temporal $S=\{m_0, \dotsc, m_k\}$ que conté les mesures de
  $S_1$ repetides temporalment a $S_2$: $S_1 \glssymbol{not:sgst:capt}
  S_2 = \{ m \glssymbol{not:sgst:int} S_1 | m \glssymbol{not:sgst:int} S_2 \}$.
\end{definition}

Propietats de la intersecció:
\begin{itemize}
\item La intersecció és commutativa però la intersecció temporal no és
  commutativa.
\item A partir de la diferència es pot definir la intersecció: $S_1
  \cap S_2 \equiv S_1 - (S_1 - S_2)$.
\end{itemize}


\subsubsection{Diferència simètrica}

La diferència simètrica de dos conjunts és un conjunt que conté els
elements no comuns dels dos conjunts. La diferència simètrica de dos
conjunts $A \ominus B$ es defineix a partir de la diferència i la
unió:
\begin{align*}
A \ominus B  & \equiv (A-B)\cup(B-A)\\
             & \equiv (A\cup B)-(A\cap B)  \\
A \ominus B  & \subseteq A\cup B
\end{align*}

Seguint aquestes propietats es defineixen dues diferències
simètriques: una a partir de la diferència i la unió de sèries
temporals i una altra a partir de la diferència temporal i la unió
temporal.  Per tal que l'operació de diferència simètrica sigui vàlida
per a les sèries temporals cal tenir en compte quan dues sèries
temporals tenen mesures en el mateix instant de temps.

En primer lloc, es defineix la diferència simètrica excloent les
mesures amb el mateix temps però de valor diferent.
\begin{definition}[diferència simètrica]
  Sigui $S_1=\{m_0^1, \dotsc, m_{k_1}^1\}$ i $S_2=\{m_0^2, \dotsc,
  m_{k_2}^2\}$ dues sèries temporals, la diferència simètrica de les
  dues sèries temporals $S_1 \glssymboldef{not:sgst:ominus} S_2$ és
  una sèrie temporal $S=\{m_0, \dotsc, m_k\}$ que conté les mesures de
  $S_1$ o exclusivament les de $S_2$: $S_1 \ominus S_2 = \{ m^1 \in
  S_1 \vee m^2 \in S_2 | m^1 \notin S_2, m^2
  \not\glssymbol{not:sgst:int} S_1 \}$.
\end{definition}

En segon lloc, es defineix la diferència simètrica temporal excloent les
mesures amb el mateix temps.
\begin{definition}[diferència simètrica temporal]
  Sigui $S_1=\{m_0^1, \dotsc, m_{k_1}^1\}$ i $S_2=\{m_0^2, \dotsc,
  m_{k_2}^2\}$ dues sèries temporals, la diferència simètrica de les
  dues sèries temporals $S_1 \glssymboldef{not:sgst:ominust} S_2$ és
  una sèrie temporal $S=\{m_0, \dotsc, m_k\}$ que conté les mesures de
  $S_1$ o exclusivament les de $S_2$: $S_1
  \glssymbol{not:sgst:ominust} S_2 = \{ m^1 \glssymbol{not:sgst:int}
  S_1 \vee m^2 \glssymbol{not:sgst:int} S_2 | m^1
  \not\glssymbol{not:sgst:int} S_2, m^2 \not\glssymbol{not:sgst:int}
  S_1 \}$.
\end{definition}



\subsubsection{Selecció}

La selecció és una operació dels \gls{SGBDR} que selecciona uns tuples
determinats d'un conjunt, a vegades també s'anomena restricció.

\begin{definition}[selecció]
  Sigui la sèrie temporal $S=\{ m_0,\dotsc,m_k\}$, $a_1$ i $a_2$ dos
  noms d'atributs que pertanyen a $S$, i $a_1 \Theta a_2$ una
  expressió booleana sobre $a_1$ i $a_2$, la selecció de $S$ per
  l'expressió booleana s'escriu com
  $\glssymboldef{not:sgst:select}_{a_1 \Theta a_2}(S)$  i es defineix
  de la mateixa manera que en els
  \gls{SGBDR} \parencite[cap.~7]{date04:introduction8}.
\end{definition}

En una forma més genèrica, l'expressió booleana pot incloure un o més
atributs i està formada per més d'una expressió lògica.


\begin{example}
  Sigui la sèrie temporal $S_1=\{(1,1),(3,1),(4,0),(5,1)\}$, la
  selecció dels temps més grans que $3$ és
  $\glssymbol{not:sgst:select}_{t>3}(S_1) = \{(4,0),(5,1)\}$.
\end{example}





\subsubsection{Projecció}


La projecció és una operació dels \gls{SGBDR} que selecciona uns
atributs determinats d'un conjunt. Aquesta operació treballa amb la
capçalera de la sèrie temporal, és a dir amb els atributs que
genèricament són $t$ i $v$ però que també poden tenir altres noms.

\begin{definition}[projecció]
  Sigui la sèrie temporal $S=\{ m_0,\dotsc,m_k\}$ i sigui $A=\{a_0,
  \dotsc, a_n\}$ un conjunt de noms d'atributs, la projecció de la
  sèrie temporal en els atributs s'escriu com
  $\glssymboldef{not:sgst:project}_A(S)$ i es defineix de la mateixa
  manera que en els
  \gls{SGBDR} \parencite[cap.~7]{date04:introduction8}. Aleshores
  aquesta nova sèrie temporal $\glssymbol{not:sgst:project}_A(S)$
  només inclou els atributs $A$ de les mesures.
\end{definition}

En l'operació de projecció, si els
atributs seleccionats no inclouen l'atribut temps o només n'inclouen
un el resultat no és una sèrie temporal sinó que és un conjunt
relacional. 


% \todo{definir? la definició embolica}
% \begin{definition}[projecció]
%   Sigui la sèrie temporal $S=\{ m_0,\dotsc,m_k\}$, on les mesures
%   poden ser multiavaluades $m_i=(t_i,v^1_i,\dotsc,v^n_i)$, expressada
%   en la forma completa amb capçalera $S = ( \{t, n_1,\dotsc,n_n \}, \{
%   \{ (t,t_0),(n_1,v^1_0),\dotsc,(n_n,v^n_0)\}, \dotsc, \{
%   (t,t_0),(n_1,v^1_1),\dotsc,(n_n,v^n_n)\} \} )$ on $t, n_1, \dotsc,
%   n_n$ són els noms dels atributs; i sigui $A=\{a_0, \dotsc, a_n\}$ un
%   conjunt de noms d'atributs. La projecció de la sèrie temporal en els
%   atributs és $\glssymboldef{not:sgst:project}_A(S) = \{ ()  \}$
% \end{definition}




\begin{example}
  Sigui la sèrie temporal $S_1=\{(1,1),(3,1),(4,0),(5,1)\}$, la
  projecció en l'atribut de temps és el conjunt
  $\glssymbol{not:sgst:project}_{\{t\}}(S_1) = \{ 1,3,4,5 \}$.  Sigui
  la sèrie temporal multivaluada $S_2 = ( (t,\text{ temp},\text{
    cons},\text{ vol}),\{ (2,1,2,3), (3,2,1,0), (6,1,2,3) \})$, la
  projecció en els atributs $t$ i \emph{temp} és la sèrie temporal
  $\glssymbol{not:sgst:project}_{ \{t,\text{temp}\}}(S_1) = (
  (t,\text{ temp}),\{ (2,1), (3,2), (6,1) \})$
\end{example}







\subsubsection{Reanomena}

El reanomena és una operació dels \gls{SGBDR} que canvia el nom dels
atributs.  Aquesta operació treballa amb la
capçalera de la sèrie temporal.


\begin{definition}[reanomena]
  Sigui la sèrie temporal $S=\{ m_0,\dotsc,m_k\}$, $a$ un nom
  d'atribut que pertany a $S$ i $b$ un que no hi pertany, reanomenar
  $a$ per $b$ s'escriu com $\glssymboldef{not:sgst:rename}_{a/b} (S)$
  i es defineix de la mateixa manera que en els
  \gls{SGBDR} \parencite[cap.~7]{date04:introduction8}.
\end{definition}

En una forma més genèrica, es poden reanomenar més d'un atribut
alhora.

\begin{example}
  Sigui la sèrie temporal multivaluada $S_2 = ( (t,\text{ temp},\text{
    cons},\text{ vol}),\{ (2,1,2,3), (3,2,1,0), (6,1,2,3) \})$, reanomenar 
  l'atribut \emph{temp} per \emph{v1} és la sèrie temporal
  $\glssymbol{not:sgst:rename}_{\text{temp}/\text{v1}}(S_1) = ( (t,\text{ v1},\text{
    cons},\text{ vol}),\{ (2,1,2,3), (3,2,1,0), (6,1,2,3) \})$.
\end{example}



\subsubsection{Producte i junció}

El producte cartesià de dos conjunts és un conjunt que conté totes les
parelles possibles d'elements d'ambdós conjunts.  Per a poder
multiplicar dos conjunts amb estructura de relació, $A \times B$, en
termes de \gls{SGBDR} cal que $A$ i $B$ no tinguin en comú noms
d'atributs.  En els \gls{SGBDR}, a diferència del producte de conjunts, el
conjunt resultant no és un conjunt de parells de tuples sinó un
conjunt de tuples.

Definim el producte de dues sèries temporals, les qual en
forma canònica tinguin els atributs $t$ i $v$, com una sèrie temporal
amb atributs $t_1$, $v_1$, $t_2$ i $v_2$. Així doncs, per a sèries
temporals el producte resulta en una sèrie temporal amb dos atributs
de temps, a la qual anomenem sèrie temporal doble (v.\
\autoref{def:sgst:st-doble}).
\begin{definition}[producte]
  Siguin $S_1=\{m_0^1, \dotsc, m_{k_1}^1\}$ i $S_2=\{m_0^2, \dotsc,
  m_{k_2}^2\}$ dues sèries temporals en forma canònica, el producte de
  les dues sèries temporals $S_1 \glssymboldef{not:sgst:times}
  S_2$ és una sèrie temporal doble $S=\{m_0, \dotsc, m_k\}$ que conté
  la unió de totes les parelles de mesures de $S_1$ i $S_2$: $S_1
  \times S_2 = \{ (t_1,v_1,t_2,v_2) | (t_1,v_1) \in S_1 \wedge
  (t_2,v_2) \in S_2 \}$
\end{definition}

Propietats del producte:
\begin{itemize}
\item El cardinal resultant és $|S|=|S_1||S_2|$
\item El grau resultant és $4$
\end{itemize}



La junció (\emph{join}) de dos conjunts és un conjunt que conté les
parelles d'elements d'ambdós conjunts que tenen el mateix valor per
als atributs comuns.  La junció de dos conjunts amb estructura de
relació, $A \join B$, es defineix com una selecció sobre el
producte \parencite[cap.~7]{date04:introduction8}.


Per a les sèries temporals, definim la junció com l'ajuntament de les
parelles que tenen el mateix atribut de temps en ambdues sèries
temporals . El resultat de la junció és una sèrie temporal
multivaluada.
\begin{definition}[junció]\label{def:sgst:join}
  Siguin $S_1=\{m_0^1, \dotsc, m_{k_1}^1\}$ i $S_2=\{m_0^2, \dotsc,
  m_{k_2}^2\}$ dues sèries temporals en forma canònica, la junció de
  les dues sèries temporals $S_1 \glssymboldef{not:sgst:join} S_2$ és
  una sèrie temporal multivaluada $S=\{m_0, \dotsc, m_k\}$ que
  selecciona del producte de $S_1$ amb $S_2$ les mesures dobles amb
  temps iguals: $S_1 \glssymbol{not:sgst:join} S_2 = \{ (t,v_1,v_2) |
  (t_1,v_1,t_2,v_2) \in S_1\times S_2 \wedge t=t_1=t_2 \}$.
\end{definition}


Propietats de la junció:
\begin{itemize}
\item El cardinal resultant és $|S|\leq\min(|S_1|,|S_2|)$
\item És commutativa; tenint en compte que els atributs tenen nom i
  per tant l'ordre no importa.
\end{itemize}




\begin{example}\label{ex:model:s1s2}
  Siguin les dues sèries temporals $S_1=\{(1,1),(3,1),(4,0),(5,1)\}$
  and $S_2=\{(2,2),(3,2),(4,0),(6,2)\}$.  La junció de totes dues és
  $S_1 \glssymbol{not:sgst:join} S_2 = \{(3,1,2),(4,0,0)\}$.
\end{example}



\subsubsection{Computacionals: mapa, agregació i plec}
\glsaddsec{not:sgst:computacional} %%%%secció d'operacions

Per a poder operar amb els conjunts, a més de l'àlgebra definida fins
ara, es necessiten operadors amb funcionalitats computacionals; és a
dir, operadors que calculin amb els valors continguts en els conjunts. 

En els \gls{SGBDR} els operadors computacionals bàsics són
\emph{extend}, \emph{aggregate} i
\emph{summarize} \parencite[cap.~7]{date04:introduction8}.  Per a les
sèries temporals definim operacions equivalents a les dues primeres de
la manera amb què habitualment s'utilitzen per als conjunts.  La
tercera, el \emph{summarize}, és una operació que s'utilitza per a
sintetitzar informació mitjançant grups, és a dir aplica operacions
\emph{aggregate} a conjunts que prèviament s'han agrupat segons un
atribut compartit.  Per a les sèries temporals, però, necessitem una
operació computacional més genèrica que ens permeti calcular
recursivament sense haver de definir grups.


 % Operacions
 %  d'agregació per intervals de temps,
 %  p.ex. $\{(gen,25),(feb,4),(mar,10)\}$, com es faria això?}
% No obstant, es pot aplicar el \emph{summarize} per a l'atribut de
% valors: summarize S per S {v} add ...  però això ja no mapa a una
% sèrie temporal.

% De l'operador \emph{aggregate} dels SGBDR definit per
% \textcite{date:introduction} cal tenir en compte que en defineix
% dues vessants. Per una banda, defineix els \emph{aggregate operator
% invocation} que retornen valors escalars. Per altra banda, defineix
% els \emph{aggregate operator invocation} que serveixen per a
% treballar amb el \emph{summarize}.

Així doncs, a continuació es defineix l'operador mapa (\emph{map}) com
a equivalent a l'\emph{extend}, l'operador agregació (\emph{aggregate})
com a equivalent a l'\emph{aggregate} i l'operador plec (\emph{fold})
com una forma més general de calcular recursivament amb les mesures
que \emph{summarize}.



L'operació de mapatge aplica una funció a cada element del conjunt.
\begin{definition}[mapa]
  Sigui $S=\{m_0, \dotsc, m_k\}$ una sèrie temporal i
  $\glssymboldef{not:sgst:fmap}$ una funció de mapatge sobre una
  mesura, $\glssymboldef{not:sgst:fmap}:m\mapsto m'$, el mapa de
  $\glssymbol{not:sgst:fmap}$ a $S$ és una sèrie temporal $S'=\{m_0',
  \dotsc, m_k'\}$ amb la funció aplicada a cada mesura:
  $\glssymboldef{not:sgst:map}(S,\glssymbol{not:sgst:fmap}) =
  \{\forall m\in S : \glssymbol{not:sgst:fmap}(m) \}$.
\end{definition}


L'operació d'agregació sintetitza en una mesura la informació dels
elements del conjunt segons un criteri, per exemple estadístics.
\begin{definition}[agregació]
  Sigui $S=\{m_0, \dotsc, m_k\}$ una sèrie temporal, $m_i$ una mesura
  inicial i $\glssymboldef{not:sgst:fagg}$ una funció d'agregació de
  dues mesures, $\glssymbol{not:sgst:fagg}: m_r \times m_b \mapsto
  m_r'$, l'agregació de $S$ segons $\glssymbol{not:sgst:fagg}$ és una
  mesura $m' = (t',v')$ que sintetitza la informació de les mesures
  seguint el criteri de la funció a partir de la mesura inicial:
  $\glssymboldef{not:sgst:aggregate}(S,m_i,\glssymbol{not:sgst:fagg})
  = \glssymbol{not:sgst:fagg}(\dots(
  \glssymbol{not:sgst:fagg}(\glssymbol{not:sgst:fagg}(\glssymbol{not:sgst:fagg}
  (m_i, m_0), m_1), m_2 )\dots),m_k)$.
\end{definition}

% Més compactament descrit amb
% \begin{align*}
%   \text{fold}: & S=\{m_0,\dotsc,m_k\} \times m_i \times f \longrightarrow m'= \\
%   & \begin{cases}
%     m_i & \text{si} |S|=0, \\
%     \text{fold}(S_1,f(m_i,m_1),f) & \text{altrament}
%   \end{cases}\\
%   \text{ a on } & m_1 \in S, S_1 = S - \{m_1\}
% \end{align*}


L'operació de plegament combina recursivament els elements del conjunt
segons un criteri.
\begin{definition}[plec]
  Sigui $S=\{m_0, \dotsc, m_k\}$ una sèrie temporal, $S_i=\{m_{i0},
  \dotsc, m_{ik}\}$ una sèrie temporal inicial i
  $\glssymboldef{not:sgst:ffold}$ una funció de plegament d'una mesura
  amb una sèrie temporal, $\glssymbol{not:sgst:ffold}: S_r \times m
  \mapsto S_r'$, el plec de $S$ segons $\glssymbol{not:sgst:ffold}$ és
  una sèrie temporal $S'= \{m_0', \dotsc, m_k'\}$ amb les mesures
  combinades seguint el criteri de la funció a partir de la sèrie
  temporal inicial:
  $\glssymboldef{not:sgst:fold}(S,S_i,\glssymbol{not:sgst:ffold}) =
  \glssymbol{not:sgst:ffold}(\dots(
  \glssymbol{not:sgst:ffold}(\glssymbol{not:sgst:ffold}(
  \glssymbol{not:sgst:ffold}(S_i,m_0),m_1),\allowbreak m_2
  )\dots),\allowbreak m_k)$.
\end{definition}


Les operacions d'agregació i plegament tal com s'han definit es
realitzen en ordre aleatori de mesures. Segons el criteri que
s'utilitzi, l'ordre és important i per tant cal una operació que
computi tenint-lo en compte. A tal efecte, a continuació s'amplia la
definició de la funció de plegament per a tenir en compte l'ordre; per
a la funció d'agregació es pot aplicar el mateix concepte.
\begin{definition}[plec amb ordre]
  Sigui $S=\{m_0, \dotsc, m_k\}$ i $S_i=\{m_{i0}, \dotsc, m_{ik}\}$
  dues sèries temporals, $\glssymbol{not:sgst:ffold}$ una funció d'una
  mesura amb una sèrie temporal i $\glssymbol{not:sgst:forder}$ una
  funció que retorna una mesura continguda a una sèrie temporal,
  $\glssymboldef{not:sgst:forder}: S_c \mapsto m'$, el plec de $S$
  segons $\glssymbol{not:sgst:ffold}$ amb valor inicial $S_i$ i ordre
  $\glssymbol{not:sgst:forder}$ és una sèrie temporal $S'= \{m_0',
  \dotsc, m_k'\}$ que combina les mesures seguint
  el criteri i l'ordre de les funcions:
  $\glssymboldef{not:sgst:ofold}(S,S_i,
  \glssymbol{not:sgst:ffold},\glssymbol{not:sgst:forder}) =
  \begin{cases}
    S_i & \text{si } |S|=0, \\
    \glssymbol{not:sgst:ofold}(S_o,\glssymbol{not:sgst:ffold}(S_i,m_o),
    \glssymbol{not:sgst:ffold},\glssymbol{not:sgst:forder}) &
    \text{altrament}
  \end{cases}$ on $m_o = \glssymbol{not:sgst:forder}(S)$ i $S_o = S
  - \{m_o\}$.
\end{definition}

El plec amb ordre és necessari quan la funció
$\glssymbol{not:sgst:ffold}$ no és associativa ni commutativa perquè
llavors l'ordre dels càlculs és important. Es pot observar que el plec
sense ordre és un plec amb ordre aleatori:
$\glssymbol{not:sgst:fold}(S,S_i,\glssymbol{not:sgst:ffold})\equiv
\glssymbol{not:sgst:ofold}(S,S_i,\glssymbol{not:sgst:ffold},
\glssymbol{not:sgst:forder})$ a on
$\glssymbol{not:sgst:forder}=\text{aleatori}(S_c)$.

% la funció de plegament en els plecs sense ordre ha de ser commutativa i ¿associativa? Si no ho és, el resultat és aleatori
%De manera semblant $\agg(S,m_i,f)\equiv \agg(S,m_i,f,o)$ on $o=\text{aleatori}(S)$.



Propietats de les operacions computacionals:
\begin{itemize}
\item El plec d'una sèrie temporal buida és la sèrie inicial;
  $\glssymbol{not:sgst:fold}: \{\} \times S_i \times
  \glssymbol{not:sgst:ffold} \mapsto S_i$.

\item El plec per una funció que sempre retorni la sèrie inicial és la
  sèrie inicial; $\glssymbol{not:sgst:fold}: S \times S_i \times \glssymbol{not:sgst:ffold} \mapsto S_i$ on
  $\glssymbol{not:sgst:ffold}: S_r \times m \mapsto S_r$.

\item El plec per una funció que només retorni la mesura original és
  una sèrie amb una sola mesura; $\glssymbol{not:sgst:fold}: S \times
  S_i \times \glssymbol{not:sgst:ffold} \mapsto S'$ on
  $\glssymbol{not:sgst:ffold}:S_r\times m \mapsto \{m\}$ i $|S'|=1$.


\item La funció d'unió en el plegament permet fer la identitat, $S
  \equiv \glssymbol{not:sgst:fold}(S,\{\},\glssymbol{not:sgst:ffold})$
  on $\glssymbol{not:sgst:ffold}: S_r \times m \mapsto S_r \cup \{m\}$.


\item Els mapes es poden implementar com a plecs,
  $\glssymbol{not:sgst:map}(S,\glssymbol{not:sgst:fmap}) \equiv
  \glssymbol{not:sgst:fold}(S,\{\},\glssymbol{not:sgst:ffold})$ on
  $\glssymbol{not:sgst:ffold}: S_r \times m \mapsto
  \{\glssymbol{not:sgst:fmap}(m)\} \cup S_r$.  De manera semblant,
  \textcite{lammel08:mapreduce} també exemplifica com els mapes es
  poden implementar com a plecs. %pàg. 5 Asides on folding

\item Les agregacions es poden implementar com a plecs,
  $\glssymbol{not:sgst:aggregate}(S,m_i,\glssymbol{not:sgst:fagg})
  \equiv
  \glssymbol{not:sgst:fold}(S,\{m_i\},\glssymbol{not:sgst:ffold})$ on
  $\glssymbol{not:sgst:ffold}: S_r \times m \mapsto
  \{\glssymbol{not:sgst:fagg}(m_r,m)\}$ i $S_r =\{m_r\}$.

\end{itemize}





\begin{example}
  Definicions de funcions d'exemple a partir de l'operació
  computacional de mapatge:

\begin{itemize}
\item $\operatorname{identitat}: S \mapsto S'$ on $S'=
  \glssymbol{not:sgst:map}(S,(t,v)\mapsto(t,v))$
\item $\operatorname{intercanvi}: S \mapsto S'$ on $S'=
  \glssymbol{not:sgst:map}(S,(t,v)\mapsto(v,t))$
\item $\glssymbolex{not:sgst:duplica-t}: S \mapsto S'$ on $S'=
  \glssymbol{not:sgst:map}(S,(t,v)\mapsto(t,t))$
\item $\operatorname{translaci\acute{o}}: S \times d \mapsto S'$ on $S'=
  \glssymbol{not:sgst:map}(S,(t,v)\mapsto(t+d,v))$
\item $\operatorname{multiplica\_tv}: S \mapsto S'$ on $S'=
  \glssymbol{not:sgst:map}(S,(t,v)\mapsto(t,t\cdot v))$

%Encara no s'ha definit ant, per tant no es pot fer l'exemple
% \item $\operatorname{tpredecessors_{v1}}: S \mapsto S'$ on $S'= \glssymbol{not:sgst:map}(S,(t,v)
%   \mapsto (t,T(\ant_S(m)))$, usant l'operació predecessor de la
%   \autoref{def:sgst:ant}
% \item $\operatorname{vpredecessors}: S \mapsto S'$ on $S'= \glssymbol{not:sgst:map}(S,(t,v)
%   \mapsto (t,V(\ant_S(m)))$, usant l'operació predecessor de la
%   \autoref{def:sgst:ant}

\end{itemize}
\end{example}

\begin{example}
  Definicions de funcions d'exemple a partir de l'operació
  computacional d'agregació. Ens els exemples següents utilitzem la
  notació $(t^i,v^i) \times(t,v)$ per a indicar la funció
  $\glssymbol{not:sgst:fagg}: m_r \times m_b \mapsto m'$ on $t^i
  = T(m_r)$, $v^i=V(m_r)$, $t=T(m_b)$ i $v=V(m_b)$.


\begin{itemize}
\item $\operatorname{cardinal}: S \mapsto v'\in \glssymbol{not:N}$ on
  $v'=V\big( \glssymbol{not:sgst:aggregate}
  (S,(0,0),\glssymbol{not:sgst:fagg}) \big)$ i
  $\glssymbol{not:sgst:fagg}: (t^i,v^i) \times(t,v) \mapsto
  (t^i,v^i+1) $. Aquesta funció és una implementació del cardinal de
  la \autoref{def:sgst:cardinal} a partir de l'agregació.


\item $\glssymbolex{not:sgst:sumav}: S \mapsto v'$ on $v'=V\big(
  \glssymbol{not:sgst:aggregate} (S,(0,0),\glssymbol{not:sgst:fagg} )
  \big)$ i $\glssymbol{not:sgst:fagg}: (t^i,v^i) \times (t,v)
  \mapsto(t^i,v+v^i)$

\item $\glssymbolex{not:sgst:mitjanav}: S \mapsto v'$ on $v'=
  \operatorname{suma\_v}(S) / \operatorname{cardinal}(S)$ 
  % alerta amb 0/0, es deixa indefinit...

\item $\operatorname{sup}: S \mapsto m'$ on $m'=
  \glssymbol{not:sgst:aggregate}(S,(-\infty,\infty),\glssymbol{not:sgst:fagg}
  )$ i $\glssymbol{not:sgst:fagg}: (t^i,v^i)\times(t,v) \mapsto \big(
  (t^i,v^i) \text{ si } t < t^i, \text{ o } (t,v) \text{ en cas
    contrari}\big) $. Aquesta funció és una implementació de
  l'operació suprem de la \autoref{def:sgst:sup} a partir de
  l'agregació.

\item $\glssymbolex{not:sgst:maxv}: S \mapsto v'$ on $v'= V\big(
  \glssymbol{not:sgst:aggregate}
  (S,(0,-\infty),\glssymbol{not:sgst:fagg} ) \big)$ i
  $\glssymbol{not:sgst:fagg}: (t^i,v^i) \times (t,v)
  \mapsto(t^i,\max(v,v^i))$. A diferència del $\sup(S)$ o del
  $\max(S)$, el $\glssymbol{not:sgst:maxv}(S)$ calcula el màxim dels
  valors.


%Encara no s'ha definit ant, per tant no es pot fer l'exemple
% \item $\operatorname{ant}: S \times m \mapsto m'$ on $m'=
%   \glssymbol{not:sgst:aggregate}(S,(-\infty,\infty),(t^i,v^i)\times(t,v)\mapsto
%   [(t,v) \text{ if } t^i < t < T(m) \text{ else } (t^i,v^i)
%   ])$. Aquesta funció és una implementació de l'operació predecessor
%   de la \autoref{def:sgst:ant} a partir de l'agregació.
\end{itemize}
\end{example}

\begin{example}
  Definicions de funcions d'exemple a partir de l'operació
  computacional de plegament. 

\begin{itemize}
\item  $\operatorname{tpredecessors}: S \mapsto S'$ on $S'=
  \glssymbol{not:sgst:fold}(S,S, \glssymbol{not:sgst:ffold})$ on $
  \glssymbol{not:sgst:ffold}: S_r\times m \mapsto \{(T(m),t_p)\} \cup
  S_r$ i $t_p=T(\sup( \glssymbol{not:sgst:select}_{t <
    T(m)}(S_r)))$. Per a cada mesura de la sèrie temporal indica quin
  és el temps de la mesura precedent.
  % , sense usar l'operació predecessor  a diferència de l'exemple $\operatorname{tpredecessors_{v1}}$ 
  %De fet aquesta operació es pot definir amb un mapa en comptes de plec i potser s'entén més? S'=mapa(S,(t',v') mapsto (t', T(sup(seleccio(S,t<t')))  ))

\item $\glssymbolex{not:sgst:vpredecessors}: S \mapsto S'$ on $S'=
  \glssymbol{not:sgst:map}(\operatorname{tpredecessors}(S),
  \glssymbol{not:sgst:fmap})$ i $\glssymbol{not:sgst:fmap}:
  (t',v')\mapsto (t, V(\sup(\glssymbol{not:sgst:select}_{t=v'}(S))) ) $.
  Per a cada mesura de la sèrie temporal indica quin és el valor de la
  mesura precedent, és una definició a partir de l'operació de
  $\operatorname{tpredecessors}$.

\end{itemize}

\end{example}


\begin{example}
  Sigui la sèrie temporal $S=\{(1,1),(3,1),(4,0),(5,1)\}$. %
  La duplicació dels temps en els valors de la sèrie temporal és
  $\glssymbol{not:sgst:duplica-t}(S)=\{(1,1),(3,3),(4,4),(5,5)\}$. %
  La mitjana dels valors de la sèrie temporal és
  $\glssymbol{not:sgst:mitjanav}(S) = 0,75$. %
  Els temps predecessors de cada mesura de la sèrie temporal és
  $\operatorname{tpredecessors}(S)=\{(1,-\infty),(3,1),(4,3),(5,4)\}$
  i els valors predecessors
  $\glssymbol{not:sgst:vpredecessors}(S)=\{(1,\infty),(3,1),(4,1),(5,0)\}$.
\end{example}




\subsubsection{Computacionals binàries amb els valors}

Una operació en els conjunts és la que aplica un operador binari a
totes les parelles possibles dels elements de dos conjunts. Per
exemple la suma, és a dir l'operador binari $+$, aplicada a dos
conjunts $A$ i $B$ és un conjunt $A + B = \{ e_a+e_b : (e_a,e_b) \in
A\times B \}$.

Per a les sèries temporals també calen operacions computacionals amb
les mesures de dues sèries temporals. En el cas d'operar amb dues
sèries temporals primer cal ajuntar les dues sèries temporals que es
volen operar i després aplicar les operacions computacionals binàries
a la sèrie temporal resultant.


El producte i la junció són els operadors que permeten crear parelles
de mesures de dues sèries temporals. Per a operar amb els valors de
dues sèries temporals la junció és més adequada ja que permet ajuntar
el valors que tenen temps comuns. Així doncs, es defineix l'aplicació
d'un operador binari de valors a dues sèries temporals a partir de la
junció.
\begin{definition}[Operació computacional binària amb els valors]
  Siguin $S_1=\{m_0^1, \dotsc, m_{k_1}^1\}$ i $S_2=\{m_0^2, \dotsc,
  m_{k_2}^2\}$ dues sèries temporals i sigui
  $\glssymboldef{not:sgst:opbinari}$ un operador binari que computi
  amb dos valors, $\glssymbol{not:sgst:opbinari}:v_1 \times v_2
  \mapsto v'$, l'aplicació d'aquest operador binari a dues sèries
  temporals és $\glssymboldef{not:sgst:computacionalbinaria}: S_1
  \times S_2 \longrightarrow S'$ on $S' = \glssymbol{not:sgst:map}
  (S_1 \glssymbol{not:sgst:join} S_2, (t,v_1,v_2) \mapsto (t,v_1
  \glssymbol{not:sgst:opbinari} v_2))$. Usem la notació
  $m=(t,v_1,v_1)$ on $m\in S_1 \glssymbol{not:sgst:join} S_2$.
\end{definition}

Cal tenir en compte que la junció de la \autoref{def:sgst:join} només
sap operar amb dues sèries temporals que tinguin el mateix vector de
temps; és a dir regulars entre elles (v.\ \autoref{def:sgst:regulars_entre_elles}). En el cas que no tinguin el mateix vector
de temps, es pot aplicar la junció temporal de la
\autoref{def:sgst:joint}.


\begin{example}
  Exemples de l'aplicació d'operacions computacionals binàries en què
  s'aplica un operador binari $ \glssymbol{not:sgst:opbinari}$ als
  valors de dues sèries temporals
  \begin{itemize}
  \item $S' = S_1 + S_2$ 
  \item $S' = S_1 \operatorname{subtraccio} S_2$. Indiquem amb el
    nom complet, $\operatorname{subtraccio}$, l'operació computacional
    de dues sèries temporals corresponent a l'operació aritmètica de
    resta, per no confondre-la amb l'operació de diferència de
    conjunts que té el guionet per símbol (\emph{-}).
  \end{itemize}
  
  Les operacions computacionals binàries també es poden usar per a
  definir altres operacions, per exemple per a calcular els increments de valor
  d'una sèrie temporal:
  \begin{itemize}
  \item $\glssymbolex{not:sgst:increments}: S \mapsto S'$ on $S'=
    S \operatorname{subtraccio} \glssymbol{not:sgst:vpredecessors}(S)$.
  \end{itemize}
\end{example}




\begin{example}
  Siguin les dues sèries temporals $S_1=\{(1,1),(3,1),(4,0),(5,1)\}$
  and $S_2=\{(1,2),(3,2),(4,0),(5,2)\}$. %
  La suma de les dues sèries temporals és
  $S_1+S_2=\{(1,3),(3,3),(3,3),(4,0)\}$. %
  Els increments de la primera sèrie temporal són
  $\glssymbol{not:sgst:increments}(S_1)=\{(1,\infty),(3,0),(4,-1),(5,1)\}$.
\end{example}




\subsection{Bàsiques de seqüències}
\glsaddsec{not:op-sequencies} %%%%secció d'operacions


Atesa la relació d'ordre induïda pel temps en una sèrie temporal
(def.\ \ref{def:model:mesura-relacio-ordre}), les sèries temporals es
poden tractar com a seqüències.  En aquest apartat definim operadors
per a les sèries temporals recollint els operadors habituals que tenen
les seqüències. 

Els operadors que treballen amb seqüències tenen en compte l'atribut
que marca un ordre total en el conjunt. En el cas de les sèries
temporals aquest atribut és el temps.



\subsubsection{Interval}

L'interval sobre una seqüència és la subseqüència compresa entre dos
elements.  Per a les sèries temporals és possible definir el concepte
d'interval sobre la seqüència com la subsèrie entre dos instants de
temps, semblant a com es fa a \cite{last:keogh,last:hetland}.  És una
operació de selecció però amb la notació habitual en les seqüències.


\begin{definition}[Interval]
  \label{def:model:st-interval}
  Sigui $S=\{m_0, \ldots, m_k\}$ una sèrie temporal i siguin $r$ i $t$
  dos instants de temps. Definim el subconjunt
  $S\glsdispdef{not:sgst:interval}{(r,t)} \subseteq S$ com la sèrie
  temporal $S(r,t)=\{m\in S | r<T(m)<t\}$.

  Tal com es fa en les seqüències, es defineix una notació de
  parèntesis i claudàtors per indicar si l'interval és obert, tancat o
  semiobert:

  $S\glsdispdef{not:sgst:intervalsemi}{[r,t)}=\{m\in S  | r\leq T(m)< t\}$

  $S(r,t]=\{m\in S  | r<T(m)\leq t\}$

  $S\glsdispdef{not:sgst:intervaltancat}{[r,t]}=\{m\in S | r\leq
  T(m)\leq t\}$
\end{definition}


Propietats:
\begin{itemize}
\item La subsèrie $S[-\infty,t)\subseteq S$ és equivalent a la sèrie
  temporal $S[-\infty,t) \equiv S[T(\inf(S)),t)$. De la mateixa manera
  $S(r,+\infty] \equiv S(r,T(\sup(S))]$.

\item L'interval degenerat $S[t,t]\subseteq S$ és equivalent a la
  sèrie temporal $S[t,t] \equiv \{m\in S | T(m)=t \}$. El intervals
  $S(t,t]\subseteq S$ i $S[t,t)\subseteq S$ són equivalents a la sèrie
  temporal buida $S(t,t] \equiv S[t,t) \equiv \emptyset$ ja que per
  ser els temps d'ordre total $\nexists T(m): t < T(m) \leq t$ o
  $\nexists T(m): t \leq T(m) < t$, respectivament. 

\item La subsèrie $S[-\infty,+\infty] \subseteq S$ és equivalent a la
  sèrie temporal original $S[-\infty,+\infty] = S$. La subsèrie
  $S(-\infty,+\infty) \subseteq S$ només és equivalent a la sèrie
  temporal original quan aquesta no conté mesures indefinides
  $S(-\infty,+\infty) \equiv S: (-\infty,v_a)\notin S \wedge
  (+\infty,v_b)\notin S$.
\end{itemize}




\subsubsection{Successió}

Atenent a la relació d'ordre induïda pel temps en una sèrie temporal, es
defineix el concepte de mesura següent i mesura anterior en una
seqüència.


\begin{definition}[Successor i
  predecessor]\label{def:sgst:seg}\label{def:sgst:ant}
  Sigui $S=\{m_0, \ldots, m_k\}$ una sèrie temporal i siguin $l$ i $n$
  dues mesures on la primera pertany a la sèrie temporal, $l\in
  S$. Direm que $l$ és el successor de $n$ en $S$ i ho notarem com
  $l=\glssymboldef{not:sgst:next}_S(n)$ si i només si
  $l=\inf(S(T(n),+\infty])$.  Direm que $l$ és el predecessor de $n$
  en $S$ i ho notarem com $l=\glssymboldef{not:sgst:prev}_S(n)$ si i
  només si $l=\sup(S[-\infty,T(n)))$.

  Quan no hi hagi dubte de la sèrie temporal que marca l'ordre, per
  exemple quan $n\in S$, podrem escriure
  $\glssymbol{not:sgst:next}(n)$ i $\glssymbol{not:sgst:prev}(n)$.
\end{definition}

S'observa que s'obtenen mesures indefinides en els casos que la
mesura següent o anterior es calcula respectivament per la mesura
suprema o ínfima de la sèrie temporal: $\glssymbol{not:sgst:next}_S(\sup
S)=(+\infty,\infty)$ i $\glssymbol{not:sgst:prev}_S(\inf S)=(-\infty,\infty)$.

De la definició anterior es dedueix que donada una sèrie temporal $S$
que no conté mesures indefinides i donada la mesura indefinida
$o=(+\infty,\infty)$, el predecessor de $o$ sempre és el suprem de la
sèrie temporal $\glssymbol{not:sgst:prev}_S( (+\infty,\infty) ) =
\sup(S): \forall m\in S:
T(m)\in\glssymbol{not:R}$.  % S\equiv S(-\infty,+\infty)
\emph{Demostració: Sigui $S$ una sèrie temporal i $o=(+\infty,\infty)$
  una mesura indefinida, el predecessor de $o$ en $S$ és una mesura
  $l=\glssymbol{not:sgst:prev}_S(o)$ que compleix
  $l=\sup(S[-\infty,T(o)))$. Substituint, s'obté que
  $l=\sup(S[-\infty,+\infty))=\sup(S-m):m\in S:T(m)=+\infty \notin
  \glssymbol{not:R}$, i per tant com que $S$ no té mesures indefinides es
  demostra que $l=\sup(S)$.  } De manera semblant es pot demostrar que
$\glssymbol{not:sgst:next}_S( (-\infty,\infty) ) = \inf(S): \forall
m\in S: T(m)\in\glssymbol{not:R}$.


\subsubsection{Concatenació}

La concatenació és una operació que uneix dues seqüències amb els
elements de la primera seqüència seguits pels de la segona. Així
doncs, la concatenació de les seqüències té un sentit semblant al que
la unió té en els conjunts. 

Per a les sèries temporals, per tal que l'operació de concatenació
uneixi amb ordre els operands, cal tenir en compte l'interval que
ocupa cada sèrie temporal segons el seu atribut de temps.  És a dir,
la concatenació de dues sèries temporals consisteix a unir la part de
la segona sèrie temporal que no està inclosa en el rang temporal de la
primera.

Per a poder concatenar dues sèries temporals cal que ambdues tinguin
la mateixa estructura, de la mateixa manera que ja s'ha vist amb
l'operació d'unió.


\begin{definition}[concatenació]
  Sigui $S_1=\{m_0^1, \dotsc, m_{k_1}^1\}$ i $S_2=\{m_0^2, \dotsc,
  m_{k_2}^2\}$ dues sèries temporals, la concatenació de les dues
  sèries temporals $S_1 \glssymboldef{not:sgst:concatenate} S_2$ és
  una sèrie temporal $S=\{m_0, \dotsc, m_k\}$ que conté totes les
  mesures de $S_1$ i les mesures de $S_2$ que no intersequen en
  l'interval de $S_1$; $S_1 \glssymbol{not:sgst:concatenate} S_2 = S_1
  \cup ( S_2 - S_2[t_1,t_2] )$ on $t_1=T(\inf S_1)$ i $t_2=T(\sup
  S_1)$.
\end{definition}


Propietats
\begin{itemize}
\item La concatenació no és commutativa
\end{itemize}







\subsection{Funció temporal}
\label{sec:sgst:operadors-temporals}
\glsaddsec{not:op-funcio} %%%%secció d'operacions

Atenent al fet que una sèrie temporal és la representació d'un funció
contínua cal definir operacions per a tractar convenientment aquesta
naturalesa.
En aquest apartat definim aquestes operacions com una redefinició de
les bàsiques anteriors, i així poder aplicar-les considerant una sèrie
temporal com una funció contínua.  


A l'\autoref{sec:model:repr} es detalla més el concepte de funció
representació contínua d'una sèrie temporal i s'ofereixen exemples de
diversos mètodes de representació. Les operacions definides a
continuació han de ser contextualitzades per a un mètode de
representació particular, ho indiquem a cada operació amb un
superíndex $r$. De fet, a partir de l'operació d'interval temporal es
poden definir els altres operadors de funció temporals; per tant cal
oferir una definició d'interval temporal per a cada mètode de
representació que es vulgui usar.



\subsubsection{Interval temporal}

Sigui $S$ una sèrie temporal i $[t_0,t_f]$ un interval de temps, per
una banda s'ha definit l'interval sobre la seqüència d'una sèrie
temporal $S(t_0,t_f)$ (v.\ \autoref{def:model:st-interval}) i per
altra banda la sèrie temporal pot tenir un mètode de representació $r$
que permet calcular la funció contínua de la sèrie temporal $S(t)^r$
(v.\ \autoref{def:model:frepr}).  Per seleccionar un interval temporal
cal tenir en compte tant l'interval sobre la seqüència com la
representació de la sèrie temporal que pot incloure noves mesures al
resultat.

\begin{definition}[Interval temporal]
  \label{def:sgst:intervalt}
  Sigui $S=\{m_0, \ldots, m_k\}$ una sèrie temporal, $[t_0,t_f]$ un
  interval de temps i $r$ un mètode de representació, l'interval
  temporal $\glsdispdef{not:sgst:intervalt}{S[t_0,t_f]^r}$ és una
  sèrie temporal amb les mesures que són dins del rang temporal $i$
  segons marca la funció de representació: $S[t_0,t_f]^r= \forall t
  \in [t_0,t_f] : S' = S(t)^r$
\end{definition}

Aquesta és una definició genèrica difícil d'implementar, per tant per
a cada mètode de representació cal interpretar una operació d'interval
temporal.  A continuació s'exemplifica utilitzant el mètode de
representació \gls{zohe}. Més endavant, un cop profunditzat en el
concepte de funció de representació, es detalla la representació
\gls{zohe} (v.\ \autoref{def:model:zohe}) i s'ofereixen més exemples
d'intervals temporals amb altres mètodes de representació (v.\
\autoref{sec:sgst:repr-intervaltemporal}).
\begin{definition}[Interval temporal \gls{zohe}]
  \label{def:sgst:interval-temporal-zohe}
  Sigui $S$ una sèrie temporal, $[t_0,t_f]$ un interval de temps i el
  mètode representació $r=\gls{zohe}$, es defineix l'interval temporal
  \gls{zohe} $S[t_0,t_f]^{\glssymbol{not:zohe}}$ com la sèrie temporal
    $S[t_0,t_f]^{\glssymbol{not:zohe}} = S(t_0,t_f] \cup \{m\}$ on $m=(t_f,v)$
    i $v= V(\inf( S[t_f,+\infty] ))$.
\end{definition}
  %Atenció S(t_0,t_f] \cup \{m\} no és equivalent a  (S \cup \{m\})(t_0,t_f] ni sabent que m=(t_f,v); comprovar-ho pel cas t_0=t_f



Propietats de l'interval temporal:

\begin{itemize}
\item Sigui $t_a$ un instant de temps, l'interval temporal
  $S[t_a,t_a]^r$ és equivalent a la funció contínua de la sèrie
  temporal avaluada en aquest punt: $S[t_a,t_a]^r\equiv \{(t_a,
  S(t_a)^r)\}$.

%$S[T(\min(S),T(\max(S)]^r \neq S$
%$S[-\infty,+\infty]^zohe = S \cup \{ (+\infty,\infty) \}$

\end{itemize}




\subsubsection{Selecció temporal}


La selecció  temporal d'una sèrie temporal permet canviar, en el
context d'una representació, la resolució a una de marcada per un
conjunt d'instants de temps. 


\begin{definition}[Selecció temporal]
  \label{def:sgst:selecciot}
  Sigui $S=\{m_0, \ldots, m_k\}$ una sèrie temporal,
  $i=\{t_0,t_1,\dotsc,t_n\}$ un conjunt d'instants de temps i $r$ un
  mètode de representació, la selecció temporal
  $\glsdispdef{not:sgst:selectt}{S[i]^r}$ és una sèrie temporal que
  conté mesures amb els temps d'$i$ segons marca el mètode de
  representació: $S[i]^r= S[t_0,t_0]^r \cup S[t_1,t_1]^r \cup \dotsb
  \cup S[t_n,t_n]^r$.
\end{definition}



Propietats de la selecció temporal:
\begin{itemize}

\item El cardinal de la sèrie temporal resultant és el mateix que el
  del conjunt d'instants de temps $|S[i]^r| = |i|$

\item La selecció temporal d'una sèrie temporal en un conjunt de temps
  equi-espaiat $i = \{\tau+n\delta | n,j\in\glssymbol{not:N}, n\leq j
  \}$ és una sèrie temporal regular $S[i]^r \equiv \{ (\tau, v_0),
  (\tau+\delta,v_1), \dotsc , (\tau+j\delta,v_j)\}$
\end{itemize}




\subsubsection{Concatenació temporal}

La concatenació temporal és l'operació de concatenació que té en
compte la representació de les sèries temporals.  És a dir, la
concatenació temporal de dues sèries temporals uneix la part de la
segona sèrie temporal que no està inclosa en l'interval temporal de la
primera.


\begin{definition}[concatenació temporal]
  Sigui $S_1=\{m_0^1, \dotsc, m_{k_1}^1\}$ i $S_2=\{m_0^2, \dotsc,
  m_{k_2}^2\}$ dues sèries temporals i $r$ un mètode de representació,
  la concatenació temporal de les dues sèries temporals $S_1
  \glssymboldef{not:sgst:concatenatet}^r S_2$, és una sèrie temporal
  que conté les mesures de $S_1$ i les mesures de $S_2$ que no
  intersequen en l'interval temporal de $S_1$; $S_1
  \glssymbol{not:sgst:concatenatet}^r S_2 = S_1[t_1,t_2]^r \cup
  S_2[-\infty,t_1]^r \cup S_2[t_2,+\infty]^r$ on $t_1=T(\inf S_1)$ i
  $t_2=T(\sup S_1)$.
\end{definition}

Propietats de la concatenació temporal:
\begin{itemize}
\item No commutativa
\end{itemize}


% Exemples $S_1=\{(2,1),(4,1),(6,1)\}$ i $S_2=\{(1,2),(2,2)\}$,
% $S_1 \glssymbol{not:sgst:concatenatet}^\gls{zohe} S_2 = \{(1,2),(2,2),(4,1),(6,1),(+\infty,\infty)\}$
% $S_2 \glssymbol{not:sgst:concatenatet}^\gls{zohe} S_1 = \{(1,1),(2,3),(4,1),(6,1),(+\infty,\infty)\}$


\subsubsection{Junció temporal}

La junció temporal de dues sèries temporals és la junció que té en
compte la representació de les sèries temporals. És a dir, la junció
temporal de dues sèries temporals ajunta parelles de mesures
seleccionant el mateix atribut de temps en ambdues sèries temporals.


\begin{definition}[junció temporal]\label{def:sgst:joint}
  Sigui $S_1=\{m_0^1, \dotsc, m_{k_1}^1\}$ i $S_2=\{m_0^2, \dotsc,
  m_{k_2}^2\}$ dues sèries temporals en forma canònica i $r$ un mètode
  de representació, la junció temporal de les dues sèries temporals
  $S_1 \glssymboldef{not:sgst:joint}^r S_2$, és una sèrie temporal
  multivaluada que ajunta les mesures seleccionant els mateixos temps
  a cada sèrie temporal segons el mètode de representació; $S_1
  \glssymbol{not:sgst:joint}^r S_2 = \{\forall t' \in T': (t',v_1,v_2)
  | (t',v_1) \in S_1[t']^r \wedge (t',v_2) \in S_2[t']^r \}$ on $T' =
  \glssymbol{not:sgst:project}_{t}(S_1) \cup
  \glssymbol{not:sgst:project}_{t}(S_2)$.
\end{definition}


Propietats de la junció temporal:
\begin{itemize}
\item El cardinal resultant és $|S'| \leq |S_1| + |S_2|$
\item És commutativa; tenint en compte que els atributs tenen nom i
  per tant l'ordre no importa.
\end{itemize}



També es defineix l'operació de semijunció temporal que és una junció
no commutativa on la primera sèrie temporal marca el vector de temps
de junció.

\begin{definition}[semijunció temporal]
  Sigui $S_1=\{m_0^1, \dotsc, m_{k_1}^1\}$ i $S_2=\{m_0^2, \dotsc,
  m_{k_2}^2\}$ dues sèries temporals en forma canònica i $r$ un mètode
  de representació, la semijunció temporal de les dues sèries
  temporals $S_1 \glssymboldef{not:sgst:semijoint}^r S_2$, és una
  sèrie temporal multivaluada que ajunta les mesures de la primera
  sèrie temporal a les mesures de la segona segons el mètode de
  representació; $S_1 \glssymboldef{not:sgst:semijoint}^r S_2 = S_1
  \glssymboldef{not:sgst:joint}^r S_2[\glssymbol{not:sgst:project}_{t}(S_1)]^r$.
\end{definition}


Propietats de la semijunció temporal:
\begin{itemize}
\item El cardinal resultant és $|S'| = |S_1|$.
\item No és commutativa.
\end{itemize}








%%% Local Variables:
%%% TeX-master: "main"
%%% End:







% LocalWords:  SGST

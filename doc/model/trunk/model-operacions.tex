\chapter{Model d'operacions}



\section{Model de dades SGST}


\subsection{Sèrie temporal}



Per tal que l'operació d'unió de conjunts sigui vàlida per les sèries
temporals cal tenir en compte quan dues sèries temporals tenen mesures
en el mateix instant de temps. En cas d'utilitzar l'operació d'unió de
conjunts la sèrie temporal resultant no compliria amb la definició
\ref{def:serie_temporal} ja que contindria mesures amb temps
repetits. Com a conseqüència, es defineix l'operació d'unió per les
sèries temporals.

\begin{definition}[unió]
  Sigui $S_1=\{m_0^1, \dotsc, m_{k_1}^1\}$ i $S_2=\{m_0^2, \dotsc,
  m_{k_2}^2\}$ dues sèries temporals, la unió de les dues sèries
  temporals $S_1 \cup S_2$ és una sèrie temporal $S=\{m_0, \dotsc,
  m_k\}$ que conté totes les mesures de $S_1$ i les mesures de $S_2$
  no repetides: $S_1 \cup S_2 = \{ m \in S_1 \} \cup \{ m^2 =
  (v^2,t^2) \in S_2 | \forall m^1 = (v^1,t^1) : t_1 \neq t_2
  \}$. Segons aquesta definició sempre es compleix que $k \leq k_1 +
  k_2$.

Nota: la unió de sèries temporals no és commutativa. Per a
  poder unir dues sèries temporals amb mesures multivaluades cal que
  totes siguin de la mateixa dimensió.
\end{definition}






\section{Model de dades MSGST}



\subsection{Buffer}


Abans de consolidar, però, cal que la sèrie temporal contingui mesures. L'operació \emph{afegeix} permet afegir una mesura a un buffer.

\begin{definition}
  L'operació \emph{afegeix} afegeix una mesura a la sèrie temporal del buffer:
  \[
  \text{afegeix}: \text{Buffer} \times \text{Mesura} \longrightarrow \text{Buffer}
  \]
  \[
   B \times m \longrightarrow B'= B \cup \{m\}
   \]
\end{definition}

Cada cop que s'afegeix una mesura a un buffer es pot comprovar si el buffer ja és consolidable mitjançant un predicat que ens retorna un booleà: cert o fals. 

\begin{definition}
  Un buffer és consolidable quan el temps d'una mesura de la sèrie temporal és més gran que el proper instant de temps de consolidació:
  \[
  \text{consolidable?}: \text{Buffer} \longrightarrow \text{Booleà}
  \]
  Sigui $B=(S,\tau,\delta,f)$ un buffer i $m=\max(S)$ la mesura màxima, $B$ és consolidable si i només si $T(m) \geq \tau+\delta$
\end{definition}


\subsection{Interpolació}
\label{sec:model:interpolador}

Sigui $S$ una sèrie temporal i $T_0$ i $T_f$ dos instants de temps, un interpolador $f$ calcula la mesura que resumeix a $S$ en un interval de temps $i=[T_0,T_f]$. 
\[
f: \text{Sèrie temporal} \times \text{interval de temps} \longrightarrow \text{Mesura}
\]

Hi poden haver diferents interpoladors depenent de com es vol resumir la sèrie temporal. A més, la representació de la sèrie temporal influeix en la manera d'interpolar. A continuació es defineixen alguns interpoladors per a sèries temporals representades amb  \emph{zero-order hold} cap enrere, a on l'interval de temps d'interpolació $i$ s'interpreta continu per l'esquerra $(T_0,T_f]$. 


\begin{definition}[Interpolador mitjana aritmètica]
  Sigui $S=\{m_0,\ldots,m_k\}$ una sèrie temporal, $S(t)$ la
  representació de la sèrie temporal amb \emph{zero-order hold} cap
  enrere i $i=[T_0,T_f]$ un interval de temps, l'interpolador mitjana
  aritmètica (MA) resumeix $S(t)$ amb una mesura que és la mitjana dels
  valors de les mesures al conjunt $S(T_0,T_f]$.
\[
MA: \text{Sèrie temporal} \times \text{interval de temps}
\longrightarrow \text{Mesura}
\]
\[
S=\{m_0,\ldots,m_k\} \times i=[T_0,T_f] \longrightarrow m'=(v',T_f)
\]
\[
S'=S(T_0:T_f]:
v' = \frac{1}{|S'|} \sum\limits_{\forall m\in S'} V(m)
\]
\end{definition}

\begin{definition}[Interpolador màxim]
  Sigui $S=\{m_0,\ldots,m_k\}$ una sèrie temporal, $S(t)$ la
  representació de la sèrie temporal amb \emph{zero-order hold} cap
  enrere i $i=(T_0,T_f]$ un interval de temps, l'interpolador màxim
  (MAX) resumeix $S(t)$ amb una mesura que és el màxim dels valors
  de les mesures al conjunt $S(T_0,T_f]$.
\[
MAX: \text{Sèrie temporal} \times \text{interval de temps} \longrightarrow \text{Mesura}
\]
\[
S=\{m_0,\ldots,m_k\} \times i=[T_0,T_f]  \longrightarrow m'=(v',T_f)
\]
\[
S'=S(T_0:T_f]:
v' = \max_{\forall m \in S'}(V(m))
\]
\end{definition}

De manera dual es pot definir l'interpolador mínim (MIN) com el
que resumeix $S(t)$ amb una mesura que és el mínim dels valors de les
mesures al conjunt $S(T_0,T_f]$.


\begin{definition}[Interpolador últim]
  Sigui $S=\{m_0,\ldots,m_k\}$ una sèrie temporal, $S(t)$ la
  representació de la sèrie temporal amb \emph{zero-order hold} cap
  enrere i $i=(T_0,T_f]$ un interval de temps, l'interpolador últim
  (LAST) resumeix $S(t)$ amb una mesura que és la mesura màxima del
  conjunt $S(T_0,T_f]$.
\[
LAST: \text{Sèrie temporal} \times \text{interval de temps} \longrightarrow \text{Mesura}
\]
\[
S=\{m_0,\ldots,m_k\} \times i=[T_0,T_f]  \longrightarrow m'=(\max(S(T_0:T_f]),T_f)
\]
\end{definition}



\begin{definition}[Interpolador àrea]
  Sigui $S=\{m_0,\ldots,m_k\}$ una sèrie temporal, $S(t)$ la
  representació de la sèrie temporal amb \emph{zero-order hold} cap
  enrere i $i=(T_0,T_f]$ un interval de temps, l'interpolador àrea
  (AREA) resumeix $S(t)$ amb una mesura que conserva l'àrea sota la
  corba de la representació \emph{zero-order hold} cap enrere del
  conjunt $S(T_0,T_f]$.
\[
AREA: \text{Sèrie temporal} \times \text{interval de temps} \longrightarrow \text{Mesura}
\]
\[
S=\{m_0,\ldots,m_k\} \times i=[T_0,T_f]  \longrightarrow m'=(v',T_f)
\]
\[
v' = 
\frac{\int_{T_0}^{T_f} S(t) dt}{T_f - T_0}
\]
per $S(t)$ estar definida a trossos, $v'$ es pot expressar com
\[
o=\min(S(T_0:T_f]),
S'= S(T_0:T_f] - \{o\},
n=\seg\limits_S \max(S'): %equivalent a n=\min(S(T_f,\infty)) quan existeix S(T_f,\infty)
\]
\[
:v'= \frac{(T(o)-T_0)V(o) 
+( T_f- T(\ant\limits_S n) )V(n) 
+\sum\limits_{\forall m \in S'}( T(m)- T(\ant\limits_S m) )V(m)}
{T_f-T_0} 
\]

Nota: s'aplica la definició $0 \times \infty = 0$ tal com es fa habitualment a la teoria de mesura, \cite{wiki:extendedreal}.
\end{definition}

%Compte amb AREA S [t<T_0,T_0] que no està definida en la definició discreta

%Quan una sèrie temporal és regular, l'intepolador mitjana aritmètica i l'interpolador àrea valen el mateix en l'interval $(T_o,n\delta]$.






\subsection{Consolidació}

Quan un buffer és consolidable, es pot calcular una mesura de consolidació de la sèrie temporal per cada interval de temps consolidable. De manera simplificada, a cada consolidació només es té en compte l'interval que comença al darrer temps de consolidació del buffer. 

Sigui $B=(S,\tau,\delta,f)$ un buffer consolidable, la mesura de consolidació de $B$ en l'interval de temps $i=[\tau,\tau+\delta]$ és $m'=(v,\tau+\delta)$ on $m'=f(S,i)$ i $f$ és un interpolador. L'operació \emph{consolida} permet consolidar la sèrie temporal del buffer calculant-ne la mesura de consolidació.

\begin{definition}
  L'operació \emph{consolida} calcula la mesura de consolidació i treu
  les mesures consolidades de la sèrie temporal del buffer, en
  l'interval de consolidació actual:
  \[
  \text{consolida}: \text{Buffer} \longrightarrow \text{Buffer} \times \text{Mesura}
  \]
  \[
  B=(S,\tau,\delta,f) \longrightarrow B' \times m'
  \]
  \[
  B'= (S',\tau+\delta,\delta,f)
  \]
  \[
  S' = S(\tau+\delta,\infty)
  \]
  \[
  m' = f(S,[\tau,\tau+\delta]): f \text{ és un interpolador}
  \]
\end{definition}






\subsection{Disc}

L'operació \emph{afegeix} permet afegir una mesura a un disc, controlant-ne el cardinal màxim.

\begin{definition}
  L'operació \emph{afegeix} afegeix una mesura a la sèrie temporal del disc:
  \[
  \text{afegeix}: \text{Disc} \times \text{Mesura} \longrightarrow \text{Disc}
  \]
  \[
  D=(S,k) \times m \longrightarrow D'= (S',k)
  \]
  \[
  S' =  
  \begin{cases}
      S\cup\{m\} &\text{si }  |S|<k\\
      (S-\{\min(S)\}) \cup \{m\} 
    \end{cases}  \
  \]
\end{definition}





\subsection{Disc resolució}


Per altra banda, les operacions dels buffers i dels discs estan relacionades amb les operacions dels discs Round Robin. 

L'operació \emph{afegeix} permet afegir una mesura a un disc Round Robin.

\begin{definition}
  L'operació \emph{afegeix} afegeix una mesura al buffer del disc Round Robin:
  \[
  \text{afegir}: \text{Disc Round Robin} \times \text{Mesura} \longrightarrow \text{Disc Round Robin}
  \]
  \[
  R=(B,D) \times m \longrightarrow R'= (B',D)
  \]
  \[
  B'= B \text{ afegeix } m
  \]
\end{definition}

Cada cop que s'afegeix una mesura a un disc Round Robin es pot comprovar si ja és consolidable. 

\begin{definition}
  Un disc Round Robin és consolidable quan el seu buffer és consolidable:
  \[
  \text{consolidable?}: \text{Disc Round Robin} \longrightarrow \text{Booleà}
  \]
  Sigui $R=(B,D)$ un disc Round Robin, $R$ és consolidable si i només
  si $B$ és consolidable.
\end{definition}


Quan un disc Round Robin és consolidable, es pot consolidar amb l'operació \emph{consolida}. 

\begin{definition}
  L'operació \emph{consolida} calcula una  mesura de consolidació del buffer, en
  l'interval de consolidació actual, i la desa al disc. 
  \[
  \text{consolida}: \text{Disc Round Robin} \longrightarrow \text{Disc Round Robin}
  \]
  \[
  R=(B,D) \longrightarrow R'= (B',D')
  \]
  \[
  B' \times m'= \text{ consolida } B 
  \]
  \[
  D'= D \text{ afegeix } m'
  \]
\end{definition}





\subsection{Base de dades multiresolució}



With reference to the operators, the add and consolidate in a multiresolution database are applied to every resolution disc it contains.


\begin{definition}
  Operator \emph{add} adds a measure to every resolution disc:
  \[
  \text{add}: \text{multiresolution database} \times \text{Measure}
  \longrightarrow \text{multiresolution database}
  \]
  \[
  M=\{R_0,\dotsc,R_d\} \times m \mapsto M' 
  \]
  \[  
  M'= \{ \forall R_i\in M: R_i \text{ add } m \}
  \]
\end{definition}


\begin{definition}
  Operator \emph{consolidate} consolidates the resolution discs that
  are ready to consolidate.
  \[
  \text{consolidate}: \text{multiresolution database} \longrightarrow
  \text{multiresolution database}
  \]
  \[
  M=\{R_0,\dotsc,R_d\} \mapsto M'
  \]
  \[
  M'= \big\{
  \forall R_i\in M: 
  \begin{cases}
    \text{ consolidate } R_i & \text{if } R_i \text{ ready to consolidate} \\
    R_i & \text{else }
  \end{cases}\big\}
  \]
\end{definition}






%%% Local Variables:
%%% TeX-master: "main"
%%% End:
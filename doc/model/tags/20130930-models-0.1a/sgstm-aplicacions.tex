\chapter{Usos de SGSTM}



\section{Estructures interessants}



\subsection{Discos enllaçats}



\paragraph{Exemple 2}

Les taules es poden veure a la \autoref{fig:model:mtsdb:cadena} a on la base de dades multiresolució és la vista-relació 
\begin{verbatim}
M_2 = ( ((M_2' RENAME S'_B AS S') JOIN (M^{series}_2 RENAME S AS S_B)) RENAME S'_D AS S') JOIN (M^{series}_2 RENAME S AS S_D)
\end{verbatim}



\begin{figure}[tp]
  \centering
  \begin{tabular}{|c|c|c|c|c|c|}
    \multicolumn{2}{c}{$M'_2$} \\ \hline
    $S'_B$  & $S'_D$ & $\tau$ & $\delta$ & $k$ & $f$ \\ \hline
    $S_{B1}$ & $S_{D1}$ & 45 & 5  & 2 & mitjana  \\
    $S_{D1}$ & $S_{D2}$ & 40 & 10 & 4 & mitjana  \\ \hline
  \end{tabular}\qquad
  \begin{tabular}{|c|c|c|}
    \multicolumn{3}{c}{$M^{series}_{2}$} \\ \hline
    \multirow{2}{*}{$S'$}  &  \multicolumn{2}{c|}{$S$} \\ \cline{2-3}
    & $t$      & $v$  \\ \hline
    \multirow{3}{*}{$S_{B1}$} & 46 & 0 \\ 
    & 48 & 0 \\ 
    & 49 & 0 \\ \hline
    \multirow{2}{*}{$S_{D1}$} & 40 & 0 \\ 
    & 45 & 0 \\ \hline
    \multirow{4}{*}{$S_{D2}$} & 10 & 0 \\ 
    & 20 & 0 \\ 
    & 30 & 0 \\ 
    & 40 & 0 \\ \hline
  \end{tabular}
  \caption{Taula d'una mtsdb en cadena}
  \label{fig:model:mtsdb:cadena}
\end{figure}

\todo{per a fer l'exemple falta conèixer els operadors estructurals}

\todo{falta definir qui són els buffer d'entrada de mesures}
definir una $M^{in}_2$.





Respecte a l'estructura general, l'estructura enllaçada restringeix
els períodes de consolidació de les sèries temporals: aquests són
múltiples dels discs anteriors.



\subsection{Data stream}



Base de dades multiresolució a on les sèries temporals dels buffers
només tenen una mesura; és a dir tenen cardinal afitat a 1.


Per a orientar a streams els buffers s'han de canviar els operadors
d'afegir i consolidar:

Es canvia l'operador d'afegir per tal que incorpori el càlcul orientat
a stream cada cop:
\[
\text{addB}^{\text{stream}}: B \times m \longrightarrow B' =
(streamB(S,m),\tau,\delta,f)
\]

Es canvia l'operador de consolidar per tal que reconegui la sèrie
temporal del buffer com a consolidada amb stream.

  \[
  \text{consolidaB}^{\text{stream}}: B \longrightarrow B' \times m'
  \]
  \[
  B'= (S',\tau+\delta,\delta,f)
  \]
  \[
  S' = S(\tau+\delta,\infty)
  \]
  \[
  m' \in S(\tau,\tau+\delta] 
  \]


Per a orientar a streams els buffers es defineix un nou operador
\[
\text{streamB}: S \times m \longrightarrow S' = \{f^{\text{stream}}(m_o,m)\}
\]
\[
m_o \in S
\]
\[
f^{\text{stream}} \text{ és un agregador d'atributs orientat a streams}
\]
 

Aleshores els agregadors d'atributs funcionen orientats a stream;
nota: no tots els agregadors d'atributs es poden definir com a
streams.


Per exemple l'interpolador mitjana orientat a stream:

\[
\text{mitjana}^{\text{stream}}: m_o \times m_n \longrightarrow m' = (T(m_n),v')
\]
\[
\text{a on } v' = (V(m_0) + V^1(m_n), V^2(m_n) + 1 )
\]


\subsection{Compartició de buffers}


Les diferents $f$ amb mateix $\delta$ poden compartir buffer.





\subsection{Arquitectura RRDtool}


RRDtool té una estructura multiresolució amb un buffer únic d'entrada
i buffers orientats a stream; segons es va avaluar anteriorment per
\textcite{llusa11:tfm}.


\subsection{Push i pull}

Estudiar Push o pull aplicada als SGSTM i quines implicacions pot tenir.


\subsection{Què fer sense coneixements a priori}

Una base de dades multiresolució requereix tenir un coneixement de l'entorn a priori per a poder establir-ne l'esquema de multiresolució. Perquè un cop establit aquest esquema només s'emmagatzemen els atributs seleccionats i es perd informació sobre la sèrie temporal original.

Per això si la informació és crítica una bona estructura de base de dades consistiria en un magatzem total de la informació recollida i un magatzem multiresolució amb un esquema inicial. La base de dades multiresolució s'utilitzaria per a les consultes habituals que s'haguessin de resoldre de forma ràpida, en cas que les respostes no fossin suficients es podria anar a buscar la informació al magatzem total, a on la resolució de la consulta tindria un temps més elevat. 
Aleshores si aquestes consultes esdevinguessin habituals es podria definir un nou esquema de multiresolució i iniciar-lo amb les dades del magatzem total (això tardaria un cert temps) per a després executar-hi les consultes de forma ràpida.


Tot i així cal notar que en moltes aplicacions les dades històriques
són prescindibles i es pot canviar l'esquema de multiresolució sense
gaires preocupacions. Per exemple un sistema de monitoratge de la
bateria que tenim disponible al portàtil.

També en altres aplicacions el que volem es resoldre una consulta del tipus la mitjana puja o baixa. En això el model de multiresolució hi encaixa molt bé ja que es base en calcular agregacions i després treballar sobre aquestes. 









\section{Operacions habituals en les sèries temporals}


\paragraph{Semblança de dues sèries temporals}


Similarity Measures for Time Series

Hi ha varis mètodes, [keogh08:vldb] n'avalua uns quants i els generalitza amb:

Given two
time series T1 and T2 , a similarity function Dist calcu-
lates the distance between the two time series, denoted by
Dist(T1 , T2 ).

Exemplifiquem amb la distància euclídia, [keogh08:vldb] nota que és
competitiva amb les altres.

Distancia euclídia segons [faloutsous94-sigmod]


\[
D(S,Q) = \left( \sum_{i=1}^{l} (S[i]-Q[i])^2  \right)^{1/2}
\]

\begin{gather*}
  D: S \times Q \longrightarrow v: \\
  S' = map(fusio(S,Q),(t,v_1,v_2)\mapsto(t,(v_1-v_2)^2)), \\
  S'' = fold(quad,(0,0),(t^1,v^1,t^2,v^2)\mapsto(t^1,v^1+v^2)), \\
  v = \sqrt{V(m)}:m\in S''
\end{gather*}


S i Q haurien de ser regulars entre elles, sinó cal aplicar una fusió amb representació/interpretació.

Amb la multiresolució la fusió es pot fer de forma eficient. Per altra banda, es podria crear un disc resolució amb agregador de semblança.


\paragraph{Semblança de dues sèries temporals amb offset}

Aquí es descriu la solució general del problema (SequentialScan),
[faloutsous94-sigmod] n'estudia implementacions amb certes
heurístiques que aconsegueixen més eficiència.





\paragraph{Filtratge senzill per mitjana mòbil}

Sigui $p$ la mida de la finestra mòbil
\begin{gather*}
  \text{MitMobil}: S \times \text{p} \longrightarrow S':\\
  \text{map}(S,(t,v)\mapsto \text{mitjanaV}(S[t,t+p]))
\end{gather*}


Mitjana mòbil sobre la multiresolució



\paragraph{Farciment de forats}

Jo tinc una sèrie temporal i vull que entre dues mesures no hi hagi més d'un cert temps. Si no es compleix dic que té forats. 

Sigui $S$ una sèrie temporal, aquesta té forats de més durada que $d$
si alguna mesura compleix $\text{forats}(S,d) = \text{selecciona}(difT(S),v>d \bigwedge v\neq\infty)$ a on $difT(S) = \text{map}(\text{tpredecessors}(S),(t,v)\mapsto(t,t-v))$.

Amb la multiresolució el farciment de forats és natural a l'estructura i és controlat per la funció agregadora d'atributs.


* Com farciria els forats manualment a una sèrie temporal?

1. Passar-ho per un esquema de multiresolució

2. Treballar sobre la sèrie temporal:

a partir del càlcul de forats anterior $\text{forats}(S,d)$ per
exemple apliquem un farciment amb representació
zohe. $\text{farciment}(S,d) = \text{unio}(S,S')$ a on fem la selecció
de resolució $S' = S[T]^{\text{zohe}}$, $\forall (t,v) \in
\text{forats}(S,d): T = \{ \tau = t - dn |
\tau\in(t-v,t),n\in\mathbb{N} \}$.







\subsection{Com treure profit de les operacions dels SGSTM}

Temes que després es poden aprofitar a les implementacions

* No hi ha updates --> les sèries temporals no s'han de canviar

* Per exemple, vull calcular la mitjana de  BDSTM(a,b] si tinc un disc resolució amb $\delta=b-a$ i $f=$mitjana aquest seria l'adequat en comptes de calcular mitjana(SerieTotal(M)(a,b])

%??
% No obstant, la base de dades multiresolució conté informació sobre la
% resolució de les subsèries i per tant aquesta operació és susceptible
% d'implementar-se aprofitant aquesta informació.  A tall d'exemple es
% defineix una operació per extreure de la base de dades multiresolució
% una sèrie temporal regular amb període $T$:


% \begin{definition}[Selecció de resolució regular]
%   \begin{gather*}
%     \text{ResolucióRegular}: M^* \times T \times r \longrightarrow S'\\
%     \forall (S_{Bi},S_{Di},\delta_i,\tau_i,k_i,f_i) \in M : \\
%     d_i = T - \delta_i , \\
%     0 \geq d_0 > d_1 \dots > d_a, 0 < d_{a+1} < \dots < d_d: \\
%     S'' = S_{D0} || S_{D1} || \dotsb || S_{Da}  ||  S_{Da+1} || \dotsb || S_{Dd}, \\
%     S' = S''[i]^r: i = {t|0+nT,n\in\mathbb{N}}
%   \end{gather*}
% \end{definition}

% Nota: les operacions no són equivalents, l'operació $\text{SerieTotal}(M)[i]^r$ és molt més potent que la $\text{ResolucióRegular}(M,T)$.




\subsection{Comparació d'operacions dels SGSTM amb les dels SGST}

Tinc una sèries temporal $S$ i l'emmagatzemo a una base de dades multiresolució $M$ amb atributs de mitjana. 

* mitjana(S) = mitjana(serieTotal(M)) ?

* Operació O, afegeix(M',O(S)) = O(serieTotal(M))?


\begin{align*}
s   \qquad   &  s'=ST(M(s))\\
r=O(s) \qquad& r'=O(s') \\
\epsilon(r,r')?
\end{align*}

on $O$ és una consulta, per exemple pot ser O=Creix la sèrie temporal? Si la resposta és Sí en els dos casos, aleshores no hi ha error.
















\section{Resum}

Aquest capítol s'acaba amb un resum dels conceptes exposats en el
model de dades. Una base de dades per sèries temporals multiresolució
és un sistema informàtic d'emmagatzematge d'una sèrie temporal entesa
com una una co\l.lecció de dades mesurades en diferents instants de
temps.

A la base de dades, la sèrie temporal queda estructurada com s'ha esquematitzat a  la figura~\ref{fig:model:bdstm}. És una forma compacta d'emmagatzemar la sèrie temporal de manera que queda repartida segons diferents funcions d'interpolació i períodes de mostreig. Aquest repartiment té lloc en els diferents discs resolució, els quals fan ús del seu buffer per interpolar les mesures i fan ús del seu disc per consolidar-les. 

El conjunt de discs resolució constitueixen la part principal d'una base de dades multiresolució tot i que hi pot haver variacions en aquest esquema, com per exemple un buffer d'entrada de mesures comú que regularitzi la sèrie temporal des d'un principi i simplifiqui els interpoladors que són complicats quan es fa el pas de sèrie temporal no regular a regular.


En el capítol \todo{? més endavant}
 utilitzant el llenguatge de programació Python es dissenya, a nivell acadèmic, un sistema de gestió de bases de dades que implementa el model de dades tal com s'ha definit en aquest capítol.


En resum, a partir del model de dades multiresolució descrit en aquest capítol per una banda es poden estudiar quin efecte té una configuració determinada de paràmetres i per altra banda es poden dissenyar sistemes de gestió de bases de dades assegurant que implementen el model i per tant que tenen el funcionament desitjat.






%%% Local Variables:
%%% TeX-master: "main"
%%% End:
% LocalWords:  SGSTM

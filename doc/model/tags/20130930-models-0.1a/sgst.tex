
\chapter{Model SGST}
\label{cap:model:sgst}


En aquest capítol es defineix un model per als sistemes de gestió de
bases de dades per a sèries temporals (SGST). Aquest model
s'estructura en base a dos objectes principals, mesures i sèries temporals.
Ambdós tenen un atribut de temps, el qual requereix un tractament
adequat. El model de SGST es dissenya en dues parts.

\begin{itemize}
\item Primer, es defineix el model d'estructura de les dades, és a
  dir, la forma com es descriuen les mesures i les sèries temporals.  
\item Segon, es defineix el model d'operacions sobre les dades, és a
  dir, els operadors bàsics que permeten modelar el comportament i la
  manipulació de les sèries temporals.
\end{itemize}

%Les sèries temporals tenen una naturalesa específica, deguda a l'atribut temps, que permet tractar-les com a funció contínua. Aquesta tipologia s'expressa més clarament al final del capítol.




\section{Model estructural de dades}

Una sèrie temporal és una relació de temps i valors. A cada parella
temps-valor l'anomenem mesura. Així doncs, una sèrie temporal és un
conjunt de mesures. 


En el nostre cas concret, una mesura es correspon amb un valor mesurat
en un instant de temps i una sèrie temporal esdevé una co\l.lecció de
mesures.





\subsection{Temps}
\label{sec:sgst:temps}

El temps és la variable que ens permet ordenar les mesures.  A tal
efecte, si denominem $T$ el domini del temps, els seus membres
presenten una relació d'ordre total. $T$ pot ser tant un conjunt finit
com infinit i normalment serà un conjunt tancat %(compactificat?)
per a poder incloure les mesures indefinides (v.\
\autoref{def:model:mesura_indefinida}) com a límits.

Per tal de facilitar la comprensió, en el document utilitzarem el
conjunt de reals com a conjunt pels temps. Representarem el conjunt
$T$ amb el conjunt estès de nombres reals $\bar{\R{}} \in
\R{} \cup
\{+\infty,-\infty\}$, \parencite{wiki:extendedreal,cantrell:extendedreal},
també anomenat recta real acabada, el qual és un conjunt tancat.

El conjunt estès de nombres reals té dos punts límits corresponents al
valor impropi infinit, aleshores en notació d'interval el conjunt $T$
es pot escriure com $\bar{\R{}} \in [-\infty,+\infty]$. En referència
amb el conjunt dels nombres reals $\R{}$, les relacions d'ordre i
algunes operacions aritmètiques s'estenen al conjunt $\bar{\R{}}$,
\cite{cantrell:extendedreal}.  Algunes expressions esdevenen
indefinides (p.ex.\ $0/0$) i altres depenen del context, com és el cas
de l'expressió indeterminada $0 \times \infty$ que per exemple en la
teoria de la mesura habitualment es defineix com $0 \times \infty =
0$, \cite{wiki:extendedreal}.


El conjunt dels reals és un espai mètric ja que té definida una funció
distància (o mètrica), com per exemple la distància euclidiana. Com a
conseqüència, ens permet distingir entre instants de temps (els
elements del conjunt) i durades (la mètrica). Observant els instants
de temps com a punts en la recta real, les durades com a segments de
la recta real i especificant un instant de temps com a marc de
referència, es pot definir el temps com a sistema de
coordenades \parencite{iep:time-supplement,wiki:coordinate}. A
continuació definim el temps de manera que puguem ordenar
esdeveniments, mesurar durades d'esdeveniments i establir quan
esdevenen; és una aproximació ingènua sense abastar detalls complicats
del concepte temps \parencite{iep:time}.

\begin{definition}[Temps]
  \label{def:model:temps}
  Siguin $t^i_i$ i $t^i_j$ dos instants de temps amb el mateix $t^R$
  com a marc de referència, definim la quantitat de temps o la durada
  $t^d$ com un valor $t^d \in\bar{\R{}}$ que mesura la distància en
  unitats de temps entre dos instants de temps $t^d = d(t^i_i,t^i_j)$
  a on $d$ és la mètrica del conjunt $T$. En el cas que els instants
  de temps es defineixin com a reals, $t^i_i , t^i_j \in \bar{\R{}}$,
  aleshores $t^d = t^i_i - t^i_j$.

  Sigui $T$ el domini del temps, definim un instant de temps $I$ com
  un element del conjunt $I \in T$. Així, un instant de temps és
  l'etiqueta d'un punt en la línia temporal. Seguint la definició de
  sistema de coordenades i prenent els nombres reals com a domini del
  temps, sigui $t^{R}$ un instant de temps marc de referència,
  aleshores els instants de temps es defineixen com un valor $t^i
  \in\bar{\R{}}$ que indica la distància de temps amb signe respecte a
  l'instant de temps de referència $t^i= d(t^{R},I)$ a on $d$ és la
  mètrica del conjunt $T$.
\end{definition}

En resum, els instants de temps es poden veure com una seqüència de
valors reals que indiquen esdeveniments amb ordre clarament definit i
entre dos instants de temps sempre hi ha una durada. Expressarem tant
els instants de temps com les durades amb un real que té unitats de
temps. Aquestes unitats són 'segons' en sistema internacional.

% El marc de referència és un instant de temps que permet definir
% unívocament la posició de qualsevol altre instant de temps en un
% sistema de coordenades.



\subsubsection{Estàndards de temps}

Els estàndards de temps especifiquen com s'ha de mesurar el pas del
temps i com s'han d'assenyalar els instants de temps.
\textcite{allen:timescales} recull diferents estàndards de temps
que existeixen, dels quals a continuació comentem els més habituals.

Actualment l'estàndard de temps habitual per mesurar el pas del temps
és el Temps Atòmic Internacional (TAI), del qual se'n deriva un altre
estàndard més conegut que és el Temps Universal Coordinat (UTC).
Ambdós estàndards assenyalen el instants de temps segons el calendari
Gregorià i el calendari julià. Actualment, de forma genèrica s'utilitza
UTC per a sincronitzar rellotges tot i que en el futur es podria
canviar per un nou estàndard, el Temps Internacional (TI), el qual
també es base en el TAI.

El calendari julià utilitza un estàndard de comptar el temps com a
nombre de dies que han passat des d'una data concreta, la qual
s'anomena època. L'època es correspon amb el concepte d'instant de
temps marc de referència de la \autoref{def:model:temps}. Per defecte
l'època se situa a l'inici del Període Julià tot i que també se solen
utilitzar altres dates assenyalades.

Un altre estàndard semblant al julià és l'Hora POSIX o Hora Unix, el
qual compta el nombre de segons des de l'1 de gener de 1970 basant-se
en el les mesures d'UTC. L'Hora Unix és l'estàndard de temps habitual
en els sistemes operatius de la família Unix. No obstant, aquest
estàndard presenta un problema d'ambigüitat a causa que no té no té en
compte els segons addicionals d'UTC.





\subsubsection{Calendari}

Un cas particular del temps és el calendari. Els calendaris són
definicions pel domini temps que consisteixen en noms per als punts de
la línia de temps i regles per establir la durada entre ells per tal
de que el temps tingui certa relació amb la rotació de la Terra. A
l'apartat anterior hem definit el domini temps de manera general
amb el conjunt de reals, els quals exemplifiquen més clarament el
concepte de sistema de coordenades de temps absolut.

\textcite{dreyer94} situen els calendaris i les seves operacions com a
essencials en els SGST. Tanmateix, pot no ser necessari modelar les
dates i regles de calendari en el model de temps. Els calendaris es
poden observar com a noms que fan referència a instants de temps
quantificables, com els de la \autoref{def:model:temps}. Aleshores,
només cal una eina que sigui capaç de convertir els noms de calendari
a instants de temps.

El fet de que un calendari sigui més o menys complicat no afecta al
model de SGST, sols té incidència en les funcions de conversió
d'instant de temps a calendari i viceversa. Tampoc afecta que els
calendaris siguin ambigus (p.ex.\ dos noms per al mateix instant o
instants sense nom) o que continguin propietats impredictibles (p.ex.\
cas dels segons addicionals en UTC) ja que aquests casos es
corresponen amb la bona definició dels sistemes de calendari.

Així doncs, els calendaris en el model de SGST es poden implementar
com una extensió del model de temps. El tipus de dades ordinal de
calendari Gregorià implementat per
\textcite[cap.~16]{date02:_tempor_data_relat_model} pot servir com a
guia per a la implementació dels calendaris en els SGST.




\subsection{Valor}
\label{sec:sgst:valor}

\todo{T: refer tota la secció}

El \gls{terme:SGBDR:valor} és qualsevol element que és d'un
\gls{terme:SGBDR:tipus}; és a dir, un objecte que
pertany a un determinat conjunt de valors i que té associat les
operacions que s'hi poden aplicar. Exemples de tipus de dades són els
enters, els reals, les cadenes de text i les estructures de dades com
vectors, llistes o \glspl{terme:SGBDR:relacio}.  

\todo{Exemplificar més amb tipus que tenen valors i operacions que se'ls poden aplicar}

%\todo{potser citar que el valor es tracta de manera semblant a l'extensió del model objecte-relacional?}

El model de dades dels valors ha d'incloure una dada que defineixi el
valor indefinit. Més endavant a la
definició~\ref{def:model:mesura_valor_indefinit} es detallen les
propietats de les mesures amb valor indefinit. Seguint l'exemple amb
els reals, el valor indefinit es defineix amb el valor impropi infinit
del conjunt dels reals estès
projectivament, \parencite{cantrell:projectivelyextendedreal},
$\R{}^*\in\R{} \cup \{\infty\}$.


En aquest exemple amb reals, el valor és un escalar però fàcilment es
pot estendre el concepte a valors multivaluats ${\R{}^*}^n$ que
representin una co\l.lecció de valors mesurats en el mateix instant de
temps, tal i com fa per exemple \textcite{assfalg08:thesis}.







\subsection{Mesura}\label{sec:model:mesura} 

Una mesura està formada per la parella de temps i valor.

\begin{definition}[Mesura]
  \label{def:model:mesura}
  Definim \emph{mesura} com el tuple $(t,v)$, en el que $v$ és el
  valor de la mesura i $t$ és l'instant de temps en que s'ha pres
  aquesta mesura.
\end{definition}


Donada una mesura $m=(t,v)$ escriurem $V(m)$ per referir-nos a $v$ i
$T(m)$ per referir-nos a $t$.

L'instant de temps de les mesures indueix una relació d'ordre entre
les mesures.
\begin{definition}[Relació d'ordre]
  \label{def:model:mesura-relacio-ordre}
  Sigui $m=(t_m,v_m)$ i $n=(t_n,v_n)$, direm que l'ordre de la mesura
  $m$ és major o igual que $n$, $m\geq n$, si i solament si $t_m\geq
  t_n$.
\end{definition}


En les definicions de temps i valor s'han estès els conjunts amb
valors impropis, concretament s'ha exemplificat amb el conjunt estès
de nombres reals afí $\bar{\R{}} \in \R{} \cup
\{+\infty,-\infty\}$ i amb el projectiu $\R{}^*\in\R{}
\cup\{\infty\}$,
\parencite{cantrell:extendedreal,cantrell:projectivelyextendedreal}. Aquesta
extensió amb l'element impropi infinit ($\infty$) dóna com a resultat
unes mesures impròpies que anomenarem mesura de valor indefinit i
mesura indefinida.

\begin{definition}[Mesura de valor indefinit]
  \label{def:model:mesura_valor_indefinit}
  Definim \emph{mesura de valor indefinit} com el tuple $(t,v)$, en el
  que el valor és $v=\infty$ i l'instant de temps és
  $t\in\bar{\R{}}$.
\end{definition}

\begin{definition}[Mesura indefinida]
  \label{def:model:mesura_indefinida}
  Definim \emph{mesura indefinida} com el tuple $(t,v)$, en el que el
  valor és $v\in\R{}^*$ i l'instant de temps és
  $t\in\{+\infty,-\infty\}$.
\end{definition}

Així doncs, donada una mesura $m$, s'anota la mesura de valor
indefinit com $m=(t,\infty)$ i les mesures indefinides com
$m=(+\infty,v)$ per la positiva i $m=(-\infty,v)$ per la negativa, les
quals normalment s'anotaran també amb valor indefinit:
$m=(+\infty,\infty)$ i $m=(-\infty,\infty)$ respectivament.


Les mesures de valor indefinit es podran utilitzar en aquells casos en
els que el valor de la mesura és desconegut. Els valors desconeguts
són aquells valors que no existeixen (es desconeixen, \emph{missing
  data} ) o que s'ignoren (es descarten, \emph{censoring} o
\emph{truncation}). Els valors que no existeixen prenen el valor
desconegut en el moment de la mesura, en canvi els valors descartats
són marcats com a desconeguts després d'un processament de les dades.

Nota: en alguns sistemes es distingeix entre valors infinits
($\infty$) i valors indefinits (NaN, \emph{not a number}),
\cite{wiki:ieee754}. Aquest no és el cas de les definicions de mesures
indefinides presents.







\subsection{Sèrie temporal}
\label{sec:model:serietemporal}

Les sèries temporals són seqüències de mesures ordenades en el temps.
Tradicionalment s'anomenen sèries temporals tot i que alguns autors les anomenen
seqüències temporals, per exemple a \cite{last:hetland}.  Les sèries
temporals són mesures del mateix fenomen i com a conseqüència el tipus
dels valors de les sèries temporals és homogeni.


\begin{definition}[Sèrie temporal]
  \label{def:serie_temporal}
  Una sèrie temporal $S$ és un conjunt de mesures
  $S=\{m_0,\ldots,m_k\}$ sense temps repetits en la qual
  $\forall i,j: i\leq k, j\leq k, i\neq j : T(m_i)\neq T(m_j)$.
\end{definition}

Per ser un conjunt, les sèries temporals tenen mesura de cardinalitat.
\begin{definition}[Cardinal]
  Sigui $S=\{m_0,\ldots,m_k\}$ una sèrie temporal, definim el nombre
  de mesures que conté la sèrie temporal com el cardinal del conjunt
  $|S|=k+1$. Una sèrie temporal sense mesures és la sèrie temporal
  buida $S_\emptyset= \emptyset = \{\}$, és a dir que no té cap element
  $|S_\emptyset|=0$.
\end{definition}



 
\subsubsection{Formes d'una sèrie temporal}

Una sèrie temporal pot tenir formes diferents segons com s'expressi.
A continuació diferenciem entre tres formes possibles d'una sèrie
temporal: canònica, multivaluada i doble.


La forma bàsica d'una sèrie temporal és la de parelles de temps i
valor a la qual anomenem forma canònica.
\begin{definition}[Forma canònica]
  Sigui $S = \{ m_0, m_1 , \dotsc, m_k \}$ una sèrie temporal,
  s'escriu com $S =  \{
  (t_0,v_0), (t_1,v_1), \dotsc, (t_k,v_k)\}$; és
  a dir com a parelles de temps i valor. A aquesta
  forma l'anomenem canònica.
\end{definition}


Les sèries temporals poden mesurar alhora més d'un fenomen quan
aquests comparteixen els instants de temps de mesura. Aquestes sèries
temporals tenen forma de sèrie temporal multivaluada.

\begin{definition}[Sèrie temporal multivaluada]
  Anomenem sèrie temporal multivaluada a una sèrie temporal que té més
  d'un atribut de valors.  Sigui $S = \{ m_0, m_1 , \dotsc, m_k \}$
  una sèrie temporal és multivaluada si cada mesura $m_i$ és un tuple
  $m_i=(t,v_1,v_2,\dotsc,v_n)$ a on $t$ és un instant de temps i
  $v_1$, $v_2$, \dots, $v_n$ són valors. 

  Una sèrie temporal multivaluada es pot escriure en forma canònica de
  parelles $(t,v)$. És a dir, la sèrie temporal multivaluada en
  forma canònica té mesures $m_i$ com a tuples
  $m_i=(t,(v_1,v_2,\dotsc,v_n))$
\end{definition}

La forma canònica s'utilitza per a generalitzar les sèries temporals
multivaluades en les operacions on el valor multivaluat no és
rellevant. En altres operacions, per exemple la selecció o la junció,
el multivalor és rellevant per treballar-hi o perquè el resultat és
una sèrie temporals multivaluada. 



Hi ha una forma no habitual de les sèries temporals que es dóna quan
tenen dos atributs de temps i a la qual anomenem forma doble.

\begin{definition}[Sèrie temporal doble]
  \label{def:sgst:st-doble}
  Anomenem sèrie temporal doble a una sèrie temporal que té dos
  atributs de temps i dos atributs de valors. Sigui $S =\{m_0, \dotsc,
  m_k\}$ una sèrie temporal és doble si cada mesura $m_i$ és un tuple
  $m_i=(t_1,v_1,t_2,v_2)$ a on $t_1$ i $t_2$ són instants de temps i
  $v_1$ i $v_2$ són valors. De la mateixa manera, a aquesta mesura
  $m_i$ l'anomenem mesura doble.  Una sèrie temporal doble no té dues
  parelles de temps repetides $|\{(t_1,t_2) | t_1,t_2\in S\}| = |S|$.
\end{definition}

La sèrie temporal doble prové d'un producte de dues sèries
temporals. Està pensada com a càlcul intermedi d'altres operacions com
per exemple la junció o el mapatge. 
% La seva forma canònica es
% correspondria de manera semblant a l'exemple de la
% \autoref{fig:model:serietemporal:serietemporal}.









\subsubsection{Relació sèrie temporal}

Una sèrie temporal s'expressa com un conjunt i com a tal és
susceptible d'aplicar-hi els conceptes del model relacional dels
SGBD. A continuació, el que s'ha descrit a l'apartat anterior es torna
a expressar seguint el concepte de relació.


Per ser un conjunt de mesures, s'observa una sèrie temporal com una
relació de grau dos a on la capçalera conté els atributs temps i
valor. Ambdós atributs tenen els dominis de temps i valor descrits a
les seccions \ref{def:model:temps} i \ref{sec:sgst:valor}, com per
exemple el tipus de dades 'reals estesos'. Les relacions de sèries
temporals inclou algunes restriccions més que les relacions:

\begin{itemize}
\item Els temps no poden estar repetits: (\emph{key restriction \{t\}})
\item L'atribut de valor ha de contenir el mateix tipus d'objecte i ha
  d'estar associat al mateix fenomen o fenòmens.
\end{itemize}

Els temps no repetits indueixen un ordre temporal a les sèries
temporals. Tot i així, les relacions, per ser conjunts, conserven la
no definció d'un ordre dels elements. En el model relacional no hi ha
ordre ni en les tuples ni en els atributs a diferència de les
relacions matemàtiques que tenen un ordre d'esquerra a
dreta \parencite[sec.\ 5.3]{date:introduction}.


%\subsubsection{Possibles representacions}

A l'apartat anterior s'han descrit diverses formes d'una sèrie
temporal.  A continuació atenem a les possibles representacions com a
relació d'una sèrie temporal segons les formes que tinguin.

% Com a relació seguint el concepte de possibles
% representacions proposat per \textcite[cap.~5]{date:introduction} ?
% i les formes normals de les relacions [Date]?



\begin{definition}[Representació canònica]
  Sigui $S = \{ m_0, m_1 , \dotsc, m_k \}$ una sèrie temporal amb
  domini $\bar{\R{}}$ pels temps i els valors de les mesures,
  representada com a relació s'escriu com $S = ( \{t: \bar{\R{}}, v:
  \bar{\R{}}\}, \{ \{t:t_0,v:v_0\}, \{t:t_1,v:v_1\}, \dotsc,
  \{t:t_k,v:v_k\} \} )$; és a dir com a parella capçalera i conjunt de
  valors certs. A aquesta representació l'anomenem forma canònica.

  Així doncs, sigui $S_{\emptyset} = \{ \}$ una sèrie temporal buida,
  modelada com a relació s'escriu com $S_{\emptyset} = ( \{t:
  \bar{\R{}}, v: \bar{\R{}}\}, \{ \} )$.
\end{definition}


Degut al format esquemàtic de les sèries temporals, en simplifiquem
l'escriptura de la forma canònica com a conjunt de tuples $(t,v)$ a on
$t$ és el temps i $v$ és el valor. Així doncs quan no hi ha dubte
sobre els dominis ni els noms d'atributs, una sèrie temporal es pot
escriure de manera simplificada com a $S = \{ (t_0,v_0), (t_1,v_1),
\dotsc, (t_k,v_k) \}$, la qual es correspon amb la forma canònica de
la sèrie temporal representada com a conjunt.

Tal com s'utilitza en les relacions, les sèries temporals es poden
visualitzar com a taules. La sèrie temporal $S$ i la $S_{\emptyset}$
es visualitzen com a taula a la
\autoref{fig:model:serietemporal:taula}.

\begin{figure}[tp]
  \centering
  \begin{tabular}[c]{|c|c|}
    \multicolumn{2}{c}{$S$} \\ \hline
    $t$  & $v$ \\ \hline
    $t_0$  & $v_0$ \\
    $t_1$  & $v_1$ \\
    $\dots$  & $\dots$ \\ 
    $t_k$  & $v_k$ \\ \hline
  \end{tabular} \qquad
  \begin{tabular}[c]{|c|c|}
    \multicolumn{2}{c}{$S_{\emptyset}$} \\ \hline
    $t$  & $v$ \\ \hline
      &  \\ \hline
  \end{tabular}
  \caption{Visualització com a taula d'una sèrie temporal}
  \label{fig:model:serietemporal:taula}
\end{figure}




\begin{definition}[Representació multivaluada]
  La representació com a relació d'una sèrie temporal multivaluada buida és
  $S_{\emptyset} = ( \{t: \bar{\R{}}, v_1: \bar{\R{}}\,
  v_2: \bar{\R{}}, \dotsc, v_n: \bar{\R{}}\}, \{ \} )$

  Una sèrie temporal multivaluada es pot escriure en forma canònica de
  parelles $(t,v)$. És a dir, la sèrie temporal multivaluada buida en
  representació canònica és $S_{\emptyset} = ( \{t: \bar{\R{}},
  v: V \}, \{ \} )$ a on el domini de l'atribut valor és de tipus
  relació $V = \{ v_1: \bar{\R{}}\, v_2: \bar{\R{}},
  \dotsc, v_n \}$ amb restricció que els valors relació que hi
  pertanyen només poden tenir un tuple $r \in V: |r| = 1$.
\end{definition}

Com ocorre en les relacions, el nom dels atributs d'una sèrie
temporal pot ser decidit per l'usuari. Per exemple una sèrie temporal
multivaluada amb tres atributs amb nom: $S_{\emptyset} = ( \{t:
\bar{\R{}}, \text{temperatura}: \bar{\R{}}\,
\text{consum}: \bar{\R{}}, \text{volum}: \bar{\R{}}\}, \{
\} )$.


Finalment, també representem una sèrie temporal doble en forma de
relació.
\begin{definition}[Representació doble]
  La representació com a relació d'una sèrie temporal doble buida és
  $S_{\emptyset} = ( \{t_1: \bar{\R{}}, v_1: \bar{\R{}}\,
  t_2: \bar{\R{}}, v_2: \bar{\R{}}\}, \{ \} )$.
\end{definition}




\subsection{Exemples}

\pgfplotsset{
    timeseriesrel/.style={
        height=4cm,
        axis x line=middle,
        axis y line=middle,
        enlarge x limits=0.2,
        enlarge y limits=0.2,
        xlabel=t,
        ylabel=v,
        xlabel style={ at={(current axis.right of origin)},anchor=mid west},
        ylabel style={ at={(current axis.above origin)},anchor=south},
        title style={at={(current axis.left of origin)},anchor=north east},
    }
}




\paragraph{Exemple 1} \emph{Valors reals}.  Sèrie temporal $S_1$ on el
temps i els valors pertanyen a $\bar{\R{}}$. Conté la mesura de
valor 1 en el temps 2, la mesura de valor 3 en el temps 2 i la mesura
de valor 1 en el temps 6. 

En la forma canònica completa s'escriu com $S_1 = ( \{t:
\bar{\R{}}, v: \bar{\R{}}\}, \{ \{t:2,v:1\}, \{t:3,v:3\},
\{t:6,v:1\} \} )$. També es pot escriure de manera simplificada com a
$S_1 = \{ (2,1), (3,3), (6,1) \}$.


La sèrie temporal $S_1$ es visualitza com a taula a la
\autoref{fig:model:serietemporal:real}, a la qual hi afegim una
visualització com diagrama de dispersió amb el temps a l'eix
horitzontal i el valor a l'eix vertical.

\begin{figure}[tp]
  \centering
  \begin{tabular}[c]{|c|c|}
    \multicolumn{2}{c}{$S_1$} \\ \hline
    $t$  & $v$ \\ \hline
    2  & 1 \\
    3  & 3 \\
    6  & 1 \\ \hline
  \end{tabular} \qquad
  \begin{tikzpicture}[baseline=(current bounding box.center)]
    \begin{axis}[
        timeseriesrel,
        title=$S_1$,
        ]
    \addplot[only marks,mark=*,blue] coordinates {
        (2,1)
        (3,3)
        (6,1)
    };
    \end{axis}
   \end{tikzpicture}
  \caption{Taula i gràfic d'una sèrie temporal amb valors reals}
  \label{fig:model:serietemporal:real}
\end{figure}


\paragraph{Exemple 2} \emph{Valors caràcters}.  Sèrie temporal $S_2$
on el temps pertany a $\bar{\R{}}$ i els valors són caràcters
que pertanyen a $C=\{a,b,\dotsc,z,\infty\}$. Conté el caràcters $a$,
$c$ i $a$ mesurats respectivament en els temps $2$, $3$ i
$6$. 

De manera simplificada s'escriu com $S_2 = \{ (2,a), (3,c), (6,a) \}$.
La sèrie temporal $S_2$ es visualitza com a taula a la
\autoref{fig:model:serietemporal:caracter}, a la qual hi afegim una
visualització com diagrama de dispersió amb el temps a l'eix
horitzontal i el valor a l'eix vertical no continu.

\begin{figure}[tp]
  \centering
  \begin{tabular}[c]{|c|c|}
    \multicolumn{2}{c}{$S_4$} \\ \hline
    $t$  & $v$ \\ \hline
    2  & a \\
    3  & c \\
    6  & a \\ \hline
  \end{tabular} \qquad
  \begin{tikzpicture}[baseline=(current bounding box.center)]
    \begin{axis}[
        timeseriesrel,
        title=$S_4$,
        yticklabels={0,0,a,b,c},
        ]
    \addplot[only marks,mark=*,blue] coordinates {
        (2,1)
        (3,3)
        (6,1)
    };
    \end{axis}
   \end{tikzpicture}
  \caption{Taula i gràfic d'una sèrie temporal amb valors caràcters}
  \label{fig:model:serietemporal:caracter}
\end{figure}




\paragraph{Exemple 3} \emph{Sèrie temporal multivaluada}.  Sèrie
temporal $S_3$ on el temps pertany a $\bar{\R{}}$ i hi ha tres
valors on cadascun pertany a $\bar{\R{}}$. En els temps $2$, $3$
i $6$ s'ha mesurat a) un atribut \emph{temp} amb valors $1$, $2$ i
$1$; b) un atribut \emph{cons} amb valors $2$, $1$ i $2$; i c) un
atribut \emph{vol} amb valors $3$, $0$ i $3$.



En la forma multivaluada s'escriu com %
$S_3 = ( \{t: \bar{\R{}}, \text{ temp}: \bar{\R{}}, \text{
  cons}: \bar{\R{}},\text{ vol}: \bar{\R{}} \}, %
\{%
\{t:2,\text{ temp}:1 , \text{ cons}:2,\text{ vol}:3 \}, %
\{t:3,\text{ temp}:2 , \text{ cons}:1,\text{ vol}:0 \}, %
\{t:6,\text{ temp}:1 , \text{ cons}:2,\text{ vol}:3 \} %
\} )$. També es pot escriure de manera simplificada com a $S_{3} = (
(t,\text{ temp},\text{ cons},\text{ vol}),\{ (2,1,2,3), (3,2,1,0),
(6,1,2,3) \})$.

La forma canònica és una sèrie temporal amb tuples $(t,v)$, és a dir
\begin{align*}
  S^C_{3} &= ( \{t: \bar{\R{}}, v: \{ \text{ temps}:
  \bar{\R{}}, \text{ cons}: \bar{\R{}},\text{ vol}:
  \bar{\R{}},\} \}, \{ \\
  & \{t:2, v: ( \{ \text{ temp}: \bar{\R{}}, \text{ cons}:
  \bar{\R{}},\text{ vol}: \bar{\R{}},\}, \{ \text{ temp}:1
  ,  \text{ cons}:2,\text{ vol}:3 \}\} ), \\
 & \{t:3, v: ( \{ \text{ temp}: \bar{\R{}}, \text{ cons}:
  \bar{\R{}},\text{ vol}: \bar{\R{}},\}, \{ \text{ temp}:2
  ,  \text{ cons}:1,\text{ vol}:0 \}\} ), \\
 & \{t:6, v: ( \{ \text{ temp}: \bar{\R{}}, \text{ cons}:
  \bar{\R{}},\text{ vol}: \bar{\R{}},\}, \{ \text{ temp}:1
  ,  \text{ cons}:2,\text{ vol}:3 \}\} ) \\
  \} )
\end{align*}


La sèrie temporal $S_3$ i la seva forma canònica es visualitzen com a
taula a la \autoref{fig:model:serietemporal:caracter}, a la qual hi
afegim una visualització com diagrama de dispersió amb el temps a
l'eix horitzontal i els valor a l'eix vertical cadascun amb color
diferent.


\begin{figure}[tp]
  \centering
  \begin{tabular}[tp]{|c|c|c|c|}
   \multicolumn{4}{c}{$S_3$} \\ \hline
    $t$  & temp & cons & vol \\ \hline
    2  & 1 & 2 & 3 \\
    3  & 2 & 1 & 0 \\
    6 & 1 & 2 & 3 \\ \hline
  \end{tabular}\qquad
  \begin{tabular}{|c|ccc|}
    \multicolumn{4}{c}{$S_3^c$} \\ \hline
    \multirow{2}{*}{$t$}  & \multicolumn{3}{c|}{$v$} \\ \cline{2-4}
       & temp & cons & vol \\ \hline
    2  & 1 & 2 & 3 \\
    3  & 2 & 1 & 0 \\
    6  & 1 & 2 & 3 \\ \hline
  \end{tabular} 
  \begin{tikzpicture}[baseline=(current bounding box.center)]
    \begin{axis}[
        timeseriesrel,
        title=$S_3$,
        legend columns = 4,
        every axis legend/.append style={
          at={(1,-0.1)},
          anchor=north east,
          draw = none},
        ]
    \addplot[only marks,mark=*,blue] coordinates {
        (2,1)
        (3,2)
        (6,1)
    };
    \addplot[only marks,mark=*,red] coordinates {
        (2,2)
        (3,1)
        (6,2)
    };
    \addplot[only marks,mark=*,green] coordinates {
        (2,3)
        (3,0)
        (6,3)
    };
    \legend{temp,cons,vol}
    \end{axis}
  \end{tikzpicture}
  %El gràfic d'una multivaluada: es poden pintar dos eixos verticals quan les diferències d'escala siguin molt grans.
  \caption{Taula d'una sèrie temporal multivaluada}
  \label{fig:model:serietemporal:multivaluada}
\end{figure}





\paragraph{Exemple 4} \emph{Valors vectors}.  Sèrie temporal $S_4$ on
el temps pertany a $\bar{\R{}}$ i el valor pertany a
$\bar{\R{}}^3$; és a dir és un vector representat amb un
tuple. Conté el valor $(1,2,3)$ en el temps $2$, el valor $(3,4,5)$ en
el temps $4$ i el valor $(1,2,3)$ en el temps $6$.

De manera simplificada s'escriu com $S_4 = \{ (2,(1,2,3)),
(3,(3,4,5)), (6,(1,2,3)) \}$ i es visualitza com a taula i com a
gràfic a la \autoref{fig:model:serietemporal:vector}.

\begin{figure}[tp]
  \centering
  \begin{tabular}{|c|c|}
    \multicolumn{2}{c}{$S_4$} \\ \hline
    $t$  & $v$ \\ \hline
    2  & (1,2,3) \\
    4  & (3,4,5) \\
    6  & (1,2,3) \\ \hline
  \end{tabular} \qquad
  \begin{tikzpicture}[baseline=(current bounding box.center)]
\begin{axis}[
        timeseriesrel,
        nodes near coords,
        title=$S_4$,
        yticklabels={},
        axis y line=none,
        ]
        \addplot+[only marks,mark=none,blue, point meta=explicit symbolic]
        coordinates {
          (2,0)  [\rotatebox{45}{(1,2,3)}]
          (4,0)  [\rotatebox{45}{(3,4,5)}]
          (5,1)  [\mbox{}]
          (6,0) [\rotatebox{45}{(1,2,3)}]
        };
      \end{axis}
    \end{tikzpicture}
    \caption{Taula d'una sèrie temporal amb valors vectors}
  \label{fig:model:serietemporal:vector}
\end{figure}



S'observa que una sèrie temporal amb valors vectors és diferent d'una
sèrie temporal multivaluada. El domini de la primera són vectors i el
de la segona són relacions d'un sol tuple en els que es pot operar
cada atribut per separat. En els vectors de forma general no es poden
operar cada component per separat sinó que formen una unitat
semàntica. Aquesta diferència de significat prové de si es considera
que es mesuren vectors o atributs diferents, el qual s'observa en la
visualització: el gràfic d'un vector és un espai $R^n$ en canvi el
gràfic d'una sèrie temporal multivaluada és un multigràfic, un gràfic
per a cada atribut.





\paragraph{Exemple 5} \emph{Valors sèrie
  temporal}. \label{par:model:exemple-relvalues} Sèrie temporal $S_5$
on el temps pertany a $\bar{\R{}}$ i el valor és una sèrie
temporal del mateix format que en l'exemple 1. Conté els tuples de
$S_1$ com a valors en el temps $1$ i $2$.

De manera simplificada s'escriu com $S_5 = \{ (1,\{ (2,1), (3,3),
(6,1) \}), (2,\{ (2,1),$ $(3,3),$ $(6,1) \}) \}$ i es visualitza com a
taula a la \autoref{fig:model:serietemporal:serietemporal}.


\begin{figure}[tp]
  \centering
  \begin{tabular}{|c|c|}
    \multicolumn{2}{c}{$S_5$} \\ \hline
    $t$  & $v$ \\ \hline
    1 &   
       \begin{tabular}{|c|c|}
         \hline
         $t$  & $v$ \\ \hline
         2  & 1 \\
         3  & 3 \\
         6  & 1 \\ \hline
       \end{tabular} \\ \hline
    2 & 
       \begin{tabular}{|c|c|}
         \hline
         $t$  & $v$ \\ \hline
         2  & 1 \\
         3  & 3 \\
         6  & 1 \\ \hline
       \end{tabular} \\ \hline
  \end{tabular}
  \caption{Taula d'una sèrie temporal amb valors sèrie temporal}
  \label{fig:model:serietemporal:serietemporal}
\end{figure}

S'observa que la capçalera de $S_5$ és $\{t:\bar{\R{}},v:
\{t:\bar{\R{}},v:\bar{\R{}}\}\}$. És a dir, el valor és
una altra relació, com es descriu per \textcite[sec.\
5.3]{date:introduction}, a on el temps i el valor pertanyen a
$\bar{\R{}}$. Per tant, el valor de $S_5$ és de tipus sèrie
temporal amb valors reals. Tot i així, aquest és un valor especial i
si es desplega la relació s'obté una sèrie temporal doble amb
capçalera $\{t^1,t^2,v\}$.

% Cal insistir que \emph{tots} el valors de $S_5$ han de pertànyer al
% mateix domini \parencite[sec.\ 5.4]{date:introduction}, el qual és
% $\{temps:\bar{\R{}},valor:\bar{\R{}}\}$.























%%% Local Variables:
%%% TeX-master: "main"
%%% End:







% LocalWords:  SGST multivaluada

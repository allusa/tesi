%--------------------
% document principal
%--------------------
% cal compilar amb `pdflatex main.tex`
%--------------------
\documentclass[paper=a4,fontsize=11pt,twoside,parskip=half,BCOR12mm]{scrbook}
%%%%BCOR12mm  factor de correcció per enquadernació
%------------- capçalera ----------------------
\input{capçalera.default}
\bibliography{bibliografia}
\ExecuteBibliographyOptions{
%  annotation=true,backref=true,
%  isbn=false,url=false,doi=false,alldates=terse,firstinits=true,abbreviate=true
}
%\bibitemsep 0cm \biblabelsep 0cm \bibhang 0cm \renewcommand{\bibfont}{\normalfont\footnotesize}
%---------- Mode esborrany --------------------
%\includeonly{resum}
\usepackage[catalan]{todonotes} %%ús: \todo{text} \missingfigure{text}
\usepackage{fancyhdr}\pagestyle{fancyplain}\chead{\fancyplain{--- esborrany \today\ ---}{\footnotesize\today}}
%%\renewcommand{\headrulewidth}{0pt}
%----------------------------------------------

%------------- format -------------------------
%%ús coma decimal sense espais:  2{,}5
\newtheorem{definition}{Definició}
\def\figureautorefname{figura} %ús: \autoref{}
\def\tableautorefname{taula} %ús: \autoref{}
\numberwithin{equation}{chapter}

\usepackage{bold-extra}
\usepackage{gantt}
%-------------- dades --------------------------
\hypersetup{
    pdftitle={Estudi i modelització dels sistemes de gestió de sèries temporals},
    pdfauthor={Aleix Llusà Serra},
    pdfcreator={DiPSE--UPC},
    pdfsubject={Proposta de tesi 2012},
    pdfkeywords={sèries temporals; sistemes de gestió de bases de dades; SGBD per a sèries temporals; SGST; dades temporals; adquisició de dades},
    pdflang=ca,
}
\title{Estudi i modelització dels sistemes de gestió de sèries temporals}
\author{Aleix Llusà Serra}
%----------------------------------------------





%\includeonly{sgbd}

\begin{document}

%------------- pàgina de portada -----------
\begin{titlepage}
  \begin{center} 

   

    {\Large \scshape Universitat Politècnica de Catalunya} \vskip 1cm 

    {Programa de Doctorat:} \vskip 0.5cm 
    
    {\scshape Automàtica, Robòtica i Visió} \vfill%\vskip 4cm 

    {Tesi Doctoral} \vskip 1cm 
    
    {\scshape \bfseries \Large Disseny i modelització d'un sistema de gestió\\
 multiresolució per a sèries temporals} \vskip 2cm

    {\bfseries Aleix Llusà Serra} \vfill%\vskip 4cm 

    {Direcció:}
       
    {Teresa Escobet Canal i
    Sebastià Vila-Marta}  \vskip 1cm 
    %\vfill 

    {Juny de 2015}

\end{center}
\end{titlepage}


%------------- pàgina de crèdits -----------
{
  \thispagestyle{empty}

  \mbox{}

  \vfill

  Primera edició: setembre de 2015. %Enquadernació en espiral, primera impressió.
  \\
  {\small Primera versió: 1.0.0 (composta a \today).} 

  \mbox{}

  {\footnotesize
  Amb el suport de la Universitat Politècnica de Catalunya (UPC).
  

  }

  \cc\bysa

  {\small
  Copyright (C) 2015 Aleix Llusà Serra.
  

  {\footnotesize
    Aquest document està sotmès a una llicència de Reconeixement-CompartirIgual 3.0 No adaptada de Creative Commons. Per veure una còpia de la llicència, visiteu \url{http://creativecommons.org/licenses/by-sa/3.0/deed.ca} o envieu una carta a Creative Commons, 444 Castro Street, Suite 900, Mountain View, California, 94041, USA.
  }

    Aleix Llusà Serra\\
    Departament de Disseny i Programació de Sistemes Electrònics
      de la Universitat Politècnica de Catalunya (DiPSE--UPC)\\
    Escola Politècnica Superior d'Enginyeria de Manresa (EPSEM),
    Av.\ de les Bases de Manresa, 61-73,
    08242 Manresa (Barcelona),
    CATALUNYA 
    }\\
    \url{aleix@dipse.upc.edu}

    {\footnotesize
      El codi font \LaTeX\ del document es troba a 
      \url{http://escriny.epsem.upc.edu/projects/rrb/}
    }
}





%%% Local Variables: 
%%% mode: latex
%%% TeX-master: "main"
%%% End: 



\tableofcontents{}


\begin{abstract}

  En aquest document es detalla el progrés en el pla de treball
  presentat el 2012 per a assolir la recerca en el disseny d'un model
  de sistema de gestió per a sèries temporals. En primer lloc es
  presenta l'actualització dels objectius, en segon lloc el progrés en
  funció de les tasques planificades realitzades i en tercer lloc la
  modificació del pla de treball per a assolir els objectius pendents.
\end{abstract}



\section{Objectius}
\label{sec:objectius}

Aquesta recerca té per objectiu l'estudi de les necessitats
específiques que comporta l'emmagatzematge i gestió de dades amb
naturalesa de sèrie temporal i la proposta d'un model de SGBD que
satisfaci aquestes necessitats. Aquest objectiu es divideix en els
següents subobjectius més concrets:

\begin{enumerate}

\item Estudi de les aplicacions en que les dades són sèries temporals
  amb la finalitat de determinar quines són les propietats i problemes
  comuns que planteja la seva gestió i emmagatzematge.

\item Estudi dels models de SGBD existents. Segons es desprèn de la
  formalització de \textcite{date:introduction} el model principal és
  el model relacional, el qual es fonamenta en dos conceptes:
  relacions i tipus de dades. 

\item Una àrea de treball important en els SGBD és la incorporació de
  nous tipus de dades complexos. És important estudiar com es modifica
  el model de dades d'un SGBD quan s'afegeix un nou tipus de dades
  complex.  Les sèries temporals es poden d'entendre com a tipus
  complex ja que presenten diferents propietats característiques i
  necessiten operadors addicionals.  

\item Disseny d'un model de SGBD per a les sèries temporals. D'aquesta
  manera els SGBD podran tractar dades amb instants de temps que
  mostrin l'evolució de variables en funció del temps. El model
  consisteix en la definició de l'estructura de les sèries temporals i
  les operacions bàsiques que necessiten.

  L'assoliment d'aquest objectiu té tres parts:

  \begin{enumerate}
  \item Disseny d'un model per a la gestió bàsica de les sèries
    temporals, el qual anomenem model de SGBD per a sèries temporals
    (SGST).  L'estructura d'aquest model és similar a l'utilitzat en
    els intervals
    temporals \parencite{date02:_tempor_data_relat_model}.  Prenent
    com a base el model de SGST, el qual és un model general per a les
    sèries temporals, s'hi poden incloure altres models per a
    propietats més específiques de les sèries temporals.

  \item Disseny d'un model específic en base del model de
    SGST. Concretament es dissenya un model pels SGST multiresolució
    (SGSTM).  En el model de SGSTM s'hi poden incloure propietats de
    les sèries temporals relacionades amb la resolució que s'han
    observat en les aplicacions pràctiques de les sèries temporals:
    regularització, canvis de resolució mitjançant agregacions,
    reconstrucció de forats, etc.
 
  \item Avaluació de diferents estructures de SGSTM. El model de SGSTM
    s'hi poden fer modificacions o simplificacions per tal
    d'aconseguir diferents estructures.  Per exemple bases de dades
    multiresolució que comparteixin informació, que treballin amb flux
    de dades (\emph{data stream}) o bé que s'especialitzin per a un
    tipus determinat de sèries temporals.

  \end{enumerate}

\item Implementació de referència dels models de SGST i SGSTM. Per una
  banda, aquesta implementació, a nivell acadèmic, ha de servir com a
  exemple per a futurs desenvolupaments de sistemes de gestió,
  acadèmics o productius. Per altra banda, ha de servir per a
  exemplificar-ne els seu funcionament amb unes dades de prova.

\item Implementació específica i reduïda del model per a una
  determinada aplicació de sèries temporals. Exemplificació de com una
  estructura de SGSTM pot ser implementada per a aconseguir una
  aplicació molt concreta.

\end{enumerate} 


El objectius 4.c i 6. són resultat d'una nova planificació com es
detalla més endavant a les tasques futures.






%%% Local Variables: 
%%% mode: latex
%%% TeX-master: "main"
%%% End: 
% LocalWords:  SGSTM multiresolució SGST


\chapter{Estat actual}%\chapter{Estat actual}
\label{cap:estat}

En aquest capítol se situen els sistemes de gestió de bases de dades (SGBD) per sèries temporals en el context de la mineria de dades de sèries temporals (\emph{time series data mining}), el qual també es considerat com mineria de dades per  detectar automàticament coneixement (\emph{knowledge discovery databases}). Els SGBD de model Round Robin (RRD) pertanyen a aquest context ja que  emmagatzemen sèries temporals  de les quals es vol aconseguir informació rellevant.


El capítol comença resumint l'estat de les sèries temporals en aquest camp de mineria; és a dir d'emmagatzematge i tractament. A continuació es llisten algunes aplicacions informàtiques que han implementat models de la mineria de sèries temporals. Finalment, es descriu l'estat actual de l'aplicació RRDtool, la qual també es classifica en aquest camp.



\section{Mineria de sèries temporals}

L'anàlisi de sèries temporals abasta camps molt diferents com ara la predicció econòmica, la medicina, la meteorologia, la qualitat industrial, etc. En aquest context,  la mineria de sèries temporals tracta de gestionar co\l.leccions cronològiques de dades que tenen una mida gran i contínuament estan en creixement. 
Aquest apartat se centra en  l'estat actual de la mineria de sèries temporals, àmbit que, en la darrera dècada, ha experimentat un important increment de la recerca.


En un article molt recent de Tak-chung Fu,~\cite{fu11}, es fa esment d'aquest increment de recerca i es resumeix l'estat actual de forma exhaustiva. Fu conclou que la recerca s'ha centrat en tasques de mineria però no s'ha resolt del tot el problema de la representació de sèries temporals.

Segons Keogh i Kasetty,~\cite{keogh02}, les quatre tasques que centren l'atenció de la recerca actual de sèries temporals són l'indexat, l'agrupament, la classificació i la segmentació. A més, Keogh compara els experiments duts a terme en aquests camps.
Un pas comú previ a aquestes quatre tasques és el de representació de la sèrie temporal. 
Keogh \emph{et al.},~\cite{keogh97,keogh98}, investiguen la representació de sèries temporals a trossos lineals (PLR, \emph{Piecewise Linear Representation}). Keogh fa notar que la representació PLR és l'habitual degut a que la visió de l'ésser humà segmenta les corbes en línies rectes.
Més tard, Keogh,~\cite{keogh00,keogh01}, explora la representació de sèries temporals per tal de reduir la dimensió d'una sèrie temporal i poder-la indexar més fàcilment  i proposa dues tècniques eficients en el càlcul: la PAA (\emph{Piecewise Aggregate Aproximation}) i  la APCA (\emph{Adaptive Piecewise Constant Approximation}), ambdues basades en la representació a trossos constants de la sèrie temporal. 
D'aquestes dues tècniques Keogh conclou que mantenen una bona aproximació a la sèrie temporal i que a més  tenen molt menys cost de càlcul que altres de més complicades, com ara la \emph{Discrete Fourier Transform} (DFT),  la  \emph{Singular Value Decomposition} (SVD) o la \emph{Discrete Wavelet Transform} (DWT).



Tal com expliquen Quevedo \emph{et al.},~\cite{quevedo10}, en un sistema complex de telecontrol hi ha una gran quantitat d'informació a manipular que s'obté de diversos sensors distribuïts pel camp de mesura, aquesta informació s'anomena variables mesurades. Un SCADA (\emph{Supervisory Control And Data Acquisition})  és el sistema encarregat de recollir i centralitzar les variables de manera periòdica en el temps. En el moment de reco\l.lecció de dades apareixen dos problemes: valors que en un instant de temps prefixat no s'han pogut recollir i valors que són falsos. Les tècniques de bases de dades no poden emmagatzemar les dades amb aquests dos tipus de problema ja que aleshores els registres històrics quedarien falsejats. Així doncs, cal comprovar que les dades emmagatzemades són correctes, segons un procés de validació, i modificar-les en el cas que siguin falses, segons un procés de reconstrucció. Quevedo,~\cite{quevedo10}, aplica aquests processos a xarxes de distribució d'aigua.

Els mètodes de validació i reconstrucció es poden basar en anàlisis senzilles del senyal o en comparacions del valor real amb models de predicció de dades. Quan les dades es tracten com a sèries temporals, hi ha mètodes de predicció específics.
Tot i que la teoria de sèries temporals permet establir aquests mètodes de predicció i reconstrucció, els SGBD habituals, com ara els de model relacional, no ho faciliten.  
Per tal d'aplicar aquests mètodes a les sèries temporals de manera eficient, els SGBD s'han d'especialitzar en el tractament de sèries temporals.



\section{SGBD per sèries temporals}


Per poder analitzar les dades de manera eficient cal disposar de bases de dades específiques, a més cada cop el volum de dades a tractar és més crític degut a que hi ha més facilitat a capturar-les i més capacitat per emmagatzemar-les. 
La diferència principal de les sèries temporals amb altres tipus de dades és que els valors són dependents d'una variable: el temps. Com a conseqüència, qualsevol base de dades que hi vulgui tractar no ho pot fer de manera independent pels valors i pel temps; ha de conservar la coherència temporal.

Tal com diu A{\ss}falg,~\cite{assfalg08:thesis}, la coherència temporal pot ser vista des de dues vessants. La primera, a la qual anomena \emph{bitemporal data}, consisteix en expressar el temps vàlid durant el qual un esdeveniment és cert i el temps de transacció durant el qual l'esdeveniment és guardat a la base de dades, és a dir consisteix a descriure dos estats, cert o fals, per cada observació. La segona, a la qual anomena \emph{time series data}, consisteix a descriure co\l.leccions de dades en funció del temps. A més diu que les primeres poden ser expressades amb les segones.

Els SGBD relacionals són capaços d'implementar el primer tipus de coherència, les \emph{bitemporal data}; llavors es classifiquen sota el nom de bases de dades temporals, \cite{date,wiki:temporal_database}. Però el model relacional no és suficient pel segon tipus: les sèries temporals. Tot i que en principi no hi hauria cap problema a utilitzar una base de dades relacional per a sèries temporals, enteses com a dades històriques, la pròpia naturalesa dels sistemes relacionals  dificulta les operacions necessàries. 
Aquestes operacions per sèries temporals es basen en rangs de temps i precisen conversions de fusos horaris i rotacions dels registres de les taules, sinó el nombre de files creixeria de forma indefinida. 

Els SGBD que implementen operacions per a sèries temporals es poden anomenar \emph{Time Series Database Systems} (TSDS),~\cite{wiki:tsds}. Les TSDS Estan optimitzades per gestionar les dades segons les operacions de temps i rotació, les quals són molt comunes en la gestió de les sèries temporals.  A més també cal controlar el creixement de la base de dades i la consulta ha de ser flexible i d'alta velocitat,~\cite{keogh10:isax}. Per exemple, s'han de poder visualitzar les evolucions tant d'una setmana com d'un any sense haver de fer càlculs complicats amb els valors emmagatzemats. 
A continuació es llisten dues bases de dades optimitzades per a sèries temporals.

A{\ss}falg,~\cite{assfalg08:thesis}, presenta un TSDS que és capaç de
cercar similituds, també anomenades distàncies, entre sèries temporals. Principalment utilitza llindars per comparar en cada interval si les dues sèries temporals s'assemblen. A partir d'aquest mètode desenvolupa algoritmes que calculen de manera eficient per a les sèries temporals i en concret els implementa en una aplicació anomenada T-Time, la qual descriu a~\cite{assfalg08:ttime}.

Keogh i Camerra~\cite{keogh08:isax,keogh10:isax}, 
estudien l'anàlisi i l'indexat de co\l.lecions massives de sèries temporals. Descriuen que el problema principal del tractament rau en l'indexat de les sèries temporals i proposen mètodes per calcular-lo de manera eficient. El mètode principal que desenvolupen està basat en l'aproximació a trossos constants de la sèrie temporal (PAA,~\cite{keogh00}) i ho implementen en una estructura de dades que anomenen iSAX (\emph{indexable Symbolic Aggregate approXimation}),~\cite{isax}. Amb aquesta eina s'obtenen representacions de sèries temporals que permeten reduir l'espai emmagatzemat i indexar tant bé com altres mètodes de representació més complexos.




En resum, aquests SGBD per sèries temporals bàsicament resolen els problemes d'anàlisis de sèries temporals.
Però cap d'aquestes sol atendre la relació entre la base de dades i el sistema de monitoratge, és a dir la manera com s'adquireixen les dades. En aquest pas intermig hi ha un sèrie de problemes, com per exemple forats, dades falses, irregularitat en els temps de mostreig, que cal gestionar correctament. Concretament un dels problemes que no s'atén és el de mostreig irregular ja que es considera que les mostres estan a intervals regulars (equi-espaiades) encara que els sistemes de monitoratge informàtics sovint no són capaços de complir-ho amb exactitud sinó que presenten una certa variació en els temps de mesura. 

Així doncs, quan es prenen mesures d'un sistema productiu, aquests problemes apareixen i són de difícil solució.
Les bases de dades RRDtool tenen en compte aquests problemes intermitjos entre el sistema de monitoratge i el sistema d'emmagatzematge i tractament. 






\section{Base de dades RRDtool}

En aquest apartat es presenta el TSDS anomenat RRDtool. Aquest sistema, que serà objecte d'un estudi acurat en els capítols \ref{cap:rrdtool} i~\ref{cap:rrdtool-etapes}, s'ha pres com a referència en aquest treball.

RRDtool és un SGBD per a sèries temporals que despunta en l'àmbit de programari lliure. Hi ha una llista de projectes que utilitzen RRDtool que poden trobar-se indicats a l'apartat \emph{Projects using RRDtool} de~\cite{rrdtool}.
Entre d'altres, s'utilitza en sistemes de monitoratge professionals com per exemple Nagios,~\cite{nagios}, o Icinga,~\cite{icinga}, també populars dins del programari lliure, o en el montior MRTG (The Multi Router Traffic Grapher),~\cite{mrtg}, del mateix creador que RRDtool. Aquests monitors fan un ús complet de les possibilitats de RRDtool i li cedeixen tot el control de l'emmagatzematge de mesures i el posterior tractament i representació gràfica de les dades. 
L'ús de RRDtool permets a aquestes aplicacions centrar-se plenament en la problemàtica de l'adquisició de dades i la gestió d'alarmes.

En l'evolució de RRDtool destaquen dues millores significatives.
La primera, descrita per Oetiker a~\cite{lisa98:oetiker}, va consistir en independitzar la base de dades RRDtool del sistema de monitoratge MRTG i dissenyar-la amb l'estructura Round Robin que la caracteritza. La segona, feta per Brutlag,~\cite{lisa00:brutlag}, ha aportat la possibilitat de fer prediccions i detecció de comportaments aberrants basant-se en algoritmes de predicció exponencials i de Holt-Winters. 


L'evolució actual de RRDtool se centra en aspectes informàtics i consisteix a millorar la rapidesa i eficiència en el processament de les sèries temporals. És el cas de Plonka i Carder que a~\cite{carder:rrdcached,lisa07:plonka} dissenyen l'aplicació \verb+rrdcached+ per incrementar el rendiment de RRDtool, la qual demostren en un sistema de monitoratge amb moltes bases de dades funcionant simultàniament.  També \verb+JRobin+,~\cite{jrobin}, que és una implementació en Java de RRDtool que millora els accessos de lectura i escriptura a la base de dades i té una eina de gràfics més perfeccionada.
És significatiu l'ús incipient d'aquest sistema en experimentació. Zhang,~\cite{zhang07}, i Chilingaryan,~\cite{chilingaryan10}, per exemple, usen RRDtool per emmagatzemar de dades experimentals i posteriorment fer predicció o validació.
  

En l'àmbit dels SGBD els sistemes relacionals van fixar una fita que ha tingut una transcendència posterior de  primer ordre. En bona part aquest èxit dels SGBD relacionals es deu al fet que es basen en un model matemàtic sòlid,~\cite{date}.
En el cas de RRDtool no existeix  cap model que descrigui el sistema i es objectiu d'aquest treball proposar-ne un. El model per a SGBD Round Robin es dissenya  al capítol~\ref{cap:model-rrd}.






%%% Local Variables: 
%%% mode: latex
%%% TeX-master: "memoria"
%%% End: 

% LocalWords:  monitoratge RRDtool SGBD RRD Round databases mining SCADA And
% LocalWords:  Supervisory Acquisition bitemporal Time Database Systems TSDS
% LocalWords:  Nagios Icinga Grapher Holt-Winters


\subsection*{Planificació del treball}


Per tal d'assolir els objectius detallats a la secció
\ref{sec:objectius}, a continuació es proposen les tasques a
realitzar.  A la figura \ref{fig:pla:futur} es detalla el pla de
treball amb temps estimat.





\begin{enumerate}


\item Estudi d'aplicacions de les sèries temporals. Per a assolir
  l'objectiu~1 estudiarem recerca actual de sèries temporals.
  Consultarem articles i llibres que tinguin les sèries temporals com
  a temàtica principal. També cercarem l'existència de programari que
  tingui en els seus objectius el tractament de sèries temporals.

\item Estudi del model relacional. Per a l'objectiu~2 estudiarem el
  model relacional com a referent pels models de SGBD. Ens basarem en
  l'estudi dels llibres de
  \textcite{date:introduction,date06,date:dictionary}. Usarem
  \emph{rel} \parencite{rel} com a implementació de referència ja que
  incorpora el llenguatge \emph{Tutorial D}, el qual és utilitzat per
  Date en els seus exemples.

\item Estudi de la gestió d'intervals temporals. Per a l'objectiu~3
  estudiarem la recerca que ha conduit a incorporar els interval
  temporals en els SGBD per a gestionar històrics. Ens basarem en
  l'estudi que relaciona el model relacional amb els intervals
  temporals de \textcite{date02:_tempor_data_relat_model}.

\item Disseny d'un model de SGBD per sèries temporals. Per a la
  primera part de l'objectiu 4 dissenyarem un model per als SGBD de
  sèries temporals formalitzat amb expressions algebraiques. A partir
  de l'estudi de l'estat de l'art de les sèries temporals, observarem
  les propietats interessants de ser modelitzades i les operacions que
  precisen els SGBD per sèries temporals.

\item Disseny d'un model de SGBD multiresolució per sèries
  temporals. Per a la segona part de l'objectiu 4 dissenyarem un model
  que contempli la multiresolució de les sèries temporals. Aquest
  model utilitzarà propietats del model anterior per les sèries
  temporals. La mulitresolució té l'objectiu d'emmagatzemar les sèries
  temporals de forma compacta, així es preveu que alguns conceptes de la recerca
  en \emph{data streams} poden prendre-hi sentit.

\item Implementació de referència. Per a l'objectiu~5 s'implementaran
  els models anteriors utilitzant un llenguatge de programació adequat
  per a models, com per exemple \texttt{Python} o \texttt{Prolog}.  Es
  prioritzarà la implementació correcte del model enfront a una
  implementació que contempli un bon rendiment. 

\item Experimentació amb dades. Per a complementar l'objectiu~5 es
  provarà la implementació amb dades experimentals per alguna
  aplicació concreta.

\end{enumerate}

\begin{figure}[tp]
\centering
\scalebox{0.8}{
\begin{gantt}[xunitlength=0.8cm,fontsize=\small,titlefontsize=\small]{15}{12}
  \begin{ganttitle}
    \numtitle{2012}{1}{2014}{4}
  \end{ganttitle}
  \begin{ganttitle}
    \numtitle{1}{1}{4}{1}
    \numtitle{1}{1}{4}{1}
    \numtitle{1}{1}{4}{1}
  \end{ganttitle}

  \ganttmilestone{Proposta de tesi}{2}

  \ganttgroup{Estudi}{0}{4}
  \ganttbar{1. Aplicacions}{0}{2}
  \ganttbar{2. Model relacional}{0}{4}
  \ganttbar{3. Intervals temporals}{2}{2}

  \ganttgroup{Disseny models}{4}{3}
  \ganttbar{4. Sèries temporals}{4}{3}
  \ganttbar{5. Multiresolució}{5}{2}
  \ganttcon{7}{7}{7}{11}

  \ganttgroup{Experimentació}{7}{2}
  \ganttbar{6. Implementació}{7}{2}
  \ganttbar{7. Dades experimentals}{8}{1}

  \ganttbar{8. Redacció memòria}{8}{2}

  \ganttmilestonecon{Lectura de tesi}{10}

\end{gantt}
}
\caption{Planificació del treball}
\label{fig:pla:futur}
\end{figure}



\subsubsection*{Treball realitzat}


La tesi de màster \parencite{llusa11:tfm} va consistir en l'estudi de
\emph{RRDtool} \parencite{rrdtool}, un sistema de gestió de bases de
dades (SGBD) específic per a sèries temporals.
%
Fruit d'aquest estudi es va formalitzar el model de \emph{RRDtool} i es va
dissenyar i implementar un prototip software que responia a la
formalització.
%
Aquest treball va permetre concloure que:
\begin{itemize}
\item Els SGBD aplicats a sèries temporals tenen aspectes propis que
  els converteixen en objecte d'estudi de \emph{per se}.
\item El concepte de multiresolució és interessant en moltes
  aplicacions reals, especialment quan existeixen restriccions d'espai
  per emmagatzemar dades.
\item El model proposat per a \emph{RDDtool} és susceptible de ser millorat
  com a mínim en el següents aspectes:
  \begin{itemize}
  \item Generalització. El model que es va presentar estava fortament
    lligat a \emph{RDDtool}. Generalitzar el model de forma que encabeixi
    altres concepcions.
  \item Incorporació d'operacions. El model presentat únicament feia
    referència a les dades. Per completar el
    model cal també considerar les operacions.
  \item Contextualització. Cal interrelacionar i descriure el model el
    context d'altres models existents per a SGBD. Específicament cal
    comparar-lo amb el model relacional usat en els SGBD
    convencionals.
  \end{itemize}
\item És convenient estudiar la feina feta per altres autors en
  l'àmbit de l'emmagatzemat i gestió de dades provinents de sèries
  temporals.
\end{itemize}

Arrel de la feina anterior, s'ha treballat en els següents aspectes:
\begin{itemize}
\item S'ha realitzat una tasca de recerca bibliogràfica i estudi dels
  treballs existents en l'àmbit de la gestió i emmagatzemat de dades
  provinents de sèries temporals. El resultat d'aquesta tasca s'ha
  reflectit a la proposta de tesi \parencite{llusa12:ptd}.
\item S'ha estudiat en profunditat el model relacional. Atesa la
  preeminència d'aquest model en els SGBD actuals, s'ha considerat
  imprescindible tenir-ne un bon coneixement que permetés estudiar les
  seves deficiències per la gestió de sèries temporals i en quina
  forma poden ser superades.
\item S'està refent el model de SGBD per sèries temporals.  S'ha
  dividit el model en dues parts ben diferenciades:
  \begin{enumerate}
  \item La primera és el model general. Aquest defineix la gestió de
    sèries temporals enteses com a co\l.lecció de dades mesurades en
    diferents instants de temps. Es basa en els conceptes de temps,
    mesura i sèrie temporal.
  \item La segona és el model de multiresolució. Aquest model explica
    la forma d'emmagatzemar una sèrie temporal amb diferents
    resolucions temporals.
  \end{enumerate}
\end{itemize}





\subsection*{Mitjans}

La recerca es duu a terme amb el suport de la Universitat Politècnica
de Catalunya (UPC) mitjançant una beca FPU-UPC adscrita al departament
d'Enginyeria del Disseny i Programació de Sistemes Electrònics
(DiPSE).

No es preveu un ús de mitjans més enllà de l'accés als
recursos bibliogràfics i d'eines informàtiques de programació i
gestió de documentació.



%%% Local Variables: 
%%% mode: latex
%%% TeX-master: "main"
%%% End: 


%------- Bibliografia ------
\cleardoublepage
%\phantomsection\addcontentsline{toc}{chapter}{\bibname}
\pdfbookmark{\bibname}{bookmark:bibliografia}
\printbibliography
%----------------------------------------------


\appendix

\chapter{Direcció}


En tractar-se d'una recerca situada entre dos àmbits, l'anàlisi de les
sèries temporals i els sistemes de gestió de bases de dades, compta
respectivament amb la direcció de dos doctors d'aquest àmbits: la
Teresa Escobet Canal i el Sebastià Vila-Marta.

\section{Teresa Escobet Canal}

És professora del departament d'Enginyeria del Disseny i Programació de
Sistemes Electrònics de la Universitat Politècnica de Catalunya. 
És membre del Programa de Doctorat en Automàtica, Robòtica i Visió.



\section{Sebastià Vila-Marta}

És professor del departament d'Enginyeria del Disseny i Programació de
Sistemes Electrònics de la Universitat Politècnica de Catalunya.
A continuació s'incorpora el seu currículum.






%%% Local Variables: 
%%% mode: latex
%%% TeX-master: "main"
%%% End: 


\end{document}


%%%%%%%%%%%%%%%%%%%%%%%%%%%%%%%%%%%%%%%%%%%%%%%%%%%%%%%%%%%%%%%%%%%%%%%%%%  
% Model dels sistemes de gestió de bases de dades per sèries temporals.
%
% Copyright (C) 2011-2012 Aleix Llusà Serra.
% 
% This LaTeX document is free software: you can redistribute it and/or
% modify it under the terms of the GNU General Public License as
% published by the Free Software Foundation, either version 3 of the
% License, or (at your option) any later version.
%
% This document is distributed in the hope that it will be useful, but
% WITHOUT ANY WARRANTY; without even the implied warranty of
% MERCHANTABILITY or FITNESS FOR A PARTICULAR PURPOSE. See the GNU
% General Public License for more details.
%
% You should have received a copy of the GNU General Public License
% along with this document. If not, see <http://www.gnu.org/licenses/>.
%
%
% Aleix Llusà Serra
% Departament de Disseny i Programació de Sistemes Electrònics de la Universitat Politècnica de Catalunya (DiPSE-UPC)
% Escola Politècnica Superior d'Enginyeria de Manresa (EPSEM)
% Av. de les Bases de Manresa, 61-73
% 08242 Manresa (Barcelona)
% PAÏSOS CATALANS 
%
% aleix (a) dipse.upc.edu
% 
% El codi font LaTeX del document es troba a 
% <http://escriny.epsem.upc.edu/projects/rrb/>
%%%%%%%%%%%%%%%%%%%%%%%%%%%%%%%%%%%%%%%%%%%%%%%%%%%%%%%%%%%%%%%%%%%%%%%%%%  


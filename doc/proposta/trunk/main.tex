%--------------------
% document principal
%--------------------
% cal compilar amb `pdflatex main.tex`
%--------------------
\documentclass[paper=a4,fontsize=10pt,twoside,parskip=half,BCOR-3mm]{scrbook}
%%%%BCOR12mm  factor de correcció per enquadernació en rústica
%%%%BCOR??mm  factor de correcció per enquadernació amb espiral
%------------- capçalera ----------------------
\input{capçalera.default}
\bibliography{bibliografia}
\ExecuteBibliographyOptions{
%  annotation=true,backref=true,
isbn=false,url=false,doi=false,alldates=terse,firstinits=true,abbreviate=true
}
%\bibitemsep 0cm \biblabelsep 0cm \bibhang 0cm \renewcommand{\bibfont}{\normalfont\footnotesize}
%---------- Mode esborrany --------------------
%\includeonly{resum}
\usepackage[catalan]{todonotes} %%ús: \todo{text} \missingfigure{text}
\usepackage{fancyhdr}\pagestyle{fancyplain}\chead{\fancyplain{--- esborrany \today\ ---}{\footnotesize\today}}
%%\renewcommand{\headrulewidth}{0pt}
%----------------------------------------------

%------------- format -------------------------
%%ús coma decimal sense espais:  2{,}5
\newtheorem{definition}{Definició}
\def\figureautorefname{figura} %ús: \autoref{}
\def\tableautorefname{taula} %ús: \autoref{}
\numberwithin{equation}{chapter}

\usepackage{bold-extra}
\usepackage{gantt}
%-------------- dades --------------------------
\hypersetup{
    pdftitle={Estudi i modelització dels sistemes de gestió de sèries temporals},
    pdfauthor={Aleix Llusà Serra},
    pdfcreator={DiPSE--UPC},
    pdfsubject={Proposta de tesi 2012},
    pdfkeywords={sèries temporals; sistemes de gestió de bases de dades; SGBD per a sèries temporals; SGST; dades temporals; adquisició de dades},
    pdflang=ca,
}
\title{Estudi i modelització dels sistemes de gestió de sèries temporals}
\author{Aleix Llusà Serra}
%----------------------------------------------





%\includeonly{sgbd}

\begin{document}

%------------- pàgina de portada -----------
\begin{titlepage}
  \begin{center} 

   

    {\Large \scshape Universitat Politècnica de Catalunya} \vskip 1cm 

    {Programa de Doctorat:} \vskip 0.5cm 
    
    {\scshape Automàtica, Robòtica i Visió} \vfill%\vskip 4cm 

    {Tesi Doctoral} \vskip 1cm 
    
    {\scshape \bfseries \Large Disseny i modelització d'un sistema de gestió\\
 multiresolució per a sèries temporals} \vskip 2cm

    {\bfseries Aleix Llusà Serra} \vfill%\vskip 4cm 

    {Direcció:}
       
    {Teresa Escobet Canal i
    Sebastià Vila-Marta}  \vskip 1cm 
    %\vfill 

    {Juny de 2015}

\end{center}
\end{titlepage}


%------------- pàgina de crèdits -----------
{
  \thispagestyle{empty}

  \mbox{}

  \vfill

  Primera edició: setembre de 2015. %Enquadernació en espiral, primera impressió.
  \\
  {\small Primera versió: 1.0.0 (composta a \today).} 

  \mbox{}

  {\footnotesize
  Amb el suport de la Universitat Politècnica de Catalunya (UPC).
  

  }

  \cc\bysa

  {\small
  Copyright (C) 2015 Aleix Llusà Serra.
  

  {\footnotesize
    Aquest document està sotmès a una llicència de Reconeixement-CompartirIgual 3.0 No adaptada de Creative Commons. Per veure una còpia de la llicència, visiteu \url{http://creativecommons.org/licenses/by-sa/3.0/deed.ca} o envieu una carta a Creative Commons, 444 Castro Street, Suite 900, Mountain View, California, 94041, USA.
  }

    Aleix Llusà Serra\\
    Departament de Disseny i Programació de Sistemes Electrònics
      de la Universitat Politècnica de Catalunya (DiPSE--UPC)\\
    Escola Politècnica Superior d'Enginyeria de Manresa (EPSEM),
    Av.\ de les Bases de Manresa, 61-73,
    08242 Manresa (Barcelona),
    CATALUNYA 
    }\\
    \url{aleix@dipse.upc.edu}

    {\footnotesize
      El codi font \LaTeX\ del document es troba a 
      \url{http://escriny.epsem.upc.edu/projects/rrb/}
    }
}





%%% Local Variables: 
%%% mode: latex
%%% TeX-master: "main"
%%% End: 



\tableofcontents{}


\begin{abstract}

  En aquest document es detalla el progrés en el pla de treball
  presentat el 2012 per a assolir la recerca en el disseny d'un model
  de sistema de gestió per a sèries temporals. En primer lloc es
  presenta l'actualització dels objectius, en segon lloc el progrés en
  funció de les tasques planificades realitzades i en tercer lloc la
  modificació del pla de treball per a assolir els objectius pendents.
\end{abstract}



\section{Objectius}
\label{sec:objectius}

Aquesta recerca té per objectiu l'estudi de les necessitats
específiques que comporta l'emmagatzematge i gestió de dades amb
naturalesa de sèrie temporal i la proposta d'un model de SGBD que
satisfaci aquestes necessitats. Aquest objectiu es divideix en els
següents subobjectius més concrets:

\begin{enumerate}

\item Estudi de les aplicacions en que les dades són sèries temporals
  amb la finalitat de determinar quines són les propietats i problemes
  comuns que planteja la seva gestió i emmagatzematge.

\item Estudi dels models de SGBD existents. Segons es desprèn de la
  formalització de \textcite{date:introduction} el model principal és
  el model relacional, el qual es fonamenta en dos conceptes:
  relacions i tipus de dades. 

\item Una àrea de treball important en els SGBD és la incorporació de
  nous tipus de dades complexos. És important estudiar com es modifica
  el model de dades d'un SGBD quan s'afegeix un nou tipus de dades
  complex.  Les sèries temporals es poden d'entendre com a tipus
  complex ja que presenten diferents propietats característiques i
  necessiten operadors addicionals.  

\item Disseny d'un model de SGBD per a les sèries temporals. D'aquesta
  manera els SGBD podran tractar dades amb instants de temps que
  mostrin l'evolució de variables en funció del temps. El model
  consisteix en la definició de l'estructura de les sèries temporals i
  les operacions bàsiques que necessiten.

  L'assoliment d'aquest objectiu té tres parts:

  \begin{enumerate}
  \item Disseny d'un model per a la gestió bàsica de les sèries
    temporals, el qual anomenem model de SGBD per a sèries temporals
    (SGST).  L'estructura d'aquest model és similar a l'utilitzat en
    els intervals
    temporals \parencite{date02:_tempor_data_relat_model}.  Prenent
    com a base el model de SGST, el qual és un model general per a les
    sèries temporals, s'hi poden incloure altres models per a
    propietats més específiques de les sèries temporals.

  \item Disseny d'un model específic en base del model de
    SGST. Concretament es dissenya un model pels SGST multiresolució
    (SGSTM).  En el model de SGSTM s'hi poden incloure propietats de
    les sèries temporals relacionades amb la resolució que s'han
    observat en les aplicacions pràctiques de les sèries temporals:
    regularització, canvis de resolució mitjançant agregacions,
    reconstrucció de forats, etc.
 
  \item Avaluació de diferents estructures de SGSTM. El model de SGSTM
    s'hi poden fer modificacions o simplificacions per tal
    d'aconseguir diferents estructures.  Per exemple bases de dades
    multiresolució que comparteixin informació, que treballin amb flux
    de dades (\emph{data stream}) o bé que s'especialitzin per a un
    tipus determinat de sèries temporals.

  \end{enumerate}

\item Implementació de referència dels models de SGST i SGSTM. Per una
  banda, aquesta implementació, a nivell acadèmic, ha de servir com a
  exemple per a futurs desenvolupaments de sistemes de gestió,
  acadèmics o productius. Per altra banda, ha de servir per a
  exemplificar-ne els seu funcionament amb unes dades de prova.

\item Implementació específica i reduïda del model per a una
  determinada aplicació de sèries temporals. Exemplificació de com una
  estructura de SGSTM pot ser implementada per a aconseguir una
  aplicació molt concreta.

\end{enumerate} 


El objectius 4.c i 6. són resultat d'una nova planificació com es
detalla més endavant a les tasques futures.






%%% Local Variables: 
%%% mode: latex
%%% TeX-master: "main"
%%% End: 
% LocalWords:  SGSTM multiresolució SGST


\chapter{Estat actual}
\label{cap:estat}


SGBD: sistema de gestió de bases de dades

SGBDR: SGBD relacional

SGST: SGBD per sèries temporals



%Estat de l'art

% * no n'hi ha d'específic del tema, potser el que més s'hi assembla són els SGST que hi ha (Cougar, RRDtool, ...)

% * Hi ha temes colaterals (monitoratge,anàlisis)

% * Temes para\l.lels que ens serveixen d'inspiració (SGBD relacionals)

%Cal introduir bé el forat de coneixement que hi ha en els SGST. Forat entre les sèries temporals i els SGBD.



%Capítol:

% * Sèries temporals
  
%   - mineria
%   - aplicacions
%   - monitoratge de sèries temporals i problemes
%      * censura
%      * mostreig

%   - sgst: 
%       ficar aquí els sgbd per sèries temporals i més endavant ja es parlarà dels sgbd en general i com modelar-los i implementar-los.


% * SGBD
%  - model relacional
%  - implementacions
%  - temporal data


  % * Sèries temporals (històrics, predicció, diagnosis, prognosis, etc.)
  % * Mostreig: docs quan període de mostreig no regular
  % * Bases de dades (docs d'emmagatzematge quan la memòria és finita, docs quan període de mostreig no és regular, altres sistemes semblants (comercials,prototips))




% El capítol comença resumint l'estat de les sèries temporals en aquest camp de mineria; és a dir d'emmagatzematge i tractament. A continuació es llisten algunes aplicacions informàtiques que han implementat models de la mineria de sèries temporals. Finalment, es descriu l'estat actual de l'aplicació RRDtool, la qual també es classifica en aquest camp.

% This paper focuses on Data Base Management Systems (DBMS) that store
% and treat data as time series.   Other DBMS are not adequate for these cases as they do not have enough facilities to manage and retrieve time series
% information \parencite{schmidt95}.

% DBMS are based from formal models that define the objects and
% operations of the abstract machine to which users interact, such is
% the relational model \parencite{date}. TSMS lack a consolidated formal
% model, although special properties and requirements for a TSMS
% have been proposed \parencite{dreyer94}.








%%% Local Variables: 
%%% mode: latex
%%% TeX-master: "main"
%%% End: 

% LocalWords:  monitoratge


\subsection*{Planificació del treball}


Per tal d'assolir els objectius detallats a la secció
\ref{sec:objectius}, a continuació es proposen les tasques a
realitzar.  A la figura \ref{fig:pla:futur} es detalla el pla de
treball amb temps estimat.





\begin{enumerate}


\item Estudi d'aplicacions de les sèries temporals. Per a assolir
  l'objectiu~1 estudiarem recerca actual de sèries temporals.
  Consultarem articles i llibres que tinguin les sèries temporals com
  a temàtica principal. També cercarem l'existència de programari que
  tingui en els seus objectius el tractament de sèries temporals.

\item Estudi del model relacional. Per a l'objectiu~2 estudiarem el
  model relacional com a referent pels models de SGBD. Ens basarem en
  l'estudi dels llibres de
  \textcite{date:introduction,date06,date:dictionary}. Usarem
  \emph{rel} \parencite{rel} com a implementació de referència ja que
  incorpora el llenguatge \emph{Tutorial D}, el qual és utilitzat per
  Date en els seus exemples.

\item Estudi de la gestió d'intervals temporals. Per a l'objectiu~3
  estudiarem la recerca que ha conduit a incorporar els interval
  temporals en els SGBD per a gestionar històrics. Ens basarem en
  l'estudi que relaciona el model relacional amb els intervals
  temporals de \textcite{date02:_tempor_data_relat_model}.

\item Disseny d'un model de SGBD per sèries temporals. Per a la
  primera part de l'objectiu 4 dissenyarem un model per als SGBD de
  sèries temporals formalitzat amb expressions algebraiques. A partir
  de l'estudi de l'estat de l'art de les sèries temporals, observarem
  les propietats interessants de ser modelitzades i les operacions que
  precisen els SGBD per sèries temporals.

\item Disseny d'un model de SGBD multiresolució per sèries
  temporals. Per a la segona part de l'objectiu 4 dissenyarem un model
  que contempli la multiresolució de les sèries temporals. Aquest
  model utilitzarà propietats del model anterior per les sèries
  temporals. La mulitresolució té l'objectiu d'emmagatzemar les sèries
  temporals de forma compacta, així es preveu que alguns conceptes de la recerca
  en \emph{data streams} poden prendre-hi sentit.

\item Implementació de referència. Per a l'objectiu~5 s'implementaran
  els models anteriors utilitzant un llenguatge de programació adequat
  per a models, com per exemple \texttt{Python} o \texttt{Prolog}.  Es
  prioritzarà la implementació correcte del model enfront a una
  implementació que contempli un bon rendiment. 

\item Experimentació amb dades. Per a complementar l'objectiu~5 es
  provarà la implementació amb dades experimentals per alguna
  aplicació concreta.

\end{enumerate}

\begin{figure}[tp]
\centering
\scalebox{0.8}{
\begin{gantt}[xunitlength=0.8cm,fontsize=\small,titlefontsize=\small]{15}{12}
  \begin{ganttitle}
    \numtitle{2012}{1}{2014}{4}
  \end{ganttitle}
  \begin{ganttitle}
    \numtitle{1}{1}{4}{1}
    \numtitle{1}{1}{4}{1}
    \numtitle{1}{1}{4}{1}
  \end{ganttitle}

  \ganttmilestone{Proposta de tesi}{2}

  \ganttgroup{Estudi}{0}{4}
  \ganttbar{1. Aplicacions}{0}{2}
  \ganttbar{2. Model relacional}{0}{4}
  \ganttbar{3. Intervals temporals}{2}{2}

  \ganttgroup{Disseny models}{4}{3}
  \ganttbar{4. Sèries temporals}{4}{3}
  \ganttbar{5. Multiresolució}{5}{2}
  \ganttcon{7}{7}{7}{11}

  \ganttgroup{Experimentació}{7}{2}
  \ganttbar{6. Implementació}{7}{2}
  \ganttbar{7. Dades experimentals}{8}{1}

  \ganttbar{8. Redacció memòria}{8}{2}

  \ganttmilestonecon{Lectura de tesi}{10}

\end{gantt}
}
\caption{Planificació del treball}
\label{fig:pla:futur}
\end{figure}



\subsubsection*{Treball realitzat}


La tesi de màster \parencite{llusa11:tfm} va consistir en l'estudi de
\emph{RRDtool} \parencite{rrdtool}, un sistema de gestió de bases de
dades (SGBD) específic per a sèries temporals.
%
Fruit d'aquest estudi es va formalitzar el model de \emph{RRDtool} i es va
dissenyar i implementar un prototip software que responia a la
formalització.
%
Aquest treball va permetre concloure que:
\begin{itemize}
\item Els SGBD aplicats a sèries temporals tenen aspectes propis que
  els converteixen en objecte d'estudi de \emph{per se}.
\item El concepte de multiresolució és interessant en moltes
  aplicacions reals, especialment quan existeixen restriccions d'espai
  per emmagatzemar dades.
\item El model proposat per a \emph{RDDtool} és susceptible de ser millorat
  com a mínim en el següents aspectes:
  \begin{itemize}
  \item Generalització. El model que es va presentar estava fortament
    lligat a \emph{RDDtool}. Generalitzar el model de forma que encabeixi
    altres concepcions.
  \item Incorporació d'operacions. El model presentat únicament feia
    referència a les dades. Per completar el
    model cal també considerar les operacions.
  \item Contextualització. Cal interrelacionar i descriure el model el
    context d'altres models existents per a SGBD. Específicament cal
    comparar-lo amb el model relacional usat en els SGBD
    convencionals.
  \end{itemize}
\item És convenient estudiar la feina feta per altres autors en
  l'àmbit de l'emmagatzemat i gestió de dades provinents de sèries
  temporals.
\end{itemize}

Arrel de la feina anterior, s'ha treballat en els següents aspectes:
\begin{itemize}
\item S'ha realitzat una tasca de recerca bibliogràfica i estudi dels
  treballs existents en l'àmbit de la gestió i emmagatzemat de dades
  provinents de sèries temporals. El resultat d'aquesta tasca s'ha
  reflectit a la proposta de tesi \parencite{llusa12:ptd}.
\item S'ha estudiat en profunditat el model relacional. Atesa la
  preeminència d'aquest model en els SGBD actuals, s'ha considerat
  imprescindible tenir-ne un bon coneixement que permetés estudiar les
  seves deficiències per la gestió de sèries temporals i en quina
  forma poden ser superades.
\item S'està refent el model de SGBD per sèries temporals.  S'ha
  dividit el model en dues parts ben diferenciades:
  \begin{enumerate}
  \item La primera és el model general. Aquest defineix la gestió de
    sèries temporals enteses com a co\l.lecció de dades mesurades en
    diferents instants de temps. Es basa en els conceptes de temps,
    mesura i sèrie temporal.
  \item La segona és el model de multiresolució. Aquest model explica
    la forma d'emmagatzemar una sèrie temporal amb diferents
    resolucions temporals.
  \end{enumerate}
\end{itemize}





\subsection*{Mitjans}

La recerca es duu a terme amb el suport de la Universitat Politècnica
de Catalunya (UPC) mitjançant una beca FPU-UPC adscrita al departament
d'Enginyeria del Disseny i Programació de Sistemes Electrònics
(DiPSE).

No es preveu un ús de mitjans més enllà de l'accés als
recursos bibliogràfics i d'eines informàtiques de programació i
gestió de documentació.



%%% Local Variables: 
%%% mode: latex
%%% TeX-master: "main"
%%% End: 


%------- Bibliografia ------
\cleardoublepage
%\phantomsection\addcontentsline{toc}{chapter}{\bibname}
\pdfbookmark{\bibname}{bookmark:bibliografia}
\printbibliography
%----------------------------------------------


\appendix

\chapter{Direcció}


En tractar-se d'una recerca situada entre dos àmbits, l'anàlisi de les
sèries temporals i els sistemes de gestió de bases de dades, compta
respectivament amb la direcció de dos doctors d'aquest àmbits: la
Teresa Escobet Canal i el Sebastià Vila-Marta.

\section{Teresa Escobet Canal}

És professora del departament d'Enginyeria del Disseny i Programació de
Sistemes Electrònics de la Universitat Politècnica de Catalunya. 
És membre del Programa de Doctorat en Automàtica, Robòtica i Visió.



\section{Sebastià Vila-Marta}

És professor del departament d'Enginyeria del Disseny i Programació de
Sistemes Electrònics de la Universitat Politècnica de Catalunya.
A continuació s'incorpora el seu currículum.






%%% Local Variables: 
%%% mode: latex
%%% TeX-master: "main"
%%% End: 


\end{document}


%%%%%%%%%%%%%%%%%%%%%%%%%%%%%%%%%%%%%%%%%%%%%%%%%%%%%%%%%%%%%%%%%%%%%%%%%%  
% Model dels sistemes de gestió de bases de dades per sèries temporals.
%
% Copyright (C) 2011-2012 Aleix Llusà Serra.
% 
% This LaTeX document is free software: you can redistribute it and/or
% modify it under the terms of the GNU General Public License as
% published by the Free Software Foundation, either version 3 of the
% License, or (at your option) any later version.
%
% This document is distributed in the hope that it will be useful, but
% WITHOUT ANY WARRANTY; without even the implied warranty of
% MERCHANTABILITY or FITNESS FOR A PARTICULAR PURPOSE. See the GNU
% General Public License for more details.
%
% You should have received a copy of the GNU General Public License
% along with this document. If not, see <http://www.gnu.org/licenses/>.
%
%
% Aleix Llusà Serra
% Departament de Disseny i Programació de Sistemes Electrònics de la Universitat Politècnica de Catalunya (DiPSE-UPC)
% Escola Politècnica Superior d'Enginyeria de Manresa (EPSEM)
% Av. de les Bases de Manresa, 61-73
% 08242 Manresa (Barcelona)
% PAÏSOS CATALANS 
%
% aleix (a) dipse.upc.edu
% 
% El codi font LaTeX del document es troba a 
% <http://escriny.epsem.upc.edu/projects/rrb/>
%%%%%%%%%%%%%%%%%%%%%%%%%%%%%%%%%%%%%%%%%%%%%%%%%%%%%%%%%%%%%%%%%%%%%%%%%%  


\subsection{Sèries temporals}
\begin{frame}{Estat sèries temporals}

  Sèries temporals:

  \begin{itemize}

  \item Conjunt de valors cadascun dels quals té associat un instant
    de temps.

  \item Emmarcades en l'àmbit de dades temporals, juntament amb dades
    bitemporals \parencite{assfalg08:thesis}.

  \end{itemize}
  

  Aplicacions:
  \begin{itemize}

  \item Ús generalitzat per a l'anàlisi i la comprensió del
    comportament temporal de variables. Tasques relacionades amb
    validació de dades, diagnòstic, prognosis, etc.

  \item Algunes aplicacions actuals: avaluació de la degradació de
    components \parencite{yu11}, anàlisi de l'estat dels sensors d'un
    vaixell \parencite{palmer07}, validació i reconstrucció de dades
    en xarxes de distribució d'aigua \parencite{quevedo10},
    optimització de la planificació semafòrica \parencite{last11},
    transmissió d'informació en xarxes de
    sensors \parencite{jainagrawal05}.


  \end{itemize}

\end{frame}


%\subsection{Adquisició}
\begin{frame}{Estat sèries temporals. Adquisició}

  Adquisició de dades mitjançant monitoratge:

  \begin{itemize}

  \item Sèries temporals com a resultat d'una adquisició de
    dades. Solucions de monitoratge per a adquirir-les.

  \item Interès actual en l'ús eficient de recursos en les xarxes de
    sensors: 
    \begin{itemize}
    \item Transmissió amb agregacions i
      aproximacions \parencite{deligiannakis07}
    \item Resolució distribuïda de consultes \parencite{bonnet01}.
    \end{itemize}

  \end{itemize}
  
    Problemes en el monitoratge:
    \begin{enumerate}

    \item Gestió d'una quantitat enorme de dades. 

    \item Necessitat de censurar les dades: validar-les i reconstruir-les.

    \item Regularitat del període de mostreig.

    \end{enumerate}
  

  Cal anàlisis de sèries temporals recolzat en els SGBD.



\end{frame}



%\subsection{Anàlisis}
\begin{frame}{Estat sèries temporals. Anàlisis}

  Anàlisis de sèries temporals:

  \begin{itemize}

  \item Aplicació de metodologies i d'algoritmes per extreure
    característiques o obtenir models de les dades.

  \item Tècniques recollides en la  mineria de sèries temporals.

  \item Recerca actual per reduir la mida de les sèries temporals i el
    temps de processat \parencite{fu11}. 

  \end{itemize}
  

  Tasques en la mineria de sèries temporals:

  \begin{itemize}

  \item Quatre tasques principals: indexat, agrupament, classificació i segmentació \parencite{keogh02}.

  \item Tasca de representació és un pas comú que s'aprofita per
    reduir la mida.  Les representacions a trossos constants són
    tècniques eficients en el càlcul \parencite{keogh00}.

  \end{itemize}

\end{frame}



%\subsection{Emmagatzematge i gestió}
\begin{frame}{Estat sèries temporals. Emmagatzematge i gestió}

  Gestió de les sèries temporals:

  \begin{itemize}

  \item Ús dels SGBD per emmagatzemar sèries temporals i consultar-ne
    informació.

  \item SGBD específics per a sèries temporals
    (SGST) amb operacions de temps adequades per
    a les dades \parencite{dreyer94}.

  \end{itemize}
  
    Cal gestionar l'arribada seqüencial i contínua de les dades.

    \begin{itemize}

    \item  Seqüències \parencite{seshadri95}

    \item \emph{Data streams} \parencite{babcock02}

    \item Vectors i matrius \parencite{stonebraker09:scidb,zhang11}

    \end{itemize}


  Alguns sistemes amb propietats de SGST: 
    \emph{Cougar} \parencite{cougar},
    \emph{iSAX} \parencite{keogh10:isax},
    \emph{RRDtool} \parencite{rrdtool},
    \emph{SciDB} \parencite{stonebraker09:scidb} o
    \emph{SciQL} \parencite{zhang11}.




\end{frame}




%%% Local Variables: 
%%% mode: latex
%%% TeX-master: "presentacio"
%%% End: 

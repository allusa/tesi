\subsection[SGBD]{Sistemes de gestió de bases de dades}
\begin{frame}{Estat sistemes gestió de bases de dades}

  Els SGBD són els sistemes informàtics que tracten amb bases de
  dades. 

  \begin{itemize}  

  \item Emmagatzemen i consulten tot tipus de dades.

  \item Es poden descriure amb models matemàtics.

  \item El model relacional és un
    referent \parencite{date:introduction}.

  \end{itemize}
  
Sistemes i models actuals:

  \begin{itemize}  

  \item Els SGBD relacionals han tingut hegemonia sota el llenguatge
    SQL, però actualment el corrent \emph{NoSQL} proposa nous SGBD amb
    més bon rendiment \parencite{stonebraker10}.


  \item El corrent \emph{NoSQL} també proposa nous models però no
    estan prou consolidats per avaluar-los. El model relacional
    proposa millores i un nou llenguatge, \emph{Tutorial
      D} \parencite{date06,date:thethirdmanifesto}.

  \end{itemize}

  Extensió mitjançant tipus de dades:

  \begin{itemize}

  \item Els tipus de dades complexos necessiten un estudi propi.

  \item Exemple recent: els intervals temporals.

  \end{itemize}
  
\end{frame}


%\subsection{Intervals temporals}
\begin{frame}{Estat sistemes gestió bases de dades. Intervals temporals}

  SGBD per a tractar dades amb històrics:

  \begin{itemize}  

  \item Emmagatzematge i consulta de dades històriques mitjançant
    intervals temporals, anomenats de forma general com a dades
    temporals.

  \item Extensió del model relacional amb intervals temporals i
    operacions \parencite{date02:_tempor_data_relat_model}.

  \item Històrics basats en temps vàlid i de transacció: dades
    bitemporals.

  \end{itemize}
  
  Les sèries temporals també són dades temporals, però:

  \begin{itemize}  

  \item SGBD per dades bitemporals no adequats per sèries
    temporals \parencite{schmidt95}.

  \end{itemize}
    


\end{frame}



\begin{frame}{Sèries temporals i dades bitemporals}



 Dades bitemporals: històrics amb intervals temporals

\begin{columns}
\column{6cm}

\begin{center}
  \begin{tikzpicture}[scale=0.5]
      \begin{axis}[
          axis y line=left,  axis x line=bottom,
          yticklabels={},
          xticklabels={,,,,,temps vàlid}]

          \addplot[only marks,mark=|] coordinates {
            (0,0) (10,0) (0,10)
            };


          \node[right] at (axis cs:2,5) {\framebox(28,10){}};
          \node[right] at (axis cs:3.5,7) {\framebox(60,10){}};
          \node[right] at (axis cs:7,6) {\framebox(30,10){}};
          \node[above] at (axis cs:9,3) {\framebox(10,10){}};


          \node[right] at (axis cs:2,2) {\color{blue}{\framebox(50,10){}}};
          \node[right] at (axis cs:4.7,4) {\color{blue}{\framebox(50,10){}}};
          \node[above] at (axis cs:9,3) {\color{blue}{\framebox(10,8){}}};

      \end{axis}
  \end{tikzpicture}
\end{center}

\column{6cm}
\begin{center}
  \begin{tikzpicture}[scale=0.5]
      \begin{axis}[
          axis y line=left,  axis x line=bottom,
          yticklabels={},
          xticklabels={,,,,,temps transacció}]

          \addplot[only marks,mark=|] coordinates {
            (0,0) (10,0) (0,10)
            };


          \node[right] at (axis cs:2,2) {\color{blue}{\framebox(50,10){}}};
          \node[right] at (axis cs:4.7,4) {\framebox(80,10){}};

      \end{axis}
  \end{tikzpicture}

\end{center}

\end{columns}


   Sèries temporals: anàlisi d'observacions seqüencials amb
    instants temporals

\begin{center}
  \begin{tikzpicture}[scale=0.5]
    \begin{axis}[
          axis y line=left,  axis x line=bottom,
          xticklabels={,,,,,,,temps},
          yticklabels={}]

          \addplot[only marks,mark=|] coordinates {
            (0,0)
            };

    \addplot[only marks,mark=*,blue] coordinates {
        (10,1)
        (20,3)
        (25,4)
        (35,5)
        (42,7)
        (45,8)
        (50,10)
        (55,6)
        (60,3)
        (70,4)
    };

    \end{axis}
  \end{tikzpicture}
 \end{center}


 

\end{frame}



%%% Local Variables: 
%%% mode: latex
%%% TeX-master: "presentacio"
%%% End: 

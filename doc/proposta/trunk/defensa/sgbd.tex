\subsection[SGBD]{Sistemes de gestió de bases de dades}
\begin{frame}{Estat sistemes gestió de bases de dades}

  Els SGBD són els sistemes informàtics que tracten amb bases de
  dades. 

  \begin{itemize}  

  \item Emmagatzemen i consulten tot tipus de dades.

  \item Es poden descriure amb teories matemàtiques: models.

  \item El model relacional és un
    referent \parencite{date:introduction}.

  \end{itemize}
  
Sistemes i models actuals:

  \begin{itemize}  

  \item Els SGBD relacionals han tingut hegemonia sota el llenguatge
    SQL, però actualment el corrent \emph{NoSQL} proposa nous SGBD amb
    més bon rendiment \parencite{stonebraker10}.


  \item El corrent \emph{NoSQL} també proposa nous models però no
    estan prou consolidats per avaluar-los. El model relacional
    proposa millores i un nou llenguatge, \emph{Tutorial
      D} \parencite{date06,date:thethirdmanifesto}.

  \end{itemize}

  Extensió mitjançant tipus de dades:

  \begin{itemize}

  \item Els tipus de dades complexos necessiten un estudi propi.

  \item Exemple recent: els intervals temporals.

  \end{itemize}
  
\end{frame}


%\subsection{Intervals temporals}
\begin{frame}{Estat sistemes gestió bases de dades. Intervals temporals}

  SGBD per a tractar dades amb històrics:

  \begin{itemize}  

  \item Emmagatzematge i consulta de dades històriques mitjançant
    intervals temporals, anomenats de forma general com a dades
    temporals.

  \item Extensió del model relacional amb intervals temporals i
    operacions \parencite{date02:_tempor_data_relat_model}.

  \item Històrics basats en temps vàlid i de transacció: dades
    bitemporals.

  \end{itemize}
  
  Les sèries temporals també són dades temporals, però:

  \begin{itemize}  

  \item SGBD per dades bitemporals no adequats per sèries
    temporals \parencite{schmidt95}.

  \item Dades bitemporals: històrics amb intervals temporals.

  \item Sèries temporals: anàlisi d'observacions seqüencials amb
    instants temporals.

  \end{itemize}
    


\end{frame}



%%% Local Variables: 
%%% mode: latex
%%% TeX-master: "presentacio"
%%% End: 

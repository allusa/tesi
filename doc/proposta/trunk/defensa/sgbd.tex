\begin{frame}{Estat sistemes gestió de bases de dades}

  \begin{itemize}  

    
  \item Implementacions: corrent NoSQL concentrat en millorar el
    rendiment dels SGBD \parencite{stonebraker10}. Cada aplicació té
    la implementació de SGBD més adequada.


  \item Models: conceptes teòrics matemàtics dels SGBD. Model
    relacional com a màxim exponent, actualment \emph{Third
      Manifesto}, especialment amb \emph{Tutorial D}. Altres models
    encara no prou potents per avaluar-los.


  \end{itemize}
  
\end{frame}



\begin{frame}{Estat sistemes gestió bases de dades. Intervals temporals}

  \begin{itemize}  

  \item Emmagatzematge i consulta de dades històriques mitjançant
    intervals temporals, anomenats de forma general com a dades
    temporals.

  \item Extensió del model relacional amb intervals temporals i
    operacions \parencite{date02:_tempor_data_relat_model}.

  \item Històrics basats en temps vàlid i de transacció: dades
    bitemporals.

  \item SGBD per dades bitemporals no adequats per sèries
    temporals \parencite{schmidt95}.
  \end{itemize}
    

  Dades bitemporals: històrics amb intervals temporals\\
  Sèries temporals: anàlisi d'observacions seqüencials amb instants
  temporals

\end{frame}



%%% Local Variables: 
%%% mode: latex
%%% TeX-master: "presentacio"
%%% End: 

\chapter{Planificació}

\section{Treball realitzat}


La tesi de màster en Automàtica i Robòtica ''Estudi i modelització
dels SGBD Round Robin pel tractament de sèries temporals'' (2011) va
consistir en l'estudi d'RRDTool, un sistema de gestió de bases de
dades (SGBD) específic per a sèries temporals.
%
Fruit d'aquest estudi es va formalitzar el model de RRDTool i es va
dissenyar i implementar un prototip software que responia a la
formalització.
%
Aquest treball va permetre concloure que:
\begin{itemize}
\item Els SGBD's aplicats a sèries temporals tenen aspectes propis que
  els converteixen en objecte d'estudi de \emph{per se}.
\item El concepte de multiresolució és interessant en moltes
  aplicacions reals, especialment quan existeixen restriccions d'espai
  per emmagatzemar dades.
\item El model proposat per a RDDTool és susceptible de ser millorat
  com a mínim en el següents aspectes:
  \begin{itemize}
  \item Generalització. El model que es va presentar estava fortament
    lligat a RDDTool. Generalitzar el model de forma que encabeixi
    altres concepcions permetria enriquir-lo.
  \item Incorporació d'operacions. El model presentat únicament feia
    referència a les dades. Era un model de dades. Per completar el
    model cal també considerar les operacions sobre les dades.
  \item Contextualització. Cal interrelacionar i descriure el model el
    context d'altres models existents per a SGBD's. Específicament cal
    comparar-lo amb el model relacional usat en els SGBD's
    convencionals.
  \end{itemize}
\item És convenient estudiar la feina feta per altres autors en
  l'àmbit de l'emmagatzemat i gestió de dades provinents de sèries
  temporals.
\end{itemize}

Arrel de la feina anterior, s'ha treballat en els següents aspectes:
\begin{itemize}
\item S'ha realitzat una tasca de recerca bibliogràfica i estudi dels
  treballs existents en l'àmbit de la gestió i emmagatzemat de dades
  provinents de sèries temporals. El resultat d'aquesta tasca s'ha
  reflectit a l'apartat~\ref{} d'aquest mateix document.
\item S'ha estudiat en profunditat el model relacional. Atesa la
  preeminència d'aquest model en els SGBD's actuals, s'ha considerat
  imprescindible tenir-ne un bon coneixement que permetés estudiar les
  seves deficiències per la gestió de sèries temporals i en quina
  forma poden ser superades.
\item S'està refent el model de SGBD per sèries temporals.  S'ha
  dividit el model en dues parts ben diferenciades:
  \begin{enumerate}
  \item La primera és el model general. Aquest defineix la gestió de
    sèries temporals enteses com a co\l.lecció de dades mesurades en
    diferents instants de temps. Es basa en els conceptes de temps,
    mesura i sèrie temporal.
  \item La segona és el model de multiresolució. Aquest model explica
    la forma d'emmagatzemar una sèrie temporal amb diferents
    resolucions temporals.
  \end{enumerate}
\end{itemize}


\section{Treball futur}


A la figura \ref{fig:pla:futur} es detalla el pla de treball amb temps
estimat per tal d'assolir els objectius detallats a la secció
\ref{sec:objectius}.

\begin{figure}[tp]
\begin{gantt}[xunitlength=0.8cm]{15}{12}
  \begin{ganttitle}
    \numtitle{2012}{1}{2014}{4}
  \end{ganttitle}
  \begin{ganttitle}
    \numtitle{1}{1}{4}{1}
    \numtitle{1}{1}{4}{1}
    \numtitle{1}{1}{4}{1}
  \end{ganttitle}

  \ganttmilestone{Proposta de tesi}{2}

  \ganttgroup{Estudi}{0}{4}
  \ganttbar{Aplicacions}{0}{2}
  \ganttbar{Model relacional}{0}{4}
  \ganttbar{Intervals temporals}{2}{2}

  \ganttgroup{Disseny}{4}{3}
  \ganttbar{Model sèries temporals}{4}{3}
  \ganttbar{Model multiresolució}{5}{2}
  \ganttcon{7}{7}{7}{11}

  \ganttgroup{Experimentació}{7}{2}
  \ganttbar{Implementació}{7}{2}
  \ganttbar{Dades experimentals}{8}{1}

  \ganttbar{Redacció memòria}{8}{2}

  \ganttmilestonecon{Lectura de tesi}{10}

\end{gantt}
\caption{Planificació del treball}
\label{fig:pla:futur}
\end{figure}


\section{Mitjans}

La recerca es duu a terme amb el suport de la Universitat Politècnica
de Catalunya (UPC) mitjançant una beca FPU-UPC adscrita al departament
d'Enginyeria del Disseny i Programació de Sistemes Electrònics
(DiPSE).


No es preveu un ús de mitjans més enllà de l'accés als
recursos bibliogràfics i d'eines informàtiques de programació i
gestió de documentació.










%%% Local Variables: 
%%% mode: latex
%%% TeX-master: "main"
%%% End: 
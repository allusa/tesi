\chapter{Planificació}

\section{Treball realitzat}


Inicialment, en la tesi final de màster en Automàtica i Robòtica
''Estudi i modelització dels SGBD Round Robin pel tractament de sèries
temporals'' (2011) es va estudiar profundament un sistema de gestió de
bases de dades (SGBD) específic per a sèries temporals, RRDtool, i com
a resultat de la seva formalització es van observar propietats
interessants de les sèries temporals: naturalesa, representació,
multiresolució i regularització.

Posteriorment, a partir de la formalització anterior, s'ha treballat
en el disseny d'un model de SGBD per sèries temporals.  S'ha dividit
el model en dues parts. Una per al model general de les sèries
temporals enteses com a co\l.lecció de dades mesurades en diferents
instants de temps; es basa en temps, mesures i sèries temporals. Una
altra pel model de multiresolució que emmagatzema la sèrie temporal
amb diferents resolucions i funcions d'interpolació; es basa en discs
resolucions que tenen un buffer on interpolen les mesures i un disc on
consoliden el resultat.


Finalment, s'ha situat l'àmbit d'estudi entre l'anàlisi de sèries
temporals i els SGBD. Així, s'ha estudiat l'estat actual d'aquests dos
temes, el qual es presenta en aquesta proposta de tesi.



\section{Treball futur}


A la figura \ref{fig:pla:futur} es detalla el pla de treball amb temps
estimat per tal d'assolir els objectius detallats a la secció
\ref{sec:objectius}.

\begin{figure}[tp]
\begin{gantt}[xunitlength=0.8cm]{15}{12}
  \begin{ganttitle}
    \numtitle{2012}{1}{2014}{4}
  \end{ganttitle}
  \begin{ganttitle}
    \numtitle{1}{1}{4}{1}
    \numtitle{1}{1}{4}{1}
    \numtitle{1}{1}{4}{1}
  \end{ganttitle}

  \ganttmilestone{Proposta de tesi}{2}

  \ganttgroup{Estudi}{0}{4}
  \ganttbar{Aplicacions}{0}{2}
  \ganttbar{Model relacional}{0}{4}
  \ganttbar{Intervals temporals}{2}{2}

  \ganttgroup{Disseny}{4}{3}
  \ganttbar{Model sèries temporals}{4}{3}
  \ganttbar{Model multiresolució}{5}{2}
  \ganttcon{7}{7}{7}{11}

  \ganttgroup{Experimentació}{7}{2}
  \ganttbar{Implementació}{7}{2}
  \ganttbar{Dades experimentals}{8}{1}

  \ganttbar{Redacció memòria}{8}{2}

  \ganttmilestonecon{Lectura de tesi}{10}

\end{gantt}
\caption{Planificació del treball}
\label{fig:pla:futur}
\end{figure}


\section{Mitjans}

La recerca es duu a terme amb el suport de la Universitat Politècnica
de Catalunya (UPC) mitjançant una beca FPU-UPC adscrita al departament
d'Enginyeria del Disseny i Programació de Sistemes Electrònics
(DiPSE).


No es preveu un ús de mitjans més enllà de l'accés als
recursos bibliogràfics i d'eines informàtiques de programació i
gestió de documentació.










%%% Local Variables: 
%%% mode: latex
%%% TeX-master: "main"
%%% End: 
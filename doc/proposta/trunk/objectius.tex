\chapter{Introducció}


Aquesta recerca s'estructura al voltant dels sistemes d'emmagatzematge
i tractament de dades com a sèries temporals.  Concretament se centra
en els sistemes de gestió de bases de dades (SGBD) que s'ocupen de
sèries temporals (SGST). Els SGST han de tenir funcionalitats
adequades per gestionar i explotar correctament la informació de les
sèries temporals.

La recerca en anàlisi de sèries temporals ha augmentat en la darrera
dècada, tal com explica \textcite{fu11}. És un camp amb interès
actual, sobretot pel que fa a processar grans volums de dades amb bon
rendiment, tant des de la vessant de temps d'execució com,
modernament, de la vessant de consum d'energia. Hi ha multitud de
metodologies i algoritmes que proposen solucions a aquests problemes
Tot i així, la recerca en aquests temes segueix avançant a la recerca
de noves solucions. Aquestes metodologies de sèries temporals són
bones candidates per a ésser resoltes de forma ordenada i integrada
mitjançant SGBD. Per a aquest objectiu és necessari disposar d'un
model per a sèries temporals a on es reculli de forma general el
problema de tractament de sèries temporals, aleshores molts d'aquests
estudis de la recerca de sèries temporals es podran aplicar en la
implementació de SGBD que segueixin el model de SGST.


Recentment s'ha observat que hi ha un forat de coneixement entre els
SGBD i les aplicacions de les sèries temporals, referides a la
literatura com a científiques \parencite{stonebraker09:scidb,zhang11}.
Hi alguns sistemes que es poden considerar com a SGST, com per exemple
RRDTool o Cougar, però no s'ha definit clarament un model de SGST.
Algunes propostes que mostren la necessitat d'estudiar conceptes de
model són les definicions de \textcite{dreyer94} d'estructures
bàsiques que han de tenir els SGST, els estudis de \textcite{bonnet01}
per a xarxes de sensors o els exemples de consultes de
\textcite{zhang11} per algunes de les propietats de les sèries
temporals.

En els SGBD es contempla l'estudi formal dels seus conceptes
mitjançant models basats en teories matemàtiques, del qual n'és una
referència el model relacional \parencite{date:introduction}.  Dins
del model relacional hi ha hagut un estudi profund per als intervals
temporals, referits com a dades
temporals \parencite{date02:_tempor_data_relat_model}, que resol el
problema dels històrics temporals en els SGBD. Les sèries temporals
necessiten un estudi similar ja que no poden ser tractades com a
intervals temporals per pertànyer a una categoria diferent de dades
temporals \parencite{assfalg08:thesis,schmidt95}.













\section{Estructura del document}

Aquest document és una proposta de recerca per a definir un model de
sistema de gestió per a sèries temporals. En una primera part, que
segueix a aquesta introducció, es presenten els objectius i les
justificacions de la recerca. En una segona part s'estudia el context
de la recerca i l'estat actual en els àmbits de les sèries temporals i
els SGBD. En una tercera part es resumeix el treball dut a terme fins
a l'actualitat i la planificació del treball futur per a assolir els
objectius proposats.




\section{Objectius i contribucions esperades}


Aquesta recerca té per objectiu l'estudi d'un model de SGBD per a
sèries temporals que en descrigui l'estructura i el comportament. A
tal efecte, es divideix en els següents objectius més concrets
d'estudi que han d'aportar solució al forat de coneixement entre
l'àmbit de SGBD i el de sèries temporals.


\begin{enumerate}

\item Observació general de les aplicacions de les sèries temporals
  per a poder trobar propietats i problemes comuns susceptibles de ser
  modelats.


\item Comprensió de l'estudi de la modelització de SGBD. Segons es
  desprèn de la formalització de \textcite{date:introduction} la
  referència principal és el model relacional, el qual es fonamenta en
  dos conceptes: relacions i tipus.

\item Estudi de la modelització de nous tipus complexos dins dels
  models de SGBD.  Les sèries temporals es poden d'entendre com a
  tipus complex ja que presenten diferents propietats i necessiten
  operadors addicionals.  Els SGBD permeten que els usuaris defineixin
  nous tipus \parencite{stonebraker86} però no hi ha un estudi teòric
  dels tipus als SGBD: \textcite{date:introduction} descriu abastament
  les relacions però no els tipus. Els tipus s'han d'estudiar i
  modelar per a poder-los considerar com a tals, oimés els tipus
  complexos ja que requereixen un estudi més complet i possiblement
  s'hagin de modelar com un propi SGBD. Una referència d'estudi és el
  cas dels intervals
  temporals \parencite{date02:_tempor_data_relat_model}.



\item Proposta de model per a les sèries temporals per tal que puguin
  ser incloses en els SGBD. D'aquesta manera els SGBD podran tractar
  dades amb instants de temps que representin l'evolució de variables
  en funció del temps. El model consisteix en la definició de
  l'estructura de les sèries temporals i les operacions bàsiques que
  necessiten.

  El model es presenta en dues parts:

  \begin{enumerate}
  \item Proposta d'un model per a les sèries temporals similar a
    l'estudi fet pels intervals
    temporals \parencite{date02:_tempor_data_relat_model}, ho anomenem
    model de SGBD per a sèries temporals (SGST). A sobre del model de
    SGST, el qual és un model general per a les sèries temporals, s'hi
    poden proposar altres models per a propietats més específiques de
    les sèries temporals.

  \item Proposta d'exemple de model específic a sobre del model de
    SGST. Concretament es proposa un model pels SGST multiresolució
    (SGSTM).  En el model de SGSTM s'hi inclouen propietats de les
    sèries temporals relacionades amb la resolució que s'han observat
    en les seves aplicacions: regularització, canvis de resolució
    mitjançant agregacions, farciment de forats, etc.  
    % (nota: a més aquesta part del model és la més sensible a ser
    % implementada com a data streams)
  \end{enumerate}

\item Implementació de referència del model SGST i del model
  SGSTM. Per una banda, aquesta implementació, a nivell acadèmic, ha
  de servir com a exemple per a futurs desenvolupaments de SGST,
  acadèmics o productius. Per altra banda, ha de servir per a
  exemplificar-ne els seu funcionament amb unes dades de prova.


%5. Possible exemplificació en algun cas pràctic?

\end{enumerate}






\section{Justificació}

Una de les grans mancances en l'àmbit de les sèries temporals és no
tenir un model que faci d'enllaç entre les diferents aplicacions i les
diferents implementacions, el que s'anomena com a forat de coneixement
entre els SGBD i les sèries
temporals \parencite{zhang11,stonebraker09:scidb}. Un model permetria,
entre d'altres, estudiar les propietats abstractes de les sèries
temporals i poder comparar entre diferents sistemes de gestió.



L'estudi d'un model de SGBD per a sèries temporals està motivat
principalment per dues bandes.


Per una banda, després d'una mirada general a algunes aplicacions de
les sèries temporals s'observa que hi ha molts algoritmes específics
per al seu tractament. A més, s'observa que en el tractament de les
sèries temporals es duen a terme un conjunt de tasques comunes: canvis
de resolució, farciment de forats, reducció del volum de dades, etc.
Tot i que hi ha sistemes dissenyats específicament per a tractar-hi,
com per exemple RRDtool \parencite{rrdtool} o
Cougar \parencite{fung02}, no s'ha identificat un model general per a
aquests problemes.

Per altra banda, en el model relacional de SGBD es troba una
formalització per als sistemes que tractin amb qualsevol tipus de
dades. Disposar d'un model matemàtic consolidat, com és el cas del
relacional, ha estat una fita important en l'àmbit dels SGBD.  No
obstant, el model relacional només presenta els conceptes bàsics dels
SGBD, quedant per resoldre la formalització dels tipus. Concretament
dins d'aquest àmbit destaca la publicació del model per a intervals
temporals \parencite{date02:_tempor_data_relat_model}, el qual
s'utilitza per a formalitzar els històrics en els SGBD.

En l'àmbit dels SGBD també s'hi inclou l'estudi d'implementacions que
obtinguin bon rendiment. Un cop s'ha formalitzat el model es poden
estudiar implementacions adequades per a cada aplicació. Per exemple
en el cas de les sèries temporals es pot aplicar la gestió mitjançant
\emph{data streams} \parencite{babcock02}.


El model relacional va marcar una fita que va situar els SGBD
relacionals com a preeminents, sobretot els que tenien en comú un
llenguatge anomenat SQL (\emph{Structured Query Language}). Tot i que
semblava que era un àmbit consolidat, recentment han aparegut corrents
crítics a aquesta supremacia, liderats pel corrent
NoSQL \parencite{edlich:nosql,stonebraker10}.  En aquest corrent han
aparegut SGBD amb més bon rendiment que els SGBD SQL, però pel que fa
a models hi ha diversitat: alguns semblen reduccions del model
relacional, \textcite[cap.~14,27]{date06} comenta els arbres i els
objectes, altres aporten conceptes nous però encara són joves per
observar-ne la potència. En el corrent crític als SGBD SQL també hi
destaquen \textcite{date:thethirdmanifesto} que proposen conceptes i
llenguatges purament relacionals.




Així doncs, tenint en compte que segons Date el model relacional és
complet, consolidat i que no hi ha cap de tant potent, la modelització
de SGST s'ha de definir propera als conceptes relacionals, tal com
s'ha efectuat pel cas dels intervals
temporals \parencite{date02:_tempor_data_relat_model}. Així els SGST
podran aprofitar els avantatges i l'experiència del model relacional,
així com també podran aprofitar els estudis d'implementacions
eficients, com per exemple els \emph{data streams} per a dades amb
naturalesa de seqüència \parencite{bai05}.

El model relacional contempla la seva extensió mitjançant nous tipus
de dades, tot i que el considera independent i només en defineix la
seva necessitat. Per a tipus de dades senzills és suficient amb
definir una possible representació amb restriccions, per exemple
definir telèfons com a subconjunts dels enters. Ara bé, per a tipus de
dades complexos cal estudiar-los prèviament i modelar-los amb les
mateixes eines matemàtiques amb les que es modelen els SGBD.  En els
SGBD podem trobar tipus complexos que han rebut una gran atenció, com
per exemple els històrics mitjançant intervals temporals o els
sistemes d'informació geogràfica. Observant aquestes condicions, es
conclou que les sèries temporals necessiten un estudi similar.

%Ara és el torn de les sèries temporals.\todo{informal}










%%% Local Variables: 
%%% mode: latex
%%% TeX-master: "main"
%%% End: 
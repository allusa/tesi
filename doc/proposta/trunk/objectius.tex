\chapter{Introducció}




Aquesta recerca s'estructura al voltant dels sistemes
d'emmagatzemament i tractament de dades com a sèries temporals.
Concretament se centra en els sistemes de gestió de bases de dades
(SGBD) que s'ocupen de sèries temporals (SGST). Els SGST han de tenir  funcionalitats adequades per gestionar i
explotar correctament la informació de les sèries temporals.

Tal com diu Fu, la recerca en mineria de sèries temporals ha augmentat. Per tant és un camp amb interès actual, sobretot hi ha interès en processar grans volums de dades. Hi ha molts que estudien com resoldre això, per tant potser es pot observar que pel que fa a implementacions d'algoritmes que tinguin bon rendiment o facin bon ús d'energia hi ha molt escrit i s'obtenen bons resultats. Tot i així sembla que encara hi ha camp a recórrer: la recerca en aquests temes segueix avançant. No obstant, nosaltres ens centrarem en obtenir un model de SGBD, en el benentès que molts d'aquests estudis es podrien aplicar en la implementació de SGBD que seguissin el model.


Recentment s'ha observat que hi ha un forat de coneixement entre els SGBD i les aplicacions de les sèries temporals, referides a la literatura com a científiques [stonebraker09:scidb i zhang]


Estan apareixent SGST (RRDTool, Cougar, ...) però no hi ha definit clarament un model de SGST. 

Alguns intents han estat Dreyer amb unes primeres propostes de que han de complir els SGST, bonnet per a xarxes de sensors, zhang amb exemples de resolucions de consultes per algunes de les propietats de les sèries temporals

Ara bé, els SGST actuals es basen en models propis, p.ex. Cougar amb models de seqüències, però no s'han estudiat com a model de SGBD, del qual actualment només es considera com a representant el model relacional [Date]. 

Dins del model relacional hi ha hagut un estudi profund per als intervals temporals, referits com a dades temporals [date02], i això confereix una nova dimensió en la resolució del problema dels històrics temporals en els SGBD. Les sèries temporals necessiten un estudi similar, tot i que no poden ser tractades com a intervals temporals ja que són una categoria diferent de dades temporals [assfalg].


\section{Estructura del document}

Aquest document és una proposta de recerca per a definir un model de
sistema de gestió per a sèries temporals. En una primera part es
presenten els objectius i les justificacions de la recerca. En una
segona part s'estudia el context de la recerca i l'estat actual en els
àmbits de les sèries temporals i els SGBD. En una tercera part es
detalla el treball dut a terme fins a l'actualitat i la planificació
del treball futur per a assolir els objectius proposats.




\section{Objectius i contribucions esperades}


Aquesta recerca té per objectiu l'estudi d'un model de SGBD per a
sèries temporals que en descrigui l'estructura i el comportament. A
tal efecte, es divideix en els següents objectius més concrets
d'estudi que han d'aportar solució al forat de coneixement entre
l'àmbit de SGBD i el de sèries temporals.


\begin{enumerate}

\item Comprensió de l'estudi de la modelització de SGBD. Segons es
  desprèn de la formalització de Date cal entendre principalment el
  model relacional, el qual es fonamenta en dos conceptes: relacions i
  tipus.

\item Estudi de la modelització de nous tipus complexos dins dels
  models de SGBD.  Les sèries temporals s'han d'entendre com a tipus
  complex.  Els SGBD permeten que els usuaris defineixin nous
  tipus \parencite{stonebraker86} però no hi ha un estudi teòric dels
  tipus als SGBD: Date ens descriu abastament les relacions però no
  els tipus. Els tipus s'han d'estudiar i modelar per a poder-los
  considerar com a tals, oimés els tipus complexos ja que requereixen
  un estudi més complet i possiblement s'hagin de modelar com un propi
  SGBD. Una referència d'estudi és el cas dels intervals
  temporals \parencite{date02:_tempor_data_relat_model}.



\item Proposta de model per a les sèries temporals per tal que puguin
  ser incloses en els SGBD. D'aquesta manera els SGBD podran tractar
  dades amb instants de temps que representin l'evolució de variables
  en funció del temps. El model consisteix en la definició de
  l'estructura de les sèries temporals i les operacions bàsiques que
  necessiten.

  El model es presenta en dues parts:

  \begin{enumerate}
  \item Proposta d'un model per a les sèries temporals similar a
    l'estudi fet pels intervals
    temporals \parencite{date02:_tempor_data_relat_model}, ho anomenem
    model de SGBD per a sèries temporals (SGST). A sobre del model de
    SGST, el qual és un model general per a les sèries temporals, s'hi
    poden proposar altres models per a propietats més específiques de
    les sèries temporals.

  \item Proposta d'exemple de model específic a sobre del model de
    SGST. Concretament es proposa un model pels SGST multiresolució
    (SGSTM).  En el model de SGSTM s'hi inclouen propietats de les
    sèries temporals relacionades amb la resolució que s'han observat
    en les seves aplicacions: regularització, canvis de resolució
    mitjançant agregacions, farciment de forats, etc.  
    % (nota: a més aquesta part del model és la més sensible a ser
    % implementada com a data streams)
  \end{enumerate}

\item Implementació de referència del model SGST i del model
  SGSTM. Per una banda, aquesta implementació, a nivell acadèmic, ha
  de servir com a exemple per a futurs desenvolupaments de SGST,
  acadèmics o productius. Per altra banda, ha de servir per a
  exemplificar-ne els seu funcionament amb unes dades de prova.


%5. Possible exemplificació en algun cas pràctic?

\end{enumerate}






\section{Motivació i justificació}


Per una banda, després d'una mirada general a algunes aplicacions de
les sèries temporals s'observa que hi ha algoritmes específics per al
seu tractament. En el tractament de les sèries temporals es duen a
terme un conjunt de tasques comunes: canvis de resolució, farciment de
forats, delmat de dades, etc. No s'ha identificat un model per a
aquests problemes, tot i que hi ha sistemes dissenyats específicament
per a tractar-hi, com per exemple RRDtool \parencite{rrdtool} o
Cougar \parencite{fung02}.

Per altra banda, en el model relacional de SGBD es troba una
formalització per als sistemes que tractin amb qualsevol tipus de
dades. No obstant, el model relacional només presenta els conceptes
bàsics dels SGBD, quedant per resoldre la formalització dels
tipus. Concretament dins d'aquest àmbit destaca la publicació del
model per a intervals
temporals \parencite{date02:_tempor_data_relat_model}, el qual
s'utilitza per a formalitzar els històrics en els SGBD.

En els SGBD també hi ha l'estudi d'implementacions que obtinguin bon rendiment. Un cop s'ha formalitzat el model es poden estudiar implementacions adequades per a cada aplicació. Per exemple en el cas de les sèries temporals es pot aplicar la gestió mitjançant \emph{data streams} \parencite{babcock02}.




Un dels grans aturadors en l'àmbit de les sèries temporals és no tenir un model de gestió que faci d'enllaç entre les diferents aplicacions i les diferents implementacions, el que s'anomena com a forat de coneixement entre els SGBD i les sèries temporals \parencite{zhang11,stonebraker09:scidb}. Un model permetria estudiar les propietats abstractes de les sèries temporals i poder comparar entre diferents sistemes de gestió.













Així doncs, tenint en compte que segons Date el model relacional és complet, que no hi ha cap de tant potent i que l'ampliació de funcionalitat dels SGBDR s'ha de fer mitjançant la creació de nous tipus, el SGST hauria de contemplar aquestes idees. 

Primer s'hauria de veure que el cas de les sèries temporals no sigui com el cas de les dades temporals a on sí que s'ha necessitat estendre el model relacional?. --> Bé de fet potser aquesta era la idea inicial en el model relacional (que s'havia de modificar per a poder tenir històrics) però ara després de modelar les dades temporals es pot veure que el model relacional ho accepta com a tipus: és a dir les relacions que tinguin atributs de tipus interval temporal passen a ser relacions temporals.

Segon, s'hauria de considerar que el SGST són un nou tipus de dades en el model relacional i per tant el model relacional ja té tota la potència per una banda constituir SGBD i per altra banda definir i incorporar nous tipus.


Com es defineixen nous tipus complexos als SGBD?

En el cas que es descarti que el cas dels SGST presenta els mateixos problemes que les dades temporals i per tant els SGST han d'esdevenir un tipus de dades, cal preguntar-se com són els tipus de dades al model relacional.

Concretament el model relacional només defineix què es un tipus de dades però dóna llibertat a la seva creació. Això és un gran avantatge. En els cas de tipus de dades senzills es defineixen amb una bona estructura i ja està però què passa quan es vol definir un nou tipus de dades complex?

Cal recercar com s'han definit nous tipus de dades complexos. Els principals problemes es donen que el tipus és complex i forma una entitat de per sí. És a dir que definir un tipus sèrie temporal no és trivial. Aleshores, com cal procedir?



Sobre la necessitat de modelar els tipus.

Cal definir un model pel tipus que volem dissenyar.
De fet, volem dissenyar un SGST. És a dir, un tipus que conformi pròpiament un SGBD. Per tant, utilitzarem les mateixes eines que es fan servir per modelar els SGBDR per a poder modelar el nostre SGST. Com que s'hauran utilitzat les mateixes eines, el SGST podrà esdevenir perfectament un tipus de dades pels SGBDR.

En resum, per a definir nous tipus complexes cal modelar-los com a entitat pròpia, fent un símil amb el model relacional. Això no vol dir, però, que un cop modelats constitueixin de per sí un nou model per als SGBD, sinó que queden dins dels SGBDR. \todo{caldria} elaborar més i trobar alguna pista de com definir nous tipus complexos, stonebraker86 només indica com es va estendre postgresql amb operadors de creació de nous tipus però no com s'han de modelar els nous tipus.

Volem crear un nou model de SGBD, Date ens diu que només hi ha el model relacional, per tant hem d'utilitzar el model relacional per a definir el nostre nou model.






\chapter{Planificació}

\section{Treball realitzat}



Tesi de màster es va estudiar RRDtool a on es van observar propietats molt interessants de les sèries temporals: multiresolució, regularització, naturalesa, representació.


S'ha estat treballant en el model de SGST (no publicat). S'ha dividit el model en dues parts: una per al model general de les sèries temporals, basat en mesures i sèries temporals, una altra pel model de multiresolució, el qual es basa en buffers i discs. De moment s'ha dissenyat una estructura, el treball continua a definir les operacions. 


A multiresolution database is an storage system for one time series, that is a col-lection of data measured in derent instants in time. The time series is compactly
stored in the database as has been shown ingure 2. The principal part of a mul-
tiresolution database is the set of resolution discs where the time series is stored
distributed by the dierent interpolation functions and sampling periods.      Each
resolution disc uses its bur to interpolate the measures and uses its disc to con-
solidate the result.


\section{Treball futur}








%%% Local Variables: 
%%% mode: latex
%%% TeX-master: "main"
%%% End: 

\renewcommand*\descriptionlabel[1]{\hspace\labelsep\normalfont #1:}

\section*{Memòria de la feina realitzada al llarg dels darrers 12 mesos:}

% \subsection*{FORMACIÓ}

% \begin{description}


% \item[Juliol 2012] Curs de ?. Professor ?. Programa de doctorat ARV-UPC, Terrassa, 10 hores.

% \end{description}


\subsection*{BEQUES COMPLEMENTÀRIES}

\begin{description}

\item[Abril--maig 2013] Concessió de beca AGAUR per a les activitats
  acadèmiques dirigides de suport al professorat de les universitats
  públiques del sistema universitari català i de la Universitat Oberta
  de Catalunya (AAD\_2013), assignat al departament DIPSE-UPC,
  tutoritzada pel Dr.\ Sebastià Vila-Marta.  S'han realitzat tasques
  de suport informàtic i elaboració de material de pràctiques per a
  l'assignatura d'Aplicacions i Serveis a Internet del grau en
  enginyeria de sistemes TIC.
\end{description}


\subsection*{DOCÈNCIA}

\begin{description}

\item[Primavera 2013] 30 h de laboratori d'Aplicacions i Serveis a
  Internet en el grau en enginyeria de sistemes TIC a l'EPSEM -- UPC.

\item[Primavera 2013] 30 h de laboratori de Tecnologia de la
  Programació en el grau en enginyeria de sistemes TIC a l'EPSEM --
  UPC.

\end{description}




\subsection*{RECERCA}

\begin{description}


\item[Desembre 2012] S'ha publicat un report de
  recerca \parencite{llusa12:report} conjuntament amb la Teresa
  Escobet-Canal i el Sebastià Vila-Marta, publicat el 17 de desembre
  de 2012 al departament de Disseny i Programació de Sistemes
  Electrònics de la Universitat Politècnica de Catalunya.


\item[Febrer 2013] Ponència a congrés \parencite{llusa13:aiked}
  conjuntament amb la Teresa Escobet-Canal i el Sebastià Vila-Marta,
  realitzada al ``International Conference on Artificial Intelligence,
  Knowledge Engineering and Data Bases'' (AIKED '13) a Cambridge (UK)
  els dies 20--22 de febrer de 2013.  


\item[Setembre 2013] Renovada la matrícula de la tutoria de tesi pel
  curs acadèmic 2013--2014 al doctorat ARV.

% \item[Octubre 2012] Aprovada la concessió del projecte d'investigació
%   TEC2012-35571 ''Nuevas aplicaciones del principio superregenerativo
%   a comunicaciones por radiofrecuencia (NASP)'' per als pròxims 3 anys
%   en finalitzar l'anterior projecte AVIC. Investigador principal Pere
%   Palà Schönwälder.



\item[Curs 2012--2013] S'ha treballat en la tesi doctoral sota la
  direcció de la Dra.\ Teresa Escobet Canal i el Dr.\ Sebastià
  Vila-Marta. A continuació es detalla el progrés en el pla de treball
  presentat el 2012 per a assolir la recerca en el disseny d'un model
  de sistema de gestió per a sèries temporals. En primer lloc es
  presenta l'actualització dels objectius, en segon lloc el progrés en
  funció de les tasques planificades realitzades


\end{description}



\subsubsection*{Objectius}


Aquesta recerca té per objectiu l'estudi de les necessitats
específiques que comporta l'emmagatzematge i gestió de dades amb
naturalesa de sèrie temporal i la proposta d'un model de SGBD que
satisfaci aquestes necessitats. Aquest objectiu es divideix en els
següents subobjectius més concrets:

\begin{enumerate}

\item Estudi de les aplicacions en que les dades són sèries temporals
  amb la finalitat de determinar quines són les propietats i problemes
  comuns que planteja la seva gestió i emmagatzematge.

\item Estudi dels models de SGBD existents. Segons es desprèn de la
  formalització de \textcite{date:introduction} el model principal és
  el model relacional, el qual es fonamenta en dos conceptes:
  relacions i tipus de dades. 

\item Una àrea de treball important en els SGBD és la incorporació de
  nous tipus de dades complexos. És important estudiar com es modifica
  el model de dades d'un SGBD quan s'afegeix un nou tipus de dades
  complex.  Les sèries temporals es poden d'entendre com a tipus
  complex ja que presenten diferents propietats característiques i
  necessiten operadors addicionals.  

\item Disseny d'un model de SGBD per a les sèries temporals. D'aquesta
  manera els SGBD podran tractar dades amb instants de temps que
  mostrin l'evolució de variables en funció del temps. El model
  consisteix en la definició de l'estructura de les sèries temporals i
  les operacions bàsiques que necessiten.

  L'assoliment d'aquest objectiu té tres parts:

  \begin{enumerate}
  \item Disseny d'un model per a la gestió bàsica de les sèries
    temporals, el qual anomenem model de SGBD per a sèries temporals
    (SGST).  L'estructura d'aquest model és similar a l'utilitzat en
    els intervals
    temporals \parencite{date02:_tempor_data_relat_model}.  Prenent
    com a base el model de SGST, el qual és un model general per a les
    sèries temporals, s'hi poden incloure altres models per a
    propietats més específiques de les sèries temporals.

  \item Disseny d'un model específic en base del model de
    SGST. Concretament es dissenya un model pels SGST multiresolució
    (SGSTM).  En el model de SGSTM s'hi poden incloure propietats de
    les sèries temporals relacionades amb la resolució que s'han
    observat en les aplicacions pràctiques de les sèries temporals:
    regularització, canvis de resolució mitjançant agregacions,
    reconstrucció de forats, etc.
 
  \item Avaluació de diferents estructures de SGSTM. El model de SGSTM
    s'hi poden fer modificacions o simplificacions per tal
    d'aconseguir diferents estructures.  Per exemple bases de dades
    multiresolució que comparteixin informació, que treballin amb flux
    de dades (\emph{data stream}) o bé que s'especialitzin per a un
    tipus determinat de sèries temporals.

  \end{enumerate}

\item Implementació de referència dels models de SGST i SGSTM. Per una
  banda, aquesta implementació, a nivell acadèmic, ha de servir com a
  exemple per a futurs desenvolupaments de sistemes de gestió,
  acadèmics o productius. Per altra banda, ha de servir per a
  exemplificar-ne els seu funcionament amb unes dades de prova.

\item Implementació específica i reduïda del model per a una
  determinada aplicació de sèries temporals. Exemplificació de com una
  estructura de SGSTM pot ser implementada per a aconseguir una
  aplicació molt concreta.

\end{enumerate} 


El objectius 4.c i 6. són resultat d'una nova planificació com es
detalla més endavant a les tasques futures.



\subsubsection*{Tasques realitzades}

Per tal d'assolir els objectius detallats a l'apartat anterior, el
juliol de 2012 es va proposar un seguit de tasques a realitzar.  A la
figura \ref{fig:pla:actual} es detalla l'estat d'aquell pla de treball
a dia d'avui.



\begin{figure}[tp]
  \centering
  \scalebox{0.8}{
  \begin{ganttchart}[
    vgrid,
    x unit=0.8cm,
    today=7.5,
    today label=PRESENT,
    title label font=\small,
    title height=1,
    bar label font = \small,
    bar/.style={draw=none, fill=black},
    %incomplete/.style={fill=lightgray},
    progress label text =,
    ]
    {12}{12}
    \gantttitle{2012}{4}
    \gantttitle{2013}{4}
    \gantttitle{2014}{4} \\
    \gantttitlelist{1,...,4}{1}\gantttitlelist{1,...,4}{1}\gantttitlelist{1,...,4}{1} \\

    \ganttmilestone{Proposta de tesi}{2} \\

    \ganttgroup{Estudi}{1}{4} \\
    \ganttbar[progress=100]{1. Aplicacions}{1}{2} \\
    \ganttbar{2. Model relacional}{1}{4} \\
    \ganttbar[progress=100]{3. Propietats temporals}{3}{3}
    \ganttbar[bar/.style={fill=white}]{}{4}{4}
    \ganttbar[progress=100]{}{7}{7} \\

    \ganttgroup[progress=90]{Disseny models}{5}{7} \\
    \ganttbar[progress=100]{4. Sèries temporals}{5}{7} \\
    \ganttbar[progress=75]{5. Multiresolució}{6}{7} \\

    \ganttgroup[progress=0]{Experimentació}{8}{9} \\
    \ganttbar[progress=0]{6. Implementació}{8}{9} \\
    \ganttlink{elem7}{elem11}
    \ganttbar[progress=0]{7. Dades experimentals}{9}{9} \\



    \ganttgroup[progress=20]{Redacció}{2}{10} \\

    \ganttbar[progress=0]{8. Redacció memòria}{9}{10} \\
    \ganttlinkedmilestone[milestone/.style={fill=lightgray}]{Dipòsit de tesi}{10} \\

    \ganttbar[progress=100]{9. Redacció articles}{4}{4} \\
    \ganttlinkedmilestone{Informe de recerca}{4} \\

    \ganttmilestone{AIKED'13}{4.5}
  \end{ganttchart}
  }
\caption{Tasques completades del treball}
\label{fig:pla:actual}
\end{figure}



\begin{enumerate}

\item Estudi d'aplicacions de les sèries temporals. Per a assolir
  l'objectiu~1 s'ha realitzat una tasca de recerca bibliogràfica i
  estudi dels treballs existents en l'àmbit de la gestió i
  emmagatzemat de dades provinents de sèries temporals. També s'ha
  cercat l'existència de programari que tingui en els seus objectius
  el tractament de sèries temporals. El resultat d'aquesta tasca es va
  reflectir a la proposta de
  tesi \parencite{llusa12:ptd}. Anteriorment a la tesi de
  màster \parencite{llusa11:tfm} ja s'havia estudiat profundament
  \emph{RRDtool} \parencite{rrdtool}, un sistema de gestió de bases de
  dades (SGBD) específic per a sèries temporals.


\item Estudi del model relacional. Per a l'objectiu~2 s'ha estudiat en
  profunditat el model relacional, prenent-lo com a referent pels
  models de SGBD. Atesa la preeminència d'aquest model en els SGBD
  actuals, es va considerar imprescindible tenir-ne un bon coneixement
  que permetés estudiar les seves deficiències per la gestió de sèries
  temporals i en quina forma poden ser superades. Aquest estudi s'ha
  utilitzat a l'objectiu~4 i ha servit per establir el punt de partida
  del model de SGST i la referència de conceptes bàsics pels SGBD.


\item Estudi de la gestió d'intervals temporals. Per a l'objectiu~3
  s'ha estudiat la recerca que ha conduit a incorporar els interval
  temporals en els SGBD per a gestionar històrics. Ens hem basat en
  l'estudi que relaciona el model relacional amb els intervals
  temporals de \textcite{date02:_tempor_data_relat_model}.  Aquest
  estudi ha servit per a complementar l'estudi del model relacional,
  ja que es detalla com es poden modelar els intervals temporals
  prenent com a base els SGBD relacionals. Tot i així, en un principi
  presenten molta diferència amb les propietats de les sèries
  temporals i s'ha trobat més adequat deixar de banda l'estudi dels
  intervals temporals i dedicar més detall sobre les propietats de les
  sèries temporals, les quals durant la tasca 4 s'ha observat que só
  variades i interessants.



\item Disseny d'un model de SGBD per sèries temporals (SGST). Per a la
  primera part de l'objectiu 4 s'ha dissenyat un model per als SGBD de
  sèries temporals formalitzat amb expressions algebraiques. Aquest
  model, com ja s'ha dit, està fortament basat amb els conceptes de
  SGBD que descriu el model relacional. A partir de l'estudi de
  l'estat de l'art de les sèries temporals, hem observat que els SGBD
  per sèries temporals precisen d'operacions de tres naturaleses
  diferents: com a conjunts, com a seqüències i com a funció temporal.
  Aquestes naturaleses han comportat la planificació d'un estudi a
  part com ja s'ha dit, i així s'ha incorporat al model de SGST les
  propietats vàries que poden tenir les sèries temporals:
  representacions temporals, regularitat del temps entre mesures,
  forats de temps, etc. Tot i així, en el model estructural i
  d'operacions de SGST s'ha pogut separar-ho i per tant el model s'ha
  pogut completar independentment.



\item Disseny d'un model de SGBD multiresolució per sèries temporals
  (SGSTM). Per a la segona part de l'objectiu 4 s'ha dissenyat un
  model que contempla la multiresolució de les sèries temporals. S'ha
  refet aquest model per a utilitzar operacions definides en el model
  anterior per les sèries temporals. Aquest model també s'ha descrit a
  excepció d'un apartat que ha d'exemplificar com es duen a terme les
  agregacions d'atributs per a aconseguir la multiresolució.



\item[9.] Redacció d'articles i informes. A partir de l'estudi fet a
  la proposta de tesi, es va iniciar la redacció d'un informe de
  recerca a on constés les mancances que observem que tenen els SGBD
  per a les sèries temporals, les propietats i requisits que han de
  complir i la idea bàsica de la nostra proposta d'un nou model
  multiresolució per a sèries temporals. Para\l.lelament també es va
  preparar un article per a congrés per tal de donar a conèixer de
  forma resumida el model multiresolució que volem dissenyar. Aquestes
  dues tasques han conclòs amb dues publicacions que es detallen a
  l'apartat següent.

\end{enumerate}






\newpage

\section*{Pla de treball per als propers 12 mesos:}

El pla de treball futur pel al curs acadèmic 2013--2014 se centra en
la realització de la tesi en el doctorat en automàtica, robòtica i
visió.  A més, com a resultat del treball dut a terme fins ara, s'ha
actualitzat alguns dels objectius, com ja s'ha notat anteriorment.
%En el document de l'\autoref{sec:pla_treball} ''\nameref{sec:pla_treball}''
A continuació, es planifica el pla de treball futur a la figura
\ref{fig:pla:futur} i es detalla la feina a realitzar en el proper
any.


\begin{figure}[tp]
\centering
  \scalebox{0.8}{
  \begin{ganttchart}[
    vgrid,
    x unit=0.8cm,
    today=7.5,
    today label=PRESENT,
    title label font=\small,
    title height=1,
    bar label font = \small,
    bar/.style={draw=none, fill=black},
    %incomplete/.style={fill=lightgray},
    progress label text =,
    ]
    {12}{12}
    \gantttitle{2012}{4}
    \gantttitle{2013}{4}
    \gantttitle{2014}{4} \\
    \gantttitlelist{1,...,4}{1}\gantttitlelist{1,...,4}{1}\gantttitlelist{1,...,4}{1} \\

    \ganttmilestone{Proposta de tesi}{2} \\

    \ganttgroup{Estudi}{1}{4} \\
    \ganttbar[progress=100]{1. Aplicacions}{1}{2} \\
    \ganttbar{2. Model relacional}{1}{4} \\
    \ganttbar[progress=100]{%3. Intervals temporals
}{3}{3} 
    \ganttbar[progress=100]{3. Propietats ST}{7}{7} \\

    \ganttgroup[progress=90]{Disseny models}{5}{8} \\
    \ganttbar[progress=100]{4. Sèries temporals}{5}{7} \\
    \ganttbar[progress=100]{5. Multiresolució}{6}{6}
    \ganttbar[progress=50]{}{8}{8} \\
    \ganttbar[bar/.style={fill=white}]{5.b Estructures}{8}{8} \\
    \ganttlink{elem5}{elem9}

    \ganttgroup[progress=0]{Experimentació}{9}{10} \\
    \ganttbar[progress=0]{6. Implementació}{9}{9} \\
    \ganttlink{elem9}{elem11}
    \ganttbar[progress=0]{7. Dades experimentals}{9}{10} \\




   \ganttgroup[progress=20]{Redacció}{2}{10} \\

    \ganttbar[progress=0]{8. Redacció memòria}{9}{10} \\
    \ganttlinkedmilestone[milestone/.style={fill=lightgray}]{Dipòsit de tesi}{10} \\

    \ganttbar[progress=100]{9. Redacció articles}{4}{4} 
    \ganttbar[bar/.style={fill=white}]{}{9}{9} \\

    \ganttmilestone{Informe de recerca}{4} \\
    \ganttlink{elem17}{elem19}

    \ganttmilestone{AIKED'13}{4.5}
  \end{ganttchart}
  }
\caption{Planificació del treball pendent}
\label{fig:pla:futur}
\end{figure}


\begin{itemize}

\item[5.] Per al disseny del model de SGSTM manca, com ja s'ha dit,
  refer un apartat sobre agregadors d'atributs, els quals són les
  funcions que permeten als SGSTM calcular diverses resolucions d'una
  sèrie temporal. L'estudi en l'objectiu~3 de les propietats variades
  de les sèries temporals indueix a varis agregadors d'atributs en el
  model de SGSTM. Així, es donaran exemples que facilitin la
  comprensió de com de variats poden ser i serveixin de model per a
  dissenyar-ne altres.


\item[5.b] Durant l'estudi de l'estat actual de les sèries temporals
  es van observar diverses aproximacions a l'hora de fer càlculs amb
  sèries temporals i en general en els SGBD. Això ha comportat la
  definició d'una tercera part en l'objectiu~4 a on s'avaluaran
  diferents estructures possibles dels SGSTM.  La multiresolució té
  l'objectiu d'emmagatzemar les sèries temporals de forma compacta,
  així es preveu que alguns conceptes de la recerca en \emph{data
    streams} poden prendre-hi sentit.



\item[6.] Implementació de referència. Per a l'objectiu~5 s'implementaran
  els models de SGST i SGSTM utilitzant un llenguatge de programació adequat
  per a models, com per exemple \texttt{Python} o \texttt{Prolog}.  Es
  prioritzarà la implementació correcte del model enfront a una
  implementació que contempli un bon rendiment. 


\item[7.] Experimentació amb dades. Per a complementar l'objectiu~5 es
  provarà la implementació amb dades experimentals per alguna
  aplicació concreta. El suport del projecte \emph{i-Sense} (FP7-
  ICT-270428) aporta dades que són sèries temporals i són candidates a
  experimentar amb elles.  També la participació en el projecte
  \emph{NAPS} (TEC2012-35571) aporta l'objectiu~6 en el qual s'està
  pensant d'implementar una part reduïda del SGSTM a baix nivell com
  pot ser amb llenguatge VHDL. És a dir la implementació d'alguna de
  les estructures específiques de SGSTM estudiades prèviament a mode
  d'exemple de com introduir una base de dades multiresolució per a
  aplicacions molt concretes.




\item[9.] Redacció d'articles i informes. Un cop s'hagi completat el
  disseny del model de SGBD per a sèries temporals, es pot presentar
  com a article en l'àmbit de les bases de dades.  També en pot
  resultar un article un cop s'hagi realitzat la implementació.


\end{itemize}




%\newpage


% \section*{Pla de treball principal}
% %\label{sec:pla_treball}

% \chapter{Introducció}




Aquesta recerca s'estructura al voltant dels sistemes
d'emmagatzemament i tractament de dades com a sèries temporals.
Concretament se centra en els sistemes de gestió de bases de dades
(SGBD) que s'ocupen de sèries temporals (SGST). Els SGST han de tenir  funcionalitats adequades per gestionar i
explotar correctament la informació de les sèries temporals.

Tal com diu Fu, la recerca en mineria de sèries temporals ha augmentat. Per tant és un camp amb interès actual, sobretot hi ha interès en processar grans volums de dades. Hi ha molts que estudien com resoldre això, per tant potser es pot observar que pel que fa a implementacions d'algoritmes que tinguin bon rendiment o facin bon ús d'energia hi ha molt escrit i s'obtenen bons resultats. Tot i així sembla que encara hi ha camp a recórrer: la recerca en aquests temes segueix avançant. No obstant, nosaltres ens centrarem en obtenir un model de SGBD, en el benentès que molts d'aquests estudis es podrien aplicar en la implementació de SGBD que seguissin el model.


Recentment s'ha observat que hi ha un forat de coneixement entre els SGBD i les aplicacions de les sèries temporals, referides a la literatura com a científiques [stonebraker09:scidb i zhang]


Estan apareixent SGST (RRDTool, Cougar, ...) però no hi ha definit clarament un model de SGST. 

Alguns intents han estat Dreyer amb unes primeres propostes de que han de complir els SGST, bonnet per a xarxes de sensors, zhang amb exemples de resolucions de consultes per algunes de les propietats de les sèries temporals

Ara bé, els SGST actuals es basen en models propis, p.ex. Cougar amb models de seqüències, però no s'han estudiat com a model de SGBD, del qual actualment només es considera com a representant el model relacional [Date]. 

Dins del model relacional hi ha hagut un estudi profund per als intervals temporals, referits com a dades temporals [date02], i això confereix una nova dimensió en la resolució del problema dels històrics temporals en els SGBD. Tot i ser una categoria diferent dels intervals temporals [assfalg], les sèries temporals necessiten un estudi similar.




\section{Objectius i contribucions}



La motivació prové per una banda de RRDtool i per altra banda de la formalització de les dades temporals [date02]



  Dissenyar un model de dades que descrigui l'estructura i el comportament
ˆ
  dels SGBD per a sèries temporals.
  Proposar una implementació de referència del model dissenyat.
ˆ


  Proposar millores i treballs futurs al voltant del model dissenyat.
ˆ


Objectius del llibre de temporal data de darwen i date, segons unes transparències de Darwen:

The Book’s Aims:
Describe a foundation for inclusion of support for temporal data in a truly
relational database management system (TRDBMS)
Focussing on problems related to data representing beliefs that hold throughout
given intervals (usually, of time).
Propose additional operators on relations and relation variables ("relvars")
having interval-valued attributes.
Propose additional constraints on relation variables having interval-valued
attributes.
[transparencies darwen]







Així doncs, tenint en compte que segons Date el model relacional és complet, que no hi ha cap de tant potent i que l'ampliació de funcionalitat dels SGBDR s'ha de fer mitjançant la creació de nous tipus, el SGST hauria de contemplar aquestes idees. 

Primer s'hauria de veure que el cas de les sèries temporals no sigui com el cas de les dades temporals a on sí que s'ha necessitat estendre el model relacional?. --> Bé de fet potser aquesta era la idea inicial en el model relacional (que s'havia de modificar per a poder tenir històrics) però ara després de modelar les dades temporals es pot veure que el model relacional ho accepta com a tipus: és a dir les relacions que tinguin atributs de tipus interval temporal passen a ser relacions temporals.

Segon, s'hauria de considerar que el SGST són un nou tipus de dades en el model relacional i per tant el model relacional ja té tota la potència per una banda constituir SGBD i per altra banda definir i incorporar nous tipus.


Com es defineixen nous tipus complexos als SGBD?

En el cas que es descarti que el cas dels SGST presenta els mateixos problemes que les dades temporals i per tant els SGST han d'esdevenir un tipus de dades, cal preguntar-se com són els tipus de dades al model relacional.

Concretament el model relacional només defineix què es un tipus de dades però dóna llibertat a la seva creació. Això és un gran avantatge. En els cas de tipus de dades senzills es defineixen amb una bona estructura i ja està però què passa quan es vol definir un nou tipus de dades complex?

Cal recercar com s'han definit nous tipus de dades complexos. Els principals problemes es donen que el tipus és complex i forma una entitat de per sí. És a dir que definir un tipus sèrie temporal no és trivial. Aleshores, com cal procedir?



Sobre la necessitat de modelar els tipus.

Cal definir un model pel tipus que volem dissenyar.
De fet, volem dissenyar un SGST. És a dir, un tipus que conformi pròpiament un SGBD. Per tant, utilitzarem les mateixes eines que es fan servir per modelar els SGBDR per a poder modelar el nostre SGST. Com que s'hauran utilitzat les mateixes eines, el SGST podrà esdevenir perfectament un tipus de dades pels SGBDR.

En resum, per a definir nous tipus complexes cal modelar-los com a entitat pròpia, fent un símil amb el model relacional. Això no vol dir, però, que un cop modelats constitueixin de per sí un nou model per als SGBD, sinó que queden dins dels SGBDR. \todo{caldria} elaborar més i trobar alguna pista de com definir nous tipus complexos, stonebraker86 només indica com es va estendre postgresql amb operadors de creació de nous tipus però no com s'han de modelar els nous tipus.

Volem crear un nou model de SGBD, Date ens diu que només hi ha el model relacional, per tant hem d'utilitzar el model relacional per a definir el nostre nou model.




\subsection{Contribucions}


1. Comprensió de l'estudi de la modelització de SGBD. Segons formalitza Date cal entendre principalment el model relacional: relacions i tipus. 

2. Estudi de la modelització de nous tipus complexos dins dels models de SGBD.
Les sèries temporals s'han d'entendre com a tipus complex.
Els SGBD permeten que els usuaris defineixin nous tipus \parencite{stonebraker86} però no hi ha un estudi teòric dels tipus als SGBD: Date ens descriu abastament les relacions però no els tipus. Els tipus s'han d'estudiar i modelar per a poder-los considerar com a tals, oimés els tipus complexos ja que requereixen un estudi més complet i possiblement s'hagin de modelar com un propi SGBD. 



3. Proposta de model per a les sèries temporals separat en dos: 

* Proposta d'un model per a les sèries temporals similar al de les dades temporals [date02], ho anomenem model de SGST. A sobre del model de SGST, el qual és un model general per a les sèries temporals, s'hi poden proposar altres models per a propietats més específiques de les sèries temporals, concretament nosaltres proposem el model multiresolució.

* Proposta d'un model multiresolució a sobre del model de les sèries temporals. Ho anomenem model de SGSTM. En el model multiresolució s'hi inclouen propietats de les sèries temporals que s'han observat en les seves aplicacions: regularització, canvis de resolució mitjançant agregacions, farciment de forats, ... 
(nota: a més aquesta part del model és la més sensible a ser implementada com a data streams)

4. Implementació de referència del model.


5. Possible exemplificació en algun cas pràctic?


A multiresolution database is an storage system for one time series, that is a col-
lection of data measured in dierent instants in time. The time series is compactly
stored in the database as has been shown in gure 2. The principal part of a mul-
tiresolution database is the set of resolution discs where the time series is stored
distributed by the dierent interpolation functions and sampling periods.      Each
resolution disc uses its buer to interpolate the measures and uses its disc to con-
solidate the result.




\section{Treball fet fins ara}



Tesi de màster es va estudiar RRDtool a on es van observar propietats molt interessants de les sèries temporals: multiresolució, regularització, naturalesa, representació.


S'ha estat treballant en el model de SGST (no publicat). S'ha dividit el model en dues parts: una per al model general de les sèries temporals, basat en mesures i sèries temporals, una altra pel model de multiresolució, el qual es basa en buffers i discs. De moment s'ha dissenyat una estructura, el treball continua a definir les operacions. 




% \chapter{Planificació}

\section{Treball realitzat}



Tesi de màster es va estudiar RRDtool a on es van observar propietats molt interessants de les sèries temporals: multiresolució, regularització, naturalesa, representació.


S'ha estat treballant en el model de SGST (no publicat). S'ha dividit el model en dues parts: una per al model general de les sèries temporals, basat en mesures i sèries temporals, una altra pel model de multiresolució, el qual es basa en buffers i discs. De moment s'ha dissenyat una estructura, el treball continua a definir les operacions. 


A multiresolution database is an storage system for one time series, that is a col-lection of data measured in derent instants in time. The time series is compactly
stored in the database as has been shown ingure 2. The principal part of a mul-
tiresolution database is the set of resolution discs where the time series is stored
distributed by the dierent interpolation functions and sampling periods.      Each
resolution disc uses its bur to interpolate the measures and uses its disc to con-
solidate the result.


\section{Treball futur}



Així doncs, tenint en compte que segons Date el model relacional és complet, que no hi ha cap de tant potent i que l'ampliació de funcionalitat dels SGBD relacionals s'ha de fer mitjançant la creació de nous tipus, el SGST hauria de contemplar aquestes idees. 

Primer s'hauria de veure que el cas de les sèries temporals no sigui com el cas de les dades temporals a on sí que s'ha necessitat estendre el model relacional?. --> Bé de fet potser aquesta era la idea inicial en el model relacional (que s'havia de modificar per a poder tenir històrics) però ara després de modelar les dades temporals es pot veure que el model relacional ho accepta com a tipus: és a dir les relacions que tinguin atributs de tipus interval temporal passen a ser relacions temporals.

Segon, s'hauria de considerar que el SGST són un nou tipus de dades en el model relacional i per tant el model relacional ja té tota la potència per una banda constituir SGBD i per altra banda definir i incorporar nous tipus.


Com es defineixen nous tipus complexos als SGBD?

En el cas que es descarti que el cas dels SGST presenta els mateixos problemes que les dades temporals i per tant els SGST han d'esdevenir un tipus de dades, cal preguntar-se com són els tipus de dades al model relacional.

Concretament el model relacional només defineix què es un tipus de dades però dóna llibertat a la seva creació. Això és un gran avantatge. En els cas de tipus de dades senzills es defineixen amb una bona estructura i ja està però què passa quan es vol definir un nou tipus de dades complex?

Cal recercar com s'han definit nous tipus de dades complexos. Els principals problemes es donen que el tipus és complex i forma una entitat de per sí. És a dir que definir un tipus sèrie temporal no és trivial. Aleshores, com cal procedir?



Sobre la necessitat de modelar els tipus.

Cal definir un model pel tipus que volem dissenyar.
De fet, volem dissenyar un SGST. És a dir, un tipus que conformi pròpiament un SGBD. Per tant, utilitzarem les mateixes eines que es fan servir per modelar els SGBDR per a poder modelar el nostre SGST. Com que s'hauran utilitzat les mateixes eines, el SGST podrà esdevenir perfectament un tipus de dades pels SGBDR.

En resum, per a definir nous tipus complexes cal modelar-los com a entitat pròpia, fent un símil amb el model relacional. Això no vol dir, però, que un cop modelats constitueixin de per sí un nou model per als SGBD, sinó que queden dins dels SGBDR. \todo{caldria} elaborar més i trobar alguna pista de com definir nous tipus complexos, stonebraker86 només indica com es va estendre postgresql amb operadors de creació de nous tipus però no com s'han de modelar els nous tipus.

Volem crear un nou model de SGBD, Date ens diu que només hi ha el model relacional, per tant hem d'utilitzar el model relacional per a definir el nostre nou model.





%%% Local Variables: 
%%% mode: latex
%%% TeX-master: "main"
%%% End: 






%%% Local Variables: 
%%% mode: latex
%%% TeX-master: "main"
%%% End: 

\renewcommand*\descriptionlabel[1]{\hspace\labelsep\normalfont #1:}

\section*{Memòria de la feina realitzada al llarg dels darrers 12 mesos:}

% \subsection*{FORMACIÓ}

% \begin{description}


% \item[Juliol 2012] Curs de ?. Professor ?. Programa de doctorat ARV-UPC, Terrassa, 10 hores.

% \end{description}


\subsection*{BEQUES COMPLEMENTÀRIES}

\begin{description}

\item[Abril--maig 2013] Concessió de beca AGAUR per a les activitats
  acadèmiques dirigides de suport al professorat de les universitats
  públiques del sistema universitari català i de la Universitat Oberta
  de Catalunya (AAD\_2013), assignat al departament DIPSE-UPC,
  tutoritzada pel Dr.\ Sebastià Vila-Marta.  S'han realitzat tasques
  de suport informàtic i elaboració de material de pràctiques per a
  l'assignatura d'Aplicacions i Serveis a Internet del grau en
  enginyeria de sistemes TIC.
\end{description}


\subsection*{DOCÈNCIA}

\begin{description}

\item[Primavera 2013] 30 h de laboratori d'Aplicacions i Serveis a
  Internet en el grau en enginyeria de sistemes TIC a l'EPSEM -- UPC.

\item[Primavera 2013] 30 h de laboratori de Tecnologia de la
  Programació en el grau en enginyeria de sistemes TIC a l'EPSEM --
  UPC.

\end{description}




\subsection*{RECERCA}

\begin{description}


\item[Desembre 2012] S'ha publicat un report de
  recerca \parencite{llusa12:report} conjuntament amb la Teresa
  Escobet-Canal i el Sebastià Vila-Marta, publicat el 17 de desembre
  de 2012 al departament de Disseny i Programació de Sistemes
  Electrònics de la Universitat Politècnica de Catalunya.


\item[Febrer 2013] Ponència a congrés \parencite{llusa13:aiked}
  conjuntament amb la Teresa Escobet-Canal i el Sebastià Vila-Marta,
  realitzada al ``International Conference on Artificial Intelligence,
  Knowledge Engineering and Data Bases'' (AIKED '13) a Cambridge (UK)
  els dies 20--22 de febrer de 2013.  


\item[Setembre 2013] Renovada la matrícula de la tutoria de tesi pel
  curs acadèmic 2013--2014 al doctorat ARV.

% \item[Octubre 2012] Aprovada la concessió del projecte d'investigació
%   TEC2012-35571 ''Nuevas aplicaciones del principio superregenerativo
%   a comunicaciones por radiofrecuencia (NASP)'' per als pròxims 3 anys
%   en finalitzar l'anterior projecte AVIC. Investigador principal Pere
%   Palà Schönwälder.



\item[Curs 2012--2013] S'ha treballat en la tesi doctoral sota la
  direcció de la Dra.\ Teresa Escobet Canal i el Dr.\ Sebastià
  Vila-Marta. A continuació es detalla el progrés en el pla de treball
  presentat el 2012 per a assolir la recerca en el disseny d'un model
  de sistema de gestió per a sèries temporals. En primer lloc es
  presenta l'actualització dels objectius, en segon lloc el progrés en
  funció de les tasques planificades realitzades


\end{description}



\subsubsection*{Objectius}


Aquesta recerca té per objectiu l'estudi de les necessitats
específiques que comporta l'emmagatzematge i gestió de dades amb
naturalesa de sèrie temporal i la proposta d'un model de SGBD que
satisfaci aquestes necessitats. Aquest objectiu es divideix en els
següents subobjectius més concrets:

\begin{enumerate}

\item Estudi de les aplicacions en que les dades són sèries temporals
  amb la finalitat de determinar quines són les propietats i problemes
  comuns que planteja la seva gestió i emmagatzematge.

\item Estudi dels models de SGBD existents. Segons es desprèn de la
  formalització de \textcite{date:introduction} el model principal és
  el model relacional, el qual es fonamenta en dos conceptes:
  relacions i tipus de dades. 

\item Una àrea de treball important en els SGBD és la incorporació de
  nous tipus de dades complexos. És important estudiar com es modifica
  el model de dades d'un SGBD quan s'afegeix un nou tipus de dades
  complex.  Les sèries temporals es poden d'entendre com a tipus
  complex ja que presenten diferents propietats característiques i
  necessiten operadors addicionals.  

\item Disseny d'un model de SGBD per a les sèries temporals. D'aquesta
  manera els SGBD podran tractar dades amb instants de temps que
  mostrin l'evolució de variables en funció del temps. El model
  consisteix en la definició de l'estructura de les sèries temporals i
  les operacions bàsiques que necessiten.

  L'assoliment d'aquest objectiu té tres parts:

  \begin{enumerate}
  \item Disseny d'un model per a la gestió bàsica de les sèries
    temporals, el qual anomenem model de SGBD per a sèries temporals
    (SGST).  L'estructura d'aquest model és similar a l'utilitzat en
    els intervals
    temporals \parencite{date02:_tempor_data_relat_model}.  Prenent
    com a base el model de SGST, el qual és un model general per a les
    sèries temporals, s'hi poden incloure altres models per a
    propietats més específiques de les sèries temporals.

  \item Disseny d'un model específic en base del model de
    SGST. Concretament es dissenya un model pels SGST multiresolució
    (SGSTM).  En el model de SGSTM s'hi poden incloure propietats de
    les sèries temporals relacionades amb la resolució que s'han
    observat en les aplicacions pràctiques de les sèries temporals:
    regularització, canvis de resolució mitjançant agregacions,
    reconstrucció de forats, etc.
 
  \item Avaluació de diferents estructures de SGSTM. El model de SGSTM
    s'hi poden fer modificacions o simplificacions per tal
    d'aconseguir diferents estructures.  Per exemple bases de dades
    multiresolució que comparteixin informació, que treballin amb flux
    de dades (\emph{data stream}) o bé que s'especialitzin per a un
    tipus determinat de sèries temporals.

  \end{enumerate}

\item Implementació de referència dels models de SGST i SGSTM. Per una
  banda, aquesta implementació, a nivell acadèmic, ha de servir com a
  exemple per a futurs desenvolupaments de sistemes de gestió,
  acadèmics o productius. Per altra banda, ha de servir per a
  exemplificar-ne els seu funcionament amb unes dades de prova.

\item Implementació específica i reduïda del model per a una
  determinada aplicació de sèries temporals. Exemplificació de com una
  estructura de SGSTM pot ser implementada per a aconseguir una
  aplicació molt concreta.

\end{enumerate} 


El objectius 4.c i 6. són resultat d'una nova planificació com es
detalla més endavant a les tasques futures.



\subsubsection*{Tasques realitzades}

Per tal d'assolir els objectius detallats a l'apartat anterior, el
juliol de 2012 es va proposar un seguit de tasques a realitzar.  A la
figura \ref{fig:pla:actual} es detalla l'estat d'aquell pla de treball
a dia d'avui.



\begin{figure}[tp]
  \centering
  \scalebox{0.8}{
  \begin{ganttchart}[
    vgrid,
    x unit=0.8cm,
    today=7.5,
    today label=PRESENT,
    title label font=\small,
    title height=1,
    bar label font = \small,
    bar/.style={draw=none, fill=black},
    %incomplete/.style={fill=lightgray},
    progress label text =,
    ]
    {12}{12}
    \gantttitle{2012}{4}
    \gantttitle{2013}{4}
    \gantttitle{2014}{4} \\
    \gantttitlelist{1,...,4}{1}\gantttitlelist{1,...,4}{1}\gantttitlelist{1,...,4}{1} \\

    \ganttmilestone{Proposta de tesi}{2} \\

    \ganttgroup{Estudi}{1}{4} \\
    \ganttbar[progress=100]{1. Aplicacions}{1}{2} \\
    \ganttbar{2. Model relacional}{1}{4} \\
    \ganttbar[progress=100]{3. Propietats temporals}{3}{3}
    \ganttbar[bar/.style={fill=white}]{}{4}{4}
    \ganttbar[progress=100]{}{7}{7} \\

    \ganttgroup[progress=90]{Disseny models}{5}{7} \\
    \ganttbar[progress=100]{4. Sèries temporals}{5}{7} \\
    \ganttbar[progress=75]{5. Multiresolució}{6}{7} \\

    \ganttgroup[progress=0]{Experimentació}{8}{9} \\
    \ganttbar[progress=0]{6. Implementació}{8}{9} \\
    \ganttlink{elem7}{elem11}
    \ganttbar[progress=0]{7. Dades experimentals}{9}{9} \\



    \ganttgroup[progress=20]{Redacció}{2}{10} \\

    \ganttbar[progress=0]{8. Redacció memòria}{9}{10} \\
    \ganttlinkedmilestone[milestone/.style={fill=lightgray}]{Dipòsit de tesi}{10} \\

    \ganttbar[progress=100]{9. Redacció articles}{4}{4} \\
    \ganttlinkedmilestone{Informe de recerca}{4} \\

    \ganttmilestone{AIKED'13}{4.5}
  \end{ganttchart}
  }
\caption{Tasques completades del treball}
\label{fig:pla:actual}
\end{figure}



\begin{enumerate}

\item Estudi d'aplicacions de les sèries temporals. Per a assolir
  l'objectiu~1 s'ha realitzat una tasca de recerca bibliogràfica i
  estudi dels treballs existents en l'àmbit de la gestió i
  emmagatzemat de dades provinents de sèries temporals. També s'ha
  cercat l'existència de programari que tingui en els seus objectius
  el tractament de sèries temporals. El resultat d'aquesta tasca es va
  reflectir a la proposta de
  tesi \parencite{llusa12:ptd}. Anteriorment a la tesi de
  màster \parencite{llusa11:tfm} ja s'havia estudiat profundament
  \emph{RRDtool} \parencite{rrdtool}, un sistema de gestió de bases de
  dades (SGBD) específic per a sèries temporals.


\item Estudi del model relacional. Per a l'objectiu~2 s'ha estudiat en
  profunditat el model relacional, prenent-lo com a referent pels
  models de SGBD. Atesa la preeminència d'aquest model en els SGBD
  actuals, es va considerar imprescindible tenir-ne un bon coneixement
  que permetés estudiar les seves deficiències per la gestió de sèries
  temporals i en quina forma poden ser superades. Aquest estudi s'ha
  utilitzat a l'objectiu~4 i ha servit per establir el punt de partida
  del model de SGST i la referència de conceptes bàsics pels SGBD.


\item Estudi de la gestió d'intervals temporals. Per a l'objectiu~3
  s'ha estudiat la recerca que ha conduit a incorporar els interval
  temporals en els SGBD per a gestionar històrics. Ens hem basat en
  l'estudi que relaciona el model relacional amb els intervals
  temporals de \textcite{date02:_tempor_data_relat_model}.  Aquest
  estudi ha servit per a complementar l'estudi del model relacional,
  ja que es detalla com es poden modelar els intervals temporals
  prenent com a base els SGBD relacionals. Tot i així, en un principi
  presenten molta diferència amb les propietats de les sèries
  temporals i s'ha trobat més adequat deixar de banda l'estudi dels
  intervals temporals i dedicar més detall sobre les propietats de les
  sèries temporals, les quals durant la tasca 4 s'ha observat que só
  variades i interessants.



\item Disseny d'un model de SGBD per sèries temporals (SGST). Per a la
  primera part de l'objectiu 4 s'ha dissenyat un model per als SGBD de
  sèries temporals formalitzat amb expressions algebraiques. Aquest
  model, com ja s'ha dit, està fortament basat amb els conceptes de
  SGBD que descriu el model relacional. A partir de l'estudi de
  l'estat de l'art de les sèries temporals, hem observat que els SGBD
  per sèries temporals precisen d'operacions de tres naturaleses
  diferents: com a conjunts, com a seqüències i com a funció temporal.
  Aquestes naturaleses han comportat la planificació d'un estudi a
  part com ja s'ha dit, i així s'ha incorporat al model de SGST les
  propietats vàries que poden tenir les sèries temporals:
  representacions temporals, regularitat del temps entre mesures,
  forats de temps, etc. Tot i així, en el model estructural i
  d'operacions de SGST s'ha pogut separar-ho i per tant el model s'ha
  pogut completar independentment.



\item Disseny d'un model de SGBD multiresolució per sèries temporals
  (SGSTM). Per a la segona part de l'objectiu 4 s'ha dissenyat un
  model que contempla la multiresolució de les sèries temporals. S'ha
  refet aquest model per a utilitzar operacions definides en el model
  anterior per les sèries temporals. Aquest model també s'ha descrit a
  excepció d'un apartat que ha d'exemplificar com es duen a terme les
  agregacions d'atributs per a aconseguir la multiresolució.



\item[9.] Redacció d'articles i informes. A partir de l'estudi fet a
  la proposta de tesi, es va iniciar la redacció d'un informe de
  recerca a on constés les mancances que observem que tenen els SGBD
  per a les sèries temporals, les propietats i requisits que han de
  complir i la idea bàsica de la nostra proposta d'un nou model
  multiresolució per a sèries temporals. Para\l.lelament també es va
  preparar un article per a congrés per tal de donar a conèixer de
  forma resumida el model multiresolució que volem dissenyar. Aquestes
  dues tasques han conclòs amb dues publicacions que es detallen a
  l'apartat següent.

\end{enumerate}






\newpage

\section*{Pla de treball per als propers 12 mesos:}

El pla de treball futur pel al curs acadèmic 2013--2014 se centra en
la realització de la tesi en el doctorat en automàtica, robòtica i
visió.  A més, com a resultat del treball dut a terme fins ara, s'ha
actualitzat alguns dels objectius, com ja s'ha notat anteriorment.
%En el document de l'\autoref{sec:pla_treball} ''\nameref{sec:pla_treball}''
A continuació, es planifica el pla de treball futur a la figura
\ref{fig:pla:futur} i es detalla la feina a realitzar en el proper
any.


\begin{figure}[tp]
\centering
  \scalebox{0.8}{
  \begin{ganttchart}[
    vgrid,
    x unit=0.8cm,
    today=7.5,
    today label=PRESENT,
    title label font=\small,
    title height=1,
    bar label font = \small,
    bar/.style={draw=none, fill=black},
    %incomplete/.style={fill=lightgray},
    progress label text =,
    ]
    {12}{12}
    \gantttitle{2012}{4}
    \gantttitle{2013}{4}
    \gantttitle{2014}{4} \\
    \gantttitlelist{1,...,4}{1}\gantttitlelist{1,...,4}{1}\gantttitlelist{1,...,4}{1} \\

    \ganttmilestone{Proposta de tesi}{2} \\

    \ganttgroup{Estudi}{1}{4} \\
    \ganttbar[progress=100]{1. Aplicacions}{1}{2} \\
    \ganttbar{2. Model relacional}{1}{4} \\
    \ganttbar[progress=100]{%3. Intervals temporals
}{3}{3} 
    \ganttbar[progress=100]{3. Propietats ST}{7}{7} \\

    \ganttgroup[progress=90]{Disseny models}{5}{8} \\
    \ganttbar[progress=100]{4. Sèries temporals}{5}{7} \\
    \ganttbar[progress=100]{5. Multiresolució}{6}{6}
    \ganttbar[progress=50]{}{8}{8} \\
    \ganttbar[bar/.style={fill=white}]{5.b Estructures}{8}{8} \\
    \ganttlink{elem5}{elem9}

    \ganttgroup[progress=0]{Experimentació}{9}{10} \\
    \ganttbar[progress=0]{6. Implementació}{9}{9} \\
    \ganttlink{elem9}{elem11}
    \ganttbar[progress=0]{7. Dades experimentals}{9}{10} \\




   \ganttgroup[progress=20]{Redacció}{2}{10} \\

    \ganttbar[progress=0]{8. Redacció memòria}{9}{10} \\
    \ganttlinkedmilestone[milestone/.style={fill=lightgray}]{Dipòsit de tesi}{10} \\

    \ganttbar[progress=100]{9. Redacció articles}{4}{4} 
    \ganttbar[bar/.style={fill=white}]{}{9}{9} \\

    \ganttmilestone{Informe de recerca}{4} \\
    \ganttlink{elem17}{elem19}

    \ganttmilestone{AIKED'13}{4.5}
  \end{ganttchart}
  }
\caption{Planificació del treball pendent}
\label{fig:pla:futur}
\end{figure}


\begin{itemize}

\item[5.] Per al disseny del model de SGSTM queda, com ja s'ha dit,
  refer un apartat sobre agregadors d'atributs, els quals són les
  funcions que permeten als SGSTM calcular diverses resolucions d'una
  sèrie temporal. L'estudi en l'objectiu~3 de les propietats variades
  de les sèries temporals indueix a varis agregadors d'atributs en el
  model de SGSTM. Així, es donaran exemples que facilitin la
  comprensió de com de variats poden ser i serveixin de model per a
  dissenyar-ne altres.


\item[5.b] Durant l'estudi de l'estat actual de les sèries temporals
  es van observar diverses aproximacions a l'hora de fer càlculs amb
  sèries temporals i en general en els SGBD. Això ha comportat la
  definició d'una tercera part en l'objectiu~4 a on s'avaluaran
  diferents estructures possibles dels SGSTM.  La multiresolució té
  l'objectiu d'emmagatzemar les sèries temporals de forma compacta,
  així es preveu que alguns conceptes de la recerca en \emph{data
    streams} poden prendre-hi sentit.



\item[6.] Implementació de referència. Per a l'objectiu~5 s'implementaran
  els models de SGST i SGSTM utilitzant un llenguatge de programació adequat
  per a models, com per exemple \texttt{Python} o \texttt{Prolog}.  Es
  prioritzarà la implementació correcte del model enfront a una
  implementació que contempli un bon rendiment. 


\item[7.] Experimentació amb dades. Per a complementar l'objectiu~5 es
  provarà la implementació amb dades experimentals per alguna
  aplicació concreta. El suport del projecte \emph{i-Sense} (FP7-
  ICT-270428) aporta dades que són sèries temporals i són candidates a
  experimentar amb elles.  També la participació en el projecte
  \emph{NAPS} (TEC2012-35571) aporta l'objectiu~6 en el qual s'està
  pensant d'implementar una part reduïda del SGSTM a baix nivell com
  pot ser amb llenguatge VHDL. És a dir la implementació d'alguna de
  les estructures específiques de SGSTM estudiades prèviament a mode
  d'exemple de com introduir una base de dades multiresolució per a
  aplicacions molt concretes.




\item[9.] Redacció d'articles i informes. Un cop s'hagi completat el
  disseny del model de SGBD per a sèries temporals, es pot presentar
  com a article en l'àmbit de les bases de dades.  També en pot
  resultar un article un cop s'hagi realitzat la implementació.


\end{itemize}




%\newpage


% \section*{Pla de treball principal}
% %\label{sec:pla_treball}

% 
\begin{abstract}

  En aquest document es detalla el progrés en el pla de treball
  presentat el 2012 per a assolir la recerca en el disseny d'un model
  de sistema de gestió per a sèries temporals. En primer lloc es
  presenta l'actualització dels objectius, en segon lloc el progrés en
  funció de les tasques planificades realitzades i en tercer lloc la
  modificació del pla de treball per a assolir els objectius pendents.
\end{abstract}



\section{Objectius}
\label{sec:objectius}

Aquesta recerca té per objectiu l'estudi de les necessitats
específiques que comporta l'emmagatzematge i gestió de dades amb
naturalesa de sèrie temporal i la proposta d'un model de SGBD que
satisfaci aquestes necessitats. Aquest objectiu es divideix en els
següents subobjectius més concrets:

\begin{enumerate}

\item Estudi de les aplicacions en que les dades són sèries temporals
  amb la finalitat de determinar quines són les propietats i problemes
  comuns que planteja la seva gestió i emmagatzematge.

\item Estudi dels models de SGBD existents. Segons es desprèn de la
  formalització de \textcite{date:introduction} el model principal és
  el model relacional, el qual es fonamenta en dos conceptes:
  relacions i tipus de dades. 

\item Una àrea de treball important en els SGBD és la incorporació de
  nous tipus de dades complexos. És important estudiar com es modifica
  el model de dades d'un SGBD quan s'afegeix un nou tipus de dades
  complex.  Les sèries temporals es poden d'entendre com a tipus
  complex ja que presenten diferents propietats característiques i
  necessiten operadors addicionals.  

\item Disseny d'un model de SGBD per a les sèries temporals. D'aquesta
  manera els SGBD podran tractar dades amb instants de temps que
  mostrin l'evolució de variables en funció del temps. El model
  consisteix en la definició de l'estructura de les sèries temporals i
  les operacions bàsiques que necessiten.

  L'assoliment d'aquest objectiu té tres parts:

  \begin{enumerate}
  \item Disseny d'un model per a la gestió bàsica de les sèries
    temporals, el qual anomenem model de SGBD per a sèries temporals
    (SGST).  L'estructura d'aquest model és similar a l'utilitzat en
    els intervals
    temporals \parencite{date02:_tempor_data_relat_model}.  Prenent
    com a base el model de SGST, el qual és un model general per a les
    sèries temporals, s'hi poden incloure altres models per a
    propietats més específiques de les sèries temporals.

  \item Disseny d'un model específic en base del model de
    SGST. Concretament es dissenya un model pels SGST multiresolució
    (SGSTM).  En el model de SGSTM s'hi poden incloure propietats de
    les sèries temporals relacionades amb la resolució que s'han
    observat en les aplicacions pràctiques de les sèries temporals:
    regularització, canvis de resolució mitjançant agregacions,
    reconstrucció de forats, etc.
 
  \item Avaluació de diferents estructures de SGSTM. El model de SGSTM
    s'hi poden fer modificacions o simplificacions per tal
    d'aconseguir diferents estructures.  Per exemple bases de dades
    multiresolució que comparteixin informació, que treballin amb flux
    de dades (\emph{data stream}) o bé que s'especialitzin per a un
    tipus determinat de sèries temporals.

  \end{enumerate}

\item Implementació de referència dels models de SGST i SGSTM. Per una
  banda, aquesta implementació, a nivell acadèmic, ha de servir com a
  exemple per a futurs desenvolupaments de sistemes de gestió,
  acadèmics o productius. Per altra banda, ha de servir per a
  exemplificar-ne els seu funcionament amb unes dades de prova.

\item Implementació específica i reduïda del model per a una
  determinada aplicació de sèries temporals. Exemplificació de com una
  estructura de SGSTM pot ser implementada per a aconseguir una
  aplicació molt concreta.

\end{enumerate} 


El objectius 4.c i 6. són resultat d'una nova planificació com es
detalla més endavant a les tasques futures.






%%% Local Variables: 
%%% mode: latex
%%% TeX-master: "main"
%%% End: 
% LocalWords:  SGSTM multiresolució SGST


% \subsection*{Planificació del treball}


Per tal d'assolir els objectius detallats a la secció
\ref{sec:objectius}, a continuació es proposen les tasques a
realitzar.  A la figura \ref{fig:pla:futur} es detalla el pla de
treball amb temps estimat.





\begin{enumerate}


\item Estudi d'aplicacions de les sèries temporals. Per a assolir
  l'objectiu~1 estudiarem recerca actual de sèries temporals.
  Consultarem articles i llibres que tinguin les sèries temporals com
  a temàtica principal. També cercarem l'existència de programari que
  tingui en els seus objectius el tractament de sèries temporals.

\item Estudi del model relacional. Per a l'objectiu~2 estudiarem el
  model relacional com a referent pels models de SGBD. Ens basarem en
  l'estudi dels llibres de
  \textcite{date:introduction,date06,date:dictionary}. Usarem
  \emph{rel} \parencite{rel} com a implementació de referència ja que
  incorpora el llenguatge \emph{Tutorial D}, el qual és utilitzat per
  Date en els seus exemples.

\item Estudi de la gestió d'intervals temporals. Per a l'objectiu~3
  estudiarem la recerca que ha conduit a incorporar els interval
  temporals en els SGBD per a gestionar històrics. Ens basarem en
  l'estudi que relaciona el model relacional amb els intervals
  temporals de \textcite{date02:_tempor_data_relat_model}.

\item Disseny d'un model de SGBD per sèries temporals. Per a la
  primera part de l'objectiu 4 dissenyarem un model per als SGBD de
  sèries temporals formalitzat amb expressions algebraiques. A partir
  de l'estudi de l'estat de l'art de les sèries temporals, observarem
  les propietats interessants de ser modelitzades i les operacions que
  precisen els SGBD per sèries temporals.

\item Disseny d'un model de SGBD multiresolució per sèries
  temporals. Per a la segona part de l'objectiu 4 dissenyarem un model
  que contempli la multiresolució de les sèries temporals. Aquest
  model utilitzarà propietats del model anterior per les sèries
  temporals. La mulitresolució té l'objectiu d'emmagatzemar les sèries
  temporals de forma compacta, així es preveu que alguns conceptes de la recerca
  en \emph{data streams} poden prendre-hi sentit.

\item Implementació de referència. Per a l'objectiu~5 s'implementaran
  els models anteriors utilitzant un llenguatge de programació adequat
  per a models, com per exemple \texttt{Python} o \texttt{Prolog}.  Es
  prioritzarà la implementació correcte del model enfront a una
  implementació que contempli un bon rendiment. 

\item Experimentació amb dades. Per a complementar l'objectiu~5 es
  provarà la implementació amb dades experimentals per alguna
  aplicació concreta.

\end{enumerate}

\begin{figure}[tp]
\centering
\scalebox{0.8}{
\begin{gantt}[xunitlength=0.8cm,fontsize=\small,titlefontsize=\small]{15}{12}
  \begin{ganttitle}
    \numtitle{2012}{1}{2014}{4}
  \end{ganttitle}
  \begin{ganttitle}
    \numtitle{1}{1}{4}{1}
    \numtitle{1}{1}{4}{1}
    \numtitle{1}{1}{4}{1}
  \end{ganttitle}

  \ganttmilestone{Proposta de tesi}{2}

  \ganttgroup{Estudi}{0}{4}
  \ganttbar{1. Aplicacions}{0}{2}
  \ganttbar{2. Model relacional}{0}{4}
  \ganttbar{3. Intervals temporals}{2}{2}

  \ganttgroup{Disseny models}{4}{3}
  \ganttbar{4. Sèries temporals}{4}{3}
  \ganttbar{5. Multiresolució}{5}{2}
  \ganttcon{7}{7}{7}{11}

  \ganttgroup{Experimentació}{7}{2}
  \ganttbar{6. Implementació}{7}{2}
  \ganttbar{7. Dades experimentals}{8}{1}

  \ganttbar{8. Redacció memòria}{8}{2}

  \ganttmilestonecon{Lectura de tesi}{10}

\end{gantt}
}
\caption{Planificació del treball}
\label{fig:pla:futur}
\end{figure}



\subsubsection*{Treball realitzat}


La tesi de màster \parencite{llusa11:tfm} va consistir en l'estudi de
\emph{RRDtool} \parencite{rrdtool}, un sistema de gestió de bases de
dades (SGBD) específic per a sèries temporals.
%
Fruit d'aquest estudi es va formalitzar el model de \emph{RRDtool} i es va
dissenyar i implementar un prototip software que responia a la
formalització.
%
Aquest treball va permetre concloure que:
\begin{itemize}
\item Els SGBD aplicats a sèries temporals tenen aspectes propis que
  els converteixen en objecte d'estudi de \emph{per se}.
\item El concepte de multiresolució és interessant en moltes
  aplicacions reals, especialment quan existeixen restriccions d'espai
  per emmagatzemar dades.
\item El model proposat per a \emph{RDDtool} és susceptible de ser millorat
  com a mínim en el següents aspectes:
  \begin{itemize}
  \item Generalització. El model que es va presentar estava fortament
    lligat a \emph{RDDtool}. Generalitzar el model de forma que encabeixi
    altres concepcions.
  \item Incorporació d'operacions. El model presentat únicament feia
    referència a les dades. Per completar el
    model cal també considerar les operacions.
  \item Contextualització. Cal interrelacionar i descriure el model el
    context d'altres models existents per a SGBD. Específicament cal
    comparar-lo amb el model relacional usat en els SGBD
    convencionals.
  \end{itemize}
\item És convenient estudiar la feina feta per altres autors en
  l'àmbit de l'emmagatzemat i gestió de dades provinents de sèries
  temporals.
\end{itemize}

Arrel de la feina anterior, s'ha treballat en els següents aspectes:
\begin{itemize}
\item S'ha realitzat una tasca de recerca bibliogràfica i estudi dels
  treballs existents en l'àmbit de la gestió i emmagatzemat de dades
  provinents de sèries temporals. El resultat d'aquesta tasca s'ha
  reflectit a la proposta de tesi \parencite{llusa12:ptd}.
\item S'ha estudiat en profunditat el model relacional. Atesa la
  preeminència d'aquest model en els SGBD actuals, s'ha considerat
  imprescindible tenir-ne un bon coneixement que permetés estudiar les
  seves deficiències per la gestió de sèries temporals i en quina
  forma poden ser superades.
\item S'està refent el model de SGBD per sèries temporals.  S'ha
  dividit el model en dues parts ben diferenciades:
  \begin{enumerate}
  \item La primera és el model general. Aquest defineix la gestió de
    sèries temporals enteses com a co\l.lecció de dades mesurades en
    diferents instants de temps. Es basa en els conceptes de temps,
    mesura i sèrie temporal.
  \item La segona és el model de multiresolució. Aquest model explica
    la forma d'emmagatzemar una sèrie temporal amb diferents
    resolucions temporals.
  \end{enumerate}
\end{itemize}





\subsection*{Mitjans}

La recerca es duu a terme amb el suport de la Universitat Politècnica
de Catalunya (UPC) mitjançant una beca FPU-UPC adscrita al departament
d'Enginyeria del Disseny i Programació de Sistemes Electrònics
(DiPSE).

No es preveu un ús de mitjans més enllà de l'accés als
recursos bibliogràfics i d'eines informàtiques de programació i
gestió de documentació.



%%% Local Variables: 
%%% mode: latex
%%% TeX-master: "main"
%%% End: 






%%% Local Variables: 
%%% mode: latex
%%% TeX-master: "main"
%%% End: 
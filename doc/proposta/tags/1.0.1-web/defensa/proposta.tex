\subsection{Objectius}
\begin{frame}{Proposta. Objectius i contribucions esperades}


\begin{enumerate}

  \item Estudi  propietats i problemes comuns de la gestió de sèries temporals.

  \item Estudi de models de SGBD existents.  \\
    Model relacional com a referència principal.

  \item Estudi de la formalització de tipus de dades complexos als
    SGBD. Intervals temporals com a
    referència.

  \item Disseny d'un model de SGBD per a les sèries temporals. 

    \begin{enumerate}
    \item Disseny d'un model per a la gestió bàsica de les sèries
      temporals, anomenat model de SGBD per a sèries temporals (SGST).

    \item Disseny d'un model pels SGST multiresolució (SGSTM), model
      específic en base del model de SGST.
    \end{enumerate}

  \item Implementació de referència dels models de SGST i SGSTM. 

\end{enumerate}

\end{frame}




\subsection{Treball realitzat}
\begin{frame}{Proposta. Treball realitzat}

  \begin{itemize}
  \item Estudi de \emph{RRDtool} \parencite{rrdtool}, un SGBD
    específic per a sèries temporals. Tesi de
    màster \parencite{llusa11:tfm}. Treball futur: generalització, operacions i
    contextualització.

  \item Recerca bibliogràfica i estudi dels treballs existents en
    l'àmbit de la gestió i emmagatzematge de dades provinents de
    sèries temporals.

  \item Estudi profund del model relacional. 

  \item Estudi preliminar del model de SGBD per sèries temporals. S'ha
    dividit en dues parts: model general i model multiresolució.

  \end{itemize}

\end{frame}


\subsection{Planificació futura}
\begin{frame}{Proposta. Planificació futura}


  \begin{enumerate}

  \item Estudi d'aplicacions de les sèries temporals i programari que
    les tracti. Relació amb el treball desenvolupat.

  \item Estudi del model relacional \parencite{date:introduction,
      date06,date:dictionary, date:thethirdmanifesto}. Llenguatge
    \emph{Tutorial D}.

  \item Estudi de la gestió d'intervals temporals, incorporació en els
    SGBD relacionals \parencite{date02:_tempor_data_relat_model}.

  \item Disseny d'un model de SGBD per sèries temporals, formalització
    de l'estructura i les operacions amb expressions algebraiques.

  \item Disseny d'un model de SGBD multiresolució per sèries
    temporals. Emmagatzematge de les sèries temporals de forma compacta.

  \item Implementació de referència. Llenguatge: \texttt{Python} o
    \texttt{Prolog}.

  \item Experimentació amb dades.

\end{enumerate}


\end{frame}



\begin{frame}{Proposta. Planificació futura}

\begin{center}
\scalebox{0.7}{
\begin{gantt}[xunitlength=0.8cm,fontsize=\small,titlefontsize=\small]{15}{12}
  \begin{ganttitle}
    \numtitle{2012}{1}{2014}{4}
  \end{ganttitle}
  \begin{ganttitle}
    \numtitle{1}{1}{4}{1}
    \numtitle{1}{1}{4}{1}
    \numtitle{1}{1}{4}{1}
  \end{ganttitle}

  \ganttmilestone{Proposta de tesi}{2}

  \ganttgroup{Estudi}{0}{4}
  \ganttbar{1. Aplicacions}{0}{2}
  \ganttbar{2. Model relacional}{0}{4}
  \ganttbar{3. Intervals temporals}{2}{2}

  \ganttgroup{Disseny models}{4}{3}
  \ganttbar{4. Sèries temporals}{4}{3}
  \ganttbar{5. Multiresolució}{5}{2}
  \ganttcon{7}{7}{7}{11}

  \ganttgroup{Experimentació}{7}{2}
  \ganttbar{6. Implementació}{7}{2}
  \ganttbar{7. Dades experimentals}{8}{1}

  \ganttbar{8. Redacció memòria}{8}{2}

  \ganttmilestonecon{Lectura de tesi}{10}

\end{gantt}
}
\end{center}


\end{frame}




%%% Local Variables: 
%%% mode: latex
%%% TeX-master: "presentacio"
%%% End: 

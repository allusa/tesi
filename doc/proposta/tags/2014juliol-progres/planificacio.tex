\section{Tasques realitzades}



Per tal d'assolir els objectius detallats a la secció
\ref{sec:objectius}, el juliol de 2012 es va proposar un seguit de
tasques a realitzar i el juliol de 2013 es va mostrar el progrés i es
van actualitzar algunes tasques.  A la figura \ref{fig:pla:actual} es
detalla l'estat d'aquell pla de treball a dia d'avui. A continuació
s'expliquen les tasques realitzades des del juliol de 2013 fins a
l'actualitat.







\begin{figure}[tp]
\centering
  \scalebox{0.8}{
  \begin{ganttchart}[
    vgrid,
    x unit=0.8cm,
    today=10,
    today label=PRESENT,
    title label font=\small,
    title height=1,
    bar label font = \small,
    bar/.style={draw=none, fill=black},
    %incomplete/.style={fill=lightgray},
    progress label text =,
    ]
    {1}{12}
    \gantttitle{2012}{4}
    \gantttitle{2013}{4}
    \gantttitle{2014}{4} \\
    \gantttitlelist{1,...,4}{1}\gantttitlelist{1,...,4}{1}\gantttitlelist{1,...,4}{1} \\

    \ganttmilestone{Proposta de tesi}{2} \\

    \ganttgroup{Estudi}{1}{7} \\
    \ganttbar[progress=100]{1. Aplicacions}{1}{2} \\
    \ganttbar{2. Model relacional}{1}{4} \\
    \ganttbar{%3. Intervals temporals
}{3}{3} 
    \ganttbar{3. Propietats ST}{7}{7} \\

    \ganttgroup[progress=80]{Disseny models}{5}{8} \\
    \ganttbar[progress=100]{4. Sèries temporals}{5}{7} \\
    \ganttbar[progress=100,name=sgstm]{5. Multiresolució}{6}{7} \\
    \ganttbar[progress=100]{5.b Esquemes}{8}{8}
    \ganttbar[bar/.style={draw,pattern=north east lines}]{}{10}{10} \\

    \ganttgroup[progress=80]{Experimentació}{8}{9} \\
    \ganttbar[progress=100,name=python]{6. Impl. referència}{8}{9} \\
    \ganttbar[bar/.style={draw,pattern=north east lines}]{6.b Impl. para\l.lela}{10}{10} \\

    \ganttbar[progress=75]{7. Dades experimentals}{9}{10} \\

   \ganttlink{sgstm}{python}


   \ganttgroup[progress=75]{Redacció}{4}{10} \\

    \ganttbar[progress=80]{8. Redacció memòria}{9}{10} \\
    \ganttlinkedmilestone[milestone/.append style={color=lightgray}]{Lectura de tesi}{10} \\

    \ganttbar[progress=100,name=articles1]{9. Redacció articles}{4}{4} 
    \ganttbar[progress=100]{}{9}{9} 
    \ganttbar[bar/.style={draw,pattern=north east lines},name=articles2]{}{10}{10} \\

    \ganttmilestone[name=rep1]{Informe de recerca}{4} \\
    \ganttlink{articles1}{rep1}

    \ganttmilestone[name=aiked13]{Revistes/congr.}{4.5}
    \ganttmilestone[name=if14]{}{10.5}
    \ganttlink{articles2}{if14}
  \end{ganttchart}
  }

\caption{Tasques completades del treball}
\label{fig:pla:actual}
\end{figure}




\begin{description}



\item[Tasca 3.] Durant el disseny del model de SGST s'ha vist que la
  naturalesa com a funció temporal de les sèries temporals hi té un
  paper important; és a dir que aleshores els operadors han de
  treballar tenint en compte que una sèrie temporal és la
  representació d'un funció contínua. S'ha estudiat profundament
  aquesta propietat de les sèries temporals conjuntament amb les
  patologies que pot presentar. S'han trobat esquemes interessants de
  gestió per a dades massives que es poden utilitzar per a les sèries
  temporals, per exemple computació para\l.lela o computació en flux.


\item[Tasca 4.] En el model de SGST s'ha incorporat un apartat amb
  l'estudi i modelització del comportament que tenen les sèries
  temporals com a funció temporal. En aquest apartat també s'han
  descrit altres patologies que poden tenir les sèries temporals:
  regularitat del temps entre mesures, forats de temps, etc.


\item[Tasca 5.] En el model de SGSTM s'ha refet el disseny per a
  utilitzar les operacions definides en el model anterior de les
  sèries temporals. A més s'ha afegit un apartat per a exemplificar
  com es duen a terme les agregacions d'atributs per a aconseguir la
  multiresolució.



\item[Tasca 5.b] Durant l'estudi de l'estat actual de les sèries
  temporals s'han observat diverses aproximacions a l'hora de fer
  càlculs amb sèries temporals i en general en els SGBD. Això ha
  comportat la definició d'una tercera part en l'objectiu~4 on
  s'avaluen els esquemes de multiresolució.  La multiresolució té
  l'objectiu d'emmagatzemar les sèries temporals de forma compacta,
  així els conceptes de la recerca en \emph{data streams} hi prenen
  sentit.  Per altra banda, s'ha pogut expressar la multiresolució com
  una funció que aplicada a una sèrie temporal retorna una nova sèrie
  temporal resultant d'aplicar-hi l'esquema de multiresolució.
  Aquesta funció de multiresolució obre la possibilitat de noves
  recerques que convé explorar i es planifiquen a la secció següent.


\item[Tasca 6.] Implementació de referència. Per a l'objectiu~5 s'ha
  implementat els models de SGST i SGSTM utilitzant el llenguatge de
  programació \texttt{Python}. S'ha prioritzat la implementació
  correcte del model enfront a una implementació que contempli un bon
  rendiment. 

\item[Tasca 6.b] Implementació para\l.lela amb Hadoop. Per a
  l'objectiu~6 s'ha experimentat amb Hadoop i MapReduce per
  tal d'entendre la computació para\l.lela en els SGBD. S'ha començat
  a dissenyar una implementació del model de SGSTM a Hadoop.



\item[Tasca 7.] Experimentació amb dades. Per a complementar
  l'objectiu~5 s'ha provat la implementació amb dades experimentals
  per a una aplicació concreta. El suport del projecte \emph{i-Sense}
  (FP7- ICT-270428) ha aportat dades que són sèries temporals i amb
  les quals s'ha pogut experimentar.  S'ha rebut també unes dades
  noves amb irregularitats interessants per a aplicar-hi el model de
  multiresolució.



% També la participació en el
%   projecte \emph{NAPS} (TEC2012-35571) aporta l'objectiu~6 en el qual
%   s'està pensant d'implementar una part reduïda del SGSTM a baix
%   nivell com pot ser amb llenguatge VHDL. És a dir la implementació
%   d'alguna de les estructures específiques de SGSTM estudiades
%   prèviament a mode d'exemple de com introduir una base de dades
%   multiresolució per a aplicacions molt concretes.




\item[Tasca 9.] Redacció d'articles i informes. Un cop s'ha completat
  el disseny del model de SGBD per a sèries temporals, s'ha escrit
  compactament en format article per a l'àmbit de les bases de dades.
  En aquest article també s'ha inclòs el disseny de la implementació
  de referència dels model.  Aquesta tasca s'ha allargat més del
  previst: ha calgut fer una tasca intensa de recerca de revistes i
  congressos que siguin afins a l'àmbit on treballem, cosa que també
  ha causat una gran feina de motivació i de recerca de treballs
  relacionats en l'actualitat de la temàtica seleccionada.  L'article
  s'ha escrit conjuntament amb la Teresa Escobet Canal i el Sebastià
  Vila Marta i està pendent per presentar-lo a la revista
  \emph{Information Systems}.
\end{description}






\section{Tasques pendents}


Seguint el pla del juliol de 2013 s'han realitzat les tasques
previstes però han aparegut uns àmbits de la multiresolució que són
interessants de ser explorats. S'ha començat a estudiar i a
experimentar amb els esquemes de la multiresolució, a tal
efecte s'ha actualitzat alguns dels objectius detallats a la secció
\ref{sec:objectius}, i a causa d'això no s'ha pogut completar la
redacció de la tesi. Així doncs, es preveu mig any més de treball en
aquests nous temes de multiresolució per a poder completar la recerca
de la tesi doctoral.  A la figura \ref{fig:pla:futur} es detalla aquest
pla de treball futur.




\begin{figure}[tp]
\centering
  \scalebox{0.8}{
  \begin{ganttchart}[
    vgrid,
    x unit=0.8cm,
    today=10,
    today label=PRESENT,
    title label font=\small,
    title height=1,
    bar label font = \small,
    bar/.style={draw=none, fill=black},
    %incomplete/.style={fill=lightgray},
    progress label text =,
    ]
    {1}{12}
    \gantttitle{2012}{4}
    \gantttitle{2013}{4}
    \gantttitle{2014}{4} \\
    \gantttitlelist{1,...,4}{1}\gantttitlelist{1,...,4}{1}\gantttitlelist{1,...,4}{1} \\

    \ganttmilestone{Proposta de tesi}{2} \\

    \ganttgroup{Estudi}{1}{7} \\
    \ganttbar[progress=100]{1. Aplicacions}{1}{2} \\
    \ganttbar{2. Model relacional}{1}{4} \\
    \ganttbar{%3. Intervals temporals
}{3}{3} 
    \ganttbar{3. Propietats ST}{7}{7} \\

    \ganttgroup[progress=80]{Disseny models}{5}{11} \\
    \ganttbar[progress=100]{4. Sèries temporals}{5}{7} \\
    \ganttbar[progress=100,name=sgstm]{5. Multiresolució}{6}{7} \\
    \ganttbar[progress=100,name=extensio]{5.b Esquemes}{8}{8} 
    \ganttbar[progress=100]{}{10}{10} 
    \ganttbar[bar/.style={fill=white,draw}]{}{11}{11} \\

    \ganttgroup[progress=70]{Experimentació}{8}{11} \\
    \ganttbar[progress=100,name=python]{6. Impl. referència}{8}{9} \\
    \ganttbar[progress=100,name=hadoop]{6.b Impl. para\l.lela}{10}{10} 
    \ganttbar[bar/.style={fill=white,draw}]{}{11}{11} \\

    \ganttbar[progress=75]{7. Dades experimentals}{9}{10} 
    \ganttbar[bar/.style={fill=white,draw}]{}{11}{11} \\

   \ganttlink{sgstm}{python}
   \ganttlink{extensio}{hadoop}



   \ganttgroup[progress=75]{Redacció}{4}{12} \\

    \ganttbar[progress=80]{8. Redacció memòria}{9}{10} 
    \ganttbar[bar/.style={fill=white,draw}]{}{12}{12}\\
    \ganttlinkedmilestone[milestone/.append style={color=lightgray}]{Lectura de tesi}{12} \\

    \ganttbar[progress=100,name=articles1]{9. Redacció articles}{4}{4} 
    \ganttbar[progress=100]{}{9}{10} 
    \ganttbar[bar/.style={fill=white,draw}]{}{12}{12} \\

    \ganttmilestone[name=rep1]{Informe de recerca}{4} \\
    \ganttlink{articles1}{rep1}

    \ganttmilestone[name=aiked13]{Revistes/congr.}{4.5}
    \ganttmilestone[name=if14]{}{10.5}
    \ganttlink{articles2}{if14}
  \end{ganttchart}
  }
\caption{Planificació del treball pendent}
\label{fig:pla:futur}
\end{figure}



\begin{description}


\item[Tasca 5.b] Recerca en els esquemes de multiresolució.  A partir
  de la multiresolució expressada com una funció simplificada, que
  calcula en temps diferit, convé explorar noves recerques. En primer
  lloc, cal avaluar formalment aquesta funció de multiresolució
  respecte a tot el model proposat de SGSTM. En segon lloc, en aquest
  càlcul en temps diferit apareix la possibilitat de computar
  para\l.lelament. En tercer lloc, a partir d'aquesta funció pot ésser
  més senzill estudiar la selecció o pèrdua d'informació que implica
  usar un model de multiresolució.


\item[Tasca 6.b] S'ha començat a dissenya una nova implementació de la
  multiresolució en Hadoop basada en conceptes de computació
  para\l.lela en els SGBD. Un cop ben establert el model d'aquest
  esquema de multiresolució cal acabar d'implementar-lo i provar-lo
  amb les mateixes dades experimentals que s'usen per a la
  implementació de referència.


\item[Tasca 7.] Experimentació amb dades. Es disposa d'unes dades
  noves amb irregularitats interessants per a aplicar-hi el model de
  multiresolució. Són unes dades de sensors reals en un sistema de
  dipòsits d'aigües. Aquestes dades pertanyen a sensors
  interrelacionats, cosa que les fa més interessants per analitzar-hi
  l'efecte que hi un SGSTM i com pot ajudar a emmagatzemar-les de
  forma compacta.


\item[Tasca 9.] La nova tasca 5.b plantejada ofereix un nou tema en
  l'àmbit de la multiresolució que, si l'experimentació que hem
  plantejat és satisfactòria, pot conduir a la redacció d'un nou
  article. Aquesta redacció, però, s'allargaria més enllà de l'entrega
  de la tesi.

\end{description}


%%% Local Variables: 
%%% mode: latex
%%% TeX-master: "main"
%%% End: 
% LocalWords:  multiresolució

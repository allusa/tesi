\section{Tasques realitzades}



Per tal d'assolir els objectius detallats a la secció
\ref{sec:objectius}, el juliol de 2012 es va proposar un seguit de
tasques a realitzar i el juliol de 2013 es va mostrar el progrés i es
van actualitzar algunes tasques.  A la figura \ref{fig:pla:actual} es
detalla l'estat d'aquell pla de treball a dia d'avui. A continuació
s'expliquen les tasques realitzades des del juliol de 2013 fins a
l'actualitat.







\begin{figure}[tp]
\centering
  \scalebox{0.8}{
  \begin{ganttchart}[
    vgrid,
    x unit=0.8cm,
    today=10,
    today label=PRESENT,
    title label font=\small,
    title height=1,
    bar label font = \small,
    bar/.style={draw=none, fill=black},
    %incomplete/.style={fill=lightgray},
    progress label text =,
    ]
    {1}{12}
    \gantttitle{2012}{4}
    \gantttitle{2013}{4}
    \gantttitle{2014}{4} \\
    \gantttitlelist{1,...,4}{1}\gantttitlelist{1,...,4}{1}\gantttitlelist{1,...,4}{1} \\

    \ganttmilestone{Proposta de tesi}{2} \\

    \ganttgroup{Estudi}{1}{7} \\
    \ganttbar[progress=100]{1. Aplicacions}{1}{2} \\
    \ganttbar{2. Model relacional}{1}{4} \\
    \ganttbar{%3. Intervals temporals
}{3}{3} 
    \ganttbar{3. Naturalesa ST}{7}{7} \\
%%NOVA    \ganttbar[bar/.style={draw,fill=white}]{4. Computació ST}{10}{11} \\

    \ganttgroup[progress=66]{Disseny models}{5}{8} \\
    \ganttbar[progress=75]{4. Sèries temporals}{5}{7} \\
    \ganttbar[progress=75]{5. Multiresolució}{6}{7} \\
    \ganttbar[bar/.style={fill=white,draw}]{5.b Estructures}{8}{8} \\

    \ganttgroup[progress=0]{Experimentació}{8}{9} \\
    \ganttbar[progress=100]{6. Implementació}{8}{9} \\
%NOVA    \ganttbar[progress=0]{6.b Para\l.lela}{10}{10} \\
    \ganttbar[progress=80]{7. Dades experimentals}{9}{10} \\




   \ganttgroup[progress=20]{Redacció}{2}{10} \\

    \ganttbar[progress=0]{8. Redacció memòria}{9}{10} \\
    \ganttlinkedmilestone[milestone/.style={fill=lightgray}]{Lectura de tesi}{10} \\

    \ganttbar[progress=100]{9. Redacció articles}{4}{4} 
    \ganttbar[progress=100]{}{9}{9} 
    \ganttbar[bar/.style={fill=white,draw}]{}{10}{10} \\

    \ganttmilestone{Informe de recerca}{4} \\
%    \ganttlink{elem16}{elem18}

    \ganttmilestone{AIKED'13}{4.5}
  \end{ganttchart}
  }

\caption{Tasques completades del treball}
\label{fig:pla:actual}
\end{figure}



















\todo{}


\begin{enumerate}

% \item Estudi d'aplicacions de les sèries temporals. Per a assolir
%   l'objectiu~1 s'ha realitzat una tasca de recerca bibliogràfica i
%   estudi dels treballs existents en l'àmbit de la gestió i
%   emmagatzemat de dades provinents de sèries temporals. També s'ha
%   cercat l'existència de programari que tingui en els seus objectius
%   el tractament de sèries temporals. El resultat d'aquesta tasca es va
%   reflectir a la proposta de
%   tesi \parencite{llusa12:ptd}. Anteriorment a la tesi de
%   màster \parencite{llusa11:tfm} ja s'havia estudiat profundament
%   \emph{RRDtool} \parencite{rrdtool}, un sistema de gestió de bases de
%   dades (SGBD) específic per a sèries temporals.


% \item Estudi del model relacional. Per a l'objectiu~2 s'ha estudiat en
%   profunditat el model relacional, prenent-lo com a referent pels
%   models de SGBD. Atesa la preeminència d'aquest model en els SGBD
%   actuals, es va considerar imprescindible tenir-ne un bon coneixement
%   que permetés estudiar les seves deficiències per la gestió de sèries
%   temporals i en quina forma poden ser superades. Aquest estudi s'ha
%   utilitzat a l'objectiu~4 i ha servit per establir el punt de partida
%   del model de SGST i la referència de conceptes bàsics pels SGBD.


% \item Estudi de la gestió d'intervals temporals. Per a l'objectiu~3
%   s'ha estudiat la recerca que ha conduit a incorporar els interval
%   temporals en els SGBD per a gestionar històrics. Ens hem basat en
%   l'estudi que relaciona el model relacional amb els intervals
%   temporals de \textcite{date02:_tempor_data_relat_model}.  Aquest
%   estudi ha servit per a complementar l'estudi del model relacional,
%   ja que es detalla com es poden modelar els intervals temporals
%   prenent com a base els SGBD relacionals. Tot i així, en un principi
%   presenten molta diferència amb les propietats de les sèries
%   temporals i s'ha trobat més adequat deixar aquest estudi fins que
%   s'hagi acabat el disseny del model de SGST per aleshores poder-los
%   comparar entre ells quan tinguem més detall sobre la naturalesa de
%   les sèries temporals.



% \item Disseny d'un model de SGBD per sèries temporals (SGST). Per a la
%   primera part de l'objectiu 4 s'ha dissenyat un model per als SGBD de
%   sèries temporals formalitzat amb expressions algebraiques. Aquest
%   model, com ja s'ha dit, està fortament basat amb els conceptes de
%   SGBD que descriu el model relacional. A partir de l'estudi de
%   l'estat de l'art de les sèries temporals, hem observat que els SGBD
%   per sèries temporals precisen d'operacions de tres naturaleses
%   diferents: com a conjunts, com a seqüències i com a funció temporal.
%   Aquesta última ha comportat la planificació d'un estudi a part com
%   es detalla a les tasques futures.  Així, a excepció d'aquesta tasca
%   separada, el model de SGST s'ha descrit completament i actualment s'està
%   revisant.



% \item Disseny d'un model de SGBD multiresolució per sèries temporals
%   (SGSTM). Per a la segona part de l'objectiu 4 s'ha dissenyat un
%   model que preveu la multiresolució de les sèries
%   temporals. Aquest model s'ha descrit i actualment s'està refent per
%   a utilitzar operacions definides en el model anterior per les sèries
%   temporals.





\item[Tasca 3.] Durant el disseny del model de SGST s'ha vist que la
  naturalesa com a funció temporal de les sèries temporals hi té un
  paper important; és a dir que aleshores els operadors han de
  treballar tenint en compte que una sèrie temporal és la
  representació d'un funció contínua. S'ha estudiat profundament
  aquesta propietat de les sèries temporals conjuntament amb les
  patologies que pot presentar. S'han trobat esquemes interessants de
  gestió per a dades massives que es poden utilitzar per a les sèries
  temporals, per exemple computació para\l.lela o computació en flux.
\todo{aquests esquemes els volem estudiar més profundament}



\item[Tasca 4.] En el model de SGST s'ha incorporat un apartat amb
  l'estudi i modelització del comportament que tenen les sèries
  temporals com a funció temporal. En aquest apartat també s'han
  descrit altres patologies que poden tenir les sèries temporals:
  regularitat del temps entre mesures, forats de temps, etc.


\item[Tasca 5.b] Durant l'estudi de l'estat actual de les sèries temporals
  es van observar diverses aproximacions a l'hora de fer càlculs amb
  sèries temporals i en general en els SGBD. Això ha comportat la
  definició d'una tercera part en l'objectiu~4 a on s'avaluaran
  diferents estructures possibles dels SGSTM.  La multiresolució té
  l'objectiu d'emmagatzemar les sèries temporals de forma compacta,
  així es preveu que alguns conceptes de la recerca en \emph{data
    streams} poden prendre-hi sentit.
\todo{ això ja s'ha dit a la tasca 3?}


\item[Tasca 6.] Implementació de referència. Per a l'objectiu~5 s'ha
  implementat els models de SGST i SGSTM utilitzant el llenguatge de
  programació \texttt{Python}. S'ha prioritzat la implementació
  correcte del model enfront a una implementació que contempli un bon
  rendiment. 



\item[Tasca 7.] Experimentació amb dades. Per a complementar
  l'objectiu~5 s'ha provat la implementació amb dades experimentals per
  a una aplicació concreta. El suport del projecte \emph{i-Sense}
  (FP7- ICT-270428) ha aportat dades que són sèries temporals i s'ha
  pogut experimentar amb elles.  
\todo{dir que han aparegut les dades de Xipre?}

% També la participació en el
%   projecte \emph{NAPS} (TEC2012-35571) aporta l'objectiu~6 en el qual
%   s'està pensant d'implementar una part reduïda del SGSTM a baix
%   nivell com pot ser amb llenguatge VHDL. És a dir la implementació
%   d'alguna de les estructures específiques de SGSTM estudiades
%   prèviament a mode d'exemple de com introduir una base de dades
%   multiresolució per a aplicacions molt concretes.




\item[9.] Redacció d'articles i informes. Un cop s'ha completat el
  disseny del model de SGBD per a sèries temporals, s'ha escrit
  compactament en format article per a l'àmbit de les bases de dades.
  En aquest article també s'ha inclòs el disseny de la implementació
  de referència dels model. Aquesta tasca s'ha allargat més del
  previst sobretot a causa de l'ampliació de l'estat de l'art que s'ha
  hagut de fer.  L'article està pendent de presentar-se com es detalla
  a l'apartat següent.  \todo{pendent de presentar?}



\end{enumerate}




TASQUES NOVES
\todo{}

\begin{enumerate}

\item[Tasca 6.b] Per altra banda, s'ha implementat amb Haddoop\todo{s'ha de dir que hem extret la funció de multiresolució, això ha de ser una tasca nova que s'ha fet!}


\end{enumerate}






\subsection{Publicacions pendents de presentar}
\todo{}

\begin{itemize}

\item Article a revista conjuntament amb la Teresa Escobet-Canal i el
  Sebastià Vila-Marta. Pendent de presentar a \emph{Information Systems}.

\end{itemize}




\section{Tasques pendents}

Seguint el pla del juliol de 2012 queden tot un seguit de tasques a
realitzar. A més, com a resultat del treball dut a terme fins ara, s'ha
actualitzat alguns dels objectius detallats a la secció
\ref{sec:objectius}.  A la figura \ref{fig:pla:futur} es detalla aquest
pla de treball futur.

\begin{figure}[tp]
\centering
  \scalebox{0.8}{
  \begin{ganttchart}[
    vgrid,
    x unit=0.8cm,
    today=6,
    today label=PRESENT,
    title label font=\small,
    title height=1,
    bar label font = \small,
    bar/.style={draw=none, fill=black},
    %incomplete/.style={fill=lightgray},
    progress label text =,
    ]
    {12}{12}
    \gantttitle{2012}{4}
    \gantttitle{2013}{4}
    \gantttitle{2014}{4} \\
    \gantttitlelist{1,...,4}{1}\gantttitlelist{1,...,4}{1}\gantttitlelist{1,...,4}{1} \\

    \ganttmilestone{Proposta de tesi}{2} \\

    \ganttgroup{Estudi}{1}{4} \\
    \ganttbar[progress=100]{1. Aplicacions}{1}{2} \\
    \ganttbar{2. Model relacional}{1}{4} \\
    \ganttbar[progress=50]{%3. Intervals temporals
}{3}{4} 
    \ganttbar[bar/.style={fill=white}]{3. Naturalesa ST}{7}{7} \\

    \ganttgroup[progress=66]{Disseny models}{5}{8} \\
    \ganttbar[progress=75]{4. Sèries temporals}{5}{7} \\
    \ganttbar[progress=75]{5. Multiresolució}{6}{7} \\
    \ganttbar[bar/.style={fill=white}]{5.b Estructures}{8}{8} \\

    \ganttgroup[progress=0]{Experimentació}{8}{9} \\
    \ganttbar[progress=0]{6. Implementació}{8}{9} \\
    \ganttlink{elem8}{elem11}
    \ganttbar[progress=0]{7. Dades experimentals}{9}{10} \\




   \ganttgroup[progress=20]{Redacció}{2}{10} \\

    \ganttbar[progress=0]{8. Redacció memòria}{9}{10} \\
    \ganttlinkedmilestone[milestone/.style={fill=lightgray}]{Lectura de tesi}{10} \\

    \ganttbar[progress=100]{9. Redacció articles}{4}{4} 
    \ganttbar[bar/.style={fill=white}]{}{9}{9} \\

    \ganttmilestone{Informe de recerca}{4} \\
    \ganttlink{elem16}{elem18}

    \ganttmilestone{AIKED'13}{4.5}
  \end{ganttchart}
  }
\caption{Planificació del treball pendent}
\label{fig:pla:futur}
\end{figure}



%%% Local Variables: 
%%% mode: latex
%%% TeX-master: "main"
%%% End: 
% LocalWords:  multiresolució

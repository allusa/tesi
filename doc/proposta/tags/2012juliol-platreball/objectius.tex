
\begin{abstract}

  En aquest document es detalla el pla de treball per a assolir la
  recerca en el disseny d'un model de sistema de gestió per a sèries
  temporals. En primer lloc es presenten els objectius, en segon lloc
  la planificació futura del treball per a assolir els objectius
  proposats a on es detalla el treball ja dut a terme fins a
  l'actualitat, i en tercer lloc els mitjans necessaris previstos.

\end{abstract}



\section{Objectius}
\label{sec:objectius}

Aquesta recerca té per objectiu l'estudi de les necessitats
específiques que comporta l'emmagatzematge i gestió de dades amb
naturalesa de sèrie temporal i la proposta d'un model de SGBD que
satisfaci aquestes necessitats. Aquest objectiu es divideix en els
següents subobjectius més concrets:

\begin{enumerate}

\item Estudi de les aplicacions en que les dades són sèries temporals
  amb la finalitat de determinar quines són les propietats i problemes
  comuns que planteja la seva gestió i emmagatzematge.

\item Estudi dels models de SGBD existents. Segons es desprèn de la
  formalització de \textcite{date:introduction} el model principal és
  el model relacional, el qual es fonamenta en dos conceptes:
  relacions i tipus de dades. 

\item Una àrea de treball important en els SGBD és la incorporació de
  nous tipus de dades complexos. És important estudiar com es modifica
  el model de dades d'un SGBD quan s'afegeix un nou tipus de dades
  complex.  Les sèries temporals es poden d'entendre com a tipus
  complex ja que presenten diferents propietats característiques i
  necessiten operadors addicionals.  Els SGBD permeten que els usuaris
  defineixin nous tipus de dades \parencite{stonebraker86} però no hi
  ha un estudi teòric dels tipus de dades en els SGBD.
  \textcite{date:introduction} descriu abastament les relacions però
  no els tipus de dades. Els tipus de dades s'han d'estudiar i modelar
  per a poder-los tractar i generar operadors, oimés els tipus
  complexos ja que requereixen un estudi més complet i possiblement
  s'hagin de modelar com un propi SGBD. Una referència d'estudi és el
  cas dels intervals
  temporals \parencite{date02:_tempor_data_relat_model}.

\item Disseny d'un model de SGBD per a les sèries temporals. D'aquesta
  manera els SGBD podran tractar dades amb instants de temps que
  mostrin l'evolució de variables en funció del temps. El model
  consisteix en la definició de l'estructura de les sèries temporals i
  les operacions bàsiques que necessiten.

  L'assoliment d'aquest objectiu té dues parts:

  \begin{enumerate}
  \item Disseny d'un model per a la gestió bàsica de les sèries
    temporals, el qual anomenem model de SGBD per a sèries temporals
    (SGST).  L'estructura d'aquest model és similar a l'utilitzat en
    els intervals
    temporals \parencite{date02:_tempor_data_relat_model}.  Prenent
    com a base el model de SGST, el qual és un model general per a les
    sèries temporals, s'hi poden incloure altres models per a
    propietats més específiques de les sèries temporals.

  \item Disseny d'un model específic en base del model de
    SGST. Concretament es dissenya un model pels SGST multiresolució
    (SGSTM).  En el model de SGSTM s'hi poden incloure propietats de
    les sèries temporals relacionades amb la resolució que s'han
    observat en les aplicacions pràctiques de les sèries temporals:
    regularització, canvis de resolució mitjançant agregacions,
    reconstrucció de forats, etc.
    % (nota: a més aquesta part del model és la més sensible a ser
    % implementada com a data streams)
  \end{enumerate}

\item Implementació de referència dels models de SGST i SGSTM. Per una
  banda, aquesta implementació, a nivell acadèmic, ha de servir com a
  exemple per a futurs desenvolupaments de sistemes de gestió,
  acadèmics o productius. Per altra banda, ha de servir per a
  exemplificar-ne els seu funcionament amb unes dades de prova.


%5. Possible exemplificació en algun cas pràctic?

\end{enumerate} 








%%% Local Variables: 
%%% mode: latex
%%% TeX-master: "main"
%%% End: 
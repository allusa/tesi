\section{Tasques realitzades}



\todo{s'ha obtingut beca AGAUR}



\subsection{Planificació 2012}


Per tal d'assolir els objectius detallats a la secció
\ref{sec:objectius}, a continuació es proposen les tasques a
realitzar.  A la figura \ref{fig:pla:futur} es detalla el pla de
treball amb temps estimat.





\begin{enumerate}


\item Estudi d'aplicacions de les sèries temporals. Per a assolir
  l'objectiu~1 estudiarem recerca actual de sèries temporals.
  Consultarem articles i llibres que tinguin les sèries temporals com
  a temàtica principal. També cercarem l'existència de programari que
  tingui en els seus objectius el tractament de sèries temporals.

\item Estudi del model relacional. Per a l'objectiu~2 estudiarem el
  model relacional com a referent pels models de SGBD. Ens basarem en
  l'estudi dels llibres de
  \textcite{date:introduction,date06,date:dictionary}. Usarem
  \emph{rel} \parencite{rel} com a implementació de referència ja que
  incorpora el llenguatge \emph{Tutorial D}, el qual és utilitzat per
  Date en els seus exemples.

\item Estudi de la gestió d'intervals temporals. Per a l'objectiu~3
  estudiarem la recerca que ha conduit a incorporar els interval
  temporals en els SGBD per a gestionar històrics. Ens basarem en
  l'estudi que relaciona el model relacional amb els intervals
  temporals de \textcite{date02:_tempor_data_relat_model}.

\item Disseny d'un model de SGBD per sèries temporals. Per a la
  primera part de l'objectiu 4 dissenyarem un model per als SGBD de
  sèries temporals formalitzat amb expressions algebraiques. A partir
  de l'estudi de l'estat de l'art de les sèries temporals, observarem
  les propietats interessants de ser modelitzades i les operacions que
  precisen els SGBD per sèries temporals.

\item Disseny d'un model de SGBD multiresolució per sèries
  temporals. Per a la segona part de l'objectiu 4 dissenyarem un model
  que contempli la multiresolució de les sèries temporals. Aquest
  model utilitzarà propietats del model anterior per les sèries
  temporals. La mulitresolució té l'objectiu d'emmagatzemar les sèries
  temporals de forma compacta, així es preveu que alguns conceptes de la recerca
  en \emph{data streams} poden prendre-hi sentit.

\item Implementació de referència. Per a l'objectiu~5 s'implementaran
  els models anteriors utilitzant un llenguatge de programació adequat
  per a models, com per exemple \texttt{Python} o \texttt{Prolog}.  Es
  prioritzarà la implementació correcte del model enfront a una
  implementació que contempli un bon rendiment. 

\item Experimentació amb dades. Per a complementar l'objectiu~5 es
  provarà la implementació amb dades experimentals per alguna
  aplicació concreta.

\end{enumerate}

\begin{figure}[tp]
  \centering
%[xunitlength=0.8cm,fontsize=\small,titlefontsize=\small]
  \begin{ganttchart}[
    vgrid,
    x unit=0.8cm,
    today=6,
    today label=PRESENT,
    title label font=\small,
    title height=1]
    {12}{12}
    \gantttitle{2012}{4}
    \gantttitle{2013}{4}
    \gantttitle{2014}{4} \\
    \gantttitlelist{1,...,4}{1}\gantttitlelist{1,...,4}{1}\gantttitlelist{1,...,4}{1} \\

    \ganttmilestone[milestone/.style={fill=green}]{Proposta de tesi}{2} \\

    \ganttgroup{Estudi}{1}{4} \\
    \ganttbar[progress=100]{1. Aplicacions}{1}{2} \\
    \ganttbar{2. Model relacional}{1}{4} \\
    \ganttbar[progress=50]{3. Intervals temporals}{3}{4} \\

    \ganttgroup{Disseny models}{5}{7} \\
    \ganttbar{4. Sèries temporals}{5}{7} \\
    \ganttbar{5. Multiresolució}{6}{7} \\

    \ganttgroup{Experimentació}{8}{9} \\
    \ganttbar{6. Implementació}{8}{9} \\
    \ganttlink{elem5}{elem9}
    \ganttbar{7. Dades experimentals}{9}{9} \\

    \ganttbar{8. Redacció memòria}{9}{10} \\

    \ganttlinkedmilestone{Lectura de tesi}{10}
  \end{ganttchart}
  
%\scalebox{0.8}{
%\begin{ganttchart}%[xunitlength=0.8cm,fontsize=\small,titlefontsize=\small]{15}{12}
  % \begin{ganttitle}
  %   \numtitle{2012}{1}{2014}{4}
  % \end{ganttitle}
  % \begin{ganttitle}
  %   \numtitle{1}{1}{4}{1}
  %   \numtitle{1}{1}{4}{1}
  %   \numtitle{1}{1}{4}{1}
  % \end{ganttitle}

  % \ganttmilestone[color=green]{Proposta de tesi}{2}

  % \ganttgroup{Estudi}{0}{4}
  % \ganttbar[color=green]{1. Aplicacions}{0}{2}
  % \ganttbar[color=green]{2. Model relacional}{0}{4}
  % \ganttbar{3. Intervals temporals}{2}{2}

  % \ganttgroup{Disseny models}{4}{3}
  % \ganttbar[color=green]{4. Sèries temporals}{4}{3}
  % \ganttbar[color=green]{5. Multiresolució}{5}{2}
  % \ganttbar[color=green]{5. Multiresolució}{7}{2}
  % \ganttcon{7}{7}{7}{11}

  % \ganttgroup{Experimentació}{7}{2}
  % \ganttbar{6. Implementació}{7}{2}
  % \ganttbar{7. Dades experimentals}{8}{1}

  % \ganttbar{8. Redacció memòria}{8}{2}

  % \ganttmilestonecon{Lectura de tesi}{10}

%\end{ganttchart}
%}
\caption{Tasques completades del treball}
\label{fig:pla:futur}
\end{figure}



\subsection{Treball realitzat}


La tesi de màster \parencite{llusa11:tfm} va consistir en l'estudi de
\emph{RRDtool} \parencite{rrdtool}, un sistema de gestió de bases de
dades (SGBD) específic per a sèries temporals.
%
Fruit d'aquest estudi es va formalitzar el model de \emph{RRDtool} i es va
dissenyar i implementar un prototip software que responia a la
formalització.
%
Aquest treball va permetre concloure que:
\begin{itemize}
\item Els SGBD aplicats a sèries temporals tenen aspectes propis que
  els converteixen en objecte d'estudi de \emph{per se}.
\item El concepte de multiresolució és interessant en moltes
  aplicacions reals, especialment quan existeixen restriccions d'espai
  per emmagatzemar dades.
\item El model proposat per a \emph{RDDtool} és susceptible de ser millorat
  com a mínim en el següents aspectes:
  \begin{itemize}
  \item Generalització. El model que es va presentar estava fortament
    lligat a \emph{RDDtool}. Generalitzar el model de forma que encabeixi
    altres concepcions.
  \item Incorporació d'operacions. El model presentat únicament feia
    referència a les dades. Per completar el
    model cal també considerar les operacions.
  \item Contextualització. Cal interrelacionar i descriure el model el
    context d'altres models existents per a SGBD. Específicament cal
    comparar-lo amb el model relacional usat en els SGBD
    convencionals.
  \end{itemize}
\item És convenient estudiar la feina feta per altres autors en
  l'àmbit de l'emmagatzemat i gestió de dades provinents de sèries
  temporals.
\end{itemize}

Arrel de la feina anterior, s'ha treballat en els següents aspectes:
\begin{itemize}
\item S'ha realitzat una tasca de recerca bibliogràfica i estudi dels
  treballs existents en l'àmbit de la gestió i emmagatzemat de dades
  provinents de sèries temporals. El resultat d'aquesta tasca s'ha
  reflectit a la proposta de tesi \parencite{llusa12:ptd}.
\item S'ha estudiat en profunditat el model relacional. Atesa la
  preeminència d'aquest model en els SGBD actuals, s'ha considerat
  imprescindible tenir-ne un bon coneixement que permetés estudiar les
  seves deficiències per la gestió de sèries temporals i en quina
  forma poden ser superades.
\item S'està refent el model de SGBD per sèries temporals.  S'ha
  dividit el model en dues parts ben diferenciades:
  \begin{enumerate}
  \item La primera és el model general. Aquest defineix la gestió de
    sèries temporals enteses com a co\l.lecció de dades mesurades en
    diferents instants de temps. Es basa en els conceptes de temps,
    mesura i sèrie temporal.
  \item La segona és el model de multiresolució. Aquest model explica
    la forma d'emmagatzemar una sèrie temporal amb diferents
    resolucions temporals.
  \end{enumerate}
\end{itemize}



\subsection{Eines utilitzades}

Eines informàtiques de programació:

Eines de gestió de documentació: s'està fent la documentació a escriny.epsem.upc.edu/rrb





\section{Nova planificació}


\begin{figure}[tp]
\centering
% \scalebox{0.8}{
% \begin{gantt}[xunitlength=0.8cm,fontsize=\small,titlefontsize=\small]{15}{12}
%   \begin{ganttitle}
%     \numtitle{2012}{1}{2014}{4}
%   \end{ganttitle}
%   \begin{ganttitle}
%     \numtitle{1}{1}{4}{1}
%     \numtitle{1}{1}{4}{1}
%     \numtitle{1}{1}{4}{1}
%   \end{ganttitle}

%   \ganttmilestone{Proposta de tesi}{2}

%   \ganttgroup{Estudi}{0}{4}
%   \ganttbar{1. Aplicacions}{0}{2}
%   \ganttbar{2. Model relacional}{0}{4}
%   \ganttbar{3. Intervals temporals}{2}{2}

%   \ganttgroup{Disseny models}{4}{3}
%   \ganttbar{4. Sèries temporals}{4}{3}
%   \ganttbar{5. Multiresolució}{5}{2}
%   \ganttcon{7}{7}{7}{11}

%   \ganttgroup{Experimentació}{7}{2}
%   \ganttbar{6. Implementació}{7}{2}
%   \ganttbar{7. Dades experimentals}{8}{1}

%   \ganttbar{8. Redacció memòria}{8}{2}

%   \ganttmilestonecon{Lectura de tesi}{10}

% \end{gantt}
% }
\caption{Planificació del treball pendent}
\label{fig:pla:futur}
\end{figure}



%%% Local Variables: 
%%% mode: latex
%%% TeX-master: "main"
%%% End: 
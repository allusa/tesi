\mbox{}\vspace{1cm}

So\l.licitud de renovació:

\vfill

\textbf{Dades personals}

Cognoms i nom: Llusà Serra, Aleix

\underline{Adreça particular}

Carrer, núm.\ i pis: Av.\ %\todo{dades}

Codi postal: 08243   Ciutat: Manresa

Telèfon:      Adreça electrònica: 


\vspace{2cm}


\textbf{Dades de la beca}

Departament: Eng.\ Disseny i Pro.\ Sist.\ Electrònics (DiPSE)

Adreça del departament: EPSEM-UPC, Av.\ Bases de Manresa 61--73, Manresa

Campus: Manresa  Telèfon: 938777254

Director de la beca: Sebastià Vila-Marta

Títol del treball de tesi: Disseny i modelització d'un sistema de gestió multiresolució de sèries temporals


\vfill

Manresa, a 19 d'octubre de 2012

\vspace{2cm}

Signatura del becari





\newpage

\renewcommand*\descriptionlabel[1]{\hspace\labelsep\normalfont #1:}

\section*{Memòria de la feina realitzada al llarg dels darrers 12 mesos:}

\subsection*{FORMACIÓ}

\begin{description}


 % S'ha assistit a un curs de  recorreguts geològics i geonaturalístics per l'Atlas (Marroc) a càrrec dels professors Kamal Tarquisti i Josep M. Mata-Perelló organitzat per l'EPSEM-UPC. Marroc, abril 2012 – 30 hores.

\item[Juliol 2012] Curs de fonaments en modelat i simulació de
  sistemes híbrids i control reset. Professor Sebastián
  Dormido. Programa de doctorat ARV-UPC, Terrassa, 10 hores.

\end{description}



\subsection*{RECERCA}

\begin{description}

\item[Curs 2011--2012] S'ha treballat en el projecte de tesi sota la
  direcció de la Dra.\ Teresa Escobet Canal i el Dr.\ Sebastià
  Vila-Marta.

\item[Març 2012] Alta com a membre del projecte d'investigació
  TEC2009-09924 ''Aplicaciones avanzadas de la inestabilidad
  controlada en circuitos a sistemas de comunicación
  (AVIC)''. Investigador principal Pere Palà Schönwälder (DiPSE-UPC).

\item[Juny 2012] Entregada i defensada la proposta de tesi en el marc
  del doctorat en automàtica, robòtica i visió amb el resultat de
  satisfactòria. La proposta s'ha titulat ''Disseny i modelització
  d'un sistema de gestió multiresolució de sèries temporals''. En la
  memòria d'aquesta proposta es presenta la recerca que estem duent a
  terme en l'àmbit dels sistemes d'emmagatzematge i tractament de
  dades com a sèries temporals amb l'objectiu principal de dissenyar
  un model d'un sistema que tracti sèries temporals amb
  multiresolució. Es defineix els objectius i la justificació de la
  recerca, s'estudia el context i l'estat actual d'altra recerca en
  aquests àmbits, i es presenta el treball dut a terme i la
  planificació del treball futur.

\item[Setembre 2012] Renovada la matrícula de la tutoria de tesi pel
  curs acadèmic 2012--2013 al doctorat ARV.

\item[Octubre 2012] Aprovada la concessió del projecte d'investigació
  TEC2012-35571 ''Nuevas aplicaciones del principio superregenerativo
  a comunicaciones por radiofrecuencia (NASP)'' per als pròxims 3 anys
  en finalitzar l'anterior projecte AVIC. Investigador principal Pere
  Palà Schönwälder.


\end{description}



\subsection*{DOCÈNCIA}

\begin{description}

\item[Tardor 2011] 30 h de laboratori d'Informàtica en el grau en
  enginyeria de sistemes TIC a l'EPSEM -- UPC.

\item[Primavera 2012] 30 h de laboratori de Tecnologia de la
  Programació en el grau en enginyeria de sistemes TIC a l'EPSEM --
  UPC.

\end{description}



\newpage

\section*{Pla de treball per als propers 12 mesos:}

El pla de treball futur pel al curs acadèmic 2012--2013 se centra en
la realització de la tesi en el doctorat en automàtica, robòtica i
visió. 
%En el document de l'\autoref{sec:pla_treball} ''\nameref{sec:pla_treball}''
A continuació es detalla el pla de treball principal de la feina a
realitzar en els propers anys, els objectius i l'interès i viabilitat
de la tesi futura.

A banda, s'està treballant en al confecció d'un article que recull la problemàtica dels sistemes de gestió de bases de dades per a sèries temporals, defineix el model multiresolució que proposem i en demostra l'ús mitjançant un exemple amb dades reals. 



%\newpage


\section*{Pla de treball principal}
%\label{sec:pla_treball}


\begin{abstract}

  En aquest document es detalla el progrés en el pla de treball
  presentat el 2012 per a assolir la recerca en el disseny d'un model
  de sistema de gestió per a sèries temporals. En primer lloc es
  presenta l'actualització dels objectius, en segon lloc el progrés en
  funció de les tasques planificades realitzades i en tercer lloc la
  modificació del pla de treball per a assolir els objectius pendents.
\end{abstract}



\section{Objectius}
\label{sec:objectius}

Aquesta recerca té per objectiu l'estudi de les necessitats
específiques que comporta l'emmagatzematge i gestió de dades amb
naturalesa de sèrie temporal i la proposta d'un model de SGBD que
satisfaci aquestes necessitats. Aquest objectiu es divideix en els
següents subobjectius més concrets:

\begin{enumerate}

\item Estudi de les aplicacions en que les dades són sèries temporals
  amb la finalitat de determinar quines són les propietats i problemes
  comuns que planteja la seva gestió i emmagatzematge.

\item Estudi dels models de SGBD existents. Segons es desprèn de la
  formalització de \textcite{date:introduction} el model principal és
  el model relacional, el qual es fonamenta en dos conceptes:
  relacions i tipus de dades. 

\item Una àrea de treball important en els SGBD és la incorporació de
  nous tipus de dades complexos. És important estudiar com es modifica
  el model de dades d'un SGBD quan s'afegeix un nou tipus de dades
  complex.  Les sèries temporals es poden d'entendre com a tipus
  complex ja que presenten diferents propietats característiques i
  necessiten operadors addicionals.  

\item Disseny d'un model de SGBD per a les sèries temporals. D'aquesta
  manera els SGBD podran tractar dades amb instants de temps que
  mostrin l'evolució de variables en funció del temps. El model
  consisteix en la definició de l'estructura de les sèries temporals i
  les operacions bàsiques que necessiten.

  L'assoliment d'aquest objectiu té tres parts:

  \begin{enumerate}
  \item Disseny d'un model per a la gestió bàsica de les sèries
    temporals, el qual anomenem model de SGBD per a sèries temporals
    (SGST).  L'estructura d'aquest model és similar a l'utilitzat en
    els intervals
    temporals \parencite{date02:_tempor_data_relat_model}.  Prenent
    com a base el model de SGST, el qual és un model general per a les
    sèries temporals, s'hi poden incloure altres models per a
    propietats més específiques de les sèries temporals.

  \item Disseny d'un model específic en base del model de
    SGST. Concretament es dissenya un model pels SGST multiresolució
    (SGSTM).  En el model de SGSTM s'hi poden incloure propietats de
    les sèries temporals relacionades amb la resolució que s'han
    observat en les aplicacions pràctiques de les sèries temporals:
    regularització, canvis de resolució mitjançant agregacions,
    reconstrucció de forats, etc.
 
  \item Avaluació de diferents estructures de SGSTM. El model de SGSTM
    s'hi poden fer modificacions o simplificacions per tal
    d'aconseguir diferents estructures.  Per exemple bases de dades
    multiresolució que comparteixin informació, que treballin amb flux
    de dades (\emph{data stream}) o bé que s'especialitzin per a un
    tipus determinat de sèries temporals.

  \end{enumerate}

\item Implementació de referència dels models de SGST i SGSTM. Per una
  banda, aquesta implementació, a nivell acadèmic, ha de servir com a
  exemple per a futurs desenvolupaments de sistemes de gestió,
  acadèmics o productius. Per altra banda, ha de servir per a
  exemplificar-ne els seu funcionament amb unes dades de prova.

\item Implementació específica i reduïda del model per a una
  determinada aplicació de sèries temporals. Exemplificació de com una
  estructura de SGSTM pot ser implementada per a aconseguir una
  aplicació molt concreta.

\end{enumerate} 


El objectius 4.c i 6. són resultat d'una nova planificació com es
detalla més endavant a les tasques futures.






%%% Local Variables: 
%%% mode: latex
%%% TeX-master: "main"
%%% End: 
% LocalWords:  SGSTM multiresolució SGST


\subsection*{Planificació del treball}


Per tal d'assolir els objectius detallats a la secció
\ref{sec:objectius}, a continuació es proposen les tasques a
realitzar.  A la figura \ref{fig:pla:futur} es detalla el pla de
treball amb temps estimat.





\begin{enumerate}


\item Estudi d'aplicacions de les sèries temporals. Per a assolir
  l'objectiu~1 estudiarem recerca actual de sèries temporals.
  Consultarem articles i llibres que tinguin les sèries temporals com
  a temàtica principal. També cercarem l'existència de programari que
  tingui en els seus objectius el tractament de sèries temporals.

\item Estudi del model relacional. Per a l'objectiu~2 estudiarem el
  model relacional com a referent pels models de SGBD. Ens basarem en
  l'estudi dels llibres de
  \textcite{date:introduction,date06,date:dictionary}. Usarem
  \emph{rel} \parencite{rel} com a implementació de referència ja que
  incorpora el llenguatge \emph{Tutorial D}, el qual és utilitzat per
  Date en els seus exemples.

\item Estudi de la gestió d'intervals temporals. Per a l'objectiu~3
  estudiarem la recerca que ha conduit a incorporar els interval
  temporals en els SGBD per a gestionar històrics. Ens basarem en
  l'estudi que relaciona el model relacional amb els intervals
  temporals de \textcite{date02:_tempor_data_relat_model}.

\item Disseny d'un model de SGBD per sèries temporals. Per a la
  primera part de l'objectiu 4 dissenyarem un model per als SGBD de
  sèries temporals formalitzat amb expressions algebraiques. A partir
  de l'estudi de l'estat de l'art de les sèries temporals, observarem
  les propietats interessants de ser modelitzades i les operacions que
  precisen els SGBD per sèries temporals.

\item Disseny d'un model de SGBD multiresolució per sèries
  temporals. Per a la segona part de l'objectiu 4 dissenyarem un model
  que contempli la multiresolució de les sèries temporals. Aquest
  model utilitzarà propietats del model anterior per les sèries
  temporals. La mulitresolució té l'objectiu d'emmagatzemar les sèries
  temporals de forma compacta, així es preveu que alguns conceptes de la recerca
  en \emph{data streams} poden prendre-hi sentit.

\item Implementació de referència. Per a l'objectiu~5 s'implementaran
  els models anteriors utilitzant un llenguatge de programació adequat
  per a models, com per exemple \texttt{Python} o \texttt{Prolog}.  Es
  prioritzarà la implementació correcte del model enfront a una
  implementació que contempli un bon rendiment. 

\item Experimentació amb dades. Per a complementar l'objectiu~5 es
  provarà la implementació amb dades experimentals per alguna
  aplicació concreta.

\end{enumerate}

\begin{figure}[tp]
\centering
\scalebox{0.8}{
\begin{gantt}[xunitlength=0.8cm,fontsize=\small,titlefontsize=\small]{15}{12}
  \begin{ganttitle}
    \numtitle{2012}{1}{2014}{4}
  \end{ganttitle}
  \begin{ganttitle}
    \numtitle{1}{1}{4}{1}
    \numtitle{1}{1}{4}{1}
    \numtitle{1}{1}{4}{1}
  \end{ganttitle}

  \ganttmilestone{Proposta de tesi}{2}

  \ganttgroup{Estudi}{0}{4}
  \ganttbar{1. Aplicacions}{0}{2}
  \ganttbar{2. Model relacional}{0}{4}
  \ganttbar{3. Intervals temporals}{2}{2}

  \ganttgroup{Disseny models}{4}{3}
  \ganttbar{4. Sèries temporals}{4}{3}
  \ganttbar{5. Multiresolució}{5}{2}
  \ganttcon{7}{7}{7}{11}

  \ganttgroup{Experimentació}{7}{2}
  \ganttbar{6. Implementació}{7}{2}
  \ganttbar{7. Dades experimentals}{8}{1}

  \ganttbar{8. Redacció memòria}{8}{2}

  \ganttmilestonecon{Lectura de tesi}{10}

\end{gantt}
}
\caption{Planificació del treball}
\label{fig:pla:futur}
\end{figure}



\subsubsection*{Treball realitzat}


La tesi de màster \parencite{llusa11:tfm} va consistir en l'estudi de
\emph{RRDtool} \parencite{rrdtool}, un sistema de gestió de bases de
dades (SGBD) específic per a sèries temporals.
%
Fruit d'aquest estudi es va formalitzar el model de \emph{RRDtool} i es va
dissenyar i implementar un prototip software que responia a la
formalització.
%
Aquest treball va permetre concloure que:
\begin{itemize}
\item Els SGBD aplicats a sèries temporals tenen aspectes propis que
  els converteixen en objecte d'estudi de \emph{per se}.
\item El concepte de multiresolució és interessant en moltes
  aplicacions reals, especialment quan existeixen restriccions d'espai
  per emmagatzemar dades.
\item El model proposat per a \emph{RDDtool} és susceptible de ser millorat
  com a mínim en el següents aspectes:
  \begin{itemize}
  \item Generalització. El model que es va presentar estava fortament
    lligat a \emph{RDDtool}. Generalitzar el model de forma que encabeixi
    altres concepcions.
  \item Incorporació d'operacions. El model presentat únicament feia
    referència a les dades. Per completar el
    model cal també considerar les operacions.
  \item Contextualització. Cal interrelacionar i descriure el model el
    context d'altres models existents per a SGBD. Específicament cal
    comparar-lo amb el model relacional usat en els SGBD
    convencionals.
  \end{itemize}
\item És convenient estudiar la feina feta per altres autors en
  l'àmbit de l'emmagatzemat i gestió de dades provinents de sèries
  temporals.
\end{itemize}

Arrel de la feina anterior, s'ha treballat en els següents aspectes:
\begin{itemize}
\item S'ha realitzat una tasca de recerca bibliogràfica i estudi dels
  treballs existents en l'àmbit de la gestió i emmagatzemat de dades
  provinents de sèries temporals. El resultat d'aquesta tasca s'ha
  reflectit a la proposta de tesi \parencite{llusa12:ptd}.
\item S'ha estudiat en profunditat el model relacional. Atesa la
  preeminència d'aquest model en els SGBD actuals, s'ha considerat
  imprescindible tenir-ne un bon coneixement que permetés estudiar les
  seves deficiències per la gestió de sèries temporals i en quina
  forma poden ser superades.
\item S'està refent el model de SGBD per sèries temporals.  S'ha
  dividit el model en dues parts ben diferenciades:
  \begin{enumerate}
  \item La primera és el model general. Aquest defineix la gestió de
    sèries temporals enteses com a co\l.lecció de dades mesurades en
    diferents instants de temps. Es basa en els conceptes de temps,
    mesura i sèrie temporal.
  \item La segona és el model de multiresolució. Aquest model explica
    la forma d'emmagatzemar una sèrie temporal amb diferents
    resolucions temporals.
  \end{enumerate}
\end{itemize}





\subsection*{Mitjans}

La recerca es duu a terme amb el suport de la Universitat Politècnica
de Catalunya (UPC) mitjançant una beca FPU-UPC adscrita al departament
d'Enginyeria del Disseny i Programació de Sistemes Electrònics
(DiPSE).

No es preveu un ús de mitjans més enllà de l'accés als
recursos bibliogràfics i d'eines informàtiques de programació i
gestió de documentació.



%%% Local Variables: 
%%% mode: latex
%%% TeX-master: "main"
%%% End: 






%%% Local Variables: 
%%% mode: latex
%%% TeX-master: "main"
%%% End: 
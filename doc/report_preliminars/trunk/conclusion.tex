\section{Implementation}\label{sec:implementation}

The six basic model definitions -- measure, time series, buffer, disc,
resolution disc, and multiresolution database -- have been implemented
with Python in an object-oriented programming paradigm. It pretends to
be a reference implementation to other productive time series data
base management systems. This implementation is yet in a primary
development and experimental examples will be shown in the future.

RRDtool \parencite{rrdtool} is a productive data base
management system and widely used in the OpenSource industry. For
instance, the performance in large systems and improvement methods can
be seen in \cite{lisa07:plonka}. Now, it can be seen as a
productive implementation of the basic model definitions presented in
this paper and much more definitions not yet described such as the
fetch operations. However, it imposes some restrictions. For example a
reduced set of interpolation functions or a special treatment of
gauges and counters inherited from the past.

\section{Conclusions}\label{sec:conclusion}

A multiresolution database is an storage system for one time series, that
is a collection of data measured in different instants in time.  The
time series is compactly stored in the database as has been shown in
figure \ref{fig:model:mtsdb}. The principal part of a multiresolution
database is the set of resolution discs where the time series is
stored distributed by the different interpolation functions and
sampling periods. Each resolution disc uses its buffer to interpolate
the measures and uses its disc to consolidate the result. 

This system allows to reduce the storage space and the analysis time
needed for large collections of times series.  With the compactness of
the time series we have in mind the scalability for large monitoring
systems.

Interpolation functions and sampling periods are degrees
of freedom for each application. Giving different values a multiresolution
database is capable to keep the desired information from a time series.


\section{Future work}\label{sec:future}

In future work the retrieval operations will be defined. Mainly, how a
time series can be restored from joining the information stored
distributed in the resolution discs. And it also will be important
how data fusion can be obtained from two time series databases.

Then it will be possible to check the database management system with
experimental data such as the proposed by \cite{keogh02}.

The MTSMS imply a data information selection and the information not considered important is discarded.  Therefore, this systems are not adequate when all the monitored data must be kept as acquired. This happens either when it is not known a priori which interpolation functions will work better with the future data monitored or when detailed questions must be retrieved such as at what hour exactly an event triggered. This may be overcome with dual DBMS: one a TSMS for the common information retrieval and the other a large size database (VLDB) that is only consulted in occasional cases. 
However, the applications when a TSMS is not appropriate are to be better understood. 

          


%%% Local Variables:
%%% TeX-master: "article"
%%% ispell-local-dictionary: "british"
%%% End:
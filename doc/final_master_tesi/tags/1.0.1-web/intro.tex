\chapter{Introducció}


En el camp de l'automàtica, hi ha un àmbit dedicat a construir
sistemes per a la interactuació i el control de l'entorn; són coneguts
com els sistemes SCADA (\emph{Supervisory Control And Data
  Acquisition}).  Una de les parts importants dels SCADA és la
reco\l.lecció de dades, a la qual també se l'anomena adquisició de
dades o monitoratge.

Monitorar o supervisar significa estar alerta de l'estat d'un sistema.
Aquesta tasca és duta a terme pels monitors, els quals recullen dades
d'uns sensors i gestionen alarmes quan hi ha problemes en les
variables mesurades.

Un monitor pot arribar a recollir un gran nombre de dades, ja que se
solen monitorar molts sensors amb una freqüència elevada. No és rar
trobar sistemes que monitoren centenars de sensors cada segon. Amb
aquestes característiques, només una petita part de les dades pot ser
observada en línia amb el procés (\emph{online}).
%http://en.wikipedia.org/wiki/Online_and_offline
Això no obstant, l'anàlisi d'aquestes dades aporta informació
important per diagnosticar el sistema, detectar errors o fer
predicció i per tant cal emmagatzemar-les. 

Els monitors usen sistemes especialitzats per gestionar l'emmagatzemat
de les dades. Aquests sistemes han de facilitar consulta les dades,
operar-hi o presentar-les gràficament fins i tot quan el volum de
dades arxivades és molt gran.

En aquest context de monitoratge, és útil considerar que les dades
recollides són sèries temporals. Aleshores es pot aplicar la teoria
sobre les sèries temporals en els diferents procediments de tractament
de dades.

Aquest document s'estructura al voltant dels sistemes
d'emmagatzemament i tractament de dades com a sèries temporals.
Concretament se centra en els sistemes de gestió de bases de dades
(SGBD) que s'ocupen de sèries temporals.  Els SGBD habituals, com per
exemple els de model relacional, no són adequats per aquests casos ja
que no disposen de les funcionalitats adequades per gestionar i
explotar correctament la informació de les sèries temporals.
%
Existeixen, però, SGBD's especialitzats en la gestió de dades
provinents de sèries temporals.  Entre aquests destaca RRDtool
per haver-se convertit en l'estàndard \emph{de facto} a nivell
productiu.  RRDtool està basat en uns conceptes que es poden
agrupar sota un model anomenat model de dades Round Robin (RRD). El
model RRD està específicament pensat per a les sèries
temporals. RRDtool, doncs, és un SGBD molt adequat per a ser usant en
l'àmbit de monitoratge.

Segons Date, \cite{date:introduction}, ``una base de dades és un contenidor
informàtic per a una co\l.lecció de dades''. En el context dels SGBD's
cal diferenciar:
\begin{itemize}
\item El model d'una bases de dades, que és el model matemàtic a on es
  descriu teòricament l'estructura de les dades, per exemple el model
  relacional o model Round Robin.

\item El sistema de gestió de bases de dades, que és la implementació
  d'un model de dades, per exemple postgresql o
  RRDtool. %Segons l'esquema de comunicació, també es poden anomenar com a sistema servidor de bases de dades, per exemple postgresql.

\item La base de dades, que és una instància d'un sistema de gestió de
  bases de dades, per exemple la base de dades dels estudiants o la
  base de dades de la temperatura de l'escola.
\end{itemize}

Fins  on coneixem,  RRDtool és  l'únic SGBD  que  implementa el
model  RRD. En  aquest projecte  s'estudia RRDtool  per  tal de
proposar,  per primera vegada,  una descripció  formal del  seu model.
L'existència d'aquest  model obre les  portes a un  millor coneixement
d'aquests tipus d'SGBD's, facilita noves implementacions i, finalment,
permet avançar en el seu estudi.


\section{Motivació}

Actualment, els sistemes de telecontrol són complexos i hi ha una gran
quantitat d'informació a gestionar. En aquests grans sistemes de
telecontrol apareixen problemes en la reco\l.lecció de dades, com per
exemple els forats, descrits per Quevedo \emph{et al.},~\cite{quevedo10}, o
el tractament de reco\l.leccions massives que descriuen Camerra i
Keogh,~\cite{keogh10:isax}. Aquestes reco\l.leccions de dades
s'estudien com a sèries temporals degut a la seva naturalesa de
seqüència de valors, però en l'emmagatzematge a les bases de dades no
es té en compte aquesta estructura.

Com s'ha dit a l'apartat anterior, els SGBD's que han de gestionar
aquesta informació haurien de poder tractar convenientment la
informació que prové de sèries temporals. En aquest sentit, actualment
s'estan investigant en algunes tècniques, com iSAX,~\cite{isax}, o
T-Time,~\cite{assfalg08:ttime}. En aquest context l'SGBD anomenat
RRDtool destaca per ser l'estàndard \emph{de facto} en
l'emmagatzematge i tractament de sèries temporals,~\cite{rrdtool}.

RRDtool no disposa de cap model matemàtic que en descrigui el
comportament. Disposar d'un model matemàtic és una fita important com
demostra l'experiència històrica en l'àmbit dels SGBD's relacionals,
\cite{date:introduction}.  Disposar d'un model per a RRDtool, permetrà
aprofundir en aquests tipus de SGBD's i estudiar com poden ajudar, per
exemple, en l'aplicació de mètodes de validació i predicció com els
que proposa Puig \emph{et al.} o les reduccions de dimensió que
proposen Camerra i Keogh.

La recerca en l'àmbit de la mineria de dades de sèries temporals,
àmbit en el que se situa l'emmagatzematge i tractament de sèries
temporals, ha experimentat un important increment darrera dècada. En
fan esment Tak-chung Fu, \cite{fu11}, que el considera un problema no
solucionat, i Yang i Wu, \cite{yangwu06}, que el qualifiquen com un
dels deu reptes a resoldre en la mineria de dades.


\section{Objectius}

En aquest projecte es plantegen els objectius següents:

\begin{itemize}

\item Situar l'estat actual de l'emmagatzematge i del tractament de
  sèries temporals.

\item Estudiar el sistema de gestió de bases de dades RRDtool.

\item Dissenyar un model de dades que descrigui l'estructura i el
  comportament dels SGBD per a sèries temporals.

\item Proposar una implementació de referència del model dissenyat.

\item Proposar millores i treballs futurs al voltant del model
  dissenyat.

\end{itemize}




\section{Estructura del document}

A continuació d'aquest primer capítol d'introducció als sistemes de
gestió de bases de dades de sèries temporals i de definició dels
objectius del projecte, aquest document s'estructura en els capítols
següents.

En el capítol~\ref{cap:estat}, complementant al capítol d'introducció,
s'exposa l'estat actual de la mineria de dades de sèries temporals. En
aquest context se situen els sistemes de gestió de bases de dades
que emmagatzemen i tracten sèries temporals.

A continuació el cos principal del document se centra en els SGBD
Round Robin.  En una primera part es descriu el SGBD RRDtool.

En el capítol~\ref{cap:velocitats} s'aclareix perquè RRDtool, en el
context de sèries temporals, està pensat per velocitats mitjanes de
comptadors i com es poden tractar variables que són magnituds.

En el capítol~\ref{cap:rrdtool} s'introdueix un concepte bàsic de base
de SGBD RRD a partir del que es pot observar a
RRDtool. S'expliquen algunes particularitats que tenen les
bases de dades gestionades per RRDtool, com per exemple que no creixen
en mida un cop creades o que són capaces de representar les dades
gràficament.

En el capítol~\ref{cap:rrdtool-etapes} es detalla el funcionament
intern de RRDtool, tot seguint les etapes per les quals passa una dada
des de que s'ha mesurat fins que queda desada.

En una segona part, a partir dels conceptes observats a RRDtool, es
proposa el model RRD.

El capítol~\ref{cap:model-rrd} correspon a la definició d'aquest model
de dades.  En el capítol~\ref{cap:roundrobinson}, s'implementa el
model utilitzant el llenguatge de programació Python. S'exposa un
exemple de funcionament.

Finalment en el capítol~\ref{cap:conclusions}, es tanca el document
amb un resum de l'exposat i les conclusions que es poden extreure del
model de dades per a sèries temporals que s'ha dissenyat.  També es
proposen els treballs futurs que es poden dur a terme a partir del
model dissenyat.





%%% Local Variables:
%%% TeX-master: "memoria"
%%% End:

% LocalWords:  monitoratge monitorar SGBD RRDtool RRD Round

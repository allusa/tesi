%\selectlanguage{english}
%\begin{abstract}
%\end{abstract}

%\selectlanguage{catalan}
%\begin{abstract}
%\end{abstract}
%\selectlanguage{catalan}

\chapter*{Abstract}

This project deals with the storage and treatment of time series obtained from data acquisition. The data acquired is treated by time series database management systems.

In this field, RRDtool is a DBMS devoted to time series data acquisition. In this project a formal model is formulated in order to describe RRDtool's data structure and behaviour, which has been absent until now. This model is known as Round Robin Database model (RRD) and improves the knowledge in time series data mining.


{\bfseries  Keywords:} Time series, data acquisition, Time series data mining, Round Robin Database model (RRD), DBMS  RRDtool. 



\section*{Resum}

Aquest projecte s'emmarca en l'emmagatzematge i el tractament de
sèries temporals que s'obtenen com a resultat d'una adquisició de
dades. Un cop adquirides, les dades són gestionades per sistemes de
gestió de bases de dades específics per a sèries temporals.

En aquest context, existeix l'SGBD RRDtool especialitzat en la gestió
d'adquisicions de dades provinents de sèries temporals. Aquest
projecte proposa un model formal, inexistent fins a l'actualitat, per
tal de descriure'n l'estructura i el comportament. Aquest model
s'anomena model de dades Round Robin (RRD) i obre les portes a
un millor coneixement dels SGBD per a sèries temporals.




{\bfseries Paraules clau:} sèries temporals, adquisició de dades, SGBD per a sèries temporals, model de dades Round Robin (RRD), SGBD RRDtool.














\newpage


\section*{Agraïments}

Amb el suport de la Universitat Politècnica de Catalunya (UPC). Amb l'agraïment al departament de Disseny i Programació de Sistemes Electrònics (DiPSE) i a la Càtedra de Programari Lliure (CPL).

Al Tobias Oetiker, l'Alex Van den Bogaerdt i companyia per dissenyar i mantenir el sistema de gestió de bases de dades més original que mai s'hagi vist.

Al Sebastià i a la Teresa per la seva gran paciència i dedicació. 

Al Melen, per ser el més ràpid de la comarca.

A la Núria, que li'n dec una de molt grossa!







%%% Local Variables: 
%%% mode: latex
%%% TeX-master: "memoria"
%%% End: 

\begin{frame}
  \titlepage
\end{frame}


\begin{frame}{Índex}
  \begin{enumerate}
  
  \item Introducció
    \begin{enumerate}
    \item Sistema de monitoratge
    \item Sèries temporals
    \item Sistemes de Gestió de Bases de Dades (SGBD)
    \end{enumerate}
    
  \item SGBD RRDtool 
    
  \item Model de dades Round Robin (RRD)
    
  \item Implementació de referència del model RRD
    
  \item Conclusions
    
  \item Treball futur
    
  \item Referències
\end{enumerate}
\end{frame}




\begin{frame}{Introducció. Sistema de monitoratge: situació del problema}

  \begin{itemize}
  \item \emph{Supervisory Control And Data Acquisition} (SCADA). 
  \item Adquisició de dades $\Rightarrow$  monitoratge:

    \begin{itemize}
    \item Moltes dades: només una petita part poden ser observades en línia.
    \item Volem extreure coneixement  de les dades  $\Rightarrow$ sèries temporals.
    \end{itemize}
  \end{itemize}

  \begin{center}
    \scriptsize 
    \begin{tikzpicture}[node distance=0.5cm]  
      \tikzset{
        mynode/.style={rectangle,rounded corners,draw=black, 
          very thick, inner sep=1em, minimum size=3em, text centered,
          groc},
        myarrow/.style={->, >=latex', shorten >=1pt, thick},
        mylabel/.style={text width=7em, text centered},
        groc/.style={top color=white, bottom color=yellow!50},
        verd/.style={top color=white, bottom color=green!50},
        roig/.style={top color=white, bottom color=red!50},
      }  

      \node[mynode]                                       (monitor)   {Sistema de monitoratge};  
      \node[mynode, below right=2cm and -2cm of monitor]  (bd)        {Base dades}; 
      \node[mynode, below=2cm of monitor, left=of bd]     (control)   {Sistema de control}; 
      \node[mynode, roig, below right=0.5cm and 1.5cm of monitor] (usuari)    {Usuari};  
      \node[mynode, verd, left=2cm of control]            (actuador)  {Actuadors};
      \node[mynode, verd, left=3cm of monitor]            (sensor)    {Sensors};  
      
      
      \draw[myarrow] (monitor.east) --   (usuari.north);	
      \draw[myarrow] (bd.east) --   (usuari.south);
      \draw[myarrow] (sensor.east) --   (monitor.west) 
         node [above,midway] {Mesures}
         node [below,midway] {Esdeveniments};
      \draw[myarrow] (control.west) -- (actuador.east);
      \draw[myarrow] (monitor.south) -- (bd.north);
      \draw[myarrow] (monitor.south) -- (control.north)
         node [above,sloped,midway] {Llaç de control};
      
    \end{tikzpicture} 
  \end{center}


\textbf{Objectiu:} Dissenyar un model pels SGBD dels sistemes de monitoratge.
\end{frame}




\begin{frame}{Introducció. Sèries temporals: estat actual.}

\begin{itemize}
\item Són co\l.leccions d'observacions cronològiques de magnituds.
\item Són de gran dimensió (gran nombre de punts).
\item Són de naturalesa contínua i numèrica.
\item Ofereixen representació, indexat, cerca de similituds,
  segmentació, visualització, extracció coneixement (predicció, cerca
  de patrons\ldots).
\end{itemize}

La recerca s'ha incrementat en la darrera dècada: és important
disposar d'esquemes de representació més efectius i eficients,
\parencite{fu11}:
\begin{itemize}
\item Reduir la dimensió: mostreig, \parencite{astrom69}, interpolació
  lineal amb PLR, \parencite{keogh97}, mitjana de segments amb PAA,
  \parencite{keogh00}, etc.
\item Indexar: iSAX, \parencite{keogh08:isax}, T-Time,
  \parencite{assfalg08:ttime}, etc.
\end{itemize}


\end{frame}


\begin{frame}{Introducció. SGBD: què són?}


Una base de dades és un contenidor informàtic per a una co\l.lecció de dades.

\begin{itemize}
\item El model d'una bases de dades, que és el model matemàtic a on es
  descriu teòricament l'estructura de les dades, per exemple el model
  relacional o model Round Robin.

\item El sistema de gestió de bases de dades, que és la implementació
  d'un model de dades, per exemple postgresql o
  RRDtool. 

\item La base de dades, que és una instància d'un sistema de gestió de
  bases de dades, per exemple la base de dades dels estudiants o la
  base de dades de la temperatura de l'escola.
\end{itemize}


Els SGBD habituals són els relacionals,  disposen d'un model matemàtic consolidat.

\end{frame}



\begin{frame}{Introducció. SGBD per sèries temporals: què aporten?}

Què no aporten els SGBD tradicionals?:
\begin{itemize}
\item SGBD relacionals permeten \emph{bitemporal data}: \emph{created} i  \emph{modified}, però dificulten operacions temporals de les sèries temporals. 

\item Les mesures són camps numèrics individuals i no un sol conjunt.

\item Cerca aproximada per les sèries temporals; en els SGBD és exacta.

\item Estructures amb dimensió temps: \emph{Time Series Database Systems}. 
\hspace{1cm}-- Extensió de tipus i operacions \parencite{stonebraker86}.

\end{itemize}

\medskip

Els TSDS habituals no consideren els sistemes de monitoratge:
\begin{itemize}
\item Temps de les mesures no regular (mostres no equi-espaiades).
\item Tractament de valors desconeguts.
\item L'ordre temporal és essencial. Valor i temps són inseparables.
\end{itemize}

\medskip

RRDtool: TSDS per a monitors de sèries temporals.


\end{frame}

%%% Local Variables: 
%%% mode: latex
%%% TeX-master: "presentacio"
%%% End: 

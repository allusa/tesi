
\chapter{Esquemes de multiresolució}

\todo{}


Definim una funció de multiresolució com una consulta sobre una sèrie
temporal que ens retorna una nova sèrie temporal resultant
d'aplicar-li un esquema de multiresolució. 

Aquesta funció de multiresolució ens permetrà:

\begin{itemize}
\item Plantejar problemes on veiem la multiresolució com una consulta sobre una sèrie temporal
\item Oferir sistemes duals de \glspl{SGSTM} i \glspl{SGST} amb operacions
  de consulta multiresolució.
\item Estudiar implementacions per a la consulta de multiresolució, p.ex. para\l.lelisme. (vegeu secció implementació \todo{})
\item Estudiar la teoria de la informació per a l'esquema de multiresolució
\end{itemize}






\section{Funció de multiresolució}

En el model de \gls{SGSTM} hem definit un model de dades de \gls{SGBD}
per a gestionar sèries temporals multiresolució. Com a \gls{SGBD},
aquest model té una estructura i per tant emmagatzema informació d'una
sèrie temporal en una forma determinada: la de multiresolució.  La
definició com a \gls{SGBD} té com a objectius l'emmagatzematge
compacte de les dades i la selecció de la informació ja preparada per
a consultes posteriors. 

Així, aquest model té capacitats de computació
sincronitzada o en línia (\emph{online}) amb el temps i té
característiques dels sistemes que tracten fluxos de dades (\emph{data
  stream}); és a dir dades que s'estan adquirint contínuament i cal
anar computant al mateix temps que es van adquirint. Això no treu,
però, que de manera més simplificada també es pugui treballar amb un
\gls{SGSTM} en temps diferit (\emph{offline}), és a dir que
s'emmagatzemin les dades adquirides i en el moment que es vulgui
aplicar-hi la consolidació.




Això no obstant, podem simplificar el problema de càlcul de
multiresolució en temps diferit com una consulta en un \gls{SGST} de
transformació d'una sèrie temporal a una nova sèrie temporal.

És a dir, sigui $S$ una sèrie temporal, $M$ una sèrie temporal
multiresolució i $e = \{ (\delta_0,f_0,\tau_0,k_0), \ldots,
(\delta_d,f_d,\tau_d,k_d)\}$ els paràmetres de l'esquema de
multiresolució de $M$. Afegim totes les mesures de la sèrie temporal a
la multiresolució, $\forall m \in S:
M=\glssymbol{addM}(M,m)$\todo{matemàticament és correcte recursivitat
  M=f(M)?}, i la consolidem, $M=\glssymbol{consolidaM}(M')$ fins que
$M$ no sigui consolidable. Consultem la sèrie temporal multiresolució
amb les dues consultes bàsiques, les quals retornen sèries temporals,
$S'=\glssymbol{not:sgstm:serietotal}(M')$ i $S_{\delta
  f}'\glssymbol{not:sgstm:seriedisc}(M',\delta,f)$ on $\delta$ i $f$
són dos paràmetres de l'esquema de multiresolució de $M$ que van
associats amb els altres dos corresponents $\tau$ i $k$.



Plantegem les funcions de transformació de la sèrie temporal original
a les consultades. És a dir, les funcions que anomenem
$\glssymbol{not:sgstm:dmap}$ i $\glssymbol{not:sgstm:multiresolucio}$
i que ens permeten calcular:

\[
\glssymbol{not:sgstm:dmap}: S \times \delta \times f \times \tau \times k \longrightarrow
S'_{\delta f}
\]


\[
 \glssymbol{not:sgstm:multiresolucio}: S \times e  \longrightarrow S'
\]



Definim la consulta de selecció de disc dels \gls{SGSTM} a partir del
mapatge dels \gls{SGST} de manera que, en computació per temps
diferit, són equivalents
\[
\glssymbol{not:sgstm:seriedisc}(M',\delta,f) \equiv
\glssymboldef{not:sgstm:dmap}(S,\delta,f,\tau,k)
\]


\begin{definition}[mapa de \glssymbol{not:sgstm:seriedisc}]
  Sigui $S$ una sèrie temporal, $M$ una sèrie temporal multiresolució
  amb esquema $e$ i $(\delta,f,\tau,k)\in e$ els paràmetres de
  multiresolució d'una subsèrie resolució. L'expressió de
  $\glssymbol{not:sgstm:seriedisc}(M,\delta,f)$ com a mapa d'una sèrie
  temporal és $\glssymboldef{not:sgstm:dmap}(S,\delta,f,\tau,k)=
  \glssymbol{not:sgst:map}(S_I,\glssymbol{not:sgst:fmap})$ on
  \[
  \glssymbol{not:sgst:fmap}: m_i\mapsto f(S, [T(m_i)-\delta,T(m_i)]),
  \]
  \[
  S_I = \{ (t,\infty) | t\in T_I  \},\;  t_M = T(\max(S)),
  \]
  \[
  T_I = \{ t_I = \tau+n\delta | n\in\glssymbol{not:Z}, t_M - k\delta <
  t_I \leq t_M \}.
  \]
\end{definition}



\begin{example}
  \label{ex:multiresolucio:dmap}
  Sigui la sèrie temporal $S=\{(1,0),(3,1),(6,0),(10,1)\}$ i els
  paràmetres de multiresolució
  $((\delta,5),(f,\glssymbol{not:sgstm:maxdd}),(\tau,0),(k,2))$.  El
  mapa de \glssymbol{not:sgstm:seriedisc} és una sèrie temporal $S'=
  \glssymboldef{not:sgstm:dmap}(S,5,0,\glssymbol{not:sgstm:maxdd},2)$
  on $S'=\{(5,1),(10,1)\}$. Expressem el càlcul pas a pas, a la
  \autoref{fig:multiresolucio:dmap} es visualitzen en taula les sèries
  temporals corresponents:
  \begin{enumerate}
  \item El primer pas és obtenir els instants de temps que
    s'emmagatzemarien al disc d'una sèrie temporal
    multiresolució. Així, els instants de consolidació possibles són
    $T_I'=\{\tau+n\delta|n\in\glssymbol{not:Z}\}=
    \{\ldots,-5,0,5,10,15,\ldots\}$. Però un cop consolidat el disc
    només hi haurà els $k=2$ més recents abans de $t_M=T(\max(S))=10$,
    és a dir $T_I=\{t_I'\in T_I'|t_M - k\delta < t_I \leq
    t_M\}=\{5,10\}$.

  \item El segon pas és obtenir a partir de $T_I$ la sèrie temporal
    $S_I$ que es correspon amb la sèrie temporal que s'inicialitzaria
    al disc encara amb valors desconeguts,
    $S_I=\{(5,\infty),(10,\infty)\}$.



  \item El tercer pas és calcular la funció d'agregació a $S$ per a
    cada intervals de consolidació del disc de la forma
    $[T(m_i)-\delta,T(m_i)]$ on $m_i\in S_i$, és a dir $f(S,[0,5])$ i
    $f(S,[5,10])$. A tal efecte utilitzem el mapa sobre $S_I$ per a
    calcular la sèrie temporal resultant $S'=\{ (5,f(S,[0,5])),
    (10,f(S,[5,10])) \}$.

    Podríem calcular un pas entremig que es correspon amb les sèries
    temporals que hi hauria en el buffer abans de cada instant de
    consolidació. Així, per a cada $T(m_i)$ hi hauria la sèrie
    temporal $S[T(m_i)-\delta,T(m_i)]$, és a dir $S_B=\{
    (5,S[0,5],(10,S[5,10]) \}$.
  \end{enumerate}


  


\begin{figure}[tp]
  \centering
  \begin{tabular}[c]{|c|c|}
    \multicolumn{2}{c}{$S$} \\ \hline
    $t$  & $v$ \\ \hline
    1  & 0 \\
    3  & 1 \\
    6  & 0 \\
    10  & 1 \\ \hline
  \end{tabular} \qquad
  \begin{tabular}[c]{|c|c|}
    \multicolumn{2}{c}{$S_I$} \\ \hline
    $t$  & $v$ \\ \hline
    5  & $\infty$ \\
    10  & $\infty$ \\ \hline
  \end{tabular} \qquad
  \begin{tabular}[c]{|c|c|}
    \multicolumn{2}{c}{$S_B$} \\ \hline
    $t$  & $v$ \\ \hline
    5  & \begin{tabular}[c]{|c|c|}\hline $t$  & $v$ \\ \hline 1&0\\ 3&0 \\\hline  \end{tabular} \\\hline
    10  & \begin{tabular}[c]{|c|c|}\hline $t$  & $v$ \\ \hline 6&0\\ 10&1 \\\hline  \end{tabular} \\ \hline
  \end{tabular} \qquad
 \begin{tabular}[c]{|c|c|}
    \multicolumn{2}{c}{$S'$} \\ \hline
    $t$  & $v$ \\ \hline
    5  & 1 \\
    10  & 1\\ \hline
  \end{tabular}
  \caption{Taules de les sèries temporals per l'operació de mapa de  \glssymbol{not:sgstm:seriedisc}}
  \label{fig:multiresolucio:dmap}
\end{figure}
 
\end{example}





Definim la consulta de sèrie temporal total dels \gls{SGSTM} a partir
del plegament dels \gls{SGST} de manera que, en computació per temps
diferit, són equivalents
\[
\glssymbol{not:sgstm:serietotal}(M') \equiv \glssymbol{not:sgstm:multiresolucio}(S,e)
\]

\begin{definition}[plec de \glssymbol{not:sgstm:serietotal}]
  Sigui $S$ una sèrie temporal i $M$ una sèrie temporal multiresolució
  amb esquema $e = \{ (\delta_0,f_0,\tau_0,k_0), \ldots,
  (\delta_d,f_d,\tau_d,k_d)\}$, el qual es pot observar com una sèrie
  temporal multivaluada.  L'expressió de
  $\glssymbol{not:sgstm:serietotal}(M)$ com a plec d'una sèrie
  temporal és $\glssymboldef{not:sgstm:multiresolucio}(S,e)=
  \glssymbol{not:sgst:ofold}(e,\{\},\glssymbol{not:sgst:ffold},\min)$
  on $\glssymbol{not:sgst:ffold}: S_i \times (\delta_c,f_c,\tau_c,k_c)
  \mapsto S_i ||
  \glssymbol{not:sgstm:dmap}(S,\delta_c,f_c,\tau_c,k_c)$.

  Així, el plec de \glssymbol{not:sgstm:serietotal} és la concatenació de
  tots els \glssymbol{not:sgstm:dmap} possibles per l'esquema $e$
  ordenats per $\delta$, assumint que $e$ no conté $\delta$ repetits.
\end{definition}



En resum: en temps diferit, s'insereixen les mateixes mesures a un
\gls{SGST} i a un \gls{SGSTM}. Per una banda es consolida el
\gls{SGSTM} i s'obté la sèrie total i per altra banda es consulta la
multiresolució en el \gls{SGST}. Aleshores s'obté la mateixa sèrie
temporal.


\begin{example}
  Sigui la sèrie temporal $S=\{(1,0),(3,1),(6,0),(10,1)\}$ i l'esquema
  de multiresolució
  $e=\{\{(\delta,5),(f,\glssymbol{not:sgstm:maxdd}),(\tau,0),(k,2)\},
  \{(\delta,2),(f,\glssymbol{not:sgstm:maxdd}),(\tau,0),(k,3)\}\}$.
  El plec de $\glssymbol{not:sgstm:serietotal}$ és una sèrie temporal
  $S'= \glssymboldef{not:sgstm:multiresolucio}(S,e)$ on
  $S'=\{(5,1),(6,0),(8,0),(10,1)\}$. Expressem el càlcul pas a pas, a la
  \autoref{fig:multiresolucio:multiresolucio} es visualitzen en taula les sèries
  temporals corresponents:

  \begin{enumerate}
  \item En primer lloc es calcula la sèrie temporal pels paràmetres de
    multiresolució $\delta_1$:
    $S_{D1}=\glssymbol{not:sgstm:dmap}(5,\glssymbol{not:sgstm:maxdd}),0,2)=\{(5,1),(10,1)\}$,
    com ja s'ha vist a l'\autoref{ex:multiresolucio:dmap}.

  \item En segon lloc, es calcula la sèrie temporal pels paràmetres de
    multiresolució $\delta_2$:
    $S_{D2}=\glssymbol{not:sgstm:dmap}(2,\glssymbol{not:sgstm:maxdd}),0,3)=\{(6,0),(8,0),(10,1)\}$,
    de manera similar a $S_{D1}$.

  \item En tercer lloc es concatenen les sèries temporals per ordre de
    $\delta$: $\delta_2<\delta_1$. Així, $S'= S_{D2} || S_{D1}$.

  \end{enumerate}
  


\begin{figure}[tp]
  \centering
  \begin{tabular}[c]{|c|c|}
    \multicolumn{2}{c}{$S$} \\ \hline
    $t$  & $v$ \\ \hline
    1  & 0 \\
    3  & 1 \\
    6  & 0 \\
    10  & 1 \\ \hline
  \end{tabular} \qquad
  \begin{tabular}[c]{|c|c|}
    \multicolumn{2}{c}{$S_{D1}$} \\ \hline
    $t$  & $v$ \\ \hline
    5  & 1 \\
    10  & 1 \\ \hline
  \end{tabular} \qquad
  \begin{tabular}[c]{|c|c|}
    \multicolumn{2}{c}{$S_{D2}$} \\ \hline
    $t$  & $v$ \\ \hline
    6  & 0 \\
    8  & 0 \\
    10  & 1 \\ \hline
  \end{tabular} \qquad
  \begin{tabular}[c]{|c|c|}
    \multicolumn{2}{c}{$S'$} \\ \hline
    $t$  & $v$ \\ \hline
    5  & 1 \\
    6  & 0 \\
    8  & 0 \\
    10  & 1 \\ \hline
  \end{tabular}
  \caption{Taules de les sèries temporals per l'operació de plec de  \glssymbol{not:sgstm:serietotal}}
  \label{fig:multiresolucio:multiresolucio}
\end{figure}
 


\end{example}








\subsection{Demostració}

Cal demostrar l'equivalència formalment\todo{}





\section{Sistemes duals amb SGST i SGSTM}
\todo{}


Oferir sistemes duals de \gls{SGSTM} i \gls{SGST} amb operacions
  de consulta multiresolució.

* Sistemes on el SGSTM funcionen com a cache per a consultes multiresolució, cache de consultes que mai s'han fet però que s'ha previst que es puguin fer -> computació en flux

* Sistemes on la informació en un warehouse SGST (que es consulta rarament) serveix per si es vol fer un canvi d'esquema  en un SGSTM poder recalcular les dades que hi hauria hagut en el SGSTM. Altrament el SGSTM ha de començar de nou. Pensem en un canvi d'esquema com per exemple ampliar un disc o canviar la $f$, un canvi com reduir un disc sí que es pot recalcular

* Posar símil amb la compressió multimèdia: es tenen fitxers lossless que s'emmagatzemen i es fan servir rarament, es fan circular fitxers lossy que ocupen menys i són més àgils per treballar; en el cas que calgui modificar un lossy  es regenera un de nou a partir del lossless. Sobretot es fa per evitar les pèrdues encadenades entre compressions lossy.


\todo{notar que els SGSTM es basen en els SGST}



La multiresolució(S) és una operació computada en temps diferit.

La SèrieTotal(S) és una operació computada en línia, és a dir seguint el flux de $S$. El temps de comput no és tant crític perquè es reparteix al llarg del temps, és a dir tal com es van adquirint les dades; més enllà del temps de càlcul de cada funció d'agregació: aquest limita quantes sèries temporals multiresolució diferents i quantes resolucions de cada es poden emmagatzemar un mateix aparell.





% \subsection{Two database structures}

% \acro{MTSMS} imply a data information selection and so the information
% not considered important is discarded.  Therefore, this systems are
% not adequate when all the monitored data must be kept as
% acquired. This can happen for example when it is not known a priori
% which aggregate functions will work better with the future data
% monitored or when detailed questions must be retrieved such as at what
% hour exactly an event triggered. 

\begin{figure}
  \centering
  %\usetikzlibrary{shapes,arrows,positioning}
\begin{tikzpicture}[scale=0.8, every node/.style={transform shape}]

      \tikzset{
        mynode/.style={rectangle,rounded corners,draw=black, 
          very thick, inner sep=1em, minimum size=3em, text centered,
          groc},
        myarrow/.style={->, shorten >=1pt, thick},
        mylabel/.style={text width=7em, text centered},
        groc/.style={top color=white, bottom color=yellow!50},
        verd/.style={top color=white, bottom color=green!50},
        roig/.style={top color=white, bottom color=red!50},
      }  






 \node[mynode] (m) {$S$};

 \node[right=2cm of m] (mdins) {};

 \node[mynode, verd, above right=0.6cm and 1cm of mdins] (tsms) {\glstext{SGST}};

 \node[mynode, verd, below right=0.6cm and 1cm of mdins] (mtsms) {\glstext{SGSTM}};

 \node[rectangle,draw,minimum height=6cm,minimum width=9.5cm,right=-0.25cm of mdins] (dual) {};

\draw[shift=( dual.south west)]   
  node[above right] {sistema dual de multiresolució};






 \node[mynode,right=3cm of mtsms] (ts) {$S'$};



 \draw (m.east) -- (mdins.east) node[above right,at start]
 {afegeix$(m)$};

 \draw[myarrow] (mdins.east) -- (tsms.west);
 \draw[myarrow] (mdins.east) -- (mtsms.west);


 \draw[myarrow] (tsms) -- (ts) node[above,midway,sloped]
 {$\glssymbol{not:sgstm:multiresolucio}(S,\glssymbol{not:esquemaM})$}; 
 
 \draw[myarrow] (mtsms) -- (ts) node[above,midway,sloped]
 {$\glssymbol{not:sgstm:serietotal}(M)$};




 \node[right=6cm of tsms] (consdins) {};

 \draw (tsms) -- (consdins.center);
 \draw (ts) -- (consdins.center);

 \node[right=2.5cm of consdins] (consultes) {};
 \draw[myarrow] (consdins.center) -- (consultes) node[above,midway,sloped]
 {consultes};



\end{tikzpicture}



%%% Local Variables:
%%% TeX-master: "../main"
%%% End:

  \caption{Sistema dual \gls{SGST}+\gls{SGSTM}}
  \label{fig:multiresolucio:dual}
\end{figure}

% This may be overcome with dual \acro{DBMS} that share measure input,
% as shown in Figure~\ref{fig:model:mtsms-tsms}. One is a \acro{TSMS}
% for long-term deposit and only consulted in occasional cases, it can
% be \acro{TSMS} with other compression techniques or large size
% \acro{DBMS}. The other is a \acro{MTSMS} that lossy compresses time
% series.

% Then generic queries can be made from \acro{TSMS} information or from
% \acro{MTSMS} selected information, as instance by TotalSeries. In
% Figure~\ref{fig:model:mtsms-tsms} the logical equivalence of applying
% $S'=\totalseries(M)$ to \acro{MTSMS} and $S'=
% \multiresolution(S,\text{sch})$ to \acro{TSMS} is plotted, however the
% implementation consideration are different. If we regard the resulting
% time series $S'$ as a view of database information, then the
% $\multiresolution$ is a view that must be computed fully at every new
% measure addition and $\totalseries$ is a view that can be computed
% incrementally at every new addition as explained next.


% Time series data volume uses to be very big and increases along
% time. However, this increase is due to a continuous acquisition of
% data. When data comes as an ordered sequence of instances it is called
% data stream, then specific \acro{DBMS} are designed to manage data
% stream data \cite{stonebraker05:sigmod}.  \acro{MTSMS} can take
% advantage of data stream orientation in order to simplify the
% consolidation process.  Assuming a time order acquisition of time
% series, the update of a \acro{MTSMS} only consists in the addition of
% new measures and the incremental consolidation of subseries.  As a
% consequence \acro{MTSMS} can be seen as a time series view
% pre-computing system for a pair of aggregation statistics and time
% resolution operations.  Then this pre-computation can be used for
% another queries, limited to the aggregations previously computed, or
% for graphical visualisations like the ones done by RRDtool
% \cite{rrdtool}.

% We then have showed a structure for manipulating in time order as then there are no updates in data and it can be managed more simpler. 

%* Important! dir que si seguim l'addició amb ordre de les mesures, aleshores es pot fer l'stream en els MTSMS ja que no hi ha possibilitat d'operacions d'UPDATE.



\subsection{Conceptes relacionats}


\textcite{marz13:nosql13, marz14:bigdata} generalitzen un concepte
similar al de \gls{SGST} dual, ho emmarquen en l'àmbit dels \gls{SGBD}
per a \emph{Big Data}.  Proposen \gls{SGBD} dissenyats amb tres
nivells, que anomenen arquitectura \emph{Lambda}:
\begin{itemize}
\item Nivell \emph{batch}: Emmagatzema totes les dades originals i
  permet realitzar qualsevol consulta sobre aquestes dades. Preveu que
  algunes consultes operen sobre dades consultades prèviament, per
  tant en aquest nivell es gestionen també aquestes consultes
  precomputades, les quals a més es poden obtenir amb computació
  para\l.lela com per exemple amb Hadoop. 

\item Nivell \emph{server}: Emmagatzema les consultes precomputades i
  n'ofereix les dades per a altres consultes. Les consultes
  precomputades s'han de tornar a calcular periòdicament i en el
  nivell \emph{server} sempre hi ha la versió calculada més
  recent. Per tant, es preveu que les consultes precomputades no
  ofereixen la informació actualitzada al moment, sinó que hi ha un
  cert temps des que es modifiquen les dades originals fins que té
  impacte en les consultes.

\item Nivell \emph{speed}: Precomputa les mateixes consultes que el
  nivell \emph{batch} però incrementalment, és a dir cada cop que
  s'afegeix una dada nova les dades de la consulta s'actualitzen
  adequadament.  Aquest nivell només s'usa per a dades recents i per
  tant complementa el problema de les dades desactualitzades en els
  nivells \emph{batch} i \emph{server}.


\end{itemize}

\todo{repassar que es digui això}
All data entering the system is dispatched to both the batch layer and the speed layer for processing.
The batch layer has two functions: (i) managing the master dataset (an immutable, append-only set of raw data), and (ii) to pre-compute the batch views.
The serving layer indexes the batch views so that they can be queried in low-latency, ad-hoc way.
The speed layer compensates for the high latency of updates to the serving layer and deals with recent data only.
Any incoming query can be answered by merging results from batch views and real-time views.





\todo{potser parlar de vistes catxejades?}
Tot i així, ens agradaria comentar els conceptes de Marz amb altres conceptes de SGBD:

és a dir en realitat volem treballar sobre consultes que es basen en altres consultes, és a dir aquestes darreres són el que en SGBD anomenem vistes que són àlies de consultes. Per tant definim una vista com una operació sobre les dades v:=op1(dades) i aleshores podem fer consultes sobre la vista op2(v) que ha de ser el mateix que fer op2(op1(dades)). Per tant en SGBD les vistes s'entén que el resultat es calculen per a cada consulta que les crida. Això no obstant:

* el resultat de les vistes es podria catxejar per a altres consultes, però fins quan és vàlid el catxeig?

  * es podria catxejar periòdicament, per tant pot ser que a vegades es computi amb dades velles. això es correspon amb les batch view que proposa marz? més o menys, ell directament afegeix computació para\l.lela en el càlcul de les view
  * es podria catxejar i tornar-les a recalcular quan canviïn les dades. Els indexos als SGBD fan una cosa similar a això?
  * es podria calcular per primer cop el resultat vista(op1(v)) i quan s'actualitzin les dades amb una operació com up(dades) aleshores traslladar aquesta operació a la vista up'(vista) on s'hauria de determinar la relació entre up i up'  -> això és el realtime que proposa marz? és el que proposen els de data stream?





% \subsection{Stream orientation}

% \todo{}
% * Dues variacions possibles interessants pels MTSMS:

 
%   - Buffers com a streams, sempre de mida fitada 
%   - Discos enllaçats





%%% Local Variables:
%%% TeX-master: "main"
%%% End:






%  LocalWords:  multiresolució


\chapter{Esquemes de multiresolució}

\todo{}


Definim una funció de multiresolució com una consulta sobre una sèrie
temporal que ens retorna una nova sèrie temporal resultant
d'aplicar-li un esquema de multiresolució. 

Aquesta funció de multiresolució ens permetrà:

\begin{itemize}
\item Plantejar problemes on veiem la multiresolució com una consulta sobre una sèrie temporal
\item Oferir sistemes duals de \gls{SGSTM} i \gls{SGST} amb operacions
  de consulta multiresolució.
\item Estudiar implementacions per a la consulta de multiresolució, p.ex. para\l.lelisme.
\item Estudiar la teoria de la informació per a l'esquema de multiresolució
\end{itemize}






\section{Funció de multiresolució}

En el model de \gls{SGSTM} hem definit un model de dades de \gls{SGBD}
per a gestionar sèries temporals multiresolució. Com a \gls{SGBD},
aquest model té una estructura i per tant emmagatzema informació d'una
sèrie temporal en una forma determinada: la de multiresolució.  La
definició com a \gls{SGBD} té com a objectius l'emmagatzematge
compacte de les dades i la selecció de la informació ja preparada per
a consultes posteriors. 

Així, aquest model té capacitats de computació
sincronitzada o en línia (\emph{online}) amb el temps i té
característiques dels sistemes que tracten fluxos de dades (\emph{data
  stream}); és a dir dades que s'estan adquirint contínuament i cal
anar computant al mateix temps que es van adquirint. Això no treu,
però, que de manera més simplificada també es pugui treballar amb un
\gls{SGSTM} en temps diferit (\emph{offline}), és a dir que
s'emmagatzemin les dades adquirides i en el moment que es vulgui
aplicar-hi la consolidació.




Això no obstant, podem simplificar el problema de càlcul de
multiresolució en temps diferit com una consulta en un \gls{SGST} de
transformació d'una sèrie temporal a una nova sèrie temporal.

És a dir, sigui $S$ una sèrie temporal, $M$ una sèrie temporal
multiresolució i $\text{e} = \{ (\delta_0,\tau_0,f_0,k_0), \ldots,
(\delta_d,\tau_d,f_d,k_d)\}$ els paràmetres de l'esquema de
multiresolució de $M$. Afegim totes les mesures de la sèrie temporal a
la multiresolució, $\forall m \in S:
M=\glssymbol{addM}(M,m)$\todo{matemàticament és correcte recursivitat M=f(M)?}, i la
consolidem, $M=\glssymbol{consolidaM}(M')$ fins que $M$ no sigui
consolidable. Consultem la sèrie temporal multiresolució amb les dues
consultes bàsiques, les quals retornen sèries temporals,
$S'=\glssymbol{not:sgstm:serietotal}(M')$ i $S_{\delta
  f}'\glssymbol{not:sgstm:seriedisc}(M',\delta,f)$ on $\delta$ i $f$
són dos paràmetres de l'esquema de multiresolució de $M$.



Plantegem les funcions de transformació de la sèrie temporal original
a les consultades. És a dir, les funcions que anomenem
$\glssymbol{not:sgstm:dmap}$ i $\glssymbol{not:sgstm:multiresolucio}$
i que ens permeten calcular:

\[
\glssymbol{not:sgstm:dmap}: S \times \delta \times f \longrightarrow
S'_{\delta f}
\]


\[
 \glssymbol{not:sgstm:multiresolucio}: S \times e  \longrightarrow S'
\]



Definim la consulta de selecció de disc dels \gls{SGSTM} a
partir del mapatge dels \gls{SGST}.
\begin{definition}[mapa de \glssymbol{not:sgstm:seriedisc}]
  Sigui $S$ una sèrie temporal i $p=(\delta,f)$ els paràmetres de
  multiresolució d'una subsèrie resolució.


$\glssymboldef{not:sgstm:dmap}(\delta,f)=$

\end{definition}


Definim la consulta de sèrie temporal total dels \gls{SGSTM} a partir
del plegament dels \gls{SGST}.
\begin{definition}[plec de \glssymbol{not:sgstm:serietotal}]
\glssymboldef{not:sgstm:multiresolucio}
\end{definition}


  %$\dmap(S,\delta,\tau,f,k)$




% The time series $S_D=\seriedisc(M,\delta,f)$ stored in a resolution
% subseries' disc $R=(B,D)$, where $B=(S_B,\tau,\delta,f)$ and
% $D=(S_D,k)$, can be expressed with a map operation over the original
% time series $S$:

% \[
% \seriedisc(M,\delta,f) \equiv \dmap(S,\delta,\tau,f,k)
% \]
% where
% \[
% \dmap(S,\delta,\tau,f,k) = \map(S_I, m_i\mapsto m') \text{ where }
% \]
% \[
%  m' = f(S, i),\;  i = [T(m_i)-\delta,T(m_i)],
% \]
% \[
%  S_I = \{ (t,\infty) | t\in T_I  \},\;  t_M = T(\max(S)),
% \]
% \[
% T_I = \{ t_I = \tau+n\delta | n\in\N, t_M - k\delta < t_I \leq t_M \}
% \]







És a dir que  si, en temps diferit, hem seguit la mateixa inserció de mesures a un \gls{SGST} i un \gls{SGSTM}, 
serà equivalent a tal tal tal en un moment concret.

És a dir, que si hem inserit les mateixes mesures a un \gls{SGST} i a
un \gls{SGSTM}, per una banda consolidem el \gls{SGSTM} i n'obtenim la
sèrie total i per altra banda consultem la multiresolució al
\gls{SGST}: obtenim la mateixa sèrie temporal.


\subsection{Demostració}

Cal demostrar l'equivalència formalment\todo{}





% \section{Multiresolution over TSMS}
% \label{sec:M+TSMS}


% \todo{explicar mes be l'objectiu d'aquesta seccio}
% \todo{s'hauria de demostrar formalment}

% We have defined the MTSMS model as an structure based on time
% series. Now we express the multiresolution as a query on a time series
% using TSMS operations. 

% Given a time series $S$ and a multiresolution time series $M$, a MTSMS
% consolidates a multiresolution schema for $S$ by applying successively
% $\forall m \in S: M=\addM(M,m)$ and then $M=\consM(M')$ until $M$ is
% no more consolidable. Then over this consolidated $M$ we have defined
% two basic queries $\seriedisc(M,\delta,f)$ and $\totalseries(M)$.

% The time series $S_D=\seriedisc(M,\delta,f)$ stored in a resolution
% subseries' disc $R=(B,D)$, where $B=(S_B,\tau,\delta,f)$ and
% $D=(S_D,k)$, can be expressed with a map operation over the original
% time series $S$:
% \[
% \seriedisc(M,\delta,f) \equiv \dmap(S,\delta,\tau,f,k)
% \]
% where
% \[
% \dmap(S,\delta,\tau,f,k) = \map(S_I, m_i\mapsto m') \text{ where }
% \]
% \[
%  m' = f(S, i),\;  i = [T(m_i)-\delta,T(m_i)],
% \]
% \[
%  S_I = \{ (t,\infty) | t\in T_I  \},\;  t_M = T(\max(S)),
% \]
% \[
% T_I = \{ t_I = \tau+n\delta | n\in\N, t_M - k\delta < t_I \leq t_M \}
% \]


% The total time series $S'=\totalseries(M)$ is a concatenation of all
% $\seriedisc$ without repeated $\delta$. It can be expressed with a
% fold operation over the original time series $S$. Let $\text{sch} = \{
% (\delta_0,\tau_0,f_0,k_0), (\delta_1,\tau_1,f_1,k_1), \ldots,
% (\delta_d,\tau_d,f_d,k_d)\}$ be the parameters of $M$ expressed as a
% multivalued time series
% \[
% \totalseries(M) \equiv \multiresolution(S,\text{sch})
% \]
% \[
% \multiresolution(S,\text{sch}) = \orderfold(\text{sch},\{\},f_r,\min) \text{ where }
% \]
% \[
% f_r: S_i \times (\delta_c,\tau_c,f_c,k_c) \mapsto S_i ||
% \dmap(S,\delta_c,\tau_c,f_c,k_c)
% \]



% Summarising, a MTSMS and the multiresolution query are equivalent on
% functionality.



% \subsection{Two database structures}

% \acro{MTSMS} imply a data information selection and so the information
% not considered important is discarded.  Therefore, this systems are
% not adequate when all the monitored data must be kept as
% acquired. This can happen for example when it is not known a priori
% which aggregate functions will work better with the future data
% monitored or when detailed questions must be retrieved such as at what
% hour exactly an event triggered. 

% \begin{figure}
%   \centering
%   %\usetikzlibrary{shapes,arrows,positioning}
\begin{tikzpicture}[scale=0.8, every node/.style={transform shape}]

 \node (m) {measure};

 \node[rectangle,draw,above right=1cm of m] (tsms) {TSMS};
 \node[rectangle,draw,below right=1cm of m] (mtsms) {MTSMS};

 \node[right=4.3cm of m] (ts) {
   \begin{tabular}[h]{|c|c|}
    \multicolumn{2}{c}{$S'$} \\ \hline
     t & v \\ \hline
       &   \\
       &   \\ \hline
   \end{tabular}
 };

 \draw[->] (m.east) -- (tsms.west);
 \draw[->] (m.east) -- (mtsms.west);

 \draw[->] (tsms.east) -- (ts) node[above,midway,sloped] {multiresolution(S)};
 \draw[->] (mtsms.east) -- (ts) node[above,midway,sloped] {TotalSeries(M)};

\end{tikzpicture}



%%% Local Variables:
%%% TeX-master: "../main"
%%% ispell-local-dictionary: "british"
%%% End:

%   \caption{Dual TSMS+MTSMS}
%   \label{fig:model:mtsms-tsms}
% \end{figure}

% This may be overcome with dual \acro{DBMS} that share measure input,
% as shown in Figure~\ref{fig:model:mtsms-tsms}. One is a \acro{TSMS}
% for long-term deposit and only consulted in occasional cases, it can
% be \acro{TSMS} with other compression techniques or large size
% \acro{DBMS}. The other is a \acro{MTSMS} that lossy compresses time
% series.

% Then generic queries can be made from \acro{TSMS} information or from
% \acro{MTSMS} selected information, as instance by TotalSeries. In
% Figure~\ref{fig:model:mtsms-tsms} the logical equivalence of applying
% $S'=\totalseries(M)$ to \acro{MTSMS} and $S'=
% \multiresolution(S,\text{sch})$ to \acro{TSMS} is plotted, however the
% implementation consideration are different. If we regard the resulting
% time series $S'$ as a view of database information, then the
% $\multiresolution$ is a view that must be computed fully at every new
% measure addition and $\totalseries$ is a view that can be computed
% incrementally at every new addition as explained next.


% Time series data volume uses to be very big and increases along
% time. However, this increase is due to a continuous acquisition of
% data. When data comes as an ordered sequence of instances it is called
% data stream, then specific \acro{DBMS} are designed to manage data
% stream data \cite{stonebraker05:sigmod}.  \acro{MTSMS} can take
% advantage of data stream orientation in order to simplify the
% consolidation process.  Assuming a time order acquisition of time
% series, the update of a \acro{MTSMS} only consists in the addition of
% new measures and the incremental consolidation of subseries.  As a
% consequence \acro{MTSMS} can be seen as a time series view
% pre-computing system for a pair of aggregation statistics and time
% resolution operations.  Then this pre-computation can be used for
% another queries, limited to the aggregations previously computed, or
% for graphical visualisations like the ones done by RRDtool
% \cite{rrdtool}.

% We then have showed a structure for manipulating in time order as then there are no updates in data and it can be managed more simpler. 

%* Important! dir que si seguim l'addició amb ordre de les mesures, aleshores es pot fer l'stream en els MTSMS ja que no hi ha possibilitat d'operacions d'UPDATE.




% \subsection{Stream orientation}

% \todo{}
% * Dues variacions possibles interessants pels MTSMS:

 
%   - Buffers com a streams, sempre de mida fitada 
%   - Discos enllaçats





%%% Local Variables:
%%% TeX-master: "main"
%%% End:






%  LocalWords:  multiresolució

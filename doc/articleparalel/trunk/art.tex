



OpenTSDB utilitza Hadoop per a l'emmagatzematge i recuperació massiva de dades. Seria molt interessant afegir-hi una etapa de processament para\l.lel d'operacions de consulta, per exemple la multiresolució.

dou14 proposa una altra mena de processament per part del hardware.




OpenTSDB \parencite{opentsdb} és un sistema
  d'emmagatzematge distribuït de sèries temporals. Basa
  l'emmagatzematge en Apache Hadoop i HBase, els quals permeten
  distribuir les dades ens diferents nodes. Gràcies a aquests
  sistemes, pot emmagatzemar totes les dades originals ja que és una
  estructura en què és ràpid d'escriure-hi i localitzar les dades, cal
  destacar que HBase crea uns índex potents de les dades i això
  s'aprofita per a indexar l'atribut de temps de les sèries temporals.
  Per a consultar les dades defineixen el concepte d'agregadors, tot i
  que només per a interpolacions lineals, i les operacions d'agregació
  es processen en el mateix moment d'executar la consulta.  Així
  doncs, si bé pot recuperar les dades de forma molt ràpida,
  restringeix les consultes a intervals temporals petits per tal que
  les execucions siguin ràpides. Per tant, és un sistema útil sobretot
  per a visualitzar i comparar intervals temporals petits de diferents
  sèries temporals.






 \textcite{dou14:historic_queries_flash_storage} se centren en
  l'àmbit de l'emmagatzematge de sèries temporals en memòries de tipus
  Flash, de les quals noten que tenen propietats diferents a
  l'emmagatzematge tradicional en discs.  Proposen emmagatzemar
  informació de cada sèrie temporal per a poder resoldre tres tipus de
  consultes: agregacions temporals, històrics basats en mostrejos
  aleatoris i cerca de patrons similars.  La tècnica d'agregacions
  temporals que utilitzen és molt semblant a la de RRDtool, és a dir
  agregar i emmagatzemar les dades amb diferents resolucions, tot i
  que implementada i particularitzada per a les memòries Flash, amb
  registre i punters. Per a la cerca de patrons similars indexen les
  sèries temporals de manera similar als algoritmes de iSAX.






Potser un àmbit de comparació són les xarxes de sensors: on la informació s'ha de calcular distribuïda, tot i que a diferència nosaltres amb la computació paralelacentralitzem totes les dades a un node i despŕes distribuïm la computació. També podríem veure-ho com que les dades estan distribuïdes en el HDFS?
Potser les xarxes de sensors lliguen més amb el model en flux de SGSTM ja plantejat?



Un dels camps recents on l'adquisició de sèries temporals hi juga un
paper fonamental és el de les xarxes de sensors. L'abaratiment del
maquinari permet monitorar el procés amb grans quantitats de sensors
inte\l.ligents \parencite{jainagrawal05,yaogehrke02}, els quals tenen
processador i ràdio incorporats però tenen recursos limitats pel que
fa a transmissió, energia i processament i estan sotmesos a la
incertesa dels sensors. Així doncs, el problema de les xarxes de
sensors rau en estudiar l'ús eficient d'aquests recursos, per la qual
cosa actualment trobem dues propostes.  Una solució consisteix en
transmetre la informació a un node central comprimint-la tant amb
agregacions o estadístics com amb
aproximacions \parencite{deligiannakis07}.  Una altra solució
consisteix en tenir les dades distribuïdes en diferents sensors i
decidir com s'ha de resoldre cada consulta tenint en compte que el
processament local és més barat que la
comunicació \parencite{yaogehrke02,gehrkemadden04,bonnet01,kim12:aggregate_sensor_networks}.



\todo{citar kim12}:
Ens parla de les consultes per agregació més freqüents que es volen fer amb les dades: MAX, MEDIAN, etc L'objectiu és reduir el consum d'energia en els nodes.

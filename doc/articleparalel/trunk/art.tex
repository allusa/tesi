






Potser un àmbit de comparació són les xarxes de sensors: on la informació s'ha de calcular distribuïda, tot i que a diferència nosaltres amb la computació paralelacentralitzem totes les dades a un node i despŕes distribuïm la computació. També podríem veure-ho com que les dades estan distribuïdes en el HDFS?
Potser les xarxes de sensors lliguen més amb el model en flux de SGSTM ja plantejat?



Un dels camps recents on l'adquisició de sèries temporals hi juga un
paper fonamental és el de les xarxes de sensors. L'abaratiment del
maquinari permet monitorar el procés amb grans quantitats de sensors
inte\l.ligents \parencite{jainagrawal05,yaogehrke02}, els quals tenen
processador i ràdio incorporats però tenen recursos limitats pel que
fa a transmissió, energia i processament i estan sotmesos a la
incertesa dels sensors. Així doncs, el problema de les xarxes de
sensors rau en estudiar l'ús eficient d'aquests recursos, per la qual
cosa actualment trobem dues propostes.  Una solució consisteix en
transmetre la informació a un node central comprimint-la tant amb
agregacions o estadístics com amb
aproximacions \parencite{deligiannakis07}.  Una altra solució
consisteix en tenir les dades distribuïdes en diferents sensors i
decidir com s'ha de resoldre cada consulta tenint en compte que el
processament local és més barat que la
comunicació \parencite{yaogehrke02,gehrkemadden04,bonnet01,kim12:aggregate_sensor_networks}.



\todo{citar kim12}:
Ens parla de les consultes per agregació més freqüents que es volen fer amb les dades: MAX, MEDIAN, etc L'objectiu és reduir el consum d'energia en els nodes.

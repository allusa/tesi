\section{Conclusions and future work} 
\label{sec:concl-future-work}

In this paper we have shown a MTSMS model, including the requirements
for these special systems and how they can be applied to an example
time series. The main objective is to store compactly a time series
and manage consistently its temporal dimension.

Our MTSMS model proposes to store a time series split into time
subseries, which we call resolution discs.  Each resolution disc has a
different resolution and is compacted with an attribute aggregate
function. Therefore, in a multiresolution database the configuration
parameters are the quantity of resolution discs and the three
parameters associated with each: the consolidation step, the attribute
aggregate function and the capacity.

The data model shown is the first step to develop a complete model for
a MTSMS but in future the operations will be defined. In this context,
there is a need for a model collecting generic properties for the
TSMS, as it can be the time series union operation or the time
interval operations. Then, the multiresolution model would be build
upon the generic TSMS model.

In an example we have shown a possible application of a MTSMS. The
resulting database has the information we have extracted with the
attribute aggregate function. We show that in this example we want
not an approximation to the original function but an extraction of
some interesting information. Then the database is ready to answer
time series questions keeping in mind that it holds this information
summary.

It may not only be interesting to keep more resolution at more recent
times. Multiresolution could be selectable, that is high resolution is
kept for periods of time considered interesting. For this purpose, the
model of multiresolution needs to be changed accordingly. Furthermore,
an interesting feature for MTSMS is to be able to configure resolution
discs on top of the others, that is setting the one's disc as the
buffer for another. The multiresolution model presented shows the
simple case where each resolution disc is independent from the others.

Time series have meta information associated \cite{dreyer94}: last
value measured, value units, classification tags, location, etc. RDBMS
are quite good at managing this kind of information. Therefore it
would be great if RDBMS and TSMS could interrelate.

Sensor networks is a promising field for TSMS. Sensor networks focus
on monitoring great amount of sensors with limited resources which
must be used efficiently \cite{yaogehrke02}. One approach is to store
data distributed in sensors and resolve each query accordingly
\cite{bonnet01}. From this point of view, it is useful to treat time
series as data streams, which means continuously arriving data in time
order, and operate with them incrementally as data arrives.  With
reference to MTSMS, most recent data could be stored distributed in
sensors and least resolution summarised data stored in central
nodes. Then queries should be resolved distributed: if adequate
resolution is available then it is calculated, else query is sent to
sensor.  With reference to data streams in MTSMS, it seems that
buffers should be reviewed to incorporate streaming capabilities and
aggregate functions should be restricted only to those being able
to calculate incrementally.


Concluding, in this paper we show that using TSMS facilitates
substantially time series management. The current field interest makes
us optimistic to expect soon an adequate time series management in
DBMS.


\section*{Acknowledgements}

This work was supported by Universitat Polit\`{e}cnica de Catalunya (UPC).

Example data comes from i-Sense project.
%http://www.i-sense.org/






%%% Local Variables:
%%% TeX-master: "main"
%%% ispell-local-dictionary: "british"
%%% End:

% LocalWords: multiresolution MTSMS DBMS RDBMS TSMS

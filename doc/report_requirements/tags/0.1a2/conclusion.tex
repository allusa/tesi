\section{Related Work}

The research in time series data management has increased last decade
as the procedure of processing and synthesizing information becomes
complicated if data varies over time. From this point of view we focus
on DBMS aimed to time series where we find two main perspectives in
current research.


On the one hand, there are DBMS with some properties and operations
for time series.

RRDtool from Oetiker \cite{rrdtool} is a professional database
management system extremely used by the free software community. It is
used in professional monitoring systems and its efficiency and speed
when processing time series is being improved. Nevertheless, it is
focused to a particular kind of data, gauges and counters, and it has
not general time series operations.

Cougar \cite{bonnet01} is a sensor database system. It has two
structures: one for sensor properties stored into relations and
another for time series stored into data sequences from sensors.  Time
series have specific operations and can combine relations and
sequences.


SciDB \cite{stonebraker09:scidb} and SciQL \cite{zhang11} are array
database systems. These systems are intended for science applications,
in which time series play a principal role. They structure time series
into arrays in order to achieve multidimensional analysis.



On the other hand, there is the relational DBMS model as the common
study for DBMS theories. The relational DBMS model is continuously
evolving \cite{date:thethirdmanifesto}.

Particularly regarding time data,
intensive research has been carried in the bitemporal data field, that
is the management of history using time intervals.  The recent
temporal data research in relational DBMS model terms
\cite{date02:_tempor_data_relat_model} marks a promising
foundation. It models bitemporal data as relations extended with time
intervals attributes and extends relational operations in order to
deal with related time aspects.

Although bitemporal data and time series data are not exactly the same
and so can not be treated interchangeably \cite{schmidt95},
time series research can benefit from two aspects of this bitemporal
data research. First, it shows the way to extend relational DBMS with
new types and how to model them. Second, it settles some time-related
concepts that can apply well to time series.






\section{Conclusions} 

In this paper we have shown a MTSMS model, including the requirements
for these special systems and how they can be applied to an example
time series. The main objective is to store compactly a time series
and manage consistently its temporal dimension.

Our MTSMS model proposes to store a time series split into time
subseries, which we call resolution discs.  Each resolution disc has a
different resolution and is compacted with an attribute interpolation
function. Therefore, in a multiresolution database the configuration
parameters are the quantity of resolution discs and the three
parameters associated with each: the consolidation step, the attribute
interpolation function and the capacity.

The data model shown is the first step to develop a complete model for
a MTSMS but in future the operations will be defined. In this context,
there is a need for a model collecting generic properties for the
TSMS, as it can be the time series union operation or the time
interval operations. Then, the multiresolution model would be build
upon the generic TSMS model.

In an example we have shown a possible application of a MTSMS. The
resulting database has the information we have extracted with the
attribute interpolation function. We show that in this example we want
not an approximation to the original function but an extraction of
some interesting information. Then the database is ready to answer
time series questions keeping in mind that it holds this information
summary.



Amb aquesta recerca volem demostrar que l'ús de SGBD per a les sèries temporals facilitarà enormement el treball amb aquestes.
L'interès actual ens fa ser optimistes per aventurar que aviat podrem gestionar les sèries temporals adequadament amb els SGBD.



\section*{Acknowledgements}

This work was supported by Universitat Polit\`{e}cnica de Catalunya (UPC).

Data comes from iSense project \todo{isense}.







%%% Local Variables:
%%% TeX-master: "main"
%%% ispell-local-dictionary: "british"
%%% End:

% LocalWords:  DBMS

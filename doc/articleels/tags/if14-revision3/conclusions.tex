\section{Conclusions}
\label{sec:concl-future-work}

In this paper, we formalised a \acro{MTSMS} model. The foundation of
\acro{MTSMS} is the \acro{TSMS} model, which we also formalised.  We
structured our models based on set theory, and heavily inspired on
relational algebra. We have gone a bit further, and we proposed a
\acro{TSMS} model that includes set, sequence, and temporal function
behaviour. We also motivated the interest of multiresolution and its
advantages.  As a reference implementation, we developed a
\emph{Python} package focused on the basic algebra, i.e., without the
extended \acro{DBMS} capabilities.


The main purpose of a \acro{MTSMS} is to store compactly a \emph{time
  series} and to operate its temporal dimension consistently.  It
stores time series using \emph{multiresolution time series}, that is,
it stores a time series at multiple resolutions called
\emph{resolution subseries}. Any resolution subseries has two key
features: its resolution step and its \emph{attribute aggregate
  function}, that is used to compact the data. According to this
structure, a multiresolution time series is configured with a few
parameters. These parameters are the number of resolution subseries,
and for every subseries: the resolution step, the initial
consolidation time, the attribute aggregate function, and the
capacity.  If we tune these parameters properly, a multiresolution
database can keep the desired data from a time series.


We have shown some aggregation functions examples with simple
aggregation statistics, mean and maximum, and simple
\emph{representation methods}, \dd{} and \zohe{}. More attribute
aggregation functions could be designed based on methods from other
fields, such as data streaming or time series data mining. Especially,
it would be interesting to design aggregations that coped with
uncertain data.  The model allows users to customise a multiresolution
database according to the actual requirements of a given context.


The queries over \acro{MTSMS} obtain time series from stored
multiresolution time series. In this way \acro{TSMS} operators can be
applied if needed. The $\seriedisc$ time series being \emph{regular}
facilitates these operations. However, the lossy storage implies that
some operations will give approximate queries and that not
every \acro{TSMS} operation will be semantically correct for a
multiresolution time series. Therefore, the correct planning of the
multiresolution schema is mandatory.


Compared to other \acro{TSMS}, we introduce a compression solution
that stores only the data that can be required in later queries. We do
not intend to reconstruct the original signal. Our multiresolution
solution copes well with some difficult aspects of time series:
regularity, data validation and data volume.  The decompression time
is minimal as data in discs get stored directly as a time series. As a
consequence, computing a query has a small time cost. Moreover, if the
query is an aggregation or resolution already calculated in a
\acro{MTSMS} consolidation, then the response is immediate.


Compared to other \acro{TSMS}, we formalised our model using set
algebra without specifying any particular query language.  This allows
us to achieve an independent model from an actual implementation or an
actual query language.  In future work, a particular query language
might be defined that facilitates to compare our multiresolution
solution with other approaches. As our model is heavily inspired on
relational algebra, we plan to implement this query language using
academically relational query languages, such as Tutorial~D
\cite{date:introduction}.  We think that this procedure will allow us
to illustrate how cumbersome the multiresolution time series use cases
are when we use relational languages.


When we apply a \acro{MTSMS} to store time series data, we discard
some data given its lossy nature.  In future work, it would be
interesting to apply information theory to measure the information
lost depending on the configuration of the multiresolution schema. We
think that it is possible to get inspired by the approaches used by
some authors on multimedia lossy compression techniques. For example,
we could evaluate whether a human can distinguish some specific
features in a time series visualisation based on the original time
series and based on some multiresolution time series. Alternatively,
given a query, we could compare the difference between the results
obtained by applying the query to a multiresolution time series or
applying the query to the original time series.

%%% Local Variables:
%%% TeX-master: "main"
%%% ispell-local-dictionary: "british"
%%% End:


%  LocalWords:  multiresolution TSMS MTSMS subseries lossy

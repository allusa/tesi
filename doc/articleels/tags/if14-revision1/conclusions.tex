





\section{Conclusions}
\label{sec:concl-future-work}


In this paper we have shown a \acro{MTSMS} model, including a
motivation example for multiresolution and an application together
with \acro{TSMS}. Our \acro{MTSMS} model is based on \acro{TSMS}
notation which we have described firmly rooted on set and relational
algebra. We have gone a bit further and proposed \acro{TSMS} including
set, sequence and temporal function behaviour.



The main objective of a \acro{MTSMS} is to store compactly a time
series and manage consistently its temporal dimension.  It stores
multiresolution time series, that is time series split into time
subseries called resolution subseries.  Each resolution subseries has
a different resolution and is compacted with an attribute aggregate
function. Therefore, each multiresolution time series is configured by
the quantity of resolution subseries and four parameters for each: the
consolidation step, the initial consolidation time, the attribute
aggregate function, and the capacity.  These configuration parameters
are degrees of freedom for each application. Giving different values a
multiresolution database is capable to keep the desired information
from a time series. %
We have showed some aggregation functions examples with simple
aggregation statistics, mean and maximum, and simple representation
methods, Delta and \zohe{}. More attribute aggregation functions could
be designed based on methods from other fields such as data streaming
or time series data mining, especially it would be interesting aggregations with uncertain data.


The queries over \acro{MTSMS} obtain time series from stored
multiresolution time series. In this way \acro{TSMS} operators can be
applied if needed. The $\seriedisc$ time series being regular
facilitates these operations. However, the lossy storage implies that
some operations will give approximate queries and that not every
\acro{TSMS} operation will be semantically correct for a
multiresolution time series. Therefore the correct planning of the
multiresolution schema is needed.


Compared to other \acro{TSMS} we propose a compression solution that
stores only the information we will require by latter queries or by
human visualisation, instead of trying to reconstruct the original
signal.  Moreover, our multiresolution solution copes well with
typical problematic properties of time series: regularity, data
validation and data volume.  The decompression time is minimal as data
in discs get stored directly as a time series. As a consequence, the
queries or visualisation computing time is only due to the computation
itself. Moreover, if the query is an aggregation or resolution already
computed in \acro{MTSMS} consolidation, then the visualisation is
immediate.


\acro{MTSMS} imply a data information selection and so the information
not considered important is discarded. 
% When this is not possible, we
% have showed a dual structure of \acro{TSMS} and \acro{MTSMS}. Then a
% \acro{TSMS} stores losslessly and a \acro{MTSMS} takes advantages of
% manipulating data in time order in order to achieve pre-computed
% queries in a stream-like orientation. 
In future work, information
theory has to be evaluated for multiresolution schemes. As multimedia
lossy compression techniques are well founded on information theory,
similar approaches could be taken for multiresolution time series,
e.g. evaluating whether a human can visualise original qualities in
the multiresoluted time series or evaluating
whether given a query it has the same validity for a multiresoluted
one as it has for the original.




A \acro{MTSMS} could be implemented as a SQL \acro{DBMS} system or as
a NoSQL one. As a referent implementation we have developed a
\emph{Python} package centred on the basic algebra, that is without
extended \acro{DBMS} capabilities. Regarding other implementations,
\emph{RRDtool} can be seen as an specific case of \acro{MTSMS} and as
a NoSQL system, although Oetiker \cite{rrdtool} has not commented
it. However, regardless of the implementation backend, we have shown
how a generic model for \acro{MTSMS} can be defined firmly rooted on
\acro{DBMS} algebra theory.



%%% Local Variables:
%%% TeX-master: "main"
%%% ispell-local-dictionary: "british"
%%% End:


%  LocalWords:  multiresolution


\section{Case study}
\label{sec:example}

In this section we introduce a real case database for a time series
data. Actual data come from a temperature distributed sensor
monitoring system~\cite{alippi10}. We focus on the data of a specific
sensor. We use \texttt{Pytsms} and \texttt{RoundRobinson}
implementations in order to create a \acro{MTSDB} and to query it.

\emph{Data}. Figure~\ref{fig:exemple:original} shows the original
data for one year and a half. The plot interpolates linearly the
measures. In this plot we can see that there are missing data and some
outlying observations. There are $146\,709$ stored values.

\begin{figure}[tp]
  \centering
  %\tikzset{every picture/.style={scale=0.8}}
  %\tikzsetnextfilename{fig_exemple_original}
  %\usetikzlibrary{dateplot}    
\begin{tikzpicture}
    \begin{axis}[
        date coordinates in=x,
%        xticklabel={\pgfcalendar{tickcal}{\tick}{\tick}{\pgfcalendarshorthand{m}{.}}},
        xticklabel={\pgfcalendarmonthshortname{\month} \year},
        xticklabel style= {rotate=15,anchor=east},
        xlabel=Time,
        ylabel=Temperature (K),
        ]
       \addplot[blue] file {dades/matriu0.originalbyday.dat};
%       \addplot[blue] table[col sep=comma] {dades/matriu0.csv};
  \end{axis}
\end{tikzpicture}

  \includegraphics{fig_exemple_original.pdf}
  \caption{Example of a temperature time series data}
  \label{fig:exemple:original}
\end{figure}

\emph{Schema}. We design a \acro{MTSDB} that stores a multiresolution
time series with higher resolution at recent times and with lower
resolution at older times. The schema is illustrated in 
Figure~\ref{fig:exemple:window}. At the top there are four discs with
different number of measures and at the bottom there is a timeline
showing the resolution subseries along time. 
%
Going from most to least granularity, disks are configured as follows:
(i) a measure every 5 h in the fourth disc, which has a capacity of 24
measures and thus it spans 5 days; (ii) a measure every 2 days in the
third disc, with a capacity of 20 measures and thus spanning 40 days;
(iii) a measure every 15 days in the second disc, with a capacity of
12 thus spanning 180 days and; (iv) a measure every 50 days in the
first disc that, with a capacity of 12 measures results in a span of
600 days. This last span is longer than the original time series so
that at least one resolution keeps some data about the complete
original time interval.

\begin{figure}[tp]
  \centering
  \setlength{\unitlength}{1.3mm}
  %\tiny
\setlength{\unitlength}{2mm}
\begin{center}
\begin{multicols}{3} 


    \begin{picture}(14,12)(-7,-6)
    \put(0,-1){\makebox(0,0)[c]{{\color{blue}5 days}}}
      \put(0,0){\circle{10}}
      \put(5,0){\circle{0.8}}
      \put(2.5,4.33){\circle{0.8}}
      \put(-2.5,4.33){\circle{0.8}}   
      \put(-5,0){\circle{0.8}}
      \put(-2.5,-4.33){\circle{0.8}} 
      \put(2.5,-4.33){\circle{0.8}} 
      \put(0,0){\vector(0,1){5}}
      \put(0,0){\oval(5,5)[t]}
      \put(-2.5,0){\makebox(0,0)[c]{$\vee$}}
    \end{picture}


    \begin{picture}(14,12)(-7,-6)
    \put(0,-1){\makebox(0,0)[c]{{\color{red}1 day}}}
      \put(0,0){\circle{10}}
      \put(5,0){\circle{0.8}}
      \put(4.33,2.5){\circle{0.8}}
      \put(2.5,4.33){\circle{0.8}}
      \put(0,5){\circle{0.8}}
      \put(-2.5,4.33){\circle{0.8}}   
      \put(-4.33,2.5){\circle{0.8}}
      \put(-5,0){\circle{0.8}}
      \put(-4.33,-2.5){\circle{0.8}}
      \put(-2.5,-4.33){\circle{0.8}} 
      \put(0,-5){\circle{0.8}}
      \put(2.5,-4.33){\circle{0.8}} 
      \put(4.33,-2.5){\circle{0.8}}
      \put(0,0){\vector(0,1){5}}
      \put(0,0){\oval(5,5)[t]}
      \put(-2.5,0){\makebox(0,0)[c]{$\vee$}}
    \end{picture}

    \tiny
    \begin{picture}(14,12)(-7,-6)
    \put(0,-1){\makebox(0,0)[c]{{\color{green}2 hours}}}
      \put(0,0){\circle{10}}
      \put(5,0){\circle{0.8}}
      \put(4.82,1.29){\circle{0.8}}
      \put(4.33,2.5){\circle{0.8}}
     \put(3.5,3.5){\circle{0.8}}
      \put(2.5,4.33){\circle{0.8}}
      \put(1.29,4.82){\circle{0.8}}
      \put(0,5){\circle{0.8}}
      \put(-1.29,4.82){\circle{0.8}}
      \put(-2.5,4.33){\circle{0.8}}
       \put(-3.5,3.5){\circle{0.8}} 
      \put(-4.33,2.5){\circle{0.8}}
    \put(-4.82,1.29){\circle{0.8}}
      \put(-5,0){\circle{0.8}}
    \put(-4.82,-1.29){\circle{0.8}}
      \put(-4.33,-2.5){\circle{0.8}}
      \put(-3.5,-3.5){\circle{0.8}} 
      \put(-2.5,-4.33){\circle{0.8 } } 
      \put(-1.29,-4.82){\circle{0.8 }}
\put(0,-5){\circle{0.8 }}
     \put(1.29,-4.82){\circle{0.8 }}
      \put(2.5,-4.33){\circle{0.8}}
      \put(3.5,-3.5){\circle{0.8}} 
      \put(4.33,-2.5){\circle{0.8}}
  \put(4.82,-1.29){\circle{0.8}}
      \put(0,0){\vector(0,1){5}}
      \put(0,0){\oval(5,5)[t]}
      \put(-2.5,0){\makebox(0,0)[c]{$\vee$}}
    \end{picture}


\end{multicols}

\vspace{-10pt}

\setlength{\unitlength}{900sp}
\begin{picture}(14460,5066)(7322,-7927)
\thinlines
{\color[rgb]{0,0,0}\put(7300,-6271){\line( 0,-1){386}}
}%
{\color[rgb]{0,0,0}\put(7782,-6271){\line( 0,-1){386}}
}%
{\color[rgb]{0,0,0}\put(8263,-6271){\line( 0,-1){386}}
}%
{\color[rgb]{0,0,0}\put(8745,-6271){\line( 0,-1){386}}
}%
{\color[rgb]{0,0,0}\put(9227,-6271){\line( 0,-1){386}}
}%
{\color[rgb]{0,0,0}\put(9709,-6271){\line( 0,-1){386}}
}%
{\color[rgb]{0,0,0}\put(10191,-6271){\line( 0,-1){386}}
}%
{\color[rgb]{0,0,0}\put(10673,-6271){\line( 0,-1){386}}
}%
{\color[rgb]{0,0,0}\put(11155,-6271){\line( 0,-1){386}}
}%
{\color[rgb]{0,0,0}\put(11637,-6271){\line( 0,-1){386}}
}%
{\color[rgb]{0,0,0}\put(12119,-6271){\line( 0,-1){386}}
}%
{\color[rgb]{0,0,0}\put(12600,-6271){\line( 0,-1){386}}
}%
{\color[rgb]{0,0,0}\put(13082,-6271){\line( 0,-1){386}}
}%
{\color[rgb]{0,0,0}\put(13564,-6271){\line( 0,-1){386}}
}%
{\color[rgb]{0,0,0}\put(14046,-6271){\line( 0,-1){386}}
}%
{\color[rgb]{0,0,0}\put(14528,-6271){\line( 0,-1){386}}
}%
{\color[rgb]{0,0,0}\put(15010,-6271){\line( 0,-1){386}}
}%
{\color[rgb]{0,0,0}\put(15492,-6271){\line( 0,-1){386}}
}%
{\color[rgb]{0,0,0}\put(15974,-6271){\line( 0,-1){386}}
}%
{\color[rgb]{0,0,0}\put(16456,-6271){\line( 0,-1){386}}
}%
{\color[rgb]{0,0,0}\put(16938,-6271){\line( 0,-1){386}}
}%
{\color[rgb]{0,0,0}\put(17419,-6271){\line( 0,-1){386}}
}%
{\color[rgb]{0,0,0}\put(17901,-6271){\line( 0,-1){386}}
}%
{\color[rgb]{0,0,0}\put(18383,-6271){\line( 0,-1){386}}
}%
{\color[rgb]{0,0,0}\put(18865,-6271){\line( 0,-1){386}}
}%
{\color[rgb]{0,0,0}\put(19347,-6271){\line( 0,-1){386}}
}%
{\color[rgb]{0,0,0}\put(19829,-6271){\line( 0,-1){386}}
}%
{\color[rgb]{0,0,0}\put(20311,-6271){\line( 0,-1){386}}
}%
{\color[rgb]{0,0,0}\put(20793,-6271){\line( 0,-1){386}}
}%
{\color[rgb]{0,0,0}\put(21275,-6271){\line( 0,-1){386}}
}%
{\color[rgb]{0,0,0}\put(7300,-6271){\line( 0,-1){1157}}
}%
{\color[rgb]{0,0,0}\put(9709,-6271){\line( 0,-1){1157}}
}%
{\color[rgb]{0,0,0}\put(12119,-6271){\line( 0,-1){1157}}
}%
{\color[rgb]{0,0,0}\put(14528,-6271){\line( 0,-1){1157}}
}%
{\color[rgb]{0,0,0}\put(16938,-6271){\line( 0,-1){1157}}
}%
{\color[rgb]{0,0,0}\put(19347,-6271){\line( 0,-1){1157}}
}%
{\color[rgb]{0,0,0}\put(21756,-6271){\line( 0,-1){1157}}
}%
{\color[rgb]{0,0,0}\put(7300,-6271){\line( 1, 0){14456}}
}%

\put(7322,-6271){\line( 0,1){3000}}
\put(21756,-7783){\makebox(0,0)[b]{now}}%
\put(7322,-7783){\makebox(0,0)[b]{30 days back}}%

\color{blue}
\put(21782,-5928){\line( -1,0){14460}}
\put(21782,-5928){\line( 0,1){779}}
\put(21782,-5149){\line( -1,0){14460}}
\put(7322,-5928){\line( 0,1){779}}
\put(14530,-5450){\makebox(0,0)[c]{30 days}}

\color{red}
\put(21782,-5149){\line( 0,1){779}}
\put(21782,-4370){\line( -1,0){5772}}
\put(16010,-5149){\line( 0,1){779}}
\put(18700,-4800){\makebox(0,0)[c]{12 days}}

\color{green}
\put(21782,-4370){\line( 0,1){779}}
\put(21782,-3591){\line( -1,0){962}}
\put(20820,-4370){\line( 0,1){779}}
\put(21400,-3950){\makebox(0,0)[c]{2 d}}
\end{picture}%


\normalsize

\end{center}
  \includegraphics{fig_exemple_window.pdf}
  \caption{Schema of multiresolution}
  \label{fig:exemple:window}
\end{figure}

\emph{Attribute aggregate functions}.  In order to illustrate this
example we consolidate all the resolution subseries using the
mean$^\zohe{}$ aggregate function and the two highest resolution
subseries using the maximum$^\zohe{}$ aggregate function. 



\emph{Consolidation}. The time subseries after consolidating the
\acro{MTSDB} are shown in Figure~\ref{fig:exemple:4mrd}. 
Each graphic corresponds to the possible $\seriedisc$ queries, that is
every resolution disc time series from the \acro{MTSDB}. Each title
shows the resolution subseries and its cardinal, and each attribute
aggregate function has different colour.  
%
Time series are plotted using \zohe{} representation function
$S(t)^\zohe{}$. The time axis has \acro{UTC} units rounded to nearest
time points and temperature axis has Kelvin units. Outliers are
marked as discontinuities, for instance see fourth plot's 2938 K
maximum.

\begin{figure}[tp]
  \centering
  % \tikzset{
  %   every picture/.style={scale=0.7},
  % }
  % 
  \begin{tikzpicture}
    \begin{axis}[
        rd,
        date coordinates in=x,
        title={RD: 5h $|24|$},
        xticklabel={\day--\hour:\minute},
        ]
       \addplot[const plot mark right, blue] table[col sep=comma] {imatges/exemple/mrdzohe-matriu0/R18000mean_zohe.csv};
  \end{axis}
\end{tikzpicture}
%
  \begin{tikzpicture}
    \begin{axis}[
        rd,
        date coordinates in=x,
        title={RD: 2d $|20|$},
        xticklabel={\pgfcalendarmonthshortname{\month} \day},
        ]
       \addplot[const plot mark right, blue] table[col sep=comma] {imatges/exemple/mrdzohe-matriu0/R172800mean_zohe.csv};
  \end{axis}
\end{tikzpicture}
%
  \begin{tikzpicture}
    \begin{axis}[
        rd,
        date coordinates in=x,
        title={RD: 15d $|12|$},
        xticklabel={\pgfcalendarmonthshortname{\month} \day},
        y filter/.code = { \pgfmathparse{(#1>320)*330+(#1<320)*#1}},
        ymax = 320,
        clip=false,
        ]
       \addplot[const plot mark right, blue] table[col sep=comma] {imatges/exemple/mrdzohe-matriu0/R1296000mean_zohe.csv};

      \addplot[const plot mark right, orange] table[col sep=comma] {imatges/exemple/mrdzohe-matriu0/R1296000maximum_zohe.csv};

      \node[right] at (axis cs:2011-10-12,330) {\footnotesize(2938)};
       \node (break) at (axis cs:2011-09-18,325)[inner sep=0pt,minimum width=0.75em, minimum height=0.5ex,fill=white] {};
    \draw [fill=red,color=orange] (break.north east) -- (break.north west) (break.south west) -- (break.south east);
       \node (break2) at (axis cs:2011-10-3,325)[inner sep=0pt,minimum width=0.75em, minimum height=0.5ex,fill=white] {};
    \draw [fill=red,color=orange] (break2.north east) -- (break2.north west) (break2.south west) -- (break2.south east);

  \end{axis}
\end{tikzpicture}
%
\begin{tikzpicture}
    \begin{axis}[
        rd,
        date coordinates in=x,
        xticklabel={\pgfcalendarmonthshortname{\month} \year},
        title={RD: 50d $|12|$},
        xlabel=Time (UTC),
        ymax = 320,
        clip=false,
%v1.6     restrict y to domain=0:320,
        y filter/.code = { \pgfmathparse{(#1>320)*330+(#1<320)*#1}},
        ]
       \addplot[const plot mark right, blue] table[col sep=comma] {imatges/exemple/mrdzohe-matriu0/R4320000mean_zohe.csv};
       \addlegendentry{meanzohe};

       \addplot[const plot mark right, orange] table[col sep=comma] {imatges/exemple/mrdzohe-matriu0/R4320000maximum_zohe.csv};
       \addlegendentry{maximumzohe};

       \node[right] at (axis cs:2011-10-12,330) {\footnotesize(2938)};
       \node (break) at (axis cs:2011-08-29,325)[inner sep=0pt,minimum width=0.75em, minimum height=0.5ex,fill=white] {};
    \draw [fill=red,color=orange] (break.north east) -- (break.north west) (break.south west) -- (break.south east);

  \end{axis}
\end{tikzpicture}




%%% Local Variables:
%%% TeX-master: "../../main"
%%% End:

  \includegraphics{fig_exemple_4mrd1.pdf}
  \includegraphics{fig_exemple_4mrd2.pdf}
  \includegraphics{fig_exemple_4mrd3.pdf}
  \includegraphics{fig_exemple_4mrd4.pdf}
  \caption{Resolution subseries in the MTSDB}
  \label{fig:exemple:4mrd}
\end{figure}

In all the four plots, we can see that mean aggregate function has
filled missing data and filtered outlier observations. This is
because the aggregate function comes from a \zohe{} interpretation.
In the 50 days step resolution, the first data point consolidated is
previous to the original time series. However, it is consolidated with
the first known data as its aggregation comes from \zohe{}
interpretation.

Figure~\ref{fig:exemple:4mrdtot} shows the $\totalseries$ queries for
the mean$^{\zohe}$ aggregate function resolution and for the
maximum$^{\zohe}$ resolution.  Each resulting time series is plotted
by interpolating linearly its measures. Note that this linear
rendering looks displaced to the right because of \zohe{} aggregation.
Comparing this figure with the original series in
Figure~\ref{fig:exemple:original}, we observe that it resembles an
incremental low-pass filter because we applied mean aggregation while
the maximum aggregation resembles an envelope function.

%\tikzsetnextfilename{fig_exemple_4mrdtot}
\begin{figure}[tp]
  \centering
  %\tikzset{every picture/.style={scale=0.8}}
  %\begin{tikzpicture}
    \begin{axis}[
        timeseries,
        xticklabel={\pgfcalendarmonthshortname{\month} \year},
        xlabel=Time (UTC),
        ymax = 320,
        clip=false,
%v1.6     restrict y to domain=0:320,
        y filter/.code = { \pgfmathparse{(#1>320)*330+(#1<320)*#1}},
        ]

       \addplot[black!15] file {imatges/matriu0.originalbyday.dat};
       \addlegendentry{original};

       \addplot[blue] table[col sep=comma] {dades/mrdzohe-matriu0/totalmean.csv};
       \addlegendentry{mean};

       \addplot[orange] table[col sep=comma] {dades/mrdzohe-matriu0/totalmax.csv};
       \addlegendentry{max};

%       \node[right] at (axis cs:2011-10-12,330) {\mbox{(2938)}};
       \node (break) at (axis cs:2011-10-01,325)[inner sep=0pt,minimum width=0.9em, minimum height=0.4ex,fill=white] {};
    \draw [fill=red,color=orange] (break.north east) -- (break.north west) (break.south west) -- (break.south east);


  \end{axis}
\end{tikzpicture}



%%% Local Variables:
%%% TeX-master: "../main"
%%% ispell-local-dictionary: "british"
%%% End:

  \includegraphics{fig_exemple_4mrdtot.pdf}
  \caption{$\totalseries$ for the mean$^{\zohe}$ and maximum$^{\zohe}$
    resolutions}
  \label{fig:exemple:4mrdtot}
\end{figure}


In conclusion, this \acro{MTSDB} example schema does not store the
complete original data but a compression of the original function
which contains more data for recent times.  Each of the
$\seriedisc$ time series is regular with $\delta$. Although
$\totalseries$ is not a regular time series, it has piece-wise
regularity as a concatenation of every disc's $\delta$.  The purpose
of this example is to show how the multiresolution is computed for a
time series, it has been computed offline as the original data had
already been acquired. However, a \acro{MTSMS} is designed to
consolidate while the original data are being acquired so that the
multiresolution computation spreads along the acquisition and the
computing time becomes less critical.


%%% Local Variables:
%%% TeX-master: "main"
%%% ispell-local-dictionary: "british"
%%% End:

% LocalWords:  multiresolution MTSDB Pytsms RoundRobinson timeline
%  LocalWords:  subseries UTC Outliers outlier MTSMS

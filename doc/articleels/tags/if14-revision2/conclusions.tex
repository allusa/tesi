\section{Conclusions}
\label{sec:concl-future-work}

In this paper we have formalised a \acro{MTSMS} model. This model is
based on a \acro{TSMS} model, which we have also formalised.  We have
formalised our models based on set theory, heavily inspired on
relational algebra. We have gone a bit further and proposed
\acro{TSMS} including set, sequence and temporal function
behaviour. We have also included the motivation for multiresolution
advantages.  As a reference implementation we have developed a
\emph{Python} package centred on the basic algebra, that is without
extended \acro{DBMS} capabilities.



The main objective of a \acro{MTSMS} is to store compactly a \emph{time
series} and to manage consistently its temporal dimension.  It stores
\emph{multiresolution time series}, that is each time series is stored at
multiple resolutions called \emph{resolution subseries}.  Each resolution
subseries has a resolution step and is compacted with an \emph{attribute
aggregate function}. Therefore, each multiresolution time series is
configured by the number of resolution subseries and four parameters
for each: the resolution step, the initial consolidation time, the
attribute aggregate function, and the capacity.  These configuration
parameters are degrees of freedom for each application. Giving
different values a multiresolution database is capable to keep the
desired data from a time series. %

We have showed some aggregation functions examples with simple
aggregation statistics, mean and maximum, and simple \emph{representation
methods}, \dd{} and \zohe{}. More attribute aggregation functions could
be designed based on methods from other fields, such as data streaming
or time series data mining. Especially, it would be interesting to
design aggregations that coped with uncertain data.  The model allows
the user to customise a multiresolution database according to the
actual requirements of a given context.

The queries over \acro{MTSMS} obtain time series from stored
multiresolution time series. In this way \acro{TSMS} operators can be
applied if needed. The $\seriedisc$ time series being \emph{regular}
facilitates these operations. However, the lossy storage implies that
some operations will give approximate queries and that not every
\acro{TSMS} operation will be semantically correct for a
multiresolution time series. Therefore the correct planning of the
multiresolution schema is needed.

Compared to other \acro{TSMS}, we propose a compression solution that
stores only the data that we will require by later queries or by
human visualisation, instead of trying to reconstruct the original
signal.  Moreover, our multiresolution solution copes well with
typical problematic properties of time series: regularity, data
validation and data volume.  The decompression time is minimal as data
in discs get stored directly as a time series. As a consequence, the
queries or the visualisation computing time is only due to the
computation itself. Moreover, if the query is an aggregation or
resolution already computed in a \acro{MTSMS} consolidation, then the
visualisation is immediate.


\acro{MTSMS} imply data selection and so what is not considered
important is discarded.  In future work, information theory has to be
evaluated for multiresolution schemes. Following multimedia lossy
compression techniques being well founded on information theory,
similar approaches could be taken for multiresolution time series. For
example, we could evaluate whether a human can visualise original
time series qualities in the multiresoluted time series. Or, given a
query, we could compare the difference of applying the query to a
multiresoluted time series instead of applying it to the original time
series.



%%% Local Variables:
%%% TeX-master: "main"
%%% ispell-local-dictionary: "british"
%%% End:


%  LocalWords:  multiresolution TSMS MTSMS subseries lossy

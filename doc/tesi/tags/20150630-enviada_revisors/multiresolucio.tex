
\chapter{Funció de multiresolució aplicada a les sèries temporals}
\label{cap:funciomultiresolucio}


En aquest capítol definim la multiresolució com una funció que
s'aplica a una sèrie temporal i retorna una nova sèrie temporal.
Aquesta nova sèrie temporal és el resultat d'aplicar, en l'àmbit dels
\glspl{SGST}, un esquema de multiresolució a la sèrie temporal
original. Aquesta funció té el mateix efecte que utilitzar, en l'àmbit
dels \glspl{SGSTM}, una sèrie temporal multiresolució amb el mateix
esquema. Tot i així, les aplicacions que resulten d'ambdós casos no
tenen les mateixes propietats.



En els capítols anteriors, hem definit la multiresolució com una
estructura de base de dades per a emmagatzemar i tractar sèries
temporals. La multiresolució també podria ser útil per a operar
directament sobre una sèrie temporal, sense la capacitat
d'emmagatzematge de dades. En aquest capítol avaluem com la funció de
multiresolució por aplicar-se directament als \glspl{SGST}. Aquesta
funció permet:
\begin{itemize}

\item Expressar problemes en què la multiresolució sigui gestionada
  com una consulta sobre una sèrie temporal; és a dir, com una
  operació dels \gls{SGST} que calcula parelles d'agregació d'atributs
  i resolucions temporals per a una sèrie temporal de la mateixa
  manera com ho calculen els \gls{SGSTM} (segons el model del
  \autoref{cap:model:sgstm}).

\item Dissenyar sistemes duals de multiresolució, en els quals una
  sèrie temporal és emmagatzemada doblement en un \glspl{SGSTM} i en
  un \glspl{SGST} amb capacitats de resoldre funcions de
  multiresolució (\seeref{sec:multiresolucio:dual}).

\item Estudiar l'aplicació de la teoria de la informació per a
  determinar l'efecte que produeix la multiresolució en comprimir unes
  dades (\seeref{sec:multiresolucio:teoriainformacio}).

\item Estudiar altres implementacions de la consulta de
  multiresolució, per exemple amb computació distribuïda i para\l.lela
  (\seeref{sec:implementacio:mapreduce}).

\end{itemize}




%es podria implementar com una operació de summarize? \todo{}
A continuació descriurem com aplicar un esquema de
multiresolució a una sèrie temporal mitjançant les operacions mapa i
plec dels \gls{SGST}. Això serà equivalent funcionalment als
\gls{SGSTM}.



\section{Funció de multiresolució}
\label{sec:multiresolucio:funcio}



En el model de \gls{SGSTM} del \autoref{cap:model:sgstm} hem definit
un model de dades per a gestionar sèries temporals
multiresolució. Aquest model té una estructura que emmagatzema la
informació d'una sèrie temporal d'una forma determinada denominada
multiresolució: en l'emmagatzematge és compacten les dades i es
resumeix la informació per a consultes posteriors.

Per aquest motiu, el model de \gls{SGSTM} té capacitats de computació
sincronitzada o en línia (\emph{online}) amb el temps i té
característiques dels sistemes que tracten fluxos de dades (\emph{data
  stream}); és a dir, dades que es van adquirint contínuament i 
gestionant al mateix temps. Això no treu, però,
que de manera més simplificada també es pugui treballar amb un
\gls{SGSTM} en temps diferit (\emph{offline}); és a dir, que primer
s'emmagatzemin les dades adquirides i després, en temps posteriors,
s'hi apliqui la consolidació.



A continuació, simplifiquem el càlcul de la multiresolució per a
poder-lo aplicar, en temps diferit, directament als \gls{SGST}.
Aquest nou càlcul consisteix en una funció que transforma una sèrie
temporal a una nova sèrie temporal. 





\section{Context de la formulació}

Expressem el context de la funció de multiresolució que ens permet
formular dues operacions mapa i plec amb un funcionament equivalent a
un \gls{SGSTM}.
%

Sigui $S=\{m_0,m_1,\dotsc,m_k\}$ una sèrie temporal, $M$ una sèrie
temporal multiresolució i $\glssymbol{not:esquemaM} = \{
(\delta_0,\glssymbol{not:sgstm:f}_0,\tau_0,\glssymbol{not:sgstm:k}_0),
\ldots,
(\delta_d,\glssymbol{not:sgstm:f}_d,\tau_d,\glssymbol{not:sgstm:k}_d)\}$
 l'esquema de multiresolució de $M$
(\seeref{def:sgstm:esquema}).  Els tres passos, de forma resumida, per
a calcular la multiresolució d'una sèrie temporal en un \gls{SGSTM}
són els següents.

\begin{enumerate}
\item S'afegeixen totes les mesures de la sèrie temporal $S$ a la sèrie temporal
multiresolució $M$, recursivament:
\[
M_0=\glssymbol{addM}(M,m_0),M_1=\glssymbol{addM}(M_0,m_1),\dotsc,M_k=\glssymbol{addM}(M_{k-1},m_k)
\]

\item Es consolida la sèrie temporal multiresolució resultant anterior, $M_k$,
  fins que no sigui consolidable
  (\seeref{def:model:buffer_consolidable}). Sigui
  $l\in\glssymbol{not:N}$, recursivament
\[
M'_0=\glssymbol{consolidaM}(M_k),M'_1=\glssymbol{consolidaM}(M_0'),\dotsc\]
\[\dotsc,M'_{l}=\glssymbol{consolidaM}(M'_{l-1}),M'=\glssymbol{consolidaM}(M'_{l})
\]
on $M_k,M'_{0},M'_{1},\dotsc,M'_{l-1},M'_{l}$ són consolidables i $M'$ no és
consolidable.  

\item Es consulta la sèrie temporal multiresolució amb les dues
  consultes bàsiques definides a \textref{sec:sgstm:consultes}, les
  quals retornen les sèries temporals
  $S'=\glssymbol{not:sgstm:serietotal}(M')$ i $S'_{\delta
    f}=\glssymbol{not:sgstm:seriedisc}(M',\delta,f)$; on $\delta$ i
  $f$ són dos paràmetres qualssevol de l'esquema de multiresolució
  $\glssymbol{not:esquemaM}$, és a dir que existeix
  $(\delta,f,\tau,\glssymbol{not:sgstm:k}) \in
  \glssymbol{not:esquemaM}$.
\end{enumerate}



En aquest context, formulem les funcions de transformació que permeten
calcular en un \gls{SGST} les sèries temporals $S'$ i $S'_{\delta f}$
a partir de la sèrie temporal original $S$. Són dues funcions de
multiresolució que anomenem mapa de multiresolució
($\glssymbol{not:sgstm:dmap}$) i plec de multiresolució
($\glssymbol{not:sgstm:multiresolucio}$). Així doncs, l'equivalència
d'aquestes funcions amb les sèries temporals calculades en un
\gls{SGSTM} és la següent:
\[
\glssymbol{not:sgstm:dmap}(S,\delta,f,\tau,\glssymbol{not:sgstm:k})=S'_{\delta f}
\]
\[
 \glssymbol{not:sgstm:multiresolucio}(S,\glssymbol{not:esquemaM})= S'
\]


és a dir amb un funcionament equivalent, en computació diferida, a

\[
\glssymbol{not:sgstm:seriedisc}(M',\delta,f) =
\glssymbol{not:sgstm:dmap}(S,\delta,f,\tau,\glssymbol{not:sgstm:k})
\]
\[
\glssymbol{not:sgstm:serietotal}(M') = \glssymbol{not:sgstm:multiresolucio}(S,\glssymbol{not:esquemaM})
\]


En resum. Primer, s'insereixen les mateixes mesures a un \gls{SGST} i
a un \gls{SGSTM}. Després, en temps diferit: per una banda es
consolida el \gls{SGSTM} i es consulta la sèrie total i el conjunt de
subsèries resolució emmagatzemades en els disc; per altra banda es
calcula en el \gls{SGST} l'operació de
$\glssymbol{not:sgstm:multiresolucio}$ i el conjunt d'operacions
possibles de $\glssymbol{not:sgstm:dmap}$. Aleshores s'obtenen,
respectivament, la mateixa sèrie temporal i el mateix conjunt de
subsèries temporals.





\section{Definicions}

En primer lloc, definim l'operació de mapa mitjançant l'operador de
mapa dels \gls{SGST} (\seeref{def:sgst:mapa}).


\begin{definition}[Mapa de multiresolució]
\label{def:multiresolucio:mapmu}
Sigui $S$ una sèrie temporal i
$(\delta,f,\tau,\glssymbol{not:sgstm:k})$ un tuple de paràmetres d'un
esquema multiresolució. El mapa de multiresolució és
\[
\glssymboldef{not:sgstm:dmap}(S,\delta,f,\tau,\glssymbol{not:sgstm:k})=
\glssymbol{not:sgst:map}(R,g)\]
 on
  \[
  R = \{ (t,\infty) | t\in Z  \},
  \]
  \[
  g(m)=f(S, T(m)-\delta,\delta) %[\tau,\tau+\delta]=[T(m_i)-\delta,T(m_i)]
  \]
 \[
 Z = \{ \tau+n\delta | n\in\glssymbol{not:Z}\;
 \glssymbol{not:wedge}\; T(\max S) - \glssymbol{not:sgstm:k}\delta <
 \tau+n\delta \leq T(\max S) \},
 \]

\end{definition}




En segon lloc, definim l'operació de plec mitjançant l'operador de
plec amb ordre dels \gls{SGST} (\seeref{def:sgst:oplec}) aplicat a
l'esquema de multiresolució.


\begin{definition}[Plec de multiresolució, amb comportament de 
  \glssymbol{not:sgstm:serietotal}]
  \label{def:multiresolucio:plecmu}
  Sigui $S$ una sèrie temporal i $\glssymbol{not:esquemaM} = \{
  (\delta_0,f_0,\tau_0,\glssymbol{not:sgstm:k}_0), \ldots,
  (\delta_d,f_d,\tau_d,\glssymbol{not:sgstm:k}_d)\}$ un esquema de
  multiresolució. El plec de multiresolució d'una sèrie
  temporal és 
  \[
  \glssymboldef{not:sgstm:multiresolucio}(S,\glssymbol{not:esquemaM})=
  \glssymbol{not:sgst:ofold}(\glssymbol{not:esquemaM},\emptyset,g,\min)
  \]
  on, usant la funció de la \textref{def:multiresolucio:mapmu},
  \[
  g(R,(\delta,f,\tau,\glssymbol{not:sgstm:k}))= R ||
  \glssymbol{not:sgstm:dmap}(S,\delta,f,\tau,\glssymbol{not:sgstm:k}).
  \]
\end{definition}

  
Així, el plec de multiresolució és la concatenació de tots els
\glssymbol{not:sgstm:dmap} possibles per a l'esquema
$\glssymbol{not:esquemaM}$. De la mateixa manera que a
\textref{def:sgstm:total}, s'ha d'assumir que
$\glssymbol{not:esquemaM}$ no conté $\delta$ repetits i que els
concatenem per odre de $\delta$. A més, noteu que en l'operació de
plec tractem l'esquema $\glssymbol{not:esquemaM}$ com una sèrie
temporal multivaluada.






\section{Exemples}

Vegem en dos exemples el càlcul de la funció de multiresolució.
Utilitzem la funció d'agregació d'atributs \glssymbol{not:sgstm:maxpd}
 (\seeref{def:sgstm:maxpd}).


\begin{example}[Mapa de multiresolució]
  \label{ex:multiresolucio:dmap}
  Sigui la sèrie temporal $S=\{(1,0),(3,1),(6,0),(10,1)\}$ i els
  paràmetres de multiresolució
  $(\delta=5,f=\glssymbol{not:sgstm:maxpd},\tau=0,\glssymbol{not:sgstm:k}=2)$.
  El mapa de multiresolució resulta en la sèrie temporal
  $S_{5\glssymbol{not:sgstm:maxpd}}'=
  \glssymboldef{not:sgstm:dmap}(S,5,\glssymbol{not:sgstm:maxpd},0,2)$
  on $S'_{5\glssymbol{not:sgstm:maxpd}}=\{(5,1),(10,1)\}$. A
  continuació expressem aquest càlcul pas a pas i a la
  \autoref{fig:multiresolucio:dmap} es visualitzen en taula les sèries
  temporals corresponents:
  \begin{enumerate}
  \item El primer pas és obtenir els instants de temps que
    s'emmagatzemarien al disc d'una sèrie temporal
    multiresolució. Així, els instants de consolidació possibles són
    $Z'=\{\tau+n\delta|n\in\glssymbol{not:Z}\}=
    \{\ldots,-5,0,5,10,15,\ldots\}$. Però un cop consolidat el disc
    només hi haurà els $\glssymbol{not:sgstm:k}=2$ més recents abans
    de $T(\max S)=10$, és a dir $Z=\{t\in Z'| T(\max S) - k\delta
    < t \leq T(\max S)\}=\{5,10\}$.

  \item El segon pas és obtenir a partir de $Z$ la sèrie temporal
    $R$ que correspon a la sèrie temporal que s'inicialitzaria
    al disc encara amb valors desconeguts,
    $R=\{(5,\infty),(10,\infty)\}$.



  \item El tercer pas és calcular la funció d'agregació a $S$ per a
    cada intervals de consolidació del disc de la forma
    $[T(m)-\delta,T(m)]$ on $m\in R$, és a dir $f(S,0,5)$ per a
    l'interval $[0,5]$ i $f(S,5,5))$ per a l'interval $[5,10]$. A tal
    efecte utilitzem el mapa sobre $R$ per a calcular la sèrie
    temporal resultant $S'_{5\glssymbol{not:sgstm:maxpd}}=\{
    (5,f(S,0,5)), (10,f(S,5,5)) \}$, en què
    $f=\glssymbol{not:sgstm:maxpd}$ i per tant resulta en els valors
    ja expressats
    $S'_{5\glssymbol{not:sgstm:maxpd}}=\{(5,1),(10,1)\}$.

    Es pot calcular un pas entremig per tal de mostrar les sèries
    temporals que hi hauria en el buffer abans de cada instant de
    consolidació. Així, per a cada $T(m)$ hi hauria la sèrie
    temporal $S[T(m)-\delta,T(m)]$, és a dir $B=\{
    (5,S[0,5]),(10,S[5,10]) \}=\{ (5,\{(1,0),(3,1)\}),
    (10,\{(5,0),(10,1)\})\}$.
  \end{enumerate}



\begin{figure}[tp]
  \centering
  \begin{tabular}[c]{|c|c|}
    \multicolumn{2}{c}{$S$} \\ \hline
    $t$  & $v$ \\ \hline
    1  & 0 \\
    3  & 1 \\
    6  & 0 \\
    10  & 1 \\ \hline
  \end{tabular} \qquad
  \begin{tabular}[c]{|c|c|}
    \multicolumn{2}{c}{$R$} \\ \hline
    $t$  & $v$ \\ \hline
    5  & $\infty$ \\
    10  & $\infty$ \\ \hline
  \end{tabular} \qquad
  \begin{tabular}[c]{|c|c|}
    \multicolumn{2}{c}{$B$} \\ \hline 
    $t$  & $v$ \\ \hline
    5  &  \raisebox{0pt}[1ex+\height][1ex+\depth]{\begin{tabular}[c]{|c|c|}\hline $t$  & $v$ \\ \hline 1&0\\ 3&1 \\\hline  \end{tabular}} \\\hline
    10  & \raisebox{0pt}[1ex+\height][1ex+\depth]{\begin{tabular}[c]{|c|c|}\hline $t$  & $v$ \\ \hline 6&0\\ 10&1 \\\hline  \end{tabular}} \\ \hline
  \end{tabular} \qquad
 \begin{tabular}[c]{|c|c|}
    \multicolumn{2}{c}{$S'_{5\glssymbol{not:sgstm:maxpd}}$} \\ \hline
    $t$  & $v$ \\ \hline
    5  & 1 \\
    10  & 1\\ \hline
  \end{tabular}
  \caption{Taules de les sèries temporals per l'operació \glssymbol{not:sgstm:dmap}}
  \label{fig:multiresolucio:dmap}
\end{figure}
 
\end{example}

\begin{example}[Plec de multiresolució]
  Sigui la sèrie temporal $S=\{(1,0),(3,1),(6,0),(10,1)\}$ i l'esquema
  de multiresolució
  $\glssymbol{not:esquemaM}=\{(\delta_0=5,f_0=\glssymbol{not:sgstm:maxpd},\tau_0=0,\glssymbol{not:sgstm:k}_0=2),
  (\delta_1=2,f_1=\glssymbol{not:sgstm:maxpd},\tau_1=0,\glssymbol{not:sgstm:k}_1=3)\}$.
  El plec de multiresolució resulta en la sèrie temporal $S'=
  \glssymboldef{not:sgstm:multiresolucio}(S,\glssymbol{not:esquemaM})$ on
  $S'=\{(5,1),(6,0),(8,0),(10,1)\}$. A continuació expressem aquest
  càlcul pas a pas I a la \autoref{fig:multiresolucio:multiresolucio}
  es visualitzen en taula les sèries temporals corresponents:

  \begin{enumerate}
  \item En primer lloc es calcula la sèrie temporal pels paràmetres de
    multiresolució de $\delta_0$:
    $S_{D0}=\glssymbol{not:sgstm:dmap}(5,\glssymbol{not:sgstm:maxpd},0,2)=\{(5,1),(10,1)\}$,
    com ja s'ha vist a l'\autoref{ex:multiresolucio:dmap}.

  \item En segon lloc, es calcula la sèrie temporal pels paràmetres de
    multiresolució de $\delta_1$:
    $S_{D1}=\glssymbol{not:sgstm:dmap}(2,\glssymbol{not:sgstm:maxpd},0,3)=\{(6,0),(8,0),(10,1)\}$,
    de manera similar a com s'ha calculat $S_{D0}$.

  \item En tercer lloc es concatenen les sèries temporals per ordre de
    $\delta$: $\delta_1<\delta_0$. Així, $S'= S_{D1} || S_{D0}$ que
    resulta en els valors ja expressats
    $S'=\{(5,1),(6,0),(8,0),(10,1)\}$ .

  \end{enumerate}
  


\begin{figure}[tp]
  \centering
  \begin{tabular}[c]{|c|c|}
    \multicolumn{2}{c}{$S$} \\ \hline
    $t$  & $v$ \\ \hline
    1  & 0 \\
    3  & 1 \\
    6  & 0 \\
    10  & 1 \\ \hline
  \end{tabular} \qquad
  \begin{tabular}[c]{|c|c|}
    \multicolumn{2}{c}{$S_{D1}$} \\ \hline
    $t$  & $v$ \\ \hline
    5  & 1 \\
    10  & 1 \\ \hline
  \end{tabular} \qquad
  \begin{tabular}[c]{|c|c|}
    \multicolumn{2}{c}{$S_{D2}$} \\ \hline
    $t$  & $v$ \\ \hline
    6  & 0 \\
    8  & 0 \\
    10  & 1 \\ \hline
  \end{tabular} \qquad
  \begin{tabular}[c]{|c|c|}
    \multicolumn{2}{c}{$S'$} \\ \hline
    $t$  & $v$ \\ \hline
    5  & 1 \\
    6  & 0 \\
    8  & 0 \\
    10  & 1 \\ \hline
  \end{tabular}
  \caption{Taules de les sèries temporals per l'operació \glssymbol{not:sgstm:multiresolucio}}
  \label{fig:multiresolucio:multiresolucio}
\end{figure}
 


\end{example}








% \section{Demostració}

% Cal demostrar l'equivalència formalment de \todo{demostrar}


% \[
% \glssymbol{not:sgstm:seriedisc}(M',\delta,f) \equiv
% \glssymbol{not:sgstm:dmap}(S,\delta,f,\tau,k)
% \]
% \[
% \glssymbol{not:sgstm:serietotal}(M') \equiv \glssymbol{not:sgstm:multiresolucio}(S,e)
% \]












\chapter{Sistemes duals de multiresolució}
\label{sec:multiresolucio:dual}


Una sèrie temporal es pot emmagatzemar i gestionar en un \gls{SGST} o
en un \gls{SGSTM}. També es pot dissenyar un sistema dual de
multiresolució en què una sèrie temporal es tracti alhora en un \gls{SGST}
i en un \gls{SGSTM}.

Les equivalències entre els \gls{SGSTM} i les funcions de
multiresolució aplicades a un \gls{SGST}, formulades a
\textref{sec:multiresolucio:funcio}, permeten dissenyar sistemes duals
que tinguin propietats complementàries.  Així, aquests sistemes duals
ofereixen altres utilitats a la multiresolució més enllà de
l'orientació de compressió amb pèrdua que hem descrit en el model del
\autoref{cap:model:sgstm}. 
% 
A continuació:
\begin{itemize}
\item Dissenyem l'estructura d'aquests sistemes duals de multiresolució.
\item Avaluem conceptes relacionats en l'àmbit genèric dels
  \gls{SGBD}. Particularment la relació que hi ha amb la precomputació
  de consultes i amb el concepte de vistes dels \glspl{SGBDR}.
\item Mostrem algunes aplicacions que permeten aquests sistemes duals:
  per a precomputar consultes, per a poder modificar els esquemes de
  multiresolució en un \gls{SGSTM}, per a conservar totes les dades
  originals en un dipòsit massiu però que no cal consultar
  freqüentment, etc.
\end{itemize}





\section{Estructura}

Un sistema dual de multiresolució està format per un \gls{SGST} i un
\gls{SGSTM} on s'emmagatzemen les mateixes sèries temporals. A
cadascun s'hi poden fer les consultes pertinents de cada model per a les
sèries temporal. A més, s'obté el mateix resultat en els dos sistemes
per a les consultes que segueixin les restriccions de la funció de
multiresolució formulades a \textref{sec:multiresolucio:funcio}.


A la \autoref{fig:multiresolucio:dual} es mostra l'estructura d'un
sistema dual de multiresolució. L'usuari percep aquest sistema com un
\gls{SGST} on emmagatzema una sèrie temporal, $S$, i hi gestiona les
consultes. %
% , on algunes d'aquestes consultes són de multiresolució o
% treballen sobre sèries temporals $S'$ que provenen de la
% multiresolució.
Internament hi ha un \gls{SGST} i un \gls{SGSTM} que comparteixen
l'entrada de mesures de la sèrie temporal. Així, quan l'usuari
so\l.licita una multiresolució, $S'$, el sistema dual tant pot
calcular-la a partir del \gls{SGST} amb l'operació de
$S'=\glssymbol{not:sgstm:multiresolucio}(S,\glssymbol{not:esquemaM})$
com a partir del \gls{SGSTM} amb l'operació de
$S'=\glssymbol{not:sgstm:serietotal}(M)$, on
$\glssymbol{not:esquemaM}$ és un esquema de multiresolució i $M$ és
una sèrie multiresolució amb aquest esquema.  La mateixa estructura
també pot servir per al cas de les operacions de
$\glssymbol{not:sgstm:dmap}$ i les de
$\glssymbol{not:sgstm:seriedisc}$.


\begin{figure}
  \centering
  %\usetikzlibrary{shapes,arrows,positioning}
\begin{tikzpicture}[scale=0.8, every node/.style={transform shape}]

      \tikzset{
        mynode/.style={rectangle,rounded corners,draw=black, 
          very thick, inner sep=1em, minimum size=3em, text centered,
          groc},
        myarrow/.style={->, shorten >=1pt, thick},
        mylabel/.style={text width=7em, text centered},
        groc/.style={top color=white, bottom color=yellow!50},
        verd/.style={top color=white, bottom color=green!50},
        roig/.style={top color=white, bottom color=red!50},
      }  






 \node[mynode] (m) {$S$};

 \node[right=2cm of m] (mdins) {};

 \node[mynode, verd, above right=0.6cm and 1cm of mdins] (tsms) {\glstext{SGST}};

 \node[mynode, verd, below right=0.6cm and 1cm of mdins] (mtsms) {\glstext{SGSTM}};

 \node[rectangle,draw,minimum height=6cm,minimum width=9.5cm,right=-0.25cm of mdins] (dual) {};

\draw[shift=( dual.south west)]   
  node[above right] {sistema dual de multiresolució};






 \node[mynode,right=3cm of mtsms] (ts) {$S'$};



 \draw (m.east) -- (mdins.east) node[above right,at start]
 {afegeix$(m)$};

 \draw[myarrow] (mdins.east) -- (tsms.west);
 \draw[myarrow] (mdins.east) -- (mtsms.west);


 \draw[myarrow] (tsms) -- (ts) node[above,midway,sloped]
 {$\glssymbol{not:sgstm:multiresolucio}(S,\glssymbol{not:esquemaM})$}; 
 
 \draw[myarrow] (mtsms) -- (ts) node[above,midway,sloped]
 {$\glssymbol{not:sgstm:serietotal}(M)$};




 \node[right=6cm of tsms] (consdins) {};

 \draw (tsms) -- (consdins.center);
 \draw (ts) -- (consdins.center);

 \node[right=2.5cm of consdins] (consultes) {};
 \draw[myarrow] (consdins.center) -- (consultes) node[above,midway,sloped]
 {consultes};



\end{tikzpicture}



%%% Local Variables:
%%% TeX-master: "../main"
%%% End:

  \caption{Arquitectura dels sistemes duals de multiresolució:
    \gls{SGST}+\gls{SGSTM}}
  \label{fig:multiresolucio:dual}
\end{figure}


Cal aclarir que el model de \gls{SGSTM} està dissenyat en base al
model de \gls{SGST} i per tant aquests primers sempre depenen dels
segons. No obstant això, cal no confondre aquesta dependència amb el
sistema dual, el qual gestiona una mateixa sèrie temporal independentment
en un \gls{SGSTM} i en un \gls{SGST}.




Tot i que per al sistema dual és equivalent calcular la sèrie temporal
resultant a partir del \gls{SGST} o del \gls{SGSTM}, no pot seguir el
mateix procediment en cada cas. Per una banda, la
$\glssymbol{not:sgstm:multiresolucio}(S)$ és una operació computada en
temps diferit; cada cop que s'afegeix una nova mesura cal tornar a
calcular tot el resultat. Per altra banda, la
$\glssymbol{not:sgstm:serietotal}(M)$ és una operació computada en
línia; és a dir seguint el flux d'adicions de les mesures.  Això no
obstant, en un sistema de multiresolució dual no cal que les mesures
s'emmagatzemin físicament en tots dos sistemes, sinó que els buffers
dels \gls{SGSTM} poden treballar amb les mesures emmagatzemades en un
\gls{SGST} massiu com hem comentat a
\textref{sec:multiresolucion:variacionsbuffer}.


El sistema dual dissenyat funciona a partir de l'adició de mesures, de
la mateixa manera que els \gls{SGSTM}. L'ordre d'arribada d'aquestes
mesures és crític en el sistema dual ja que, un cop el \gls{SGSTM}
s'ha consolidat, les dades més antigues que arribin no seran tingudes
en compte i per tant l'equivalència entre les consultes de \gls{SGST}
i \gls{SGSTM} ja no serà certa. Així doncs, si es vol mantenir
l'equivalència, el sistema dual dissenyat té dues
restriccions: només permet operacions d'adició i l'ordre d'adició és
important.  Més endavant, en les aplicacions d'aquests sistemes,
descriurem l'abast d'aquestes restriccions.




\section{Conceptes relacionats}

En l'àmbit genèric dels \gls{SGBD}, hi ha altres sistemes o altres
conceptes semblants a l'estructura de sistema dual que
proposem. Principalment s'utilitzen en dos àmbits similars: en la
precomputació de consultes i en la precomputació de vistes.



\textcite{marz13:nosql13, marz14:bigdata} generalitzen un concepte
similar al de sistema dual, ho emmarquen en l'àmbit dels \gls{SGBD}
per a \emph{Big Data}.  Proposen \gls{SGBD} dissenyats amb tres
nivells, que anomenen arquitectura \emph{Lambda}:
\begin{itemize}
\item Nivell \emph{batch}: Emmagatzema totes les dades originals i
  permet realitzar qualsevol consulta sobre aquestes dades. Preveu que
  algunes consultes operin sobre dades consultades prèviament, per
  tant en aquest nivell es gestionen també aquestes consultes
  precomputades, les quals a més es poden obtenir amb computació
  para\l.lela com per exemple amb Hadoop. Es considera
  que les dades originals són immutables, és a dir que les bases de
  dades només permeten afegir però no modificar.

\item Nivell \emph{server}: Emmagatzema les consultes precomputades i
  n'ofereix les dades per a altres consultes. Les consultes
  precomputades s'han de tornar a calcular periòdicament i en el
  nivell \emph{server} sempre hi ha la versió calculada més
  recent. Per tant, es preveu que les consultes precomputades no
  ofereixen la informació actualitzada al moment, sinó que hi ha un
  cert temps des que es modifiquen les dades originals fins que té
  impacte en les consultes.

\item Nivell \emph{speed}: Precomputa les mateixes consultes que el
  nivell \emph{batch} però incrementalment, és a dir cada cop que
  s'afegeix una dada nova les dades de la consulta \emph{speed}
  s'actualitzen adequadament.  Aquest nivell només s'usa per a dades
  recents per tal de complementar el problema de les dades
  desactualitzades en els nivells \emph{batch} i \emph{server}.
\end{itemize}

L'arquitectura Lambda té moltes similituds amb el treball amb vistes
en els \glspl{SGBDR}.  Una vista (\seeref{sec:estat:sgbdr}) és un
àlies per a una expressió relacional, és a dir una consulta que
s'utilitza en altres consultes. Així doncs, una vista $v$ és un àlies
d'una consulta \emph{op1} sobre unes \emph{dades},
$v:=\text{op1}(\text{dades})$, que si s'utilitza en una altra
consulta, $\text{op2}(v)$, és equivalent a executar totes dues
operacions sobre les dades, $\text{op2}(v) \equiv
\text{op2}(\text{op1}(\text{dades}))$. En aquest sentit, el concepte
de vista s'assembla a les consultes que es basen en altres consultes
proposades per l'arquitectura Lambda de \citeauthor{marz14:bigdata} i
també a les sèries temporals multiresolució que proposem, les quals
podem observar com a vista multiresolució d'una sèrie temporal
original.

En el model relacional \cite[cap.~10.\ Views]{date04:introduction8} es
considera, conceptualment, que les vistes no s'avaluen quan es
defineixen sinó cada cop que s'executa una consulta una variable de la
qual és una vista.  En les implementacions, però, les vistes poden ser
precomputades per tal d'emmagatzemar-ne temporalment els resultats i
poder-los reutilitzar en altres consultes; aleshores les vistes
s'anomenen \emph{snapshots} o \emph{materialized views}. En el context
de sistemes de suport a les decisions, la precomputació també es
preveu en el càlcul de taules resum per a agregacions de les dades
\cite[cap.~22.\ Decision support]{date04:introduction8}.  Això no
obstant, la precomputació de vistes no sempre comporta un millor
rendiment; el concepte de vista del model permet la substitució
algebraica i per tant també permet l'optimització global de la consulta i
l'operació continguda a la vista.



% PostgreSQL no té materialized views però aquí expliquen com definir-ne a partir de triggers. \url{http://tech.jonathangardner.net/wiki/PostgreSQL/Materialized_Views}.



Les vistes precomputades tenen associada una acció per actualitzar de
nou el seu valor, és a dir per a recalcular la consulta que contenen
quan les dades originals han canviat. En usar vistes precomputades cal
preveure el termini de validesa dels càlculs precomputats, com ocorre
en el nivell \emph{server} de l'arquitectura \emph{Lambda}. Així
doncs, les vistes precomputades es poden actualitzar de vàries
maneres:
\begin{itemize}

\item L'usuari decideix manualment quan s'han de tornar a
  computar. Per exemple, es pot utilitzar per a treballar amb les
  dades congelades a un cert instant en el temps (\emph{snapshots})
  sense haver de blocar les operacions de modificació de la base de
  dades \cite[\S{}10.5]{date04:introduction8}.

\item Es computen periòdicament, com també es proposa en el nivell
  \emph{batch} de l'arquitectura \emph{Lambda}.

\item Quan s'utilitzen per primer cop, es computen associades a un
  termini a partir del qual si es tornen a utilitzar s'hauran
  de tornar a computar. 


\item Es computen cada cop que es modifiquen les dades amb les quals
  operen, és a dir quan les dades originals reben una operació
  d'afegir, de modificar o d'actualitzar es torna a computar tota la
  vista.

\item Es computen incrementalment. Quan es modifiquen les dades,
  s'aplica la mateixa operació a la vista precomputada; requereix,
  però, especificar quina operació s'ha d'aplicar per a actualitzar el
  resultat que ja hi ha a la vista precomputada. És a dir, inicialment
  es precomputa el resultat de la vista, sigui
  $v:=\text{op1}(\text{dades})$. Quan es modifiquen les dades
  originals, amb una operació $\text{dades} :=
  \text{op3}(\text{dades})$, s'ha de traslladar aquesta operació a la
  precomputació de la vista, amb una nova operació $v:=
  \text{op3}'(v)$, de manera que s'aconsegueixi que
  $\text{op3}'(v)=\text{op3}(\text{op1}(\text{dades}))$. Així doncs,
  cal determinar $\text{op3}'$ as partir de $\text{op3}$, cosa que pot
  requerir un estudi complicat o fins i tot no ser possible.  Aquesta
  translació és més senzilla quan només hi ha possibilitat
  d'operacions d'afegir noves dades però no de modificar-les: és el
  que es proposa en el nivell \emph{speed} de l'arquitectura
  \emph{Lambda} i el que admet el model de \gls{SGSTM} que
  proposem. \textcite{jagadish95} també proposen una solució semblant
  de mantenir vistes computades incrementalment segons arriben les
  dades amb un estudi contextualitzat en els \emph{data stream}.
% When data comes as an ordered sequence of instances it is called
% data stream, then specific \acro{DBMS} are designed to manage data
% stream data \cite{stonebraker05:sigmod}.  \acro{MTSMS} can take
% advantage of data stream orientation in order to simplify the
% consolidation process.  Assuming a time order acquisition of time
% series, the update of a \acro{MTSMS} only consists in the addition of
% new measures and the incremental consolidation of subseries. 
\end{itemize}



En resum, la sèrie temporal resultant dels sistemes duals de
multiresolució pot ser considerada com una vista calculada sobre les
sèries temporals originals. Aquesta vista pot ser precomputada, cosa
que en els sistemes duals de multiresolució es pot fer de dues
maneres: mitjançant la funció de multiresolució en l'\gls{SGST}, que
s'ha de computar totalment cada cop que s'afegeix una nova mesura, i
mitjançant l'\gls{SGSTM}, que computa incrementalment.  Aleshores
aquestes vistes es poden usar per a altres consultes que tinguin com a
context l'aproximació de multiresolució realitzada o per a
visualitzacions gràfiques, com les que ofereix RRDtool \cite{rrdtool}.






%Estudiar també Aurora
%http://www.cs.brown.edu/research/aurora/vldb03_journal.pdf
% Abadi, D., Carney, D., Cetintemel, U., Cherniack, M., Convey, C., Lee, S., Stonebraker, M., Tatbul, N., and Zdonik, S. Aurora: A New Model and Architecture for Data Stream Management. In VLDB Journal (12)2:120-139, August 2003.





\section{Aplicacions}



Les sèries temporals són dades que s'adquireixen contínuament i per
tant cada cop és més gran el volum de dades que s'ha d'emmagatzemar i
tractar. Aquest gran volum de dades és un problema per a operar amb
les sèries temporals i és un problema en els sistemes que tenen
l'emmagatzematge limitat. En aquest sentit, originalment hem
plantejat el model de \gls{SGSTM} per tal d'oferir una solució
d'emmagatzematge que comprimeix la informació seleccionant-ne una
multiresolució determinada.


Així, un \gls{SGSTM} implica un selecció d'informació i la informació
que no es considera importat és descartada. Aquests sistemes, per
tant, no són adequats quan totes les dades monitorades han de ser
emmagatzemades tal com s'adquireixen. Un cas d'aquests és quan no es
coneixen quines funcions d'agregació són les més escaients per a les
dades futures que s'adquiriran. Un altre cas és quan volem resoldre
consultes detallades sobre les dades, com per exemple: a quina hora
exacta ha ocorregut un esdeveniment.


Els sistemes duals de multiresolució ofereixen una solució per tal
d'emmagatzemar totes les dades i alhora mantenen una gestió de
multiresolució.  En el sistema dual, s'ha d'entendre l'\gls{SGST} com
un emmagatzematge a llarg termini que no és consultat freqüentment;
així pot estar implementat com a \gls{SGBD} per a dades massives o
basat en tècniques de compressió sense pèrdua. L'\gls{SGSTM} s'ha
d'entendre com un emmagatzematge de compressió amb pèrdua que conté
multiresolucions precomputades de la sèries temporal.  El temps de
còmput no és tant crític en els \gls{SGSTM} perquè es reparteix al
llarg del temps, és a dir tal com es van adquirint les dades; més
enllà del temps de còmput de cada funció d'agregació d'atributs, el
qual limita la quantitat de multiresolucions diferents que pot
gestionar un mateix \gls{SGSTM}.


En la compressió de dades multimèdia s'utilitza una tècnica de gestió
similar.  Les dades s'emmagatzemen inicialment amb compressió sense
pèrdua, a partir d'aquestes es generen dades amb compressió amb pèrdua
que ocupen menys i són més àgils per a treballar. En el cas que calgui
modificar les dades, es canvien les comprimides sense pèrdua i es
regeneren de nou les comprimides amb pèrdua. Amb aquesta gestió
s'evita el problema que les compressions amb pèrdua acumulin pèrdua
entre successives modificacions (\emph{generation loss}).



En resum, les aplicacions del sistema dual de multiresolució són les
següents:
\begin{itemize}
\item Sistemes on els \gls{SGSTM} precomputen incrementalment les
  consultes de multiresolució. És a dir, funcionen com a precomputació
  d'informació que es preveuen que es necessitarà; per tant al llarg
  del temps es creen i eliminen vistes segons les necessitats que es
  preveuen. Aleshores els \gls{SGST} funcionen com a emmagatzematge a
  llarg termini que es consulta rarament.  Aquesta aplicació és
  similar a la proposta de l'arquitectura Lambda i a la de les vistes
  precomputades incrementalment.

\item Les dades emmagatzemades en els \gls{SGST} s'utilitzen per al
  farciment inicial dels \gls{SGSTM} gràcies a la funció de
  multiresolució que permet computar les sèries temporals dels discs a
  partir de l'operació \glssymbol{not:sgstm:dmap}. Pot tenir diversos
  objectius:
  \begin{itemize}
  \item Quan es creen les vistes precomputades anteriors, inicialment
    el \gls{SGSTM} contindrà sèries temporals amb valors desconeguts;
    amb la funció de multiresolució es poden inicialitzar amb els
    valors correctes.

  \item Es pot usar per a canviar l'esquema de multiresolució dels
    \gls{SGSTM}. En alguns canvis d'esquema, per exemple ampliar un
    disc, inicialment hi ha dades desconegudes però que es poden
    computar amb la funció de multiresolució.

  \item Es pot usar per a canviar d'un emmagatzematge de sèries
    temporals en \gls{SGST} a un emmagatzematge en \gls{SGSTM}. Cal
    notar que és un canvi irreversible perquè l'emmagatzematge en els
    \gls{SGSTM} és amb pèrdua.
  \end{itemize}

\item Es poden usar els \gls{SGST} per a experimentar amb diversos
  esquemes de multiresolució per a les dades adquirides i així
  observar-ne la idoneïtat i escollir-ne un de millor.

\item En el cas que no es compleixi la restricció d'ordre d'arribada
  de les mesures per a l'equivalència entre els \gls{SGST} i els
  \gls{SGSTM}, es podria refer la informació emmagatzemada en els
  \gls{SGSTM} a partir dels \gls{SGST}.

\end{itemize}


Com a contrapartida, però, en els sistemes duals apareix un \gls{SGST}
amb una gran quantitat de dades. Per tant, cal tenir en compte que si
la informació computada pels \gls{SGSTM} és suficient per a les
consultes que s'han de realitzar, aleshores la informació
emmagatzemada en els \gls{SGST} és redundant. Això no obstant, no és
senzill identificar i predir quan la informació emmagatzemada en el
\gls{SGSTM} serà totalment suficient; en
\textref{sec:multiresolucio:teoriainformacio} descrivim el problema
d'identificar la informació que selecciona i la que perd un
\gls{SGSTM}.


En conclusió, encara que l'objectiu final sigui l'emmagatzematge de
les sèries temporals comprimides amb pèrdua en un \gls{SGSTM}, és a
dir el model proposat originalment, els sistemes duals de
multiresolució es poden utilitzar mentre hi hagi dubtes sobre quin
esquema de multiresolució escollir i eliminar-los un cop es consideri
que l'esquema és correcte. Aleshores, l'estructura de sistema dual
serveix per a observar clarament que els \gls{SGSTM} ofereixen una
sèrie temporal que és una aproximació de l'original i que, per tant,
permeten resoldre consultes aproximades. %
La \autoref{fig:multiresolucio:comparacio-propietats} resumeix les
propietats descrites dels \gls{SGST}, dels \gls{SGSTM} i dels
duals \gls{SGST}+\gls{SGSTM}, on es comparen segons si les compleixen (S) o
si no les compleixen (N).


\begin{table}[tp]
  \centering
  \begin{tabular}[c]{r|ccc}
  & \gls{SGST} & \gls{SGSTM} & \gls{SGST}+\gls{SGSTM}\\\hline
  càlcul de la multiresolució & S & S & S \\
  emmagatzematge comprimit i limitat & N & S & N+S\\
  pèrdua de dades & N & S & N\\
  redundància de dades & N & N & S\\ 
  actualització de la multiresolució & S & N & S\\
  precomputacions en flux & N & S & S\\
  precomputacions en para\l.lel & S & N & S\\
\end{tabular}
  \caption{Comparació de les propietats dels \gls{SGST}, dels \gls{SGSTM} i dels duals \gls{SGST}+\gls{SGSTM}}
  \label{fig:multiresolucio:comparacio-propietats}
\end{table}











%%% Local Variables:
%%% TeX-master: "main"
%%% End:


%  LocalWords:  multiresolució multiresolucions subsèries SGST SGSTM
%  LocalWords:  precomputada

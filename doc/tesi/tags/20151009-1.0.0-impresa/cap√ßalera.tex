%--------------------
% capçalera del document 
%--------------------
\usepackage[utf8]{inputenc}
\usepackage[catalan]{babel}
\usepackage{lmodern}
%\usepackage[T1]{fontenc} 
%\usepackage{parskip}
%---------- Gràfics ------------------------
%\usepackage[final]{graphicx}
%\usepackage{epstopdf}
%\usepackage{tikz}
%\usepackage{epic,xcolor,multicol}
\usepackage{subcaption}
\usepackage{multirow}
\usepackage{pgfplots}
\usetikzlibrary{dateplot}
\def\pgfcalendarmonthshortname#1{%
%\translate{
\ifcase#1\or gen.\or febr.\or març\or abr.\or maig\or juny\or jul.\or ag.\or set.\or oct.\or nov.\or des.\fi}%}
% \def\pgfcalendarmonthname#1{%
% \translate{\ifcase#1\or JJJ\or FFF\or MMM\or AAA\or
% MAY\or JJU\or JJY\or AAU\or SSS\or OOO\or
% NNN\or DDD\fi}}
\usetikzlibrary{shapes,arrows,positioning}
\usetikzlibrary{patterns}
%%%\usepackage{arydshln} %hdashline a les taules --> conflicte amb supertabular 
%----------- Format -------------------------
\usepackage{cclicenses} %\usepackage{ccicons}
%---------- Símbols ------------------------
\usepackage{url} %\url i \path
\usepackage{eurosym}
\usepackage[cmex10]{amsmath}
\usepackage{amssymb}
\usepackage{ifsym}
%%ús coma decimal sense espais:  2{,}5
%per tal que trenqui la coma en math inline mode 
%http://tex.stackexchange.com/questions/19094/allowing-line-break-at-in-inline-math-mode-breaks-citations/
\AtBeginDocument{%
  \mathchardef\mathcomma\mathcode`\,
  \mathcode`\,="8000 
}
{\catcode`,=\active
  \gdef,{\mathcomma\discretionary{}{}{}}
}
%---------- Teoremes ------------------------
\usepackage{amsthm}
\usepackage{thmtools} %listoftheorems??
%teoremes
\theoremstyle{plain}
\newtheorem{definition}{Definició}
%\declaretheorem{definition}
\theoremstyle{definition}
\newtheorem{example}{Exemple}
%\declaretheorem{example}
%numeracions d'equacions, definicions, etc.
\numberwithin{equation}{chapter}
\numberwithin{definition}{chapter}
\numberwithin{example}{chapter}
%%format del listoftherorems
\makeatletter
\def\ll@example{%
  \protect\numberline{\csname the\thmt@envname\endcsname}%
  \ifx\@empty\thmt@shortoptarg
    \thmt@thmname
  \else
    \thmt@shortoptarg
  \fi}
\def\ll@definition{%
  \protect\numberline{\csname the\thmt@envname\endcsname}%
  \ifx\@empty\thmt@shortoptarg
    \thmt@thmname
  \else
    \thmt@shortoptarg
  \fi}
\makeatother
%---------- Bibliografia -------------------
%ús: \cite{} \textcite{} \parencite{} \citeauthor{}
\usepackage[
style=numeric-comp,%style=authoryear
sortcites=true,
%backref=true,
 ]{biblatex}
%\ExecuteBibliographyOptions{annotation=true,backref=true,}
%backref=true, urldate=long, abbreviate=false,%%format web 
% isbn=false,url=false,doi=false,alldates=terse,firstinits=true,abbreviate=true %%format en paper

\DefineBibliographyStrings{catalan}{%
%    andothers = {i altres},
    backrefpage  = {citat a la p\adddot}, % for single page number
    backrefpages = {citat a les pp\adddot} % for multiple page numbers
}

\newcommand{\bibendash}{--}
%\bibparsep 0.2cm
%\bibhang 0.25cm

%afegeix camp eprinttype=hdl
\DeclareFieldFormat{eprint:hdl}{%
  \mkbibacro{HDL}\addcolon\space
  \ifhyperref
    {\href{http://hdl.handle.net/#1}{\nolinkurl{#1}}}
    {\nolinkurl{#1}}}

%anotacions a la bibliografia
\newboolean{bbx@annotation}% (same as biblatex-dw)
\DeclareBibliographyOption{annotation}[true]{%
\setboolean{bbx@annotation}{#1}}
%
\renewbibmacro{finentry}{%
\finentry%
\iffieldundef{annotation}%
{}%
{\ifbool{bbx@annotation}%
{\color{blue}
\begin{quotation}\noindent%
\printfield{annotation}%
\end{quotation}}%
{}}%
}

%cites bibliogràfiques: números de citacions entre comes
%cal escapar els accents: \parencite{tal} {é}s
%http://tex.stackexchange.com/questions/28461/biblatex-tighter-integration-of-textcite-in-the-flow-of-text
%http://tex.stackexchange.com/questions/19627/biblatex-idiom-for-testing-contents-of-list-field
%http://tex.stackexchange.com/questions/26401/help-to-develop-a-textcite-command-to-be-used-with-verbose-citation-styles-in-b
%http://tex.stackexchange.com/questions/28461/biblatex-tighter-integration-of-textcite-in-the-flow-of-text
%
% \makeatletter
% %\DeclareAutoPunctuation{.,;:!?}
% %per defecte \DeclareRangeChars{~,;-+/}
% \DeclareRangeChars*{:}
% %
% \newcommand{\citacomes}[4]{
% \blx@addpunct{comma}\space\cite[#1][#2]{#3}%
% %\blx@imc@ifpunctmark{#4}{\blx@addpunct{comma}\space#4}{#4}%
% \blx@imc@ifnumerals{#4}{%per separar accents
% \blx@imc@ifpunctmark{#4}{\blx@addpunct{comma}\space#4}{#4}}%
% {\blx@addpunct{comma}\space#4}%
% }
% %
% \renewrobustcmd*{\textcite}{\blx@citeargs\cbx@textcite}
% \newcommand{\cbx@textcite}[4]{%
% \citeauthor{#3}%
% \citacomes{#1}{#2}{#3}{#4}%
% }

% \renewrobustcmd*{\parencite}{\blx@citeargs\cbx@parencite}
% \newcommand{\cbx@parencite}[4]{%
% \citacomes{#1}{#2}{#3}{#4}%
% }
% %
% \DeclareCiteCommand{\citeauthor}
%   {\usebibmacro{cite:init}%
%     \boolfalse{citetracker}%
%     \boolfalse{pagetracker}%
%     \usebibmacro{prenote}}%pre
%   {\ifciteindex
%      {\indexnames{labelname}}
%      {}%
%      \iffieldequals{namehash}{\cbx@lasthash}%
%      {}%repetit 
%      {\ifnumequal{\value{citecount}}{1}{}{\multicitedelim}%
%        \printnames{labelname}}%
%      \savefield{namehash}{\cbx@lasthash}%
% }%post
%   {}%\multicitedelim}%sep
%   {\usebibmacro{postnote}}
% %
% \makeatother
%---------- Codi ---------------------------
%% ús del lstlisting
%%\begin{lstlisting}[language=C,caption=Titol del llistat,label=lst:etiqueta,numbers=left]
%%\lstinline[style=sh]!for i:integer;!

\usepackage{upquote} %perquè en verbatim surtin les cometes `
\usepackage{listings}
\lstloadlanguages{bash,C,HTML,Python,XML}
\lstset{numberstyle=\tiny,frame=single,frameround=tttt,
        breaklines=true,breakindent=0pt,
%        prebreak=\mbox{{\color{blue}\tiny$\searrow$}},
%        postbreak=\mbox{{\color{blue}\tiny$\hookrightarrow$}},
        columns=[l]fullflexible,
        xleftmargin=1em,
        extendedchars=true,
        literate={à}{{\`a}}1 {è}{{\`e}}1 {é}{{\'e}}1 {í}{{\'\i}}1 {ï}{{\"\i}}1
                 {ò}{{\`o}}1 {ó}{{\'o}}1 {ú}{{\'u}}1 {ü}{{\"u}}1
                 {ç}{{\c{c}}}1 {l·l}{{\l.l}}1
                 {À}{{\`A}}1 {È}{{\`E}}1 {É}{{\'E}}1 {Í}{{\'I}}1 {Ï}{{\"I}}1
                 {Ò}{{\`O}}1 {Ó}{{\'O}}1 {Ú}{{\'U}}1 {Ü}{{\"U}}1
                 {Ç}{{\c{C}}}1 {L·L}{{\L.L}}1, 
        escapechar=æ,
        }

\lstdefinestyle{py}{
  style=pycolor
}

\lstdefinestyle{pynocolor}{
  language=python,
  frame=none,
  inputencoding=utf8,
  backgroundcolor=\color[gray]{0.95},
}


\lstdefinestyle{pycolor}{
        style=pynocolor,
        basicstyle=\sffamily\footnotesize,
        stringstyle=\color{brown},
        showstringspaces=false,
        alsoletter={1234567890},
        otherkeywords={\ , \}, \{},
        keywordstyle=\color{blue},
        emph={access,and,as,break,class,continue,def,del,elif,else,%
          except,exec,finally,for,from,global,if,import,in,is,%
          lambda,not,or,pass,print,raise,return,try,while,assert},
        emphstyle=\color{orange}\bfseries,
        emph={[2]self},
        emphstyle=[2]\color{gray},
        emph={[4]ArithmeticError,AssertionError,AttributeError,BaseException,%
          DeprecationWarning,EOFError,Ellipsis,EnvironmentError,Exception,%
          False,FloatingPointError,FutureWarning,GeneratorExit,IOError,%
          ImportError,ImportWarning,IndentationError,IndexError,KeyError,%
          KeyboardInterrupt,LookupError,MemoryError,NameError,None,%
          NotImplemented,NotImplementedError,OSError,OverflowError,%
          PendingDeprecationWarning,ReferenceError,RuntimeError,RuntimeWarning,%
          StandardError,StopIteration,SyntaxError,SyntaxWarning,SystemError,%
          SystemExit,TabError,True,TypeError,UnboundLocalError,UnicodeDecodeError,%
          UnicodeEncodeError,UnicodeError,UnicodeTranslateError,UnicodeWarning,%
          UserWarning,ValueError,Warning,ZeroDivisionError,abs,all,any,apply,%
          basestring,bool,buffer,callable,chr,classmethod,cmp,coerce,compile,%
          complex,copyright,credits,delattr,dict,dir,divmod,enumerate,eval,%
          execfile,exit,file,filter,float,frozenset,getattr,globals,hasattr,%
          hash,help,hex,id,input,int,intern,isinstance,issubclass,iter,len,%
          license,list,locals,long,map,max,min,object,oct,open,ord,pow,property,%
          quit,range,raw_input,reduce,reload,repr,reversed,round,set,setattr,%
          slice,sorted,staticmethod,str,sum,super,tuple,type,unichr,unicode,%
          vars,xrange,zip},
        emphstyle=[4]\color{purple}\bfseries,
        morecomment=[s][\color{lightgreen}]{"""}{"""},
        commentstyle=\color{gray}\slshape,
        literate=
          {>>>}{\textbf{\textcolor{blue}{$>$\kern-.5ex$>$\kern-.5ex$>$}~}}3%
          {...}{{\textcolor{gray}{...}}}3%
          {à}{{\`a}}1 {è}{{\`e}}1 {é}{{\'e}}1 {í}{{\'\i}}1 {ï}{{\"\i}}1%
          {ò}{{\`o}}1 {ó}{{\'o}}1 {ú}{{\'u}}1 {ü}{{\"u}}1 {ç}{{\c{c}}}1%
          {l·l}{{\l.l}}1 {À}{{\`A}}1 {È}{{\`E}}1 {É}{{\'E}}1 {Í}{{\'\I}}1%
          {Ï}{{\"\I}}1 {Ò}{{\`O}}1 {Ó}{{\'O}}1 {Ú}{{\'U}}1 {Ü}{{\"U}}1%
          {Ç}{{\c{C}}}1 {L·L}{{\L.L}}1, 
        rulesepcolor=\color{blue},
} 




\lstdefinelanguage{tutorialD}{
  morekeywords={\ , \}, \{, var, operator,with,as,end,begin,type,possrep},
  sensitive=false,
  morecomment=[l]{//},
%  morecomment=[s]{/*}{*/},
%  morestring=[b]",
}

\makeatletter
\newcommand{\lstuppercase}{\uppercase\expandafter{\expandafter\lst@token
                           \expandafter{\the\lst@token}}}
\newcommand{\lstlowercase}{\lowercase\expandafter{\expandafter\lst@token
                           \expandafter{\the\lst@token}}}
\makeatother


\lstdefinestyle{tutorialD}{
        language=tutorialD,
        frame=none,
        inputencoding=utf8,
        backgroundcolor=\color[gray]{0.95},
        basicstyle=\sffamily\footnotesize,
        stringstyle=\color{green},
        keywordstyle={\color{blue}\lstlowercase},
        emph={tuple,relation,base,key,same_type_as,same_heading_as,return,returns,from,execute,init,private,all,but,%
%          where,rename,union,join,minus,extend,add,summarize,group,ungroup
        },
        emphstyle={\color{orange}\bfseries\lstlowercase},
        emphstyle=[2]\color{gray},
        emph={[4]rational,char},
        emphstyle=[4]\color{purple}\bfseries,
        morecomment=[s][\color{lightgreen}]{"""}{"""},
        commentstyle=\color{red}\slshape,
        rulesepcolor=\color{blue},
} 







\lstdefinestyle{sh}{
  language=bash,
  frame=none,
  prebreak =\textbackslash,
  postbreak ={},
  basicstyle=\ttfamily,
  showspaces=false,
  keepspaces=true,
  backgroundcolor=\color[gray]{0.95},
}
\lstdefinestyle{file}{
  frame=none,
  showspaces=false,
  keepspaces=true,
  backgroundcolor=\color{yellow!20!white}
}

\lstdefinestyle{stdout}{
  frame=none,
  keepspaces=true,
  columns=fixed,
  backgroundcolor=\color{yellow!20!white}
}
%--------------------------------------------


%---------- hyperref ------------------------
\usepackage[bookmarks,pdfborder={0 0 0},pdfusetitle]{hyperref}
\usepackage{bookmark}%per netejar l'ordre de \part{}

%modificar autoref per babel catala
\let\orgautoref\autoref
%per: (v. fig. 3)
\providecommand{\seeref}
{%
\def\figureautorefname{fig.}%
\def\definitionautorefname{def.}%
\def\exampleautorefname{ex.}%
\def\chapterautorefname{\gls{capitol}}%
\def\sectionautorefname{\gls{seccio}\ignorespaces}%
\def\subsectionautorefname{\gls{seccio}\ignorespaces}% 
\gls{vegeu}~\orgautoref%
}
%per: a la figura 3 !Problemes en les contraccions amb noms masculins! evitar a i de davant de capítol!
\providecommand{\textref}
{%
\def\figureautorefname{la figura}%
\def\definitionautorefname{la definició}%
\def\exampleautorefname{l'exemple}%
\def\chapterautorefname{el capítol}% al/del capítol???
\def\sectionautorefname{la secció}%
\def\subsectionautorefname{l'apartat}% 
\orgautoref%
}
%per defecte
\renewcommand{\autoref}
{%
\def\figureautorefname{figura}%ús: \autoref{}
\def\tableautorefname{taula}%ús: \autoref{}
\def\definitionautorefname{definició}%ús: \autoref{}
\def\exampleautorefname{exemple}%ús: \autoref{}
\def\chapterautorefname{capítol}%ús: \autoref{}
\def\sectionautorefname{secció}%ús: \autoref{}
\def\subsectionautorefname{apartat}%ús: \autoref{}
\orgautoref%
}


\renewcommand\lstlistingname{Llistat} %%%PENSAR bé el nom
\renewcommand\lstlistlistingname{Índex de llistats}
\def\lstlistingautorefname{llistat} %ús: \autoref{}
%---------- Glossaris -------------------
\usepackage{tabu}
%\usepackage{longtable}
%\usepackage{supertabular}
\usepackage[
          nomain,% Remember if you don't want to use the main glossary.
          acronym,
          %%nonumberlist,
          toc,
          section,
          numberedsection=false,%numberedsection=autolabel,
          sanitize=none, %pels accents en el vegeu
          ]{glossaries}
%\renewcommand{\glossarypreamble}{Text com a préambul}
%\renewcommand*{\glspostdescription}{}%anul·la el punt final
\renewcommand*{\acronymname}{Sigles}%{Índex de sigles}??si té refs pàgines 
%Índex d'abreviacions?? si conté abreviatures o símbols
\renewcommand{\seename}{vegeu}
\renewcommand{\entryname}{Notació}
\renewcommand{\descriptionname}{Descripció}
% \short<type>name,
%Nous usos de símbols
\newcommand{\glssymboldef}{\glssymbol[format=hyperbf,counter=definition]}
\newcommand{\glsdispdef}{\glsdisp[format=hyperbf,counter=definition]}
\newcommand{\hyperbfsec}[1]{\textbf{\S\hypersf{#1}}}
\newcommand{\hyperbfex}[1]{\textbf{ex.\hypersf{#1}}}
\newcommand{\glsaddsec}{\glsadd[format=hyperbfsec,counter=subsection]}
\newcommand{\glsaddchap}{\glsadd[format=hyperbfsec,counter=chapter]}
\newcommand{\glsaddsection}{\glsadd[format=hyperbfsec,counter=section]}
\newcommand{\glsdispsec}{\glsdisp[format=hyperbfsec,counter=subsection]}
\newcommand{\glssymbolsec}{\glssymbol[format=hyperbfsec,counter=subsection]}
\newcommand{\glssymbolex}{\glssymbol[format=hyperbfex,counter=example]}
%Nous glossaris
\newglossary{abreviatura}{abr}{brv}{Abreviatures}
\newglossary{notation}{not}{ntn}{Símbols i notació}
\newglossarystyle{estil-notation}{%
%  \renewcommand{\glsgroupskip}{}% make nothing happen between groups
%  \renewcommand*{\glossaryheader}{}
% {\begin{longtable*}{llp{10em}}}{\end{longtable*}}
  \renewenvironment{theglossary}
  {\begin{center}\begin{supertabular*}{\textwidth}{llp{10em}}}{\end{supertabular*}\end{center}} %
  \renewcommand*{\glossarysubentryfield}[6]{%
    \ifglshaschildren{##2} %
    {\glstarget{##2}{\textbf{\Glsentryname{##2}}}} %hi ha subentrades
    {\space \glstarget{##2}{\Glsentryname{##2}}} % the entry name
    %& (##5) % the symbol in brackets
    & ##4 % the description
    & ##6 % the number list 
%    & \parbox[t]{\hsize}{hola\\ ei} 
    % ##1 entry level
    \\
  }%
  \renewcommand*{\glossaryentryfield}[5]{%   
    %\glossarysubentryfield{##2}{##1}{##2}{##3}{##4}{##5}
    \\    
    \glstarget{##1}{\textbf{\Glsentryname{##1}}}  & ##3 & ##5 \\\hline
  }
}

%Executa els glossaris
\makeglossaries
%--------------------------------------------




%%% Local Variables: 
%%% mode: latex
%%% TeX-master: "main"
%%% End: 

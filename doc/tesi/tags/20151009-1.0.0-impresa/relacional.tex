\chapter{Implementació relacional amb Tutorial~D}
\label{sec:implementacio:relacional}

Els models \gls{SGST} i \gls{SGSTM} es basen fortament en l'àlgebra
relacional.  Així doncs, n'explorem una implementació en un
\gls{SGBDR}. Per tal de mantenir la màxima fidelitat amb els models i
per les consideracions de \textcite[cap.~1--4]{date04:introduction8}
sobre la idoneïtat del llenguatge \gls{SQL} en el model relacional,
implementem els models amb el llenguatge acadèmic dels \gls{SGBDR}:
\emph{Tutorial~D} \parencite{date04:introduction8,date:thethirdmanifesto,date:tutoriald}. Aquest
llenguatge és orientat a l'àlgebra relacional i per tant és còmode per
a implementar els models que hem definit fortament basats en aquesta
àlgebra.  Com a intèrpret per a aquest llenguatge utilitzem
Rel \parencite{rel}, el qual és un \gls{SGBDR} de propòsit acadèmic
per a experimentar amb Tutorial~D.


% Tutorial D is a specific D which is defined and used for illustration in The Third Manifesto. Implementations of D need not have the same syntax as Tutorial D. The purpose of Tutorial D is both educational and to show what a D might be like. Rel is an implementation of Tutorial D.
% Rel is an open source true relational database management system that implements a significant portion of Chris Date and Hugh Darwen's Tutorial D query language.
% Primarily intended for teaching purposes, Rel is written in the Java programming language.



Implementem la part essencial, és a dir l'àlgebra definida en els
models lògics. Com que són de propòsit acadèmic, ni Tutorial~D ni Rel
implementen els complements que tenen els \gls{SGBDR} en entorns
d'explotació. Com a conseqüència, les implementacions resultants són
útils per a comprovar l'àlgebra relacional però no són còmodes per a
treballar amb dades reals, les quals necessiten sovint operacions
per canviar-ne el format, per dibuixar-les, etc.


En primer lloc, implementem un \gls{SGST} relacional anomenat
\emph{Reltsms}. En segon lloc, pel cas dels \gls{SGSTM}, implementem
la funció de multiresolució sobre Reltsms.



\section{Reltsms}


\emph{Reltsms} és una co\l.lecció d'operadors definits amb Tutorial~D
sobre una estructura de sèrie temporal definida com a relació. A
continuació en destaquem algunes definicions i alguns casos d'ús.  La
sintaxi de Tutorial~D utilitzada es correspon amb l'acceptada per
Rel \parencite{rel}.



Tal com s'ha definit el model estructural de \gls{SGST}, en el model
relacional les sèries temporals són relacions amb dos atributs: $t$
pel temps i $v$ pels valors, on l'atribut $t$ fa de clau primària en
les variables relació.  Les mesures són els tuples d'aquestes
relacions.  Per tant, la implementació relacional consisteix a definir
un tipus sèrie temporal i totes les operacions associades. Aquestes
definicions es basaran en Tutorial~D, és a dir que no cal cap
llenguatge no relacional extern.

Tutorial~D defineix estàticament els tipus dels paràmetres de les
operacions, la qual cosa dificulta generalitzar-les per a qualsevol
tipus de temps i valor com s'ha fet a la implementació amb Python. Com
a conseqüència, fixarem a reals els tipus dels atributs temps i dels
valors, \emph{Rational} a Tutorial~D.


Així doncs, definim el valor de sèrie temporal mitjançant
una relació d'atributs $t$ i $v$. Per exemple la sèrie temporal $s=\{
(2,3), (3,4) \}$:
\begin{lstlisting}[style=tutorialD]
relation {
   tuple { t 2.0, v 3.0 },
   tuple { t 3.0, v 4.0 }
 }
\end{lstlisting}
\begin{lstlisting}[style=stdout]
 t   v 
--- ---
2.0 3.0
3.0 4.0
\end{lstlisting}


A partir d'aquest valors relació s'hauria de definir el tipus sèrie
temporal. No obstant això, la definició de tipus es troba en un estat
massa experimental a Tutorial~D, com es detalla a
l'apartat~\ref{sec:implementacio:tipus-relacional}. Així doncs,
definim una variable relació \emph{timeseries} per a utilitzar-la com a
equivalent del tipus en les definicions dels operadors:
\begin{lstlisting}[style=tutorialD]
var timeseries base relation
    { t rational, v rational }  key { t } ;
\end{lstlisting}


En les sèries temporals com a relació, els tuples són les mesures del
model de \gls{SGST}. Hi ha alguns operadors dels \gls{SGST} que tenen
mesures com a paràmetres, 
% però els tuples sense formar part d'una
% relació no són un tipus vàlid a Tutorial D. 
així doncs caldria definir també un tipus per a les mesures. Per a
simplificar, definim aquestes mesures com a sèries temporals d'un sol
element, per exemple la mesura $m=(2,3)$:
\begin{lstlisting}[style=tutorialD]
relation {
   tuple { t 2.0, v 3.0 }
 }
\end{lstlisting}


A partir d'aquestes mesures com a valors relació definim els operadors
que tenen mesures com a paràmetres. Així, els dos operadors bàsics
dels \gls{SGST} per a obtenir els atributs de temps i de valor d'una mesura
són:
\begin{lstlisting}[style=tutorialD]
operator ts.t(m same_type_as (timeseries)) returns rational;
  return t from tuple from m;
end operator;

operator ts.v(m same_type_as  (timeseries)) returns rational;
  return v from tuple from m;
end operator;
\end{lstlisting}


Per exemple, per a obtenir l'atribut temps $T(m)$ de la mesura:
\begin{lstlisting}[style=tutorialD]
with relation {
   tuple { t 2.0, v 3.0 }
  } as m1: 
ts.t(m1)
\end{lstlisting}
\begin{lstlisting}[style=stdout]
2.0
\end{lstlisting}
i per a obtenir el valor $V(m)$:
\begin{lstlisting}[style=tutorialD]
with relation {
   tuple { t 2.0, v 3.0 }
  } as m1: 
ts.v(m1)
\end{lstlisting}
\begin{lstlisting}[style=stdout]
3.0
\end{lstlisting}



\subsection{Operadors}

Un cop implementat el model estructural és el torn d'implementar el
model d'operacions.  Algunes operacions dels \gls{SGST} es poden
implementar directament amb les operacions relacionals de Tutorial~D
aplicades a les relacions \emph{timeseries}, és el cas de:

\begin{itemize}
\item La projecció
\begin{lstlisting}[style=tutorialD]
with
 RELATION {
   TUPLE { t 2.0, v 3.0 },
   TUPLE { t 3.0, v 4.0 }
 } AS s:
s {t}
\end{lstlisting}
\begin{lstlisting}[style=stdout]
 t
---
2.0
3.0
\end{lstlisting}

\item La selecció
\begin{lstlisting}[style=tutorialD]
with
 RELATION {
   TUPLE { t 2.0, v 3.0 },
   TUPLE { t 3.0, v 4.0 }
 } AS s:
s where v>3.0
\end{lstlisting}
\begin{lstlisting}[style=stdout]
 t   v 
--- ---
3.0 4.0
\end{lstlisting}


\item El reanomena
\begin{lstlisting}[style=tutorialD]
with
 RELATION {
   TUPLE { t 2.0, v 3.0 },
   TUPLE { t 3.0, v 4.0 }
 } AS s:
s rename ( v as temperatura )
\end{lstlisting}
\begin{lstlisting}[style=stdout]
 t  temperatura 
--- -----------
3.0         4.0
\end{lstlisting}


\item El mapa 
  % \glsdesc{not:sgst:map} és equivalent a un \emph{extend} més una
  % projecció més un reanomena.
 % EXTEND s ADD ( <texpr> AS tp, <vexpr> as vp ) {tp,vp} RENAME (tp AS t, vp AS v)
\begin{lstlisting}[style=tutorialD]
with
 RELATION {
   TUPLE { t 2.0, v 3.0 },
   TUPLE { t 3.0, v 4.0 }
 } AS s:
extend s add ( t+v as vp) 
\end{lstlisting}
\begin{lstlisting}[style=stdout]
 t   v  vp
--- --- ---
2.0 5.0 5.0
3.0 7.0 7.0
\end{lstlisting}


\item L'agregació, per algunes funcions
\begin{lstlisting}[style=tutorialD]
with
 RELATION {
   TUPLE { t 2.0, v 3.0 },
   TUPLE { t 3.0, v 4.0 }
 } AS s:
summarize s add (max(t) as t, sum(v) as v )
\end{lstlisting}
\begin{lstlisting}[style=stdout]
 t   v  
--- ---
3.0 1.0
\end{lstlisting}


\end{itemize}





La resta d'operacions s'han de definir, les implementem basades en
altres operadors de l'àlgebra relacional de Tutorial~D..  A
continuació en mostrem algunes d'exemple, tots els operadors que
definim els anomenem prefixats amb `ts.'.



L'operació d'unió és equivalent a unir la primera sèrie temporal amb
les mesures de la segona que no tenen un atribut de temps igual a
algun temps de la primera. Així, l'operador d'unió \emph{ts.union}
 és:
\begin{lstlisting}[style=tutorialD]
OPERATOR ts.union(
      s1 SAME_TYPE_AS  (timeseries), 
      s2 SAME_TYPE_AS  (timeseries)) 
      RETURNS RELATION SAME_HEADING_AS  (timeseries);
  return s1 union (s2 join (s2 {t} minus s1 {t}));
END OPERATOR;
\end{lstlisting}

Per exemple, siguin dues sèries temporals $s_1=\{(2,3),(4,2),(6,4)\}$ i $s_2=\{(1,2),(5,3),(6,5),(10,1)\}$ la unió $s_1\cup s_2$ és:
\begin{lstlisting}[style=tutorialD]
with 
 RELATION {
   TUPLE { t 2.0, v 3.0 },
   TUPLE { t 4.0, v 2.0 },
   TUPLE { t 6.0, v 4.0 }
  } AS s1,
 RELATION {
   TUPLE { t 1.0, v 2.0 },
   TUPLE { t 5.0, v 3.0 },
   TUPLE { t 6.0, v 5.0 },
   TUPLE { t 9.0, v 1.0 }
  } AS s2: 
ts.union(s1,s2)
\end{lstlisting}
\begin{lstlisting}[style=stdout]
 t   v 
--- ---
1.0 2.0
2.0 3.0
4.0 2.0
5.0 3.0
6.0 4.0
9.0 1.0
\end{lstlisting}



Les operacions de funció temporal operen en base a un mètode de
representació de les sèries temporals. Els operadors de funció
temporal definits en el model de \gls{SGST} depenen de l'operador
d'interval temporal, per tant cal definir-ne un per a cada
representació i els altres operadors utilitzaran el que correspongui a
la representació sol·licitada.



Així doncs, com a exemple definim l'operador d'interval temporal
\gls{zohe} \emph{ts.interval.zohe}:
\begin{lstlisting}[style=tutorialD]
OPERATOR ts.interval.zohe(
         s SAME_TYPE_AS  (timeseries),
         l rational, 
         h rational ) 
         RETURNS RELATION SAME_HEADING_AS  (timeseries);
  BEGIN;
    VAR x rational init(0.0);
    VAR sp private SAME_TYPE_AS (timeseries) KEY { t };
    x := ts.v(ts.inf(ts.interval.closed(s,h,1.0/0.0)));
    sp := RELATION {
      TUPLE {t h, v x}
      };
    return ts.union(ts.interval.left(s,l,h),sp);
  END;
END OPERATOR;
\end{lstlisting}


Per exemple, l'interval temporal \gls{zohe} $s_1[1,3]^{\glssymbol{not:zohe}}$ és:
\begin{lstlisting}[style=tutorialD]
with 
 RELATION {
   TUPLE { t 2.0, v 3.0 },
   TUPLE { t 4.0, v 2.0 },
   TUPLE { t 6.0, v 4.0 }
  } AS s1: 
ts.interval.zohe(s1,1.0,3.0)
\end{lstlisting}
\begin{lstlisting}[style=stdout]
 t   v 
--- ---
2.0 3.0
3.0 2.0
\end{lstlisting}



% Dave Voorhis,  [TTM] Questions about Tutorial D and/or Rel, 17/10/14 15:30:50
% Rel provides EXECUTE for compiling and executing strings of arbitrary code at run-time, but such mechanisms should never be used by anyone, ever, for anything.  Unless you have to, of course.


Els altres operadors de funció temporal tenen un paràmetre \emph{repr}
a partir del qual s'executa l'operador d'interval temporal
corresponent, el qual ha d'estar definit prèviament a l'execució.  Ni
Tutorial~D ni Rel ofereixen una manera còmode d'executar un operador
escollit segons un paràmetre tot i que
\textcite{voorhis12:rel_anonymous_op} està desenvolupant millores en
aquest sentit. Una solució actualment és l'execució d'una cadena de
text mitjançant la sentència \emph{execute}. A continuació ho mostrem
en la implementació de l'operació de concatenació temporal
\emph{ts.concatenation.t}:
\begin{lstlisting}[style=tutorialD]
OPERATOR ts.concatenation.t(
         s1 SAME_TYPE_AS  (timeseries),
         s2 SAME_TYPE_AS  (timeseries),
         repr char)
         RETURNS RELATION SAME_HEADING_AS  (timeseries);
  BEGIN;
    VAR t1 rational init(0.0);
    VAR t2 rational init(0.0);
    VAR exp char init('');

    t1 := ts.t(ts.inf(s1));
    t2 := ts.t(ts.sup(s1));

    //Cal executar: ts.globalexecute := ts.interval.<repr>(s2,t1,t2); 
    ts.globalexecute := s2;
    exp := 'ts.globalexecute := ts.interval.' || repr || '(ts.globalexecute,' || t1 || ',' || t2 || ');';
    execute exp;
  
    return ts.union(s1,s2 minus ts.globalexecute);
  END;
END OPERATOR;
\end{lstlisting}


Per exemple, la concatenació \gls{zohe} de les sèries
temporals $s1 ||^{\glssymbol{not:zohe}} s2$ és:
\begin{lstlisting}[style=tutorialD]
with 
 RELATION {
   TUPLE { t 2.0, v 3.0 },
   TUPLE { t 4.0, v 2.0 },
   TUPLE { t 6.0, v 4.0 }
  } AS s1,
 RELATION {
   TUPLE { t 1.0, v 2.0 },
   TUPLE { t 5.0, v 3.0 },
   TUPLE { t 6.0, v 5.0 },
   TUPLE { t 9.0, v 1.0 }
  } AS s2: 
ts.concatenation.t(s1,s2,'zohe')
\end{lstlisting}
\begin{lstlisting}[style=stdout]
 t   v 
--- ---
1.0 2.0
2.0 3.0
4.0 2.0
6.0 4.0
9.0 1.0
\end{lstlisting}





\subsection{Quant a definir el tipus sèrie temporal}
\label{sec:implementacio:tipus-relacional}

% As defined, Tutorial D easily lets you add specific relational operators but not generic ones.  It could be easily extended to support generic relational operators, but that's really beyond the scope of Tutorial D. [DaveVoorhis TTM distribution list 24/09/14 09:01:12 "Is an INTEGER an integer"]



Tutorial~D permet definir nous tipus de dades, que poden ser usats amb
la mateixa funcionalitat que els tipus predefinits --\emph{char},
\emph{rational}, etc.  A l'hora de definir nous tipus, Tutorial~D usa
el concepte de possibles representacions (\emph{possrep}) del model
relacional. Les possibles representacions són les diverses maneres
d'instanciar el valor d'un tipus.

Així doncs, en comptes de definir una variable \emph{timeseries} com
hem fet en el Reltsms seria millor definir un tipus \emph{timeseries}
que tingui una possible representació \emph{ts} que és la relació de
sèrie temporal:
\begin{lstlisting}[style=tutorialD]
TYPE timeseries
  POSSREP {ts relation {t rational, v rational}};
\end{lstlisting}


Aleshores es poden instanciar sèries temporals:
\begin{lstlisting}[style=tutorialD]
timeseries(
 RELATION {
   TUPLE { t 2.0, v 3.0 },
   TUPLE { t 3.0, v 4.0 }
 }
)
\end{lstlisting}

de les quals es pot demanar la relació continguda en la representació \emph{ts}:
\begin{lstlisting}[style=tutorialD]
with
 timeseries(
  RELATION {
    TUPLE { t 2.0, v 3.0 },
    TUPLE { t 3.0, v 4.0 }
  }
 ) AS s:
THE_ts(s)
\end{lstlisting}
\begin{lstlisting}[style=stdout]
 t   v 
--- ---
2.0 3.0
3.0 4.0
\end{lstlisting}


Però actualment, la definició de les sèries temporals com a tipus a
Tutorial~D és incòmode principalment per dos motius:
\begin{itemize}
\item Emmascara el concepte de tipus relació per als propòsits
  acadèmics que tenim. Principalment, Tutorial~D no permet definir
  objectes que siguin del mateix ordre que les relacions.


\item La definició estàtica de tipus en les operacions i la rigidesa
  en les capçaleres de les relacions. Aquest, de fet, és també un
  inconvenient que ens trobem en la definició dels operadors de
  Reltsms. Impedeix definir sèries temporals multivaluades i sèries
  amb atributs de qualssevol tipus.
\end{itemize}



\subsubsection{La sèrie temporal a partir del tipus relació}



Els valors relació en l'àlgebra relacional sempre són d'un tipus i
aquest tipus es correspon amb la seva capçalera. Però les relacions de
Tutorial~D són uns objectes de més ordre que la resta de tipus, tant
els predefinits com els d'usuari, ja que tenen un tractament molt
diferenciat i no es poden declarar nous tipus amb la mateixa
funcionalitat que les relacions.  Per exemple no es pot heretar del
tipus relació i per tant no es poden definir tipus que siguin admesos
com a arguments en els operadors predefinits de l'àlgebra relacional.


A Reltsms ens ha sigut còmode definir les sèries temporals com a tipus
relació perquè conserva molta correspondència amb el model algebraic.
Així en els operadors hem definit els paràmetres de sèrie temporal com
de tipus relació d'atributs $t$ i $v$ reals, és a dir en Tutorial~D %
\lstinline[style=tutorialD]+relation {t rational, v rational}+. %
De fet hem utilitzat la variable \emph{timeseries} per a agrupar
aquesta capçalera i així en els operadors hem definit el tipus a
partir de %
\lstinline[style=tutorialD]+s SAME_TYPE_AS(timeseries)+.%


Si haguéssim definit el tipus \emph{timeseries} aleshores hauríem
definit els operadors de la forma següent, per exemple pel cas de la
unió a partir de l'operador definit de \emph{ts.union}:
\begin{lstlisting}[style=tutorialD]
OPERATOR tstype.union(
      s1 timeseries, 
      s2 timeseries, 
      RETURNS timeseries;
  return timeseries(ts.union(THE_ts(s1),THE_ts(s2));
END OPERATOR;
\end{lstlisting}

És a dir que \emph{s1} i \emph{s2} són de tipus \emph{timeseries} i
per obtenir-ne la relació continguda usem l'operador \emph{THE\_ts}. El
resultat també és de tipus \emph{timeseries}, el qual s'instancia a
partir d'un valor relació.

Aleshores, el mateix exemple que per la unió ara s'executaria amb:
\begin{lstlisting}[style=tutorialD]
tstype.union(timeseries(s1),timeseries(s2))
\end{lstlisting}

A més, en els usos d'operadors de l'àlgebra relacional també s'hauria
d'executar usant l'operador \emph{THE\_ts} i tornant a instanciar com
a sèrie temporal. Per exemple:
\begin{lstlisting}[style=tutorialD]
with
timeseries(
 RELATION {
   TUPLE { t 2.0, v 3.0 },
   TUPLE { t 3.0, v 4.0 }
  }) AS s:
timeseries(THE_ts(s) where v>3.0)
\end{lstlisting}

En conclusió, definir la sèrie temporal com a tipus \emph{timeseries}
en comptes de directament com a tipus relació dificulta poder observar
la relació directa entre l'àlgebra relacional del model i la
implementació en un \gls{SGBDR}. Si bé és cert que ambdues es poden
usar amb la mateixa funcionalitat, en el cas dels tipus les crides als
operadors queden emmascarades amb l'ús de l'operador d'atribut
\emph{THE\_ts} i la instància de nou de la \emph{timeseries}.  A
Tutorial~D faria falta un sistema d'herència més clar que pogués
declarar que el tipus \emph{timeseries} té l'atribut \emph{ts} que és
de tipus relació i pot actuar com a tal allà on calgui. Així, per bé
que l'emmascarament del tipus podria ser útil en un sistema productiu,
dificulta el propòsit acadèmic d'aquesta implementació.


\subsubsection{Sèries temporals multivaluades i de qualssevol tipus}



Els operadors de Reltsms definits treballen amb la sèrie temporal en
forma canònica ja que així és com hem definit la variable
\emph{timeseries}, com a relació de dos atributs $t$ i $v$ de tipus
real per a simplificar.

Per una banda, es poden definir sèries temporals multivaluades, que
són relacions amb l'atribut $t$ i altres atributs per als valors
multivaluats. Per exemple la sèrie temporal multivaluada $m = \{
(2,1,2),(3,1,2) \}$:
\begin{lstlisting}[style=tutorialD]
 RELATION {
   TUPLE { t 2.0, v1 1.0, v2 2.0},
   TUPLE { t 3.0, v1 1.0, v2 2.0}
  }
\end{lstlisting}
\begin{lstlisting}[style=stdout]
 t   v1  v2
--- --- ---
2.0 1.0 2.0
3.0 1.0 2.0
\end{lstlisting}

També es poden definir sèries temporals multivaluades en la forma
canònica, que son relacions amb l'atribut $t$ i $v$ on el $v$ és una
relació dels atributs multivaluats. Per exemple la sèrie temporal
multivaluada $m$ en la forma canònica és $m_c = \{ (2,(1,2)),(3,(1,2))
\}$:
\begin{lstlisting}[style=tutorialD]
 RELATION {
   TUPLE { t 2.0, v relation { tuple {v1 1.0, v2 2.0 }} },
   TUPLE { t 3.0, v relation { tuple {v1 1.0, v2 2.0 }} }
  }
\end{lstlisting}
\begin{lstlisting}[style=stdout]
        v
 t   ------
     v1  v2
--- --- ---
2.0 1.0 2.0
3.0 1.0 2.0
\end{lstlisting}


Els operadors d'agrupament de l'àlgebra relacional permeten
transformar de la forma multivaluada a la forma canònica i a la
inversa. Així, amb Tutorial~D l'operació de transformació de la forma
multivaluada a la canònica és:
\begin{lstlisting}[style=tutorialD]
s group ({all but t} as v)
\end{lstlisting}
i la transformació de la forma canònica a la multivaluada és:
\begin{lstlisting}[style=tutorialD]
s ungroup (v)
\end{lstlisting}

% //operators using Tutorial D:
% //multivalued2canonical ->   s group ({all but t} as v)
% //canonical2multivalued ->   s ungroup (v)



% També és possible definir la multivaluada canònica amb tuples en comptes de relacions i treballar amb l'operador WRAP:

% WITH 
%  RELATION {
%    TUPLE { t 4.0, v1 1.0, v2 2.0}
%   } AS s:
% s WRAP ({v1,v2} AS v)

% WITH 
%  RELATION {
%    TUPLE { t 4.0, v tuple {v1 1.0, v2 2.0}}
%   } AS s:
% s unwrap (v)




Per altra banda, es poden definir sèries temporals on els atributs no
siguin de tipus real sinó per exemple enters:
\begin{lstlisting}[style=tutorialD]
 RELATION {
   TUPLE { t 2, v 1},
   TUPLE { t 3, v 1}
  }
\end{lstlisting}
\begin{lstlisting}[style=stdout]
 t   v 
--- ---
 2   1
 3   1
\end{lstlisting}





Això no obstant, a Tutorial~D aquestes relacions de sèries temporals
multivaluades o d'altres tipus que no siguin reals no són acceptades
com a arguments en els operadors definits de Reltsms.  Les sèries
temporals multivaluades són relacions amb la capçalera $\{t, v1, v2,
\ldots \}$ o bé $\{t, v: \{v1, v2,\} \}$ en la forma multivaluada
canònica, les quals a Tutorial~D no s'adiuen amb la capçalera
$\{t:\glssymbol{not:R}, v:\glssymbol{not:R}\}$ de les
\emph{timeseries}. El mateix ocorre amb relacions amb capçalera on
els tipus no són reals, per exemple on són enters
$\{t:\glssymbol{not:Z}, v:\glssymbol{not:Z}\}$.

Això és degut al fet que a Tutorial~D hi ha definició estàtica dels
tipus i en els operadors de Reltsms els paràmetres són del mateix
tipus que \emph{timeseries}, és a dir exactament de tipus
\lstinline[style=tutorialD]+relation {t rational, v rational}+.  El
mateix problema també ocorre en definir el tipus \emph{timeseries} amb
una possible representació d'aquest tipus, aleshores no es poden
instanciar sèries temporals on els atributs no siguin de tipus real.


Per tant, necessitaríem definir les sèries temporals com a relacions
de tipus on un dels atributs és $t$ el temps i els altres atributs són
qualssevol.  \textcite{darwen13:generic_relation_type} proposa una
ampliació de Tutorial~D en aquest sentit que permetria definir el
tipus relació amb asteriscs per a indicar qualsevol nom d'atribut o
qualsevol tipus, per exemple el més genèric seria
\lstinline[style=tutorialD]+relation {*}+ que indicaria una relació de
qualsevol tipus.  Cal tenir en compte, però, que per ara només és una proposta.


Així, més específicament, ens permetria definir les sèries temporals
en forma canònica com de tipus %
\lstinline[style=tutorialD]+relation {t *, v *}+ %
i la forma multivaluada com de tipus %
\lstinline[style=tutorialD]+relation {t * ,*}+. %
Aleshores podríem definir correctament el tipus sèrie temporal amb les
dues possibles representacions de canònica i multivaluada:
\begin{lstlisting}[style=tutorialD]
type timeseries
	POSSREP ts.canonical {ts relation {t *, v * }}}
	POSSREP ts.multivalued {multivalued relation {t * ,*}} 
	INIT  ts.canonical (multivalued:= ts ungroup (v))
	      ts.multivalued (ts:= multivalued group ({all but t} as v));
\end{lstlisting}



Aleshores podríem instanciar sèries temporals amb altres tipus en els
atributs, com per exemple enters:
\begin{lstlisting}[style=tutorialD]
ts.canonical(
 RELATION {
   TUPLE { t 2, v 1},
   TUPLE { t 3, v 1}
  }
)
\end{lstlisting}


o bé instanciar sèries temporals multivaluades, per exemple:
\begin{lstlisting}[style=tutorialD]
ts.multivalued(
 RELATION {
   TUPLE { t 2.0, v1 1.0, v2 2.0},
   TUPLE { t 3.0, v1 1.0, v2 2.0}
  }
)
\end{lstlisting}

de les quals se'n podria obtenir tant la possible representació en
forma multivaluada
\begin{lstlisting}[style=tutorialD]
THE_multivalued(m)
\end{lstlisting}
com en la forma canònica
\begin{lstlisting}[style=tutorialD]
THE_ts(m)
\end{lstlisting}




En conclusió, definir el tipus sèrie temporal seria de molta utilitat
si les capçaleres de les relacions poguessin ser genèriques, sobretot
el concepte de possibles representacions és molt útil per a
implementar la forma canònica i la multivaluada.






% \todo{per entendre les possrep potser s'hauria d'explicar això:}

% type prova2 
%   POSSREP pa {a rational}
%   POSSREP pb {b rational}
%   INIT pa (b:=a)
%          pb (a:=b);

% pa(1.0)
% pb(1.0)



%\subsection{Sèries temporals dobles}
% \begin{lstlisting}[style=tutorialD]
% VAR timeseriesdouble BASE RELATION
%     { t1 RATIONAL, v1 RATIONAL, t2 RATIONAL, v2 RATIONAL }  KEY { t1, t2 } ;
% \end{lstlisting}




\section{Multiresolució relacional}


Implementem la funció de \glssymbol{not:sgstm:multiresolucio} 
 de \textref{def:multiresolucio:plecmu}
sobre els \gls{SGST} amb l'operador \emph{mtsms.multiresolution}. Té
dos paràmetres: la sèrie temporal \emph{s} que és una
\emph{timeseries} de Reltsms i l'esquema multiresolució \emph{schema}
que és una relació, és a dir un conjunt, d'atributs \emph{delta},
\emph{tau}, \emph{f} i \emph{k}:

\begin{lstlisting}[style=tutorialD]
OPERATOR mtsms.multiresolution(
         s SAME_TYPE_AS  (timeseries),   
         schema relation {delta rational, tau rational, f char, k rational}) 
         RETURNS RELATION SAME_HEADING_AS  (timeseries);
     [...]
\end{lstlisting}
       



Els atributs de l'esquema multiresolució són de tipus real, seguint
amb la simplificació de Reltsms, llevat de \emph{f} que és un nom
d'un operador que actua com a funció d'agregació d'atributs i per tant
permet que l'usuari se'ls defineixi. Per a calcular la subsèrie
temporal de cada disc a partir de cada tuple de paràmetres
multiresolució, l'operador de multiresolució depèn d'un operador
\emph{mtsms.dmap}:
\begin{lstlisting}[style=tutorialD]
OPERATOR mtsms.dmap(
         s SAME_TYPE_AS  (timeseries),   
         delta rational, 
         tau rational,         
         f char,         
         k rational) 
         RETURNS RELATION SAME_HEADING_AS  (timeseries);
     [...]
\end{lstlisting}


L'operador \emph{mtsms.dmap} és el que executarà l'operador
d'agregació d'atributs corresponent al nom mitjançant la sentència
\emph{execute}, de manera similar a com s'ha descrit en el Reltsms.
Els operadors d'agregació d'atributs tenen tres paràmetres: la sèrie
temporal \emph{s}, l'instant de temps menor $l$ i l'instant de temps
major $h$. %
Noteu que en el model hem
definit una funció d'agregació d'atributs sobre una sèrie temporal, un
instant de consolidació $\tau$ i un pas de consolidació $\delta$,
mentre que les implementem sobre una sèrie temporal i directament
sobre l'interval de consolidació $[l=\tau,h=\tau+\delta]$.


Així doncs, prèviament a l'execució de la multiresolució cal tenir
definit un operador d'agregació d'atributs. Com a exemple, definim
l'operador \emph{mtsms.aaf.maxzohe} per a la funció d'agregació
\glssymbol{not:sgstm:maxzohe}:
\begin{lstlisting}[style=tutorialD]
OPERATOR mtsms.aaf.maxzohe(
         s SAME_TYPE_AS  (timeseries), 
         l rational, 
         h rational)        
         RETURNS RELATION SAME_HEADING_AS  (timeseries);
  return summarize ts.interval.zohe(s,l,h) add (max(t) as t, max(v) as v);
END OPERATOR;
\end{lstlisting}


Per exemple, sigui la sèrie temporal $s=\{(4,1),(5,6),(8,2)\}$ i la
sèrie temporal multiresolució amb esquema $ \glssymbol{not:esquemaM} = \{ \{
\tau : 0, \delta : 2, \glssymbol{not:sgstm:k}: 4, f : \glssymbol{not:sgstm:maxzohe} \} , \{
\tau : 0, \delta : 5, \glssymbol{not:sgstm:k}: 2, f :\glssymbol{not:sgstm:maxzohe} \} \}$,
la $\glssymbol{not:sgstm:multiresolucio}(s, \glssymbol{not:esquemaM})$ és


\begin{lstlisting}[style=tutorialD]
WITH 
 RELATION {
   TUPLE { t 4.0, v 1.0 },
   TUPLE { t 5.0, v 6.0 },
   TUPLE { t 8.0, v 2.0 }
  } AS s,
 RELATION {
   TUPLE { delta 2.0, tau 0.0, f 'mtsms.aaf.maxzohe', k 4.0},
   TUPLE { delta 5.0, tau 0.0, f 'mtsms.aaf.maxzohe', k 2.0}
  } AS schema:
mtsms.multiresolution(s,schema)
\end{lstlisting}
\begin{lstlisting}[style=stdout]
 t   v 
--- ---
0.0 1.0
2.0 1.0
4.0 1.0
6.0 6.0
8.0 2.0
\end{lstlisting}



\section{Resum}

En resum, mitjançant el llenguatge acadèmic dels \gls{SGBDR},
Tutorial~D, hem implementat un \gls{SGST} a partir del qual hem pogut
implementar les operacions de multiresolució.  En aquest capítol hem
presentat les operacions més destacables de les implementacions per
tal d'experimentar com els models dissenyats s'adapten al model
relacional. Si bé no hem implementat un \gls{SGSTM}, es podria seguir
un raonament similar al que hem fet per als \gls{SGST}.











% \subsection{Pertinença i inclusió}


% with RELATION {
% TUPLE { t 2.0, v 3.0 }
%  } AS m,
% RELATION {
% TUPLE { t 2.0, v 3.0 },
% TUPLE { t 5.0, v 3.0 },
% TUPLE { t 6.0, v 5.0 },
% TUPLE { t 10.0, v 1.0 }
%  } AS ts: 
% ts.in(m,ts)


% \subsection{Màxim i suprem}

% Tutorial D:
% \begin{lstlisting}[style=tutorialD]
% OPERATOR ts.max(s1 SAME_TYPE_AS  (timeseries)) RETURNS RELATION SAME_HEADING_AS  (timeseries);
% return s1 JOIN ( SUMMARIZE s1 {t} PER (s1 {}) ADD (MAX (t) AS t));
% END OPERATOR;

% OPERATOR ts.sup(s1 SAME_TYPE_AS  (timeseries)) RETURNS RELATION SAME_HEADING_AS  (timeseries);
% return ts.max(ts.union(s1,(RELATION { TUPLE {t -1.0/0.0, v 1.0/0.0} })));
% END OPERATOR;
% \end{lstlisting}



% Exemple:
% \begin{lstlisting}[style=tutorialD]
% with RELATION {
% TUPLE { t 2.0, v 3.0 },
% TUPLE { t 4.0, v 2.0 },
% TUPLE { t 6.0, v 4.0 }
%  } AS ts1: 
% ts.max(ts1)
% \end{lstlisting}
% \begin{lstlisting}[style=tutorialD]
% RELATION {
% TUPLE { t 1.0, v 2.0 },
% TUPLE { t 5.0, v 3.0 },
% TUPLE { t 6.0, v 5.0 },
% TUPLE { t 1.0/0.0, v 1.0 }  //1.0/0.0 infinit
%  } AS ts2: 
% ts.max(ts2)
% \end{lstlisting}
% \begin{lstlisting}[style=tutorialD]
% with RELATION {
% TUPLE { t 2.0, v 3.0 },
% TUPLE { t 4.0, v 2.0 },
% TUPLE { t 6.0, v 4.0 }
%  } AS ts1: 
% ts.sup(ts1)
% \end{lstlisting}
% \begin{lstlisting}[style=tutorialD]
% with RELATION {
% TUPLE { t 2.0, v 3.0 },
% TUPLE { t 4.0, v 2.0 },
% TUPLE { t 6.0, v 4.0 }
%  } AS ts1: 
% ts.sup(timeseries)
% \end{lstlisting}




% \subsection{Producte}

% with RELATION {
% TUPLE { t 1.0, v 4.0 }
%  } AS s1,
% RELATION {
% TUPLE { t 2.0, v 3.0 },
% TUPLE { t 5.0, v 3.0 },
% TUPLE { t 6.0, v 5.0 },
% TUPLE { t 10.0, v 1.0 }
%  } AS s2: 
% ts.product(s1,s2)


% \subsection{Unió exclusiva}


% TutorialD:
% \begin{lstlisting}[style=tutorialD]
% OPERATOR ts.xunion(s1 SAME_TYPE_AS  (timeseries), s2 SAME_TYPE_AS  (timeseries)) RETURNS RELATION SAME_HEADING_AS  (timeseries);
% return ts.union(s1,s2) MINUS ts.intersect(s1,s2) ;
% END OPERATOR;
% \end{lstlisting}


% Exemple:
% \begin{lstlisting}[style=tutorialD]
% with RELATION {
% TUPLE { t 2.0, v 3.0 },
% TUPLE { t 4.0, v 2.0 },
% TUPLE { t 6.0, v 4.0 }
%  } AS ts1,
% RELATION {
% TUPLE { t 1.0, v 2.0 },
% TUPLE { t 5.0, v 3.0 },
% TUPLE { t 6.0, v 5.0 },
% TUPLE { t 10.0, v 1.0 }
%  } AS ts2: 
% ts.xunion(ts1,ts2)
% \end{lstlisting}


% \subsection{Selecció temporal}



% TutorialD:
% \begin{lstlisting}[style=tutorialD]
% OPERATOR ts.temporal.select.zohe(s SAME_TYPE_AS  (timeseries), l RATIONAL, h RATIONAL ) RETURNS RELATION SAME_HEADING_AS  (timeseries);
% BEGIN;
% VAR x RATIONAL init(0.0);
% VAR sp PRIVATE SAME_TYPE_AS ( timeseries) KEY { t };
% x := ts.v(ts.inf(s MINUS ts.interval.ni(s,h)));
% sp := RELATION {
% TUPLE {t h, v x}
% };
% return ts.union(ts.interval(s,l,h),sp);
% END;
% END OPERATOR;
% \end{lstlisting}

% Exemple:
% \begin{lstlisting}[style=tutorialD]
% with RELATION {
% TUPLE { t 2.0, v 3.0 },
% TUPLE { t 4.0, v 2.0 },
% TUPLE { t 6.0, v 4.0 }
%  } AS ts1,
% RELATION {
% TUPLE { t 1.0, v 2.0 },
% TUPLE { t 5.0, v 3.0 },
% TUPLE { t 6.0, v 5.0 },
% TUPLE { t 10.0, v 1.0 }
%  } AS ts2:
% ts.temporal.select.zohe(ts1,1.0,5.0)
% \end{lstlisting}
% \begin{lstlisting}[style=tutorialD]
% with RELATION {
% TUPLE { t 2.0, v 3.0 },
% TUPLE { t 4.0, v 2.0 },
% TUPLE { t 6.0, v 4.0 }
%  } AS ts1,
% RELATION {
% TUPLE { t 1.0, v 2.0 },
% TUPLE { t 5.0, v 3.0 },
% TUPLE { t 6.0, v 5.0 },
% TUPLE { t 10.0, v 1.0 }
%  } AS ts2:
% ts.temporal.select.zohe(ts1,-1.0/0.0,-1.0/0.0)
% \end{lstlisting}



% \subsection{Map i fold}



% TutorialD:
% \begin{lstlisting}[style=tutorialD]
% with RELATION {
% TUPLE { t 2.0, v 3.0 },
% TUPLE { t 4.0, v 2.0 },
% TUPLE { t 6.0, v 4.0 }
% } AS ts1: 
% ts.map(ts1,'t','t*v/2.0')
% \end{lstlisting}

% TutorialD:
% \begin{lstlisting}[style=tutorialD]
% with RELATION {
% TUPLE { t 2.0, v 3.0 },
% TUPLE { t 4.0, v 2.0 },
% TUPLE { t 6.0, v 4.0 }
%  } AS ts1,
% RELATION {
% TUPLE { t 0.0, v 0.0}
% } AS mi: 
% ts.fold(ts1,mi,'t','v+vi')
% \end{lstlisting}

% antM TutorialD:
% \begin{lstlisting}[style=tutorialD]
% with RELATION {
% TUPLE { t 2.0, v 3.0 },
% TUPLE { t 4.0, v 2.0 },
% TUPLE { t 6.0, v 4.0 }
%  } AS ts1,
% RELATION {
% TUPLE { t 5.0, v 0.0}
% } AS m: 
% ts.sup(ts1 WHERE t < ts.t(m))
% \end{lstlisting}

% sup fold TutorialD:
% \begin{lstlisting}[style=tutorialD]
% with RELATION {
% TUPLE { t 2.0, v 3.0 },
% TUPLE { t 4.0, v 2.0 },
% TUPLE { t 6.0, v 4.0 }
%  } AS ts1,
% RELATION {
% TUPLE { t -1.0/0.0, v 1.0/0.0}
% } AS mi: 
% ts.fold(ts1,mi,'max {t,ti}','v FROM TUPLE FROM (RELATION { TUPLE {t t, v v, e True}, TUPLE {t ti, v vi,  e False} } JOIN RELATION { TUPLE {e t > ti}})')
% \end{lstlisting}

% predecessors mapfold TutorialD:
% \begin{lstlisting}[style=tutorialD]
% with RELATION {
% TUPLE { t 2.0, v 3.0 },
% TUPLE { t 4.0, v 2.0 },
% TUPLE { t 6.0, v 4.0 }
%  } AS ts1,
% EXTEND ts1 {t} ADD ( -1.0/0.0 as v)
% AS si: 
% ts.fold(ts1,si,'t','v FROM TUPLE FROM (RELATION { TUPLE {v ti, e True}, TUPLE {v v,  e False} } JOIN RELATION { TUPLE {e v < ti and ti<t}})')
% \end{lstlisting}


% \section{SGSTM relacional}














%%% Local Variables:
%%% TeX-master: "main"
%%% End:

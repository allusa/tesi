\section{Model d'operacions}

En aquesta secció es defineixen els operadors que permeten modelar el
comportament i la manipulació de les dades en el model de \gls{SGSTM}.

Per a treballar amb les sèries temporals multiresolució s'utilitzen
els conceptes descrits al model d'operacions de \gls{SGST}. El model
de \gls{SGSTM} es defineix a partir del model de \gls{SGST} i per tant
les operacions dels \gls{SGSTM} també hi estan basades. Tot i així cal
tenir en compte dues particularitats.

Per una banda, el model de \gls{SGSTM} té una estructura específica que
requereix ser manipulada coherentment. Així, es defineixen operadors
que saben treballar amb aquesta estructura agrupats en dos grups.  El
primer grup són els operadors requerits pel model estructural;
operadors que són inseparables de l'estructura i són utilitzats en el
procés d'emmagatzemar les mesures. El segon grup són els operadors
necessaris per a manipular l'estructura; és a dir operadors que
permeten fer canvis en l'esquema de la base de dades o consultar
paràmetres de l'esquema actual.

Per altra banda, el model de \gls{SGSTM} treballa amb sèries temporals
multiresolució. Així, es defineixen operadors que permeten extreure
les sèries temporals emmagatzemades en aquestes bases de dades amb
l'objectiu d'aplicar-hi posteriorment els operadors dels \gls{SGST}.


En el disseny del model d'operacions següent es distingeixen tres
grups d'operadors segons els casos anteriors:

\begin{itemize}
\item Estructurals: operadors requerits pel model estructural.
\item Manipulació de l'esquema: operadors per a manipular l'esquema de
  multiresolució.
\item Consultes: operadors per a extreure les sèries temporals
  emmagatzemades.
\end{itemize}





\subsection{Estructurals}
\label{sec:model:sgstm-estructurals}

En el model estructural de \gls{SGSTM} hem definit les sèries temporals
multiresolució com un conjunt de subsèries resolució a les quals es
van afegint mesures compactant-les i consolidant-les. En aquest
apartat definim els operadors que permeten inserir mesures noves i
consolidar-les al seu lloc corresponent en l'estructura.

A continuació es descriuen els operadors associats a cada objecte del
model de \gls{SGSTM}.


\subsubsection{Buffer}

Els buffers reben les noves mesures i les consoliden a cada instant de
consolidació. Així, tenen dos operadors associats: un per afegir noves
mesures al buffer i un altre per consolidar-les.


L'operació d'afegir una mesura al buffer consisteix en afegir-la a la
sèrie temporal pendent de consolidar.
\begin{definition}[Afegeix mesura al buffer]
  \label{def:sgstm:addB}
  Sigui $B=(S,\tau,\delta,f)$ un buffer i $m=(t,v)$ una mesura, la
  inserció de la mesura al buffer $\glssymboldef{addB}(B,m)$ és un
  buffer $B'=(S',\tau,\delta,f)$ amb la mesura afegida a la sèrie
  temporal del buffer: $\glssymbol{addB}(B,m) = (S',\tau,\delta,f)$ a
  on $S'=S\cup \{m\}$.
\end{definition}


L'operació de consolidació d'un buffer consisteix en compactar les
mesures segons els intervals de consolidació i la funció d'agregació i
a suprimir la part ja consolidada de la sèrie temporal.  Així doncs,
la consolidació d'un buffer per cada interval de temps
$i=[\tau,\tau+\delta]$ dóna com a resultat una mesura calculada en
funció de l'agregador d'atributs i un nou buffer amb la sèrie temporal
reduïda.
\begin{definition}[Consolida el buffer]\label{def:model:consolidacio-buffer}
  Sigui $B=(S,\tau,\delta,f)$ un buffer, la consolidació del buffer
  $\glssymboldef{consolidaB}(B)$ en l'interval de temps
  $[\tau,\tau+\delta]$ és un nou buffer amb el nou instant de
  consolidació i una mesura resultant de la consolidació:
  $\glssymbol{consolidaB}(B) = (B',m')$ on $B'=(S',\tau',\delta,f)$,
  $m'=f(S,[\tau,\tau+\delta])$, $\tau'=\tau+\delta$ i $S'$ és el
  resultat d'eliminar les dades històriques que no es necessiten més.

  \emph{Sobre eliminar les dades històriques}: En el model teòric es
  pot donar $S'=S$ tot i que a les implementacions normalment caldrà
  eliminar les dades ja no necessàries per no ocupar espai amb per
  exemple $S'= S[\tau+\delta,+\infty]$.
\end{definition}


De manera simplificada, hem definit que cada consolidació només
s'aplica a l'interval de consolidació actual; així la consolidació
total del buffer és l'aplicació successiva de l'operació de
consolidació. 

Aquesta consolidació successiva requereix que les mesures s'insereixen
al buffer ordenades en el temps, sinó un cop duta a terme la
consolidació les mesures inserides desordenades poden no ser tingudes
en compte. Si es duu a terme aquesta inserció ordenada, aleshores un
buffer té l'estat de consolidable quan el temps d'una mesura de la
sèrie temporal és més gran que el següent instant de temps de
consolidació del buffer.
\begin{definition}[Buffer consolidable]\label{def:model:buffer_consolidable}
  Sigui $B=(S,\tau,\delta,f)$ un buffer, definim que
  \glsdispdef{consolidableB}{$B$ és
    consolidable} si i només si $T(m) \geq \tau+\delta$ a on
  $m=\sup(S)$ és la mesura suprema de la sèrie temporal del buffer.
\end{definition}



\subsubsection{Disc}

Els discs reben les mesures consolidades per a emmagatzemar-les de
forma acotada. Així, tenen un operador associat que afegeix noves
mesures al disc mantenint sota control el seu cardinal.


L'operació d'afegir una mesura al disc consisteix en afegir-la a la
sèrie temporal i a eliminar la mesura mínima d'aquesta si se supera el
cardinal permès.
\begin{definition}[Afegeix mesura al disc]
  Sigui $D=(S,k)$ un disc i $m=(t,v)$ una mesura, la inserció de la
  mesura al disc $\glssymboldef{addD}(D,m)$ és un disc $D'=(S',k)$
  amb la mesura afegida a la sèrie temporal del disc mantenint el
  cardinal màxim: $\glssymbol{addD}(D,m) = (S',k)$ on $S'=
  \begin{cases}
      S\cup\{m\} &\text{si }  |S|<k\\
      (S-\{\min(S)\}) \cup \{m\} &\text{altrament}
    \end{cases}$.
\end{definition}




\subsubsection{Subsèrie resolució}

Les subsèries resolució són l'aparellament d'un buffer amb un disc.
Així tenen dos operadors associats, els quals treballen amb els
operadors del buffer i del disc : un per afegir una mesura al buffer i
un altre per consolidar el buffer i afegir la mesura resultant al disc


L'operació d'afegir una mesura a la subsèrie resolució consisteix en
afegir-la al buffer.
\begin{definition}[Afegeix mesura a la subsèrie resolució]
  Sigui $R=(B,D)$ una subsèrie resolució i $m=(t,v)$ una mesura, la
  inserció de la mesura a la subsèrie resolució
  $\glssymboldef{addR}(R,m)$ és una subsèrie resolució $R'=(B',D)$
  amb la mesura afegida al buffer: $\glssymbol{addR}(R,m) = (B',D)$ 
  on $B'=\glssymbol{addB}(B,m)$.
\end{definition}


L'operació de consolidar una subsèrie resolució consisteix en calcular
una mesura de consolidació del buffer, en l'interval de consolidació
actual, i desar-la al disc. 
\begin{definition}[Consolida la subsèrie resolució]
  Sigui $R=(B,D)$ una subsèrie resolució, la consolidació de la
  subsèrie resolució $\glssymboldef{consolidaR}(R)$ és una subsèrie
  resolució $R'=(B',D')$ on $(B',m') = \glssymbol{consolidaB}(B)$ i
  $D'=\glssymbol{addD}(D,m')$.
\end{definition}

Una subsèrie resolució és consolidable quan ho és el seu buffer.
\begin{definition}[Subsèrie resolució consolidable]
  Sigui $R=(B,D)$ una subsèrie resolució, definim que
  \glsdispdef{consolidableR}{$R$ és
    consolidable} si i només si \glsdisp{consolidableB}{$B$ és
    consolidable}
\end{definition}



\subsubsection{Sèrie temporal multiresolució}

Les sèries temporals multiresolució són un conjunt de subsèries
resolució. Així tenen dos operadors per a treballar globalment amb
totes les subèries que contingui: un per a afegir una mesura a cada
subsèrie i un altre per a consolidar cadascuna de les subsèries.


L'operació d'afegir una mesura a la sèrie temporal multiresolució
consisteix en afegir-la a cadascuna de les subsèries resolució.
\begin{definition}[Afegeix mesura a la sèrie temporal multiresolució]
  Sigui $M=\{R_0,\dotsc,R_d\}$ una sèrie temporal multiresolució i
  $m=(t,v)$ una mesura, la inserció de la mesura a la sèrie temporal
  multiresolució $\glssymboldef{addM}(M,m)$ és una sèrie temporal
  multiresolució $M'=\{R_0',\dotsc,R_d'\}$ amb la mesura afegida a
  cada subsèrie resolució: $\glssymbol{addM}(M,m) = \{ \forall R_i\in
  M: \glssymbol{addR}(R_i,m) \}$.
\end{definition}


L'operació de consolidar una sèrie temporal multiresolució consisteix
en consolidar cadascuna de les subsèries resolució que siguin
consolidables.
\begin{definition}[Consolida la sèrie temporal multiresolució]
  Sigui $M=\{R_0,\dotsc,R_d\}$ una sèrie temporal multiresolució, la
  consolidació de la sèrie temporal multiresolució
  $\glssymboldef{consolidaM}(M)$ és una sèrie temporal multiresolució
  $M'=\{R_0',\dotsc,R_d'\}$ que consolida les subsèries resolució
  consolidables:
  \[
  \glssymbol{consolidaM}(M) = \left\{ \forall R_i\in M:
  \begin{cases}
    \glssymbol{consolidaR}(R_i) & \text{si }
    \glsdisp{consolidableR}{R_i \text{ és
        consolidable}}\\
    R_i & \text{altrament}
  \end{cases}\right\}
  \].
\end{definition}



Una sèrie temporal multiresolució és un conjunt de subsèries
resolució. Per a tractar amb tots i cadascun dels elements del conjunt
és útil tenir l'operació de mapatge, de manera similar a com s'ha
definit per les sèries temporals però que es pugui obtenir qualsevol
conjunt de valors.
 \begin{definition}[Mapa d'una sèrie temporal multiresolució]
   \label{def:sgstm:mapa}
   Sigui $M=\{R_0,\dotsc,R_d\}$ una sèrie temporal multiresolució i
   \glssymbol{not:F} una funció sobre una subsèrie resolució que
   retorna un valor $v'$ on $\glssymbol{not:F}:R\mapsto v'$, el mapa
   de \glssymbol{not:F} a $M$ és un conjunt de valors resultant
   d'aplicar la funció a cada subsèrie resolució:
   $\glssymboldef{not:sgstm:map}(M,\glssymbol{not:F}) = \{
   \glssymbol{not:F}(R_0), \dotsc, \glssymbol{not:F}(R_d) \}$.
\end{definition}
Atès que una sèrie temporal multiresolució és un conjunt de subsèries
resolució, un dels resultats possible del mapa és una nova sèrie
temporal multiresolució si la funció és de la forma
$\glssymbol{not:F}:R\mapsto R'$.


% Aplicació de fold a una BDM
% \begin{gather*}
%   \text{fold}: M \times M_i \times f \longrightarrow M' :\\
%   = f(\dots(f(f(f(M_i,R_0),R_1),R_2)\dots),R_k), \\
%    \text{a on } f: M_a \times R_b \mapsto M'
% \end{gather*}





\subsection{Manipulació de l'esquema}
\label{sec:model:sgstm-manipulacio-esquema}


El model de \gls{SGSTM} associa a cada sèrie temporal un esquema de
multiresolució. En aquest apartat definim els operadors que permeten
consultar i manipular aquest esquema de multiresolució de forma
coherent amb el model de \gls{SGSTM}.
Segons s'ha descrit a la~\autoref{def:sgstm:esquema}, l'esquema de
multiresolució de cada sèrie temporal multiresolució consisteix en el
nombre de subsèries resolució i els quatre paràmetres de cadascuna: el
darrer instant de consolidació ($\tau$), el pas de consolidació
($\delta$), el cardinal màxim ($k$) i la funció d'agregació d'atributs
($f$).  Així doncs, quan es manipula una base de dades multiresolució
cal conservar o tractar adequadament aquests esquemes de
multiresolució.


A continuació es descriuen operadors per a poder estudiar aquest
esquema, operadors per a canviar-lo i operadors per a unir o ajuntar
dos esquemes.  Una sèrie temporal multiresolució és un conjunt de
subsèries resolució i per tant podem observar també l'esquema de
multiresolució com a conjunt de l'esquema de cada subsèrie.  Així, per
simplicitat, definim la majoria dels operadors de manipulació sobre
les subsèries resolució, tot i que es poden estendre fàcilment a una sèrie
temporal multiresolució mitjançant l'operació de mapatge
(v.~\autoref{def:sgstm:mapa}), on la funció $\glssymbol{not:F}$
correspon a l'operador d'esquema que es vol aplicar a totes i
cadascuna de les subsèries resolució.




\subsubsection{Propietats de l'esquema}


La configuració dels paràmetres de l'esquema de multiresolució
infereix diverses propietats a les sèries temporals multiresolució. A
continuació definim algunes de les propietats que es poden estudiar a
partir d'un esquema de multiresolució.


A partir d'un esquema de multiresolució es pot dibuixar la situació
relativa en el temps que prendran les mesures. A aquest dibuix de
l'esquema de multiresolució l'anomenem cronograma. Per exemple, sigui
la sèrie temporal multiresolució $M_3=\{R_1,R_2\}$ de
l'\autoref{ex:model:bdm-desfasaments} dibuixem el seu cronograma a la
\autoref{fig:model:cronograma} per a l'instant de temps $30$, just
abans de la seva consolidació. Les subsèries resolució de $M_3$ tenen
els paràmetres $\delta_1=5$, $\delta_2=10$, $k_1=4$, $k_2=3$,
$f_1=\text{mitjana}$ i $f_2=\text{mitjanad5}$ on la funció
$\text{mitjanad5}$, ja utilitzada a
l'\autoref{ex:model:bdm-desfasaments}, té un desfasament de $5$
unitats de temps. Tot seguit definim els conceptes que apareixen al
cronograma.

\begin{figure}[tp]
  \centering
  %\usetikzlibrary{positioning}
\begin{tikzpicture}

  %referencia
  \def\ut{1.2}
  \node (-10) {};

  \foreach \x in {-9,...,0}
  {
    \pgfmathparse{int(\x-1)}
    \let\antx=\pgfmathresult 
    \pgfmathparse{int(5*\x+30)}
    \let\nomx=\pgfmathresult
    
    \node[align=center] (\x) [right=\ut cm of \antx.center,anchor=center] {
    %  \x
    \nomx  %noms de les x
    };
    \node [above=0mm of \x] {\tiny$\vert$};
  }



  %eixos
  \node (et0) [above=0mm of 0] {};
  \node (eti) [left=11cm of et0] {};
  \node (etf) [right=0.5cm of et0] {};
  \draw[->] (eti) to (etf);
  \node [below=2mm of -10,anchor=west] {temps (u.t.)};

  \node (ara) [above=4cm of et0] {};
  \node (ara2) [below=5mm of et0] {};
  \draw[-] (ara) to (ara2);
  \node (ara3) [below=-2mm of ara2] {};

  


 %R1 d2 k4 tau0 f-1
  \node (R1) [above=1cm of 0] {};
  \node[ellipse,draw,minimum width=3*\ut cm] (R1b) [left=0mm of R1.center, anchor=east] {$S_{B2}$};
  \node[rectangle,draw,minimum width=2*\ut cm] (R1d1) [left=0mm of R1b.west, anchor=east] {$S_{D2}^1$};
  \node[rectangle,draw,minimum width=2*\ut cm,align=center] (R1d2) [left=0mm of R1d1.west, anchor=east] {$S_{D2}^2$};
  \node[rectangle,draw,minimum width=2*\ut cm,align=center] (R1d3) [left=0mm of R1d2.west, anchor=east] {$S_{D2}^3$};



  %R0 d1 k4 tau0
  \node (R0) [above=2.5cm of R1.center] {};
  \node[ellipse,draw,minimum width=\ut cm,align=center] (R0b) [left=0mm of R0.center, anchor=east] {$S_{B1}$};
  \node[rectangle,draw,minimum width=\ut cm,align=center] (R0d1) [left=0mm of R0b.west, anchor=east] {$S_{D1}^1$};
  \node[rectangle,draw,minimum width=\ut cm,align=center] (R0d2) [left=0mm of R0d1.west, anchor=east] {$S_{D1}^2$};
  \node[rectangle,draw,minimum width=\ut cm,align=center] (R0d3) [left=0mm of R0d2.west, anchor=east] {$S_{D1}^3$};
  \node[rectangle,draw,minimum width=\ut cm,align=center] (R0d4) [left=0mm of R0d3.west, anchor=east] {$S_{D1}^4$};


  %definicions
  \node (R0la) [above=5mm of R0d4.west] {};
  \node (R0lb) [above=5mm of R0d1.east] {};
  \draw[<->] (R0la.center) -- (R0lb.center) node [above,sloped,midway] {$lapseR(R_1)$};

  \node (R1la) [above=5mm of R1d3.west] {};
  \node (R1lb) [above=5mm of R1d1.east] {};
  \draw[<->] (R1la) -- (R1lb) node [above,sloped,midway] {$lapseR(R_2)$};


  \node (R0ba) [above=5mm of R0b.west] {};
  \node (R0bb) [above=5mm of R0b.east] {};
  \draw[<->] (R0ba) -- (R0bb) node [above,sloped,midway] {$periodeB(R_1)$};

  \node (R1ba) [above=12mm of R1b.west] {};
  \node (R1bb) [above=12mm of R1b.east] {};
  \draw[<->] (R1ba) -- (R1bb) node [above,sloped,midway] {$periodeB(R_2)$};

  \node (R1da) [above=5mm of R1b.west] {};
  \node (R1db) [above=5mm of -2] {};
  \draw[<->] (R1da.center) -- (R1da-|R1db) node [above,sloped,midway] {$desfasamentR(R_2)$};

  \node [below=0mm of -1] {$\tau_1$};
  \node [below=0mm of -2] {$\tau_2$};
  \node (tau0d) [below=0mm of 0] {$\tau_1+\delta_1$};
  \node [below=0mm of tau0d] {$\tau_2+\delta_2$};

\end{tikzpicture}
  \caption{Cronograma d'un esquema multiresolució just abans de la consolidació,  on $\glssymbol{not:sgstm:periodeB}(R_i)=\glssymbol{not:sgstm:lapseB}(R_i)$}
  \label{fig:model:cronograma}
\end{figure}


Els paràmetres $k$, $\delta$ i $f$ d'una subsèrie resolució són fixats
per l'esquema, mentre que el paràmetre $\tau$ és fixat a un valor
inicial i va essent canviat per l'operació de consolidació. Així doncs,
les propietats que impliquin a $\tau$ dependran de l'instant temporal
en què es faci la consulta i les que no l'impliquin seran fixes per a
cada esquema.

Una propietat que observem en el cronograma és el lapse temporal d'una
subsèrie resolució, és a dir la mida temporal que ocupa
la sèrie temporal emmagatzemada en el disc.
\begin{definition}[Lapse de la subsèrie resolució] %angl. span
  Sigui $R=(S_B,S_D,\tau,\delta,k,f)$ una subsèrie resolució, el lapse
  $\glssymboldef{not:sgstm:lapseR}(R)$ és una durada de temps que
  indica la mida de l'interval que ocupa el disc:
  $\glssymbol{not:sgstm:lapseR}(R) = k\delta$.
\end{definition}


Si definim l'interval de temps del lapse com a $[\tau - k\delta,
\tau]$ i l'interval de temps real de la sèrie temporal del disc com a
$[\min(S_D),\max(S_D)]$, aleshores normalment es complirà que
$\max(S_D)=\tau$ i $\min(S_D)=\tau - (k-1)\delta$. No obstant això,
pot no complir-se per exemple si la sèrie temporal no és regular o per
exemple si $\tau$ i $\max(S_D)$ no coincideixen.  Aquest darrer fet
que $\tau$ i $\max(S_D)$ no coincideixen ocorre quan la funció
d'agregació d'atributs causa un desfasament; ho anomenem desfasament
de la subsèrie resolució.
\begin{definition}[Desfasament de la subsèrie resolució] %angl. offset
  \label{def:sgstm:desdsamentR}
  Sigui $R=(S_B,S_D,\tau,\delta,k,f)$ una subsèrie resolució, el seu
  desfasament $\glssymboldef{not:sgstm:desfasamentR}(R)$ és una durada
  de temps que indica la distància entre $\tau$ i $\max(S_D)$ causada
  per la funció d'agregació $f$. Així havent definit la consolidació
  sobre $m'=f(S,[\tau,\tau+\delta]$
  (v.~\autoref{def:model:consolidacio-buffer}), una funció d'agregació
  amb desfasament retorna una mesura $m'=(\tau - r,v)$ con
  $\glssymbol{not:sgstm:desfasamentR}(R) = r$.
\end{definition}



Un altre desfasament que es produeix, aquest però variable, és la
variació de temps que hi ha entre el darrer instant de consolidació i
l'instant de temps actual. Aquest interval de temps és durant el qual
el buffer emmagatzema les noves mesures que arriben i l'anomenen
període de buffer.
\begin{definition}[Període de buffer de la subsèrie resolució]
  Sigui $R=(S_B,S_D,\tau,\delta,k,f)$ una subsèrie resolució,
  $\glssymbol{not:sgstm:desfasamentR}(R)$ el seu desfasament i $t_N$
  l'instant de temps actual, el període de buffer de la subsèrie
  resolució $\glssymboldef{not:sgstm:periodeB}(R)$ és una durada de
  temps que indica la distància entre $\tau$ i $t_N$ tenint en compte
  el desfasament: $\glssymbol{not:sgstm:periodeB}(R) = t_N - (\tau -
  \glssymbol{not:sgstm:desfasamentR}(R))$.
\end{definition}

Si la consolidació de la subsèrie resolució es realitza immediatament
cada cop que estigui en estat de consolidable i existeix una mesura
$m\in S_B$ que compleix $T(m)=t_N$, aleshores el període de buffer té
una variació afitada atès que, per ser la consolidació immediata, $t_N
- \tau \leq \delta$. En aquest cas, el mínim que pot prendre el
període de buffer és el desfasament,
$\glssymbol{not:sgstm:desfasamentR}(R)\leq
\glssymbol{not:sgstm:periodeB}(R)$, que correspon a l'instant
$t_N=\tau$. El màxim que pot prendre l'anomenem lapse de buffer de la
subsèrie resolució.
\begin{definition}[Lapse de buffer de la subsèrie resolució]
  Sigui $R=(S_B,S_D,\tau,\delta,k,f)$ una subsèrie resolució i
  $\glssymbol{not:sgstm:desfasamentR}(R)$ el seu desfasament, el lapse
  de buffer de la subsèrie resolució
  $\glssymboldef{not:sgstm:lapseB}(R)$ és una durada de temps que
  indica el període de buffer màxim que pot prendre:
  $\glssymbol{not:sgstm:lapseB}(R) = \delta +
  \glssymbol{not:sgstm:desfasamentR}(R)$.  Sempre es compleix que
  $\glssymbol{not:sgstm:periodeB}(R) \leq
  \glssymbol{not:sgstm:lapseB}(R)$ atès que, per ser la consolidació
  immediata, $t_N - \tau \leq \delta$.
\end{definition}


Una altra propietat de l'esquema de multiresolució és la
regularització de les sèries temporals emmagatzemades en el disc. El
període d'aquesta sèrie temporal del disc normalment es correspondrà
amb $\delta$.  Així, segons el pas de consolidació descrit a
l'esquema de multiresolució, descrivim quina subsèrie resolució conté
més resolució.
\begin{definition}[Subsèrie resolució amb més resolució]
  Siguin $R_1=(S_{B1},S_{D1},\tau_1,\delta_1,k_1,f_1)$ i
  $R_2=(S_{B2},S_{D2},\tau_2,\delta_2,k_2,f_2)$ dues subsèries
  resolució, la subsèrie amb més resolució
  $\glssymboldef{not:sgstm:maxR}(R_1,R_2)$ és la que té el pas de
  consolidació més petit: $\glssymbol{not:sgstm:maxR}(R_1,R_2) = R_i$
  on $\delta_i = \min(\delta_1,\delta_2)$.
\end{definition}

Això no obstant, notem que en determinats instants el pas de
consolidació descrit a l'esquema de multiresolució pot no coincidir
amb el període de regularitat de la sèrie temporal emmagatzemada al
disc, com per exemple durant un canvi en l'esquema del pas de
consolidació (vegeu més endavant
la \autoref{def:sgstm:canviad}). 


Resumint, podem dir que una subsèrie resolució té, per una banda,
informació consolidada per un lapse temporal de
$\glssymbol{not:sgstm:lapseR}(R)$, el qual es posiciona absolutament
des de $\tau + \glssymbol{not:sgstm:desfasamentR}(R)$ enrere. Per
altra banda, la subsèrie resolució té informació no consolidada al
davant de la consolidada per un interval de mida
$\glssymbol{not:sgstm:periodeB}(R)$ i de com a màxim
$\glssymbol{not:sgstm:lapseB}(R)$. A continuació mostrem aquestes
propietats en cronogrames d'esquemes multiresolució per a la mateixa
sèrie temporal multiresolució però en diferents situacions temporals
d'exemple.



\begin{example}[Instant de temps just abans de la consolidació]
  \label{ex:sgstm:abanscons}
  El cronograma de la \autoref{fig:model:cronograma} mostra la sèrie
  temporal multiresolució $M_3=\{R_1,R_2\}$ de
  l'\autoref{ex:model:bdm-desfasaments} a l'instant de temps
  determinat, el $t_N=30$. Exactament mostra una fotografia en
  l'instant just abans d'aplicar l'operació de consolidació per a
  totes dues subsèries, les quals ja són consolidables:
  $t_N\geq\delta_1+\tau_1$ i $t_N\geq\delta_2+\tau_2$ assumint que
  existeixen les mesures $m_1\in S_{D1}: T(m_1)=t_N$ i $m_2\in S_{D2}:
  T(m_2)=t_N$.  Aquest és un moment en què es compleix la propietat
  $\glssymbol{not:sgstm:periodeB}(R_i)=\glssymbol{not:sgstm:lapseB}(R_i)$,
  és a dir en què el període de buffer té la mida màxima.

  En aquest cronograma es poden observar totes les propietats d'un
  cronograma multiresolució que ara, un cop definides, en calculem els
  valors concrets per l'exemple:
  \begin{itemize}
  \item Lapses de les subsèries:
    $\glssymbol{not:sgstm:lapseR}(R_1)=k_1\delta_1=20$ i
    $\glssymbol{not:sgstm:lapseR}(R_2)=k_2\delta_2=30$.
  \item Desfasaments de les subsèries:
    $\glssymbol{not:sgstm:desfasamentR}(R_1)=0$ i
    $\glssymbol{not:sgstm:desfasamentR}(R_2)=5$, segons interpretació
    de la $f_1=\text{mitjana}$ i $f_2=\text{mitjanad5}$.
  \item Períodes de buffer: $\glssymbol{not:sgstm:periodeB}(R_1)
    =t_N-(\tau_1-\glssymbol{not:sgstm:desfasamentR}(R_1))
    =30-(25-0)=5$ i $\glssymbol{not:sgstm:periodeB}(R_2)
    =t_N-(\tau_2-\glssymbol{not:sgstm:desfasamentR}(R_2)))
    =30-(20-5)=15$ atès que en l'instant de temps actual $t_N=30$ just
    abans de consolidar-se $\tau_1=25$ i $\tau_2=20$.
  \item Lapses de buffer: $\glssymbol{not:sgstm:lapseB}(R_1)=
    \delta_1+\glssymbol{not:sgstm:desfasamentR}(R_1)=5$ i
    $\glssymbol{not:sgstm:lapseB}(R_2)=
    \delta_2+\glssymbol{not:sgstm:desfasamentR}(R_2)=15$.
  \item Subsèrie amb més resolució:
    $\glssymboldef{not:sgstm:maxR}(R_1,R_2)=R_1$ atès que
    $\delta_1=\min(\delta_1,\delta_2)$.
  \end{itemize}




\end{example}



\begin{example}[Instant de temps just després de la consolidació]
   \label{ex:sgstm:desprescons}


   El cronograma canvia amb el pas del temps i per tant també canvia
   la mida dels buffers i la posició temporal dels discs.  Ara el
   cronograma de la \autoref{fig:model:cronograma} mostra el mateix
   cas que l'\autoref{ex:sgstm:abanscons} per $t_N=30$ però exactament
   en una fotografia a l'instant just després d'aplicar l'operació de
   consolidació per a totes dues subsèries. Aquest és un moment en què
   es compleix la propietat $\glssymbol{not:sgstm:periodeB}(R_i)=
   \glssymbol{not:sgstm:desfasamentR}(R_i)$, és a dir en què el
   període de buffer té la mida mínima.

\begin{figure}[tp]
  \centering
  %\usetikzlibrary{positioning}
\begin{tikzpicture}

  %referencia
  \def\ut{1.2}
  \node (-8) {};

  \foreach \x in {-7,...,2}
  {
    \pgfmathparse{int(\x-1)}
    \let\antx=\pgfmathresult
    \pgfmathparse{int(5*\x+30)}
    \let\nomx=\pgfmathresult

    \node[align=center] (\x) [right=\ut cm of \antx.center,anchor=center] {
    %  \x
    \nomx  %noms de les x      
    };
    \node [above=0mm of \x] {\tiny$\vert$};
  }


  %eixos
  \node (et0) [above=0mm of 0] {};
  \node (eti) [left=8.5cm of et0] {};
  \node (etf) [right=3cm of et0] {};
  \draw[->] (eti) to (etf);
  \node [below=2mm of -8,anchor=west] {temps (u.t.)};

  \node (ara) [above=4cm of et0] {};
  \node (ara2) [below=5mm of et0] {};
  \draw[-] (ara) to (ara2);
  \node (ara3) [below=-2mm of ara2] {};

  


 %R1 d2 k4 tau0 f-1
  \node (R1) [above=1cm of 0] {};
  \node[ellipse,draw,minimum width=\ut cm] (R1b) [left=0mm of R1.center, anchor=east] {$S_{B1}$};
  \node[rectangle,draw,minimum width=2*\ut cm] (R1d1) [left=0mm of R1b.west, anchor=east] {$S_{D1}^1$};
  \node[rectangle,draw,minimum width=2*\ut cm,align=center] (R1d2) [left=0mm of R1d1.west, anchor=east] {$S_{D1}^2$};
  \node[rectangle,draw,minimum width=2*\ut cm,align=center] (R1d3) [left=0mm of R1d2.west, anchor=east] {$S_{D1}^3$};



  %R0 d1 k4 tau0
  \node (R0) [above=2cm of R1.center] {};
%  \node[ellipse,draw,minimum width=\ut cm,align=center] (R0b) [left=0mm of R0.center, anchor=east] {$S_{B1}$};
  \node[rectangle,draw,minimum width=\ut cm,align=center] (R0d1) [left=0mm of R0.center, anchor=east] {$S_{D0}^1$};
  \node[rectangle,draw,minimum width=\ut cm,align=center] (R0d2) [left=0mm of R0d1.west, anchor=east] {$S_{D0}^2$};
  \node[rectangle,draw,minimum width=\ut cm,align=center] (R0d3) [left=0mm of R0d2.west, anchor=east] {$S_{D0}^3$};
  \node[rectangle,draw,minimum width=\ut cm,align=center] (R0d4) [left=0mm of R0d3.west, anchor=east] {$S_{D0}^4$};


  %definicions
  \node (R0la) [above=5mm of R0d4.west] {};
  \node (R0lb) [above=5mm of R0d1.east] {};
  \draw[<->] (R0la.center) -- (R0lb.center) node [above,sloped,midway] {$lapseR(R_0)$};

  \node (R1la) [above=5mm of R1d3.west] {};
  \node (R1lb) [above=5mm of R1d1.east] {};
  \draw[<->] (R1la) -- (R1lb) node [above,sloped,midway] {$lapseR(R_1)$};


%  \node (R0ba) [above=5mm of R0b.west] {};
%  \node (R0bb) [above=5mm of R0b.east] {};
%  \draw[<->] (R0ba) -- (R0bb) node [above,sloped,midway] {$periodeB(R_0)$};

  \node (R1ba) [above=5mm of R1b.west] {};
  \node (R1bb) [above=5mm of R1b.east] {};
  \draw[<->] (R1ba) -- (R1bb) node [above,sloped,midway] {$periodeB(R_1)$};

  % \node (R1da) [above=5mm of R1b.west] {};
  % \node (R1db) [above=5mm of -2] {};
  % \draw[<->] (R1da.center) -- (R1da-|R1db) node [above,sloped,midway] {$desfasamentB(R_1)$};


  \node (tau0d) [below=0mm of 0] {$\tau_0$};
  \node [below=0mm of tau0d] {$\tau_1$};
  \node [below=0mm of 1] {$\tau_0+\delta_0$};
  \node [below=0mm of 2] {$\tau_1+\delta_1$};

\end{tikzpicture}
  \caption{Cronograma d'un esquema multiresolució just després de la consolidació, on $\glssymbol{not:sgstm:periodeB}(R_i)=\glssymbol{not:sgstm:desfasamentR}(R_i)$}
  \label{fig:model:cronograma-consolidat}
\end{figure}



Pel que fa als valors de les propietats del cronograma, tots són els
mateixos que a l'\autoref{ex:sgstm:abanscons} llevat de:
 \begin{itemize}
  \item Períodes de buffer: $\glssymbol{not:sgstm:periodeB}(R_1)
    =t_N-(\tau_1-\glssymbol{not:sgstm:desfasamentR}(R_1))
    =30-(30-0)=0$ i $\glssymbol{not:sgstm:periodeB}(R_2)
    =t_N-(\tau_2-\glssymbol{not:sgstm:desfasamentR}(R_2)))
    =30-(30-5)=5$ atès que en l'instant de temps actual $t_N=30$ just
    després de consolidar-se $\tau_1=30$ i $\tau_2=30$.
  \end{itemize}
 

\end{example}




\begin{example}[Cas no ideal amb consolidació retardada]

  L'\autoref{ex:sgstm:desprescons} és la continuació temporal
  (instantània) de l'\autoref{ex:sgstm:abanscons} assumint que en
  l'instant $t_N=30$ s'aplica l'operació de consolidació.  Ara
  descrivim un cas no ideal en què aquesta operació de consolidació no
  s'aplica fins a l'instant $t_N=35$. Això podria ser degut que no es
  vol executar l'operació en el mateix moment que les subsèries siguin
  consolidables o bé que encara no són consolidables perquè no
  existeixen les mesures $m_1\in S_{B1}: 30\leq T(m_1)\leq 35$ i
  $m_2\in S_{B2}: 30\leq T(m_2)\leq 35$.




\begin{figure}[tp]
  \centering
  %\usetikzlibrary{positioning}
\begin{tikzpicture}

  %referencia
  \def\ut{1.2}
  \node (-10) {};

  \foreach \x in {-9,...,1}
  {
    \pgfmathparse{int(\x-1)}
    \let\antx=\pgfmathresult 
    \pgfmathparse{int(5*\x+30)}
    \let\nomx=\pgfmathresult
    
    \node[align=center] (\x) [right=\ut cm of \antx.center,anchor=center] {
    %  \x
    \nomx  %noms de les x
    };
    \node [above=0mm of \x] {\tiny$\vert$};
  }



  %eixos
  \node (et0) [above=0mm of 0] {};
  \node (eti) [left=11cm of et0] {};
  \node (etf) [right=1.5cm of et0] {};
  \draw[->] (eti) to (etf);
  \node [below=2mm of -10,anchor=west] {temps (u.t.)};

  \node (ara) [above=6cm of et0] {};
  \node (ara2) [below=5mm of et0] {};
  \draw[dashed,gray] (ara) to (ara2);
  \node (ara3) [below=-2mm of ara2] {};


  \node(et1) [above=0mm of 1] {};
  \node (ara) [above=6cm of et1] {};
  \node (ara2) [below=5mm of et1] {};
  \draw[-] (ara) to (ara2);
  \node (ara3) [below=-2mm of ara2] {};

  


 %R1 d2 k4 tau0 f-1
  \node (R1) [above=1cm of 1] {};
  \node[ellipse,draw,minimum width=4*\ut cm] (R1b) [left=0mm of R1.center, anchor=east] {$S_{B2}$};
  \node[rectangle,draw,minimum width=2*\ut cm] (R1d1) [left=0mm of R1b.west, anchor=east] {$S_{D2}^1$};
  \node[rectangle,draw,minimum width=2*\ut cm,align=center] (R1d2) [left=0mm of R1d1.west, anchor=east] {$S_{D2}^2$};
  \node[rectangle,draw,minimum width=2*\ut cm,align=center] (R1d3) [left=0mm of R1d2.west, anchor=east] {$S_{D2}^3$};



  %R0 d1 k4 tau0
  \node (R0) [above=3.2cm of R1.center] {};
  \node[ellipse,draw,minimum width=2*\ut cm,align=center] (R0b) [left=0mm of R0.center, anchor=east] {$S_{B1}$};
  \node[rectangle,draw,minimum width=\ut cm,align=center] (R0d1) [left=0mm of R0b.west, anchor=east] {$S_{D1}^1$};
  \node[rectangle,draw,minimum width=\ut cm,align=center] (R0d2) [left=0mm of R0d1.west, anchor=east] {$S_{D1}^2$};
  \node[rectangle,draw,minimum width=\ut cm,align=center] (R0d3) [left=0mm of R0d2.west, anchor=east] {$S_{D1}^3$};
  \node[rectangle,draw,minimum width=\ut cm,align=center] (R0d4) [left=0mm of R0d3.west, anchor=east] {$S_{D1}^4$};


  %definicions
  \node (R0la) [above=5mm of R0d4.west] {};
  \node (R0lb) [above=5mm of R0d1.east] {};
  \draw[<->] (R0la.center) -- (R0lb.center) node [above,sloped,midway] {$\glssymbol{not:sgstm:lapseR}(R_1)$};

  \node (R1la) [above=5mm of R1d3.west] {};
  \node (R1lb) [above=5mm of R1d1.east] {};
  \draw[<->] (R1la) -- (R1lb) node [above,sloped,midway] {$\glssymbol{not:sgstm:lapseR}(R_2)$};


  \node (R0ba) [above=12mm of R0b.west] {};
  \node (R0bb) [above=12mm of R0b.east] {};
  \draw[<->] (R0ba) -- (R0bb) node [above,sloped,midway] {$\glssymbol{not:sgstm:periodeB}(R_1)$};

  \node (R1ba) [above=19mm of R1b.west] {};
  \node (R1bb) [above=19mm of R1b.east] {};
  \draw[<->] (R1ba) -- (R1bb) node [above,sloped,midway] {$\glssymbol{not:sgstm:periodeB}(R_2)$};

  \node (R1da) [above=5mm of R1b.west] {};
  \node (R1db) [above=5mm of -2] {};
  \draw[<->] (R1da.center) -- (R1da-|R1db) node [above,sloped,midway] {$\glssymbol{not:sgstm:desfasamentR}(R_2)$};


  \node (R0lba) [above=5mm of R0b.west] {};
  \node (R0lbb) [right=\ut of R0lba.west] {};
  \draw[<->] (R0lba) -- (R0lbb) node [above,sloped,midway] {$\glssymbol{not:sgstm:lapseB}(R_1)$};

  \node (R1lba) [above=12mm of R1b.west] {};
  \node (R1lbb) [right=3*\ut of R1lba.west] {};
  \draw[<->] (R1lba) -- (R1lbb) node [above,sloped,midway] {$\glssymbol{not:sgstm:lapseB}(R_2)$};




  \node [below=0mm of -1] {$\tau_1$};
  \node [below=0mm of -2] {$\tau_2$};
  \node (tau0d) [below=0mm of 0] {$\tau_1+\delta_1$};
  \node [below=0mm of tau0d] {$\tau_2+\delta_2$};

\end{tikzpicture}
  \caption{Cronograma d'un esquema multiresolució amb consolidació retardada,  on $\glssymbol{not:sgstm:periodeB}(R_i)>\glssymbol{not:sgstm:lapseB}(R_i)$}
  \label{fig:model:cronograma-noideal}
\end{figure}


  Així doncs, reprenem l'\autoref{ex:sgstm:abanscons} i avancem el
  temps actual fins a l'instant $t_N=35$ però sense aplicar la
  consolidació. El cronograma es mostra a la
  \autoref{fig:model:cronograma-noideal}, en línia contínua vertical
  es mostra l'instant de temps actual $t_N=30$ i en línia discontínua
  es mostra la consolidació immediata ideal que hauria estat a
  l'instant $30$.  Aquest és un moment en què apareix la propietat
  $\glssymbol{not:sgstm:periodeB}(R_i)>\glssymbol{not:sgstm:lapseB}(R_i)$,
  és a dir en què el període de buffer supera la mida màxima que té en
  cas de consolidació immediata.



Pel que fa als valors de les propietats del cronograma, tots són els
mateixos que a l'\autoref{ex:sgstm:abanscons} llevat de:
 \begin{itemize}
 \item Períodes de buffer: $\glssymbol{not:sgstm:periodeB}(R_1)
   =t_N-(\tau_1-\glssymbol{not:sgstm:desfasamentR}(R_1))
   =35-(25-0)=10$ i $\glssymbol{not:sgstm:periodeB}(R_2)
   =t_N-(\tau_2-\glssymbol{not:sgstm:desfasamentR}(R_2)))
   =35-(20-5)=20$ atès que en l'instant de temps actual $t_N=35$
   encara queda pendent de consolidar-se $\tau_1=25$ i $\tau_2=20$.
  \end{itemize}

\end{example}



\begin{figure}[tp]
  \centering
  \begin{tikzpicture}

  %R0
  \node[circle,draw,minimum width=2 cm] (R0d) {$\delta_{0}$};
  \draw [->] (R0d.center)++(.4:.4cm) arc(0:180:.4cm);

  \node[circle,draw] [left=0mm of R0d.north,anchor=center] {};
  \node[circle,draw] [left=0mm of R0d.east,anchor=center] {};
  \node[circle,draw] [left=0mm of R0d.south,anchor=center] {};
  \node[circle,draw] [left=0mm of R0d.west,anchor=center] {};

  %R1
  \node[circle,draw,minimum width=2 cm] (R1d) [right=of R0d] {$\delta_{1}$};
  \draw [->] (R1d.center)++(.4:.4cm) arc(0:180:.4cm);

  \node[circle,draw] [left=0mm of R1d.north,anchor=center] {};
  \node[circle,draw] [left=0mm of R1d.south west,anchor=center] {};
  \node[circle,draw] [left=0mm of R1d.south east,anchor=center] {};

\end{tikzpicture}
\begin{tikzpicture}

  %referencia
  \def\ut{1.2}
  \node (-8) {};

  \foreach \x in {-7,...,0}
  {
    \pgfmathparse{int(\x-1)}
    \let\antx=\pgfmathresult
    \pgfmathparse{int(5*\x)}
    \let\nomx=\pgfmathresult

    \node[align=center] (\x) [right=\ut cm of \antx.center,anchor=center] {
    %  \x
    \nomx  %noms de les x      
    };
    \node [above=0mm of \x] {\tiny$\vert$};
  }


  %eixos
  \node (et0) [above=0mm of 0] {};
  \node (eti) [left=8.5cm of et0] {};
  \node (etf) [right=0.5cm of et0] {};
  \draw[->] (eti) to (etf);
  \node [below=2mm of -8,anchor=west] {temps (u.t.)};

  \node (ara) [above=3cm of et0] {};
  \node (ara2) [below=5mm of et0] {};
  \draw[-] (ara) to (ara2);
  \node (ara3) [below=-2mm of ara2] {};

  


 %R1 d2 k4 tau0 f-1
  \node (R1) [above=1cm of 0] {};
  \node[rectangle,draw,minimum width=6*\ut cm] (R1d) [left=\ut cm of R1.center, anchor=east] {$\text{lapseR}(R_2)$};




  %R0 d1 k4 tau0
  \node (R0) [above= of R1.center] {};
  \node[rectangle,draw,minimum width=4*\ut cm] (R0d) [left=0 cm of R0.center, anchor=east] {$\text{lapseR}(R_1)$};
\end{tikzpicture}

  \caption{Cronograma periòdic d'un esquema multiresolució}
  \label{fig:model:cronograma-simplificat}
\end{figure}


\paragraph{Cronograma periòdic}
El cronograma evoluciona amb el temps i a cada instant determinat
presenta una forma. Això no obstant totes les propietats de l'esquema
són fixes llevat del període de buffer que varia segons $t_N$, a més
si assumim el cas ideal de consolidació immediata aleshores les formes
es repeteixen periòdicament en el temps de $\delta$ per a cada
subsèrie resolució.  Per a simplificar, dibuixem un cronograma on no
aparegui el període de buffer i que només mostri la informació
periòdica del cronograma.  A tal efecte, establim l'instant 0 com
aquell on totes les subsèries resolució es consoliden al mateix temps
i dibuixem els lapses enrere en el temps. A la
\autoref{fig:model:cronograma-simplificat} es pot veure el cronograma
periòdic per als exemples anteriors on l'eix del temps  indica
les durades i les posicions relatives dels discs de les subsèries
resolució. La informació referent al pas de consolidació i la
resolució dels discs es dibuixa a la part superior amb formes
circulars, de manera que s'observa clarament quina subsèrie té més
resolució.




\subsubsection{Canvis en l'esquema}


Canviar l'esquema multiresolució d'una sèrie temporal significa crear
un esquema nou de multiresolució i emmagatzemar-hi les dades de
l'esquema vell de forma que es conservi la coherència que tenien
aquestes dades. Així doncs, a continuació definim algunes operacions
de canvi d'esquema que es poden aplicar als objectes d'una base de
dades multiresolució de forma coherent amb les dades emmagatzemades.


L'operació que canvia la mida del disc d'una subsèrie resolució ha de
controlar que si la mida disminueix s'han d'eliminar dades.
\begin{definition}[Canvi de mida d'una subsèrie resolució]
  Sigui $R=(B,D)$ una subsèrie resolució, en què $D=(S_D,k)$ és el
  disc de la subsèrie, i $k'$ un nou cardinal màxim, el canvi de mida
  de la subsèrie resolució $\glssymboldef{not:sgstm:canviaK}(R,k')$ és
  una nova subsèrie resolució $R'=(B,D')$ amb les dades antigues
  afegides de nou al disc: $\glssymbol{not:sgstm:canviaK}(R,k')=
  (B,D')$ aplicant l'operació $\forall m_i \in S_D:
  \glssymbol{addD}(D',m_i)$ amb $D'=(\{\},k')$ com a disc inicial.
\end{definition}


L'operació que canvia el pas de consolidació d'una subsèrie resolució
no cal que tingui en compte les dades ja que la sèrie temporal
emmagatzemada s'anirà canviant quan es consolidin noves mesures.
\begin{definition}[Canvi de pas de consolidació d'una subsèrie resolució]
  \label{def:sgstm:canviad}
  Sigui $R = (S_B,S_D,\delta,\tau,k,f)$ una subsèrie resolució i
  $\delta'$ un nou pas de consolidació, el canvi del pas de
  consolidació de la subsèrie resolució
  $\glssymboldef{not:sgstm:canviad}(R,\delta')$ és la nova subsèrie resolució
  $\glssymbol{not:sgstm:canviad}(R,\delta')=(S_B,S_D,\delta',\tau,k,f)$.
\end{definition}

Hi ha altres canvis de l'esquema que tampoc no requereixen tenir en
compte les dades, com per exemple canviar la funció d'agregació
d'atribut. Atès que són molt semblants al canvi de pas de
consolidació, no mostrem les definicions específics per a aquestes
operacions.


L'operació que canvia la resolució d'una subsèrie resolució modifica
alhora la mida i el pas de consolidació tenint en compte les dades; és
a dir, canvia el període de la sèrie temporal emmagatzemada seguint el
criteri de representació.
\begin{definition}[Canvi de resolució d'una subsèrie resolució]
  Sigui $R = (S_B,S_D,\delta,\tau,k,f)$ una subsèrie resolució, $r$
  una funció de representació per a la sèrie temporal, $k'$ un nou
  cardinal màxim i $\delta'$ un nou pas de consolidació, el canvi de
  resolució de la subsèrie resolució
  $\glssymboldef{not:sgstm:canviaR}(R,k',\delta',r)$ és una nova subsèrie
  resolució $R' = (S_B,S_D',\delta',\tau,k',f)$ amb una selecció
  temporal segons el criteri de representació en el nou conjunt
  regular de temps:
  $\glssymbol{not:sgstm:canviaR}(R,k',\delta',r)=(S_B,S_D',\delta',\tau,k',f)$
  a on $S_D' = S_D[i]^r$ i $i=\{ \tau-n\delta' | n\in\glssymbol{not:N},n<k'
  \}$.
\end{definition}


Per a treballar amb sèries temporals multivaluades cal que les sèries
temporals emmagatzemades als buffers i al discs tinguin la mateixa
forma. Així afegir un nou multivalor significa afegir un nou atribut a
les sèries temporals del buffer i dels disc de cada subsèrie
resolució. A continuació definim com afegir un multivalor a una
subsèrie resolució considerant sèries temporals en la forma canònica;
a la pràctica, per comoditat, habitualment cada atribut de multivalor
tindrà un nom.
\begin{definition}[Afegeix multivalor a una subsèrie resolució]
  Sigui $R = (S_B,S_D,\delta,\tau,k,f)$ una subsèrie resolució,
  l'addició d'un nou multivalor és una subsèrie resolució
  $\glssymboldef{not:sgstm:addMV}(R) = (S_B',S_D',\delta,\tau,k,f)$
  amb les sèries temporals ampliades amb un nou atribut inicialment de
  valor indefinit $S'_{B} =
  \glssymbol{not:sgst:map}(S_B,(t,v)\mapsto(t,(v,\infty)))$ i $S'_{D}
  = \glssymbol{not:sgst:map}(S_D,(t,v)\mapsto(t,(v,\infty)))$.
\end{definition}

Per a eliminar un multivalor només cal eliminar l'atribut a totes les
sèries temporals del buffer i dels disc de cada subsèrie resolució.
Atès que és més senzill que afegir un multivalor, no en mostrem la
definició específica.


\subsubsection{Unió i junció de dos esquemes}



Dos casos particulars de canvi d'esquema multiresolució són el d'unió
i el de junció de dos esquemes. En aquests canvis cal crear un nou
esquema tot conservant la coherència de dades emmagatzemades en dos
esquemes diferents.

Un exemple d'aplicació de la unió de dos esquemes és el següent. Es
mesuren els valors d'una sèrie temporal i durant un temps
s'emmagatzemen com a sèrie temporal multiresolució en una base de
dades però durant un altre temps s'emmagatzemen en una altra base de
dades. En acabar es volen unir els valors emmagatzemats a les dues
bases de dades.



L'operació d'unió de dues subsèries resolució és una unió de les
sèries temporals de cada un.
\begin{definition}[Unió de dues subsèries resolució]
  Sigui $R_1=(S_{B1},S_{D1},\delta_1,\tau_1,k_1,f_1)$ i
  $R_2=(S_{B2},S_{D2},\delta_2,\tau_2,k_2,f_2)$ dues subsèries
  resolució, la unió de les dues subsèries resolució
  $\glssymboldef{not:sgstm:unioR}(R_1,R_2)$ és una subsèrie resolució $R' =
  (S'_B,S'_D,\delta',\tau',k',f')$ que conté la unió de les sèries
  temporals dels seus buffers i discs: $\glssymbol{not:sgstm:unioR}(R_1,R_2)=
  (S'_B,S'_D,\delta_1,\max(\tau_1,\tau_2),k_1+k_2,f_1)$ a on $S'_B =
  S_{B1} \cup S_{B2}$ i $S'_D = S_{D1} \cup S_{D2}$.
\end{definition}
La unió de dues subsèries resolució no és commutativa degut a que les
unions de les sèries temporals no ho són. Tampoc ho és degut a que si
s'uneixen dues subsèries resolució amb diferent $\delta$ i $f$ llavors
es determina que la primera marca quins són els $\delta'$ i $f'$
resultants.  És a dir que en cas que es vulguin unir subsèries
resolució que continguin mesures en el mateix temps o informació
diferent; es prioritza la primera.


L'operació d'unió de dues sèries temporals multiresolució és la unió
de conjunts per a les subsèries resolució quan no intersequen en les
claus $(\delta,f)$. En cas que intersequin cal unir les dues subsèries
resolució.
\begin{definition}[Unió de dues sèries temporals  multiresolució]
  Sigui $M_1=\{R_0^1,\dotsc,R_{d1}^1\}$ i
  $M_2=\{R_0^2,\dotsc,R_{d2}^2\}$ dues sèries temporals
  multiresolució, la unió de les dues
  $\glssymboldef{not:sgstm:unioM}(M_1, M_2)$ és una sèrie temporal
  multiresolució $M'=\{R_0',\dotsc,R_d'\}$ que conté les subsèries que
  no intersequen i la unió de les subsèries que intersequen:
  $\glssymbol{not:sgstm:unioM}(M_1, M_2)= M_{a} \cup M'_{1} \cup
  M'_{2}$ a on $M_a = \{\forall R_1\in M_1,R_2\in M_2:
  \glssymbol{not:sgstm:unioR}(R_1,R_2) | (\delta_1,f_1) =
  (\delta_2,f_2) \}$, $M_1' =\{\forall R_1\in M_1: R_1 | \nexists R_2
  \in M_2 : (\delta_1,f_1) = (\delta_2,f_2) \}$ i $M_2' =\{\forall
  R_2\in M_2: R_2 | \nexists R_1 \in M_1 : (\delta_1,f_1) =
  (\delta_2,f_2) \}$.
\end{definition}
La unió de dues sèries temporals multiresolució no és commutativa
degut a que la unió de subsèries resolució no ho és.





L'operació de junció de dues subsèries resolució és la fusió de les
dades de les dues en una sèrie temporal multivalor. En el cas que els
vectors de temps siguin els mateixos es pot utilitzar la junció de
sèries temporals. En el cas que alguns temps no coincideixin cal
utilitzar la junció temporal amb una representació que donarà valors
allà on manquin.
\begin{definition}[Junció de dues subsèries resolució]
  Sigui $R_1=(S_{B1},S_{D1},\delta_1,\tau_1,k_1,f_1)$ i
  $R_2=(S_{B2},S_{D2},\delta_2,\tau_2,k_2,f_2)$ dues subsèries
  resolució i $r$ una funció de representació de sèries temporals, la
  junció de les dues subsèries resolució
  $\glssymboldef{not:sgstm:joinR}^r(R_1,R_2)$ és una subsèrie
  resolució $R' = (S'_B,S'_D,\delta',\tau',k',f')$ que conté la junció
  temporal de les sèries temporals dels seus buffers i discs:
  $\glssymbol{not:sgstm:joinR}^r(R_1,R_2)=
  (S'_B,S'_D,\delta_1,\max(\tau_1,\tau_2),k_1+k_2,f_1)$ a on $S'_B =
  S_{B1} \glssymbol{not:sgst:joint}^r S_{B2}$ i $S'_D = S_{D1}
  \glssymbol{not:sgst:joint}^r S_{D2}$.
\end{definition}
La junció de dues subsèries resolució no és commutativa degut a que si
s'uneixen dues subsèries resolució amb diferent $\delta$ i $f$ llavors
es determina que la primera marca quins són els $\delta'$ i $f'$
resultants.



L'operació de junció de dues sèries temporals multiresolució
és la unió de conjunts per a les subsèries resolució, ampliades amb un
multivalor indefinit, quan no intersequen en les claus $(\delta,f)$. En
cas que intersequin cal ajuntar les dues subsèries resolució.
\begin{definition}[Junció de dues sèries temporals multiresolució]
  Sigui $M_1=\{R_0^1,\dotsc,R_{d1}^1\}$ i
  $M_2=\{R_0^2,\dotsc,R_{d2}^2\}$ dues sèries temporals multiresolució
  i $r$ una funció de representació de sèries temporals, la junció de
  les dues $\glssymboldef{not:sgstm:joinM}^r(M_1, M_2)$ és una sèrie
  temporal multiresolució $M'=\{R_0',\dotsc,R_d'\}$ que conté les
  subsèries que no intersequen ampliades amb un multivalor i la junció
  de les subsèries que intersequen:
  $\glssymbol{not:sgstm:joinM}^r(M_1, M_2)= M_{a} \cup M'_{1} \cup
  M'_{2}$ a on $M_a = \{\forall R_1\in M_1,R_2\in M_2:
  \glssymbol{not:sgstm:joinR}(R_1,R_2) | (\delta_1,f_1) =
  (\delta_2,f_2) \}$, $M_1' = \{\forall R_1\in M_1:
  \glssymbol{not:sgstm:addMV}(R_1)| \nexists R_2 \in M_2 :
  (\delta_1,f_1) = (\delta_2,f_2) \}$ i $M_2' = \{\forall R_2\in M_2:
  \glssymbol{not:sgstm:addMV}(R_2)| \nexists R_1 \in M_1 :
  (\delta_1,f_1) = (\delta_2,f_2) \}$.
\end{definition}





\subsection{Consultes}


En els \glspl{SGBD} les consultes són les
operacions que permeten treballar amb les dades emmagatzemades. En el
cas dels \gls{SGSTM} les dades són sèries temporals emmagatzemades amb forma
de multiresolució. Així doncs, els operadors de consulta que es
defineixen a continuació permeten extreure les sèries temporals que hi
ha emmagatzemades amb l'objectiu d’aplicar-hi posteriorment els
operadors propis de les sèries temporals, com s'ha definit en el model
dels \gls{SGST}.

Els \glsdispsec{not:sgstm:consulta}{operadors de consulta} en un
\gls{SGSTM} es basen en obtenir sèries temporals de la base de dades
multiresolució per a aplicar-hi operacions dels \gls{SGST}.  En
aquest apartat definim el operadors que permeten obtenir sèries
temporals a partir d'una sèrie temporal multiresolució; l'aplicació
posterior d'operadors a les sèries temporals segueix les mateixes
definicions que en el model d'operacions dels \gls{SGST}
(v.~\autoref{sec:model:sgst-operacions}).

Distingim entre dos tipus d'operadors de consulta. Uns permeten
extreure les subsèries temporals de les subsèries resolució i uns altres
permeten abstreure una sèrie temporal resolució com si fos una sola
sèrie temporal amb diferents períodes de mostreig.


\subsubsection{Extracció de subsèries}


Les subsèries resolució estan caracteritzades per la parella de pas de
consolidació i funció d'agregació d'atributs $(\delta,f)$, ja que són
els atributs clau del conjunt. Seleccionant aquests dos paràmetres
d'una sèrie temporal multiresolució s'obté la subsèrie resolució
corresponent, de la qual es pot extreure la sèrie temporal
emmagatzemada al seu buffer o al seu disc. A continuació mostrem con
seleccionar la sèrie temporal del disc, per a la del buffer es pot
procedir de forma similar.

\begin{definition}[Selecció de la sèrie temporal d'un disc]
  Sigui $M=\{R_0,\dotsc,R_{d}\}$ una sèrie temporal multiresolució, on
  cada $R_i=(S_{Bi},S_{Di},\delta_i,\tau_i,k_i,f_i)$ és una subsèrie
  resolució, $\delta$ un pas de consolidació i $f$ una funció
  d'agregació d'atributs, la selecció de la sèrie temporal d'un disc
  de la sèrie temporal multiresolució és la sèrie temporal
  $\glssymboldef{not:sgstm:seriedisc}(M,\delta,f)= S_{Di}$ on
  $(S_{Bi},S_{Di},\delta,\tau_i,k_i,f) \in M$ essent $S_{Bi}$,
  $\tau_i$ i $k_i$ qualsevol valor.
\end{definition}




\subsubsection{Sèrie temporal total}

%Alerta: en el punt d'unió de les conatenacions queda mal unit perquè el valor de més resolució fa un span més gros del que li toca



La sèrie temporal total és la sèrie temporal que ofereix la màxima
resolució de la sèrie temporal multiresolució; és a dir que concatena
les subsèries resolució tenint en compte el pas de consolidació de
cada una. El resultat és una sèrie temporal amb períodes de mostreig
regulars a trossos.
\begin{definition}[Sèrie temporal total]
  Sigui $M^*=\{R_0,\dotsc,R_{d}\}$ una sèrie temporal multiresolució a
  on no hi ha $\delta$ repetits, la sèrie temporal total de la sèrie
  temporal multiresolució $\glssymboldef{not:sgstm:serietotal}(M^*)$
  és la sèrie temporal que resulta de la concatenació de les sèries
  temporals dels discs per ordre de pas de consolidació:
  $\glssymbol{not:sgstm:serietotal}(M^*)=S'$ a on $S'= S_{D0}
  \glssymbol{not:sgst:concatenate} S_{D1} \glssymbol{not:sgst:concatenate} S_{D2} \glssymbol{not:sgst:concatenate} \dotsb \glssymbol{not:sgst:concatenate}
  S_{Dd}$ complint que $\forall
  (S_{Bi},S_{Di},\delta_i,\tau_i,k_i,f_i) \in M^* : \delta_0 <
  \delta_1 < \delta_2 < \dots < \delta_d$.
\end{definition}

La sèrie temporal total s'ha definit a partir d'una sèrie temporal
multiresolució sense $\delta$ repetits. Com a conseqüència, els
$\delta_i$ pertanyents a la sèrie temporal multiresolució tenen un
ordre estricte i la sèrie temporal total resultant no és ambigua.
En el cas que una sèrie temporal multiresolució tingui $\delta$
repetits, prèviament a l'obtenció de la sèrie temporal total cal
decidir com seleccionar un conjunt únic de $\delta$. Proposem dos
exemples de selecció prèvia:
\begin{itemize}
\item Una sèrie temporal multiresolució és un conjunt amb $(\delta,f)$
  com a atributs clau i una selecció prèvia habitual pot ser la
  selecció de les subsèries resolució que comparteixin un determinat
  agregador d'atribut $f$.
\item Un altra selecció prèvia possible és utilitzar l'operació d'unió
  de subsèries resolució per a les subsèries que tinguin el mateix pas
  de consolidació $\delta$.
\end{itemize}



L'operació de sèrie temporal total definida és una operació genèrica
per a extreure una sèrie temporal amb la màxima resolució, però se'n
poden definir altres casos particulars. Per exemple, una operació de
sèrie temporal total on la concatenació sigui temporal: $S_{D0}
\glssymbol{not:sgst:concatenate}^r \dotsb
\glssymbol{not:sgst:concatenatet}^r S_{Dd}$; o una on es consulti una
subsèrie resolució en particular
$S_{Di}=\glssymboldef{not:sgstm:seriedisc}(M,\delta,f)$ i s'acabi de
completar la informació amb les dades d'altres resolucions: $S_{Di}
\glssymbol{not:sgst:concatenate}^r S_{D0}
\glssymbol{not:sgst:concatenate}^r \dotsb
\glssymbol{not:sgst:concatenatet}^r S_{Dd}$.


Així doncs, la sèrie temporal total és una abstracció d'una sèrie
temporal multiresolució en forma de sèrie temporal. Aleshores es poden
aplicar totes les operacions de les sèries temporals dels
\gls{SGST}. En mostrem dos exemples:
\begin{itemize}
\item Per a extreure una resolució determinada de la sèrie temporal
  emmagatzemada a una base de dades multiresolució, es consulta la
  sèrie temporal total i s'aplica una selecció de resolució:
  $\glssymbol{not:sgstm:serietotal}(M)[i]^r$ on $i$ és un conjunt
  d'instants de temps i $r$ la representació de la sèrie temporal.

\item Per a sumar dues sèries temporals emmagatzemades a una base de
  dades multiresolució, es consulta la sèrie temporal total de cada
  una i s'hi aplica l'operació computacional de sumar:
  $\glssymbol{not:sgstm:serietotal}(M_1)+\glssymbol{not:sgstm:serietotal}(M_2)$.

\end{itemize}






%%% Local Variables:
%%% TeX-master: "main"
%%% End:
% LocalWords:  SGSTM l'agregador buffer multiresolució subsèries
% LocalWords:  subsèrie

\lstMakeShortInline[style=py,basicstyle=\sffamily]{@}


\chapter{Implementació de referència}
\label{sec:implementacio:python}


La implementació de referència dels models de \gls{SGST} i de \gls{SGSTM}
es realitza amb Python \parencite{python:doc2}. L'objectiu d'aquesta
implementació de referència és mantenir la fidelitat al model per tal
de poder experimentar-hi amb tota la potència matemàtica.
Implementem els dos models de \gls{SGST} i \gls{SGSTM} com a dues biblioteques
diferents: \emph{Pytsms} i \emph{RoundRobinson} respectivament. La
RoundRobinson, però, està fortament relacionada amb la
Pytsms de la mateixa manera que hem definit els \gls{SGSTM} en base
als \gls{SGST}.


En la implementació amb Python utilitzem el paradigma de programació
d'orientació a objectes. Aquest paradigma permet definir objectes que
tenen atributs per a estructurar les dades i mètodes associats per a
manipular-les, cosa que ens permet relacionar de manera senzilla la
implementació amb el model.  Per a mostrar les relacions entre els
diversos objectes utilitzem diagrames de classe de \gls{UML}.




Implementem la part essencial dels models, és a dir l'àlgebra definida
en els models lògics. No implementem complements habituals dels
\gls{SGBD} que podrien ser necessaris en entorns d'explotació, com per
exemple gestió d'usuaris i permisos, còpies de seguretat, llenguatges
estàndards de consulta, etc.  Tampoc es tenen en compte paràmetres de
rendiment; per exemple en el cas de dues subsèries resolució que tenen
el mateix pas de consolidació i les funcions d'agregació d'atributs
tenen la mateixa funció de representació --per exemple
\glssymbol{not:sgstm:meanzohe} i \glssymbol{not:sgstm:maxzohe}--,
aquestes podrien compartir la mateixa operació de
selecció temporal d'interval \gls{zohe}.



\section{Pytsms}

La biblioteca Pytsms implementa un \gls{SGST} de referència. Així
doncs, seguint el model, els objectes principals són les mesures, les
sèries temporals i les representacions de les sèries temporals. Tots
tres s'implementen respectivament com a classes @Measure@,
@TimeSeries@ i @Representation@.


\begin{figure}[tp]
  \centering
  \begin{tikzpicture}

  %Timeseries
  \umlclass[x=0,y=0] {TimeSeries}{}{}  

  % Measure
  \umlclass[x=-4] {Measure}{}{}
  \umluniaggreg[mult=0..*]  {TimeSeries}{Measure}
  \umlclass[x=-5.2,y=-2] {MFloat}{}{}
  \umlclass[x=-2.8,y=-2] {MChar}{}{}
  \umlinherit {MFloat}{Measure}
  \umlinherit {MChar}{Measure}

  %Repr
  \umlclass[x=4] {Representation}{}{} %,type=abstract
  \umlassoc[mult1=1,mult2=1]  {TimeSeries}{Representation}
  \umlclass[x=3,y=-2] {Zohe}{}{}
  \umlclass[x=5,y=-2] {Delta}{}{}
  \umlinherit {Zohe}{Representation}
  \umlinherit {Delta}{Representation}

  %Associacions visitor
  \umlclass[x=-1.5,y=-4] {RegularProp}{}{}
  \umluniassoc  {TimeSeries}{RegularProp}
  \umlclass[x=1.5,y=-4] {Storage}{}{}
  \umluniassoc {TimeSeries}{Storage}

  %Dependencies
  \umlemptypackage[x=5,y=-5]{Matplotlib}
  \umldep{Zohe}{Matplotlib}
  \umldep{Delta}{Matplotlib}


  \end{tikzpicture}



  \caption{Diagrama \gls{UML} de Pytsms}
  \label{fig:implementacio:pytsms-uml}
\end{figure}





\begin{figure}
  \centering

\begin{tikzpicture}

  %Timeseries
  \umlclass[x=0,y=0] {TimeSeries}{}{}  
  %Realisations 
  \umlclass[x=-3.5,y=-3] {Structure}{}{}
  \umlclass[x=-1.2,y=-3] {OpSet}{}{}
  \umlclass[x=1.2,y=-3] {OpSeq}{}{}
  \umlclass[x=3.5,y=-3] {OpFunc}{}{}
  %\umlreal[geometry=|-|]{Structure}{TimeSeries}
  \umlinherit[geometry=|-|]{TimeSeries}{Structure}
  \umlinherit[geometry=|-|]{TimeSeries}{OpSet}
  \umlinherit[geometry=|-|]{TimeSeries}{OpSeq}
  \umlinherit[geometry=|-|]{TimeSeries}{OpFunc}
  %Subrealisations
  \umlclass[x=-3,y=-6] {SetNoTemporal}{}{}
  \umlclass[x=0.7,y=-6] {SetTemporal}{}{}
  \umlclass[x=4,y=-6] {SetRelacional}{}{}
  \umlinherit[geometry=|-|]{OpSet}{SetNoTemporal}
  \umlinherit[geometry=|-|]{OpSet}{SetTemporal}
  \umlinherit[geometry=|-|]{OpSet}{SetRelacional}
  \umlclass[x=-6,y=-6,type=python] {set}{}{}
  \umlinherit{Structure}{set}

\end{tikzpicture}

  \caption{Diagrama \gls{UML} de la realització de sèries temporals a Pytsms}
  \label{fig:implementacio:pytsms-uml-ts}
\end{figure}





La \autoref{fig:implementacio:pytsms-uml} mostra amb un diagrama
\gls{UML} la relació entre aquests tres objectes principals. Així, per una
banda, una @TimeSeries@ té una relació d'agregació amb les
@Measure@, és a dir que una sèrie temporal conté cap, una o més
d'una mesura.  Per altra banda, les sèries temporals i les
representacions són ortogonals i això s'implementa mitjançant una
relació d'associació bidireccional entre una @TimeSeries@ i una
@Representation@, és a dir que una instància de sèrie temporal té
associada una representació i una instància de representació coneix la
sèrie temporal que representa.




Una @TimeSeries@ és un objecte amb una gran quantitat de
mètodes. Com a conseqüència, la implementació de funcionalitats
essencials s'ha dividit en diversos objectes, els quals es mostren a
la \autoref{fig:implementacio:pytsms-uml-ts}. Els mètodes que
implementen el model estructural i el model d'operacions bàsiques
s'han agrupat en objectes segons la seva funcionalitat. Així hi ha
l'objecte @Structure@ que implementa el model estructural de les
sèries temporals, l'@OpSet@ pel model d'operacions de conjunts,
l'@OpSeq@ pel model d'operacions de seqüències i l'@OpFunc@
pel model d'operacions de funció temporal.  Aleshores l'objecte
@TimeSeries@ multihereta les funcionalitats d'aquests quatre
objectes, cosa que s'implementa a Python com a
\emph{Mixin} \parencite[\S 8.3.6, \S 20.17]{python:doc2}.  L'objecte
@OpSet@ també té una gran quantitat de mètodes i, de la mateixa
manera, hereta la funcionalitat de tres objectes: el
@SetNoTemporal@ per a les operacions basades en l'orde parcial de
les sèries temporals, el @SetTemporal@ basat en l'ordre temporal
i el @SetRelacional@ per a les operacions específiques de
l'àlgebra relacional. Pel que fa a l'@Structure@ hereta
funcionalitats dels objectes @set@, que són un tipus predefinit a
Python \parencite[\S 5.7]{python:doc2}.


Les funcionalitats complementàries de les @TimeSeries@ s'han
implementat amb relacions d'associació unidireccionals, les quals es
mostren a la \autoref{fig:implementacio:pytsms-uml}. Així, hi ha dues
funcionalitats complementàries: @RegularProp@ és un objecte que agrupa
les operacions relacionades amb la regularitat de les sèries temporals
i @Storage@ agrupa les operacions d'emmagatzematge i de recuperació en
fitxers. En aquests casos, l'associació unidireccional indica que les
@TimeSeries@ són objectes que admeten les operacions complementàries i
s'implementa a Python seguint el patró de disseny
\emph{Visitor} \parencite{ziade08:expert_python_programming:visitor,martin02:visitor}. Amb
aquest patró les @TimeSeries@ esdevenen \emph{Visitable}, és a dir
accepten objectes \emph{Visitor} que aporten funcionalitats
extres. Així doncs, els objectes @RegularProperties@ i @Storage@ són
\emph{Visitor} i les @TimeSeries@ tenen un mètode @accept@ que permet
acceptar-los.


La \autoref{fig:implementacio:pytsms-uml} mostra exemples
d'especialitzacions de les mesures i de les representacions.
%
Pel que fa a les @Measure@, poden tenir especialitzacions segons
els tipus dels atributs de temps i de valor. Amb aquesta relació
implementem la propietat homogènia de les sèries temporals i la
definició de mesura indefinida i de valor indefinit, és a dir que
totes les mesures que conté una sèrie temporal són del mateix tipus i
cada tipus de mesura té uns valors de l'atribut temps que la
defineixen indefinida i uns valors de l'atribut valor que la
defineixen de valor indefinit.  Així, per defecte, una @Measure@
defineix els reals $-\infty$ i $+\infty$ per a les mesures indefinida
negativa i positiva respectivament, i defineix el valor @None@ de
Python per a la mesura de valor indefinit. Aleshores, mitjançant
especialitzacions es poden definir altres tipus de mesures; per
exemple la @MFloat@ que defineix el real $\infty$ com a valor
indefinit o bé la @MChar@ que defineix mesures de tipus caràcter.




Pel que fa a les representacions, cada representació en concret és una
especialització de @Representation@. Per exemple @Zohe@ i
@Delta@ implementen la funció de representació \gls{zohe} i la delta
respectivament.
%\emph{Representation} és una classe abstracta?
Bàsicament, cada representació particular ha de definir l'operació que
calcula l'interval temporal i l'operació que permet trobar-ne el graf.
També cadascuna implementa una operació que dibuixi correctament el
gràfic de la sèrie temporal segons la representació; en les
representacions definides s'usa la biblioteca
\emph{Matplotlib} \parencite{python:matplotlib}  per a fer els gràfics.




% \subsubsection{Encaix a Python}

% \todo{}

% Explicar més detalladament quins mètodes tenen les sèries temporals: unió, selecció, etc.

% Heretem de sets, implementem els mètodes especials de sets

% implementem els mètodes especials de seqüències







\section{RoundRobinson}

La biblioteca RoundRobinson implementa un \gls{SGSTM} de
referència. Així doncs, seguint el model, els objectes principals són
les sèries temporals multiresolució, les subsèries resolució, els
buffers, els discs i les funcions d'agregació
d'atributs. Respectivament s'implementen com a classes
@MultiresolutionSeries@, @Resolution@, @Buffer@, @Disc@ i
@Function@ --les funcions d'agregació d'atributs són realitzades per
@Function@ de Python.


\begin{figure}[tp]
  \centering

\begin{tikzpicture}

  %MultiTimeseries
  \umlclass[x=0,y=0] {MultiresolutionSeries}{}{}  
  \umlclass[x=-4,y=0,type=python] {set}{}{}
  \umlinherit{MultiresolutionSeries}{set}
  %Components 
  \umlclass[x=0,y=-3] {Resolution}{}{}
  \umluniaggreg  {MultiresolutionSeries}{Resolution}
  %SubComponents 
  \umlclass[x=-1.2,y=-6] {Buffer}{}{}
  \umlclass[x=1.2,y=-6] {Disc}{}{}
  \begin{umlpackage}[x=-3,y=-9]{aggregators}
    \umlclass[template={s,i},type=interface] {Function}{}{}
  \end{umlpackage}
  \umlunicompo[mult=1]  {Resolution}{Buffer}
  \umlunicompo[mult=1]  {Resolution}{Disc}
  \umluniassoc[mult=1]  {Buffer}{Function}

  %Agregadors
  \umlclass[x=-4,y=-12] {max\_zohe}{}{}  
  \umlclass[x=-1,y=-12] {mean\_zohe}{}{}  
  \umlinherit{max\_zohe}{Function}
  \umlinherit{mean\_zohe}{Function}

  %TimeSeries
  \begin{umlpackage}[x=1,y=-9]{Pytsms}
    \umlclass{TimeSeries}{}{}  
  \end{umlpackage}
  \umluniassoc[mult=1]  {Buffer}{TimeSeries}
  \umluniassoc[mult=1]  {Disc}{TimeSeries}

  %Associacions visitor
  \umlclass[x=4,y=-3] {Storage}{}{}
  \umluniassoc {MultiresolutionSeries}{Storage}
  \umlclass[x=4,y=-1] {Plot}{}{}
  \umluniassoc {MultiresolutionSeries}{Plot}

  

\end{tikzpicture}

  \caption{Diagrama \gls{UML} de RoundRobinson}
  \label{fig:implementacio:roundrobinson-uml}
\end{figure}



La \autoref{fig:implementacio:roundrobinson-uml} mostra amb un
diagrama \gls{UML} la relació entre aquests cinc objectes principals. Així,
una @MultiresolutionSeries@ té una relació d'agregació amb les
@Resolution@, és a dir que una sèrie temporal multiresolució
conté subsèries resolucions.  Una @Resolution@ té una relació de
composició amb un @Buffer@ i una altra amb un @Disc@, és a
dir que cada subsèrie resolució està formada exactament per un buffer
i un disc. Cada @Buffer@ té una relació d'associació amb una
@TimeSeries@, és a dir amb la sèrie temporal del buffer; de
manera similar per la sèrie temporal del disc cada @Disc@
s'associa a una @TimeSeries@. A més, cada @Buffer@ també té
una relació d'associació amb una @Function@ que ha de tenir dos
paràmetres: la sèrie temporal (@s@) i l'interval de consolidació
(@i@).




Les @MultiresolutionSeries@ tenen funcionalitats complementàries
que s'han implementat amb relacions d'associació
unidireccionals. Així, hi ha dues funcionalitats complementàries:
@Plot@ per a les operacions relacionades amb la visualització
gràfica i @Storage@ per a operacions d'emmagatzematge i de
recuperació en fitxers. Aquests dos objectes també s'han implementat
amb patró de disseny \emph{Visitor} i les @MultiresolutionSeries@
són \emph{Visitable} de la mateixa manera que en el cas de les
funcionalitats complementàries de Pytsms.


Les @MultiresolutionSeries@ com a conjunts formats per
@Resolution@ s'han implementat heretant funcionalitats dels
@set@ de Python. Així, per a definir l'esquema multiresolució hi
ha un mètode @addResolution@ que permet afegir subsèries
resolució. Cadascuna es configura amb quatre paràmetres: @delta@,
@k@, @f@ i @tau@, els quals creen una @Resolution@
amb el @Buffer@ i el @Disc@ corresponent.

Les @MultiresolutionSeries@ tenen mètodes que operen amb totes les
@Resolution@ contingudes. Els més importants són el mètode @add@ per a
afegir noves mesures, el @consolidable@ per a determinar si alguna
subsèrie és @consolidable@ i el @consolidate@ que consolida totes les
consolidables.

Per a les consultes, hi ha dos mètodes: @discSeries@ i @total@. El
@discSeries@ té com a paràmetres @delta@ i @f@ i retorna la
@TimeSeries@ del @Disc@ corresponent. El @total@ retorna la
concatenació de tots els @discSeries@ possibles ordenats per @delta@;
com que no hi pot haver @delta@ repetits, el mètode @total@ té un
paràmetre @f@ que permet seleccionar només aquells @discSeries@ que
tenen unes determinades funcions d'agregació d'atributs.



En un mòdul @aggregators@ hi ha predefinides algunes funcions
d'agregació d'atributs com per exemple la @max_zohe@ o la
@mean_zohe@ que respectivament agreguen el màxim i la mitjana seguint
la representació \gls{zohe}.  Això no obstant, els usuaris en poden
definir de pròpies ja que són @Function@ de Python amb els paràmetres
@s@ una @TimeSeries@ i @i@ una parella d'instants de temps com ara
@[t0,tf]@ i que retornen una @Measure@. Per exemple la funció
@mean_zohe@ retorna una mesura amb @tf@ com a temps i com a valor
l'àrea mitjana a partir la selecció temporal \gls{zohe} de la sèrie
temporal en l'interval de temps donat.






\section{Exemples d'ús}


Amb les biblioteques Pytsms i RoundRobinson podem treballar amb les
sèries temporals i les sèries temporals multiresolució de manera molt
semblant als models algebraics de \gls{SGST} i de \gls{SGSTM}.




El \autoref{lst:pytsms:ex1} és un exemple d'ús de Pytsms on definim
dues sèries temporals $s_1$ i $s_2$ i hi apliquem diverses operacions:
unió, unió temporal, concatenació, interval tancat, interval temporal
\gls{zohe}, selecció temporal \gls{zohe}, comprovació de propietats de
regularitat i consultes de gràfics. Noteu que @Measure@ s'abrevia amb
@m@.

\begin{lstlisting}[style=py,caption=Exemple d'operacions amb Pytsms,label=lst:pytsms:ex1]
#Importació dels objectes necessaris
>>> from pytsms import TimeSeries, Measure as m
>>> from pytsms.representation import Zohe
>>> from pytsms.properties import isRegular

#Definició de les dues sèries temporals d'exemple
>>> s1 = TimeSeries([ m(1,1), m(3,1), m(4,0), m(5,1) ])
>>> s2 = TimeSeries([ m(2,2), m(3,2), m(4,0), m(6,2) ])

#Manipulacions de les dues sèries temporals
# æ$s_1 \glssymbol{not:sgst:cup} s_2$æ
>>> s1.union(s2)
TimeSeries([m(1,1), m(2,2), m(3,1), m(4,0), m(5,1), m(6,2)])
# æ$s_1 \glssymbol{not:sgst:cupt} s_2$æ
>>> s1.union_temporal(s2)
TimeSeries([m(1,1), m(2,2), m(4,0), m(5,1), m(6,2)])
# æ$s_1 \glssymbol{not:sgst:concatenate} s_2$æ
s1.concatenate(s2) 
TimeSeries([m(1,1), m(3,1), m(4,0), m(5,1), m(6,2)])
# æ$s_2[2,5]$æ
>>> s2.interval_closed(2,5)
TimeSeries([m(2,2), m(3,2), m(4,0)])
# æ$s_2[2,5]^{\glssymbol{not:zohe}}$æ
>>> s2.interval_temporal(2,5,Zohe)
TimeSeries([m(3,2), m(4,0), m(5,2)])

#Comprovació de la regularitat
# æ$s_2$æ no és regular
>>> s2.accept(isRegular())
False
# regularitzem æ$s_2$æ amb la selecció temporal æ$s_2[{0,2,4}]^{\glssymbol{not:zohe}}$æ
>>> r2 = s2.selection_temporal([0,2,4],Zohe)
>>> r2
TimeSeries([m(0,2), m(2,2), m(4,0)])
>>> r2.accept(isRegular())
True

Gràfic de la sèrie temporal æ$s_2$æ amb representació æ\gls{zohe}æ
>>> s2.plot(rpr=Zohe)
[æv.~\autoref{fig:python:plotzohe}æ]
\end{lstlisting}




\begin{figure}[tp]
  \centering
  \begin{tikzpicture}
    \begin{axis}[
        timeseries,
        ylabel={},
        xlabel={},
        ]
       \addplot[const plot mark right, blue,mark=*] coordinates {
         (2,2)
         (3,2)
         (4,0)
         (6,2)
       };
    \end{axis}
   \end{tikzpicture}
   \caption{Gràfic de la sèrie temporal d'exemple $s_2$ amb
     representació \gls{zohe}}
   \label{fig:python:plotzohe}
\end{figure}




Al~\autoref{lst:pytsms:ex1} es mostra l'ús de funcionalitats
complementàries de Pytsms, el qual s'ha dissenyat extensible per a
poder incorporar noves operacions, per a la regularitat.  En
el~\autoref{lst:pytsms:storage} cas es mostra l'ús d'operacions
d'emmagatzematge en fitxers, del mòdul @storage@. Primer s'emmagatzema
la @s2@ del~\autoref{lst:pytsms:ex1} en un fitxer amb format de
\gls{CSV}, en què cada línia és una parella de temps i valor separats
per una coma, i després es recupera del fitxer la sèrie temporal
emmagatzemada. Les dades emmagatzemades al fitxer es mostren
al~\autoref{lst:pytsms:csv}.
\begin{lstlisting}[style=py,caption=Operacions complementàries de Pytsms per a l'emmagatzematge,label=lst:pytsms:storage]
#Importació dels objectes necessaris
>>> from pytsms.storage import SaveCsv, LoadCsv

#Emmagatzematge en format æ\gls{CSV}æ de nom st2.csv
>>> s2.accept(SaveCsv('st2.csv'))
#Recuperació a partir de format æ\gls{CSV}æ
>>> sr = TimeSeries([])
>>> sr.accept(LoadCsv('st2.csv'))
TimeSeries([m(2,2), m(3,2), m(4,0), m(6,2)])
\end{lstlisting}

\begin{lstlisting}[style=file,caption=Dades del fitxer st2.csv,label=lst:pytsms:csv]
2,2
3,2
4,0
6,2
\end{lstlisting}






El \autoref{lst:roundrobinson:ex1} és un exemple d'ús de RoundRobinson
on definim una sèrie temporal multiresolució $M$, hi afegim una sèrie
temporal $s$, hi apliquem la consolidació fins que no sigui
consolidable. Finalment, consultem el resultat amb les dues consultes
bàsiques --$\glssymbol{not:sgstm:seriedisc}$ i
$\glssymbol{not:sgstm:serietotal}$-- i ho dibuixem gràficament.  Les
dades són les mateixes que a \textref{ex:model:bdm1} i també la
consolidació resultant, tot i que ara les sèries temporals tenen
valors diferents de zero per a poder observar els resultats de les
funcions d'agregació d'atributs.



\begin{lstlisting}[style=py,caption=Exemple d'operacions amb RoundRobinson,label=lst:roundrobinson:ex1]
#Importació dels objectes necessaris
>>> from pytsms import TimeSeries, Measure as m
>>> from roundrobinson import MultiresolutionSeries
>>> from roundrobinson.aggregators import mean_zohe,maximum_zohe
>>> from roundrobinson.plot import Plot
>>> from pytsms.representation import Zohe

#Definició de la sèrie temporal d'exemple
>>> s = TimeSeries([m(1,6),m(5,2),m(8,5),m(10,0),m(14,1),m(19,6),m(22,11),m(26,6),m(29,0)])

#Definició de la sèrie temporal multiresolució
>>> M = MultiresolutionSeries()
#Definició de l'esquema multiresolució
>>> M.addResolution(delta=5,k=4,f=mean_zohe,tau=0)
>>> M.addResolution(delta=10,k=3,f=maximum_zohe,tau=0)

#Addició de totes les mesures de la sèrie temporal
>>> for m in s: M.add(m)
#M ja és consolidable
>>> M.consolidable()
True
#Consolidació fins que no sigui consolidable
>>> while M.consolidable(): M.consolidate()

#Consulta æ$\glssymbol{not:sgstm:seriedisc}(M,5,\glssymbol{not:sgstm:meanzohe})$æ
>>> M.discSeries(5,mean_zohe)
TimeSeries([m(10,3), m(15,2), m(20,7), m(25,8)])
#Consulta æ$\glssymbol{not:sgstm:seriedisc}(M,10,\glssymbol{not:sgstm:maxzohe})$æ
>>> M.discSeries(10,maximum_zohe)
TimeSeries([m(10,6), m(20,11)])
#Consulta æ$\glssymbol{not:sgstm:serietotal}(M)$æ
>>> M.total()
TimeSeries([m(10,3), m(15,2), m(20,7), m(25,8)])
#Gràfic multiresolució
>>> M.accept(Plot())
[æv.~\autoref{fig:python:BDMm}æ]
#Gràfic Pytsms de la sèrie temporal total amb representació æ\gls{zohe}æ
>>> M.total().plot(rpr=Zohe)
[æv.~\autoref{fig:python:BDMmtotal}æ]
\end{lstlisting}
%from pytsms import TimeSeries, Measure; from roundrobinson import MultiresolutionSeries; from aggregators import mean_zohe,maximum_zohe; from plot import Plot
%s = TimeSeries([Measure(1,6),Measure(5,2),Measure(8,5),Measure(10,0),Measure(14,1),Measure(19,6),Measure(22,11),Measure(26,6),Measure(29,0)]); M = MultiresolutionSeries(); M.addResolution(delta=5,k=4,f=mean_zohe,tau=0); M.addResolution(delta=10,k=2,f=maximum_zohe,tau=0)



\begin{figure}[tp]
  \centering
  \begin{tikzpicture}
    \begin{axis}[
        multiresolucio,
        title={R: 5/mean\_zohe $|4|$},
        ylabel={},
        xlabel={},
        ]
       \addplot[const plot mark right, blue,mark=*] coordinates {
         (10,3)
         (15,2)
         (20,7)
         (25,8)
       };
    \end{axis}
   \end{tikzpicture}
  \begin{tikzpicture}
    \begin{axis}[
        multiresolucio,
        title={R: 10/maximum\_zohe $|3|$},
        ylabel={},
        xlabel={},
        ]
       \addplot[const plot mark right, blue,mark=*] coordinates {
         (10,6)
         (20,11)
       };
    \end{axis}
   \end{tikzpicture}
   \caption{Base de dades multiresolució $M$}
   \label{fig:python:BDMm}
\end{figure}

\begin{figure}[tp]
  \centering
  \begin{tikzpicture}
    \begin{axis}[
        timeseries,
        ylabel={},
        xlabel={},
        ]
       \addplot[const plot mark right, blue,mark=*] coordinates {
         (10,3)
         (15,2)
         (20,7)
         (25,8)
       };
    \end{axis}
   \end{tikzpicture}
   \caption{$\glssymbol{not:sgstm:serietotal}(M)$ amb representació \gls{zohe}}
   \label{fig:python:BDMmtotal}
\end{figure}



Al~\autoref{lst:roundrobinson:storage} es mostra l'ús de
funcionalitats complementàries de RoundRobinson, el qual s'ha
dissenyat extensible per a poder incorporar noves operacions, per als
gràfics.  En el~\autoref{lst:roundrobinson:storage} es mostra l'ús
d'operacions d'emmagatzematge en fitxers, del mòdul @storage@. Primer
s'emmagatzema la @M@ del~\autoref{lst:roundrobinson:storage} en un
fitxer amb format Pickle, que és un format d'emmagatzematge de
Python \parencite{python:doc2}, i després es recupera del fitxer la
sèrie temporal multiresolució emmagatzemada.  Segon s'emmagatzema la
@M@ del~\autoref{lst:roundrobinson:storage} en un fitxer amb format
\gls{CSV} i després es recupera del fitxer la sèrie temporal
multiresolució emmagatzemada.  Les dades emmagatzemades al fitxer \gls{CSV} es
mostren al~\autoref{lst:roundrobinson:csv}.

\begin{lstlisting}[style=py,caption=Operacions complementàries de RoundRobinson per a l'emmagatzematge,label=lst:roundrobinson:storage]
#Importació dels objectes necessaris
>>> from roundrobinson.storage import SavePickle, LoadPickle, SaveCsv, LoadCsv

#Emmagatzematge en format Pickle de nom mrd.pickle
>>> M.accept(SavePickle('mrd.pickle'))
#Recuperació a partir de format Pickle
>>> Mr = MultiresolutionSeries([])
>>> Mr = Mr.accept(LoadPickle('mrd.pickle'))
>>> Mr == M
True

#Emmagatzematge en format æ\gls{CSV}æ de nom mrd.csv
>>> M.accept(SaveCsv('mrd.csv'))
#Recuperació a partir de format æ\gls{CSV}æ
>>> Mr = M.accept(LoadCsv('mrd.csv'))
\end{lstlisting}


\begin{lstlisting}[style=file,caption=Dades del fitxer mrd.csv,label=lst:roundrobinson:csv]
5 mean_zohe 25 8
5 mean_zohe 10 3
5 mean_zohe 20 7
5 mean_zohe 15 2
10 maximum_zohe 10 6
10 maximum_zohe 20 11
\end{lstlisting}


Cal notar que en el cas de l'emmagatzematge en \gls{CSV} no es tenen
en compte els buffers, per tant la recuperació no és estrictament el
mateix que la sèrie temporal multiresolució original.




% \todo{}
% Un cas d'exemple és demostrar l'equivalència entre l'operació multiresolució dels SGST sobre una sèrie temporal i la sèrie temporal resultant d'una consolidació dels SGSTM. 



% def MTSMSequivalenceTSMS(s,schema):
%     """
%     >>> from pytsms.consult import multiresolution
%     >>> s = TimeSeries([Measure(5,5),Measure(11,1),Measure(12,2),Measure(16,1),Measure(21,1),Measure(26,1)])
%     >>> def _max(s,i): sp=s[i[0]:i[1]]; return Measure(i[1], None if len(sp)==0 else max(sp.projection('v')))
%     >>> schema = [(5,4,_max,10),(10,3,_max,0)]
%     >>> MTSMSequivalenceTSMS(s,schema)
%     True
%     """
%     from pytsms.consult import multiresolution
    
%     schemats = TimeSeries()
%     m = MultiresolutionSeries()
%     for delta,k,f,tau in schema:
%         m.addResolution(delta,k,f,tau)
%         schemats.add(Measure(delta,(tau,f,k)))
    
%     m.update(s)
%     m.consolidateTotal()
    
%     return multiresolution(s,schemats) == m.total()




\lstDeleteShortInline{@}
%%% Local Variables:
%%% TeX-master: "main"
%%% End:
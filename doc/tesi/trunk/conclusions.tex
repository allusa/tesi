\chapter{Conclusions}
\label{sec:conclusions}


\todo{fer}

En aquest capítol es resumeix l'exposat en el cos principal del
document i s'extreuen conclusions del model dissenyat i de les
implementacions.


1. Resumim els models formalitzats i la tècnica de multiresolució

2. Resumim les variacions fetes a partir dels models

3. Implementacions




\section{Resum dels models}
%Resum del model de multiresolució


El principal objectiu de la multiresolució és l'emmagatzematge
comprimit de sèries temporals. Així, la multiresolució es
contextualitza en l'àmbit dels \gls{SGBD}, per la qual cosa hem
formalitzat el model de dades dels \gls{SGSTM}, els sistemes que
gestionen sèries temporals amb la tècnica de multiresolució. Els
\gls{SGSTM} basen el tractament de les sèries temporals en els
\gls{SGST}, per la qual cosa també hem descrit el model d'aquests
sistemes. El principal objectiu dels \gls{SGST} és gestionar les
sèries temporals de manera coherent a la dimensió temporal.




La multiresolució emmagatzema una sèrie temporal mitjançant diversos
resums d'atributs i resolucions.  Breument, el model de \gls{SGSTM}
s'estructura a partir de \emph{sèries temporals multiresolució} com a
conjunt de \emph{subsèries resolució}, les quals acumulen temporalment
les mesures en un \emph{buffer} per tal de tractar-les abans
d'emmagatzemar-les a un \emph{disc}. El tractament principal
consisteix a canviar els intervals de temps entre mesures i a
agregar-ne atributs, amb l'objectiu de compactar la informació de la
sèrie temporal.  Així, cada sèrie temporal multiresolució té diferents
paràmetres per a configurar de quina manera s'ha de resumir la
informació i calcular les resolucions. És el que anomenem
\emph{esquema de multiresolució} i consisteix en definir la quantitat
de subsèries resolució i quatre paràmetres per a cadascuna: el pas de
consolidació, l'instant de temps d'inici de la consolidació, la funció
d'agregació d'atributs i la capacitat d'emmagatzematge.



El model de \gls{SGSTM} també inclou les operacions que formalitzen el
comportament d'aquests sistemes. En primer lloc, per a emmagatzemar
una sèrie temporals multiresolució és indispensable que hi hagi
operacions per a afegir mesures i per a consolidar-les. En segon lloc,
hi ha operacions per a manipular l'esquema de multiresolució i per a
observar-ne propietats. En tercer lloc, per a consultar les dades emmagatzemades  hi ha dues operacions bàsiques: obtenir 


També les manipulacions d'esquema (sobretot com observar les dades d'un esquema) i les consultes: dues de bàsiques



Un apartat específic
En el model de dades multiresolució es defineixen els operadors específics per a aquestes tasques anomenats \emph{agregadors d'atributs}
com interpretar les agregacions, hem introduït el problema de cooperar amb la validació de dades.

% We have showed some aggregation functions examples with simple
% aggregation statistics, mean and maximum, and simple representation
% methods, Delta and \zohe{}. More attribute aggregation functions could
% be designed based on methods from other fields such as data streaming
% or time series data mining, especially it would be interesting aggregations with uncertain data.


Pel que fa als \gls{SGST}

We have gone a bit further and proposed \acro{TSMS} including
 set, sequence and temporal function behaviour.

Hem observat propietats problemàtiques, naturalesa
la multiresolució ha de cooperar en els propietats problemàtiques dels \gls{SGST}: regularitat, gran volum de dades, etc.




\section{Consideracions}

* La multiresolució és una selecció de la informació de la sèrie temporal. Per tant és un emmagatzematge amb pèrdua. Això vol dir que l'usuari ha d'escollir un esquema de multiresolució adequat al context en què vulgui treballar.


 These configuration parameters
% are degrees of freedom for each application. Giving different values a
% multiresolution database is capable to keep the desired information
% from a time series. %



% The queries over \acro{MTSMS} obtain time series from stored
% multiresolution time series. In this way \acro{TSMS} operators can be
% applied if needed. The $\seriedisc$ time series being regular
% facilitates these operations. However, the lossy storage implies that
% some operations will give approximate queries and that not every
% \acro{TSMS} operation will be semantically correct for a
% multiresolution time series. Therefore the correct planning of the
% multiresolution schema is needed.


% Compared to other \acro{TSMS} we propose a compression solution that
% stores only the information we will require by latter queries or by
% human visualisation, instead of trying to reconstruct the original
% signal.  Moreover, our multiresolution solution copes well with
% typical problematic properties of time series: regularity, data
% validation and data volume.  The decompression time is minimal as data
% in discs get stored directly as a time series. As a consequence, the
% queries or visualisation computing time is only due to the computation
% itself. Moreover, if the query is an aggregation or resolution already
% computed in \acro{MTSMS} consolidation, then the visualisation is
% immediate.


% \acro{MTSMS} imply a data information selection and so the information
% not considered important is discarded. 
% % When this is not possible, we
% % have showed a dual structure of \acro{TSMS} and \acro{MTSMS}. Then a
% % \acro{TSMS} stores losslessly and a \acro{MTSMS} takes advantages of
% % manipulating data in time order in order to achieve pre-computed
% % queries in a stream-like orientation. 
% In future work, information
% theory has to be evaluated for multiresolution schemes. As multimedia
% lossy compression techniques are well founded on information theory,
% similar approaches could be taken for multiresolution time series,
% e.g. evaluating whether a human can visualise original qualities in
% the multiresoluted time series or evaluating
% whether given a query it has the same validity for a multiresoluted
% one as it has for the original.




% A \acro{MTSMS} could be implemented as a SQL \acro{DBMS} system or as
% a NoSQL one. As a referent implementation we have developed a
% \emph{Python} package centred on the basic algebra, that is without
% extended \acro{DBMS} capabilities. Regarding other implementations,
% \emph{RRDtool} can be seen as an specific case of \acro{MTSMS} and as
% a NoSQL system, although Oetiker \cite{rrdtool} has not commented
% it. However, regardless of the implementation backend, we have shown
% how a generic model for \acro{MTSMS} can be defined firmly rooted on
% \acro{DBMS} algebra theory.





% \subsubsection{Resum SGSTM?}

% Aquest capítol s'acaba amb un resum dels conceptes exposats en el
% model de dades. Una base de dades per sèries temporals multiresolució
% és un sistema informàtic d'emmagatzematge d'una sèrie temporal entesa
% com una una co\l.lecció de dades mesurades en diferents instants de
% temps.

% A la base de dades, la sèrie temporal queda estructurada com s'ha esquematitzat a  la figura~\ref{fig:model:bdstm}. És una forma compacta d'emmagatzemar la sèrie temporal de manera que queda repartida segons diferents funcions d'interpolació i períodes de mostreig. Aquest repartiment té lloc en els diferents discs resolució, els quals fan ús del seu buffer per interpolar les mesures i fan ús del seu disc per consolidar-les. 

% El conjunt de discs resolució constitueixen la part principal d'una base de dades multiresolució tot i que hi pot haver variacions en aquest esquema, com per exemple un buffer d'entrada de mesures comú que regularitzi la sèrie temporal des d'un principi i simplifiqui els interpoladors que són complicats quan es fa el pas de sèrie temporal no regular a regular.


% En el capítol \todo{? més endavant}
%  utilitzant el llenguatge de programació Python es dissenya, a nivell acadèmic, un sistema de gestió de bases de dades que implementa el model de dades tal com s'ha definit en aquest capítol.


% En resum, a partir del model de dades multiresolució descrit en aquest capítol per una banda es poden estudiar quin efecte té una configuració determinada de paràmetres i per altra banda es poden dissenyar sistemes de gestió de bases de dades assegurant que implementen el model i per tant que tenen el funcionament desitjat.










% \subsubsection{Què fer sense coneixements a priori}

% Una base de dades multiresolució requereix tenir un coneixement de l'entorn a priori per a poder establir-ne l'esquema de multiresolució. Perquè un cop establit aquest esquema només s'emmagatzemen els atributs seleccionats i es perd informació sobre la sèrie temporal original.

% Per això si la informació és crítica una bona estructura de base de dades consistiria en un magatzem total de la informació recollida i un magatzem multiresolució amb un esquema inicial. La base de dades multiresolució s'utilitzaria per a les consultes habituals que s'haguessin de resoldre de forma ràpida, en cas que les respostes no fossin suficients es podria anar a buscar la informació al magatzem total, a on la resolució de la consulta tindria un temps més elevat. 
% Aleshores si aquestes consultes esdevinguessin habituals es podria definir un nou esquema de multiresolució i iniciar-lo amb les dades del magatzem total (això tardaria un cert temps) per a després executar-hi les consultes de forma ràpida.


% Tot i així cal notar que en moltes aplicacions les dades històriques
% són prescindibles i es pot canviar l'esquema de multiresolució sense
% gaires preocupacions. Per exemple un sistema de monitoratge de la
% bateria que tenim disponible al portàtil.

% També en altres aplicacions el que volem es resoldre una consulta del tipus la mitjana puja o baixa. En això el model de multiresolució hi encaixa molt bé ja que es base en calcular agregacions i després treballar sobre aquestes. 



% \subsubsection{Arquitectura RRDtool}


% RRDtool té una estructura multiresolució amb un buffer únic d'entrada
% i buffers orientats a stream; segons havíem avaluat anteriorment \parencite{llusa11:tfm}.


% S'ha d'estudiar com es fan les consultes a RRDtool

% \url{http://en.wikipedia.org/wiki/RRD_Editor}



% Podem considerar que:

% 1. RRDtool és un SGBD NoSQL?
% 2. Nosaltres n'hem formalitzat un model lògic?
% 3. És el primer model lògic per a un producte NoSQL?
% 4. Aquest model lògic es pot implementar tant en productes relacionals com amb NoSQL? i per tant es demostra que els models lògics són extremadament potents i necessaris?
% 5. La implementació que fa RRDtool és molt eficient per a un determinat camp d'aplicació?
% 6. La implementació relacional seria molt genèrica i propera al model però no tan eficient? més aviat subjecte a l'eficiència genèrica dels SGBDR?
% 7. Els SGST són uns SGBD més simples? no tenen tantes actualitzacions de valors, no hi ha tantes relationships en l'esquema... Els SGST només es preocupen de sèries temporals i per tant només d'un tipus de dades en concret, això no obstant tal com s'ha dissenyat el model aquest tipus de dades es pot implementar en SGBD més complexos. 


% \subsubsection{Rapidesa/eficiència dels SGSTM}

% Aquí només hem definit el nivell lògic i les implementacions que volem
% fer només són per exemplificar el model lògic; és a dir que les farem
% tan properes al model lògic com es pugui. La rapidesa/eficiència dels
% SGSTM només la podríem avaluar a les implementacions; a on es podrien
% aplicar models d'implementacions que se saben eficients. Com que no és
% el cas, no té sentit avaluar ni comparar l'eficiència de les nostres
% implementacions amb altres de semblants.




% \subsubsection{Comparació de SGST i SGSTM amb altres models del mercat}


% SGST:

% * Respecte dels models de seqüència: 

%   - una seqüència es defineix com una funció el domini de la qual és un conjunt comptable i totalment ordenat. La definició genèrica que fem del temps no és un conjunt comptable i per tant creiem que un model genèric de sèries temporals no es pot descriure amb comoditat des de les seqüències. 

%   - generalment s'assumeix que la distància entre els elements d'una seqüència és regular

%   - els instants de temps en una sèrie temporal no es poden canviar, en canvi en una seqüencia els índexs només marquen l'ordre i no significat de posicionament en un marc de referència.






% \subsubsection{Lapses de buffer o no lapses}

% Podem tenir esquemes de multiresolució a on les diferents subsèries
% resolució coincideixin en els temps recents o a on no coincideixin:
% les subsèries més velles acabin on comencen les noves de manera
% semblant a l'estructura de buffers enllaçats.

% La primera opció pot servir per quan hi ha moltes dades tenir diferents resums preparats per a ser visualitzats, així permet triar ràpidament entre diferents zooms de les dades.

% La segona opció serveix per aprofitar al màxim la resolució i l'espai d'emmagatzematge, sense que cap subsèrie desi informació per al mateix interval de temps. Així permet conservar una sèrie temporal al llarg del seu temps amb diferents resolucions. També pot servir per usar la informació d'altres buffers i no haver de repetir emmagatzematge de buffers.








% * Sobre la reflexió de la informació:
% L'anàlisi que formulem és una introducció a la reflexió sobre l'error
% en la informació de la multiresolució. Així, de forma simple,
% analitzem si hi ha error o si no n'hi ha, sense pretendre
% quantificar-lo. A més, ho analitzem en base a l'esquema de
% multiresolució que s'utilitzi, particularment de quines funcions
% d'agregació d'atributs s'utilitzin i de com siguin les consultes
% posteriors.

% En el cas que es conegui més bé el context i el comportament de la
% sèrie temporal a què s'aplica la multiresolució, es pot detallar més
% bé la quantificació de l'error: és a dir en termes de la teoria de
% la informació aleshores tenim més coneixement sobre la predicció del
% comportament de les dades i podem utilitzar-ho per a avaluar
% característiques més concretes.

% Cal notar que no avaluem la idoneïtat d'aplicar un estadístic o un
% altre a unes dades ni quin és el que més bé va per a obtenir una
% informació, només plantegem el cas que algú vol aplicar una consulta
% amb una agregació determinada a una sèrie temporal i quin error
% tindria si en comptes de a les dades originals ho aplica a les dades
% multiresolucionades.




%%% Local Variables:
%%% TeX-master: "main"
%%% End:
% LocalWords:  SGSTM multiresolució subsèries

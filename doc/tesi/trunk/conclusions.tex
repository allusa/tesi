\chapter{Conclusions}
\label{sec:conclusions}


\todo{fer}



% \subsubsection{Resum del model de multiresolució}
% El model de dades multiresolució s'estructura a partir de \emph{sèries temporals multiresolució} com a conjunt de \emph{subsèries resolució}, les quals  acumulen temporalment les mesures en un \emph{buffer} per tal de tractar-les abans d'emmagatzemar-les  a un \emph{disc}. El tractament principal consisteix en canviar els intervals de temps entre mesures amb l'objectiu de compactar la informació de la sèrie temporal.
% Així doncs, la sèrie temporal queda emmagatzemada com una sèrie temporal multiresolució en intervals de temps diferents, repartits en les subsèries resolució. 

% Pel que fa a les operacions, és indispensable que el model multiresolució pugui fer aquests canvis d'intervals de temps, els quals s'aconsegueixen amb les operacions d'\emph{agregació} i \emph{consolidació}. En el model de dades multiresolució es defineixen els operadors específics per a aquestes tasques anomenats \emph{agregadors d'atributs}



% \subsubsection{Resum SGSTM?}

% Aquest capítol s'acaba amb un resum dels conceptes exposats en el
% model de dades. Una base de dades per sèries temporals multiresolució
% és un sistema informàtic d'emmagatzematge d'una sèrie temporal entesa
% com una una co\l.lecció de dades mesurades en diferents instants de
% temps.

% A la base de dades, la sèrie temporal queda estructurada com s'ha esquematitzat a  la figura~\ref{fig:model:bdstm}. És una forma compacta d'emmagatzemar la sèrie temporal de manera que queda repartida segons diferents funcions d'interpolació i períodes de mostreig. Aquest repartiment té lloc en els diferents discs resolució, els quals fan ús del seu buffer per interpolar les mesures i fan ús del seu disc per consolidar-les. 

% El conjunt de discs resolució constitueixen la part principal d'una base de dades multiresolució tot i que hi pot haver variacions en aquest esquema, com per exemple un buffer d'entrada de mesures comú que regularitzi la sèrie temporal des d'un principi i simplifiqui els interpoladors que són complicats quan es fa el pas de sèrie temporal no regular a regular.


% En el capítol \todo{? més endavant}
%  utilitzant el llenguatge de programació Python es dissenya, a nivell acadèmic, un sistema de gestió de bases de dades que implementa el model de dades tal com s'ha definit en aquest capítol.


% En resum, a partir del model de dades multiresolució descrit en aquest capítol per una banda es poden estudiar quin efecte té una configuració determinada de paràmetres i per altra banda es poden dissenyar sistemes de gestió de bases de dades assegurant que implementen el model i per tant que tenen el funcionament desitjat.










% \subsubsection{Què fer sense coneixements a priori}

% Una base de dades multiresolució requereix tenir un coneixement de l'entorn a priori per a poder establir-ne l'esquema de multiresolució. Perquè un cop establit aquest esquema només s'emmagatzemen els atributs seleccionats i es perd informació sobre la sèrie temporal original.

% Per això si la informació és crítica una bona estructura de base de dades consistiria en un magatzem total de la informació recollida i un magatzem multiresolució amb un esquema inicial. La base de dades multiresolució s'utilitzaria per a les consultes habituals que s'haguessin de resoldre de forma ràpida, en cas que les respostes no fossin suficients es podria anar a buscar la informació al magatzem total, a on la resolució de la consulta tindria un temps més elevat. 
% Aleshores si aquestes consultes esdevinguessin habituals es podria definir un nou esquema de multiresolució i iniciar-lo amb les dades del magatzem total (això tardaria un cert temps) per a després executar-hi les consultes de forma ràpida.


% Tot i així cal notar que en moltes aplicacions les dades històriques
% són prescindibles i es pot canviar l'esquema de multiresolució sense
% gaires preocupacions. Per exemple un sistema de monitoratge de la
% bateria que tenim disponible al portàtil.

% També en altres aplicacions el que volem es resoldre una consulta del tipus la mitjana puja o baixa. En això el model de multiresolució hi encaixa molt bé ja que es base en calcular agregacions i després treballar sobre aquestes. 



% \subsubsection{Arquitectura RRDtool}


% RRDtool té una estructura multiresolució amb un buffer únic d'entrada
% i buffers orientats a stream; segons havíem avaluat anteriorment \parencite{llusa11:tfm}.


% S'ha d'estudiar com es fan les consultes a RRDtool

% \url{http://en.wikipedia.org/wiki/RRD_Editor}



% Podem considerar que:

% 1. RRDtool és un SGBD NoSQL?
% 2. Nosaltres n'hem formalitzat un model lògic?
% 3. És el primer model lògic per a un producte NoSQL?
% 4. Aquest model lògic es pot implementar tant en productes relacionals com amb NoSQL? i per tant es demostra que els models lògics són extremadament potents i necessaris?
% 5. La implementació que fa RRDtool és molt eficient per a un determinat camp d'aplicació?
% 6. La implementació relacional seria molt genèrica i propera al model però no tan eficient? més aviat subjecte a l'eficiència genèrica dels SGBDR?
% 7. Els SGST són uns SGBD més simples? no tenen tantes actualitzacions de valors, no hi ha tantes relationships en l'esquema... Els SGST només es preocupen de sèries temporals i per tant només d'un tipus de dades en concret, això no obstant tal com s'ha dissenyat el model aquest tipus de dades es pot implementar en SGBD més complexos. 


% \subsubsection{Rapidesa/eficiència dels SGSTM}

% Aquí només hem definit el nivell lògic i les implementacions que volem
% fer només són per exemplificar el model lògic; és a dir que les farem
% tan properes al model lògic com es pugui. La rapidesa/eficiència dels
% SGSTM només la podríem avaluar a les implementacions; a on es podrien
% aplicar models d'implementacions que se saben eficients. Com que no és
% el cas, no té sentit avaluar ni comparar l'eficiència de les nostres
% implementacions amb altres de semblants.




% \subsubsection{Comparació de SGST i SGSTM amb altres models del mercat}


% SGST:

% * Respecte dels models de seqüència: 

%   - una seqüència es defineix com una funció el domini de la qual és un conjunt comptable i totalment ordenat. La definició genèrica que fem del temps no és un conjunt comptable i per tant creiem que un model genèric de sèries temporals no es pot descriure amb comoditat des de les seqüències. 

%   - generalment s'assumeix que la distància entre els elements d'una seqüència és regular

%   - els instants de temps en una sèrie temporal no es poden canviar, en canvi en una seqüencia els índexs només marquen l'ordre i no significat de posicionament en un marc de referència.






% \subsubsection{Lapses de buffer o no lapses}

% Podem tenir esquemes de multiresolució a on les diferents subsèries
% resolució coincideixin en els temps recents o a on no coincideixin:
% les subsèries més velles acabin on comencen les noves de manera
% semblant a l'estructura de buffers enllaçats.

% La primera opció pot servir per quan hi ha moltes dades tenir diferents resums preparats per a ser visualitzats, així permet triar ràpidament entre diferents zooms de les dades.

% La segona opció serveix per aprofitar al màxim la resolució i l'espai d'emmagatzematge, sense que cap subsèrie desi informació per al mateix interval de temps. Així permet conservar una sèrie temporal al llarg del seu temps amb diferents resolucions. També pot servir per usar la informació d'altres buffers i no haver de repetir emmagatzematge de buffers.









%%% Local Variables:
%%% TeX-master: "main"
%%% End:
% LocalWords:  SGSTM

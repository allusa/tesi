\chapter{Conclusions}
\label{sec:conclusions}


\todo{fer}

En aquest capítol es resumeix l'exposat en el cos principal del
document i s'extreuen conclusions del model dissenyat i de les
implementacions.


\todo{hauríem de contestar a les preguntes de la intro}
``Però, cal emmagatzemar totes les dades que aquests sistemes de
monitoratge són capaços de recollir?, cal emmagatzemar-les
indefinidament en el temps? O bé es pot emmagatzemar només la
informació que es desitgi monitorar o controlar i rebutjar la resta?,
de manera que no calgui gestionar un volum de dades massiu que
realment no s'utilitzarà? En aquesta tesi dissertem sobre un mètode
capaç de seleccionar i emmagatzemar una determinada informació de les
dades, com gestionar aquesta informació i fins a quin límit això es pot
aplicar. L'anomenem multiresolució.''

``Contextualitzem la multiresolució en dos àmbits: les sèries temporals
i els \glspl{SGBD}. En aquesta tesi unim conceptes d'aquest dos àmbits
i proposem solucions per alguns dels problemes que en resulten.''


Destacar que no només volem calcular ràpid sinó també tenir en compte altres recursos: capacitat, energia, comunicacions, etc.


\todo{}
* La multiresolució és un càlcul de diferents agregacions i resolucions d'una sèrie temporal. És a dir, el que podria ser una consulta de multiresolució que voldríem fer a unes dades per exemple veure'n la mitjana de cada mes al llarg d'un any i veure la mitjana del darrer dia cada 5 minuts. Això és costós de calcular però nosaltres plantegem alternatives a aquest càlcul. En el model, però, es tracta de definir el problema mitjançant l'àlgebra, de fer una bona definició senzilla i compacta del sistema; l'àlgebra en el nivell lògic dels models de SGBD no coneix els problemes de rendiment ni de recursos limitats, per això no hi apareixen específicament aquestes qüestions, de fet és un avantatge ja que permet definir el model de totalment genèrica i molt potent.

* Tenen problemes les sèries temporals? Sí, ja hem analitzat algunes de les seves propietats problemàtiques, sobretot el gran volum que tenen i al qual s'ocupa de trobar algoritmes per a solucionar-ho el data mining. Ara bé, això no treu que se segueixin gestionant amb SGBD, nosaltres proposem diferents implementacions que intenten gestionar les propietats problemàtiques.


1. Resumim els models formalitzats i la tècnica de multiresolució

2. Resumim les consideracions i variacions fetes a partir dels models

3. Implementacions




% Compared to other \acro{TSMS} we propose a compression solution that
% stores only the information we will require by latter queries or by
% human visualisation, instead of trying to reconstruct the original
% signal.  Moreover, our multiresolution solution copes well with
% typical problematic properties of time series: regularity, data
% validation and data volume.  The decompression time is minimal as data
% in discs get stored directly as a time series. As a consequence, the
% queries or visualisation computing time is only due to the computation
% itself. Moreover, if the query is an aggregation or resolution already
% computed in \acro{MTSMS} consolidation, then the visualisation is
% immediate.





% \subsubsection{Comparació de SGST i SGSTM amb altres models del mercat}


% SGST:

% * Respecte dels models de seqüència: 

%   - una seqüència es defineix com una funció el domini de la qual és un conjunt comptable i totalment ordenat. La definició genèrica que fem del temps no és un conjunt comptable i per tant creiem que un model genèric de sèries temporals no es pot descriure amb comoditat des de les seqüències. 

%   - generalment s'assumeix que la distància entre els elements d'una seqüència és regular

%   - els instants de temps en una sèrie temporal no es poden canviar, en canvi en una seqüencia els índexs només marquen l'ordre i no significat de posicionament en un marc de referència.







\section{Resum dels models}
%Resum del model de multiresolució


El principal objectiu de la multiresolució és l'emmagatzematge
comprimit de sèries temporals. Així, la multiresolució es
contextualitza en l'àmbit dels \gls{SGBD}, per la qual cosa hem
formalitzat el model de dades dels \gls{SGSTM}, els sistemes que
gestionen sèries temporals amb la tècnica de multiresolució. Els
\gls{SGSTM} basen el tractament de les sèries temporals en els
\gls{SGST}, per la qual cosa també hem descrit el model d'aquests
sistemes. El principal objectiu dels \gls{SGST} és gestionar les
sèries temporals de manera coherent a la dimensió temporal.



La multiresolució emmagatzema una sèrie temporal mitjançant diversos
resums d'atributs i resolucions.  Breument, el model de \gls{SGSTM}
s'estructura a partir de \emph{sèries temporals multiresolució} com a
conjunt de \emph{subsèries resolució}, les quals acumulen temporalment
les mesures en un \emph{buffer} per tal de tractar-les abans
d'emmagatzemar-les a un \emph{disc}. El tractament principal
consisteix a canviar els intervals de temps entre mesures i a
agregar-ne atributs, amb l'objectiu de compactar la informació de la
sèrie temporal.  Així, cada sèrie temporal multiresolució té diferents
paràmetres per a configurar de quina manera s'ha de resumir la
informació i calcular les resolucions. És el que anomenem
\emph{esquema de multiresolució} i consisteix en definir la quantitat
de subsèries resolució i quatre paràmetres per a cadascuna: el
\emph{pas de consolidació}, l'\emph{instant de temps d'inici de la
  consolidació}, la \emph{funció d'agregació d'atributs} i la
\emph{capacitat d'emmagatzematge}.



El model de \gls{SGSTM} també inclou les operacions que formalitzen el
comportament d'aquests sistemes. En primer lloc, per a emmagatzemar
una sèrie temporals multiresolució és indispensable que hi hagi
operacions per a afegir mesures i per a consolidar-les. En segon lloc,
hi ha operacions per a manipular l'esquema de multiresolució i per a
observar-ne propietats. En tercer lloc, per a consultar les dades
emmagatzemades hi ha dues operacions bàsiques: obtenir les subsèries
temporals consolidades i obtenir una sèrie temporal total resultant de
concatenar les subsèries. Aquestes consultes resulten en sèries
temporals, així per a elaborar consultes més complexes es poden
utilitzar les operacions dels \gls{SGST}.



Els \gls{SGSTM} necessiten les \emph{funcions d'agregació d'atributs}
per a consolidar les sèries temporals originals.  Aquestes funcions,
però, es formulen com a objectes independents del model per tal que
l'usuari en pugui definir de pròpies.  Així, hem contextualitzat les
agregacions en les interpretacions de mètodes de representació de
sèries temporals i hem introduït el problema de cooperar amb la
validació de dades durant l'agregació.  Hem exemplificat les funcions
d'agregació d'atributs amb estadístics d'agregació simples --mitjana,
màxim i darrer-- interpretats amb tres mètodes de representació
--\gls{pd}, \gls{dd} i \gls{zohe}.




Pel que fa al model de \gls{SGST}, s'estructura en \emph{mesures} i
\emph{sèries temporals}, les quals tenen atributs de \emph{temps} i
\emph{valor}. En les operacions es distingeix el comportament de les
sèries temporals en conjunt, en seqüència i en funció temporal.  També
s'observen les propietats de les sèries temporals en contextos
determinats: els trets semàntics, els mètodes de representació i les
patologies.  Particularment, la multiresolució ha de cooperar per a
solucionar algunes de les propietats problemàtiques de les sèries
temporals, com per exemple la regularitat, un gran volum de dades o
els diversos mètodes de representació.


En resum, proposem una solució de compressió que només emmagatzemi la
informació prevista que es requerirà en futures consultes o bé en
visualitzacions gràfiques.  En comparació, generalment, altres
sistemes per sèries temporals comprimeixen i reconstrueixen el senyal
original o bé emmagatzemen massivament les dades.  Com a tècnica de
compressió, en els \gls{SGSTM} el procés de descompressió és nul quan
es consulta exactament una resolució de les emmagatzemades.  A més, la
multiresolució coopera amb propietats problemàtiques típiques de les
sèries temporals i s'adequa a l'àmbit de monitoratge.


\section{Consideracions}

La multiresolució és una selecció d'una part de la informació d'una
sèrie temporal. Per tant, és una solució d'emmagatzematge amb
pèrdua. Com a conseqüència, l'usuari ha d'escollir un esquema de
multiresolució adequat al context en què vulgui treballar.



El model presentat permet aplicar els \gls{SGSTM} en dispositius que
requerixen un emmagatzematge reduït i afitat. De fet, en els
\gls{SGSTM} es pot preveure a priori la quantitat total de mesures
emmagatzemades per una durada de temps suficientment llarga, tot i que
evidentment finita. Així, es pot evitar la pèrdua de la informació per
manca de capacitat d'emmagatzematge o per manca de temps d'enviar les
dades a un node central, cosa que per exemple pot ocorre en sistemes
integrats petits.


El model és abstracte i genèric, sobretot pel que fa als buffers, per
tal d'abastar el màxim de contexts. Així, principalment es defineix
una consolidació independent de cada resolució en què és possible
utilitzar qualsevol esquema de multiresolució.



Ara bé, considerant diversos contextos en què s'utilitzarà la
multiresolució, es poden fer petits variacions en el comportament dels
buffers. Aquestes variacions, però, impliquen restriccions en els
paràmetres, per exemple limitem les funcions d'agregació d'atributs o
els passos de consolidació que es poden utilitzar.  Les variacions
explorades són: diverses resolucions que comparteixen el mateix
buffer; resolucions encadenades, en què unes depenen de les sèries
temporals consolidades en altres; funcions d'agregació d'atributs
orientades a flux, és a dir que acumulen incrementalment la computació
de les agregacions; i rellotges de consolidació amb diferent
naturalesa --externs al sistema, interns o bé relatius a altres
resolucions.




Pel que fa a la inserció de mesures i el procés posterior de
consolidació, el model de \gls{SGSTM} permet tant una computació en
temps diferit com en línia. És a dir, tant permet tenir una sèrie
temporal ja capturada i processar-la en un \gls{SGSTM} com
processar-la al mateix temps que es va adquirint i per tant seguint el
flux d'entrada de dades.  Aquesta possibilitat de processar en línia
permet fer menys crítica la computació ja que la reparteix durant
l'adquisició de la sèrie temporal. A més permet disposar en línia de
la multiresolució computada i per tant visualitzar-la al mateix temps
en què es va adquirint.






\subsection{Funcions de multiresolució i sistemes duals}


Expressant el model de multiresolució en computació diferida, es pot
observar el problema com una funció sobre una sèrie temporal que
retorna una nova sèrie temporal. Així, es poden dissenyar dues
funcions de multiresolució que tenen una funcionalitat equivalent a
les dues consultes bàsiques dels \gls{SGSTM}. Una, fa un mapa sobre la
sèrie temporal original i un paràmetre de multiresolució per a obtenir
la subsèrie consolidada. L'altra, fa un plec sobre la sèrie temporal
original i un esquema de multiresolució per a obtenir la sèrie
temporal total.

Aquestes funcions de multiresolució són operacions que es poden
resoldre en els \gls{SGST}. De fet, és interessant la possibilitat de
computar-les amb tècniques distribuïdes i para\l.leles com comentem
després.




A partir de les funcions de multiresolució es poden dissenyar
estructures més complexes en què intervingui la multiresolució. Aquest
és el cas de sistemes duals de multiresolució en què una mateixa sèrie
temporal s'emmagatzema alhora en un \gls{SGST} i en un
\gls{SGSTM}. Aleshores, segons les necessitats, es poden consultar les
sèries temporals consolidades emmagatzemades al \gls{SGSTM} o bé, de
forma equivalent, consultar les funcions de multiresolució al
\gls{SGST}.  
Els sistemes duals ofereixen altres aplicacions de la
multiresolució.

En primer lloc, la multiresolució implica pèrdua de dades i
l'\gls{SGST} dels sistemes duals pot servir com a dipòsit a llarg
termini de les dades originals, el qual no es consulta
freqüentment. Habitualment es consultarien els resums d'informació
emmagatzemats a l'\gls{SGSTM} i en cas que aquests no fossin
suficients es podrien usar les dades de l'\gls{SGST}. Per exemple per
a resoldre consultes detallades sobre les dades semblants a `a quina
hora exacta ha ocorregut un esdeveniment'.

En segon lloc, la multiresolució requereix definir un esquema de
multiresolució abans de començar a recollir les dades.  Però cal
conèixer l'entorn monitorat per a establir-ne un esquema de
multiresolució, per exemple per a determinar quines funcions
d'agregació són les més escaients.  En aquest període de transició,
fins que es conegui l'esquema de multiresolució més escaient, es pot
utilitzar un sistema dual per a experimentar amb diferents esquemes. I
un cop determinats, ja es pot utilitzar només l'emmagatzematge en
l'\gls{SGSTM}.

En tercer lloc, la computació en flux dels \gls{SGSTM} és una
estratègia que convé tenir present ja que permet repartir el temps de
computació en línia amb l'adquisició de les dades.  Així, prenent com
a referència altres conceptes similars en les vistes dels \gls{SGBD} i
en l'arquitectura Lambda, els sistemes duals es poden aplicar seguint
l'estratègia següent. L'\gls{SGSTM} processa el flux de dades al
mateix temps en què s'adquireixen i sempre ofereix la resposta de
multiresolució forma immediata, és a dir té els resultats
precomputats. L'\gls{SGST} actua per acabar de completar les consultes
o bé si s'ha de canviar l'esquema de multiresolució i tornar a
començar de nou.



Això no obstant, cal tenir present que els \gls{SGST} són un
emmagatzematge massiu de les dades. A més, els sistemes duals
impliquen en certa manera un emmagatzematge redundant de la informació
en els dos sistemes.  En els sistemes duals, assumim que l'usuari vol
obtenir la multiresolució de la sèrie temporal original: ja sigui per
a visualitzar-la directament o per a utilitzar-la com a base per a
altre consultes.  Així, hi ha contextos en què els resum de la
multiresolució són suficients i es pot prescindir de conservar totes
les dades històriques, per exemple en el monitoratge de l'evolució de
la bateria disponible en un portàtil és suficient visualitzar-ne un
resum. També en altres contextos la multiresolució és suficient per a
respondre determinades consultes sobre les dades, per exemple una
consulta similar a `en aquestes dades la mitjana dels darrers dies és
similar a l'habitual o bé creix'. En aquests contextos, el model de
multiresolució és molt adequat; la reflexió sobre la qualitat de la
multiresolució aclareix quan les dades emmagatzemades són
suficients i quan la informació emmagatzemada és redundant.






\subsection{Reflexió sobre la qualitat}


La multiresolució és una tècnica de compressió amb pèrdua de dades i
els paràmetres de l'esquema de multiresolució són graus de llibertat
per a cada context d'aplicació. Així, depenent d'aquests paràmetres,
se selecciona una informació o una altra de la sèrie temporal
original.  La compressió amb pèrdua implica que algunes operacions de
\gls{SGST} són consultes aproximades quan es resolen a partir de les
sèries temporals emmagatzemades en els \gls{SGSTM}. Com a
conseqüència, cal determinar la qualitat que té la compressió de la
multiresolució per tal de poder dissenyar correctament els esquemes de
multiresolució i poder quantificar l'aproximació de les consultes que
s'hi basin.


En altres àmbits, per al cas de pèrdua principalment la compressió de
multimèdia, la teoria de la informació avalua formalment l'efecte que
té la compressió en la informació que hi ha originalment a les dades.
L'anàlisi que formulem, però, és una introducció a la reflexió sobre
l'error en la informació de la multiresolució. Així, de forma simple,
analitzem si hi ha error o si no n'hi ha, sense pretendre
quantificar-lo més detalladament. 



Si bé definim el problema genèric d'error en la multiresolució, és un
problema massa abstracte per a treballar-lo directament. Així doncs,
analitzem alguns casos particulars per tal que serveixin com a exemple
per a reflexionar sobre l'efecte que té una configuració determinada
de paràmetres multiresolució. Concretament, es pot avaluar l'error de
multiresolució per a algunes funcions d'agregació d'atributs
particulars i de com siguin les consultes posteriors. Cal destacar el
cas específic dels comptadors, en què es pot conservar la informació
de comptatge que contenen coherentment amb la seva naturalesa.



\section{Experimentació}


Un cop s'ha definit el model dels \gls{SGST} i dels \gls{SGSTM}, es
poden implementar. D'aquesta manera, el model descriu exactament el
comportament de les implementacions i es pot assegurar que aquestes
tenen el funcionament desitjat.  De fet, el model definit descriu
essencialment les operacions, entrades, sortides i tipus de dades
bàsics amb els quals han de treballar les implementacions. Això es
coneix com a interfície de programació d'aplicacions, tot i que
habitualment s'especifiquen i representen amb llenguatges informàtics,
en comptes de models formals matemàtics, i incorporen detalls de la
implementació en qüestió.


L'objectiu de les implementacions és principalment acadèmic per a
mostrar el funcionament dels models, tot i que també són útils per
poder experimentar amb dades reals. Així, dissenyem tres
implementacions de naturalesa diferent. 



\todo{fer}


En primer lloc, 
Python, orientació a objectes va bé per a tenir una bona correspondència entre implementació i model. A més té una llibreria extensa que ens permet afegir complements com fer gràfics i gestionar l'emmagatzematge en fitxers
Tant admet la computació en línia com en diferit.
%  utilitzant el llenguatge de programació Python es dissenya, a nivell acadèmic, un sistema de gestió de bases de dades que implementa el model de dades tal com s'ha definit en aquest capítol.


En segon lloc, MapReduce i Hadoop. Tècniques de computació para\l.lela i distribuïda. Basat en les funcions de multiresolució i per tant en la computació en diferit. Aquí, però, hem de fer simplificacions del model, perquè aquestes tècniques no s'hi ajusten. 


En tercer lloc, Tutorial~D, i Rel com a intèrpret. En aquest cas el
llenguatge acadèmic dels \gls{SGBDR} i un intèrpret en
experimentació. Per tant, és la implementació més experimental però
que permet enfocar els \gls{SGST} en una altra línia de treball,
d'alguna manera assenyala els avantatges i els límits del model
relacional per a tipus complexos com són les sèries temporals i la
multiresolució. En aquest cas també podríem haver fet un estudi
similar pels \gls{SGSTM} però només hem implementat les funcions de
multiresolució. Aquesta implementació és la més acadèmica per tant
també està limitada per a provar amb dades reals.




Finalment, posem a prova les implementacions amb dades reals massives.
Aquest exemple conclou el funcionament de la multiresolució,
principalment s'assenyala el cronograma de multiresolució sobre unes
dades originals i mostra les característiques dels mètodes de
representació i diferents funcions d'agregació d'atributs,
particularment del màxim i la mitjana de la família \gls{zohe}.


És difícil comparar les implementacions perquè no tenen res a veure. A més l'exemple es basa en computació diferida. 

També cal dir que en la programació de les implementacions no hem tingut en compte eficiència, de fet hem preferit conservar les operacions algebraiques i lògiques fetes en les definicions del models.

Això no obstant, el rendiment ja no és només calcular en poc temps, sinó que en alguns contextos pot ser un consum baix d'energia, ocupar poc espai, etc.
És a dir, que hi ha contextos i en cadascun el rendiment vol dir una cosa o una altra.
A més, hi ha la computació en flux. Aquesta ens permet anar calcular a mesura que van arribant les dades en comptes d'acumular-ho tot i fer-ho de cop. Si potdem utilitzar un esquema de flux aleshores els càlculs quedem repartits en el temps per la mateixa naturalesa del flux. (per altra banda també estudiaderem càlculs en temps diferent i oferirem una solució amb computació para\l.lela).
També hi ha l'emmagatzematge limitat per a sistemes petits.



* Sobre la multiresolució calculada en flux en un SGSTM o amb computació paral·lela: la primera és computació distribuïda al llarg del temps i la segona és computació distribuïda en l'espai (o en els recursos físics)



* Quan les dades són grans són grans i no s'hi pot fer res més. Si es
vol calcular amb elles doncs té un cost elevat en temps. Nosaltres hem
definit unes operacions de multiresolució sobre els SGST que triguen
una estona a calcular-se perquè s'han de fer moltes operacions o s'han
de fer moltes vegades. Això no obstant, aquest és l'únic punt crític
en recursos de les sèries temporals; la resta (afegir una nova dada,
emmagatzemar amb redundància, posar etiquetes a les sèries temporals,
...) es pot fer perfectament amb els SGBDR, de fet són la millor eina.
Com es resol el problema de múltiples càlculs? proposem dues
implementacions dels SGSTM: una on hi ha execució paral·lela i una
altra on hi ha execució en flux.  L'execució paral·lela permet
aprofitar quan es poden fer operacions en paral·lel, que és el cas de
la multiresolució. L'execució en flux permet repartir al llarg del
temps els càlculs quan és possible, en la multiresolució podem anar
fent els càlculs cada cop que entra una dada nova així sempre hi ha a
punt els càlculs fets i els podem visualitzar instantàniament i hem
pogut anar repartint el temps d'execució i no ens cal fer-ho tot de
cop (això sí: requereix planificació dels càlculs que voldrem
consultar o visualitzar); per tant ja no tenim problemes de recursos,
hem reconvertit el problema de dades grans, l'hem simplificat.
Aquesta nova estructura de SGSTM en flux es pot implementar en qualsevol lloc, pròpia, SGBDR, etc, ja que no hi ha tanta criticitat en els recursos. 




















%%% Local Variables:
%%% TeX-master: "main"
%%% End:
% LocalWords:  SGSTM multiresolució subsèries

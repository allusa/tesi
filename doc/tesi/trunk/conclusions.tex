\chapter{Conclusions}
\label{sec:conclusions}


En aquest capítol es resumeix l'exposat en el cos principal del
document i s'extreuen conclusions del model dissenyat i de les
implementacions.


Hem formalitzat la multiresolució per a les sèries temporals com un
model de \gls{SGBD}. La multiresolució permet seleccionar una
determinada informació d'una sèrie temporal i només emmagatzemar-ne
una quantitat de dades afitada. Això no obstant és una solució de
compressió amb pèrdua i cal avaluar i adequar els paràmetres de la
multiresolució a cada context.


El model defineix els sistemes de multiresolució de forma senzilla i
compacta, la formalització permet definir un model totalment genèric i
molt potent. Les implementacions de sistemes multiresolució exploren
diverses maneres de computar la multiresolució, el model de
multiresolució està pensat per a tenir en compte el context dels
recursos: capacitat d'emmagatzematge limitada, processament en temps
diferit o en línia amb el flux de dades, distribució de les dades,
transmissió de resums de la informació, etc.



En comparació a altres sistemes que gestionen sèries temporals, la
multiresolució és una solució de compressió que només emmagatzema les
dades que es requeriran en consultes posteriors o per a
visualitzacions gràfiques. El procés de descompressió és mínim ja que
les dades s'emmagatzemen directament com a subsèries temporals. A més,
la multiresolució co\l.labora a solucionar les propietats
problemàtiques de les sèries temporals: la regularitat, les dades
desconegudes o un volum massiu de dades.  Les sèries temporals es
defineixen genèriques més enllà de considerar-les simples seqüències:
els instants de temps poden marcar alhora ordre i posició absoluta
temporal, la distància entre les mesures pot ser irregular i
consideren el seu àmbit d'adquisició en sistemes de monitoratge.



A continuació, més particularment:

\begin{itemize}
\item Es resumeixen els models formalitzats i la tècnica de
  multiresolució

\item Es reflexiona sobre algunes consideracions i variacions dels models

\item Es resumeix l'experimentació en el disseny d'implementacions per
  als models

\end{itemize}








\section{Resum dels models}
%Resum del model de multiresolució


El principal objectiu de la multiresolució és l'emmagatzematge
comprimit de sèries temporals. Així, la multiresolució es
contextualitza en l'àmbit dels \gls{SGBD}, per la qual cosa hem
formalitzat el model de dades dels \gls{SGSTM}, els sistemes que
gestionen sèries temporals amb la tècnica de multiresolució. Els
\gls{SGSTM} basen el tractament de les sèries temporals en els
\gls{SGST}, per la qual cosa també hem descrit el model d'aquests
sistemes. El principal objectiu dels \gls{SGST} és gestionar les
sèries temporals de manera coherent a la dimensió temporal.



La multiresolució emmagatzema una sèrie temporal mitjançant diversos
resums d'atributs i resolucions.  Breument, el model de \gls{SGSTM}
s'estructura a partir de \emph{sèries temporals multiresolució} com a
conjunt de \emph{subsèries resolució}, les quals acumulen temporalment
les mesures en un \emph{buffer} per tal de tractar-les abans
d'emmagatzemar-les a un \emph{disc}. El tractament principal
consisteix a canviar els intervals de temps entre mesures i a
agregar-ne atributs, amb l'objectiu de compactar la informació de la
sèrie temporal.  Així, cada sèrie temporal multiresolució té diferents
paràmetres per a configurar de quina manera s'ha de resumir la
informació i calcular les resolucions. És el que anomenem
\emph{esquema de multiresolució} i consisteix en definir la quantitat
de subsèries resolució i quatre paràmetres per a cadascuna: el
\emph{pas de consolidació}, l'\emph{instant de temps d'inici de la
  consolidació}, la \emph{funció d'agregació d'atributs} i la
\emph{capacitat d'emmagatzematge}.



El model de \gls{SGSTM} també inclou les operacions que formalitzen el
comportament d'aquests sistemes. En primer lloc, per a emmagatzemar
una sèrie temporals multiresolució és indispensable que hi hagi
operacions per a afegir mesures i per a consolidar-les. En segon lloc,
hi ha operacions per a manipular l'esquema de multiresolució i per a
observar-ne propietats. En tercer lloc, per a consultar les dades
emmagatzemades hi ha dues operacions bàsiques: obtenir les subsèries
temporals consolidades i obtenir una sèrie temporal total resultant de
concatenar les subsèries. Aquestes consultes resulten en sèries
temporals, així per a elaborar consultes més complexes es poden
utilitzar les operacions dels \gls{SGST}.



Els \gls{SGSTM} necessiten les \emph{funcions d'agregació d'atributs}
per a consolidar les sèries temporals originals.  Aquestes funcions,
però, es formulen com a objectes independents del model per tal que
l'usuari en pugui definir de pròpies.  Així, hem contextualitzat les
agregacions en les interpretacions de mètodes de representació de
sèries temporals i hem introduït el problema de cooperar amb la
validació de dades durant l'agregació.  Hem exemplificat les funcions
d'agregació d'atributs amb estadístics d'agregació simples --mitjana,
màxim i darrer-- interpretats amb tres mètodes de representació
--\gls{pd}, \gls{dd} i \gls{zohe}.




Pel que fa al model de \gls{SGST}, s'estructura en \emph{mesures} i
\emph{sèries temporals}, les quals tenen atributs de \emph{temps} i
\emph{valor}. En les operacions es distingeix el comportament de les
sèries temporals en conjunt, en seqüència i en funció temporal.  També
s'observen les propietats de les sèries temporals en contextos
determinats: els trets semàntics, els mètodes de representació i les
patologies.  Particularment, la multiresolució ha de cooperar per a
solucionar algunes de les propietats problemàtiques de les sèries
temporals, com per exemple la regularitat, un gran volum de dades o
els diversos mètodes de representació.


En resum, proposem una solució de compressió que només emmagatzemi la
informació prevista que es requerirà en futures consultes o bé en
visualitzacions gràfiques.  En comparació, generalment, altres
sistemes per sèries temporals comprimeixen i reconstrueixen el senyal
original o bé emmagatzemen massivament les dades.  Com a tècnica de
compressió, en els \gls{SGSTM} el procés de descompressió és nul quan
es consulta exactament una resolució de les emmagatzemades.  A més, la
multiresolució coopera amb propietats problemàtiques típiques de les
sèries temporals i s'adequa a l'àmbit de monitoratge.


\section{Consideracions}

La multiresolució és una selecció d'una part de la informació d'una
sèrie temporal. Per tant, és una solució d'emmagatzematge amb
pèrdua. Com a conseqüència, l'usuari ha d'escollir un esquema de
multiresolució adequat al context en què vulgui treballar.



El model presentat permet aplicar els \gls{SGSTM} en dispositius que
requerixen un emmagatzematge reduït i afitat. De fet, en els
\gls{SGSTM} es pot preveure a priori la quantitat total de mesures
emmagatzemades per una durada de temps suficientment llarga, tot i que
evidentment finita. Així, es pot evitar la pèrdua de la informació per
manca de capacitat d'emmagatzematge o per manca de temps d'enviar les
dades a un node central, cosa que per exemple pot ocorre en sistemes
integrats petits. 


El model és abstracte i genèric, sobretot pel que fa als buffers, per
tal d'abastar el màxim de contexts. Així, principalment es defineix
una consolidació independent de cada resolució en què és possible
utilitzar qualsevol esquema de multiresolució.



Ara bé, considerant diversos contextos en què s'utilitzarà la
multiresolució, es poden fer petits variacions en el comportament dels
buffers. Aquestes variacions, però, impliquen restriccions en els
paràmetres, per exemple limitem les funcions d'agregació d'atributs o
els passos de consolidació que es poden utilitzar.  Les variacions
explorades són: diverses resolucions que comparteixen el mateix
buffer; resolucions encadenades, en què unes depenen de les sèries
temporals consolidades en altres; funcions d'agregació d'atributs
orientades a flux, és a dir que acumulen incrementalment la computació
de les agregacions; i rellotges de consolidació amb diferent
naturalesa --externs al sistema, interns o bé relatius a altres
resolucions.




Pel que fa a la inserció de mesures i el procés posterior de
consolidació, el model de \gls{SGSTM} permet tant una computació en
temps diferit com en línia. És a dir, tant permet tenir una sèrie
temporal ja capturada i processar-la en un \gls{SGSTM} com
processar-la al mateix temps que es va adquirint i per tant seguint el
flux d'entrada de dades.  Aquesta possibilitat de processar en línia
permet fer menys crítica la computació ja que la reparteix durant
l'adquisició de la sèrie temporal. A més permet disposar en línia de
la multiresolució computada i per tant visualitzar-la al mateix temps
en què es va adquirint.






\subsection{Funcions de multiresolució i sistemes duals}


Expressant el model de multiresolució en computació diferida, es pot
observar el problema com una funció sobre una sèrie temporal que
retorna una nova sèrie temporal. Així, es poden dissenyar dues
funcions de multiresolució que tenen una funcionalitat equivalent a
les dues consultes bàsiques dels \gls{SGSTM}. Una, fa un mapa sobre la
sèrie temporal original i un paràmetre de multiresolució per a obtenir
la subsèrie consolidada. L'altra, fa un plec sobre la sèrie temporal
original i un esquema de multiresolució per a obtenir la sèrie
temporal total.

Aquestes funcions de multiresolució són operacions que es poden
resoldre en els \gls{SGST}. De fet, és interessant la possibilitat de
computar-les amb tècniques distribuïdes i para\l.leles com comentem
després.




A partir de les funcions de multiresolució es poden dissenyar
estructures més complexes en què intervingui la multiresolució. Aquest
és el cas de sistemes duals de multiresolució en què una mateixa sèrie
temporal s'emmagatzema alhora en un \gls{SGST} i en un
\gls{SGSTM}. Aleshores, segons les necessitats, es poden consultar les
sèries temporals consolidades emmagatzemades al \gls{SGSTM} o bé, de
forma equivalent, consultar les funcions de multiresolució al
\gls{SGST}.  
Els sistemes duals ofereixen altres aplicacions de la
multiresolució.

En primer lloc, la multiresolució implica pèrdua de dades i
l'\gls{SGST} dels sistemes duals pot servir com a dipòsit a llarg
termini de les dades originals, el qual no es consulta
freqüentment. Habitualment es consultarien els resums d'informació
emmagatzemats a l'\gls{SGSTM} i en cas que aquests no fossin
suficients es podrien usar les dades de l'\gls{SGST}. Per exemple per
a resoldre consultes detallades sobre les dades semblants a `a quina
hora exacta ha ocorregut un esdeveniment'.

En segon lloc, la multiresolució requereix definir un esquema de
multiresolució abans de començar a recollir les dades.  Però cal
conèixer l'entorn monitorat per a establir-ne un esquema de
multiresolució, per exemple per a determinar quines funcions
d'agregació són les més escaients.  En aquest període de transició,
fins que es conegui l'esquema de multiresolució més escaient, es pot
utilitzar un sistema dual per a experimentar amb diferents esquemes. I
un cop determinats, ja es pot utilitzar només l'emmagatzematge en
l'\gls{SGSTM}.

En tercer lloc, la computació en flux dels \gls{SGSTM} és una
estratègia que convé tenir present ja que permet repartir el temps de
computació en línia amb l'adquisició de les dades.  Així, prenent com
a referència altres conceptes similars en les vistes dels \gls{SGBD} i
en l'arquitectura Lambda, els sistemes duals es poden aplicar seguint
l'estratègia següent. L'\gls{SGSTM} processa el flux de dades al
mateix temps en què s'adquireixen i sempre ofereix la resposta de
multiresolució forma immediata, és a dir té els resultats
precomputats. L'\gls{SGST} actua per acabar de completar les consultes
o bé si s'ha de canviar l'esquema de multiresolució i tornar a
començar de nou.



Això no obstant, cal tenir present que els \gls{SGST} són un
emmagatzematge massiu de les dades. A més, els sistemes duals
impliquen en certa manera un emmagatzematge redundant de la informació
en els dos sistemes.  En els sistemes duals, assumim que l'usuari vol
obtenir la multiresolució de la sèrie temporal original: ja sigui per
a visualitzar-la directament o per a utilitzar-la com a base per a
altre consultes.  Així, hi ha contextos en què els resum de la
multiresolució són suficients i es pot prescindir de conservar totes
les dades històriques, per exemple en el monitoratge de l'evolució de
la bateria disponible en un portàtil és suficient visualitzar-ne un
resum. També en altres contextos la multiresolució és suficient per a
respondre determinades consultes sobre les dades, per exemple una
consulta similar a `en aquestes dades la mitjana dels darrers dies és
similar a l'habitual o bé creix'. En aquests contextos, el model de
multiresolució és molt adequat; la reflexió sobre la qualitat de la
multiresolució aclareix quan les dades emmagatzemades són
suficients i quan la informació emmagatzemada és redundant.






\subsection{Reflexió sobre la qualitat}


La multiresolució és una tècnica de compressió amb pèrdua de dades i
els paràmetres de l'esquema de multiresolució són graus de llibertat
per a cada context d'aplicació. Així, depenent d'aquests paràmetres,
se selecciona una informació o una altra de la sèrie temporal
original.  La compressió amb pèrdua implica que algunes operacions de
\gls{SGST} són consultes aproximades quan es resolen a partir de les
sèries temporals emmagatzemades en els \gls{SGSTM}. Com a
conseqüència, cal determinar la qualitat que té la compressió de la
multiresolució per tal de poder dissenyar correctament els esquemes de
multiresolució i poder quantificar l'aproximació de les consultes que
s'hi basin.


En altres àmbits, per al cas de pèrdua principalment la compressió de
multimèdia, la teoria de la informació avalua formalment l'efecte que
té la compressió en la informació que hi ha originalment a les dades.
L'anàlisi que formulem, però, és una introducció a la reflexió sobre
l'error en la informació de la multiresolució. Així, de forma simple,
analitzem si hi ha error o si no n'hi ha, sense pretendre
quantificar-lo més detalladament. 



Si bé definim el problema genèric d'error en la multiresolució, és un
problema massa abstracte per a treballar-lo directament. Així doncs,
analitzem alguns casos particulars per tal que serveixin com a exemple
per a reflexionar sobre l'efecte que té una configuració determinada
de paràmetres multiresolució. Concretament, es pot avaluar l'error de
multiresolució per a algunes funcions d'agregació d'atributs
particulars i de com siguin les consultes posteriors. Cal destacar el
cas específic dels comptadors, en què es pot conservar la informació
de comptatge que contenen coherentment amb la seva naturalesa.



\section{Experimentació}


Un cop s'ha definit el model dels \gls{SGST} i dels \gls{SGSTM}, es
poden implementar. D'aquesta manera, el model descriu exactament el
comportament de les implementacions i es pot assegurar que aquestes
tenen el funcionament desitjat.  De fet, el model definit descriu
essencialment les operacions, entrades, sortides i tipus de dades
bàsics amb els quals han de treballar les implementacions. Això és
conegut com a interfície de programació d'aplicacions, tot i que
habitualment s'especifiquen i es representen amb llenguatges
informàtics, en comptes de models formals matemàtics, i incorporen
detalls de la implementació en qüestió.


L'objectiu de les implementacions és principalment acadèmic per a
mostrar el funcionament dels models, tot i que també són útils per
poder experimentar amb dades reals. Així, dissenyem tres
implementacions de naturalesa diferent. 



En primer lloc, Pytsms i RoundRobinson implementats amb llenguatge
Python.  Aprofita el paradigma d'orientació a objectes per a obtenir
una bona correspondència entre implementació i model. A més Python té
unes biblioteques extensa que permet afegir complements com per
exemple gràfics o gestionar l'emmagatzematge en fitxers.  Aquesta és
la implementació de referència, és a dir la que considerem que mostra
com es pot implementar el model de multiresolució definit.  Tant admet
la computació en línia com en diferit.

En segon lloc, RoundRobindoop implementat amb MapReduce i
Hadoop. Utilitza MapReduce com a tècnica de programació para\l.lela i
distribuïda i usa Hadoop com a sistema de computació d'aquesta
tècnica. Està basat en les funcions de multiresolució i per tant és
una computació en diferit. Hi ha, però, algunes simplificacions del
model de multiresolució per a poder-se ajustar a MapReduce.

En tercer lloc, Reltsms implementat amb Tutorial~D, el llenguatge
acadèmic dels \gls{SGBDR}, i Rel com a intèrpret.  Aquesta és la
implementació més experimental, de fet Rel està en desenvolupament.
Permet situar els \gls{SGST} en els \gls{SGBDR} i alhora avaluar
l'adequació del model relacional per a tipus complexos com són les
sèries temporals i la multiresolució.  Pel que fa a la multiresolució,
s'han implementat les funcions de multiresolució. Aquesta
implementació, però, és la més acadèmica i és limitada a l'hora de
provar amb dades reals.



Finalment, posem a prova les implementacions amb dades reals massives.
Aquestes proves demostren el bon funcionament de la multiresolució en
un context determinat. A més, permeten exemplificar la cobertura del
cronograma de multiresolució sobre unes dades originals i mostrar les
característiques dels diferents mètodes de representació i funcions
d'agregació d'atributs, particularment del màxim i la mitjana de la
família \gls{zohe}.




En una comparació entre les diverses implementacions de la
multiresolució, s'ha observat que computen de manera similar per a un
mateix cas. A excepció de Reltsms que no admet casos reals massius,
car es basa en un \gls{SGBDR} totalment acadèmic i encara
desenvolupament.  Si bé cal notar que en la programació de les
implementacions no hem tingut en compte l'eficiència, de fet hem
preferit conservar certa similitud amb les operacions algebraiques i
lògiques fetes en les definicions del models.    Així
doncs, basar-se en el model relacional ha resultat adequat per a
aquests objectius abstractes d'implementació. A més, ha permès
implementacions de diversa natura, com són Pytsms i Reltsms.
RoundRobindoop és la implementació més específica de totes, la podem
qualificar com a sistema de càlcul de la multiresolució. En canvi,
RoundRobinson i Pytsms són les més genèriques, car les hem dissenyat
com a referència dels models. Però també gràcies que Python disposa de
biblioteques extenses, les quals són útils a l'hora d'implementar
funcionalitats complementàries.



Pel que fa a la computació de la multiresolució, s'ha vist que pot ser
costosa a causa dels múltiples càlculs que s'han de computar. Hem
estudiat dues estratègies remarcables.  D'una banda, la computació
para\l.lela i distribuïda, que distribueix el còmput en els recursos
físics disponibles i per tant permet processar ràpidament.  D'altra
banda, la computació en línia amb el flux de dades, que distribueix el
còmput al llarg del temps d'adquisició i per tant el fa menys crític.
En aquest cas, la multiresolució és precomputada i sempre està
disponible per a consultar-la o visualitzar-la immediatament. Ara bé,
requereix planificar prèviament l'esquema de multiresolució.



















%%% Local Variables:
%%% TeX-master: "main"
%%% End:
% LocalWords:  SGSTM multiresolució subsèries

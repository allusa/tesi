\chapter{Introducció als models}
\label{sec:model}
\glsreset{SGST}
\glsreset{SGSTM}

En els capítols següents es dissenya un model matemàtic per a la
gestió en bases de dades de la multiresolució de sèries temporals.  Es
defineixen els objectes que ens permeten modelar l'estructura de les
dades i els operadors que s'hi poden aplicar.

La definició del model s'estructura en dos capítols:

\begin{itemize}
\item Un model pels \glspl{SGST}  que defineix mesura i sèrie temporal.
\item Un model pels \glspl{SGSTM} que defineix buffer, disc, subsèrie
  resolució, sèrie temporal multiresolució i esquema de
  multiresolució. Aquest model es defineix a partir del model de \gls{SGST}.
\end{itemize}



  
En aquest capítol d'introducció, resumim els objectius de la definició
dels models i aclarim alguns termes i conceptes que altrament podrien
resultar confusos.  Primer, relacionem el concepte de model matemàtic
pels \gls{SGBD} de \textref{sec:art:sgbd} amb el model que
proposem. Segon, introduïm el context i els supòsits que utilitzarem a
l'hora de treballar amb les sèries temporals en els conceptes de
\textref{sec:art:seriestemporals}.  Tercer, introduïm les
característiques principals de la multiresolució i la motivació per a
definir-la.


\section{Introducció als conceptes de model}

%Sobre què és una base de dades i un sistema que la gestiona
\paragraph{Què és una base de dades i un sistema que la gestiona.}
Una definició més particular de base de dades que l'exposada en
la~\autoref{sec:art:sgbd} és ``conjunt de dades organitzades segons
una estructura coherent i accessibles des de més d'un programa o
aplicació, de manera que qualsevol d'aquestes dades pot ésser extreta
del conjunt i actualitzada, sense que això afecti ni l'estructura del
conjunt ni les altres dades'' \parencite[s.~v.~base de dades]{termcat}
amb la corresponent definició per als sistemes que les gestionen de
``sistema informàtic que permet la gestió automàtica d'una base de
dades, generalment la creació, l'emmagatzematge, la modificació i la
protecció de les dades que s'hi contenen'' \parencite[s.~v.~sistema de
gestió de bases de dades]{termcat}.  En aquest cas, hem de precisar
que en els capítols següents ens centrem en la teoria dels sistemes
d'informació, basant-nos en el model
relacional \parencite{date04:introduction8}, per a proposar el model
lògic i deixem de banda els aspectes més d'implementació informàtica,
com per exemple accedir o manipular físicament les dades.

D'aquestes definicions cal destacar la precisió en els termes
d'estructura, organitzada i coherent, és a dir que s'enumeren els
requisits per al correcte emmagatzematge de les dades en concordança
amb l'expressat en el model relacional.  En aquestes definicions, a
més, caldria afegir que els \gls{SGBD} han de ser capaços d'inferir
informació, és a dir a partir de les dades emmagatzemades deduir-ne de
noves mitjançant les consultes.





%Sobre model aproximat i model exacte
\paragraph{Què es modelitza amb exactitud i què amb aproximació.}
Els models de \gls{SGST} i \gls{SGSTM} que proposem són models lògics
matemàtics, és a dir són models formals i abstractes que defineixen
amb exactitud l'estructura i la manipulació d'unes dades
independentment de la realitat.  
% Així doncs es defineixen models de
% \gls{SGBD} i es presenten com un model matemàtic formal, per tant la
% realitat que es modelitza són els \gls{SGBD}.
Així, definim els models com models matemàtics formals i els utilitzem
com a eines simbòliques per a modelitzar la
realitat.% , és a dir en el procés de dissenyar un model
% que aproximi i simplifiqui una realitat
Cal no confondre la definició
del model formal matemàtic amb la interpretació del model en una
realitat concreta.

Tot i tenir tot el sentit matemàtic, un model lògic abstracte no té
sentit pràctic si no es té en compte la descripció que fa de la
realitat, és a dir que també s'ha d'interpretar el significat que té
el model en la realitat.  Aquest procés d'interpretació construeix un
model aproximat i simplificat de la realitat; consisteix en definir,
per a un determinat context, quines variables hi ha, de quin tipus són
els valors, quina forma tenen, com s'interpreten, etc.  Els \gls{SGBD}
descriuen la realitat mitjançant predicats que indiquen quins fets es
consideren certs. Per tant, els models de \gls{SGBD} assumeixen que
aquests predicats són la realitat i deixen per a altres teories
l'avaluació de com els predicats s'aproximen a la realitat. Alhora,
deleguen als usuaris la interpretació del significat dels predicats a
la realitat llevat d'allò que es pot expressar mitjançant regles
d'integritat.


En l'àmbit dels \gls{SGST}, el model assumeix com a predicat la mesura
d'un valor i deixen que altres teories, com per exemple la teoria de
la mesura, avaluïn fins a quin punt el valor mesurat s'aproxima a la
realitat. Emfatitzant en aquest aspecte, es poden formular mesures
desconegudes per a descriure errors en l'adquisició de les mesures. En
el cas particular dels \gls{SGSTM}, el model que es proposa descriu
com a fet cert unes resolucions de la sèrie temporal i mitjançant
altres teories, com per exemple la teoria de la informació, s'avalua
com s'aproximen a la informació d'una sèrie temporal original.



%  D'alguna manera, els SGBD descriuen exactament els
% fets que es coneixen i mitjançant altres teories es dedueix en quina
% mesura aquests fets coneguts s'aproximen a la realitat (particularment
% un fet descrit com a cert en un SGBD pot ser totalment esbiaixat de la
% realitat però això no se n'ocupen els SGBD; és a dir el predicat exacte és el d'allò que és observat tot i que mitjançant altres teories es pot assegurar un predicat d'allò que és la realitat).
% ; i per tant
% la interpretació del significat dels models matemàtics a la realitat,
% i com a conseqüència la modelització aproximada de la realitat, depèn del
% conjunt del model lògic de SGBD més altres teories. 


% Kopetz en el llibre de temps real (capítol 2 o 3?) diu que un model té l'objectiu d'estudiar una realitat simplificada per a facilitar la comprensió d'una determinada característica. Potser la definició que fa i la intencionalitat que té no és ben bé la mateixa que el tipus de model que parlem aquí? Si és el cas potser estaria bé fer notar que hi ha diferències en el concepte de model segon l'àmbit i que aquí s'utilitza en tal sentit.

% Per exemple, Fabian Pascal parla de representar la realitat de manera simple (i no tant de simplificar la realitat):
% ``For the informational purpose that RM satisfies--inferencing facts that are logical implications of facts represented in databases--the RM is superior, because it is the simplest way to guarantee logically correct results with respect to the real world and it has the highest scope-to-simplicity ratio: it can represent any reality with the least and simplest of constructs''




\paragraph{Quin nivell es modela.}
En la~\autoref{sec:art:sgbd} s'han nombrat els tres nivells de
l'arquitectura dels \gls {SGBD}: el físic, el lògic i el d'usuari.
Els models que es proposen pertanyen al nivell lògic, és a dir són
models lògics de l'estructura de dades i del comportament dels
\gls{SGST} i dels \gls{SGSTM}.  En \textref{sec:implementacio:python} es
proposen implementacions per a aquests models i per tant pertanyen al
nivell físic.  En alguns exemples i descripcions de propietats dels
models, s'avalua el significat en un context particular del model, és
a dir la semàntica del model, cosa que pertany a descriure com els
usuaris poden interpretar el model lògic per a modelitzar la realitat.
% és a dir descriu la relació entre el nivell lògic i el d'usuari \todo{???}





\paragraph{Àlgebra o càlcul relacional.} Els models que definim són
similars a l'àlgebra relacional. L'àlgebra relacional es basa en la
teoria de conjunts, que és més propera a la definició d'una sèrie
temporal com a conjunt de mesures i a aplicar-hi operacions de manera
prescriptiva. Alternativament, el càlcul relacional, que com s'ha
descrit a la~\autoref{sec:art:sgbd} és equivalent a
l'àlgebra relacional, es basa en la lògica de predicats i és més
proper a aplicar les operacions de manera descriptiva. Això no
obstant, en les definicions usem tant l'àlgebra com la lògica de
conjunts segons convingui i faciliti la comprensió de les definicions.




% \paragraph{Model com a relació o com a tipus de dades.} 
% Aquesta és una pregunta complicada. Sobretot perquè el model
% relacional no ho aclareix; descriu les relacions com si no fossin un
% tipus de dades tot i que després accepta que hi hagi relacions amb
% atributs de relacions. Nosaltres volem presentar un model de gestió
% de dades per a tipus complexos, les sèries temporals i les
% multiresolució, i aleshores veiem la necessitat d'abordar-ho des del
% model relacional al complet (com si fossin uns SGBD específics per a
% aquelles tasques). Ara bé, en un SGBD genèric aquests models de
% dades s'entendrien com a tipus de dades.







\section{Introducció a les sèries temporals}


Una sèrie temporal és una representació per a unes variables o
magnituds físiques que evolucionen al llarg del temps.  En els models
usarem les sèries temporals des de la visió més genèrica possible, per
tant considerem una sèrie temporal com a conjunt de dades que s'han
adquirit en uns certs instants de temps.  En aquest sentit, les sèries
temporals poden representar dades molt variades i que pertanyen a
àmbits molt diferents.


La variació en el temps de magnituds com a sèries temporals són
estudiades en altres teories com per exemple la teoria del senyal. 
% ; de
% fet les sèries temporals són una de les eines en aquesta teoria.
% Això
% no obstant, en les anàlisis aquests senyals normalment s'assumeixen
% com a periòdics, afitats en freqüència, valors adquirits com a
% seqüències equiespaiades i amb una forta anàlisi en els components
% freqüencials. %
% L'aproximació que presentem en sèries temporals és un raonament
% similar més genèric però més propi de l'àlgebra discreta matemàtica
% mentre que la teoria del senyal és més pròpia de l'àmbit del càlcul
% matemàtic.
L'aproximació que presentem de les sèries temporals és un raonament
similar al de les altres teories però des d'un punt de vista més
genèric i més propi de l'àlgebra discreta matemàtica.  És a dir,
les sèries temporals tenen una forma més genèrica on, per exemple, es
 té en compte la posició absoluta en el temps de les mostres o
es pot tolerar l'inframostreig.  Això no treu, però, que quan una
sèrie temporal compleix amb els paràmetres de senyal digital, les
operacions més adients a aplicar-hi siguin les del processament
digital del senyal.

%5El teorema de mostreig de Shanon-Nyquist és vital en els senyals digitals?
% Un senyal digital és un senyal (una quantitat que té informació) representat per una seqüència de valors discrets de la quantitat. Un senyal analògic és un senyal representat per una quantitat que varia contínuament.



Així doncs, l'estudi genèric proposat de les sèries temporals no
pretén substituir aquests estudis propis de cada àmbit sinó que pretén
oferir una visió més àmplia i comuna a totes aquestes dades i
complementar-lo amb aquelles dades que no tenen un comportament
clarament definit. Aquest és el cas, per exemple, de les dades
adquirides en monitoratges en entorns no controlats d'una variable
física, on aquestes variables són aleatòries i el temps d'adquisició pot
ser irregular, i per tant cal estudiar-les com a sèries temporals
genèriques.  Cal dir, que a vegades les sèries temporals es redueixen
a seqüències temporals, és a dir a estudis de dades on només importa
l'ordre en què s'han adquirit i el període d'adquisició es
constant. No pretenem fer aquesta reducció sinó que tractem les sèries
temporals des del punt de vista més genèric on cal saber també la
posició de temps absoluta que ocupen i la distància de temps entre els
valors.  Aquests estudis més particulars de les sèries temporals, els
quals es focalitzen i simplifiquen algunes propietats, permeten
concentrar-se més en àmbits específics i oferir solucions molt ben
raonades.  Per tant és interessant poder incorporar aquests estudis en
els models, en aquest sentit per exemple utilitzarem conceptes de la
teoria del senyal per a interpretar propietats de les sèries
temporals.




\paragraph{Interpretació de la sèrie temporal.} 
La interpretació genèrica d'una sèrie temporal és un conjunt de
predicats `en el temps $t$ la variable observada té el valor $v$'
pertanyents a una mateixa variable o fenomen físic.  De forma més
particular, i en una interpretació més lligada a l'adquisició i
monitoratge continu de fenòmens, una sèrie temporal indica la mesura
d'un valor en un temps, és a dir que el fet que es constata com a cert
és que segons un rellotge i un aparell de mesura s'ha adquirit una
parella de temps i valor.  


El models de \gls{SGBD}, com el model relacional, defineixen la
metodologia per tal d'assegurar la correctesa en la inferència
d'informació a partir dels fets que es donen com a certs. Alhora es
donen com a falsos els fets que no són constatats; és a dir que si en
una sèrie temporal hi apareix una mesura en un temps $t$ de valor $v$
significa alhora que és fals que en aquell instant s'ha mesurat un
altre valor diferent de $v$. Particularment, en el cas que en una
sèrie temporal no hi apareix un temps $t$ significa que és fals que en
aquell instant s'ha mesurat qualsevol valor; així, si s'ha mesurat
però s'ha obtingut un valor erroni aleshores hauria d'aparèixer marcat
amb un valor especial.

Atesa aquesta interpretació de valor adquirit en un instant de
temps per a una mateixa variable observada, en una sèrie temporal no
hi pot haver instants de temps repetits. Altrament, no tindria sentit
que un mateix aparell hagués mesurat alhora dos valors diferents en el
mateix instant.
  


% En alguns casos la interpretació, a més, es pot particularitzar,
% com en els casos descrits de senyal i so. O al revés, unes dades
% que es consideraven un senyal de so es tracten genèricament com a
% sèries temporals.

% Particularment, en els casos similars a la consulta de les mitjanes
% mensuals de temperatures i a les particularitzacions amb seqüències,
% el temps és discret i no fa tanta referència a un posicionament
% absolut sobre la línia temporal.



%\paragraph{Modelització de la realitat amb sèries temporals.} %?
\paragraph{Semàntica de les sèries temporals.}
Les sèries temporals s'utilitzen per a modelitzar aspectes de la
realitat i per tant per a cada cas cal estudiar-ne l'adequació.  Això
no obstant, en el nostre cas ens centrem en els aspectes formals dels
models, tot i que cal exposar alguns aspectes semàntics per a poder
comprendre'n la utilitat. Així, cal tenir en compte dues
consideracions.


Per una banda, en el context del monitoratge, per a estudiar la
deducció d'informació sobre la variable mesurada a partir del procés
d'adquisició s'ha de complementar amb les teories adients.  Per
exemple si l'aparell de mesura està avariat la informació inferida en
el \gls{SGST} serà certa des del punt de vista que aquell és el valor
mesurat per l'aparell però, evidentment, no serà cert que la variable
hagi tingut aquell valor. De fet, aquest és el principi d'inferència
d'informació que segueix el model relacional de bases de dades: a
partir dels fet que es donen com a certs estableixen com a certa la
informació que s'infereix, els models asseguren que aquest raonament
sigui correcte, però responsabilitzen a l'usuari d'interpretar si
aquella operació efectivament es correspon amb la informació que vol
inferir. És a dir, els operadors només tenen en compte l'estructura de
les dades mentre que el significat contextualitzat dels operadors és
extern al model; com de fet és el cas d'altres àlgebres que tampoc no
defineixen com s'han d'interpretar la validesa dels resultats.

%per exemple pensem en el cas de la suma on
%  la definició matemàtica no explica com s'ha d''interpretar la
%  validesa del resultat)

Per altra banda, no totes les sèries temporals són adquirides
directament, car poden ser resultat d'operacions amb altres sèries
temporals o resultat de consultes on el temps sigui una variable, com
per exemple consultar les mitjanes mensuals de temperatures. %
 %(p.ex també l'evolució al llarg del temps de la  quantitat de visites a un portal segons el país d'origen)
En aquests casos també és important no perdre de vista la
interpretació de la validesa dels resultats.




Així doncs, en els models no incloem l'etapa d'adquisició ni de
mostreig de les sèries temporals sinó que partim del fet que les
mesures ja han estat capturades i s'ha raonat sobre l'adequació
d'aquest procés. Això no obstant, cal notar que alguns \gls{SGST}
proposen el fet d'influir sobre el procés d'adquisició a partir de la
informació gestionada \parencite{madden05}, per exemple per poder
so\l.licitar d'obtenir més mostres si s'observen variacions estranyes
o per reduir la freqüència de mostreig si es considera que el sistema
té un estat estable.






En resum, les sèries temporals tenen un atribut, l'instant de temps,
que ofereix unes particularitats a l'estructura de dades i amb què cal
operar coherentment. Tant els instants de temps com els valors poden
ser de qualsevol tipus, tot i així els exemplificarem amb nombres
reals per tal de facilitar-ne la comprensió i per ser més propers a
les anàlisi de sèries temporals basades en variables
numèriques \parencite{last04:book}.


% * El temps és un nom donat al camp, qualsevol objecte que tingui la mateixa interfície que el temps pot funcionar. En el cas del valor pot ser qualsevol objecte, s'exemplifica amb reals per facilitar-ne la comprensió i per ser el més proper al time series analysis: statistical methods focused on sequences of values representing a single numeric variable [llibre-last].






\section{Introducció a la multiresolució}

La multiresolució és una tècnica que s'aplica a una sèrie temporal
per tal de compactar-ne i resumir-ne certa informació.  Bàsicament la
multiresolució consisteix a calcular un conjunt de resolucions d'una
sèrie temporal on cada resolució consisteix en aplicar una funció
d'agregació a les mesures cada cert període de temps. A més, cada
resolució inclou un paràmetre per tal d'afitar el nombre de valors
emmagatzemats.  

La idea bàsica és utilitzar la multiresolució per a descriure sèries
temporals de forma que hi hagi més resolució per a les dades més
recents i menys resolució per a les dades més antigues. Tot i així, es
podrien establir variacions d'aquesta estructura mitjançant les quals,
per exemple, es pogués retenir una resolució per a un període temporal
que ha resultat interessant o que calgués investigar més profundament.
Aquesta idea de multiresolució prové de l'\gls{SGBD}
RRDtool \parencite{rrdtool}, del qual en estudis anteriors n'hem
analitzat profundament els conceptes i n'hem abstret i formalitzat les
característiques essencials \parencite{llusa11:tfm,llusa12:ptd}.  Els
objectius principals són la formalització d'un model abstracte de
\gls{SGBD} per a la multiresolució i la inclusió de conceptes més
genèrics per tal de descriure els \gls{SGSTM} contextualitzats en els
\gls{SGST}.

% En el mateix sentit que hem comparat les sèries temporals i els
% senyals digitals, en altres àmbits també s'utilitzen tècniques amb
% propòsits semblants als de la multiresolució.  Per exemple en el
% processament del senyal digital s'utilitzen anàlisis temps-freqüència
% mitjançant Fourier o \emph{wavelets} per tal d'identificar atributs en
% els senyals. També com en el cas per les sèries temporals, aquestes
% anàlisis del senyal tenen un fort component freqüencial i en canvi
% proposem la multiresolció des d'un punt de vista més genèric.  Això no
% treu, però, que fóra interessant estudiar en els \gls{SGSTM}
% agregacions en el domini freqüencial.




La multiresolució aplicada a una sèrie temporal implica una selecció
d'informació i per tant és alhora una compressió de dades amb
pèrdua. Abans d'aplicar la multiresolució cal decidir quins atributs
se seleccionaran mitjançant uns paràmetres, els quals bàsicament són les
definicions del períodes de temps, les funcions que han d'agregar els atributs
i el nombre màxim de valors que s'han d'emmagatzemar.  Així doncs,
aplicar la multiresolució ha de ser una decisió consensuada, s'ha de
tenir en compte que és una compressió amb pèrdua i s'ha de pensar
adequadament la configuració dels paràmetres. O, dit d'una altra
manera, l'usuari ha de ser conscient que vol gestionar les sèries
temporals amb multiresolució; en certa manera un sistema no pot
decidir autònomament d'utilitzar la multiresolució com a equivalent a
la sèrie temporal original sense avisar l'usuari.
%
Com en els \gls{SGST}, en el cas dels \gls{SGSTM} l'usuari també ha
d'interpretar la validesa dels resultats. En aquest cas, però, la
multiresolució produeix l'efecte de compressió amb pèrdua que
estudiarem amb més detall en
\textref{sec:multiresolucio:teoriainformacio}.



El model que presentem d'\gls{SGSTM} defineix els termes i conceptes
de la multiresolució i els operadors genèrics que hi treballen. En els
capítols posteriors al model d'\gls{SGSTM}, comentarem aplicacions i variacions interessants dels \gls{SGSTM}.






\subsection{Característiques de la multiresolució}

Un \gls{SGSTM} és un \gls{SGST} amb capacitats de multiresolució.  A
continuació resumim les característiques que els \gls{SGSTM} milloren
respecte als \gls{SGST}.

\begin{itemize}

\item Gran volum de dades. Els sistemes de monitoratge adquireixen una
  gran quantitat de dades dels sensors. Aquestes dades contenen
  informació que ha de ser observada, tant en línia amb l'adquisició
  com en diferit, i per tal de poder processar-la cal reduir el volum
  de dades. Una de les característiques de la multiresolució és la
  selecció i l'emmagatzematge dels segments més interessants de les
  dades. Aquests segments són el conjunt de resolucions per a cada
  sèrie temporal que l'usuari pot configurar com extreure i resumir
  mitjançant diversos períodes de temps i funcions. En la
  multiresolució la mida de les sèries temporals queda afitada, cosa
  que és útil per a sistemes d'emmagatzematge que necessiten
  controlar-ne l'espai.


\item Visualització. La multiresolució també és útil quan es
  visualitzen gràficament les sèries temporals ja que permet a
  l'usuari seleccionar el millor rang temporal i la resolució que
  s'adeqüen a la pantalla. No cal processar amb més quantitat de dades que
  la que realment es pot mostrar.


\item Validació de dades. Els sistemes de monitoratge adquireixen
  dades però poden ocórrer alguns problemes que tenen efecte en el
  procés posterior d'anàlisi de les sèries temporals. Un dels
  principals problemes ocorre quan els monitors no poden adquirir una
  dada, cosa que es coneix com a forats, o bé quan adquireixen una
  dada erròniament, com per exemple les dades atípiques o
  aberrants \parencite{quevedo10}.  Les funcions d'agregació
  d'atributs de la multiresolució poden cooperar en la validació, el
  filtratge i la reconstrucció d'aquestes dades desconegudes per tal
  de conservar històrics consistents.

\item Regularització de les sèries temporals. Un altre efecte
  secundari del monitoratge ocorre quan el període de mostreig no és
  constant, és a dir quan les dades resultants no estant equiespaiades
  en el temps. Aquestes no regularitats poden provenir de fluctuacions
  en el rellotge de mostrejos periòdics o bé de mostrejos no periòdics
  basats en esdeveniments \parencite{kopetz11:realtime}. Un dels
  objectius de la multiresolució és regularitzar els intervals de
  temps de la sèrie temporal, és a dir que les resolucions que en
  resulten són regulars en el temps. Aquest procés de regularització
  és útil per a aplicar posteriorment algoritmes d'anàlisi de sèries
  temporals que assumeixen que les sèries temporals són regulars, però
  també per a calcular altres resolucions de la sèrie temporal com per
  exemple observar dades periòdiques amb intervals de mes o d'any.


\item Resum de la informació.  La multiresolució extreu i selecciona
  les característiques desitjades de les dades mitjançant diverses
  funcions d'agregació d'atributs. Per tant, emmagatzema resumidament la
  informació que l'usuari posteriorment pot consultar. Això no
  obstant, aquesta selecció de la informació s'ha de determinar a
  priori considerant el context en què les consultes futures s'hauran
  de realitzar.


\item Computació en flux. Els resums d'informació que resulten de la
  multiresolució són cars de computar, sobretot si s'han emmagatzemat
  grans quantitats de sèries temporals o bé si es té en compte que en
  monitoratges periòdics constantment arriben dades noves. El
  model de \gls{SGSTM} permet calcular la multiresolució en flux amb
  l'adquisició de dades, és a dir al mateix temps que arriben noves
  mesures de la sèrie temporal computa incrementalment el nou resultat
  de multiresolució. D'aquesta manera el temps de còmput es pot
  repartir des del moment d'adquisició fins al moment de la
  consulta. Això, però, requereix que les mesures s'insereixin amb
  ordre als \gls{SGSTM}.

\item Computació distribuïda i para\l.lela. En el cas que es vulgui
  computar la multiresolució en temps diferit del procés d'adquisició
  de dades, és a dir que es vulgui emmagatzemar tota la sèrie temporal
  i després computar amb totes les dades, cal una computació
  intensiva. A tal efecte, la multiresolució també es pot computar
  distribuïdament i para\l.elament en diversos nodes de computació.

 
\end{itemize}


Tot i així, també pot ser útil de complementar els \gls{SGSTM} amb les
capacitats d'altres \gls{SGBD}. Per una banda, es poden usar per a
emmagatzemar els valors originals en qualitat de dipòsit a llarg
termini en cas que calgui realitzar alguna consulta imprevista en
temps diferit.  Par altra banda, s'hi pot emmagatzemar
informació relacionada amb les sèries temporals com per exemple les
unitats dels valors, la localització del sensor, etiquetes de
classificació, darrer valor mesurat, etc.


% Però el model de SGSTM també es pot fer servir per altres aplicacions:

% * Regularitzar en línia (temps real) una sèrie temporal en diferents períodes de mostreig

% * Tenir unes vistes (consultes) a punt (ja processades) amb diferents resolucions d'una sèrie temporal

% * Comprimir per decimació (downsampling) o bé farcir forats (reconstrucció del senyal)


% Tres possibles camps d'aplicació de la multiresolució: comptadors (conservar els totals), soroll d'un senyal (conservar la mitjana), temperatura (aplicar a priori la DFT i determinar deltes bons per a després poder interpolar mitjanes).





\subsection{Motivació per a la multiresolució}


A continuació mostrem la motivació per a la multiresolució mitjançant
dos exemples: \textref{fig:model:multiresolucio-motivacio} i
\textref{fig:model:multiresolucio-motivacio-irregular}. Les figures
mostren el càlcul de la multiresolució per a una sèrie temporal
regular i una d'irregular, respectivament.  Assumim que les figures
mostren una instantània entre els instants de temps nou i deu.


A la part superior de les figures hi ha el gràfic d'una sèrie temporal
en què l'eix de temps té qualssevol unitats de temps
(\gls{unitatstemps}) i l'eix de valors qualssevol unitats. És una
sèrie temporal senzilla que val 1 en l'instant -1, 0 en el 0, 1 en
l'1, 2 en el 2, etc.  L'eix vertical \emph{ara} indica l'instant en
què s'ha pres la instantània, així que el temps anterior és el passat
i el temps posterior és el futur, els esdeveniments del qual estan
pintats en gris. L'eix \emph{inici} indica l'instant zero u.t. en què
el sistema ha començat a adquirir dades, així les dades anteriors són
desconegudes.


\begin{figure}[tp]
  \centering
  %\usetikzlibrary{positioning}
\begin{tikzpicture}%[scale=0.77, every node/.style={transform shape}]

  %referencia
  \node (-6) {};

  \foreach \x in {-5,...,12}
  {
    \pgfkeys{/pgf/number format/.cd,int trunc}
    \pgfmathparse{abs(\x)}
    \let\absx=\pgfmathresult
    \pgfmathparse{\x-1}
    \let\antx=\pgfmathresult
    %time
    \node[node distance=1mm] (\x) [right=of \antx] 
    {\ifnum\x<11 \x \else \phantom{9} \fi};

    %graph values
    \node [above=\absx mm of \x] 
    {\ifnum\x=10 \color{gray} \fi \ifnum\x<11 $\bullet$ \fi};    

    %values
    \node[rectangle,draw] (s\x) [below=of \x] 
    {\ifnum\x<10 \pgfmathprintnumber{\absx} \else \phantom{9} \fi};
  }

  \node [below=of 10] {\color{gray}10}; 
  

  
  %rd: 5s |inf| mean
  \node [circle,draw] (rd5-5) [below=3cm of s-5] {u};
  \node [circle,draw] (rd50) [below=3cm of s0] {u};
  \node [circle,draw] (rd55) [below=3cm of s5] {3};
  \node [circle,draw] (rd510) [below=3cm of s10] {u};
  \node [below=3.3cm of s10] {\color{gray}8};
 
  \draw[->,bend right] (s5) to (rd55);
  \draw[->,bend right] (s4) to (rd55);
  \draw[->,bend right] (s3) to (rd55);
  \draw[->,bend right] (s2) to (rd55);
  \draw[->,bend right] (s1) to (rd55);

  \draw[->,dotted,bend right] (s10) to (rd510);
  \draw[->,bend right] (s9) to (rd510);
  \draw[->,bend right] (s8) to (rd510);
  \draw[->,bend right] (s7) to (rd510);
  \draw[->,bend right] (s6) to (rd510);

  
  %rd: 3s |inf| mean
  \node [circle,draw] (rd3-3) [below=of s-3] {u};
  \node [circle,draw] (rd30) [below=of s0] {u};
  \node [circle,draw,fill=white] (rd33) [below=of s3] {2};
  \node [circle,draw,fill=white] (rd36) [below=of s6] {5};
  \node [circle,draw,fill=white] (rd39) [below=of s9] {8};
  \node [circle,draw] (rd312) [below=of s12] {u};

  \draw[->] (s3) to (rd33);
  \draw[->] (s2) to (rd33);
  \draw[->] (s1) to (rd33);

  \draw[->] (s6) to (rd36);
  \draw[->] (s5) to (rd36);
  \draw[->] (s4) to (rd36);

  \draw[->] (s9) to (rd39);
  \draw[->] (s8) to (rd39);
  \draw[->] (s7) to (rd39);

  \draw[->,dotted] (s12) to (rd312);
  \draw[->,dotted] (s11) to (rd312);
  \draw[->,dotted] (s10) to (rd312);



  %eixos
  \node (et0) [above=1mm of -5] {};
  \node (et12) [above=1mm of 11] {};
  \node [right=-2mm of et12] {temps};
  \draw[->] (et0) to (et12);
  \node (y5) [above=5mm of 0] {--};
  \node [left=-1.5mm of y5] {5};
  \node (y10) [above=10mm of 0] {--};
  \node [left=-1.5mm of y10] {10};

  \node (inici) [above=4cm of s0] {inici};
  \node (inici2) [below=3.8cm of s0] {};
  \draw[-,dotted] (inici) to (inici2);

  \node (fi) [above=4.4cm of s9.east] {ara};
  \node (fi2) [below=4.2cm of s9.east] {};
  \draw[-,dotted] (fi) to (fi2);


  \node (fut) [below right=1mm and 1mm of fi] {futur};
  \draw[->] (fut.south west) to (fut.south east);

  \node (pas) [below left=1mm and 1mm of fi] {passat};
  \draw[->] (pas.south east) to (pas.south west);

  \node (unk) [below left=1mm and 1mm of inici] {desconegut};
  \draw[->] (unk.south east) to (unk.south west);



  \node [above=0cm of s-5] {\makebox[0cm][l]{mostra cada 1 u.t.}};
  \node [below=0.5cm of s-5] {\makebox[0cm][l]{mitjana cada 3 u.t.}};
  \node [below=2.5cm of s-5] {\makebox[0cm][l]{mitjana cada 5 u.t.}};


\end{tikzpicture}



%%% Local Variables:
%%% TeX-master: "../main"
%%% ispell-local-dictionary: "british"
%%% End:

  \caption{Diagrama d'una instantània en la multiresolució d'una sèrie temporal amb mostreig regular}
  \label{fig:model:multiresolucio-motivacio}
\end{figure}


\begin{figure}[tp]
  \centering
  %\usetikzlibrary{positioning}
\begin{tikzpicture}

  \node[node distance=1mm] (0) {0};
  \node[node distance=1mm] (-1) [left=of 0]{\phantom{9}};
  \node[node distance=1mm] (1) [right=of 0] {\phantom{1}};
  \node[node distance=1mm] (2) [right=of 1] {2};
  \node[node distance=1mm] (3) [right=of 2] {\phantom{3}};
  \node[node distance=1mm] (4) [right=of 3] {4};
  \node[node distance=1mm] (5) [right=of 4] {\phantom{5}};
  \node[node distance=1mm] (6) [right=of 5] {6};
  \node[node distance=1mm] (7) [right=of 6] {\phantom{7}};
  \node[node distance=1mm] (8) [right=of 7] {8};
  \node[node distance=1mm] (9) [right=of 8] {\phantom{9}};
  \node[node distance=1mm] (10) [right=of 9] {10};
  \node[node distance=1mm] (11) [right=of 10] {\phantom{9}};
  \node[node distance=1mm] (12) [right=of 11] {\phantom{9}};


  \node [above=0 mm of 0] {$\bullet$}; 
  \node [above=2 mm of 2] (v2) {$\bullet$}; 
  \node [above=4 mm of 4] {?}; 
  \node [above=6 mm of 6] (v6) {$\bullet$}; 
  \node [above=7 mm of 7] {$\bullet$}; 
  \node [above=9 mm of 9] {$\bullet$}; 
  \node [above=10 mm of 10] (v10) {\color{gray} $\bullet$}; 


  \node[rectangle,draw] (s0) [below=of 0] {0};
  \node[rectangle,draw] (s2) [below=of 2] {2};
  \node[rectangle,draw] (s4) [below=of 4] {u};
  \node[rectangle,draw] (s6) [below=of 6] {6};
  \node[rectangle,draw] (s7) [below=of 7] {7};
  \node[rectangle,draw] (s9) [below=of 9] {9};
  \node[rectangle,draw] (s10) [below=of 10] {\color{gray}10};
  \node[rectangle,draw] (s11) [below=of 11] {\phantom{9}};
  \node[rectangle,draw] (s12) [below=of 12] {\phantom{9}};


  \draw[<->] (v2.north east) to (v6.north west)
  node [above,sloped,midway] {\small gap};

  \draw[<->] (v6.south east) to (v10.south west)
  node [below,sloped,midway] {\small disrupt};

  
  %rd: 5s |inf| mean
  \node [circle,draw] (rd50) [below=4cm of 0] {u};
  \node [circle,draw] (rd55) [below=4cm of 5] {2};
  \node [circle,draw] (rd510) [below=4cm of 10] {u};
  \node [below=4.3cm of 10] {\color{gray}8};
 
  \draw[->,bend right] (s4) to (rd55);
  \draw[->,bend right] (s2) to (rd55);

  \draw[->,dotted,bend right] (s10) to (rd510);
  \draw[->,bend right] (s9) to (rd510);
  \draw[->,bend right] (s7) to (rd510);
  \draw[->,bend right] (s6) to (rd510);

  
  %rd: 3s |inf| mean
  \node [circle,draw] (rd30) [below=of s0] {u};
  \node [circle,draw,fill=white] (rd33) [below=2.5cm of 3] {2};
  \node [circle,draw,fill=white] (rd36) [below=2.5cm of 6] {6};
  \node [circle,draw,fill=white] (rd39) [below=2.5cm of 9] {8};
  \node [circle,draw] (rd312) [below=2.5cm of 12] {u};

  \draw[->] (s2) to (rd33);

  \draw[->] (s6) to (rd36);
  \draw[->] (s4) to (rd36);

  \draw[->] (s9) to (rd39);
  \draw[->] (s7) to (rd39);

  \draw[->,dotted] (s12) to (rd312);
  \draw[->,dotted] (s11) to (rd312);
  \draw[->,dotted] (s10) to (rd312);



  %eixos
  \node (et0) [above=1mm of -1] {};
  \node (et12) [above=1mm of 11] {};
  \node [right=-2mm of et12] {time};
  \draw[->] (et0) to (et12);
  \node (y5) [above=5mm of 0] {--};
  \node [left=-1.5mm of y5] {5};
  \node (y10) [above=10mm of 0] {--};
  \node [left=-1.5mm of y10] {10};

  \node (inici) [above=3.1cm of s0] {init};
  \node (inici2) [below=3.3cm of s0] {};
  \draw[-,dotted] (inici) to (inici2);

  \node (fi) [above=3.4cm of s9.east] {now};
  \node (fi2) [below=3.5cm of s9.east] {};
  \draw[-,dotted] (fi) to (fi2);


  % \node (fut) [below right=1mm and 1mm of fi] {future};
  % \draw[->] (fut.south west) to (fut.south east);

  % \node (pas) [below left=1mm and 1mm of fi] {past};
  % \draw[->] (pas.south east) to (pas.south west);

  \node [above=0cm of s0] {\makebox[0.5cm][l]{sample every 2 u.t.}};
  \node [below=0.5cm of s0] {\makebox[0.5cm][l]{mean/3}};
  \node [below=2cm of s0] {\makebox[0.5cm][l]{mean/5}};

\end{tikzpicture}



%%% Local Variables:
%%% TeX-master: "../main"
%%% ispell-local-dictionary: "british"
%%% End:

  \caption{Diagrama d'una instantània en la multiresolució d'una sèrie temporal amb mostreig irregular}
  \label{fig:model:multiresolucio-motivacio-irregular}
\end{figure}




A la part inferior de les figures hi ha un diagrama que mostra l'acció
de la multiresolució.  La primera línia mostra enquadrades els valors
numèrics adquirits de la sèrie temporal. La segona i la tercera línia
mostren encerclades les dades que s'emmagatzemaran a la base de dades
segons un esquema de multiresolució particular que consisteix en
calcular dues resolucions de la sèrie temporal: una calcula la mitjana
dels valors cada tres \gls{unitatstemps}\ i l'altra calcula la mitjana
cada cinc \gls{unitatstemps} En aquest cas, la mitjana és la funció
que actua com a selector d'informació de la sèrie temporal mitjançant
estadístics d'agregació.  Totes les dades emmagatzemades abans de
l'instant zero són desconegudes (\emph{u}) i totes les dades futures
també són marcades com a desconegudes fins que el temps avanci, tot i
que en alguns casos en gris mostrem els valors que prendran.


A \textref{fig:model:multiresolucio-motivacio-irregular} la sèrie temporal s'ha
adquirit cada una \gls{unitatstemps}\ de forma regular, és a dir que
hi ha valors per a cada adquisició. Les fletxes mostren com es
resumeixen les dades adquirides per tal d'emmagatzemar-les: així de
cada tres mostres se n'emmagatzema la mitjana i, independentment, cada
cinc mostres se n'emmagatzema una altra mitjana.  Per als valors
futurs, quan el temps avanci una \gls{unitatstemps}\ aleshores
s'adquirirà el valor 10 i es podrà calcular la mitjana de cada 5
\gls{unitatstemps}\ per a l'instant 10, la qual resultarà en el valor
8, però no es podrà calcular la mitjana de cada 3 \gls{unitatstemps}\
fins a l'instant 12.



A \textref{fig:model:multiresolucio-motivacio} la sèrie temporal s'ha
adquirit cada dues \gls{unitatstemps}\ de forma irregular, és a dir
que manquen valors en alguna adquisició, cosa que es marca com un
forat i un valors adquirit desconegut, o bé l'adquisició no s'ha fet
exactament cada dues \gls{unitatstemps}, cosa que es marca com una
ruptura. Aquests són dos exemples de possibles problemes en el
monitoratge, és a dir que es volia mostrejar cada dues
\gls{unitatstemps}\ però per alguna raó no s'ha pogut fer en l'instant
4 i al voltant de l'instant 8 hi ha hagut una ruptura en el rellotge
que ha adquirit en els instants 7 i 9.  L'esquema de multiresolució
emmagatzemat té la mateixa forma que en el cas regular, és a dir sense
que hi afectin les irregularitats. Aquí, les fletxes mostren com es
calcula la mitjana a partir dels valors adquirits, és a dir que ara no
hi ha ni tres ni cinc valors disponibles per a fer la mitjana de cada
resolució. Alguns valors emmagatzemats coincideixen però altres
difereixen, especialment quan la mitjana opera amb valors desconeguts
dels quals en aquest exemple operem sense tenir-los en compte; en els
models detallarem més bé aquests casos.








%%% Local Variables:
%%% TeX-master: "main"
%%% End:
% LocalWords: buffer buffers multiresolució SGSTM SGST






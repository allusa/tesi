\section{Model d'operacions}
\label{sec:model:sgst-operacions}

En aquesta secció definim les operacions d'un \gls{SGST} que permeten
manipular les sèries temporals.  Una sèrie temporal té un atribut de
temps que ha de ser tingut en compte pels operadors que la manipulin.
Així, atenent a aquest atribut de temps, el comportament d'una sèrie
temporal pot tenir naturaleses diferents:
\begin{itemize}
\item Conjunt, és a dir els operadors només atenent a la forma
  estructural bàsica.
\item Seqüència, en la qual els operadors la tracten com a conjunts
  amb ordre.
\item Funció temporal, en la qual els operadors assumeixen que una
  sèrie temporal és la representació d'una funció temporal.
\end{itemize}



En el disseny del model d'operacions següent es distingeix el
comportament per als tres casos anteriors.  Es dissenyen les
operacions bàsiques que permeten que posteriorment es combinin per
a elaborar-ne de més complexes.


Les manipulacions de les sèries temporals es defineixen abstractament
per a qualsevol sèrie temporal que tingui l'estructura de \gls{SGST}.
Les definicions dels operadors avaluen els conceptes algebraics i
lògics de les dades però no avaluen la semàntica en un context
particular, com també ocorre en el model d'operacions del
model relacional. És a dir, en cada context particular de manipulació
d'una sèrie temporal s'ha de decidir si aquella àlgebra té significat
o, al contrari, no pot ser aplicada. Per exemple una suma de valors de
diferents unitats podria ser semànticament errònia. A
la \autoref{sec:sgst:propietats} estudiem el significat d'algunes
propietats de les sèries temporals.


% es treballa amb la forma canònica de les sèries temporals llevat que
% s'indiqui el contrari. 



\subsection{Bàsiques de conjunts}
\glsaddsec{not:op-conjunts} %%%%secció d'operacions

En el model estructural d'\gls{SGST} hem definit les sèries temporals
utilitzant conjunts. En aquest apartat definim operadors per a les
sèries temporals recollint els operadors habituals que tenen els
conjunts.   

El model relacional d'\gls{SGBDR} defineix els seus operadors bàsics
a partir de l'àlgebra de
conjunts \parencite[cap.~7]{date04:introduction8}. En aquest apartat
apliquem el mateix estudi per al model d'\gls{SGST}. Tot i així de
manera simplificada, a les definicions no es descriuen les sèries
temporals com a relacions amb capçaleres sinó que se n'escriuen només
els conjunts de valors. Seguint el model relacional es poden estendre
les definicions i introduir el model complet de relacions.



En les operacions binàries de sèries temporals, entre les mesures
d'ambdós conjunts es poden aplicar les dues relacions d'ordre de la
\autoref{def:model:mesura-relacio-ordre}, és a dir poden tenir ordre
semitemporal o temporal. Com a conseqüència, aquests dos ordres
indueixen dues definicions per a alguns operadors de conjunts. Als
operadors amb ordre temporal els anomenarem temporals i hi afegirem un
superíndex $t$ per a indicar-ho.


Les operacions que agrupen els elements dels conjunts habitualment se
solen representar gràficament mitjançant diagrames de Venn. A la
\autoref{fig:model:sgst:venn} es mostren els diagrames de Venn per a
cinc de les operacions dels \gls{SGSTM} que descrivim a continuació:
inclusió, unió, diferència, intersecció i diferència simètrica; tant
en la seva vessant parcial com en la seva vessant temporal.  

Per a dibuixar aquest diagrames de Venn, a banda del conjunt
corresponent a cada sèrie temporal $S_1$ i $S_2$ i la seva intersecció,
cal dibuixar uns subconjunts $T_1$ i $T_2$ que indiquen les mesures
que comparteixen el mateix temps amb una altra mesura de l'altre
conjunt però no el mateix valor. És a dir $T_1 = \{ m
\glsdisp{not::}{|} m \in S_1 \wedge (\exists n \in S_2: m =^t n) \}$ i
$T_2 = \{ m | m \in S_2 \wedge (\exists n \in S_1: m =^t n) \}$.  Aquests dos
subconjunts $T_1$ i $T_2$ són importants per a les operacions dels
\gls{SGST} perquè no hi pot haver cap sèrie temporal resultant que els
inclogui a tots dos, car significaria que conté temps repetits. Per
exemple, una operació que tingui la sèrie resultant $T_1 \cup T_2$,
com es mostra a la \autoref{fig:model:sgst:venn-impossible}, és
impossible


\begin{figure}[tp]
  \centering
  \def\escala{0.7}

\def\nodeA{node [anchor=east] {$A$}}
\def\nodeB{node [anchor=west] {$B$}}
\def\nodeT{node [left=0.4cm] {\tiny $t_A$} node [right=0.4cm] {\tiny $t_B$}}
% Definition of circles
\def\firstcircle{(0,0) circle (1.5cm)}
\def\secondcircle{(0:2cm) circle (1.5cm)}
\def\thirdcircle{(0:1cm) circle (1.11cm)}

\colorlet{circle edge}{blue!50}
\colorlet{circle area}{blue!20}

\tikzset{
  filled/.style={fill=circle area, draw=circle edge, thick},
  outline/.style={draw=circle edge, thick},
  every node/.style={transform shape}
}

%\setlength{\parskip}{5mm}



%Set A in B
\begin{tikzpicture}[scale=\escala]
    \begin{scope}
        \clip \secondcircle;
        \draw[even odd rule,blue] \firstcircle \nodeA
                                 \secondcircle ;
                                 %\thirdcircle;
            \fill[filled] \firstcircle;
   \end{scope}
      \draw[outline] \secondcircle \nodeB;

   \node[anchor=south] at (current bounding box.north) {$A \subset B$};
   \node[anchor=west] at (current bounding box.west) {$A$};
\end{tikzpicture}
%Set temporal A in B
\begin{tikzpicture}[scale=\escala]
    \begin{scope}
        \clip \firstcircle;
        \fill[filled] \thirdcircle;
      \draw[outline] \thirdcircle \nodeT;
    \end{scope}
    \begin{scope}
        \clip \secondcircle;
        \draw \thirdcircle \nodeT;
    \end{scope}
      \draw[circle edge] \thirdcircle;
    \begin{scope}
        \clip \secondcircle;
        \draw[even odd rule,blue] \firstcircle \nodeA
                                 \secondcircle ;
                                 %\thirdcircle;
            \draw[outline] \firstcircle;
   \end{scope}
      \draw[outline] \secondcircle \nodeB;

   \node[anchor=south] at (current bounding box.north) {$A \subset^t B$};
   \node[anchor=east] at (current bounding box.center) {$A$};
\end{tikzpicture}







%Set A or B
\begin{tikzpicture}[scale=\escala]
  \draw[filled] \firstcircle \nodeA;
    \begin{scope}
        \clip \secondcircle;
        \draw[filled, even odd rule] \firstcircle \nodeA
                                 \secondcircle 
                                 \thirdcircle;
   \end{scope}
    \draw[outline] \firstcircle
                   \secondcircle \nodeB
                   \thirdcircle \nodeT;

   \node[anchor=south] at (current bounding box.north) {$A \cup B$};
\end{tikzpicture}
%Set temporal A or B
\begin{tikzpicture}[scale=\escala]
    \draw[filled, even odd rule] \firstcircle \nodeA
                                 \secondcircle \nodeB
                                 \thirdcircle \nodeT;
    \node[anchor=south] at (current bounding box.north) {$A \cup^t B$};
\end{tikzpicture}




% Set A but not B
\begin{tikzpicture}[scale=\escala]
    \begin{scope}
        \clip \firstcircle;
        \draw[filled, even odd rule] \firstcircle \nodeA
                                     \secondcircle;

    \end{scope}
    \draw[outline] \firstcircle
                   \secondcircle \nodeB
                   \thirdcircle \nodeT;
    \node[anchor=south] at (current bounding box.north) {$A - B$};
\end{tikzpicture}
% Set temporal A but not B
\begin{tikzpicture}[scale=\escala]
    \begin{scope}
        \clip \firstcircle;
        \draw[filled, even odd rule] \firstcircle \nodeA
                                     \thirdcircle;

    \end{scope}
    \draw[outline] \firstcircle
                   \secondcircle \nodeB
                   \thirdcircle \nodeT;
    \node[anchor=south] at (current bounding box.north) {$A -^t B$};
\end{tikzpicture}





% % Set A and B
% \begin{tikzpicture}
%     \begin{scope}
%         \clip \firstcircle;
%         \fill[filled] \secondcircle;
%     \end{scope}
%     \draw[outline] \firstcircle \nodeA;
%     \draw[outline] \secondcircle \nodeB;
%     \draw[outline] \thirdcircle \nodeT;
%     \node[anchor=south] at (current bounding box.north) {$A \cap B$};
% \end{tikzpicture}
% % Set temporal A and B
% \begin{tikzpicture}
%     \begin{scope}
%         \clip \firstcircle;
%         \fill[filled] \thirdcircle;
%     \end{scope}
%     \draw[outline] \firstcircle \nodeA;
%     \draw[outline] \secondcircle \nodeB;
%     \draw[outline] \thirdcircle \nodeT;
%     \node[anchor=south] at (current bounding box.north) {$A \cap^t B$};
% \end{tikzpicture}



% %Set A or B but not (A and B) also known a A xor B
% \begin{tikzpicture}
%     \begin{scope}
%         \clip \firstcircle;
%         \draw[filled, even odd rule] \firstcircle
%                                      \secondcircle;
%     \end{scope}
%     \begin{scope}
%         \clip \secondcircle;
%         \draw[filled, even odd rule] \secondcircle 
%                                      \thirdcircle;

%     \end{scope}
%     \draw[outline] \firstcircle \nodeA;
%     \draw[outline] \secondcircle \nodeB;
%     \draw[outline] \thirdcircle \nodeT;
%     \node[anchor=south] at (current bounding box.north) {$A \ominus B$};
% \end{tikzpicture}
% %Set temporal A or B but not (A and B) also known a A xor B
% \begin{tikzpicture}
%     \begin{scope}
%         \clip \firstcircle;
%         \draw[filled, even odd rule] \firstcircle
%                                      \thirdcircle;
%     \end{scope}
%     \begin{scope}
%         \clip \secondcircle;
%         \draw[filled, even odd rule] \secondcircle 
%                                      \thirdcircle;

%     \end{scope}
%     \draw[outline] \firstcircle \nodeA;
%     \draw[outline] \secondcircle \nodeB;
%     \draw[outline] \thirdcircle \nodeT;
%     \node[anchor=south] at (current bounding box.north) {$A \ominus^t B$};
% \end{tikzpicture}
  \caption{Diagrames de Venn per a les operacions dels \gls{SGSTM}. Els
    subconjunts $T_1$ i $T_2$ indiquen les mesures $m =^t n$.}
  \label{fig:model:sgst:venn}
\end{figure}


\begin{figure}[tp]
  \centering
  \def\nodeA{node [anchor=east] {$S_1$}}
\def\nodeB{node [anchor=west] {$S_2$}}
\def\nodeT{node [left=0.4cm] {\tiny $T_1$} node [right=0.4cm] {\tiny $T_2$}}
% Definition of circles
\def\firstcircle{(0,0) circle (1.5cm)}
\def\secondcircle{(0:2cm) circle (1.5cm)}
\def\thirdcircle{(0:1cm) circle (1.11cm)}

\colorlet{circle edge}{blue!50}
\colorlet{circle area}{blue!20}

\tikzset{filled/.style={fill=circle area, draw=circle edge, thick,color=lightgray},
    outline/.style={draw=circle edge, thick}}


% Set  A impossible B
\begin{tikzpicture}
    \begin{scope}
        \clip \firstcircle;
        \draw[filled, even odd rule] \secondcircle
                                     \thirdcircle;
    \end{scope}
    \begin{scope}
        \clip \secondcircle;
        \draw[filled, even odd rule] \firstcircle
                                     \thirdcircle;
    \end{scope}

    \draw[outline] \firstcircle \nodeA;
    \draw[outline] \secondcircle \nodeB;
    \draw[outline] \thirdcircle \nodeT;
    \node[anchor=south] at (current bounding box.north) {\color{gray}$T_1 \cup T_2$};
\end{tikzpicture}
  \caption{Diagrama de Venn impossible per a les operacions dels
    \gls{SGSTM}. Els subconjunts $T_1$ i $T_2$ indiquen les mesures
    $m =^t n$.}
  \label{fig:model:sgst:venn-impossible}
\end{figure}


\subsubsection{Pertinença i inclusió}


La pertinença determina si un element pertany a un conjunt.  Sigui $S$
una sèrie temporal i $m$ una mesura, es defineix la pertinença de $m$
a $S$ de la forma habitual en els conjunts. Aquesta pertinença es
defineix a partir de l'ordre semitemporal, és a dir que dues mesures
són iguals quan ho són els seus temps i valor.
\begin{definition}[Pertinença]
  Sigui $S$ una sèrie temporal i $m$ una
  mesura, direm que la mesura pertany a la sèrie temporal
  $m \glssymboldef{not:sgst:in} S \iff \exists n \in S :
  T(m) = T(n) \glssymbol{not:wedge} V(m) = V(n)$.
\end{definition}




A partir de l'ordre temporal, és a dir que dues mesures són iguals quan
ho són els seus temps, es defineix la pertinença temporal d'una mesura
a una sèrie temporal.
\begin{definition}[Pertinença temporal]
  Sigui $S$ una sèrie temporal i $m$ una
  mesura, direm que la mesura pertany temporalment a la sèrie temporal
  $m \glssymboldef{not:sgst:int} S \iff \exists n \in S :
  T(m) = T(n)$.
\end{definition}


Si una mesura pertany a una sèrie temporal, $m\in S$, aleshores també
hi pertany temporalment, $m \glssymbol{not:sgst:int} S$.



La inclusió determina si tots els elements d'un conjunt pertanyen a un
altre conjunt. Atenent a la pertinença, es defineix la inclusió d'una
mesura a una sèrie temporal.
\begin{definition}[Inclusió]
  Siguin $S_1$ i $S_2$
  dues sèries temporals, la primera sèrie temporal està inclosa en la
  segona $S_1 \glssymboldef{not:sgst:sub} S_2 \iff \forall
  m \in S_1: m \in S_2$. Aleshores, $S_1$ és una subsèrie temporal de
  $S_2$.
\end{definition}



Atenent a la pertinença temporal, es defineix la inclusió temporal
d'una mesura a una sèrie temporal.
\begin{definition}[Inclusió temporal]
  Siguin $S_1$ i $S_2$
  dues sèries temporals, la primera sèrie temporal està inclosa
  temporalment en la segona $S_1 \glssymboldef{not:sgst:subt} S_2 \iff \forall m \in S_1: m \glssymbol{not:sgst:int} S_2$.
\end{definition}




\begin{example}[Pertinença i inclusió]
  Siguin les mesures $m_1=(1,1)$, $m_2=(3,1)$, $m_3=(3,2)$ i
  $m_4=(4,0)$ i les sèries temporals $S_1=\{m_1,m_2\}$ i
  $S_2=\{m_1,m_3,m_4\}$ aleshores les operacions següents són certes:
  $m_2 \in S_1$, $m_2 \notin S_2$, $m_2 \glssymbol{not:sgst:int} S_1$,
  $m_2 \glssymbol{not:sgst:int} S_2$, $S_1\not\subseteq S_2$ i
  $S_1\glssymbol{not:sgst:subt} S_2$.
\end{example}

\subsubsection{Màxim i suprem}


En una sèrie temporal les mesures tenen relació d'ordre total. Com que
la sèrie temporal s'ha considerat finita i sense elements repetits,
quan la sèrie temporal no és buida això comporta l'existència d'un
màxim i d'un mínim. 

\begin{definition}[Màxim i mínim]
  Sigui $S$ una sèrie temporal. El \emph{màxim} de $S$, notat com a
  $\glssymboldef{not:sgst:max}(S)$, és un element de $S$ tal que
  $\forall m \in S: \max(S) \geq m$.  El \emph{mínim} de $S$, notat com a
  $\glssymboldef{not:sgst:min}(S)$, és un element de $S$ tal que
  $\forall m \in S: \min(S) \leq m$.
\end{definition}

El $\max(S)$ i el $\min(S)$ no estan definits quan la sèrie temporal
és buida: $S= \emptyset$. En canvi, com que el domini de temps un
conjunt tancat, el suprem i l'ínfim estan definits per qualsevol sèrie
temporal. De fet, es poden definir de manera similar al conjunt estès
de nombres reals on $\sup(\emptyset)=-\infty$ i
$\inf(\emptyset)=+\infty$ \parencite{cantrell:extendedreal}.
% A continuació seguim identificant amb
% infinit ($\infty$) els valors impropis que tanquen el domini, tal com
% hem exemplificat en el model estructural.
\begin{definition}[Suprem i
  ínfim]\label{def:sgst:sup}\label{def:sgst:inf}
  Sigui $S$ una sèrie temporal, $m=(-\infty,\infty)$ una mesura
  indefinida negativa i $n=(+\infty,\infty)$ una mesura indefinida
  positiva. 

  El \emph{suprem} de $S$, notat com a
  $\glssymboldef{not:sgst:sup}(S)$, és
  \[
  \sup(S) = 
  \begin{cases}
    m & \text{quan } S=\emptyset\\
    \max(S) & \text{altrament}
  \end{cases}
  \]

  L'\emph{ínfim} de $S$, notat com a
  $\glssymboldef{not:sgst:inf}(S)$, és
  \[
  \inf(S) = 
  \begin{cases}
    n & \text{quan } S=\emptyset\\
    \min(S) & \text{altrament}
  \end{cases}
  \]
\end{definition}

Quan la sèrie temporal no és buida, per
ser un conjunt finit i d'ordre total, sempre hi ha un i només un màxim
i un mínim i per tant es corresponen amb el suprem i l'ínfim
respectivament.


\begin{example}[Mínim i suprem]
  Siguin les sèries temporals $S_1=\{(1,1),m_2=(3,1)\}$ i $S_2=\{\}$
  aleshores les operacions següents són certes:
  $\min(S_1)=\inf(S_1)=(1,1)$ i $\sup(S_2)=(-\infty,\infty)$.
\end{example}





\subsubsection{Unió}


La unió de dos conjunts és un conjunt que conté tots els elements
d'ambdós conjunts.  Per a poder unir dos conjunts amb estructura de
relació, $A \cup B$, cal que tots dos tinguin la mateixa estructura;
és a dir, en termes de \gls{SGBDR} cal que $A$ i $B$ tinguin la
mateixa capçalera.

Per tal que l'operació d'unió de conjunts sigui vàlida per a les
sèries temporals cal, a més, tenir en compte quan dues sèries
temporals tenen mesures en el mateix instant de temps. En cas
d'utilitzar l'operació d'unió de conjunts la sèrie temporal resultant
no compliria amb la \autoref{def:serie_temporal} ja que contindria
mesures amb temps repetits. Com a conseqüència, es defineixen dues
operacions d'unió per a les sèries temporals que resolen la restricció
del temps de forma diferent.  Per a definir ambdues unions cal usar la
pertinença temporal ja que cal treballar amb els conjunts que
comparteixen instants de temps, per tant és difícil establir la
referència de pertinença per a cada una.  Definim la primera unió com
la més propera possible a la unió de conjunts.



En primer lloc, es defineix la unió de dues sèries temporals que
escull les mesures del primer operand en cas de mesures amb el mateix
temps però diferent valor.
\begin{definition}[Unió]
  Siguin $S_1$ i $S_2$ dues sèries temporals en què $\glssymbol{dom}
  S_1 = \glssymbol{dom} S_2$, la unió de les dues sèries temporals,
  notada com a $S_1 \glssymboldef{not:sgst:cup} S_2$, és una sèrie
  temporal que conté totes les mesures de $S_1$ i les mesures de $S_2$
  que no tenen temps repetits: $S_1 \cup S_2 = \{m | m \in S_1
  \glssymbol{not:vee} (m \in S_2 \wedge m \not\glssymbol{not:sgst:int}
  S_1 )\}$.
\end{definition}

Propietats de la unió de sèries temporals:
\begin{itemize}
\item El cardinal de la sèrie temporal resultant està fitat a
  $|S_1| \leq |S| \leq |S_1| + |S_2|$. 
\item No commutativa. En general
  $S_1\cup S_2 \neq S_2\cup S_1$ tot i que sí que es compleix
  l'equivalència respecte al cardinal $|S_1 \cup S_2| = |S_2\cup S_1|$.
\end{itemize}

En segon lloc, es defineix la unió temporal de dues sèries temporals,
la qual és la unió sense tenir en compte les mesures que tenen el mateix
instant de temps i diferent valor.
\begin{definition}[Unió temporal]
  Siguin $S_1$ i $S_2$ dues sèries temporals en què $\glssymbol{dom}
  S_1 = \glssymbol{dom} S_2$, la unió temporal de les
  dues sèries temporals, notada com a $S_1
  \glssymboldef{not:sgst:cupt} S_2$, és una sèrie temporal que conté
  les mesures de $S_1$ i de $S_2$ excloent les que només comparteixen
  el temps: $S_1 \glssymbol{not:sgst:cupt} S_2 = \{ m | (m \in S_1
  \wedge m \in S_2 ) \vee ( m \in S_2 \wedge m
  \not\glssymbol{not:sgst:int} S_2) \vee (m \in S_2 \wedge m
  \not\glssymbol{not:sgst:int} S_1 )\}$.
\end{definition}


Propietats de la unió temporal:
\begin{itemize}
\item Commutativa
\end{itemize}

%DEfinir quina de les dues és la unió i quina la unió temporal és difícil. Fem l'anterior definició perquè per la unió és la que més s'acosta a la unió de conjunts i compleix que $A \subseteq A \cup B$. El definim amb la pertinença temporal per comoditat, si no la definició és més rebuscada: $\{ m_1 \in S_1 \vee (m_2 \in S_2 \wedge (m_2\leq m_1 \vee m_1 \geq m_2) )$



\begin{example}[Unió de dues sèries temporals]
  Siguin les dues sèries temporals $S_1=\{(1,1),(3,1),(4,0),(5,1)\}$ i
  $S_2=\{(2,2),(3,2),(4,0),(6,2)\}$.  La unió de la primera amb la
  segona és $S_1 \cup S_2 = \{(1,1),(2,2), (3,1), (4,0),(5,1),(6,2)\}$
  i, com que no és commutativa, la unió de segona amb la primera és
  $S_2 \cup S_1 = \{(1,1),(2,2), (3,2), (4,0),(5,1),(6,2)\}$. La unió
  temporal de totes dues, que és commutativa, és $S_1
  \glssymbol{not:sgst:cupt} S_2 = S_2 \glssymbol{not:sgst:cupt} S_1 =
  \{(1,1),(2,2),(4,0),(5,1),(6,2)\}$. %

  A la \autoref{fig:model:venn-unio} es mostren els diagrames Venn per
  a les tres operacions, on l'àrea pintada és la sèrie temporal
  resultant. L'àrea central d'intersecció dels dos conjunts són les
  mesures que comparteixen temps i valor, en aquest cas la mesura
  $(4,0)$. L'àrea central esquerra són les mesures de $S_1$ que només
  comparteixen temps amb una mesura de $S_2$, és a dir la $(3,1)$, i
  dualment a l'àrea central dreta hi ha la $(3,2)$. Les àrees més
  externes es corresponen amb la resta de mesures.  A la
  \autoref{fig:model:venn-unio} també es mostren les mateixes
  operacions amb la visualització en taula de les sèries temporals.

  \begin{figure}
    \centering 
    \def\escala{0.9}

\def\nodeA{node [above left=0.5cm and 0.1cm] {$(1,1)$} node [below left=0.5cm and 0.1cm] {$(5,1)$}}
\def\nodeB{node [above right=0.5cm and 0.1cm] {$(2,2)$} node [below right=0.5cm and 0.1cm] {$(6,2)$}}
\def\nodeT{node [above=0.1cm] {$(4,0)$} node [left=0.4cm] {$(3,1)$} node [right=0.4cm] {$(3,2)$}}
% Definition of circles
\def\firstcircle{(0,0) circle (1.5cm)}
\def\secondcircle{(0:2cm) circle (1.5cm)}
\def\thirdcircle{(0:1cm) circle (1.11cm)}

\colorlet{circle edge}{blue!50}
\colorlet{circle area}{blue!20}

\tikzset{
  filled/.style={fill=circle area, draw=circle edge, thick},
  outline/.style={draw=circle edge, thick},
  every node/.style={transform shape}
}

%\setlength{\parskip}{5mm}






%Set A or B
\begin{tikzpicture}[scale=\escala]
  \draw[filled] \firstcircle \nodeA;
    \begin{scope}
        \clip \secondcircle;
        \draw[filled, even odd rule] \firstcircle \nodeA
                                 \secondcircle 
                                 \thirdcircle;
   \end{scope}
    \draw[outline] \firstcircle
                   \secondcircle \nodeB
                   \thirdcircle \nodeT;

   \node[anchor=south] at (current bounding box.north) {$S_1 \cup S_2$};
\end{tikzpicture}
%Set temporal A or B
\begin{tikzpicture}[scale=\escala]
    \draw[filled, even odd rule] \firstcircle \nodeA
                                 \secondcircle \nodeB
                                 \thirdcircle \nodeT;
    \node[anchor=south] at (current bounding box.north) {$S_1 \cup^t S_2$};
\end{tikzpicture}






%%% Local Variables:
%%% TeX-master: "../main"
%%% ispell-local-dictionary: "british"
%%% End:


  \begin{tabular}[c]{|c|c|}
    \multicolumn{2}{c}{$S_1$} \\ \hline
    $t$  & $v$ \\ \hline
    1  & 1 \\
    3  & 1 \\
    4  & 0 \\
    5  & 1 \\ \hline
  \end{tabular} \qquad
  \begin{tabular}[c]{|c|c|}
    \multicolumn{2}{c}{$S_2$} \\ \hline
    $t$  & $v$ \\ \hline
    2  & 2 \\
    3  & 2 \\
    4  & 0 \\
    6  & 2 \\ \hline
  \end{tabular} \qquad
  \begin{tabular}[c]{|c|c|}
    \multicolumn{2}{c}{$S_1 \cup S_2$} \\ \hline
    $t$  & $v$ \\ \hline
    1  & 1 \\
    2  & 2 \\
    3  & 1 \\
    4  & 0 \\
    5  & 1 \\
    6  & 2 \\ \hline
  \end{tabular} \qquad
  \begin{tabular}[c]{|c|c|}
    \multicolumn{2}{c}{$S_2 \cup S_1$} \\ \hline
    $t$  & $v$ \\ \hline
    1  & 1 \\
    2  & 2 \\
    3  & 2 \\
    4  & 0 \\
    5  & 1 \\
    6  & 2 \\ \hline
  \end{tabular} \qquad
  \begin{tabular}[c]{|c|c|}
    \multicolumn{2}{c}{$S_1 \glssymbol{not:sgst:cupt} S_2$} \\ \hline
    $t$  & $v$ \\ \hline
    1  & 1 \\
    2  & 2 \\
    4  & 0 \\
    5  & 1 \\
    6  & 2 \\ \hline
  \end{tabular} 

    \caption{Diagrames Venn i taules per als exemples d'unió i
      d'unió temporal}
    \label{fig:model:venn-unio}
  \end{figure}

 \end{example}





\subsubsection{Diferència}

La diferència de dos conjunts és un conjunt que conté tots els
elements del primer conjunt que no pertanyen al segon.  Per a poder
restar dos conjunts amb estructura de relació, $A - B$, cal que tots
dos tinguin la mateixa estructura; és a dir, en termes de \gls{SGBDR}
cal que $A$ i $B$ tinguin la mateixa capçalera.
En la definició de l'operació de diferència cal tenir en compte les
dues pertinences possibles.

En primer lloc, es defineix la diferència atenent a la pertinença
estricta de conjunts. És a dir s'aplica la diferència de
conjunts a les sèries temporals.
\begin{definition}[Diferència]
  Siguin $S_1$ i $S_2$ dues sèries temporals  en què $\glssymbol{dom}
  S_1 = \glssymbol{dom} S_2$, la diferència de les dues sèries
  temporals, $S_1 \glssymboldef{not:sgst:minus} S_2$, és una sèrie
  temporal que conté totes les mesures de
  $S_1$ que no pertanyen a $S_2$: $S_1 - S_2 = \{ m| m \in S_1 \wedge m \notin
  S_2 \}$.
\end{definition}

En segon lloc, es defineix la diferència atenent a la pertinença
temporal.
\begin{definition}[Diferència temporal]
  Siguin $S_1$ i $S_2$ dues sèries temporals en què $\glssymbol{dom}
  S_1 = \glssymbol{dom} S_2$, la diferència temporal de les dues
  sèries temporals, $S_1 \glssymboldef{not:sgst:minust} S_2$, és una
  sèrie temporal que conté totes les mesures de $S_1$ que no pertanyen
  temporalment a $S_2$: $S_1 \glssymbol{not:sgst:minust} S_2 = \{ m |
  m \glssymbol{not:sgst:int} S_1 \wedge m \not\glssymbol{not:sgst:int}
  S_2 \}$.
\end{definition}




\subsubsection{Intersecció}

La intersecció de dos conjunts és un conjunt que conté els elements
comuns als dos conjunts.  Per a poder intersecar dos conjunts amb estructura
de relació, $A \cap B$, cal que tots dos tinguin la mateixa
estructura; és a dir, en termes de \gls{SGBDR} cal que $A$ i $B$ tinguin la
mateixa capçalera.

En la definició de l'operació d'intersecció cal tenir en compte les
dues pertinences possibles.

En primer lloc, es defineix la diferència atenent a la pertinença
estricta de conjunts. És a dir s'aplica l'operació d'intersecció de
conjunts.
\begin{definition}[Intersecció]
  Siguin $S_1$ i $S_2$ dues sèries temporals  en què $\glssymbol{dom}
  S_1 = \glssymbol{dom} S_2$, la intersecció de les dues
  sèries temporals, $S_1 \glssymboldef{not:sgst:cap} S_2$, és una sèrie
  temporal  que conté les mesures de $S_1$
  repetides a $S_2$: $S_1 \cap S_2 = \{ m | m \in S_1 \wedge m \in S_2 \}$.
\end{definition}

En segon lloc, es defineix la intersecció atenent a la pertinença
temporal tenint en compte quan dues sèries temporals tenen mesures en
el mateix instant de temps però de valor diferent.
\begin{definition}[Intersecció temporal]
  Siguin $S_1$ i $S_2$ dues sèries temporals  en què $\glssymbol{dom}
  S_1 = \glssymbol{dom} S_2$, la intersecció temporal de les
  dues sèries temporals, $S_1 \glssymboldef{not:sgst:capt} S_2$, és una
  sèrie temporal que conté les mesures de
  $S_1$ repetides temporalment a $S_2$: $S_1 \glssymbol{not:sgst:capt}
  S_2 = \{ m | m \glssymbol{not:sgst:int} S_1 \wedge m \glssymbol{not:sgst:int} S_2 \}$.
\end{definition}

Propietats de la intersecció:
\begin{itemize}
\item La intersecció és commutativa però la intersecció temporal no és
  commutativa.
\item A partir de la diferència es pot definir la intersecció: $S_1
  \cap S_2 = S_1 - (S_1 - S_2)$.
\end{itemize}


\subsubsection{Diferència simètrica}

La diferència simètrica de dos conjunts és un conjunt que conté els
elements no comuns dels dos conjunts. La diferència simètrica de dos
conjunts $A \ominus B$ es defineix a partir de la diferència i la
unió:
\begin{align*}
A \ominus B  & = (A-B)\cup(B-A)\\
             & = (A\cup B)-(A\cap B)  \\
A \ominus B  & \subseteq A\cup B
\end{align*}

Seguint aquestes propietats es defineixen dues diferències
simètriques: una a partir de la diferència i la unió de sèries
temporals i una altra a partir de la diferència temporal i la unió
temporal.  Per tal que l'operació de diferència simètrica sigui vàlida
per a les sèries temporals cal tenir en compte quan dues sèries
temporals tenen mesures en el mateix instant de temps.

En primer lloc, es defineix la diferència simètrica excloent les
mesures amb el mateix temps però de valor diferent.
\begin{definition}[Diferència simètrica]
  Siguin $S_1$ i $S_2$ dues sèries temporals  en què $\glssymbol{dom}
  S_1 = \glssymbol{dom} S_2$, la diferència simètrica de les
  dues sèries temporals, $S_1 \glssymboldef{not:sgst:ominus} S_2$, és
  una sèrie temporal que conté les mesures de
  $S_1$ o exclusivament les de $S_2$: $S_1 \ominus S_2 = \{ m | (m \in
  S_1 \wedge m \notin S_2) \vee (m \in S_2 \wedge m
  \not\glssymbol{not:sgst:int} S_1) \}$.
\end{definition}

En segon lloc, es defineix la diferència simètrica temporal excloent les
mesures amb el mateix temps.
\begin{definition}[Diferència simètrica temporal]
  Siguin $S_1$ i $S_2$ dues sèries temporals en què $\glssymbol{dom}
  S_1 = \glssymbol{dom} S_2$, la diferència simètrica de les dues
  sèries temporals, $S_1 \glssymboldef{not:sgst:ominust} S_2$, és una
  sèrie temporal que conté les mesures de $S_1$ o exclusivament les de
  $S_2$: $S_1 \glssymbol{not:sgst:ominust} S_2 = \{ m | (m
  \glssymbol{not:sgst:int} S_1 \wedge m \not\glssymbol{not:sgst:int}
  S_2 ) \vee (m \glssymbol{not:sgst:int} S_2 \wedge m
  \not\glssymbol{not:sgst:int} S_1 )\}$.
\end{definition}



\subsubsection{Selecció}

La selecció és una operació dels \gls{SGBDR} que selecciona uns tuples
determinats d'un conjunt, a vegades també s'anomena restricció.

\begin{definition}[Selecció]
  Sigui la sèrie temporal $S$, $a_1$ i $a_2$ dos
  noms d'atributs que pertanyen a $S$, i $a_1 \Theta a_2$ una
  expressió booleana sobre $a_1$ i $a_2$, la selecció de $S$ per
  l'expressió booleana s'escriu com
  $\glssymboldef{not:sgst:select}_{a_1 \Theta a_2}(S)$  i es defineix
  de la mateixa manera que en els
  \gls{SGBDR} \parencite[cap.~7]{date04:introduction8}.
\end{definition}

En una forma més genèrica, l'expressió booleana pot incloure un o més
atributs i està formada per més d'una expressió lògica.


\begin{example}[Selecció de les mesures majors a un instant de temps]
  Sigui la sèrie temporal $S_1=\{(1,1),(3,1),(4,0),(5,1)\}$, la
  selecció dels temps més grans que $3$ és
  $\glssymbol{not:sgst:select}_{t>3}(S_1) = \{(4,0),(5,1)\}$.
\end{example}





\subsubsection{Projecció}


La projecció és una operació dels \gls{SGBDR} que selecciona uns
atributs determinats d'un conjunt. Aquesta operació treballa amb la
capçalera de la sèrie temporal, és a dir amb els atributs que
genèricament són $t$ i $v$ però que també poden tenir altres noms.

\begin{definition}[Projecció]
  Sigui la sèrie temporal $S$ i sigui $A=\{a_0,
  \dotsc, a_n\}$ un conjunt de noms d'atributs, la projecció de la
  sèrie temporal en els atributs s'escriu com
  $\glssymboldef{not:sgst:project}_A(S)$ i es defineix de la mateixa
  manera que en els
  \gls{SGBDR} \parencite[cap.~7]{date04:introduction8}. Aleshores
  aquesta nova sèrie temporal $\glssymbol{not:sgst:project}_A(S)$
  només inclou els atributs $A$ de les mesures.
\end{definition}

En l'operació de projecció, si els
atributs seleccionats no inclouen l'atribut temps o només inclouen
un atribut el resultat no és una sèrie temporal sinó que és un conjunt
relacional. 


% \todo{definir? la definició embolica}
% \begin{definition}[projecció]
%   Sigui la sèrie temporal $S=\{ m_0,\dotsc,m_k\}$, on les mesures
%   poden ser multiavaluades $m_i=(t_i,v^1_i,\dotsc,v^n_i)$, expressada
%   en la forma completa amb capçalera $S = ( \{t, n_1,\dotsc,n_n \}, \{
%   \{ (t,t_0),(n_1,v^1_0),\dotsc,(n_n,v^n_0)\}, \dotsc, \{
%   (t,t_0),(n_1,v^1_1),\dotsc,(n_n,v^n_n)\} \} )$ on $t, n_1, \dotsc,
%   n_n$ són els noms dels atributs; i sigui $A=\{a_0, \dotsc, a_n\}$ un
%   conjunt de noms d'atributs. La projecció de la sèrie temporal en els
%   atributs és $\glssymboldef{not:sgst:project}_A(S) = \{ ()  \}$
% \end{definition}




\begin{example}[Projecció d'alguns atributs de la sèrie temporal]
  Sigui la sèrie temporal $S_1=\{(1,1),(3,1),(4,0),(5,1)\}$, la
  projecció en l'atribut de temps és el conjunt
  $\glssymbol{not:sgst:project}_{\{t\}}(S_1) = \{ 1,3,4,5 \}$.  Sigui
  la sèrie temporal multivaluada $S_2 = ( (t,\text{ temp},\text{
    cons},\text{ vol}),\{ (2,1,2,3), (3,2,1,0), (6,1,2,3) \})$, la
  projecció en els atributs $t$ i \emph{temp} és la sèrie temporal
  $\glssymbol{not:sgst:project}_{ \{t,\text{temp}\}}(S_1) = (
  (t,\text{ temp}),\{ (2,1), (3,2), (6,1) \})$
\end{example}







\subsubsection{Reanomena}

El reanomena és una operació dels \gls{SGBDR} que canvia el nom dels
atributs.  Aquesta operació treballa amb la
capçalera de la sèrie temporal.


\begin{definition}[Reanomena]
  Sigui la sèrie temporal $S$, $a$ un nom
  d'atribut que pertany a $S$ i $b$ un que no hi pertany, reanomenar
  $a$ per $b$ s'escriu com $\glssymboldef{not:sgst:rename}_{a/b} (S)$
  i es defineix de la mateixa manera que en els
  \gls{SGBDR} \parencite[cap.~7]{date04:introduction8}.
\end{definition}

En una forma més genèrica, es poden reanomenar més d'un atribut
alhora.

\begin{example}[Reanomena els atributs de la sèrie temporal]
  Sigui la sèrie temporal multivaluada $S_2 = ( (t,\text{ temp},\text{
    cons},\text{ vol}),\{ (2,1,2,3), (3,2,1,0), (6,1,2,3) \})$, reanomenar 
  l'atribut \emph{temp} per \emph{v1} és la sèrie temporal
  $\glssymbol{not:sgst:rename}_{\text{temp}/\text{v1}}(S_1) = ( (t,\text{ v1},\text{
    cons},\text{ vol}),\{ (2,1,2,3), (3,2,1,0), (6,1,2,3) \})$.
\end{example}



\subsubsection{Producte i junció}

El producte cartesià de dos conjunts és un conjunt que conté totes les
parelles possibles d'elements d'ambdós conjunts.  Per a poder
multiplicar dos conjunts amb estructura de relació, $A
\glsdisp{not:times}{\times} B$, en termes de \gls{SGBDR} cal que $A$ i
$B$ no tinguin en comú noms d'atributs.  En els \gls{SGBDR}, a
diferència del producte de conjunts, el conjunt resultant no és un
conjunt de parells de tuples sinó un conjunt de tuples.

Definim el producte de dues sèries temporals, les qual en
forma canònica tinguin els atributs $t$ i $v$, com una sèrie temporal
amb atributs $t_1$, $v_1$, $t_2$ i $v_2$. Així doncs, per a sèries
temporals el producte resulta en una sèrie temporal amb dos atributs
de temps, a la qual anomenem sèrie temporal doble (v.\
\autoref{def:sgst:st-doble}).
\begin{definition}[Producte]
  Siguin $S_1$ i $S_2$ dues sèries temporals en forma canònica, el producte de
  les dues sèries temporals $S_1 \glssymboldef{not:sgst:times}
  S_2$ és una sèrie temporal doble que conté
  la unió de totes les parelles de mesures de $S_1$ i $S_2$: $S_1
  \times S_2 = \{ (t_1,v_1,t_2,v_2) | (t_1,v_1) \in S_1 \wedge
  (t_2,v_2) \in S_2 \}$
\end{definition}

Propietats del producte:
\begin{itemize}
\item El cardinal resultant és $|S|=|S_1||S_2|$
\item El grau resultant és $4$
\end{itemize}



La junció (\emph{join}) de dos conjunts és un conjunt que conté les
parelles d'elements d'ambdós conjunts que tenen el mateix valor per
als atributs comuns.  La junció de dos conjunts amb estructura de
relació es defineix com una selecció sobre el
producte \parencite[cap.~7]{date04:introduction8}.


Per a les sèries temporals, definim la junció com l'ajuntament de les
parelles que tenen el mateix atribut de temps en ambdues sèries
temporals . El resultat de la junció és una sèrie temporal
multivaluada.
\begin{definition}[Junció]\label{def:sgst:join}
  Siguin $S_1$ i $S_2$ dues sèries temporals en forma canònica, la junció de
  les dues sèries temporals, $S_1 \glssymboldef{not:sgst:join} S_2$, és
  una sèrie temporal multivaluada que
  selecciona del producte de $S_1$ amb $S_2$ les mesures dobles amb
  temps iguals: $S_1 \glssymbol{not:sgst:join} S_2 = \{ (t,v_1,v_2) |
  (t,v_1) \in S_1 \wedge (t,v_2) \in S_2  \}$.
\end{definition}



Propietats de la junció:
\begin{itemize}
\item $\glssymbol{dom}(S_1 \glssymbol{not:sgst:join}
  S_2)=\glssymbol{dom}S_1 \times \glssymbol{dom}S_2$
\item El cardinal resultant és $|S|\leq\min(|S_1|,|S_2|)$
\item És commutativa; tenint en compte que els atributs tenen nom i
  per tant l'ordre no importa.
\end{itemize}


Cal tenir en compte que la junció només sap operar amb dues sèries
temporals que tinguin el mateix vector de temps; és a dir regulars
entre elles (v.\ \autoref{def:sgst:regulars_entre_elles}). En el cas
que no tinguin el mateix vector de temps, es pot aplicar la junció
temporal de la \autoref{def:sgst:joint}.



\begin{example}[Junció de dues sèries temporals]
  Siguin les dues sèries temporals $S_1=\{(1,1),(3,1),(4,0),(5,1)\}$
  and $S_2=\{(2,2),(3,2),(4,0),(6,2)\}$.  La junció de totes dues és
  $S_1 \glssymbol{not:sgst:join} S_2 = \{(3,1,2),(4,0,0)\}$.
\end{example}



\subsubsection{Computacionals: mapa, agregació i plec}
\glsaddsec{not:sgst:computacional} %%%%secció d'operacions

Per a poder operar amb els conjunts, a més de l'àlgebra definida fins
ara, es necessiten operadors amb funcionalitats computacionals; és a
dir, operadors que calculin amb els valors continguts en els conjunts. 

En els \gls{SGBDR} els operadors computacionals bàsics són
\emph{extend}, \emph{aggregate} i
\emph{summarize} \parencite[cap.~7]{date04:introduction8}.  Per a les
sèries temporals definim operacions equivalents a les dues primeres de
la manera amb què habitualment s'utilitzen per als conjunts.  La
tercera, el \emph{summarize}, és una operació que s'utilitza per a
sintetitzar informació mitjançant grups, és a dir aplica operacions
\emph{aggregate} a conjunts que prèviament s'han agrupat segons un
atribut compartit.  Per a les sèries temporals, però, necessitem una
operació computacional més genèrica que ens permeti calcular
recursivament sense haver de definir grups.


 % Operacions
 %  d'agregació per intervals de temps,
 %  p.ex. $\{(gen,25),(feb,4),(mar,10)\}$, com es faria això?}
% No obstant, es pot aplicar el \emph{summarize} per a l'atribut de
% valors: summarize S per S {v} add ...  però això ja no mapa a una
% sèrie temporal.

% De l'operador \emph{aggregate} dels SGBDR definit per
% \textcite{date:introduction} cal tenir en compte que en defineix
% dues vessants. Per una banda, defineix els \emph{aggregate operator
% invocation} que retornen valors escalars. Per altra banda, defineix
% els \emph{aggregate operator invocation} que serveixen per a
% treballar amb el \emph{summarize}.

Així doncs, a continuació es defineix l'operador mapa (\emph{map}) com
a equivalent a l'\emph{extend}, l'operador agregació (\emph{aggregate})
com a equivalent a l'\emph{aggregate} i l'operador plec (\emph{fold})
com una forma més general de calcular recursivament amb les mesures
que \emph{summarize}.



L'operació de mapatge aplica una funció a cada element del conjunt.
\begin{definition}[Mapa]
  \label{def:sgst:mapa}
  Sigui $S$ una sèrie temporal en què
  $\glssymbol{not:valor-domini}=\glssymbol{dom} S$ i sigui
  $f:\glssymbol{not:temps-domini}\times\glssymbol{not:valor-domini}\rightarrow\glssymbol{not:temps-domini}'\times\glssymbol{not:valor-domini}'$
  una funció sobre una mesura que retorna una mesura.  El \emph{mapa} de $f$
  a $S$ és una sèrie temporal amb la funció aplicada a cada mesura:
  $\glssymboldef{not:sgst:map}(S,f) = \{ f(m) | m\in S \}$. Noteu que
  $\glssymbol{dom}(\glssymbol{not:sgst:map}(S,f))=\glssymbol{not:valor-domini}'$.
\end{definition}


L'operació d'agregació sintetitza en una mesura la informació dels
elements del conjunt segons un criteri, per exemple estadístics.
\begin{definition}[Agregació]
  Sigui $S=\{m_0, \dotsc, m_k\}$ una sèrie temporal en què
  $\glssymbol{not:valor-domini}=\glssymbol{dom} S$, sigui $m$ una
  mesura amb $\glssymbol{not:valor-domini}=\glssymbol{dom} m$ i sigui
  $f:(\glssymbol{not:temps-domini}\times\glssymbol{not:valor-domini})\times(\glssymbol{not:temps-domini}\times\glssymbol{not:valor-domini})\rightarrow
  \glssymbol{not:temps-domini}\times\glssymbol{not:valor-domini}$ una
  funció sobre dues mesures que retorna una mesura.
  L'\emph{agregació} de $S$ segons $f$ amb valor inicial $m$ és una
  mesura que sintetitza la informació de les mesures:
  $\glssymboldef{not:sgst:aggregate}(S,m,f) = f(\cdots(f(f(f(m,m_0),
  m_1), m_2)\cdots), m_k)$.
\end{definition}

% Més compactament descrit amb
% \begin{align*}
%   \text{fold}: & S=\{m_0,\dotsc,m_k\} \times m_i \times f \longrightarrow m'= \\
%   & \begin{cases}
%     m_i & \text{si} |S|=0, \\
%     \text{fold}(S_1,f(m_i,m_1),f) & \text{altrament}
%   \end{cases}\\
%   \text{ a on } & m_1 \in S, S_1 = S - \{m_1\}
% \end{align*}


L'operació de plegament combina recursivament els elements del conjunt
segons un criteri. Cal notar que l'agregació definida anteriorment és
un cas específic del plegament. Assumiu que $\glssymbol{powerset}(C)$
és el conjunt potència (\emph{powerset}) de $C$.
\begin{definition}[Plec]
  \label{def:sgst:plec}
  Siguin $S=\{m_0, \dotsc, m_k\}$ i $R$ dues sèries temporals en les quals
  $\glssymbol{not:valor-domini}=\glssymbol{dom} S$ i
  $\glssymbol{not:valor-domini}'=\glssymbol{dom} R$, i sigui
  $f:\glssymbol{powerset}(\glssymbol{not:temps-domini}\times\glssymbol{not:valor-domini}')
  \times
  (\glssymbol{not:temps-domini}\times\glssymbol{not:valor-domini})
  \rightarrow
  \glssymbol{powerset}(\glssymbol{not:temps-domini}\times\glssymbol{not:valor-domini}')$
  una funció sobre una sèrie temporal i una mesura que retorna una
  sèrie temporal.  El \emph{plec} de $S$ per $f$ amb valor inicial $R$
  és una sèrie temporal amb les mesures combinades:
  $\glssymboldef{not:sgst:fold}(S,R,f) = f(\cdots(f(f(f(R,m_0),
  m_1),m_2)\cdots) m_k)$.
\end{definition}


Les operacions d'agregació i plegament tal com s'han definit es
realitzen en ordre aleatori de mesures. Segons el criteri que
s'utilitzi, l'ordre és important i per tant cal una operació que
computi tenint-lo en compte. A tal efecte, a continuació s'amplia la
definició de la funció de plegament per a tenir en compte l'ordre; per
a la funció d'agregació es pot aplicar el mateix concepte.
\begin{definition}[Plec amb ordre]
  \label{def:sgst:oplec}
  Siguin $S=\{m_0, \dotsc, m_k\}$ i $R$ dues sèries temporals en les
  quals $\glssymbol{not:valor-domini}=\glssymbol{dom} S$ i
  $\glssymbol{not:valor-domini}'=\glssymbol{dom} R$, sigui
  $f:\glssymbol{powerset}(\glssymbol{not:temps-domini}\times\glssymbol{not:valor-domini}')
  \times
  (\glssymbol{not:temps-domini}\times\glssymbol{not:valor-domini})
  \rightarrow
  \glssymbol{powerset}(\glssymbol{not:temps-domini}\times\glssymbol{not:valor-domini}')$
  una funció sobre una sèrie temporal i una mesura que retorna una
  sèrie temporal, i sigui
  $g:\glssymbol{powerset}(\glssymbol{not:temps-domini}\times\glssymbol{not:valor-domini})
  \rightarrow
  (\glssymbol{not:temps-domini}\times\glssymbol{not:valor-domini})$
  una funció que retorna una mesura d'una sèrie temporal.
  El \emph{plec} de $S$ per $f$ amb valor inicial $R$ i ordre $g$ és
  una sèrie temporal que combina les mesures seguint l'ordre:
  \[\glssymboldef{not:sgst:ofold}(S,R,f,g) =
  \begin{cases}
    R & \text{si } |S|=0, \\
    \glssymbol{not:sgst:ofold}(Q,f(R,q),
    f,g) &
    \text{altrament}
  \end{cases}\] on $q = g(S)$ i $Q = S - \{q\}$.
\end{definition}

El plec amb ordre és necessari quan la funció
$f$ no és associativa ni commutativa perquè
llavors l'ordre dels càlculs és important. 

% la funció de plegament en els plecs sense ordre ha de ser commutativa i ¿associativa? Si no ho és, el resultat és aleatori
%De manera semblant $\agg(S,m_i,f)\equiv \agg(S,m_i,f,o)$ on $o=\text{aleatori}(S)$.



Propietats de les operacions computacionals:
\begin{itemize}
\item El plec sense ordre és un plec amb ordre aleatori:
  $\glssymbol{not:sgst:fold}(S,R,f)=
  \glssymbol{not:sgst:ofold}(S,R,f,\text{aleatori})$.

\item El plec d'una sèrie temporal buida és la sèrie inicial:
  $\glssymbol{not:sgst:fold}(\emptyset, R, f) = R$.

\item El plec per una funció que sempre retorni la sèrie inicial és la
  sèrie inicial: $\glssymbol{not:sgst:fold}(S, R, f)= R$ on
  $f(Q, m)=Q$.

\item El plec per una funció que només retorni la mesura original és
  una sèrie amb una sola mesura: $\glssymbol{not:sgst:fold}(S, R, f)=
  S'$ on $f(Q,m)=\{m\}$ i $|S'|=1$.


\item La funció d'unió en el plegament permet fer la identitat: $S
  = \glssymbol{not:sgst:fold}(S,\emptyset,f)$
  on $f(Q,m)= Q \cup \{m\}$.


\item Els mapes es poden implementar com a plecs:
  $\glssymbol{not:sgst:map}(S,f) =
  \glssymbol{not:sgst:fold}(S,\emptyset,g)$ on
  $g(Q, m)= \{f(m)\} \cup Q$.  De manera semblant,
  \textcite{lammel08:mapreduce} també exemplifica com els mapes es
  poden implementar com a plecs. %pàg. 5 Asides on folding

\item Les agregacions es poden implementar com a plecs:
  $\glssymbol{not:sgst:aggregate}(S,m,f)=
  \glssymbol{not:sgst:fold}(S,\{m\},g)$ on
  $g(\{n\}, o)= \{f(n,o)\}$.

\end{itemize}





\begin{example}[Mapes de sèries temporals]
  Definicions de funcions d'exemple a partir de l'operació
  computacional de mapatge:

\label{ex:sgst:duplicat}
\begin{itemize}
\item $\operatorname{identitat}(S)= \glssymbol{not:sgst:map}(S,f)$ on
  $f(t,v)=(t,v)$
\item $\operatorname{intercanvi}(S)=
  \glssymbol{not:sgst:map}(S,f)$ on $f(t,v)=(v,t)$
\item $\glssymbolex{not:sgst:duplica-t}(S)=
  \glssymbol{not:sgst:map}(S,f)$ on $f(t,v)=(t,t)$
\item $\operatorname{translaci\acute{o}}(S,d)=
  \glssymbol{not:sgst:map}(S,f)$ on $f(t,v)=(t+d,v)$ i $d$ és una
  durada de temps
\item $\operatorname{multiplica\_tv}(S)=
  \glssymbol{not:sgst:map}(S,f)$ on $f(t,v)=(t,t\cdot v)$

%Encara no s'ha definit ant, per tant no es pot fer l'exemple
% \item $\operatorname{tpredecessors_{v1}}: S \mapsto S'$ on $S'= \glssymbol{not:sgst:map}(S,(t,v)
%   \mapsto (t,T(\ant_S(m)))$, usant l'operació predecessor de la
%   \autoref{def:sgst:ant}
% \item $\operatorname{vpredecessors}: S \mapsto S'$ on $S'= \glssymbol{not:sgst:map}(S,(t,v)
%   \mapsto (t,V(\ant_S(m)))$, usant l'operació predecessor de la
%   \autoref{def:sgst:ant}

\end{itemize}
\end{example}

\begin{example}[Agregacions de sèries temporals]
  Definicions de funcions d'exemple a partir de l'operació
  computacional d'agregació. Ens els exemples següents utilitzem la
  notació descrita anteriorment $f(m,n)$ per a definir les funcions
  d'agregació, en què $m$ pot ser la mesura inicial o les mesures
  resultants i $n$ és una mesura pertanyent a la sèrie temporal
  agregada.

\begin{itemize}
\item $\operatorname{cardinal}( S) = V\big(
  \glssymbol{not:sgst:aggregate} (S,(0,0),f) \big)$ on
  $f(m,n)=(0,V(m)+1)$. Aquesta funció és una implementació del
  cardinal de la \autoref{def:sgst:cardinal} a partir de
  l'agregació. Noteu que l'atribut de temps no té cap sentit en
  aquesta computació.


\item $\glssymbolex{not:sgst:sumav}(S) =V\big(
  \glssymbol{not:sgst:aggregate} (S,(0,0),f ) \big)$ on
  $f(m,n)=(0,V(m)+V(n))$ \label{def:sgst:sumav}. Noteu que en aquesta
  computació l'atribut de temps tampoc té cap sentit.

\item $\glssymbolex{not:sgst:mitjanav}(S)=
  \operatorname{suma\_v}(S) / \operatorname{cardinal}(S)$ \label{def:sgst:mitjanav}
  % alerta amb 0/0, es deixa indefinit... 

\item $\operatorname{sup}(S)=
  \glssymbol{not:sgst:aggregate}(S,(-\infty,\infty),f )$ on $f(m,n)=
  m$ si $T(n)<T(m)$ o $f(m,n)= n$ en cas contrari. Aquesta funció és una
  implementació de l'operació suprem de la \autoref{def:sgst:sup} a
  partir de l'agregació.

\item $\glssymbolex{not:sgst:maxv}(S)= V\big(
  \glssymbol{not:sgst:aggregate}
  (S,(0,-\infty),f ) \big)$ on
  $f(m,n)=(0,\max(V(m),V(n))$. A diferència del $\sup(S)$ o del
  $\max(S)$, el $\glssymbol{not:sgst:maxv}(S)$ calcula el màxim dels
  valors.   \label{def:sgst:maxv}



%Encara no s'ha definit ant, per tant no es pot fer l'exemple
% \item $\operatorname{ant}: S \times m \mapsto m'$ on $m'=
%   \glssymbol{not:sgst:aggregate}(S,(-\infty,\infty),(t^i,v^i)\times(t,v)\mapsto
%   [(t,v) \text{ if } t^i < t < T(m) \text{ else } (t^i,v^i)
%   ])$. Aquesta funció és una implementació de l'operació predecessor
%   de la \autoref{def:sgst:ant} a partir de l'agregació.
\end{itemize}
\end{example}

\begin{example}[Plecs de sèries temporals]
  Definicions de funcions d'exemple a partir de l'operació
  computacional de plegament. 

\begin{itemize}
\item $\operatorname{tpredecessors}(S)=
  \glssymbol{not:sgst:fold}(S,S,f)$ on $ f(R,m)= \{(T(m),s)\} \cup R$
  i $s=T(\sup( \allowbreak \glssymbol{not:sgst:select}_{t <
    T(m)}R))$. Per a cada mesura $(t,v)$ de la sèrie temporal, el
  resultat conté una mesura $(t,s)$ en què $s$ és el temps de la
  mesura precedent a $(t,v)$.
  % , sense usar l'operació predecessor  a diferència de l'exemple $\operatorname{tpredecessors_{v1}}$ 
  %De fet aquesta operació es pot definir amb un mapa en comptes de plec i potser s'entén més? S'=mapa(S,(t',v') mapsto (t', T(sup(seleccio(S,t<t')))  ))

\item $\glssymbolex{not:sgst:vpredecessors}(S) =
  \glssymbol{not:sgst:map}(\operatorname{tpredecessors}(S), f)$ on
  $f(m)= (T(m), V(\sup(R))$ i $R=
  \glssymbol{not:sgst:select}_{t=V(m)}S$.  Per a cada mesura de la
  sèrie temporal indica quin és el valor de la mesura precedent, és
  una definició a partir de l'operació de
  $\operatorname{tpredecessors}$.

\end{itemize}

\end{example}


\begin{example}[Aplicacions de les operacions computacionals]
  Sigui la sèrie temporal $S=\{(1,1),(3,1),(4,0),(5,1)\}$. %
  La duplicació dels temps en els valors de la sèrie temporal és
  $\glssymbol{not:sgst:duplica-t}(S)=\{(1,1),(3,3),(4,4),(5,5)\}$. %
  La mitjana dels valors de la sèrie temporal és
  $\glssymbol{not:sgst:mitjanav}(S) = 0{,}75$. %
  Els temps predecessors de cada mesura de la sèrie temporal són
  $\operatorname{tpredecessors}(S)=\{(1,-\infty),(3,1),(4,3),(5,4)\}$
  i els valors predecessors són
  $\glssymbol{not:sgst:vpredecessors}(S)=\{(1,\infty),(3,1),(4,1),(5,0)\}$.
\end{example}




\subsubsection{Computacionals binàries amb els valors}

Una operació en els conjunts és la que aplica un operador binari a
totes les parelles possibles dels elements de dos conjunts. Per
exemple la suma, és a dir l'operador binari $+$, aplicada a dos
conjunts $A$ i $B$ és un conjunt $A + B = \{ a+b | (a,b) \in
A\times B \}$.

Per a les sèries temporals també calen operacions computacionals amb
les mesures de dues sèries temporals. En el cas d'operar amb dues
sèries temporals primer cal ajuntar les dues sèries temporals que es
volen operar i després aplicar les operacions computacionals binàries
a la sèrie temporal resultant.


El producte i la junció són els operadors que permeten crear parelles
de mesures de dues sèries temporals. Per a operar amb els valors de
dues sèries temporals la junció és més adequada ja que permet ajuntar
el valors que tenen temps comuns. Així doncs, es defineix l'aplicació
d'un operador binari de valors a dues sèries temporals a partir de la
junció.
\begin{definition}[Operació computacional binària amb els valors]
  Siguin $S_1$ i $S_2$ dues sèries temporals i sigui
  $\glssymboldef{not:sgst:opbinari}$ un operador binari en el domini
  dels valors,
  $\glssymbol{not:sgst:opbinari}:\glssymbol{not:valor-domini}_1 \times
  \glssymbol{not:valor-domini}_2 \rightarrow
  \glssymbol{not:valor-domini}'$. L'aplicació d'aquest operador binari
  a dues sèries temporals és $S_1
  \glssymboldef{not:sgst:computacionalbinaria} S_2 =
  \glssymbol{not:sgst:map} (S_1 \glssymbol{not:sgst:join} S_2,f)$ on
  $f (t,v,w)=(t,v \glssymbol{not:sgst:opbinari} w)$.
\end{definition}




\begin{example}[Aplicacions de les operacions computacionals binàries]
  Exemples de l'aplicació d'operacions computacionals binàries en què
  s'aplica un operador binari $ \glssymbol{not:sgst:opbinari}$ als
  valors de dues sèries temporals
  \begin{itemize}
  \item $S' = S_1 + S_2$ 
  \item $S' = \operatorname{subtracci\acute{o}}(S_1, S_2)$. Noteu que
    indiquem amb el nom complet, $\operatorname{subtracci\acute{o}}$,
    l'operació computacional de dues sèries temporals corresponent a
    l'operació aritmètica de resta per tal de no confondre-la amb l'operació
    de diferència de conjunts que té el guionet per símbol (\emph{-}).
  \end{itemize}
  
  Les operacions computacionals binàries també es poden usar per a
  definir altres operacions, per exemple per a calcular els increments
  de valor d'una sèrie temporal $\glssymbolex{not:sgst:increments}(S)=
  \operatorname{subtracci\acute{o}}(S,
  \glssymbol{not:sgst:vpredecessors}(S))$. \label{ex:sgst:increments}
\end{example}




\begin{example}[Suma de dues sèries temporals i increments d'una]
  Siguin les dues sèries temporals $S_1=\{(1,1),(3,1),(4,0),(5,1)\}$
  and $S_2=\{(1,2),(3,2),(4,0),(5,2)\}$. %
  La suma de les dues sèries temporals és
  $S_1+S_2=\{(1,3),(3,3),(4,0),(5,3)\}$. %
  Els increments de la primera sèrie temporal són
  $\glssymbol{not:sgst:increments}(S_1)=\{(1,\infty),(3,0),(4,-1),(5,1)\}$.
\end{example}




\subsection{Bàsiques de seqüències}
\glsaddsec{not:op-sequencies} %%%%secció d'operacions


Atesa la relació d'ordre induïda pel temps en una sèrie temporal
(def.\ \ref{def:model:mesura-relacio-ordre}), les sèries temporals es
poden tractar com a seqüències.  En aquest apartat definim operadors
per a les sèries temporals recollint els operadors habituals que tenen
les seqüències. 

Els operadors que treballen amb seqüències tenen en compte l'atribut
que marca un ordre total en el conjunt. En el cas de les sèries
temporals aquest atribut és el temps.



\subsubsection{Interval}

L'interval sobre una seqüència és la subseqüència compresa entre dos
elements.  Per a les sèries temporals és possible definir el concepte
d'interval sobre la seqüència com la subsèrie entre dos instants de
temps, semblant a com es fa a \cite{last:keogh,last:hetland}.  És una
operació de selecció però amb la notació habitual en les seqüències.


\begin{definition}[Interval]
  \label{def:model:st-interval}
  Sigui $S$ una sèrie temporal i siguin $s$ i $t$
  dos instants de temps. Definim el subconjunt
  $S\glsdispdef{not:sgst:interval}{(s,t)} \subseteq S$ com la sèrie
  temporal $S(s,t)=\{m| m\in S \wedge s<T(m)<t\}$.

  Tal com es fa en les seqüències, es defineix una notació de
  parèntesis i claudàtors per indicar si l'interval és obert, tancat o
  semiobert:

  $S\glsdispdef{not:sgst:intervalsemi}{[s,t)}=\{m | m\in S  \wedge s\leq T(m)< t\}$

  $S(s,t]=\{m| m\in S  \wedge s<T(m)\leq t\}$

  $S\glsdispdef{not:sgst:intervaltancat}{[s,t]}=\{m | m\in S \wedge s\leq
  T(m)\leq t\}$
\end{definition}


Propietats:
\begin{itemize}
\item La subsèrie $S[-\infty,t)\subseteq S$ és equivalent a la sèrie
  temporal $S[-\infty,t) = S[T(\inf(S)),t)$. De la mateixa manera
  $S(s,+\infty] = S(s,T(\sup(S))]$.

\item L'interval degenerat $S[t,t]\subseteq S$ és equivalent a la
  sèrie temporal $S[t,t] = \{m | m\in S \wedge T(m)=t \}$. Els intervals
  $S(t,t]\subseteq S$ i $S[t,t)\subseteq S$ són equivalents a la sèrie
  temporal buida $S(t,t] = S[t,t) = \emptyset$ ja que per
  ser els temps d'ordre total $\nexists T(m): t < T(m) \leq t$ o
  $\nexists T(m): t \leq T(m) < t$, respectivament. 

\item La subsèrie $S[-\infty,+\infty] \subseteq S$ és equivalent a la
  sèrie temporal original $S[-\infty,+\infty] = S$. La subsèrie
  $S(-\infty,+\infty) \subseteq S$ només és equivalent a la sèrie
  temporal original quan aquesta no conté mesures indefinides
  $S(-\infty,+\infty) \iff S: (-\infty,v)\notin S \wedge
  (+\infty,w)\notin S$ on $v$ i $w$ són dos valors qualssevol.
\end{itemize}




\subsubsection{Successió}

Atenent a la relació d'ordre induïda pel temps en una sèrie temporal,
es defineix el concepte de successor i predecessor en una seqüència. A
partir d'una mesura, aquests conceptes determinen quina és la mesura
immediatament següent i la mesura immediatament anterior contingudes
en una sèrie temporal.
\begin{definition}[Successor i
  predecessor]\label{def:sgst:seg}\label{def:sgst:ant}
  Sigui $S$ una sèrie temporal i sigui $m$ una mesura. El successor de
  $m$ en $S$, notat com $\glssymboldef{not:sgst:next}_S(m)$, és
  $\glssymboldef{not:sgst:next}_S(m)=\inf(S(T(m),+\infty])$.  El
  predecessor de $m$ en $S$, notat com
  $\glssymboldef{not:sgst:prev}_S(m)$, és
  $\glssymboldef{not:sgst:prev}_S(m)=\sup(S[-\infty,T(m)))$.

  Quan no hi hagi dubte de la sèrie temporal que marca l'ordre, per
  exemple quan $m\in S$, podem escriure
  $\glssymbol{not:sgst:next}(m)$ i $\glssymbol{not:sgst:prev}(m)$.
\end{definition}

S'observa que s'obtenen mesures indefinides en els casos que la
mesura següent o anterior es calcula respectivament per la mesura
suprema o ínfima de la sèrie temporal: $\glssymbol{not:sgst:next}_S(\sup
S)=(+\infty,\infty)$ i $\glssymbol{not:sgst:prev}_S(\inf S)=(-\infty,\infty)$.


% De la definició anterior es dedueix que sigui una sèrie temporal $S$
% que no conté mesures indefinides i sigui la mesura indefinida positiva
% $(+\infty,\infty)$, el predecessor de la mesura indefinida positiva
% sempre és el suprem de la sèrie temporal $\glssymbol{not:sgst:prev}_S(
% (+\infty,\infty) ) = \sup(S): \forall m\in S:
% T(m)\in\glssymbol{not:R}$.  % S\equiv S(-\infty,+\infty)
% \emph{Demostració: Sigui $S$ una sèrie temporal i $o=(+\infty,\infty)$
%   una mesura indefinida, el predecessor de $o$ en $S$ és una mesura
%   $l=\glssymbol{not:sgst:prev}_S(o)$ que compleix
%   $l=\sup(S[-\infty,T(o)))$. Substituint, s'obté que
%   $l=\sup(S[-\infty,+\infty))=\sup(S-m):m\in S:T(m)=+\infty \notin
%   \glssymbol{not:R}$, i per tant com que $S$ no té mesures indefinides
%   es demostra que $l=\sup(S)$.  } De manera semblant es pot demostrar
% que $\glssymbol{not:sgst:next}_S( (-\infty,\infty) ) = \inf(S):
% \forall m\in S: T(m)\in\glssymbol{not:R}$.


\subsubsection{Concatenació}

La concatenació és una operació que uneix dues seqüències amb els
elements de la primera seqüència seguits pels de la segona. Així
doncs, la concatenació de les seqüències té un sentit semblant al que
la unió té en els conjunts. 

Per a les sèries temporals, per tal que l'operació de concatenació
uneixi amb ordre els operands, cal tenir en compte l'interval que
ocupa cada sèrie temporal segons el seu atribut de temps.  És a dir,
la concatenació de dues sèries temporals consisteix a unir la part de
la segona sèrie temporal que no està inclosa en el rang temporal de la
primera.

Per a poder concatenar dues sèries temporals cal que ambdues tinguin
la mateixa estructura, de la mateixa manera que ja s'ha vist amb
l'operació d'unió.


\begin{definition}[Concatenació]
  Siguin $S_1$ i $S_2$ dues sèries temporals en què $\glssymbol{dom}
S_1 = \glssymbol{dom} S_2$, la concatenació de les dues sèries
temporals, $S_1 \glssymboldef{not:sgst:concatenate} S_2$, és una sèrie
temporal que conté totes les mesures de $S_1$ i les mesures de $S_2$
que no intersequen en l'interval de $S_1$: $S_1
\glssymbol{not:sgst:concatenate} S_2 = S_1 \cup ( S_2 - S_2[T(\inf
S_1),T(\sup S_1)] )$.
\end{definition}


Propietats
\begin{itemize}
\item La concatenació no és commutativa
\end{itemize}







\subsection{Funció temporal}
\label{sec:sgst:operadors-temporals}
\glsaddsec{not:op-funcio} %%%%secció d'operacions

Atenent al fet que una sèrie temporal pot representar-se com una funció
temporal cal definir operacions per a tractar convenientment aquesta
naturalesa.
En aquest apartat definim aquestes operacions com una redefinició de
les bàsiques anteriors, i així poder aplicar-les considerant una sèrie
temporal com una funció temporal.  


A l'\autoref{sec:model:repr} es detalla més el concepte de
representació en funció temporal d'una sèrie temporal i s'ofereixen
exemples de diversos mètodes de representació. Les operacions
definides a continuació han de ser contextualitzades per a un mètode
de representació particular. A causa d'això indiquem cada operació de
funció temporal amb un superíndex, per exemple $r$, que indica que el
nom $r$ del mètode de representació usat.


%  De fet, a partir de l'operació d'interval temporal es
% poden definir els altres operadors de funció temporals; per tant cal
% oferir una definició d'interval temporal per a cada mètode de
% representació que es vulgui usar.



\subsubsection{Interval temporal}

Sigui $S$ una sèrie temporal i $[s,t]$ un interval de temps, per una
banda s'ha definit l'interval sobre la seqüència d'una sèrie temporal
$S(s,t)$ (v.\ \autoref{def:model:st-interval}) i per altra banda la
sèrie temporal pot tenir un mètode de representació $r$ que permet
calcular la funció temporal de la sèrie temporal $S(x)^r$ (v.\
\autoref{def:model:frepr}) on $x\in\glssymbol{not:temps-domini}$.  Per
seleccionar un interval temporal cal tenir en compte tant l'interval
sobre la seqüència com la funció temporal de la sèrie
temporal que pot incloure noves mesures al resultat.

\begin{definition}[Interval temporal]
  \label{def:sgst:intervalt}
  Sigui $S$ una sèrie temporal, $[s,t]$ un interval de temps i $r$ un
  mètode de representació, l'interval temporal
  $\glsdispdef{not:sgst:intervalt}{S[s,t]^r}$ és una sèrie temporal
  amb les mesures que són dins del rang temporal de l'interval $[s,t]$
  segons marca la funció de representació: $S[s,t]^r\equiv S(x)^r$ 
  per tot $x \in [s,t]$
\end{definition}

Aquesta és una definició genèrica difícil d'implementar, per tant per
a cada mètode de representació cal interpretar una operació d'interval
temporal. Més endavant, un cop haguem profunditzat el
concepte de mètode de representació, oferirem  exemples
d'intervals temporals particularitzats per mètodes de representació (v.\
\autoref{sec:sgst:repr-intervaltemporal})





Propietats de l'interval temporal:

\begin{itemize}
\item Sigui $t$ un instant de temps, l'interval temporal
  $S[t,t]^r$ és equivalent a la funció temporal de la sèrie
  temporal avaluada en aquest instant: $S[t,t]^r= \{(t,
  S(t)^r)\}$.

%$S[T(\min(S),T(\max(S)]^r \neq S$
%$S[-\infty,+\infty]^zohe = S \cup \{ (+\infty,\infty) \}$

\end{itemize}




\subsubsection{Selecció temporal}


La selecció temporal d'una sèrie temporal permet seleccionar, en el
context d'una representació, un conjunt d'instants de temps
determinats.  Així, també es pot utilitzar aquesta operació per a
canviar la resolució d'una sèrie temporal.


\begin{definition}[Selecció temporal]
  \label{def:sgst:selecciot}
  Sigui $S$ una sèrie temporal,
  $I=\{t_0,t_1,\dotsc,t_n\}$ un conjunt d'instants de temps i $r$ un
  mètode de representació. La selecció temporal, notada com
  $\glsdispdef{not:sgst:selectt}{S[I]^r}$, és una sèrie temporal que
  conté mesures amb els temps d'$I$ segons marca el mètode de
  representació: $S[I]^r= S[t_0,t_0]^r \cup S[t_1,t_1]^r \cup \dotsb
  \cup S[t_n,t_n]^r$.
\end{definition}



Propietats de la selecció temporal:
\begin{itemize}

\item El cardinal de la sèrie temporal resultant és el mateix que el
  del conjunt d'instants de temps $|S[I]^r| = |I|$

% \item La selecció temporal d'una sèrie temporal en un conjunt de temps
%   equi-espaiat $i = \{\tau+n\delta | n,j\in\glssymbol{not:N}, n\leq j
%   \}$ és una sèrie temporal regular $S[i]^r \equiv \{ (\tau, v_0),
%   (\tau+\delta,v_1), \dotsc , (\tau+j\delta,v_j)\}$ \todo{posar-ho a regularitat?}
\end{itemize}




\subsubsection{Concatenació temporal}

La concatenació temporal és l'operació de concatenació que té en
compte la representació de les sèries temporals.  És a dir, la
concatenació temporal de dues sèries temporals uneix la part de la
segona sèrie temporal que no està inclosa en l'interval temporal de la
primera.


\begin{definition}[Concatenació temporal]
  Siguin $S_1$ i $S_2\}$ dues sèries temporals i $r$ un mètode de representació,
  la concatenació temporal de les dues sèries temporals, $S_1
  \glssymboldef{not:sgst:concatenatet}^r S_2$, és una sèrie temporal
  que conté les mesures de $S_1$ i les mesures de $S_2$ que no
  intersequen en l'interval temporal de $S_1$: $S_1
  \glssymbol{not:sgst:concatenatet}^r S_2 = S_1[s,t]^r \cup
  S_2[-\infty,s]^r \cup S_2[t,+\infty]^r$ on $s=T(\inf S_1)$ i
  $t=T(\sup S_1)$.
\end{definition}

Propietats de la concatenació temporal:
\begin{itemize}
\item No commutativa
\end{itemize}


% Exemples $S_1=\{(2,1),(4,1),(6,1)\}$ i $S_2=\{(1,2),(2,2)\}$,
% $S_1 \glssymbol{not:sgst:concatenatet}^\gls{zohe} S_2 = \{(1,2),(2,2),(4,1),(6,1),(+\infty,\infty)\}$
% $S_2 \glssymbol{not:sgst:concatenatet}^\gls{zohe} S_1 = \{(1,1),(2,3),(4,1),(6,1),(+\infty,\infty)\}$


\subsubsection{Junció temporal}

La junció temporal de dues sèries temporals és la junció que té en
compte la representació de les sèries temporals. És a dir, la junció
temporal de dues sèries temporals ajunta parelles de mesures
seleccionant el mateix atribut de temps en ambdues sèries temporals.


\begin{definition}[Junció temporal]\label{def:sgst:joint}
  Siguin $S_1$ i $S_2$ dues sèries temporals i $r$ un mètode de
  representació, la junció temporal de les dues sèries temporals, $S_1
  \glssymboldef{not:sgst:joint}^r S_2$, és una sèrie temporal
  multivaluada que ajunta les mesures seleccionant els mateixos temps
  a cada sèrie temporal segons el mètode de representació: $S_1
  \glssymbol{not:sgst:joint}^r S_2 = \{ (x,v,w) | x \in
  (\glssymbol{not:sgst:project}_{t}(S_1) \cup
  \glssymbol{not:sgst:project}_{t}(S_2)) \wedge (x,v) \in S_1[x,x]^r
  \wedge (x,w) \in S_2[x,x]^r \}$
\end{definition}


Propietats de la junció temporal:
\begin{itemize}
\item El cardinal resultant és $|S_1 \glssymbol{not:sgst:joint}^r S_2|
  \leq |S_1| + |S_2|$
\item És commutativa; tenint en compte que els atributs tenen nom i
  per tant l'ordre no importa.
\end{itemize}



També es defineix l'operació de semijunció temporal que és una junció
no commutativa on la primera sèrie temporal marca el vector de temps
de junció.

\begin{definition}[Semijunció temporal]
  Sigui $S_1$ i $S_2$ dues sèries temporals i $r$ un mètode
  de representació, la semijunció temporal de les dues sèries
  temporals, $S_1 \glssymboldef{not:sgst:semijoint}^r S_2$, és una
  sèrie temporal multivaluada que ajunta les mesures de la primera
  sèrie temporal a les mesures de la segona segons el mètode de
  representació: $S_1 \glssymboldef{not:sgst:semijoint}^r S_2 = S_1
  \glssymboldef{not:sgst:joint}^r S_2[\glssymbol{not:sgst:project}_{t}(S_1)]^r$.
\end{definition}


Propietats de la semijunció temporal:
\begin{itemize}
\item El cardinal resultant és $|S_1
  \glssymboldef{not:sgst:semijoint}^r S_2| = |S_1|$.
\item No és commutativa.
\end{itemize}








%%% Local Variables:
%%% TeX-master: "main"
%%% End:







% LocalWords:  SGST

%--------------------
% document principal
%--------------------
% cal compilar amb 
% 1. `pdflatex main.tex`
% 2. `biber main`
% 3. `makeglossaries main`
% 4. `pdflatex main.tex`
%--------------------
\documentclass[paper=a4,parskip=half,
twoside,fontsize=11pt,BCOR12mm,
%oneside,fontsize=11pt, %%format web
% twoside,openany,fontsize=10pt,DIV = 10,BCOR12mm, %%format en paper
%,toc=listof, %afegir índexs a l'índex
]{scrbook}
%%%%BCOR12mm  factor de correcció per enquadernació en rústica
%%%%BCOR??mm  factor de correcció per enquadernació amb espiral -0.5mm??
%------------- capçalera ----------------------
%--------------------
% capçalera del document 
%--------------------
\usepackage[utf8]{inputenc}
\usepackage[catalan]{babel}
\usepackage{lmodern}
%\usepackage[T1]{fontenc} 
%\usepackage{parskip}
%---------- Gràfics ------------------------
%\usepackage[final]{graphicx}
%\usepackage{epstopdf}
%\usepackage{tikz}
%\usepackage{epic,xcolor,multicol}
\usepackage{subcaption}
\usepackage{multirow}
\usepackage{pgfplots}
\usetikzlibrary{dateplot}
\def\pgfcalendarmonthshortname#1{%
%\translate{
\ifcase#1\or gen.\or febr.\or març\or abr.\or maig\or juny\or jul.\or ag.\or set.\or oct.\or nov.\or des.\fi}%}
% \def\pgfcalendarmonthname#1{%
% \translate{\ifcase#1\or JJJ\or FFF\or MMM\or AAA\or
% MAY\or JJU\or JJY\or AAU\or SSS\or OOO\or
% NNN\or DDD\fi}}
\usetikzlibrary{shapes,arrows,positioning}
\usetikzlibrary{patterns}
%%%\usepackage{arydshln} %hdashline a les taules --> conflicte amb supertabular 
%----------- Format -------------------------
\usepackage{cclicenses} %\usepackage{ccicons}
%---------- Símbols ------------------------
\usepackage{url} %\url i \path
\usepackage{eurosym}
\usepackage[cmex10]{amsmath}
\usepackage{amssymb}
%%ús coma decimal sense espais:  2{,}5
%per tal que trenqui la coma en math inline mode 
%http://tex.stackexchange.com/questions/19094/allowing-line-break-at-in-inline-math-mode-breaks-citations/
\AtBeginDocument{%
  \mathchardef\mathcomma\mathcode`\,
  \mathcode`\,="8000 
}
{\catcode`,=\active
  \gdef,{\mathcomma\discretionary{}{}{}}
}
%---------- Teoremes ------------------------
\usepackage{amsthm}
\usepackage{thmtools} %listoftheorems??
%teoremes
\theoremstyle{plain}
\newtheorem{definition}{Definició}
%\declaretheorem{definition}
\theoremstyle{definition}
\newtheorem{example}{Exemple}
%\declaretheorem{example}
%numeracions d'equacions, definicions, etc.
\numberwithin{equation}{chapter}
\numberwithin{definition}{chapter}
\numberwithin{example}{chapter}
%%format del listoftherorems
\makeatletter
\def\ll@example{%
  \protect\numberline{\csname the\thmt@envname\endcsname}%
  \ifx\@empty\thmt@shortoptarg
    \thmt@thmname
  \else
    \thmt@shortoptarg
  \fi}
\def\ll@definition{%
  \protect\numberline{\csname the\thmt@envname\endcsname}%
  \ifx\@empty\thmt@shortoptarg
    \thmt@thmname
  \else
    \thmt@shortoptarg
  \fi}
\makeatother
%---------- Bibliografia -------------------
%ús: \cite{} \textcite{} \parencite{} \citeauthor{}
\usepackage[
style=numeric-comp,%style=authoryear
sortcites=true,
%backref=true,
 ]{biblatex}
%\ExecuteBibliographyOptions{annotation=true,backref=true,}
%backref=true, urldate=long, abbreviate=false,%%format web 
% isbn=false,url=false,doi=false,alldates=terse,firstinits=true,abbreviate=true %%format en paper

\DefineBibliographyStrings{catalan}{%
%    andothers = {i altres},
    backrefpage  = {citat a la p\adddot}, % for single page number
    backrefpages = {citat a les pp\adddot} % for multiple page numbers
}

\newcommand{\bibendash}{--}
%\bibparsep 0.2cm
%\bibhang 0.25cm

%afegeix camp eprinttype=hdl
\DeclareFieldFormat{eprint:hdl}{%
  \mkbibacro{HDL}\addcolon\space
  \ifhyperref
    {\href{http://hdl.handle.net/#1}{\nolinkurl{#1}}}
    {\nolinkurl{#1}}}

%anotacions a la bibliografia
\newboolean{bbx@annotation}% (same as biblatex-dw)
\DeclareBibliographyOption{annotation}[true]{%
\setboolean{bbx@annotation}{#1}}
%
\renewbibmacro{finentry}{%
\finentry%
\iffieldundef{annotation}%
{}%
{\ifbool{bbx@annotation}%
{\color{blue}
\begin{quotation}\noindent%
\printfield{annotation}%
\end{quotation}}%
{}}%
}

%cites bibliogràfiques: números de citacions entre comes
%cal escapar els accents: \parencite{tal} {é}s
%http://tex.stackexchange.com/questions/28461/biblatex-tighter-integration-of-textcite-in-the-flow-of-text
%http://tex.stackexchange.com/questions/19627/biblatex-idiom-for-testing-contents-of-list-field
%http://tex.stackexchange.com/questions/26401/help-to-develop-a-textcite-command-to-be-used-with-verbose-citation-styles-in-b
%http://tex.stackexchange.com/questions/28461/biblatex-tighter-integration-of-textcite-in-the-flow-of-text
%
% \makeatletter
% %\DeclareAutoPunctuation{.,;:!?}
% %per defecte \DeclareRangeChars{~,;-+/}
% \DeclareRangeChars*{:}
% %
% \newcommand{\citacomes}[4]{
% \blx@addpunct{comma}\space\cite[#1][#2]{#3}%
% %\blx@imc@ifpunctmark{#4}{\blx@addpunct{comma}\space#4}{#4}%
% \blx@imc@ifnumerals{#4}{%per separar accents
% \blx@imc@ifpunctmark{#4}{\blx@addpunct{comma}\space#4}{#4}}%
% {\blx@addpunct{comma}\space#4}%
% }
% %
% \renewrobustcmd*{\textcite}{\blx@citeargs\cbx@textcite}
% \newcommand{\cbx@textcite}[4]{%
% \citeauthor{#3}%
% \citacomes{#1}{#2}{#3}{#4}%
% }

% \renewrobustcmd*{\parencite}{\blx@citeargs\cbx@parencite}
% \newcommand{\cbx@parencite}[4]{%
% \citacomes{#1}{#2}{#3}{#4}%
% }
% %
% \DeclareCiteCommand{\citeauthor}
%   {\usebibmacro{cite:init}%
%     \boolfalse{citetracker}%
%     \boolfalse{pagetracker}%
%     \usebibmacro{prenote}}%pre
%   {\ifciteindex
%      {\indexnames{labelname}}
%      {}%
%      \iffieldequals{namehash}{\cbx@lasthash}%
%      {}%repetit 
%      {\ifnumequal{\value{citecount}}{1}{}{\multicitedelim}%
%        \printnames{labelname}}%
%      \savefield{namehash}{\cbx@lasthash}%
% }%post
%   {}%\multicitedelim}%sep
%   {\usebibmacro{postnote}}
% %
% \makeatother
%---------- Codi ---------------------------
%% ús del lstlisting
%%\begin{lstlisting}[language=C,caption=Titol del llistat,label=lst:etiqueta,numbers=left]
%%\lstinline[style=sh]!for i:integer;!

\usepackage{upquote} %perquè en verbatim surtin les cometes `
\usepackage{listings}
\lstloadlanguages{bash,C,HTML,Python,XML}
\lstset{numberstyle=\tiny,frame=single,frameround=tttt,
        breaklines=true,breakindent=0pt,
%        prebreak=\mbox{{\color{blue}\tiny$\searrow$}},
%        postbreak=\mbox{{\color{blue}\tiny$\hookrightarrow$}},
        columns=[l]fullflexible,
        xleftmargin=1em,
        extendedchars=true,
        literate={à}{{\`a}}1 {è}{{\`e}}1 {é}{{\'e}}1 {í}{{\'\i}}1 {ï}{{\"\i}}1
                 {ò}{{\`o}}1 {ó}{{\'o}}1 {ú}{{\'u}}1 {ü}{{\"u}}1
                 {ç}{{\c{c}}}1 {l·l}{{\l.l}}1
                 {À}{{\`A}}1 {È}{{\`E}}1 {É}{{\'E}}1 {Í}{{\'I}}1 {Ï}{{\"I}}1
                 {Ò}{{\`O}}1 {Ó}{{\'O}}1 {Ú}{{\'U}}1 {Ü}{{\"U}}1
                 {Ç}{{\c{C}}}1 {L·L}{{\L.L}}1, 
        escapechar=æ,
        }

\lstdefinestyle{py}{
  style=pycolor
}

\lstdefinestyle{pynocolor}{
  language=python,
  frame=none,
  inputencoding=utf8,
  backgroundcolor=\color[gray]{0.95},
}


\lstdefinestyle{pycolor}{
        style=pynocolor,
        basicstyle=\sffamily\footnotesize,
        stringstyle=\color{brown},
        showstringspaces=false,
        alsoletter={1234567890},
        otherkeywords={\ , \}, \{},
        keywordstyle=\color{blue},
        emph={access,and,as,break,class,continue,def,del,elif,else,%
          except,exec,finally,for,from,global,if,import,in,is,%
          lambda,not,or,pass,print,raise,return,try,while,assert},
        emphstyle=\color{orange}\bfseries,
        emph={[2]self},
        emphstyle=[2]\color{gray},
        emph={[4]ArithmeticError,AssertionError,AttributeError,BaseException,%
          DeprecationWarning,EOFError,Ellipsis,EnvironmentError,Exception,%
          False,FloatingPointError,FutureWarning,GeneratorExit,IOError,%
          ImportError,ImportWarning,IndentationError,IndexError,KeyError,%
          KeyboardInterrupt,LookupError,MemoryError,NameError,None,%
          NotImplemented,NotImplementedError,OSError,OverflowError,%
          PendingDeprecationWarning,ReferenceError,RuntimeError,RuntimeWarning,%
          StandardError,StopIteration,SyntaxError,SyntaxWarning,SystemError,%
          SystemExit,TabError,True,TypeError,UnboundLocalError,UnicodeDecodeError,%
          UnicodeEncodeError,UnicodeError,UnicodeTranslateError,UnicodeWarning,%
          UserWarning,ValueError,Warning,ZeroDivisionError,abs,all,any,apply,%
          basestring,bool,buffer,callable,chr,classmethod,cmp,coerce,compile,%
          complex,copyright,credits,delattr,dict,dir,divmod,enumerate,eval,%
          execfile,exit,file,filter,float,frozenset,getattr,globals,hasattr,%
          hash,help,hex,id,input,int,intern,isinstance,issubclass,iter,len,%
          license,list,locals,long,map,max,min,object,oct,open,ord,pow,property,%
          quit,range,raw_input,reduce,reload,repr,reversed,round,set,setattr,%
          slice,sorted,staticmethod,str,sum,super,tuple,type,unichr,unicode,%
          vars,xrange,zip},
        emphstyle=[4]\color{purple}\bfseries,
        morecomment=[s][\color{lightgreen}]{"""}{"""},
        commentstyle=\color{gray}\slshape,
        literate=
          {>>>}{\textbf{\textcolor{blue}{$>$\kern-.5ex$>$\kern-.5ex$>$}~}}3%
          {...}{{\textcolor{gray}{...}}}3%
          {à}{{\`a}}1 {è}{{\`e}}1 {é}{{\'e}}1 {í}{{\'\i}}1 {ï}{{\"\i}}1%
          {ò}{{\`o}}1 {ó}{{\'o}}1 {ú}{{\'u}}1 {ü}{{\"u}}1 {ç}{{\c{c}}}1%
          {l·l}{{\l.l}}1 {À}{{\`A}}1 {È}{{\`E}}1 {É}{{\'E}}1 {Í}{{\'\I}}1%
          {Ï}{{\"\I}}1 {Ò}{{\`O}}1 {Ó}{{\'O}}1 {Ú}{{\'U}}1 {Ü}{{\"U}}1%
          {Ç}{{\c{C}}}1 {L·L}{{\L.L}}1, 
        rulesepcolor=\color{blue},
} 




\lstdefinelanguage{tutorialD}{
  morekeywords={\ , \}, \{, var, operator,with,as,end,begin,type,possrep},
  sensitive=false,
  morecomment=[l]{//},
%  morecomment=[s]{/*}{*/},
%  morestring=[b]",
}

\makeatletter
\newcommand{\lstuppercase}{\uppercase\expandafter{\expandafter\lst@token
                           \expandafter{\the\lst@token}}}
\newcommand{\lstlowercase}{\lowercase\expandafter{\expandafter\lst@token
                           \expandafter{\the\lst@token}}}
\makeatother


\lstdefinestyle{tutorialD}{
        language=tutorialD,
        frame=none,
        inputencoding=utf8,
        backgroundcolor=\color[gray]{0.95},
        basicstyle=\sffamily\footnotesize,
        stringstyle=\color{green},
        keywordstyle={\color{blue}\lstlowercase},
        emph={tuple,relation,base,key,same_type_as,same_heading_as,return,returns,from,execute,init,private,all,but,%
%          where,rename,union,join,minus,extend,add,summarize,group,ungroup
        },
        emphstyle={\color{orange}\bfseries\lstlowercase},
        emphstyle=[2]\color{gray},
        emph={[4]rational,char},
        emphstyle=[4]\color{purple}\bfseries,
        morecomment=[s][\color{lightgreen}]{"""}{"""},
        commentstyle=\color{red}\slshape,
        rulesepcolor=\color{blue},
} 







\lstdefinestyle{sh}{
  language=bash,
  frame=none,
  prebreak =\textbackslash,
  postbreak ={},
  basicstyle=\ttfamily,
  showspaces=false,
  keepspaces=true,
  backgroundcolor=\color[gray]{0.95},
}
\lstdefinestyle{file}{
  frame=none,
  showspaces=false,
  keepspaces=true,
  backgroundcolor=\color{yellow!20!white}
}

\lstdefinestyle{stdout}{
  frame=none,
  keepspaces=true,
  columns=fixed,
  backgroundcolor=\color{yellow!20!white}
}
%--------------------------------------------


%---------- hyperref ------------------------
\usepackage[bookmarks,pdfborder={0 0 0},pdfusetitle]{hyperref}
\usepackage{bookmark}%per netejar l'ordre de \part{}

%modificar autoref per babel catala
\let\orgautoref\autoref
%per: (v. fig. 3)
\providecommand{\seeref}
{%
\def\figureautorefname{fig.}%
\def\definitionautorefname{def.}%
\def\exampleautorefname{ex.}%
\def\chapterautorefname{\gls{capitol}}%
\def\sectionautorefname{\gls{seccio}\ignorespaces}%
\def\subsectionautorefname{\gls{seccio}\ignorespaces}% 
\gls{vegeu}~\orgautoref%
}
%per: a la figura 3 !Problemes en les contraccions amb noms masculins! evitar a i de davant de capítol!
\providecommand{\textref}
{%
\def\figureautorefname{la figura}%
\def\definitionautorefname{la definició}%
\def\exampleautorefname{l'exemple}%
\def\chapterautorefname{el capítol}% al/del capítol???
\def\sectionautorefname{la secció}%
\def\subsectionautorefname{l'apartat}% 
\orgautoref%
}
%per defecte
\renewcommand{\autoref}
{%
\def\figureautorefname{figura}%ús: \autoref{}
\def\tableautorefname{taula}%ús: \autoref{}
\def\definitionautorefname{definició}%ús: \autoref{}
\def\exampleautorefname{exemple}%ús: \autoref{}
\def\chapterautorefname{capítol}%ús: \autoref{}
\def\sectionautorefname{secció}%ús: \autoref{}
\def\subsectionautorefname{apartat}%ús: \autoref{}
\orgautoref%
}


\renewcommand\lstlistingname{Llistat} %%%PENSAR bé el nom
\renewcommand\lstlistlistingname{Índex de llistats}
\def\lstlistingautorefname{llistat} %ús: \autoref{}
%---------- Glossaris -------------------
\usepackage{tabu}
%\usepackage{longtable}
%\usepackage{supertabular}
\usepackage[
          nomain,% Remember if you don't want to use the main glossary.
          acronym,
          %%nonumberlist,
          toc,
          section,
          numberedsection=false,%numberedsection=autolabel,
          sanitize=none, %pels accents en el vegeu
          ]{glossaries}
%\renewcommand{\glossarypreamble}{Text com a préambul}
%\renewcommand*{\glspostdescription}{}%anul·la el punt final
\renewcommand*{\acronymname}{Sigles}%{Índex de sigles}??si té refs pàgines 
%Índex d'abreviacions?? si conté abreviatures o símbols
\renewcommand{\seename}{vegeu}
\renewcommand{\entryname}{Notació}
\renewcommand{\descriptionname}{Descripció}
% \short<type>name,
%Nous usos de símbols
\newcommand{\glssymboldef}{\glssymbol[format=hyperbf,counter=definition]}
\newcommand{\glsdispdef}{\glsdisp[format=hyperbf,counter=definition]}
\newcommand{\hyperbfsec}[1]{\textbf{\S\hypersf{#1}}}
\newcommand{\hyperbfex}[1]{\textbf{ex.\hypersf{#1}}}
\newcommand{\glsaddsec}{\glsadd[format=hyperbfsec,counter=subsection]}
\newcommand{\glsaddchap}{\glsadd[format=hyperbfsec,counter=chapter]}
\newcommand{\glsaddsection}{\glsadd[format=hyperbfsec,counter=section]}
\newcommand{\glsdispsec}{\glsdisp[format=hyperbfsec,counter=subsection]}
\newcommand{\glssymbolsec}{\glssymbol[format=hyperbfsec,counter=subsection]}
\newcommand{\glssymbolex}{\glssymbol[format=hyperbfex,counter=example]}
%Nous glossaris
\newglossary{abreviatura}{abr}{brv}{Abreviatures}
\newglossary{notation}{not}{ntn}{Símbols i notació}
\newglossarystyle{estil-notation}{%
%  \renewcommand{\glsgroupskip}{}% make nothing happen between groups
%  \renewcommand*{\glossaryheader}{}
% {\begin{longtable*}{llp{10em}}}{\end{longtable*}}
  \renewenvironment{theglossary}
  {\begin{center}\begin{supertabular*}{\textwidth}{llp{10em}}}{\end{supertabular*}\end{center}} %
  \renewcommand*{\glossarysubentryfield}[6]{%
    \ifglshaschildren{##2} %
    {\glstarget{##2}{\textbf{\Glsentryname{##2}}}} %hi ha subentrades
    {\space \glstarget{##2}{\Glsentryname{##2}}} % the entry name
    %& (##5) % the symbol in brackets
    & ##4 % the description
    & ##6 % the number list 
%    & \parbox[t]{\hsize}{hola\\ ei} 
    % ##1 entry level
    \\
  }%
  \renewcommand*{\glossaryentryfield}[5]{%   
    %\glossarysubentryfield{##2}{##1}{##2}{##3}{##4}{##5}
    \\    
    \glstarget{##1}{\textbf{\Glsentryname{##1}}}  & ##3 & ##5 \\\hline
  }
}

%Executa els glossaris
\makeglossaries
%--------------------------------------------




%%% Local Variables: 
%%% mode: latex
%%% TeX-master: "main"
%%% End: 

\ExecuteBibliographyOptions{backref=true}
%backref=true, urldate=long, abbreviate=false,%%format web 
%annotation=true %per veure comentaris
%---------- Esborrany --------------------
%\includeonly
%{resum}
%{intro}
%{art,seriestemporals,sgbd,sistemessimilars}
%{model,sgst,sgst-operacions,sgst-naturalesa,sgstm,sgstm-operacions,sgstm-interpoladors}
%{variacions,multiresolucio,informacio}
%{implementacio,python,mapreduce,relacional}
%{exemple}
%{conclusions,futur}
%\usepackage[catalan]{todonotes} %%ús: \todo{text} \missingfigure{text}
%\usepackage{scrpage2}\pagestyle{scrheadings}\chead[--- esborrany \today\ ---]{\color{gray}{\today}}   
%\usepackage[showframe,a4paper]{geometry}%page margins %\geometry{layoutheight=230mm,layoutwidth=160mm,layoutvoffset=30mm,showcrop}
%\usepackage[cam,a3,center]{crop} %frame
%------------- Format -------------------------
\usepackage{imatges/tikz-uml}
%\usetikzlibrary{dateplot}  
%\usetikzlibrary{pgfplots.groupplots}






% \pgfplotsset{
%    rrbtimeseries/.style={
%         % date coordinates in=x,
%         ylabel=Temperature (K),
%         % legend style={font=\footnotesize},
%         % tick label style={font=\footnotesize},
%         % every axis x label/.style={
%         %   at={(1.3,0)},
%         %   anchor=north,
%         %   },
%         % label style={font=\footnotesize},
%         % xticklabel style= {rotate=17,anchor=north east},
% %        every axis title shift=0pt,
% %        max space between ticks=15,
%        %  every mark/.append style={mark size=6},
%        %  major tick length=0.1cm,
%        %  minor tick length=0.066cm,
%        %  very thin,
%        %  every axis legend/.append style={
%        %    at={(1.2,0)},
%        %    anchor=south east,
%        %    draw = none},
%        % legend columns = 4,
%        % unbounded coords=jump, %v>1.4
%     },

%   % rrbrs/.style={
%   %       rrbtimeseries,
%   %       width = \textwidth,
%   %       height = 0.25\textwidth,
%   %       every axis x label/.style={
%   %         at={(1.3,-1)},
%   %         anchor=north,
%   %         },
%   %       ylabel = {},  
%   %       max space between ticks=50,
%   %       every axis legend/.append style={
%   %         at={(1,-1.1)},
%   %         anchor=north east,
%   %         draw = none},
%   %       title style={font=\small,below,anchor=north,fill=white},
%   %   },
%  }




\pgfplotsset{
   timeseries/.style={
%        date coordinates in=x,
        ylabel=Temperature (K),
        legend style={font=\footnotesize},
        tick label style={font=\footnotesize},
        every axis x label/.style={
          at={(1.3,0)},
          anchor=north,
          },
        label style={font=\footnotesize},
        xticklabel style= {rotate=17,anchor=north east},
%        every axis title shift=0pt,
%        max space between ticks=15,
        every mark/.append style={mark size=6},
        major tick length=0.1cm,
        minor tick length=0.066cm,
        very thin,
        every axis legend/.append style={
          at={(1.2,0)},
          anchor=south east,
          draw = none},
       legend columns = 4,
    },
    rd/.style={
        timeseries,
        % every axis x label/.style={
        %   at={(1.3,-1)},
        %   anchor=north,
        %   },
        label style={font=\footnotesize},
        ylabel = {},
        width=\textwidth,
        height=3.5cm,   
        max space between ticks=50,
        every axis legend/.append style={
          at={(1,-1.1)},
          anchor=north east,
          draw = none},
%        title style={font=\small,below, at={(0.7,1.7)},anchor=north},
    }
}


%        unbounded coords=jump, %v>1.4
%        unbounded coords=discard, %v>1.4


%http://tex.stackexchange.com/questions/46422/axis-break-in-pgfplots

%http://tex.stackexchange.com/questions/52409/insert-a-separate-mark-inside-a-pgfplots-graph
%---------- Bibliografia i glossaris ------------
\bibliography{bibliografia}
\loadglsentries{vocabulari/abreviacions.tex}
\loadglsentries[notation]{vocabulari/notacio.tex}
%-------------- dades --------------------------
\hypersetup{
    pdftitle={Disseny i modelització d'un sistema de gestió multiresolució per a sèries temporals},
    pdfauthor={Aleix Llusà Serra},
    pdfcreator={DiPSE--UPC},
    pdfsubject={Tesi 2011--2015},
    pdfkeywords={sèries temporals; model de dades; sistemes de bases de dades; sistemes de monitoratge},
    pdflang={ca},
}

\title{Disseny i modelització d'un sistema de gestió multiresolució per a sèries temporals}
\author{Aleix Llusà Serra}
%----------------------------------------------


\begin{document}
%\frontmatter
%------------- Pàgina de títol ------------
%\maketitle
%------------- pàgina de portada -----------
\begin{titlepage}
  \begin{center} 

   

    {\Large \scshape Universitat Politècnica de Catalunya} \vskip 1cm 

    {Programa de Doctorat:} \vskip 0.5cm 
    
    {\scshape Automàtica, Robòtica i Visió} \vfill%\vskip 4cm 

    {Tesi Doctoral} \vskip 1cm 
    
    {\scshape \bfseries \Large Disseny i modelització d'un sistema de gestió\\
 multiresolució per a sèries temporals} \vskip 2cm

    {\bfseries Aleix Llusà Serra} \vfill%\vskip 4cm 

    {Direcció:}
       
    {Teresa Escobet Canal i
    Sebastià Vila-Marta}  \vskip 1cm 
    %\vfill 

    {Juny de 2015}

\end{center}
\end{titlepage}

%----------------------------------------------

%------------- Abstract ------------
%\begin{abstract}
\chapter*{Resum}

\todo{repassar la taula de notació que a vegades fa un salt de pàgina on no toca}

Actualment és possible d'adquirir una gran quantitat de dades,
principalment gràcies a la facilitat de disposar de sistemes de
monitoratge amb grans xarxes de sensors. Això no obstant, no és tan
senzill de gestionar posteriorment totes aquestes dades.  A més, també
cal tenir en compte com s'emmagatzemen aquestes dades.


D'una banda, l'adquisició de valors d'una variable al llarg del temps
es formalitza com a sèrie temporal. Així, hi ha multitud d'algoritmes
i metodologies d'anàlisi de sèries temporals que descriuen com
extreure informació de les dades. D'altra banda, l'emmagatzematge i la
gestió de les dades es formalitza com a \glspl{SGBD}. Així, hi ha
sistemes informàtics dedicats a inferir la informació que un usuari
vol consultar. Aquests sistemes són descrits per models lògics
formals, entre els quals el model relacional n'és la referència
principal.


En aquesta tesi dissertem sobre el fet d'emmagatzemar només aquella
part de les dades originals que conté una certa informació
seleccionada. Aquesta selecció de la informació es duu a terme
mitjançant el resum de diferents resolucions de les dades, cadascuna
de les quals bàsicament són agregacions de les dades a intervals de
temps periòdics. A aquesta tècnica l'anomenem multiresolució.



La multiresolució s'aplica a les sèries temporals. Com a resultat,
s'obtenen subsèries temporals de mida finita i amb la informació
resumida. Per tal de gestionar les sèries temporals, s'utilitzen
\gls{SGBD} específics anomenats \glspl{SGST}. Així doncs, proposem
\gls{SGST} amb capacitats de multiresolució i els anomenem
\glspl{SGSTM}. De la mateixa manera que en els \gls{SGBD}, formalitzem
un model pels \gls{SGST} i pels \gls{SGSTM}.



A causa de la naturalesa de variable capturada al llarg del temps, en
l'adquisició de les sèries temporals apareixen propietats
problemàtiques. Els \gls{SGSTM} tenen en compte algunes d'aquestes
propietats com:
\begin{itemize}
\item La sincronització dels rellotges en els diferents sistemes
  d'adquisició.
\item L'aparició de dades desconegudes perquè no s'han pogut adquirir
  o perquè són errònies.
\item La gestió d'una quantitat enorme de dades, i que a més segueix
  creixent al llarg del temps.
\item Les consultes amb dades que no s'han recollit de manera uniforme
  en el temps.
\end{itemize}


Ara bé, els \gls{SGSTM} són uns sistemes que emmagatzemen unes dades
segons una selecció d'informació i descarten les que no es consideren
importants. Per tant, prèviament a l'emmagatzematge, cal decidir els
paràmetres de selecció de la informació. Per tal d'avaluar la qualitat
d'aquests sistemes, depenent dels paràmetres que s'escullin, es pot
utilitzar la teoria de la informació. En aquest sentit, la
multiresolució es pot considerar com una tècnica de compressió amb
pèrdua. Així doncs, introduïm una reflexió sobre com avaluar l'error
que es comet amb la multiresolució en comparació amb disposar de totes
les dades originals.


Com es diu actualment en l'àmbit dels \gls{SGBD}, un mateix sistema no
pot ser adequat per a tots els contextos. A més, els sistemes han de
tenir en compte un bon rendiment en altres recursos a part del temps
de computació, com per exemple la capacitat finita, el consum
d'energia o la transmissió per la xarxa. Així doncs, dissenyem
diverses implementacions del model dels \gls{SGSTM}. Aquestes
implementacions exploren diverses tècniques de computació: computació
incremental seguint el flux de dades, computació para\l.lela i
computació de bases de dades relacional.


En resum, en aquesta tesi dissenyem els \gls{SGSTM} i en formalitzem
un model.  Els \gls{SGSTM} són útils per a emmagatzemar sèries
temporals en sistemes amb capacitat finita i per a precomputar la
multiresolució. D'aquesta manera, permeten disposar de consultes i
visualitzacions immediates de les sèries temporals de forma
resumida. Això no obstant, impliquen una selecció de la informació que
cal decidir prèviament. En aquesta tesi proposem consideracions i
reflexions sobre els límits de la multiresolució.





\chapter*{Abstract}






\glsreset{SGBD}
\glsreset{SGST}
\glsreset{SGSTM}


%%% Local Variables: 
%%% mode: latex
%%% TeX-master: "main"
%%% End: 

%  LocalWords:  multiresolució

%\end{abstract}
%----------------------------------------------

%------------- Índex de continguts ------------
\cleardoublepage\pdfbookmark{\contentsname}{bookmark:index}\tableofcontents{}
%\addtocontents{toc}{\protect\enlargethispage{1cm}}
%----------------------------------------------



%------------- Cos ------------
%\mainmatter

\part{Introducció}

\begin{abstract}
  Current monitoring systems are an essential part of supervising
  control as they manage a large amount of information. Data
  collection is normally studied as a time series owing to it fits
  into sequence of values.  Thanks to the facility of designing
  monitoring hardware, the measurement of data has increased the last
  decade and there is not enough capacity to store nor process all the
  time series. Therefore, we need to design database management
  systems capable of storing and processing efficiently the time
  series. Moreover, this systems have to cope with the measurements
  not happening at regular time intervals as it is a restriction
  imposed by some time series treatment algorithms.

  In this paper a formal model for a time series database management
  system is designed.  It is called Multiresoltion Time Series
  Database Management Systems model (MTSMS). A Time series is
  compactly stored in the database and the information is summarised
  by different interpolation functions. From this model this kind of
  DBMS will be better understood, new implementations will be possible
  and we will be able to enhance its potential.
\end{abstract}



\section{Introduction}

This paper focuses on Data Base Management Systems (DBMS) that store
and treat data as time series.  Traditional DBMS, as is ones derived
from relational model, are not adequate for these cases as they do not
have enough facilities to manage and retrieve time series
information \parencite{schmidt95}.

Some DBMS have already taken into account the specificities of time
series, then called Time Series Data Base Management Systems
(TSMS) \parencite{dreyer94}.  Time Series Data
Server \parencite{weigel10} allows to select a data range from a time
series and to apply a filter when the data is retrieved.
RRDtool \parencite{rrdtool} applies filters and stores different data
ranges when data is stored, moreover it considers that the sampling
times can not be equally spaced, the temporal order is essential and
the value and time must be stored together. However, this TSMS lack
the complete definition of the relation between the three main fields
involved: time series, monitoring systems and DBMS.

Monitoring sensor data and processing this data to achieve diagnosis,
prognosis, prediction, data fusion or other time series analysis tasks are common in many fields such as prognosis in degradation models \parencite{yu11}, qualify sensor health in navy vessels \parencite{palmer07}, validate and reconstruct data in water distribution networks \parencite{quevedo10}, sensor networks information dissemination \parencite{deligiannakis07} or economic stock classification \parencite{dreyer95}. Time series data mining formalises the knowledge discovery in time series databases \parencite{last01}. 


DBMS are based from formal models that define the objects and
operations of the abstract machine to which users interact, such is
the relational model \parencite{date}. TSMS lack a consolidated formal
model, although special properties and requirements for a TSMS
have been proposed \parencite{dreyer94}.

In this paper
a model is proposed for an storage system that will keep a time series
in a multiresolution and bounded way.  In this first proposal there
are six definitions that are related to the data storage mechanism:
measure, time series, buffer, disc, resolution disc, and multiresolution
database. Some of this concepts are familiar with RRDtool
operating mode.


\todo{estructura article}
The paper is organised as follows. In \autoref{sec:preliminaries} the preliminaries concerning TSMS are summarised. In \autoref{sec:TSMS} the state of TSMS is shown as well as known similar systems that are designed with some TSMS properties. In \autoref{sec:MTSMS} a model for a multiresolution TSMS is presented for the first time. Possible reference implementations of this model are explored in \autoref{sec:implementation}. In sections \ref{sec:conclusion} and \ref{sec:future}  the MTSMS is concluded and future work is proposed. In \autoref{sec:notation} the main nomenclature and notation used in the paper is summarised. 




%%% Local Variables: 
%%% mode: latex
%%% TeX-master: "article"
%%% ispell-local-dictionary: "british"
%%% End: 


%-- ART --
\section{State of the art}


\todo{
  * Sèries temporals (històrics, predicció, diagnosis, prognosis, etc.)
  * Mostreig: docs quan període de mostreig no regular
  * Bases de dades (docs d'emmagatzematge quan la memòria és finita, docs quan període de mostreig no és regular, altres sistemes semblants (comercials,prototips))
}

En aquest capítol se situen els sistemes de gestió de bases de dades (SGBD) per sèries temporals en el context de la mineria de dades de sèries temporals (\emph{time series data mining}), el qual també es considerat com mineria de dades per  detectar automàticament coneixement (\emph{knowledge discovery databases}). Els SGBD de model Round Robin (RRD) pertanyen a aquest context ja que  emmagatzemen sèries temporals  de les quals es vol aconseguir informació rellevant.


El capítol comença resumint l'estat de les sèries temporals en aquest camp de mineria; és a dir d'emmagatzematge i tractament. A continuació es llisten algunes aplicacions informàtiques que han implementat models de la mineria de sèries temporals. Finalment, es descriu l'estat actual de l'aplicació RRDtool, la qual també es classifica en aquest camp.



\subsection{Time series data mining}

The analysis of time series includes different activities such as prediction in economics, weather forecasts, quality control in business, etc. In this context, time series data mining is the study and management of large collections of chronological data. Moreover, this data is continually increasing in size as time goes on.

Mining research in time series has increased the last decade as shown in \textcite{fu11}. Fu summarises exhaustively the current state of the art in time series data mining  and concludes that there is still room for us to further investigate and develop'. Mining tasks have had intensively research but the problem of  time series representation needs better solutions because of high dimensionality of time data. Furthermore, it is said to be one of the ten challenging problems in data mining \parencite{yangwu06}.
 
Time series data mining research  is given at mainly four tasks \parencite{keogh02}: indexing, clustering, classification and segmentation. In order to solve these tasks, lots of experimental algorithms have been proposed by different authors. \textcite{keogh02} compare and evaluate some of these algorithms with the same datasets and they recommend the time series data mining community to follow the same benchmark when evaluating the performance of similar algorithms. 

Representation of time series is a common step before the last four tasks. 
\emph{Piecewise Linear Representation} (PLR) \parencite{keogh97,keogh98}  is one representation widely used as it is said that human vision segments curvatures into lineal segments.




Un pas comú previ a aquestes quatre tasques és el de representació de la sèrie temporal. 
Keogh \emph{et al.},~\cite{keogh97,keogh98}, investiguen la representació de sèries temporals a trossos lineals (PLR, \emph{Piecewise Linear Representation}). Keogh fa notar que la representació PLR és l'habitual degut a que la visió de l'ésser humà segmenta les corbes en línies rectes.
Més tard, Keogh,~\cite{keogh00,keogh01}, explora la representació de sèries temporals per tal de reduir la dimensió d'una sèrie temporal i poder-la indexar més fàcilment  i proposa dues tècniques eficients en el càlcul: la PAA (\emph{Piecewise Aggregate Aproximation}) i  la APCA (\emph{Adaptive Piecewise Constant Approximation}), ambdues basades en la representació a trossos constants de la sèrie temporal. 
D'aquestes dues tècniques Keogh conclou que mantenen una bona aproximació a la sèrie temporal i que a més  tenen molt menys cost de càlcul que altres de més complicades, com ara la \emph{Discrete Fourier Transform} (DFT),  la  \emph{Singular Value Decomposition} (SVD) o la \emph{Discrete Wavelet Transform} (DWT).



\subsection{Monitoring}
Tal com expliquen Quevedo \emph{et al.},~\cite{quevedo10}, en un sistema complex de telecontrol hi ha una gran quantitat d'informació a manipular que s'obté de diversos sensors distribuïts pel camp de mesura, aquesta informació s'anomena variables mesurades. Un SCADA (\emph{Supervisory Control And Data Acquisition})  és el sistema encarregat de recollir i centralitzar les variables de manera periòdica en el temps. En el moment de reco\l.lecció de dades apareixen dos problemes: valors que en un instant de temps prefixat no s'han pogut recollir i valors que són falsos. Les tècniques de bases de dades no poden emmagatzemar les dades amb aquests dos tipus de problema ja que aleshores els registres històrics quedarien falsejats. Així doncs, cal comprovar que les dades emmagatzemades són correctes, segons un procés de validació, i modificar-les en el cas que siguin falses, segons un procés de reconstrucció. Quevedo,~\cite{quevedo10}, aplica aquests processos a xarxes de distribució d'aigua.

Els mètodes de validació i reconstrucció es poden basar en anàlisis senzilles del senyal o en comparacions del valor real amb models de predicció de dades. Quan les dades es tracten com a sèries temporals, hi ha mètodes de predicció específics.
Tot i que la teoria de sèries temporals permet establir aquests mètodes de predicció i reconstrucció, els SGBD habituals, com ara els de model relacional, no ho faciliten.  
Per tal d'aplicar aquests mètodes a les sèries temporals de manera eficient, els SGBD s'han d'especialitzar en el tractament de sèries temporals.



\subsection{SGBD per sèries temporals}


Per poder analitzar les dades de manera eficient cal disposar de bases de dades específiques, a més cada cop el volum de dades a tractar és més crític degut a que hi ha més facilitat a capturar-les i més capacitat per emmagatzemar-les. 
La diferència principal de les sèries temporals amb altres tipus de dades és que els valors són dependents d'una variable: el temps. Com a conseqüència, qualsevol base de dades que hi vulgui tractar no ho pot fer de manera independent pels valors i pel temps; ha de conservar la coherència temporal.

Tal com diu A{\ss}falg,~\cite{assfalg08:_advan_analy_temp}, la coherència temporal pot ser vista des de dues vessants. La primera, a la qual anomena \emph{bitemporal data}, consisteix en expressar el temps vàlid durant el qual un esdeveniment és cert i el temps de transacció durant el qual l'esdeveniment és guardat a la base de dades, és a dir consisteix a descriure dos estats, cert o fals, per cada observació. La segona, a la qual anomena \emph{time series data}, consisteix a descriure co\l.leccions de dades en funció del temps. A més diu que les primeres poden ser expressades amb les segones.

Els SGBD relacionals són capaços d'implementar el primer tipus de coherència, les \emph{bitemporal data}; llavors es classifiquen sota el nom de bases de dades temporals, \cite{date,wiki:temporal_database}. Però el model relacional no és suficient pel segon tipus: les sèries temporals. Tot i que en principi no hi hauria cap problema a utilitzar una base de dades relacional per a sèries temporals, enteses com a dades històriques, la pròpia naturalesa dels sistemes relacionals  dificulta les operacions necessàries. 
Aquestes operacions per sèries temporals es basen en rangs de temps i precisen conversions de fusos horaris i rotacions dels registres de les taules, sinó el nombre de files creixeria de forma indefinida. 

Els SGBD que implementen operacions per a sèries temporals es poden anomenar \emph{Time Series Database Systems} (TSDS),~\cite{tsds}. Les TSDS Estan optimitzades per gestionar les dades segons les operacions de temps i rotació, les quals són molt comunes en la gestió de les sèries temporals.  A més també cal controlar el creixement de la base de dades i la consulta ha de ser flexible i d'alta velocitat,~\cite{keogh10:isax}. Per exemple, s'han de poder visualitzar les evolucions tant d'una setmana com d'un any sense haver de fer càlculs complicats amb els valors emmagatzemats. 
A continuació es llisten dues bases de dades optimitzades per a sèries temporals.

A{\ss}falg,~\cite{assfalg08:_advan_analy_temp}, presenta un TSDS que és capaç de
cercar similituds, també anomenades distàncies, entre sèries temporals. Principalment utilitza llindars per comparar en cada interval si les dues sèries temporals s'assemblen. A partir d'aquest mètode desenvolupa algoritmes que calculen de manera eficient per a les sèries temporals i en concret els implementa en una aplicació anomenada T-Time, la qual descriu a~\cite{assfalg08:ttime}.

Keogh i Camerra~\cite{keogh08:isax,keogh10:isax}, 
estudien l'anàlisi i l'indexat de co\l.lecions massives de sèries temporals. Descriuen que el problema principal del tractament rau en l'indexat de les sèries temporals i proposen mètodes per calcular-lo de manera eficient. El mètode principal que desenvolupen està basat en l'aproximació a trossos constants de la sèrie temporal (PAA,~\cite{keogh00}) i ho implementen en una estructura de dades que anomenen iSAX (\emph{indexable Symbolic Aggregate approXimation}),~\cite{isax}. Amb aquesta eina s'obtenen representacions de sèries temporals que permeten reduir l'espai emmagatzemat i indexar tant bé com altres mètodes de representació més complexos.




En resum, aquests SGBD per sèries temporals bàsicament resolen els problemes d'anàlisis de sèries temporals.
Però cap d'aquestes sol atendre la relació entre la base de dades i el sistema de monitoratge, és a dir la manera com s'adquireixen les dades. En aquest pas intermig hi ha un sèrie de problemes, com per exemple forats, dades falses, irregularitat en els temps de mostreig, que cal gestionar correctament. Concretament un dels problemes que no s'atén és el de mostreig irregular ja que es considera que les mostres estan a intervals regulars (equi-espaiades) encara que els sistemes de monitoratge informàtics sovint no són capaços de complir-ho amb exactitud sinó que presenten una certa variació en els temps de mesura. 

Així doncs, quan es prenen mesures d'un sistema productiu, aquests problemes apareixen i són de difícil solució.
Les bases de dades RRDtool tenen en compte aquests problemes intermitjos entre el sistema de monitoratge i el sistema d'emmagatzematge i tractament. 






\subsection{Base de dades RRDtool}

En aquest apartat es presenta el TSDS anomenat RRDtool. Aquest sistema, que serà objecte d'un estudi acurat en els capítols \ref{cap:rrdtool} i~\ref{cap:rrdtool-etapes}, s'ha pres com a referència en aquest treball.

RRDtool és un SGBD per a sèries temporals que despunta en l'àmbit de programari lliure. Hi ha una llista de projectes que utilitzen RRDtool que poden trobar-se indicats a l'apartat \emph{Projects using RRDtool} de~\cite{rrdtool}.
Entre d'altres, s'utilitza en sistemes de monitoratge professionals com per exemple Nagios,~\cite{nagios}, o Icinga,~\cite{icinga}, també populars dins del programari lliure, o en el montior MRTG (The Multi Router Traffic Grapher),~\cite{mrtg}, del mateix creador que RRDtool. Aquests monitors fan un ús complet de les possibilitats de RRDtool i li cedeixen tot el control de l'emmagatzematge de mesures i el posterior tractament i representació gràfica de les dades. 
L'ús de RRDtool permets a aquestes aplicacions centrar-se plenament en la problemàtica de l'adquisició de dades i la gestió d'alarmes.

En l'evolució de RRDtool destaquen dues millores significatives.
La primera, descrita per Oetiker a~\cite{lisa98:oetiker}, va consistir en independitzar la base de dades RRDtool del sistema de monitoratge MRTG i dissenyar-la amb l'estructura Round Robin que la caracteritza. La segona, feta per Brutlag,~\cite{lisa00:brutlag}, ha aportat la possibilitat de fer prediccions i detecció de comportaments aberrants basant-se en algoritmes de predicció exponencials i de Holt-Winters. 


L'evolució actual de RRDtool se centra en aspectes informàtics i consisteix a millorar la rapidesa i eficiència en el processament de les sèries temporals. És el cas de Plonka i Carder que a~\cite{carder:rrdcached,lisa07:plonka} dissenyen l'aplicació \verb+rrdcached+ per incrementar el rendiment de RRDtool, la qual demostren en un sistema de monitoratge amb moltes bases de dades funcionant simultàniament.  També \verb+JRobin+,~\cite{jrobin}, que és una implementació en Java de RRDtool que millora els accessos de lectura i escriptura a la base de dades i té una eina de gràfics més perfeccionada.
És significatiu l'ús incipient d'aquest sistema en experimentació. Zhang,~\cite{zhang07}, i Chilingaryan,~\cite{chilingaryan10}, per exemple, usen RRDtool per emmagatzemar de dades experimentals i posteriorment fer predicció o validació.
  

En l'àmbit dels SGBD els sistemes relacionals van fixar una fita que ha tingut una transcendència posterior de  primer ordre. En bona part aquest èxit dels SGBD relacionals es deu al fet que es basen en un model matemàtic sòlid,~\cite{date}.
En el cas de RRDtool no existeix  cap model que descrigui el sistema i es objectiu d'aquest treball proposar-ne un. El model per a SGBD Round Robin es dissenya  al capítol~\ref{cap:model-rrd}.






%%% Local Variables: 
%%% mode: latex
%%% TeX-master: "article"
%%% ispell-local-dictionary: "british"
%%% End: 




\part{Models}
%-- MODELS --

\section{Model}

\acro{MTSMS}
\acro{MTSM}


La definició del model s'estructura en dues parts:

\begin{itemize}
\item Un model pels (SGST)  que defineix mesura i sèrie temporals.
\item Un model pels (SGSTM) que defineix buffer, disc i subsèrie
  resolució, el qual treballa sobre el model de SGST.
\end{itemize}

\todo{sobre tres nivells}
A l'estat de l'art s'ha d'haver explicat els tres nivell de model de dades segons Date i deixar clar aquí que nosaltres definim un model pel segon nivell: nivell de model lògic. Els models lògics modelen les dades, en canvi els models conceptuals modelen la realitat, Fabian Pascal posa d'exemple conceptual el model E/RM.


Objectius:

En el model de SGST s'observen algunes patologies que poden presentar les sèries temporals. El model de SGSTM soluciona algunes d'aquestes patologies:

\begin{itemize}
\item Regularitza les sèries temporals
\item Tracta i validar les sèries temporals: gestiona els casos de dades errònies o desconegudes i marca quan hi ha valors erronis.
\item És una solució de compressió per a quantitats enormes de dades
\end{itemize}


Però el model de SGSTM també es pot fer servir per altres aplicacions:

* Regularitzar en línia (temps real) una sèrie temporal en diferents períodes de mostreig

* Tenir unes vistes (consultes) a punt (ja processades) amb diferents resolucions d'una sèrie temporal

* Comprimir per decimació (downsampling) o bé farcir forats (reconstrucció del senyal)




\section{Time series preliminaries}
\label{sec:model:preliminaries}

% In this section we introduce some background concepts and the
% nomenclature which we will use later.  First we define the main
% objects of a \acro{MTSMS} which are measures and time series.

% A \emph{measure} is a value measured in a time instant. More formally
% it is a tuple $(v,t)$ where $v$ is the value of the measure and $t \in
% \mathbb{R}$ is the time instant of measurement.  The values of a time
% series can be of any type. For simplicity examples are presented with
% integers or real numbers but can also be strings or vectors.  Let $m =
% (v,t)$ be a measure, $v$ is written as $V(m)$ and $t$ is written as
% $T(m)$.

% The time value defines the canonical order between measures.  Let $m =
% (v_m, t_m)$ and $n = (v_n, t_n)$ be two measures, then $m\geq n$ if
% and only if $t_m\geq t_n$.

% A \emph{time series} is sequence of measures of the same phenomena
% that are ordered in time.
% \begin{definition}[Time series]
%   A \emph{time series} $S$ is a a set of measures of the same
%   phenomena $S = \{m_0, \ldots, m_k\}$ without repeated time values
%   $\forall i,j: i\leq k, j\leq k, i\neq j : T(m_i)\neq T(m_j)$. Given
%   a time series $|S|$, we note its size by $|S|=k+1$. Observe that,
%   because measures in $S$ are of the same phenomena, the type of $S$
%   values is homogeneous.
% \end{definition}

% The order defined by measures implies a total order in a time
% series. As a time series is a finite set, if it is not empty it has a
% maximum and a minimum.  Let $S=\{m_0,\ldots,m_k\}$ be a time series
% and $n\in S$ be a measure. The time series' maximum is $n=\max(S)$ if
% and only if $\forall m \in S: n \geq m $.  Similarly, the time series'
% minimum is $n=\min(S)$ if and only if $\forall m \in S: n \leq m$.

% Given the order defined by time, in a time series we define the
% sequence interval following \cite{last:keogh,last:hetland}.  Let
% $S=\{m_0, \ldots, m_k\}$ be a time series. We define the subset
% $S(r,t] \subseteq S$ as the time series $S(r,t]=\{m\in S | r<T(m)\leq
% t\}$, where $r$ and $t$ are two instants in time.  We also define the
% subset $S(r,+\infty)\subseteq S$ as the time series $S(r,+\infty) =
% \{m\in S | r< T(m) \leq T(\max(S))\}$ and the subset
% $S(-\infty,t)\subseteq S$ as the time series $S(-\infty,t) = \{m\in S
% | T(\min(S))\leq T(m) < t\}$.

% The time order in time series also implies the sequence concept of
% next and previous measure.  Let $S=\{m_0, \ldots, m_k\}$ be a time
% series and $l\in S$ and $n$ be two measures. We define the next
% measure of $n$ in $S$ as $l=\nex_S(n)$ where $l =
% \min(S(T(n),+\infty))$. We define the previous measure of $n$ in $S$
% as $l=\prev_S(n)$ where $l = \max(S(-\infty,T(n)))$.

% Let $S$ be a time series, $t$ be a time instant and $\delta$ be a
% time duration, then the time series' measures can be located in the
% time interval $i_0=[t, t+\delta]$ and its multiples $i_j=[t+j\delta,
% t+(j+1)\delta]$ for $j=0,1,2,\ldots$. When time series' measures are
% equally spaced we say it to be regular.
% \begin{definition}[Regular time series]
%   Let $S=\{m_0,$ $ldots,$ $m_k\}$ be a time series and $\delta$ a time
%   duration. $S$ is regular if and only if $\forall m \in
%   S(T(\min(S),+\infty):T(m) - T(\prev_S(m)) = \delta$.
% \end{definition}









\section{Multiresolution model}
\label{sec:MTSMS}

The \acro{MTSMS} are \acro{TSMS} that store time series with a lossy
compression approach, that is some information is selected and spread in
different time resolutions. The \acro{MTSMS} model is based on the
concepts of measures and time series as defined in
Section~\ref{sec:model:preliminaries}.


The multiresolution concept comes from thoroughly analysis of the
RRDtool \cite{rrdtool} \acro{TSMS}. Our objective is to formalise its
essential parts into an abstract model, where what we call
multiresolution plays a main role, and to include more genericity in
order to describe \acro{MTSMS} as fully \acro{TSMS}. Then we will be
able to apply these systems to other applications.



A \acro{MTSMS} stores multiresolution time series where each has a
multiresolution schema as shown in Figure~\ref{fig:model:mtsdb}. A
multiresolution time series is a collection of resolution subseries
which temporarily accumulate measures in a buffer in order to select
some information and finally store it in a disc. The information
selection process changes the time intervals between measures to
compact information by aggregating the time series attributes. 

\begin{figure}[tp]
  \centering
  \input{imatges/mtsms-arquitectura_interna.tex}
  \smallskip
  \caption{Architecture of \acro{MTSMS} model}
  \label{fig:model:mtsdb}
\end{figure}


In this way, the original time series gets stored spread in the discs,
each with a different time resolution and attribute aggregation.
Discs are size bounded so they only contain a fixed amount of
measures. When a disc becomes full it discards a measure. Thus,
multiresolution database is bounded in size and the time series gets
stored in pieces, that is time subseries.

Regarding to operations, \acro{MTSMS} structure needs operators to
change the time intervals between measures and to select
attributes. Mainly, these operators are measure additions and time
series consolidations which some functionality is delegated to operators called 
attribute aggregate functions.
 Most of these operators
are attribute aggregate functions and consolidation actions.
the operations to
create a multiresolution database, to add measures, and to consolidate
time series.
 Attribute aggregate functions are required but not linked
to the model.\todo{}

Following we define the \acro{MTSMS} model by: (i) four basic
structure model elements ---buffer, disc, resolution subserie, and
multiresolution time series--- with its structure operators, (ii) the
operations to change and consult a multiresolution schema, and (iii)
the attribute aggregate functions.



\subsection{Structure}



\subsection{operations}




\section{The proposed data model}



Regarding to operations, \acro{MTSMS} structure needs operators to
change the time intervals between measures. Most of these operators
are attribute aggregate functions and consolidation actions.

In what follows we describe the basic \acro{MTSMS} model centered in:
(i) the four basic data model elements ---buffer, disc, resolution
disc, and multiresolution database---, and (ii) the operations to
create a multiresolution database, to add measures, and to consolidate
time series. Attribute aggregate functions are required but not linked
to the model. They are defined in the
Section~\ref{sec:model:interpolador}.

A \emph{buffer} is a container for a regular or a no-regular time
series. The buffer objective is to regularise the time series using a
predetermined step and an attribute function. We name
\emph{consolidation} to this action.
\begin{definition}[Buffer]
  A \emph{buffer} is defined as the tuple $(S,\tau,\delta,f)$ where
  $S$ is a time series, $\tau$ is the last consolidation time,
  $\delta$ is the duration of the consolidation step and $f$ is an
  attribute aggregate function.

  An empty buffer $B_{\emptyset} = (\emptyset,t_0, \delta, f)$ has an
  empty time series, an initial consolidation time $t_0$ and
  predetermined $\delta$ and $f$. From the $B_{\emptyset}$ all the
  consolidation time instants can be calculated as $t_0+i\delta,
  i\in\mathbb{N}$.
\end{definition}

Operator \emph{addBuffer} adds a measure to its time series:
$\text{addBuffer}: B = (S,\tau,\delta,f) \times m \mapsto
(S',\tau,\delta,f)$ where $S' = S \cup \{m\} $.

A buffer is ready to consolidate when the time of some measure is
bigger than the buffer's next consolidation time.  Let
$B=(S,\tau,\delta,f)$ be a buffer and $m=\max(S)$ the maximum measure,
$B$ is ready to consolidate if and only if $T(m) \geq \tau+\delta$.
The consolidation of $B$ in the time interval $i=[\tau,\tau+\delta]$
results in a measure $m'=(v,\tau+\delta)$ where $m'=f(S,i)$ and $f$ is
an attribute aggregate function $f$. Operator \emph{consolidateBuffer}
consolidates a set of measures and removes the consolidated part of
the time series from the buffer. Usually consolidateBuffer is only
applied to the present consolidation interval and it is defined as
follows: $\text{consolidateBuffer}: B=(S,\tau,\delta,f) \mapsto B'
\times m' $ where $ B'= (S',\tau+\delta,\delta,f)$, $S' = S$ and $m' =
f(S,[\tau,\tau+\delta])$. When historic data is not needed anymore the
consolidated buffer measures can be removed applying $S' =
S(\tau+\delta,\infty)$.

A \emph{disc} is a finite capacity measures container. A time series
stored in a disc has its cardinal bounded. When the cardinal of the
time series is to overcome the limit, some measures need to be
discarded.
\begin{definition}[Disc]
  A \emph{disc} is a tuple $(S,k)$ where $S$ is a time series and
  $k\in\mathbb{N}$ is the maximum allowed cardinal of $S$.  An empty
  disc $D_{\emptyset} = (\emptyset,k)$ has an empty time series and
  the $k$ maximum cardinal allowed.
\end{definition}

The cardinal of the times series is kept under control by the add
operator, $\text{addDisc}:D=(S,k)\times m\mapsto (S',k)$ where 
$$
S' = \begin{cases}
  S\cup\{m\}                 & \text{if } |S|<k  \\
  (S-\{\min(S)\}) \cup \{m\} & \text{otherwise}
\end{cases}  
$$

A \emph{resolution disc} is a disc which stores a regular time
series. It is composed of a buffer, that contains the partial time
series to be regularised, and a disc, that contains the regularised
time series.
\begin{definition}[Resolution disc]
  A \emph{resolution disc} is a tuple $(B,D)$ where $B$ is a buffer
  and $D$ is a disc.  An empty buffer and empty disc imply an empty
  resolution disc $R_{\emptyset} = (B_{\emptyset},D_{\emptyset})$.
\end{definition}
 
The operators of a resolution disc extend the buffer and disc ones:
(i) The addition of a measure to the buffer of the resolution disc,
$\text{addRD}:R=(B,D) \times m \mapsto R'$ where $R'= (B',D)$, and
$B'= \text{addBuffer}(B,m)$; (ii) The consolidation of the resolution
disc by consolidating its buffer and adding the consolidation measure
to its disc, $\text{consolidateRD}:R=(B,D) \mapsto R'$ where $R'=
(B',D')$ and $(B',m') = \text{consolidateBuffer}(B)$ and $D'=
\text{addDisc}(B,m')$.
% \]

A \emph{multiresolution database} is a set of resolution discs which
share the input of measures, that is they store the same time
series. A time series is stored regularised and distributed with
different resolutions in the various resolution discs, as it was shown
in the Figure~\ref{fig:model:mtsdb}.
\begin{definition}[Multiresolution Database]
  A \emph{Multi\-re\-solution Database} is a set of resolution discs
  $M=\{R_0, \dots, R_d\}$.  An empty multiresolution database has
  empty resolution discs $M_{\emptyset}=\{R_{0_\emptyset}, \dots,
  R_{d_\emptyset}\}$.
\end{definition}

We define the addition of a measure to every resolution disc as
$\text{addMD} : M=\{R_0, \dots, R_d\} \times m \mapsto \{R'_0, \dots,
R'_d\}$ where $R'_i=\text{addRD}(R_i,m)$.

The consolidation of all resolution discs can be defined as follows:
$\text{consolidateMD}: M=\{R_0, \dots, R_d\} \mapsto \{R'_0, \dots,
R'_d\}$ where
$$ 
R'_i = \begin{cases}
  \text{consolidateRD}(R_i) & \text{if } R_i \text{ ready to consolidate} \\
  R_i                       & \text{otherwise}
\end{cases}
$$.


\subsection{Attribute aggregate function}
\label{sec:model:interpolador}

When a buffer is consolidated we summarise the time series information
using an attribute aggregate function.  Let $S$ be a time series and
$t_0$ and $t_f$ two time instants, an attribute aggregate function $f$
calculates a measure that summarises the measures of $S$ included in
the time interval $i=[T_0,T_f]$:
\begin{align*}
f&:S=\{m_0,\ldots,m_k\} \times [T_0,T_f] \mapsto m'
\end{align*}

To summarise a time series we can use different attribute aggregate
functions.  For instance, we can calculate an statistic indicator of
the time series such as the average or we can apply a more complex
digital signal processing operation, \cite{zhang11}.

Below there are some examples. Let $S'=S(T_0,T_f]$. Then:
\begin{itemize}
\renewcommand{\labelitemi}{--}
\item maximum$^d$: $S \times i \mapsto m'$ where $V(m') =
  \max_{\forall m \in S'}(V(m))$. It summarises $S'$ with the maximum
  of the measure values.
\item last$^d$: $S \times i \mapsto m'$ where $V(m') = \max(S')$. It
  summarises $S'$ with the maximum measure.
\item arithmetic mean$^d$: $S \times i \mapsto m'$ where $V(m') =
  \frac{1}{|S'|} \sum\limits_{\forall m\in S'} V(m)$. It
  summarises $S'$ with the mean of the measure values.
\end{itemize}

% With reference to data validation, attribute aggregate functions
% can cope with this process. When data has not been captured or has
% been captured erroneously, it must be treated as unknown data.
% \begin{itemize}
% \item When data has not been captured it is unknown by nature. For
%   example, we try to capture data from a sensor and there is no
%   response.
% \item When data is erroneously it must be marked as unknown. For
%   example, we capture data from a sensor but it responses in a not
%   reasonable time or we capture data that is clearly outside a
%   reasonable limits.
% \end{itemize}
% As a consequence, attribute aggregate functions deals with these two
% subprocesses: treating unknown data and marking data as
% unknown. Following with real numbers example, we extend the
% domain with a value that means 'unknown', let this unknown value be
% represented by the improper element infinity ($\infty$).

% An attribute aggregate functions treating unknown
% data is a one that can calculate a result when there are unknown
% values in the original time series, $f^u: S \times i \mapsto m'$ where
% $\exists m \in S: V(m)=\infty$. Although from a strict point of view
% operating with unknown data makes unknown result, aggregate functions
% are free to calculate whatever is needed such as time series analysis
% does with data reconstruction.

% For example, arithmetic mean$^{d}$ aggregate function returns
% $V(m')=\infty$ if $\exists m \in S: V(m)=\infty$.  We can define a new
% mean function, based on the original arithmetic mean$^{d}$ aggregate,
% that naively treats unknown values by keeping the
% known mean; in other words, it ignores unknown values found in the time
% interval: arithmetic mean$^{du}$: $S \times i \mapsto m'$ where $m' =
% \text{arithmetic mean}^{d}(S'',i)$ and $S''= \{m''\in S':V(m'')\neq
% \infty\}$.
% % ignore$^{u}$: $S \mapsto S'$ where $S'= \{m''\in S':V(m'')\neq
% % \infty\}$,
% % arithmetic mean$^{du}$: $S \times i \mapsto m'$ where $m' =
% % \text{arithmetic mean}^{d}(\text{ignore}^u(S),i)$.

% An attribute aggregate functions marking data as unknown is a one
% that can give unknown value as the resulting measure's value, $f^{mu}:
% S \times i \mapsto m'$ where $V(m')\in \mathbb{R}\cup\{\infty\}$.

% For example, we can define a maximum aggregate, based on the
% maximum$^d$ aggregate, that returns unknown if there is a
% measure's value bigger than 2:  maximum$^{dmu2}$: $S \times i
% \mapsto m'$ where $V(m') = 
% \begin{cases}
%   \infty &\text{if }  m''>2\\
%   m'' & \text{else }
% \end{cases}$ and $m''=\text{maximum}^d(S,i)$.

% %Per exemple definim un termini, si les dades estan més espaiades que 2 es marca com a desconeguda

In the design of the attribute aggregate function we can interpret a
time series in different ways, that is what we call the representation
of a time series. \citeauthor{last:keogh}, \cite{last:keogh}, cite
some possible representations for time series such as Fourier
transforms, wavelets, symbolic mappings or piecewise linear
representation. The last one is very usual due to its simplicity,
\cite{keogh01}.

Time series representations can be taken into account when computing
with the measures of the time series.  For example, a maximum
attribute aggregate function may give different values if we consider
a linear or a constant piecewise representation.

Following we show a possible family of attribute aggregate functions
for time series represented by a staircase function, that is with a
piecewise constant representation.  We define a new representation for
time series named \emph{zero-order hold backwards} (zohe). This
representation holds back each value until the preceding value. 
RRDtool, \cite{lisa98:oetiker}, has a similar aggregate function.

Let $S=\{m_0,\ldots,m_k\}$ be a time series, we define
$S(t)^{\text{zohe}}$ as its continuous representation along time $t$:
$\forall t \in \mathbb{R} ,\forall m \in S:$
\begin{equation}
 S(t)^{\text{zohe}} =  
\begin{cases}
  \infty & \text{if } t > T(\max S) \\
  V(m)   & \text{if } t\in (T(\prev_S m),T(m)]
\end{cases}
\label{eq:zohe}
\end{equation}


In conclusion, we can define many attribute aggregate functions and
thus no global assumptions can be made about them. Each user has to
decide which combination of aggregation and representation fits better
with the measured phenomena.  Therefore, \acro{MTSMS} must allow to
define user aggregate functions.







%%% Local Variables:
%%% TeX-master: "main"
%%% ispell-local-dictionary: "british"
%%% End:

% LocalWords:  genericity



\chapter{Model SGST}

En aquest capítol es defineixen els objectes que ens permeten modelar l'estructura de les dades.

Dos models estructurals:

* Hi ha un model pels SGST (TSMS) que inclou mesura i sèries temporals.

* Hi ha un model pels SGSTM (MTSMS) que té buffer, discs i mtsdb, els quals inclouen el model de sèrie temporal del SGST.




\section{Model estructural de dades}

Una sèrie temporal és una relació de temps i valors. A cada parella
temps-valor l'anomenem mesura. Així doncs, una sèrie temporal és un
conjunt de mesures. Una mesura és un tuple temps-valor.



Una mesura és un valor mesurat en un instant de temps i una sèrie
temporal és un co\l.lecció de mesures.





\subsection{Temps}

Utilitzem el temps com un valor que ens permet ordenar les mesures.  A
tal efecte, el domini del temps es defineix com un conjunt tancat
(compactificat) i amb ordre total. No obstant, pot ser tant un conjunt
finit com infinit.

Per facilitar la comprensió, en el document utilitzarem el conjunt de
reals com a conjunt pels temps. Concretament, per a complir que sigui
un conjunt tancat usarem el conjunt estès de nombres reals
$\bar{\mathbb{R}} \in \mathbb{R} \cup
\{+\infty,-\infty\}$, \parencite{wiki:extendedreal,cantrell:extendedreal},
també anomenat recta real acabada.


El conjunt estès de nombres reals té dos punts límits corresponents al
valor impropi infinit, aleshores en notació d'interval el conjunt es
pot escriure com $\bar{\mathbb{R}} \in [-\infty,+\infty]$.  Més
endavant a la definició~\ref{def:model:mesura_indefinida} es detallen
algunes propietats induïdes a les mesures com a resultat d'aquesta
extensió.

Les relacions d'ordre i algunes operacions aritmètiques s'estenen al
conjunt $\bar{\mathbb{R}}$, \cite{cantrell:extendedreal}.  Algunes
expressions esdevenen indefinides (p.ex.\ $0/0$) i altres depenen del
context, com és el cas de l'expressió indeterminada $0 \times \infty$ que
per exemple en la teoria de la mesura habitualment es defineix com $0 \times
\infty = 0$, \cite{wiki:extendedreal}.


El conjunt dels reals és un espai mètric ja que té definida una funció
distància (o mètrica), com per exemple la distància euclidiana. Com a
conseqüència, ens permet distingir entre instants de temps (els
elements del conjunt) i durades (la mètrica). Observant els instants
de temps com a punts en la recta real i les durades com a segments de
la recta real, es pot definir el temps com a sistema de coordenades
especificant un instant com a marc de
referència, \parencite{iep:time-supplement,wiki:coordinate}.


\begin{definition}[Temps]
  \label{def:model:temps}
  Siguin $t^i_i$ i $t^i_j$ dos instants de temps, observem la quantitat
  de temps o la durada $t^d$ com un valor $t^d \in\bar{\mathbb{R}}$
  que mesura la distància en unitats de temps entre dos temps
  absoluts $t^d = t^i_i - t^i_j$.
  
  Sigui $t^d$ una durada i $t^{R}$ un temps absolut de referència,
  observem un instant de temps $t^i$ com un valor $t^i
  \in\bar{\mathbb{R}}$ que mesura la quantitat de temps respecte al
  temps de referència $t^i= t^{R} + t^d$ . Aquest valor de referència
  $t^{R}\in\mathbb{R}$ és també un instant de temps però que permet
  definir unívocament la posició de qualsevol altre instant de temps.


\end{definition}

En resum, els instants de temps es poden veure com una seqüència de
valors reals que indiquen esdeveniments amb ordre clarament definit i
entre dos instants de temps sempre hi ha una durada. Tant els instants
de temps com les durades s'expressen amb un real que té unitats de
temps. Aquestes unitats són 'segons' en sistema internacional.



\subsubsection{Calendari}
\textcite{dreyer94} situen els calendaris i les seves operacions com a
essencials en els SGST. Tanmateix, pot no ser necessari modelar les
dates de calendari en el model de temps. El temps és la línia contínua
de temps, el calendari són nom especials a certs punts de la línia de
temps. Només cal una eina que sigui capaç de convertir de noms a
instants de temps.

Per una banda, no afecta al model SGST que els calendaris siguin més o
menys complicats, en aquest cas només es veuen complicades les
funcions de conversió de temps a calendari i viceversa.  Per altra
banda, tampoc afecta que els calendaris siguin ambigus (p.ex.\ dos
noms per al mateix instant o instants sense nom) o que continguin
propietats impredictibles (p.ex.\ cas dels segons addicionals
(intercalats) en UTC) ja que la responsabilitat d'aquests problemes
correspon a la bona definició dels sistemes de calendari.


Unix time (posix) no incorpora els leap seconds.
Millor TAI, unix time (right), ja que és una mesura totalment lineal del temps. 
Unix time, UTC i TAI: http://lwn.net/Articles/504744/


\subsection{Valor}

El \gls{terme:SGBDR:valor} és qualsevol element que és d'un
\gls{terme:SGBDR:tipus}; és a dir, un objecte que
pertany a un determinat conjunt de valors i que té associat les
operacions que s'hi poden aplicar. Exemples de tipus de dades són els
enters, els reals, les cadenes de text i les estructures de dades com
vectors, llistes o \glspl{terme:SGBDR:relacio}.  \todo{vigilar amb
  date, ell en diu escalars i no escalars (amb components visibles) i
  per exemple considera que un punt és escalar}

El model de dades dels valors ha d'incloure una dada que defineixi el
valor indefinit. Més endavant a la
definició~\ref{def:model:mesura_valor_indefinit} es detallen les
propietats de les mesures amb valor indefinit. Seguint l'exemple amb
els reals, el valor indefinit es defineix amb el valor impropi infinit
del conjunt dels reals estès
projectivament, \parencite{cantrell:projectivelyextendedreal},
$\mathbb{R}^*\in\mathbb{R} \cup \{\infty\}$.  En aquest cas el valor
és un escalar però fàcilment es pot estendre el concepte a valors
multivaluats ${\mathbb{R}^*}^n$ que representin una co\l.lecció de
valors mesurats en el mateix instant de temps, tal i com fa per
exemple \textcite{assfalg08:thesis}.





\subsection{Mesura}\label{sec:model:mesura} 

Una mesura és una parella de temps i valor.

\begin{definition}[Mesura]
  \label{def:model:mesura}
  Definim \emph{mesura} com el tuple $(t,v)$, en el que $v$ és el
  valor de la mesura i $t$ és l'instant de temps en que s'ha pres
  aquesta mesura.
\end{definition}


Donada una mesura $m=(t,v)$ escriurem $V(m)$ per referir-nos a $v$ i
$T(m)$ per referir-nos a $t$.

Donades dues mesures és fàcil establir la relació d'ordre induïda pel
temps.

\begin{definition}[Relació d'ordre]
  \label{def:model:mesura-relacio-ordre}
  Sigui $m=(t_m,v_m)$ i $n=(t_n,v_n)$. Direm que $m\geq n$ si i solament
  si $t_m\geq t_n$.
\end{definition}


En les definicions de temps i valor s'han estès els conjunts amb
valors impropis, concretament s'ha exemplificat amb el conjunt estès
de nombres reals afí $\bar{\mathbb{R}} \in \mathbb{R} \cup
\{+\infty,-\infty\}$ i amb el projectiu $\mathbb{R}^*\in\mathbb{R}
\cup\{\infty\}$,
\parencite{cantrell:extendedreal,cantrell:projectivelyextendedreal}. Aquesta
extensió amb l'element impropi infinit ($\infty$) dóna com a resultat
unes mesures impròpies que anomenarem mesura de valor indefinit i
mesura indefinida.

\begin{definition}[Mesura de valor indefinit]
  \label{def:model:mesura_valor_indefinit}
  Definim \emph{mesura de valor indefinit} com el tuple $(t,v)$, en el
  que el valor és $v=\infty$ i l'instant de temps és
  $t\in\bar{\mathbb{R}}$.
\end{definition}

\begin{definition}[Mesura indefinida]
  \label{def:model:mesura_indefinida}
  Definim \emph{mesura indefinida} com el tuple $(t,v)$, en el que el
  valor és $v\in\mathbb{R}^*$ i l'instant de temps és
  $t\in\{+\infty,-\infty\}$.
\end{definition}

Així doncs, sigui $m$ una mesura, es podrà notar la mesura de valor
indefinit com $m=(t,\infty)$ i les mesures indefinides com
$m=(+\infty,v)$ per la positiva i $m=(-\infty,v)$ per la negativa, les
quals normalment s'anotaran també amb valor indefinit:
$m=(+\infty,\infty)$ i $m=(-\infty,\infty)$ respectivament.


Les mesures de valor indefinit es podran utilitzar en aquells casos en
els que el valor de la mesura és desconegut. Els valors desconeguts
són aquells valors que no existeixen (es desconeixen, \emph{missing
  data} ) o que s'ignoren (es descarten, \emph{censoring} o
\emph{truncation}). Els valors que no existeixen prenen el valor
desconegut en el moment de la mesura, en canvi els valors descartats
són marcats com a desconeguts després d'un processament de les dades.

Nota: en alguns sistemes es distingeix entre valors infinits
($\infty$) i valors indefinits (NaN, \emph{not a number}),
\cite{wiki:ieee754}. Aquest no és el cas de les definicions de mesures
indefinides presents.



\subsection{Sèrie temporal}
\label{sec:model:serietemporal}

Les sèries temporals són seqüències de mesures ordenades en el temps.
Tradicionalment s'anomenen sèries temporals tot i que també s'anomenen
seqüències temporals, per exemple a \cite{last:hetland}.

\begin{definition}[Sèrie temporal]
  \label{def:serie_temporal}
  Una sèrie temporal $S$ és un conjunt de mesures
  $S=\{m_0,\ldots,m_k\}$ sense temps repetits
  $\forall i,j: i\leq k, j\leq k, i\neq j : T(m_i)\neq T(m_j)$.
\end{definition}

Per ser un conjunt, les sèries temporals tenen mesura de cardinalitat.
\begin{definition}[Cardinal]
  Sigui $S=\{m_0,\ldots,m_k\}$ una sèrie temporal, definim el nombre
  de mesures que conté la sèrie temporal com el cardinal del conjunt
  $|S|=k+1$. Una sèrie temporal sense mesures és la sèrie temporal
  buida $S_\emptyset= \emptyset$, és a dir que no té cap element
  $|S_\emptyset|=0$.
\end{definition}


 





\subsection{Relació sèrie temporal}

Una sèrie temporal és una relació de temps i valors. A cada parella temps-valor l'anomenem mesura. Així doncs, una sèrie temporal és un conjunt de mesures.

Una sèrie temporal és un conjunt de mesures, així doncs s'observa com una relació de grau dos (relació binària)  a on la capçalera conté els atributs temps i valor, ambdós amb els dominis de temps i valor ja vistos com per exemple el tipus de dades 'reals estesos'. Inclou algunes restriccions més que les relacions:

* Els temps no poden estar repetits

* Els valors han de contenir el mateix tipus d'objecte.

Els temps no repetits indueixen un ordre temporal a les sèries temporals. Tot i així, les relacions, per ser conjunts, conserven la no definció d'un ordre dels elements. 


En el model relacional no hi ha ordre en els atributs a diferència de les relacions matemàtiques que tenen un ordre d'esquerra a dreta \parencite[sec.\ 5.3]{date:introduction}.

\subsection{Exemples}

\paragraph{Exemple 1}
Sèrie temporal $S_1$ on el temps i els valors pertanyen a $\bar{\mathbb{R}}$. Conté la mesura de valor 1 en el temps 5, la mesura de valor 3 en el temps 7 i la mesura de valor 1 en el temps 10. Modelada com a relació, és a dir com a parella capçalera i conjunt de valors certs, s'escriu com 
$S_1 = ( \{temps: \bar{\mathbb{R}}, valor: \bar{\mathbb{R}}\}, \{ \{temps:5,valor:1\}, \{temps:7,valor:3\}, \{temps:10,valor:1\} \} )$.

Degut al format esquemàtic, simplifiquem l'escriptura de les sèries temporals com a conjunt de tuples $(t,v)$ a on $t$ és el temps i $v$ és el valor. Així doncs la sèrie temporal $S$ es pot escriure de manera simplificada com a 
$S = \{ (5,1), (7,3), (10,1) \}$.

Tal com s'utilitza en les relacions, les sèries temporals es poden visualitzar com a taules. La sèrie temporal $S_1$ es visualitza com a taula a la \autoref{fig:model:serietemporal:real}.

\begin{figure}[tp]
  \centering
  \begin{tabular}{|c|c|}
    \multicolumn{2}{c}{$S_1$} \\ \hline
    $t$  & $v$ \\ \hline
    5  & 1 \\
    7  & 3 \\
    10 & 1 \\ \hline
  \end{tabular}
  \caption{Taula d'una sèrie temporal amb valors reals}
  \label{fig:model:serietemporal:real}
\end{figure}


\paragraph{Exemple 2}
Sèrie temporal $S_2$ on el temps pertany a $\bar{\mathbb{R}}$ i el valor pertany a  $\bar{\mathbb{R}}^3$; és a dir és un vector. Conté el valor (1,2,3) en el temps 5, el valor (3,4,5) en el temps 7 i el valor (1,2,3) en el temps 10.

De manera simplificada s'escriu com 
$S_2 = \{ (5,(1,2,3)), (7,(3,4,5)), (10,(1,2,3)) \}$ i es visualitza com a taula a la \autoref{fig:model:serietemporal:vector}. No obstant, es pot visualitzar de forma més còmode com a $S_2^b = \{ (5,1,2,3), (7,3,4,5), (10,1,2,3) \}$

\begin{figure}[tp]
  \centering
  \begin{tabular}{|c|c|}
    \multicolumn{2}{c}{$S_2$} \\ \hline
    $t$  & $v$ \\ \hline
    5  & (1,2,3) \\
    7  & (3,4,5) \\
    10 & (1,2,3) \\ \hline
  \end{tabular} \qquad
  \begin{tabular}[tp]{|c|c|c|c|}
   \multicolumn{4}{c}{$S_2^b$} \\ \hline
    $t$  & $v_1$ & $v_2$ & $v_3$ \\ \hline
    5  & 1 & 2 & 3 \\
    7  & 3 & 4 & 5 \\
    10 & 1 & 2 & 3 \\ \hline
  \end{tabular}

  \caption{Taula d'una sèrie temporal amb valors vectors}
  \label{fig:model:serietemporal:vector}
\end{figure}


\paragraph{Exemple 3} \emph{Valors relació}. \label{par:model:exemple-relvalues}
Sèrie temporal $S_3$ on el temps pertany a $\bar{\mathbb{R}}$ i el valor és una sèrie temporal del mateix format que en l'exemple 1. Conté les tuples de $S_1$ com a valors en el temps 1 i 2. 

De manera simplificada s'escriu com
$S_3 =  \{ (1,\{ (5,1), (7,3), (10,1) \}), 
(2,\{ (5,1),$ $(7,3),$ $(10,1) \}) \}$
i es visualitza com a taula a la \autoref{fig:model:serietemporal:serietemporal}.


\begin{figure}[tp]
  \centering
  \begin{tabular}{|c|c|}
    \multicolumn{2}{c}{$S_3$} \\ \hline
    $t$  & $v$ \\ \hline
    1 &   
       \begin{tabular}{|c|c|}
         \hline
         $t$  & $v$ \\ \hline
         5  & 1 \\
         7  & 3 \\
         10 & 1 \\ \hline
       \end{tabular} \\ \hline
    2 & 
       \begin{tabular}{|c|c|}
         \hline
         $t$  & $v$ \\ \hline
         5  & 1 \\
         7  & 3 \\
         10 & 1 \\ \hline
       \end{tabular} \\ \hline
  \end{tabular}
  \caption{Taula d'una sèrie temporal amb valors sèrie temporal}
  \label{fig:model:serietemporal:serietemporal}
\end{figure}


S'observa que la capçalera de $S3$ és $\{temps:\bar{\mathbb{R}},valor:
relacio\{temps:\bar{\mathbb{R}},valor:\bar{\mathbb{R}}\}\}$ \parencite[sec.\ 5.3]{date:introduction}. És a dir, el valor és de tipus relació que es defineix amb la capçalera de la relació on el temps i el valor pertanyen a $\bar{\mathbb{R}}$. Per tant, el valor de $S3$ és de tipus sèrie temporal amb valors reals. Cal insistir que \emph{tots} el valors de $S3$ han de pertànyer al mateix domini \parencite[sec.\ 5.4]{date:introduction}, el qual és $relacio\{temps:\bar{\mathbb{R}},valor:\bar{\mathbb{R}}\}$.



\paragraph{Exemple 4} \emph{Variable relació}.\todo{això no es pot fer, perquè no existeix el tipus relvar?? però les tuples poden contenir expressions?? No existeixen les tuplevar (Date rebutja ferotjament els apuntadors a dins dels DBMS) [Date on database :writings 2000-2006 / C.J. Date]}
Sèrie temporal $S_4$ on el temps pertany a $\bar{\mathbb{R}}$ i el valor és una referència a una sèrie temporal. Conté $S_1$ com a valors en el temps 1 i 2. 

De manera simplificada s'escriu com
$S_4 =  \{ (1,S_1) , (2,S_1) \}$ 
i es visualitza com a taula a la \autoref{fig:model:serietemporal:relvar}.

S'aplica el concepte de variable relació (\emph{relvar}) dels SGBDR \parencite[sec.\ 3.3]{date:introduction}.
Així doncs, cal notar que $S_4$  no és el mateix que $S_3$.
\begin{figure}[tp]
  \centering
  \begin{tabular}{|c|c|}
    \multicolumn{2}{c}{$S_4$} \\ \hline
    $t$  & $v$ \\ \hline
    1 & $S_1$ \\
    2 & $S_1$ \\ \hline
  \end{tabular}
  \caption{Taula d'una sèrie temporal amb valors \emph{relvar}}
  \label{fig:model:serietemporal:relvar}
\end{figure}


Relació de noms i sèries temporals $R =  ((nom:string,serie:relacio\{temps:\bar{\mathbb{R}},valor:\bar{\mathbb{R}}),\{ ('S_1',S_1),('S_2',S_2)  \})$

Sèrie temporal amb strings com a valors:
$N= ( (temps:\bar{\mathbb{R}},valor:string) ,\{ (1,'S_1') , (2,'S_1') \})$

Sèrie temporal com a variable relació de vista (relvar view)
$S_4 =  (N RENAME valor as nom) JOIN R$
\todo{cal definir una view}



\subsection{Naturalesa de les sèries temporals}


Perquè RRDtool diferencia entre comptadors i magnituds?

[segev87] diferencia entre step-wise constant, discret (potser aquest tal com se'l defineix són intervals temporals), continu. Ho anomena tipus de la sèrie temporal i diu que es poden definir interpolacions per cada una.


[John G. Proakis, Dimitris G. Manolakis 2007 Tratamiento digital de señales/Digital signal processing 4a ed pp11-12(segons wikipedia)] Acquisition: Discrete signals may have several origins, but can usually be classified into one of two groups:[1]
*By acquiring values of an analog signal at constant or variable rate. This process is called sampling.[2]
*By recording the number of events of a given kind over finite time periods. For example, this could be the number of people taking a certain elevator every day.



\subsubsection{Regularitat de les sèries temporals} 

Sigui $S=\{m_0,\ldots,m_k\}$ una sèrie temporal, $t$ un instant de
temps i $\delta$ una durada de temps, les mesures de la sèrie temporal
es poden localitzar en l'interval de temps $i_0=[t,t+\delta]$ i els
seus múltiples $i_j=[t+j\delta \,,\, t+(j+1)\delta]: j=0,1,2,\ldots$.
En processat de senyal aquests intervals de temps s'anomenen intervals
de mostreig, $\delta$ s'anomena període de mostreig i $t$ s'anomena
temps inicial del mostreig.  La sèrie temporal $S$ és de naturalesa
diferent segons la situació dels temps $T(m_i)$ en els intervals de
temps $i_j$.

Una sèrie temporal és regular quan les mesures són equidistants en el
temps, tal com ho anomenen a \cite{last:hetland}.

\begin{definition}[Sèrie temporal regular]
  Sigui $S=\{m_0,\ldots,m_k\}$ una sèrie temporal, $t$ un instant de
  temps i $\delta$ una durada de temps. Direm que $S$ és regular si i
  només si $\forall m \in S(T(\min(S),\infty):T(m) - T(\ant(m)) =
  \delta$ i $T(\min(S))=t$.
\end{definition}

Si una sèrie temporal és regular, l'anomenem sèrie temporal mostrejada
regularment amb període de mostreig $\delta$. Noteu que si es complís
la definició excepte que s'iniciés en el temps que exigim
$T(\min(S))=t$, aleshores la sèrie temporal seria equidistant però a
efectes de mostreig no la podríem anomenar regular; sí que seria una sèrie temporal de temps real (v.\ def.~\ref{def:st:tempsreal}).


Una sèrie temporal és no regular quan no és regular. 
En les sèries temporals no regulars es poden distingir tres casos: temps real, ultramostreig i inframostreig.

Una sèrie temporal és de temps real quan a cada interval de mostreig hi ha una i només una mesura. L'interval de mostreig pot estar acotat per una durada anomenada termini.

\begin{definition}[Sèrie temporal de temps real]\label{def:st:tempsreal}
  Sigui $S=\{m_0,\dotsc,m_k\}$ una sèrie temporal, $t$ un instant de
  temps, $\delta$ una durada de temps i $D$ una durada que indica
  termini. Direm que $S$ és de temps real si i només si $D\leq\delta$
  i $\forall n\in\{0,\ldots,|S|-1\}: \exists!m \in
  S(t+n\delta,t+n\delta+D]$.  Aleshores la sèrie temporal està
  mostrejada en temps real per al temps de mostreig $\delta$ amb
  compliment del termini $D$.
\end{definition}

Si una sèrie temporal és de temps real, l'anomenem  sèrie temporal mostrejada
en temps real amb període de mostreig $\delta$ i compliment del termini $D$.
Si $D=\delta$, es pot anomenar que $S$ és una sèrie temporal de temps real sense termini.


% \paragraph{Ultramostreig} Una sèrie temporal està ultramostrejada (\emph{upsampling}) quan a cada interval de mostreig hi ha una mesura o més d'una. 
% \[
% \text{Ultramostrejada?}: \text{Sèrie temporal} \times T_0 \times \delta \longrightarrow \text{Booleà}
% \]

% Una sèrie temporal $S$ està ultramostrejada ssi $S$ no és de temps real i $\exists m_i=(v_i,t_i)\in S:T_0+(n-1)\delta \leq t_i < T_0+n\delta:\forall n\in\{1,\ldots,|S|\}$.

% \paragraph{Inframostreig} Una sèrie temporal està inframostrejada (\emph{downsampling}) quan en algun interval de mostreig no hi ha cap mesura. 
% \[
% \text{Inframostrejada?}: \text{Sèrie temporal} \times T_0 \times \delta \longrightarrow \text{Booleà}
% \]

% Una sèrie temporal $S$ està inframostrejada ssi $\nexists m_i=(v_i,t_i)\in S:T_0+(n-1)\delta \leq t_i < T_0+n\delta:\forall n\in\{1,\ldots,|S|\}$.








\subsubsection{Representació de les sèries temporals}



La naturalesa indueix representacions?
Jo puc utilitzar qualsevol representació donada una sèrie temporal, però això em pot causa perjudici si no s'adiu amb la naturalesa.


La representació serveix per interpolar:

zoh, zoh cap enrere, lineal, etc.


Una sèrie temporal és la representació discreta d'una funció contínua. A partir de la sèrie temporal es pot definir una funció contínua. 

A teoria de senyal s'estudia com fer que aquesta s'aproximi a la real. Estudiant com a senyal fan: donada una sèrie temporal dir quina funció s'hi 'ajusta' més. 

Però jo puc preguntar donada una sèrie temporal quina funció representa i puc dir per representar a zohe és tal, per representar a lineal és qual. 

Potser millor dir-li interpretació?



\paragraph{Representació de sèries temporals}

\textcite{last:keogh}, cita vàries representacions per les sèries temporals com per exemple \emph{Fourier Transforms}, \emph{Wavelets}, \emph{Symbolic Mappings} o \emph{Piecewise Linear Representation} (PLR), però assenyala aquesta última com la representació més utilitzada. 
La PLR, funció definida a trossos lineal, és l'aproximació d'una sèrie temporal $S$, de llargada $n$, amb $K$ segments rectes. Els segments podrien ser polinomis de qualsevol grau, però la manera més comuna de representar sèries temporals és amb funcions lineals, segons Keogh, \cite{keogh02}.
Per aproximar el segment $S(t_a:t_b]$ d'una sèrie $S$, Keogh defineix dues tècniques: interpolació lineal, la recta que connecta $t_a$ i $t_b$, i regressió lineal, la millor recta que aproxima per mínims quadrats el segment entre $t_a$ i $t_b$.

Però també es pot representar una sèrie temporal amb una funció esglaó (\emph{step} o \emph{staircase function}); és a dir, amb una funció definida a trossos constant (\emph{piecewise constant representation}).
La representació a trossos constant és utilitzada en electrònica als convertidors digital-analògic (DAC, \emph{digital-to-analog converter}). En aquest cas, un senyal discret es considera una sèrie temporal i per reconstruir el senyal continu típicament s'aplica el model de \emph{zero-order hold}, equivalent a la representació a trossos constant,  o el de \emph{first-order hold},  equivalent a la representació a trossos lineal.
El model de \emph{zero-order hold} consisteix en mantenir constant cada valor fins al proper. S'obté una representació a trossos constant que en electrònica s'anomena seqüència de pulsos rectangulars (\emph{rectangular pulses}).

%http://en.wikipedia.org/wiki/Piecewise

%http://ca.wikipedia.org/wiki/Funció_definida_a_trossos

%http://en.wikipedia.org/wiki/Rectangular_function

%http://en.wikipedia.org/wiki/Step_function

% Piecewise Aggregate Approximation (PAA) \cite{keogh00}: aproxima una sèrie temporal partint-la en segments de la mateixa mida i emmagatzemant la mitjana dels punts que cauen dins del segment. Redueix de dimensió $n$ a dimensió $N$

% Adaptive Piecewise Constant Approximation (APCA) \cite{keogh01}: com el PAA però amb segments de mida variable.

A continuació,  la representació  d'una sèrie temporal segons el model de \emph{zero-order hold} s'estén per diferents continuïtats en els intervals de temps de representació.

Sigui $S$ una sèrie temporal, es defineix $S(t)$ com la representació
de la sèrie temporal contínuament al llarg del temps $t$.  En primer
lloc, es representa amb \emph{zero-order hold} a partir de funcions
graó contínues per la dreta (\emph{right-continuous}).

\begin{definition}[Representació amb \emph{zero-order hold}]
Sigui $S=\{m_0,\ldots,m_k\}$ una sèrie temporal, la representació  $S(t)$ amb \emph{zero-order hold} es defineix
\[
\forall t \in \mathbb{R} ,\forall m \in S: S(t) =
\begin{cases}
  V(\min S) & \text{si } t < T(\min S) \\
  V(m) & \text{si }  t\in [T(m),T(\seg m))
\end{cases}
\]
\end{definition}

En segon lloc, es representa $S(t)$ amb \emph{zero-order hold} centrada en
l'interval, definit també a partir de funcions graó contínues per la
dreta.

\begin{definition}[Representació amb \emph{zero-order hold} centrada en l'interval]
  Sigui $S=\{m_0,\ldots,m_k\}$ una sèrie temporal, la representació
  $S(t)$ amb \emph{zero-order hold} centrada en l'interval es defineix
\[
\forall t \in \mathbb{R}  ,\forall m \in S:
S(t) =  
\begin{cases}
  V(m) & \text{si } t = \frac{T(\ant m)+T(m)}{2} \\
  V(m) & \text{si } t\in \left( \frac{T(\ant m)+T(m)}{2},\frac{T(m)+T(\seg m)}{2} \right) \
\end{cases}
\]
\end{definition}

En tercer lloc, es representa $S(t)$ amb \emph{zero-order hold} cap enrere, ara definit a partir de funcions graó contínues per l'esquerra.
\begin{definition}[Representació en \emph{zero-order hold} cap enrere]
  Sigui $S=\{m_0,\ldots,m_k\}$ una sèrie temporal, la representació
  $S(t)$ amb \emph{zero-order hold} cap enrere es defineix
\[
\forall t \in \mathbb{R}  ,\forall m \in S:
S(t) =  
\begin{cases}
  V(\max S) & \text{si } t > T(\max S) \\
  V(m) & \text{si }  t\in (T(\ant m),T(m)]
\end{cases}
\]
\end{definition}

Sigui $S$ una sèrie temporal regular i $\delta$ una durada de temps, aleshores la representació de $S(t)$ amb \emph{zero-order hold} és la mateixa que la de $S(t-\delta)$ amb \emph{zero-order hold} cap enrere i és la mateixa que la de $S(t-\frac{\delta}{2})$ centrada en l'interval. 














%%% Local Variables:
%%% TeX-master: "main"
%%% End:







% LocalWords:  SGST

\section{Model d'operacions}

Una sèrie temporal té un atribut de temps que ha de ser tingut en
compte pels operadors que la manipulin.  Així, atenent a aquest
atribut de temps, el comportament d'una sèrie temporal pot tenir
naturaleses diferents:
\begin{itemize}
\item Conjunt, és a dir els operadors només atenen a la forma
  estructural bàsica.
\item Seqüència, en la qual els operadors la tracten com a conjunts
  amb ordre.
\item Funció temporal, en la qual els operadors treballen tenint en
  compte que una sèrie temporal és la representació d'un funció
  contínua.
\end{itemize}


En el disseny del model d'operacions següent es distingeix el
comportament per als tres casos anteriors.  Es dissenyen les
operacions bàsiques que permeten que posteriorment es combinin per
elaborar-ne de més complexes.




\subsection{Bàsiques de conjunts}

En el model estructural de SGST hem definit les sèries temporals
utilitzant conjunts. En aquest apartat definim operadors per a les
sèries temporals recollint els operadors habituals que tenen els
conjunts.

El model relacional de SGBD defineix els seus operadors bàsics a
partir de l'àlgebra de
conjunts \parencite[cap.~6]{date:introduction}. En aquest apartat
apliquem el mateix estudi per al model de SGST. Tot i així de manera
simplificada, a les definicions no es descriuen les sèries temporals
com a relacions amb capçaleres sinó que se n'escriuen només els
conjunts de valors. Seguint el model de \citeauthor{date:introduction}
es poden estendre les definicions i introduir el model complet de
relacions.



\subsubsection{Pertinença i inclusió}

La pertinença determina si un element pertany a un conjunt.  Sigui
$S=\{m_0,\ldots,m_k\}$ una sèrie temporal i $m$ una mesura, es
defineix de la mateixa manera que en els conjunts la pertinença de la
mesura a la sèrie temporal $m \in S$. Atenent a l'atribut de temps, es
defineix la pertinença temporal d'una mesura a una sèrie temporal.
\begin{definition}[Pertinença temporal]
  Sigui $S=\{m_0,\ldots,m_k\}$ una sèrie temporal i $m=(t,v)$ una
  mesura.  Direm que la mesura pertany temporalment a la sèrie
  temporal $m \inst S$ si i només si $\exists m_a \in S : T(m) =
  T(m_a)$.
\end{definition}


La inclusió determina si tots els elements d'un conjunt pertanyen a un
altre conjunt. Siguin $S_1=\{m_0^1,\ldots,m_k^1\}$ i
$S_2=\{m_0^2,\ldots,m_l^2\}$ dues sèries temporals, la primera sèrie
temporal està inclosa en la segona $S_1 \subseteq S_2$ si $\forall m
\in S_1: m \in S_2$. Aleshores, $S_1$ és una subsèrie temporal de
$S_2$. Aquesta definició d'inclusió determina un ordre parcial entre
les sèries temporals.



\subsubsection{Màxim i suprem}

La relació definida a~\ref{def:model:mesura-relacio-ordre} indueix
sobre una sèrie temporal una relació d'ordre total. Com que la sèrie
temporal s'ha considerat finita i sense elements repetits, quan la
sèrie temporal no és buida això comporta l'existència d'un màxim i
d'un mínim.  Si $S$ és una sèrie temporal, $\max(S)$ i $\min(S)$ són
respectivament la mesura màxima i mínima d'$S$.

\begin{definition}[Màxim i mínim]
  Sigui $S=\{m_0,\ldots,m_k\}$ una sèrie temporal i $n\in S$ una
  mesura.  Direm que $n=\max(S)$ és el màxim de la sèrie temporal si i
  només si $\forall m \in S: n \geq m $.  Direm que $n=\min(S)$ és el
  mínim de la sèrie temporal si i només si $\forall m \in S: n \leq
  m$.
\end{definition}

El $\max(S)$ i el $\min(S)$ no estan definits quan la sèrie temporal
és buida: $S= \emptyset$. En
canvi, el suprem i l'ínfim estan definits per qualsevol
sèrie temporal tal com passa amb el conjunt estès de nombres reals,
\cite{cantrell:extendedreal}.  

\begin{definition}[Suprem i ínfim]\label{def:sgst:sup}\label{def:sgst:inf}
  Sigui $S=\{m_0,\ldots,m_k\}$ una sèrie temporal i $n\in S$ una
  mesura.  Direm que $n=\sup(S)$ és el suprem de la sèrie temporal si
  $n=\max(S)$ en cas que el màxim estigui definit o
  $n=(-\infty,\infty)$ en cas contrari.  Direm que $n=\inf(S)$ és
  l'ínfim de la sèrie temporal si $n=\min(S)$ en cas que el mínim
  estigui definit o $n=(+\infty,\infty)$ en cas contrari.
\end{definition}

Quan la sèrie temporal no és buida, per
ser un conjunt finit i d'ordre total, sempre hi ha un i només un màxim
i un mínim i per tant es corresponen amb el suprem i l'ínfim
respectivament.




\subsubsection{Unió}

La unió de dos conjunts és un conjunt que conté tots els elements
d'ambdós conjunts.  Per a poder unir dos conjunts amb estructura de
relació, $A \cup B$, cal que tots dos tinguin la mateixa estructura;
és a dir, en termes de SGBDR cal que $A$ i $B$ tinguin la mateixa
capçalera.

Per tal que l'operació d'unió de conjunts sigui vàlida per a les
sèries temporals cal, a més, tenir en compte quan dues sèries
temporals tenen mesures en el mateix instant de temps. En cas
d'utilitzar l'operació d'unió de conjunts la sèrie temporal resultant
no compliria amb la definició \ref{def:serie_temporal} ja que
contindria mesures amb temps repetits. Com a conseqüència, es
defineixen dues operacions d'unió per a les sèries temporals que
resolen la restricció del temps de forma diferent.

En primer lloc, es defineix la unió de dues sèries temporals que
escull les mesures del primer operand en cas de mesures repetides.
\begin{definition}[unió]
  Sigui $S_1=\{m_0^1, \dotsc, m_{k_1}^1\}$ i $S_2=\{m_0^2, \dotsc,
  m_{k_2}^2\}$ dues sèries temporals, la unió de les dues
  sèries temporals $S_1 \cup S_2$ és una sèrie temporal $S=\{m_0,
  \dotsc, m_k\}$ que conté totes les mesures de $S_1$ i les mesures de
  $S_2$ no repetides: $S_1 \cup S_2 = \{m^1 \in S_1 \vee m^2 \in S_2
  | m^2 \notinst S_1 \}$.
\end{definition}

Propietats de la unió de sèries temporals:
\begin{itemize}
\item La dimensió $k$ de la sèrie temporal resultant està fitada a
  $k_1 \leq k \leq k_1 + k_2$. 
\item No commutativa. En general
  $S_1\cup S_2 \neq S_2\cup S_1$ tot i que sí que es compleix
  l'equivalència respecte al cardinal $|S_1 \cup S_2| = |S_2\cup S_1|$.
\end{itemize}

En segon lloc, es defineix la unió temporal de dues sèries temporals
que és la unió sense tenir en compte les mesures que tenen el mateix
instant de temps i diferent valor.
\begin{definition}[unió temporal]
  Sigui $S_1=\{m_0^1, \dotsc, m_{k_1}^1\}$ i $S_2=\{m_0^2, \dotsc,
  m_{k_2}^2\}$ dues sèries temporals, la unió temporal de les dues
  sèries temporals $S_1 \cupt S_2$ és una sèrie temporal $S=\{m_0,
  \dotsc, m_k\}$ que conté les mesures de $S_1$ i de $S_2$ excloent
  les que només comparteixen el temps: $S_1 \cupt S_2 = \{ m^1 \in S_1
  \vee m^2 \in S_2 | m^1 \notinst S_2 \vee m^1 \in S_2, m^2 \notinst
  S_1 \}$.
\end{definition}


Propietats de la unió temporal:
\begin{itemize}
\item Commutativa
\end{itemize}



\subsubsection{Diferència}

La diferència de dos conjunts és un conjunt que conté tots els
elements del primer conjunt que no pertanyen al segon.  Per a poder
restar dos conjunts amb estructura de relació, $A - B$, cal que
tots dos tinguin la mateixa estructura; és a dir, en termes de SGBDR
cal que $A$ i $B$ tinguin la mateixa capçalera.

En la definició de l'operació de diferència cal tenir en compte les
dues pertinences possibles.

En primer lloc, es defineix la diferència atenent a la pertinença
estricta de conjunts. És a dir s'aplica la diferència de
conjunts a les sèries temporals.
\begin{definition}[diferència]
  Sigui $S_1=\{m_0^1, \dotsc, m_{k_1}^1\}$ i $S_2=\{m_0^2, \dotsc,
  m_{k_2}^2\}$ dues sèries temporals, la diferència de les dues
  sèries temporals $S_1 - S_2$ és una sèrie temporal $S=\{m_0,
  \dotsc, m_k\}$ que conté totes les mesures de $S_1$ que no pertanyen a
  $S_2$: $S_1 - S_2 = \{ m \in S_1 | m \notin S_2  \}$.
\end{definition}

En segon lloc, es defineix la diferència atenent a la pertinença
temporal.
\begin{definition}[diferència temporal]
  Sigui $S_1=\{m_0^1, \dotsc, m_{k_1}^1\}$ i $S_2=\{m_0^2, \dotsc,
  m_{k_2}^2\}$ dues sèries temporals, la diferència temporal de les
  dues sèries temporals $S_1 -^t S_2$ és una sèrie temporal
  $S=\{m_0, \dotsc, m_k\}$ que conté totes les mesures de $S_1$ que no
  pertanyen temporalment a $S_2$: $S_1 -^t S_2 = \{ m \in S_1 | m
  \notinst S_2 \}$.
\end{definition}




\subsubsection{Intersecció}

La intersecció de dos conjunts és un conjunt que conté els elements
comuns als dos conjunts.  Per a poder intersecar dos conjunts amb estructura
de relació, $A \cap B$, cal que tots dos tinguin la mateixa
estructura; és a dir, en termes de SGBDR cal que $A$ i $B$ tinguin la
mateixa capçalera.

En la definició de l'operació d'intersecció cal tenir en compte les
dues pertinences possibles.

En primer lloc, es defineix la diferència atenent a la pertinença
estricta de conjunts. És a dir s'aplica l'operació d'intersecció de
conjunts.
\begin{definition}[intersecció]
  Sigui $S_1=\{m_0^1, \dotsc, m_{k_1}^1\}$ i $S_2=\{m_0^2, \dotsc,
  m_{k_2}^2\}$ dues sèries temporals, la intersecció de les dues
  sèries temporals $S_1 \cap S_2$ és una sèrie temporal $S=\{m_0,
  \dotsc, m_k\}$ que conté les mesures de $S_1$ repetides a $S_2$:
  $S_1 \cap S_2 = \{ m \in S_1 | m \in S_2 \}$.
\end{definition}

En segon lloc, es defineix la intersecció atenent a la pertinença
temporal tenint en compte quan dues sèries temporals tenen mesures en
el mateix instant de temps però de valor diferent.
\begin{definition}[intersecció temporal]
  Sigui $S_1=\{m_0^1, \dotsc, m_{k_1}^1\}$ i $S_2=\{m_0^2, \dotsc,
  m_{k_2}^2\}$ dues sèries temporals, la intersecció temporal de les
  dues sèries temporals $S_1 \capt S_2$ és una sèrie temporal
  $S=\{m_0, \dotsc, m_k\}$ que conté les mesures de $S_1$ repetides
  temporalment a $S_2$: $S_1 \capt S_2 = \{ m \in S_1 | m \inst S_2
  \}$.
\end{definition}

Propietats de la intersecció:
\begin{itemize}
\item La intersecció és commutativa però la intersecció temporal no és
  commutativa.
\item A partir de la diferència es pot definir la intersecció: $S_1
  \cap S_2 \equiv S_1 - (S_1 - S_2)$.
\end{itemize}


\subsubsection{Diferència simètrica}

La diferència simètrica de dos conjunts és un conjunt que conté els
elements no comuns dels dos conjunts. La diferència simètrica de dos
conjunts $A \ominus B$ es defineix a partir de la diferència i la
unió:
\begin{align*}
A \ominus B  & \equiv (A-B)\cup(B-A)\\
             & \equiv (A\cup B)-(A\cap B)  \\
A \ominus B  & \subseteq A\cup B
\end{align*}

Seguint aquestes propietats es defineixen dues diferències
simètriques: una a partir de la diferència i la unió de sèries
temporals i una altra a partir de la diferència temporal i la unió
temporal.  Per tal que l'operació de diferència simètrica sigui vàlida
per a les sèries temporals cal tenir en compte quan dues sèries
temporals tenen mesures en el mateix instant de temps.

En primer lloc, es defineix la diferència simètrica excloent les
mesures amb el mateix temps però de valor diferent.
\begin{definition}[diferència simètrica]
  Sigui $S_1=\{m_0^1, \dotsc, m_{k_1}^1\}$ i $S_2=\{m_0^2, \dotsc,
  m_{k_2}^2\}$ dues sèries temporals, la diferència simètrica de les
  dues sèries temporals $S_1 \ominus S_2$ és una sèrie temporal
  $S=\{m_0, \dotsc, m_k\}$ que conté les mesures de $S_1$ o
  exclusivament les de $S_2$: $S_1 \ominus S_2 = \{ m^1 \in S_1 \vee
  m^2 \in S_2 | m^1 \notinst S_2, m^2 \notin S_1 \}$.
\end{definition}

En segon lloc, es defineix la diferència simètrica temporal excloent les
mesures amb el mateix temps.
\begin{definition}[diferència simètrica temporal]
  Sigui $S_1=\{m_0^1, \dotsc, m_{k_1}^1\}$ i $S_2=\{m_0^2, \dotsc,
  m_{k_2}^2\}$ dues sèries temporals, la diferència simètrica de les
  dues sèries temporals $S_1 \ominus S_2$ és una sèrie temporal
  $S=\{m_0, \dotsc, m_k\}$ que conté les mesures de $S_1$ o
  exclusivament les de $S_2$: $S_1 \ominus S_2 = \{ m^1 \in S_1 \vee
  m^2 \in S_2 | m^1 \notinst S_2, m^2 \notinst S_1 \}$.
\end{definition}



\subsubsection{Projecció}

La projecció és una operació dels SGBDR que selecciona uns atributs
determinats d'un conjunt. Es pot aplicar la projecció a les sèries
temporals de la mateixa manera que als
SGBDR \parencite{date:introduction}. 

Sigui $S =\{m_0, \dotsc, m_k\}$ una sèrie temporal i $A=\{a_0, \dotsc,
a_n\}$ un conjunt de noms d'atributs, la projecció de $S$ en $A$
s'escriu com $S\{a_0, \dotsc, a_n\}$.




\subsubsection{Selecció}

La selecció és una operació dels SGBDR que selecciona uns tuples
determinats d'un conjunt. Es pot aplicar la selecció a les sèries
temporals de la mateixa manera que als
SGBDR \parencite{date:introduction}. 

Sigui $S =\{m_0, \dotsc, m_k\}$ una sèrie temporal, $a_1$ i $a_2$ dos
noms d'atributs que pertanyen a $S$, i $a_1 \Theta a_2$ una expressió
booleana sobre $a_1$ i $a_2$, la selecció de $S$ per l'expressió
booleana s'escriu com $S \where a_1 \Theta a_2$.


\subsubsection{Reanomena}

El reanomena és una operació dels SGBDR que canvia el nom dels
atributs. Es pot aplicar el reanomena les sèries temporals de la
mateixa manera que als SGBDR \parencite{date:introduction}.

Sigui $S =\{m_0, \dotsc, m_k\}$ una sèrie temporal, $a$ un nom
d'atribut que pertany a $S$ i $b$ un que no hi pertany, el reanomenat
de $a$ per $b$ s'escriu com $S \rename a \as b$.





\subsubsection{Producte i junció}

El producte cartesià de dos conjunts és un conjunt que conté totes les
parelles possibles d'elements d'ambdós conjunts.  Per a poder
multiplicar dos conjunts amb estructura de relació, $A \times B$, en
termes de SGBDR cal que $A$ i $B$ no tinguin en comú noms d'atributs.
En els SGBDR, a diferència del producte de conjunts, el conjunt
resultant no és un conjunt de parells de tuples sinó un conjunt de
tuples.

\todo{} Definim el producte de dues sèries temporals, les qual en
forma canònica tinguin els atributs $t$ i $v$, com una sèrie temporal
amb atributs $t_1$, $v_1$, $t_2$ i $v_2$. Així doncs, per a sèries
temporals el producte resulta en una sèrie temporals amb dos atributs
de temps. Per aquest fet l'anomenem sèrie temporal doble
(v.\ \autoref{def:sgst:st-doble}).
\begin{definition}[producte]
  Sigui $S_1=\{m_0^1, \dotsc, m_{k_1}^1\}$ i $S_2=\{m_0^2, \dotsc,
  m_{k_2}^2\}$ dues sèries temporals en forma canònica, el producte de
  les dues sèries temporals $S_1 \times S_2$ és una sèrie temporal
  doble $S=\{m_0, \dotsc, m_k\}$ que conté la unió de totes les
  parelles de mesures de $S_1$ i $S_2$: $S_1 \times S_2 = \{
  (t_1,v_1,t_2,v_2) | (t_1,v_1) \in S_1 \wedge (t_2,v_2) \in S_2 \}$
\end{definition}

Propietats del producte:
\begin{itemize}
\item El cardinal resultant és $|S|=k_1k_2$
\item El grau resultant és $4$
\end{itemize}



La junció (\emph{join}) de dos conjunts és un conjunt que conté les
parelles d'elements d'ambdós conjunts que tenen el mateix valor per
als atributs comuns.  La junció de dos conjunts amb estructura de
relació, $A \join B$, es defineix com una selecció sobre el
producte \parencite{date:introduction}.


Per a les sèries temporals, definim la junció com l'ajuntament de les
parelles que tenen el mateix atribut de temps en ambdues sèries
temporals . El resultat de la junció és una sèrie temporal
multivaluada.
\begin{definition}[junció]\label{def:sgst:join}
  Sigui $S_1=\{m_0^1, \dotsc, m_{k_1}^1\}$ i $S_2=\{m_0^2, \dotsc,
  m_{k_2}^2\}$ dues sèries temporals en forma canònica, la junció de
  les dues sèries temporals $S_1 \join S_2$ és una sèrie temporal
  multivaluada $S=\{m_0, \dotsc, m_k\}$ que selecciona del producte de
  $S_1$ amb $S_2$ les mesures dobles amb temps iguals: $S_1 \join S_2
  = \{ (t,v_1,v_2) | (t_1,v_1,t_2,v_2) \in S_1\times S_2 \wedge
  t=t_1=t_2 \}$.
\end{definition}


Propietats de la junció:
\begin{itemize}
\item El cardinal resultant és $|S|\leq\min(k_1,k_2)$
\item És commutativa; tenint en compte que els atributs tenen nom i
  per tant l'ordre no importa.
\end{itemize}







\subsubsection{Computacionals: mapa, agregat i plec}

Per a poder operar amb els conjunts, a més de l'àlgebra definida fins
ara, es necessiten operadors amb funcionalitats computacionals; és a
dir, operadors que calculin amb els valors continguts en els conjunts. 

En els SGBDR els operadors computacionals bàsics són \emph{extend},
\emph{aggregate} i \emph{summarize} \parencite{date:introduction}.
Per a les sèries temporals definim operacions equivalents a les dues
primeres de la manera amb que habitualment s'utilitzen per als
conjunts. La tercera, el \emph{summarize}, és una operació creada a
partir de les altres dues que s'utilitza per a sintetitzar per grups i
per tant en les sèries temporals no té sentit perquè l'atribut temps
d'una sèrie temporal no pot tenir instants repetits i per tant no se'n
poden fer grups d'instants compartits.
% No obstant, es pot aplicar el \emph{summarize} per a l'atribut de
% valors: summarize S per S {v} add ...  però això ja no mapa a una
% sèrie temporal.

% De l'operador \emph{aggregate} dels SGBDR definit per
% \textcite{date:introduction} cal tenir en compte que en defineix
% dues vessants. Per una banda, defineix els \emph{aggregate operator
% invocation} que retornen valors escalars. Per altra banda, defineix
% els \emph{aggregate operator invocation} que serveixen per a
% treballar amb el \emph{summarize}.

Així doncs, a continuació es defineix l'operador mapa (\emph{map}) com
a equivalent a l'\emph{extend}, l'operador agregat (\emph{aggregate})
com a equivalent a l'\emph{aggregate} i l'operador plec (\emph{fold})
com una forma més general de calcular recursivament amb les mesures
que l'\emph{aggregate}.




L'operació de mapatge aplica una funció a cada element del conjunt.
\begin{definition}[mapa]
  Sigui $S=\{m_0, \dotsc, m_k\}$ una sèrie temporal i $f$ una funció
  sobre una mesura a on $f:m\mapsto m'$, el mapa de $f$ a $S$ és una
  sèrie temporal $S'=\{m_0', \dotsc, m_k'\}$ amb la funció aplicada a
  cada mesura: $\map(S,f) = \{\forall m\in S : f(m) \}$.
\end{definition}


L'operació d'agregació agrupa en una mesura els elements del conjunt
segons un criteri, per exemple estadístics.
\begin{definition}[agregat]
  Sigui $S=\{m_0, \dotsc, m_k\}$ una sèrie temporal, $m_i$ una mesura
  i $f$ una funció de dues mesures a on $f: m_a \times m_b \mapsto
  m_r$, l'agregat de $S$ segons $f$ amb valor inicial $m_i$ és una
  mesura $m' = (t',v')$ amb l'agrupament de les mesures seguint el
  criteri de la funció: $\agg(S,m_i,f) = f(\dots(
  f(f(f(m_i,m_0),m_1),\allowbreak m_2 )\dots),\allowbreak m_k)$.
\end{definition}

% Més compactament descrit amb
% \begin{align*}
%   \text{fold}: & S=\{m_0,\dotsc,m_k\} \times m_i \times f \longrightarrow m'= \\
%   & \begin{cases}
%     m_i & \text{si} |S|=0, \\
%     \text{fold}(S_1,f(m_i,m_1),f) & \text{altrament}
%   \end{cases}\\
%   \text{ a on } & m_1 \in S, S_1 = S - \{m_1\}
% \end{align*}


L'operació de plegament combina recursivament els elements del conjunt
segons un criteri.
\begin{definition}[plec]
  Sigui $S=\{m_0, \dotsc, m_k\}$ i $S_i=\{m_{i0}, \dotsc, m_{ik}\}$
  dues sèries temporals i $f$ una funció d'una mesura amb una sèrie
  temporal a on $f: S_a \times m_b \mapsto S_r$, el plec de $S$ segons
  $f$ amb valor inicial $S_i$ és una sèrie temporal $S'= \{m_0',
  \dotsc, m_k'\}$ amb l'agrupament de les mesures seguint el criteri
  de la funció: $\fold(S,S_i,f) = f(\dots(
  f(f(f(S_i,m_0),m_1),\allowbreak m_2 )\dots),\allowbreak m_k)$.
\end{definition}


Les operacions d'agregació i plegament tal com s'han definit es
realitzen en ordre aleatori de mesures. Segons el criteri que
s'utilitzi, l'ordre és important i per tant cal una operació que
computi tenint-lo en compte. A tal efecte, a continuació s'amplia la
funció de plegament. Per a la funció d'agregació es pot aplicar el
mateix concepte d'ordre.
\begin{definition}[plec amb ordre]
  Sigui $S=\{m_0, \dotsc, m_k\}$ i $S_i=\{m_{i0}, \dotsc, m_{ik}\}$
  dues sèries temporals, $f$ una funció d'una mesura amb una sèrie
  temporal a on $f: S_a \times m_b \mapsto S_r$ i $o$ una funció que
  treu una mesura d'una sèrie temporal a on $o: S_c \mapsto m_c$, el
  plec de $S$ segons $f$ amb valor inicial $S_i$ i ordre $o$ és una
  sèrie temporal $S'= \{m_0', \dotsc, m_k'\}$ amb l'agrupament de les
  mesures seguint el criteri i l'ordre de les funcions:
  $\fold(S,S_i,f,o) =
  \begin{cases}
    S_i & \text{si } |S|=0, \\
    \fold(S_o,f(S_i,m_o),f,o) & \text{altrament}
  \end{cases}$ a on $m_o = o(S)$ i $S_o = S - \{m_o\}$.
\end{definition}

El plec amb ordre és necessari quan la funció $f$ no és associativa ni
commutativa perquè llavors l'ordre dels càlculs importa. Es pot
observar que el plec sense ordre és un plec amb ordre aleatori:
$\fold(S,S_i,f)\equiv \fold(S,S_i,f,o)$ a on $o=\text{aleatori}(S)$.
De manera semblant $\agg(S,m_i,f)\equiv \agg(S,m_i,f,o)$ a on
$o=\text{aleatori}(S)$.



Propietats d'operacions de plegament:
\begin{itemize}
\item El plec d'una sèrie temporal buida és la sèrie inicial; $\fold:
  \{\} \times f \times S_i \mapsto S_i$.

\item El plec per una funció que sempre retorni la sèrie inicial és la
  sèrie inicial; $\fold: S \times S_i \times f \mapsto S_i$ a on
  $f: S_i \times m \mapsto S_i$.

\item El plec per una funció que només retorni la mesura original és
  una sèrie amb una sola mesura; $\fold: S \times S_i \times f \mapsto
  S'$ a on $f:S_i\times m \mapsto \{m\}$ i $|S'|=1$.


\item La funció d'unió en el plegament permet fer la identitat, $S
  \equiv \fold(S,\{\},(S_i,m_i) \mapsto S_i \cup \{m_i\}$.


\item Els mapes es poden implementar com a plecs; $\map(S,f) \equiv
  \fold(S,\{\},f')$ a on $f': S_i \times m_a \mapsto \{f(m_a)\}
  \cup S_i$.

\item Els agregats es poden implementar com a plecs; $\agg(S,m_i,f) \
  equiv \fold(S,\{m_i\},f')$ a on $f': \{m_i\} \times m \mapsto
  \{f(m_i,m)\}$.

\end{itemize}


\paragraph{Exemples} Definicions de funcions d'exemple a partir de les
operacions computacionals.

Exemple d'operacions de mapatge:
\begin{itemize}
\item $\operatorname{identitat}: S \mapsto S'$ a on $S'=
  \map(S,(t,v)\mapsto(t,v))$
\item $\operatorname{intercanvi}: S \mapsto S'$ a on $S'=
  \map(S,(t,v)\mapsto(v,t))$
\item $\operatorname{translaci\acute{o}}: S \times \delta \mapsto S'$ a on $S'=
  \map(S,(t,v)\mapsto(t+\delta,v))$
\item $\operatorname{t\times v}: S \mapsto S'$ a on $S'=
  \map(S,(t,v)\mapsto(t,t\cdot v))$
\item $\operatorname{tpredecessors_{v1}}: S \mapsto S'$ a on $S'= \map(S,(t,v)
  \mapsto (t,T(\ant_S(m)))$, usant l'operació predecessor de la
  \autoref{def:sgst:ant}
\item $\operatorname{vpredecessors}: S \mapsto S'$ a on $S'= \map(S,(t,v)
  \mapsto (t,V(\ant_S(m)))$, usant l'operació predecessor de la
  \autoref{def:sgst:ant}
\end{itemize}

Exemple d'operacions d'agregació:
\begin{itemize}
\item $\operatorname{cardinal}: S \mapsto m'$ a on $m'=
  \agg(S,(0,0),(t^i,v^i)\times(t,v)\mapsto(t^i,v^i+1)$
\item $\operatorname{sumaV}: S \mapsto m'$ a on $m'=
  \agg(S,(0,0),(t^i,v^i)\times(t,v)\mapsto(t,v+v^i))$
\item $\operatorname{sup}: S \mapsto m'$ a on $m'=
  \agg(S,(-\infty,\infty),(t^i,v^i)\times(t,v)\mapsto [(t^i,v^i)
  \text{ if } t < t^i \text{ else } (t,v) ])$, implementació de
  l'operació suprem de la \autoref{def:sgst:sup} a partir de
  l'agregació
\item $\operatorname{ant}: S \times m \mapsto m'$ a on $m'=
  \agg(S,(-\infty,\infty),(t^i,v^i)\times(t,v)\mapsto [(t,v)
  \text{ if } t^i < t < T(m) \text{ else } (t^i,v^i) ])$,
  implementació de l'operació predecessor de la \autoref{def:sgst:ant} a
  partir de l'agregació 
\end{itemize}


Exemple d'operació de plegament:
\begin{itemize}
\item $\operatorname{tpredecessors}_{v2}: S \mapsto S'$ a on $S'=
  \fold(S,S^b,S_i\times (t,v) \mapsto f)$ a on $S^b =
  \map(S,(t^b,v^b)\mapsto(t^b,-\infty))$, $f= \{(t,tp)\} \cup S_i$ i
  $t_p=T(\sup(S^i \where t^i < t))$, sense usar l'operació predecessor
  a diferència de l'exemple $\operatorname{tpredecessors_{v1}}$
\end{itemize}




\subsubsection{Computacionals per a dues sèries temporals}

Una operació en els conjunts és la d'aplicar un operador binari a
totes les parelles possibles dels elements de dos conjunts. Per
exemple la suma aplicada a dos conjunts A i B és un conjunt $A + B =
\{ e_a+e_b : (e_a,e_b) \in A\times B \}$.

Per a les sèries temporals també calen operacions computacionals amb
els valors de dues sèries temporals. En el cas d'operar amb dues
sèries temporals primer cal ajuntar les dues sèries temporals amb les
que es vol operar i després aplicar les operacions computacionals a la
sèrie temporal resultant.


El producte i la junció són els operadors que permeten crear parelles
de mesures de dues sèries temporals. Per a operar amb els valors de
dues sèries temporals la junció és més adequada ja que permet ajuntar
el valors que tenen temps comuns. Així doncs, per a aplicar un
operador binari $\operatorname{op}$ que calculi amb els valors de
dues sèries temporals:

\[
\operatorname{op}: S_1 \times S_2 \longrightarrow S'
\]
\[
\text{a on } S' = \map(\join(S_1,S_2),(t,v^1,v^2)\mapsto(t^1,v^1
\operatorname{op} v^2))
\]

Cal tenir en compte que la junció de la \autoref{def:sgst:join} només
sap operar amb dues sèries temporals que tinguin el mateix vector de
temps; és a dir regulars entre elles (v.\
def.~\ref{def:st:regular}). En el cas que no tinguin el mateix vector
de temps, es pot aplicar la junció temporal de la
\autoref{def:sgst:joint}.


Exemples de l'aplicació d'operacions computacionals per a dues sèries
temporals
\begin{itemize}
\item $S' = S_1 + S_2$
\item $\operatorname{gradient}: S \mapsto S'$ a on $S'= S -
  \operatorname{vpredecessors}(S)$
\end{itemize}









\subsection{Bàsiques de seqüències}

Atesa la relació d'ordre induïda pel temps en una sèrie temporal
(def.\ \ref{def:model:mesura-relacio-ordre}), les sèries temporals es
poden tractar com a seqüències.  En aquest apartat definim operadors
per a les sèries temporals recollint els operadors habituals que tenen
les seqüències.

Els operadors que treballen amb seqüències tenen en compte l'atribut
que marca un ordre total en el conjunt. En el cas de les sèries
temporals aquest atribut és el temps.



\subsubsection{Interval}

L'interval sobre una seqüència és la subseqüència compresa entre dos
elements.  Per a les sèries temporals és possible definir el concepte
d'interval sobre la seqüència com la subsèrie entre dos instants de
temps, semblant a com es fa a \cite{last:keogh,last:hetland}.

\begin{definition}[Interval]
  \label{def:model:st-interval}
  Sigui $S=\{m_0, \ldots, m_k\}$ una sèrie temporal. Definirem el subconjunt
  $S(r,t) \subseteq S$ com la sèrie temporal $S(r,t)=\{m\in S
  | r<T(m)<t\}$, a on $r$ i $t$ són dos instants de temps.

  Tal com es fa en les seqüències, es defineix una notació de
  parèntesis i claudàtors per indicar si l'interval és obert, tancat o
  semiobert:

  $S[r,t)=\{m\in S  | r\leq T(m)< t\}$

  $S(r,t]=\{m\in S  | r<T(m)\leq t\}$

  $S[r,t]=\{m\in S  | r\leq T(m)\leq t\}$
\end{definition}


Propietats:
\begin{itemize}
\item La subsèrie $S[-\infty,t)\subseteq S$ és equivalent a la sèrie
  temporal $S[-\infty,t) \equiv S[T(\inf(S)),t)$. De la mateixa manera
  $S(r,+\infty] \equiv S(r,T(\sup(S))]$.

\item L'interval degenerat $S[t,t]\subseteq S$ és equivalent a la
  sèrie temporal $S[t,t] \equiv \{m\in S | T(m)=t \}$. El intervals
  $S(t,t]\subseteq S$ i $S[t,t)\subseteq S$ són equivalents a la sèrie
  temporal buida $S(t,t] \equiv S[t,t) \equiv \emptyset$ ja que per
  ser els temps d'ordre total $\nexists T(m): t < T(m) \leq t$ o
  $\nexists T(m): t \leq T(m) < t$, respectivament. 

\item La subsèrie $S[-\infty,+\infty] \subseteq S$ és equivalent a la
  sèrie temporal original $S[-\infty,+\infty] = S$. La subsèrie
  $S(-\infty,+\infty) \subseteq S$ només és equivalent a la sèrie
  temporal original quan aquesta no conté mesures indefinides
  $S(-\infty,+\infty) \equiv S: (-\infty,v_a)\notin S \wedge
  (+\infty,v_b)\notin S$.
\end{itemize}




\subsubsection{Successió}

També atenent a la relació d'ordre induïda pel temps en una sèrie temporal, es
defineix el concepte de mesura següent i mesura anterior en una
seqüència.


\begin{definition}[Successor i
  predecessor]\label{def:sgst:seg}\label{def:sgst:ant}
  Sigui $S=\{m_0, \ldots, m_k\}$ una sèrie temporal i $l\in S$ i $n$ dues
  mesures. Direm que $l$ és el successor de $n$ en $S$ i ho notarem
  com $l=\seg\limits_S(n)$ si i només si $l=\inf(S(T(n),+\infty])$.
  Direm que $l$ és el predecessor de $n$ en $S$ i ho notarem com
  $l=\ant\limits_S(n)$ si i només si $l=\sup(S[-\infty,T(n)))$.

Quan no hi hagi dubte de la sèrie temporal que marca l'ordre, per
exemple quan $n\in S$, podrem escriure $\seg(n)$ i $\ant(n)$.
\end{definition}

S'observa que s'obtenen mesures indefinides en els casos que la
mesura següent o anterior es calcula respectivament per la mesura
suprema o ínfima de la sèrie temporal: $\seg\limits_S(\sup
S)=(+\infty,\infty)$ i $\ant\limits_S(\inf S)=(-\infty,\infty)$.

De la definició anterior es dedueix que donada una sèrie temporal $S$
que no conté mesures indefinides i donada la mesura indefinida
$o=(+\infty,\infty)$, el predecessor de $o$ sempre és el suprem de la
sèrie temporal $\ant\limits_S( (+\infty,\infty) ) = \sup(S): \forall
m\in S: T(m)\in\mathbb{R}$.  % S\equiv S(-\infty,+\infty)
\emph{Demostració: Sigui $S$ una sèrie temporal i $o=(+\infty,\infty)$
  una mesura indefinida, el predecessor de $o$ en $S$ és una mesura
  $l=\ant\limits_S(o)$ que compleix
  $l=\sup(S[-\infty,T(o)))$. Substituint, s'obté que
  $l=\sup(S[-\infty,+\infty))=\sup(S-m):m\in S:T(m)=+\infty \notin
  \mathbb{R}$, i per tant com que $S$ no té mesures indefinides es
  demostra que $l=\sup(S)$.  } De manera semblant es pot demostrar que
$\seg\limits_S( (-\infty,\infty) ) = \inf(S): \forall m\in S:
T(m)\in\mathbb{R}$.


\subsubsection{Concatenació}

La concatenació és una operació que uneix dues seqüències amb els
elements de la primera seqüència seguits pels de la segona. Així
doncs, la concatenació de les seqüències té un sentit semblant al que
la unió té en els conjunts. 

Per a les sèries temporals, per tal que l'operació de concatenació
uneixi amb ordre els operands, cal tenir en compte l'interval que
ocupa cada sèrie temporal segons el seu atribut de temps.  És a dir,
la concatenació de dues sèries temporals consisteix a unir la part de
la segona sèrie temporal que no està inclosa en el rang temporal de la
primera.

Per a poder concatenar dues sèries temporals cal que ambdues tinguin
la mateixa estructura, de la mateixa manera que ja s'ha vist amb
l'operació d'unió.


\begin{definition}[concatenació]
  Sigui $S_1=\{m_0^1, \dotsc, m_{k_1}^1\}$ i $S_2=\{m_0^2, \dotsc,
  m_{k_2}^2\}$ dues sèries temporals, la concatenació de les dues
  sèries temporals $S_1 || S_2$ és una sèrie temporal $S=\{m_0,
  \dotsc, m_k\}$ que conté totes les mesures de $S_1$ i les mesures de
  $S_2$ que no intersequen en l'interval de $S_1$; $S_1 || S_2 = S_1
  \cup ( S_2 - S_2[t_1,t_2] )$ a on $t_1=T(\inf S_1)$ i $t_2=T(\sup
  S_1)$.
\end{definition}

Propietats
\begin{itemize}
\item La concatenació no és commutativa
\end{itemize}







\subsection{Funció temporal}

Atenent a que una sèrie temporal és la representació d'un funció
contínua (v.\ cap.~\todo{Ref!Falta fer el capítol de representació})
cal definir operacions per a tractar convenientment aquesta
naturalesa.

En aquest apartat definim aquestes operacions com una redefinició de
les bàsiques anteriors per a aplicar-les tenint en compte la sèrie
temporal com una funció contínua.



\subsubsection{Interval temporal}

Sigui $S$ una sèrie temporal i $i=[t_0,t_f]$ un interval de temps, per
una banda s'ha definit l'interval sobre la seqüència d'una sèrie
temporal $S(t_0,t_f]$ (def.~\ref{def:model:st-interval}) i per altra
banda s'ha definit la representació contínua $r$ d'una sèrie temporal
$S(t)^r$ \todo{referenciar la definició de repr}.  Per seleccionar un
interval temporal cal tenir en compte tant l'interval sobre la
seqüència com la representació contínua, Aquest interval temporal
s'anota com interval temporal de $S$ en $i$ amb representació $r$ o bé
$S[t_o,t_f]^r$.



\begin{definition}[Interval temporal]
  Sigui $S=\{m_0, \ldots, m_k\}$ una sèrie temporal, $i=[t_0,t_f]$ un
  interval de temps i $r$ una funció de representació, l'interval
  temporal de $S$ en $i$ amb representació $r$, $S[t_o,t_f]^r$, és una
  sèrie temporal $S'=\{m_0, \ldots, m_{k'}\}$ amb les mesures que són
  dins del rang temporal $i$ segons marca la funció de representació:
  $S[t_o,t_f]^r= \forall t \in [t_0,t_f] : S' = S(t)^r $
\end{definition}


A continuació s'exemplifica utilitzant la representació \emph{zohe}
\todo{ref}.
\begin{definition}[Interval temporal \emph{zohe}]
  Sigui $S$ una sèrie temporal, $i=[t_0,t_f]$ un interval de temps i
  \emph{zohe} la representació $S(t)$ amb \emph{zero-order-hold} cap
  enrere, es defineix la subsèrie $S[t_0,t_f]^{\text{zohe}}$ com la
  sèrie temporal $S[t_0,t_f]^{\text{zohe}} = S(t_0,t_f] \cup \{m\}$ a
  on $m=(t_f,v)$ i $v= V(\inf( S[t_f,+\infty] ))$.
\end{definition}
  %Atenció S(t_0,t_f] \cup \{m\} no és equivalent a  (S \cup \{m\})(t_0,t_f] ni sabent que m=(t_f,v); comprovar-ho pel cas t_0=t_f



Propietats de l'interval temporal:

\begin{itemize}
\item Sigui $t_a$ un instant de temps, la selecció de
  $S$ en $[t_a,t_a]^r$ és equivalent a la representació contínua
  $S(t_a)^r$.
\end{itemize}




\subsubsection{Selecció temporal}


La selecció  temporal d'una sèrie temporal permet canviar, en el
context d'una representació, la resolució a una de marcada per un
conjunt d'instants de temps. 

Sigui $S$ una sèrie temporal, $i= \{t_0,t_1,\dotsc,t_n\}$ un conjunt
d'instants de temps i la representació contínua $r$ de la sèrie
temporal $S(t)$, la selecció de resolució s'anota com resolució de $S$
en $i$ amb representació $r$ o bé $S[i]^r$.


\begin{definition}[Selecció temporal]
  Sigui $S=\{m_0, \ldots, m_k\}$ una sèrie temporal,
  $i=\{t_0,t_1,\dotsc,t_n\}$ un conjunt d'instants de temps i $r$ una
  funció de representació, la selecció temporal de $S$ en $i$ amb
  representació $r$, $S[i]^r$, és una sèrie temporal $S'=\{m_0, \ldots, m_n\}$
  amb les mesures amb els temps d'$i$ segons marca la funció de
  representació: $S[i]^r= S[t_0,t_0]^r \cup S[t_1,t_1]^r \cup \dotsb
  \cup S[t_n,t_n]^r$.
\end{definition}

Propietats de la selecció temporal:
\begin{itemize}

\item El cardinal de la sèrie temporal resultant és el mateix que el
  del conjunt d'instants de temps $|S[i]^r| = |i|$

\item La selecció temporal d'una sèrie temporal en un conjunt de temps
  equi-espaiat $i = \{\tau+n\delta | n\in\mathbb{N}, n\leq s \}$ és una
  sèrie temporal regular $S[i]^r \equiv \{ (\tau, v_0),
  (\tau+\delta,v_1), \dotsc , (\tau+s\delta,v_1)\}$
\end{itemize}




\subsubsection{Concatenació temporal}

La concatenació temporal és l'operació de concatenació que té en
compte la representació de les sèries temporals.  És a dir, la
concatenació temporal de dues sèries temporals uneix la part de la
segona sèrie temporal que no està inclosa en l'interval temporal de la
primera.


\begin{definition}[concatenació temporal]
  Sigui $S_1=\{m_0^1, \dotsc, m_{k_1}^1\}$ i $S_2=\{m_0^2, \dotsc,
  m_{k_2}^2\}$ dues sèries temporals i $r$ una funció de
  representació, la concatenació temporal de les dues sèries temporals
  amb representació $r$, $S_1 ||^r S_2$, és una sèrie temporal $S=\{m_0,
  \dotsc, m_k\}$ que conté totes les mesures de $S_1$ i les mesures de
  $S_2$ que no intersequen en l'interval temporal de $S_1$; $S_1 ||^r
  S_2 = S_1 \cup ( S_2 - S_2[t_1,t_2]^r )$ a on $t_1=T(\inf S_1)$ i
  $t_2=T(\sup S_1)$.
\end{definition}

Propietats de la concatenació temporal:
\begin{itemize}
\item No commutativa
\end{itemize}




\subsubsection{Junció temporal}

La junció temporal de dues sèries temporals és la junció que té en
compte la representació de les sèries temporals. És a dir, la junció
temporal de dues sèries temporals ajunta parelles de mesures
seleccionant el mateix atribut de temps en ambdues sèries temporals.


\begin{definition}[junció temporal]
  Sigui $S_1=\{m_0^1, \dotsc, m_{k_1}^1\}$ i $S_2=\{m_0^2, \dotsc,
  m_{k_2}^2\}$ dues sèries temporals en forma canònica i $r$ una
  funció de representació, la junció temporal de les dues sèries
  temporals amb representació $r$, $S_1 \join^r S_2$, és una sèrie
  temporal multivaluada $S=\{m_0, \dotsc, m_k\}$ que ajunta les
  mesures seleccionant els mateixos temps a cada sèrie temporal segons
  la funció de representació; $S_1 \join^r S_2 = \{m=(t',v_1,v_2) |
  (t',v_1) \in S_1[t']^r \wedge (t',v_2) \in S_2[t']^r \}$ a on $t'\in
  S_1\{t\} \cup S_2\{t\}$.
\end{definition}


Propietats de la junció temporal:
\begin{itemize}
\item El cardinal resultant és $|S'| \leq k_1 + k_2$
\item És commutativa; tenint en compte que els atributs tenen nom i
  per tant l'ordre no importa.
\end{itemize}



\todo{}
També es defineix l'operació de semijunció temporal que és una junció
no commutativa a on la primera sèrie temporal marca el vector de temps
de fusió,

\todo{}
Semifusió de dues sèries temporals $S_1 \text{ semifusió } S_2$, a on la primera sèrie temporal marca el vector de temps de fusió, 
\[
S_1 \text{ semifusió } S_2 = S_1 \text{ fusió } S_2[S_1\{t\}]^r
\]











%%% Local Variables:
%%% TeX-master: "main"
%%% End:







% LocalWords:  SGST

\section{Naturalesa/tipologia/patologies de les sèries temporals}


Perquè RRDtool diferencia entre comptadors i magnituds?

[segev87] diferencia entre step-wise constant, discret (potser aquest tal com se'l defineix són intervals temporals), continu. Ho anomena (semantic behavior, property, interpret)
 tipus de la sèrie temporal i diu que es poden definir interpolacions per cada una.




[John G. Proakis, Dimitris G. Manolakis 2007 Tratamiento digital de señales/Digital signal processing 4a ed pp11-12(segons wikipedia)] Acquisition: Discrete signals may have several origins, but can usually be classified into one of two groups:[1]
*By acquiring values of an analog signal at constant or variable rate. This process is called sampling.[2]
*By recording the number of events of a given kind over finite time periods. For example, this could be the number of people taking a certain elevator every day.



\subsubsection{Regularitat de les sèries temporals} 

Sigui $S=\{m_0,\ldots,m_k\}$ una sèrie temporal, $t$ un instant de
temps i $\delta$ una durada de temps, les mesures de la sèrie temporal
es poden localitzar en l'interval de temps $i_0=[t,t+\delta]$ i els
seus múltiples $i_j=[t+j\delta \,,\, t+(j+1)\delta]$ per $j=0,1,2,\ldots$.
En processat de senyal aquests intervals de temps s'anomenen intervals
de mostreig, $\delta$ s'anomena període de mostreig i $t$ s'anomena
temps inicial del mostreig.  La sèrie temporal $S$ és de naturalesa
diferent segons la situació dels temps $T(m_i)$ en els intervals de
temps $i_j$.

Una sèrie temporal és regular quan les mesures són equidistants en el
temps, tal com ho anomenen a \cite{last:hetland}.

\begin{definition}[Sèrie temporal regular]
  \label{def:st:regular}
  Sigui $S=\{m_0,\ldots,m_k\}$ una sèrie temporal, $t$ un instant de
  temps i $\delta$ una durada de temps. Direm que $S$ és regular si i
  només si $\forall m \in S(T(\min(S),\infty):T(m) - T(\ant(m)) =
  \delta$ i $T(\min(S))=t$.
\end{definition}

Si una sèrie temporal és regular, l'anomenem sèrie temporal mostrejada
regularment amb període de mostreig $\delta$ iniciada a $t$. Si el $t$
pot ser qualsevol llavors simplement l'anomenem sèrie temporal
mostrejada regularment amb període de mostreig $\delta$.

Noteu que si es complís
la definició excepte que no s'iniciés en el temps que exigim
$T(\min(S))=t$, aleshores la sèrie temporal seria equidistant però a
efectes de mostreig no la podríem anomenar regular; sí que seria una
sèrie temporal de temps real (v.\ def.~\ref{def:st:tempsreal}).


Una sèrie temporal és no regular quan no és regular. 
En les sèries temporals no regulars es poden distingir tres casos: temps real, ultramostreig i inframostreig.

Una sèrie temporal és de temps real quan a cada interval de mostreig hi ha una i només una mesura. L'interval de mostreig pot estar acotat per una durada anomenada termini.

\begin{definition}[Sèrie temporal de temps real]\label{def:st:tempsreal}
  Sigui $S=\{m_0,\dotsc,m_k\}$ una sèrie temporal, $t$ un instant de
  temps, $\delta$ una durada de temps i $D$ una durada que indica
  termini. Direm que $S$ és de temps real si i només si $D\leq\delta$
  i $\forall n\in\{0,\ldots,|S|-1\}: \exists!m \in
  S(t+n\delta,t+n\delta+D]$.  Aleshores la sèrie temporal està
  mostrejada en temps real per al temps de mostreig $\delta$ amb
  compliment del termini $D$.
\end{definition}

Si una sèrie temporal és de temps real, l'anomenem  sèrie temporal mostrejada
en temps real amb període de mostreig $\delta$ i compliment del termini $D$.
Si $D=\delta$, es pot anomenar que $S$ és una sèrie temporal de temps real sense termini.


% \paragraph{Ultramostreig} Una sèrie temporal està ultramostrejada (\emph{upsampling}) quan a cada interval de mostreig hi ha una mesura o més d'una. 
% \[
% \text{Ultramostrejada?}: \text{Sèrie temporal} \times T_0 \times \delta \longrightarrow \text{Booleà}
% \]

% Una sèrie temporal $S$ està ultramostrejada ssi $S$ no és de temps real i $\exists m_i=(v_i,t_i)\in S:T_0+(n-1)\delta \leq t_i < T_0+n\delta:\forall n\in\{1,\ldots,|S|\}$.

% \paragraph{Inframostreig} Una sèrie temporal està inframostrejada (\emph{downsampling}) quan en algun interval de mostreig no hi ha cap mesura. 
% \[
% \text{Inframostrejada?}: \text{Sèrie temporal} \times T_0 \times \delta \longrightarrow \text{Booleà}
% \]

% Una sèrie temporal $S$ està inframostrejada ssi $\nexists m_i=(v_i,t_i)\in S:T_0+(n-1)\delta \leq t_i < T_0+n\delta:\forall n\in\{1,\ldots,|S|\}$.








\subsubsection{Representació de les sèries temporals}



La naturalesa indueix representacions?
Jo puc utilitzar qualsevol representació donada una sèrie temporal, però això em pot causa perjudici si no s'adiu amb la naturalesa.


La representació serveix per interpolar:

zoh, zoh cap enrere, lineal, etc.


Una sèrie temporal és la representació discreta d'una funció contínua. A partir de la sèrie temporal es pot definir una funció contínua. 

A teoria de senyal s'estudia com fer que aquesta s'aproximi a la real. Estudiant com a senyal fan: donada una sèrie temporal dir quina funció s'hi 'ajusta' més. 

Però jo puc preguntar donada una sèrie temporal quina funció representa i puc dir per representar a zohe és tal, per representar a lineal és qual. 

Potser millor dir-li interpretació?



\paragraph{Representació de sèries temporals}

\textcite{last:keogh}, cita vàries representacions per les sèries temporals com per exemple \emph{Fourier Transforms}, \emph{Wavelets}, \emph{Symbolic Mappings} o \emph{Piecewise Linear Representation} (PLR), però assenyala aquesta última com la representació més utilitzada. 
La PLR, funció definida a trossos lineal, és l'aproximació d'una sèrie temporal $S$, de llargada $n$, amb $K$ segments rectes. Els segments podrien ser polinomis de qualsevol grau, però la manera més comuna de representar sèries temporals és amb funcions lineals, segons Keogh, \cite{keogh02}.
Per aproximar el segment $S(t_a:t_b]$ d'una sèrie $S$, Keogh defineix dues tècniques: interpolació lineal, la recta que connecta $t_a$ i $t_b$, i regressió lineal, la millor recta que aproxima per mínims quadrats el segment entre $t_a$ i $t_b$.

Però també es pot representar una sèrie temporal amb una funció esglaó (\emph{step} o \emph{staircase function}); és a dir, amb una funció definida a trossos constant (\emph{piecewise constant representation}).
La representació a trossos constant és utilitzada en electrònica als convertidors digital-analògic (DAC, \emph{digital-to-analog converter}). En aquest cas, un senyal discret es considera una sèrie temporal i per reconstruir el senyal continu típicament s'aplica el model de \emph{zero-order hold}, equivalent a la representació a trossos constant,  o el de \emph{first-order hold},  equivalent a la representació a trossos lineal.
El model de \emph{zero-order hold} consisteix en mantenir constant cada valor fins al proper. S'obté una representació a trossos constant que en electrònica s'anomena seqüència de pulsos rectangulars (\emph{rectangular pulses}).

%http://en.wikipedia.org/wiki/Piecewise

%http://ca.wikipedia.org/wiki/Funció_definida_a_trossos

%http://en.wikipedia.org/wiki/Rectangular_function

%http://en.wikipedia.org/wiki/Step_function

% Piecewise Aggregate Approximation (PAA) \cite{keogh00}: aproxima una sèrie temporal partint-la en segments de la mateixa mida i emmagatzemant la mitjana dels punts que cauen dins del segment. Redueix de dimensió $n$ a dimensió $N$

% Adaptive Piecewise Constant Approximation (APCA) \cite{keogh01}: com el PAA però amb segments de mida variable.

A continuació,  la representació  d'una sèrie temporal segons el model de \emph{zero-order hold} s'estén per diferents continuïtats en els intervals de temps de representació.

Sigui $S$ una sèrie temporal, es defineix $S(t)$ com la representació
de la sèrie temporal contínuament al llarg del temps $t$.  En primer
lloc, es representa amb \emph{zero-order hold} a partir de funcions
graó contínues per la dreta (\emph{right-continuous}).

\begin{definition}[Representació amb \emph{zero-order hold}]
Sigui $S=\{m_0,\ldots,m_k\}$ una sèrie temporal, la representació  $S(t)$ amb \emph{zero-order hold} es defineix
\[
\forall t \in \mathbb{R} ,\forall m \in S: S(t) =
\begin{cases}
  V(\min S) & \text{si } t < T(\min S) \\
  V(m) & \text{si }  t\in [T(m),T(\seg m))
\end{cases}
\]
\end{definition}

En segon lloc, es representa $S(t)$ amb \emph{zero-order hold} centrada en
l'interval, definit també a partir de funcions graó contínues per la
dreta.

\begin{definition}[Representació amb \emph{zero-order hold} centrada en l'interval]
  Sigui $S=\{m_0,\ldots,m_k\}$ una sèrie temporal, la representació
  $S(t)$ amb \emph{zero-order hold} centrada en l'interval es defineix
\[
\forall t \in \mathbb{R}  ,\forall m \in S:
S(t) =  
\begin{cases}
  V(m) & \text{si } t = \frac{T(\ant m)+T(m)}{2} \\
  V(m) & \text{si } t\in \left( \frac{T(\ant m)+T(m)}{2},\frac{T(m)+T(\seg m)}{2} \right) \
\end{cases}
\]
\end{definition}


En tercer lloc, es representa $S(t)$ amb \emph{zero-order hold} cap enrere, ara definit a partir de funcions graó contínues per l'esquerra.
\begin{definition}[Representació en \emph{zero-order hold} cap enrere]
  Sigui $S=\{m_0,\ldots,m_k\}$ una sèrie temporal, la representació
  $S(t)$ amb \emph{zero-order hold} cap enrere es defineix
\[
\forall t \in \mathbb{R}  ,\forall m \in S:
S(t) =  
\begin{cases}
  V(\max S) & \text{si } t > T(\max S) \\
  V(m) & \text{si }  t\in (T(\ant m),T(m)]
\end{cases}
\]
\end{definition}

Sigui $S$ una sèrie temporal regular i $\delta$ una durada de temps, aleshores la representació de $S(t)$ amb \emph{zero-order hold} és la mateixa que la de $S(t-\delta)$ amb \emph{zero-order hold} cap enrere i és la mateixa que la de $S(t-\frac{\delta}{2})$ centrada en l'interval. 




\subsubsection{Equivalència en representació}


$S_1=\{(1,1),(3,0),(5,1)\}$ i $S_2=\{(1,1),(2,0),(3,0)(4,1),(5,1)\}$ són equivalents amb representació zohe però no són equivalents amb representació lineal ja que hauria de ser $S_2'=\{(1,1),(2, 0{,}5 ),(3,0),(4, 0{,}5),(5,1)\}$. 




%%% Local Variables:
%%% TeX-master: "main"
%%% End:







% LocalWords:  SGST


\chapter{Model SGSTM}


En aquest capítol es defineixen els operadors que permeten modelar el comportament i la manipulació de les dades.



\section{Model estructural de dades}

Una MTSDB és una relació de buffers amb discs. 


\begin{figure}[tp]
\centering
\input{imatges/model/mtsms-arquitectura_interna.tex}
\caption{Arquitectura del model SGSTM}
\label{fig:model:bdstm}
\end{figure}


\subsection{Buffer}\label{sec:model:buffer}\todo{falta parlar de regularitat de ST}\todo{falta parlar de representació de ST}

Un buffer és un contenidor d'una sèrie temporal, regular o no regular, que mitjançant una funció permet regularitzar aquesta sèrie temporal amb un període de mostreig constant. A l'acció de regularitzar un interval d'una sèrie temporal l'anomenarem consolidació, al període de mostreig contant l'anomenarem pas de consolidació i a la funció de regularització l'anomenarem agregador d'atributs.

\begin{definition}[Buffer]
  Definim \emph{buffer} com el tuple $(S,\tau,\delta,f)$, en el que
  $S$ és una sèrie temporal, $\tau$ és el darrer instant de temps de
  consolidació, $\delta$ és la durada del pas de consolidació i $f$ és
  un agregador d'atributs.
\end{definition}

La consolidació d'una sèrie temporal s'inicia en un instant de temps concret i té lloc a cada pas de consolidació. Amb la finalitat d'establir els intervals de consolidació de la sèrie temporal, es defineix un buffer inicial.

\begin{definition}\label{def:model:buffer_buit}
  Definim buffer inicial o buffer buit com el buffer $B_{\emptyset} =
  (\emptyset,t_0, \delta_0, f)$, el qual
  conté una sèrie temporal buida, l'instant de temps inicial de
  consolidació, una durada que indica el pas de consolidació i un
  agregador d'atributs.
\end{definition}

A partir del buffer buit es poden conèixer tots els instants de temps de consolidació del buffer, els quals seran $t_0+k\delta, k\in\mathbb{N}$. 



\subsection{Disc}\label{sec:model:disc}

Un disc és un contenidor d'una sèrie temporal regular amb un nombre acotat de mesures. En arribar al nombre màxim de mesures permeses, cada cop que s'afegeix una mesura nova s'elimina la mesura mínima de la sèrie temporal.
Així doncs, un disc és semblant a una cua \emph{First In First Out} (FIFO), a on el primer d'arribar és el primer de sortir.  

\begin{definition}[Disc]
  Definim \emph{disc} com el tuple $(S,k)$, en el que $S$
  és una sèrie temporal i $k\in\mathbb{N}$ és el cardinal màxim de $S$.
\end{definition}

A l'inici, un disc no conté mesures però cal que estigui caracteritzat pel cardinal màxim. Amb aquesta finalitat es defineix un disc inicial.

\begin{definition}\label{def:model:disc_buit}
  Definim disc inicial o disc buit com el disc $D_{\emptyset} =
  (\emptyset,k)$, el qual conté una sèrie temporal buida i el cardinal
  màxim que podrà prendre $S$.
\end{definition}




\subsection{Disc resolució}\label{sec:model:disc_multiresolucio}

Un disc resolució és un disc amb buffer. En el buffer hi ha la part d'una sèrie temporal a regularitzar i en el disc hi ha l'altra part ja regularitzada, amb un nombre acotat de mesures. 

\begin{definition}[Disc resolució]
  Definim \emph{disc resolució} com el tuple $(B,D)$, en el que $B$
  és un buffer i $D$ és un disc.
\end{definition}
 
La definició de buffer buit (def.~\ref{def:model:buffer_buit}) i de disc buit (def.~\ref{def:model:disc_buit}) indueixen a una definició de disc resolució buit. 

\begin{definition}\label{def:model:disc_resolucio_buit}
  Definim disc resolució buit com el disc resolució $R_{\emptyset}
  = (B_{\emptyset},D_{\emptyset})$, el qual conté un buffer buit i un
  disc buit.
\end{definition}




\subsection{Base de dades multiresolució}\label{sec:model:bdstm}

Una base de dades multiresolució és un conjunt de discs resolució que comparteixen l'entrada de mesures, les quals provenen d'una mateixa sèrie temporal. La sèrie temporal queda regularitzada i distribuïda  en els diferents discs resolució amb resolucions diferents, tal com s'ha vist a la \autoref{fig:model:bdstm}


\begin{definition}[Base de dades multiresolució]
  Definim \emph{base de dades multiresolució} com el conjunt de discs resolució
  $M=\{R_0,\dotsc,R_d\}$.
\end{definition}

A partir de la definició de disc resolució buit (def.~\ref{def:model:disc_resolucio_buit}) és defineix la base de dades multiresolució buida. 
 
\begin{definition}\label{def:model:bd_multiresolucio_buit}
  Definim base de dades multiresolució buida com el conjunt de discs
  resolució buits
  $M_{\emptyset}=\{R_{0_{\emptyset}},\dotsc,R_{d_{\emptyset}\}}$.
\end{definition}

Normalment, en una base de dades multiresolució no hi ha dos discs
resolució amb la mateixa informació. És a dir, donats dos discs
resolució $R_a = (B_a, D_a)$ i $R_b = (B_b, D_b)$, 
els seus respectius buffers 
$B_a=(S_a,\tau_a,\delta_a,f_a)$ i
$B_b=(S_b,\tau_b,\delta_b,f_b)$ no tenen el mateix interval de
consolidació i agregador d'atributs: 
$\delta_a \neq \delta_b \wedge f_a \neq f_b$.









\subsection{Exemples}

\paragraph{Exemple 1}


S'observa que per tal de complir amb les propietats de les relacions, totes les sèries temporals dels buffers han de ser del mateix tipus, és a dir tenir la mateixa capçalera. El mateix succeeix amb les sèries temporals dels discs. (Vegeu els exemples de la secció \ref{par:model:exemple-relvalues} s'obre valors relació).

\begin{figure}[tp]
  \centering
  \begin{tabular}{|c|c|c|c|c|c|}
    \multicolumn{2}{c}{$M_1$} \\ \hline
    $S_B$  & $S_D$ & $\tau$ & $\delta$ & $k$ & $f$ \\ \hline
    $S_{B1}$ & $S_{D1}$ & 0 & 5  & 2 & mitjana  \\
    $S_{B2}$ & $S_{D2}$ & 0 & 10 & 4 & mitjana  \\ \hline
  \end{tabular}
  \caption{Taula d'una mtsdb independent}
  \label{fig:model:mtsdb:independent}
\end{figure}



\paragraph{Exemple 2}\todo{Compte! que no existeix el tipus relvar i potser no es pot definir una relació que contingui relvars (apuntadors). Cal pensar amb l'exemple 4 suprimit del model dels SGST}

\begin{figure}[tp]
  \centering
  \begin{tabular}{|c|c|c|c|c|c|}
    \multicolumn{2}{c}{$M_2$} \\ \hline
    $S_B$  & $S_D$ & $\tau$ & $\delta$ & $k$ & $f$ \\ \hline
    $S_{B1}$ & $S_{D1}$ & 0 & 5  & 2 & mitjana  \\
    $S_{D1}$ & $S_{D2}$ & 0 & 10 & 4 & mitjana  \\ \hline
  \end{tabular}
  \caption{Taula d'una mtsdb en cadena}
  \label{fig:model:mtsdb:cadena}
\end{figure}












\section{Model d'operacions}


\subsection{Estructurals}

\subsubsection{Buffer}


Abans de consolidar, però, cal que la sèrie temporal contingui mesures. L'operació \emph{afegeix} permet afegir una mesura a un buffer.

\begin{definition}
  L'operació \emph{afegeix} afegeix una mesura a la sèrie temporal del buffer:
  \[
  \text{afegeix}: \text{Buffer} \times \text{Mesura} \longrightarrow \text{Buffer}
  \]
  \[
   B \times m \longrightarrow B'= B \cup \{m\}
   \]
\end{definition}

Cada cop que s'afegeix una mesura a un buffer es pot comprovar si el buffer ja és consolidable mitjançant un predicat que ens retorna un booleà: cert o fals. 

\begin{definition}
  Un buffer és consolidable quan el temps d'una mesura de la sèrie temporal és més gran que el proper instant de temps de consolidació:
  \[
  \text{consolidable?}: \text{Buffer} \longrightarrow \text{Booleà}
  \]
  Sigui $B=(S,\tau,\delta,f)$ un buffer i $m=\max(S)$ la mesura màxima, $B$ és consolidable si i només si $T(m) \geq \tau+\delta$
\end{definition}



Propietats:

\begin{itemize}
\item Les mesures habitualment s'insereixen ordenades en el temps,
  sinó un cop duta a terme la consolidació les mesures inserides
  desordenades poden no ser tingudes en compte.
\end{itemize}



\subsubsection{Consolidació}

Quan un buffer és consolidable, es pot calcular una mesura de consolidació de la sèrie temporal per cada interval de temps consolidable. De manera simplificada, a cada consolidació només es té en compte l'interval que comença al darrer temps de consolidació del buffer. 

Sigui $B=(S,\tau,\delta,f)$ un buffer consolidable, la mesura de consolidació de $B$ en l'interval de temps $i=[\tau,\tau+\delta]$ és $m'=(v,\tau+\delta)$ on $m'=f(S,i)$ i $f$ és un agregador d'atributs. L'operació \emph{consolida} permet consolidar la sèrie temporal del buffer calculant-ne la mesura de consolidació.

\begin{definition}
  L'operació \emph{consolida} calcula la mesura de consolidació i treu
  les mesures consolidades de la sèrie temporal del buffer, en
  l'interval de consolidació actual:
  \[
  \text{consolida}: \text{Buffer} \longrightarrow \text{Buffer} \times \text{Mesura}
  \]
  \[
  B=(S,\tau,\delta,f) \longrightarrow B' \times m'
  \]
  \[
  B'= (S',\tau+\delta,\delta,f)
  \]
  \[
  S' = S(\tau+\delta,\infty)
  \]
  \[
  m' = f(S,[\tau,\tau+\delta]): f \text{ és un agregador d'atributs}
  \]
\end{definition}
\todo{$S'$ pot ser $S$ en el model, en tot cas fer una nota que en la implementació normalment es reduirà per no ocupar espai}





\subsubsection{Disc}

L'operació \emph{afegeix} permet afegir una mesura a un disc, controlant-ne el cardinal màxim.

\begin{definition}
  L'operació \emph{afegeix} afegeix una mesura a la sèrie temporal del disc:
  \[
  \text{afegeix}: \text{Disc} \times \text{Mesura} \longrightarrow \text{Disc}
  \]
  \[
  D=(S,k) \times m \longrightarrow D'= (S',k)
  \]
  \[
  S' =  
  \begin{cases}
      S\cup\{m\} &\text{si }  |S|<k\\
      (S-\{\min(S)\}) \cup \{m\} 
    \end{cases}  \
  \]
\end{definition}





\subsubsection{Disc resolució}


Per altra banda, les operacions dels buffers i dels discs estan relacionades amb les operacions dels discs Round Robin. 

L'operació \emph{afegeix} permet afegir una mesura a un disc Round Robin.

\begin{definition}
  L'operació \emph{afegeix} afegeix una mesura al buffer del disc Round Robin:
  \[
  \text{afegir}: \text{Disc Round Robin} \times \text{Mesura} \longrightarrow \text{Disc Round Robin}
  \]
  \[
  R=(B,D) \times m \longrightarrow R'= (B',D)
  \]
  \[
  B'= B \text{ afegeix } m
  \]
\end{definition}

Cada cop que s'afegeix una mesura a un disc Round Robin es pot comprovar si ja és consolidable. 

\begin{definition}
  Un disc Round Robin és consolidable quan el seu buffer és consolidable:
  \[
  \text{consolidable?}: \text{Disc Round Robin} \longrightarrow \text{Booleà}
  \]
  Sigui $R=(B,D)$ un disc Round Robin, $R$ és consolidable si i només
  si $B$ és consolidable.
\end{definition}


Quan un disc Round Robin és consolidable, es pot consolidar amb l'operació \emph{consolida}. 

\begin{definition}
  L'operació \emph{consolida} calcula una  mesura de consolidació del buffer, en
  l'interval de consolidació actual, i la desa al disc. 
  \[
  \text{consolida}: \text{Disc Round Robin} \longrightarrow \text{Disc Round Robin}
  \]
  \[
  R=(B,D) \longrightarrow R'= (B',D')
  \]
  \[
  B' \times m'= \text{ consolida } B 
  \]
  \[
  D'= D \text{ afegeix } m'
  \]
\end{definition}





\subsubsection{Base de dades multiresolució}



With reference to the operators, the add and consolidate in a multiresolution database are applied to every resolution disc it contains.


\begin{definition}
  Operator \emph{add} adds a measure to every resolution disc:
  \[
  \text{add}: \text{multiresolution database} \times \text{Measure}
  \longrightarrow \text{multiresolution database}
  \]
  \[
  M=\{R_0,\dotsc,R_d\} \times m \mapsto M' 
  \]
  \[  
  M'= \{ \forall R_i\in M: R_i \text{ add } m \}
  \]
\end{definition}


\begin{definition}
  Operator \emph{consolidate} consolidates the resolution discs that
  are ready to consolidate.
  \[
  \text{consolidate}: \text{multiresolution database} \longrightarrow
  \text{multiresolution database}
  \]
  \[
  M=\{R_0,\dotsc,R_d\} \mapsto M'
  \]
  \[
  M'= \big\{
  \forall R_i\in M: 
  \begin{cases}
    \text{ consolidate } R_i & \text{if } R_i \text{ ready to consolidate} \\
    R_i & \text{else }
  \end{cases}\big\}
  \]
\end{definition}





\subsection{Consultes}


Abstracció d'una BDSTM com a sèrie temporal

És possible treballar amb una BDSTM com si fos una sèrie temporal?

Com a consulta total: $\text{SerieTotal}(M)$
Com a consulta amb informació multiresolució: $\text{DiscSelecció}(M,\delta,f)$


\subsubsection{Selecció de disc}


Consulta la subsèrie de la BDSTM que té una resolució i atribut
determinat. 


\begin{definition}[DiscSelecció]
  \begin{gather*}
    \text{DiscSelecció}: M \times \delta \times f \longrightarrow S' = S_D: \\
    (S_B,S_D,\delta,\tau,k,f) \in M
\end{gather*}
\end{definition}



\subsubsection{Sèrie temporal total}



\begin{definition}[Sèrie temporal total]
  Sigui $M^*$ una base de dades multiresolució a on no hi ha $\delta$ repetits
  \begin{gather*}
    \text{SerieTotal}: M^* \longrightarrow S': \\
    \forall (S_{Bi},S_{Di},\delta_i,\tau_i,k_i,f_i) \in M : \\
    \delta_0 < \delta_1 < \delta_2 < \dots < \delta_d : \\
    S' = S_{D0} || S_{D1} || S_{D2} || \dotsb || S_{Dd}
\end{gather*}
\end{definition}

Prèviament es pot fer una selecció dels discs resolució que
comparteixin un determinat agregador d'atributs. \todo{També hi podria
  haver una operació estructural que sabés fusionar dos discs
  resolució}



L'operació de consulta de la sèrie temporal total també es pot aplicar
tenint en compte la representació.
\begin{definition}[Sèrie total amb representació]
  Sigui $M^*$ una base de dades multiresolució a on no hi ha $\delta$
  repetits i $r$ una representació
  \begin{gather*}
    \text{SerieTotal}: M^* \times r \longrightarrow S': \\
    \forall (S_{Bi},S_{Di},\delta_i,\tau_i,k_i,f_i) \in M : \\
    \delta_0 < \delta_1 < \delta_2 < \dots < \delta_d : \\
    S' = S_{D0} \cup^r S_{D1} \cup^r  S_{D2}  \cup^r \dotsb \cup^r  S_{Dd}
\end{gather*}
\end{definition}



\paragraph{Selecció de resolució}


Per a extreure una resolució determinada de la sèrie temporal
emmagatzemada a la base de dades multiresolució, es consulta la sèrie
temporal total i s'aplica una selecció de resolució
$\text{SerieTotal}(M)[i]^r$ a on $i$ és el conjunt d'instants de
temps.






\subsection{Operacions sobre l'estructura}

* Fusió d'esquemes de BDM
* Canvis d'esquemes de BDM (afegir multivaluat, canvi de delta d'un disc,canvi de la k,...)
* Estudiar Push o pull?




\subsubsection{Estudis en l'esquema}

Quin és el període de la sèrie temporal d'un disc?
  \begin{gather*}
    \text{periodeR}: R \longrightarrow \delta':\\
    \delta'=
    \begin{cases}
      \delta_r &\text{si } S_D \text{ regular o temps real amb } \delta_r\\
      \delta &\text{altrament}
    \end{cases}
  \end{gather*}
  
Quin és el l'interval temporal de la sèrie temporal d'un disc?
  \begin{gather*}
    \text{intervalR}: R \longrightarrow [T_0,T_f] :\\
    T_0 = T(\min(S_D)),     T_f = T(\max(S_D))
  \end{gather*}

Quin és el lapse temporal d'un disc?
  \begin{gather*}
    \text{lapseR}: R \longrightarrow [T_0,T_f] :\\
    T_0 = \tau - k\delta,  T_f = \tau
  \end{gather*}




Quin disc conté més resolució?
  \begin{gather*}
    \text{maxR}: R_1 \times R_2 \longrightarrow R_i' | d_i = \max(d_1,d_2) : \\
    d_1 = periodeR(R_1), d_2 = periodeR(R_1)
  \end{gather*}
  





\subsubsection{Canvis en l'esquema}


Redueix o augmenta la mida d'un disc
  \begin{gather*}
    \text{CanviaK}: R \times k' \longrightarrow R': \\
    R' = (S_B,S'_D,\delta,\tau,k',f) : \\
    k_d = |S_D|:\\
    S'_D = \begin{cases}
      S_D         & \text{si } k' \geq k_d   \\
      treuN(S_D,k_d-k')    & \text{altrament}
    \end{cases}, \\
    treuN: S \times n \mapsto S'=  
    \begin{cases}
      S                & \text{si } n=0   \\
      treuN(S - \{\min(S)\},n-1)  & \text{altrament}
    \end{cases}
\end{gather*}


Redueix o augmenta el pas de consolidació d'un disc (sense canviar la sèrie temporal emmagatzemada; ja s'anirà canviant quan es consolidin noves mesures)
  \begin{gather*}
    \text{Canvia}\delta: R \times \delta' \longrightarrow R': \\
    R' = (S_B,S_D,\delta',\tau,k,f)
  \end{gather*}


Redueix o augmenta alhora el pas de consolidació i la mida d'un disc
  \begin{gather*}
    \text{CanviaK}\delta: R \times k' \times \delta' \longrightarrow R': \\
    R' = (S_B,S_D',\delta',\tau,k',f): \\    
    t = \{ \tau-n\delta' | n\in\mathbb{N},n<k' \} \\
    S_D' = \text{seleccioResolucio}(S_D,t)
  \end{gather*}




Afegeix un multivalor per a emmagatzemar sèries temporals multivaluades
  \begin{gather*}
    \text{afegeixMultivalor}: R \longrightarrow R': \\
    R' = (S'_{B},S'_{D},\delta,\tau,k,f): \\
    S'_{B} = \text{map}(S_B,(t,v)\mapsto(t,v,\infty)), \\
    S'_{D} = \text{map}(S_D,(t,v)\mapsto(t,v,\infty))
  \end{gather*}





\subsubsection{Unió de multiresolució}

Cas típic:
Mesuro una sèrie temporal. Durant un temps emmagatzemo valors a una
base de dades i després els emmagatzemo a una altra base de dades. Al final vull unir les dues bases de dades.


Unió de dos discs resolució que tenen el mateix $\delta$ i $f$ és un
disc resolució que conté la unió de les sèries de cada un.  Sigui
$R_1^*=(S_{B1},S_{D1},\delta,\tau_1,k_1,f)$ i
$R_2^*=(S_{B2},S_{D2},\delta,\tau_2,k_2,f)$
  \begin{gather*}
    \text{unioR}: R_1^* \times R_2^* \longrightarrow R': \\
    R' = (S'_B,S'_D,\delta,\max(\tau_1,\tau_2),k_1+k_2,f), \\
    S_{Di}, S_{Dj} | R_i = \text{maxR}(R_1,R_2), j \neq i:  \\
    S'_B = \text{unio}(S_{B1},S_{B2})\\
    S'_D = \text{unio}^r(S_{Di},S_{Dj})
\end{gather*}

També es pot unir dos discs resolució amb diferent $\delta$ i $f$,
però llavors s'ha de determinar quins són els $\delta'$ i $f'$
resultants.


Com a relacions multiresolució, dues bases de dades multiresolució es
poden unir si no intersecten en les claus $(\delta,f)$.  En cas que
intersectin, podem definir la unió multiresolució com la unió que sap unir els discs resolució repetits.

\begin{gather*}
    \text{UnioM}: M_1 \times M_2 \longrightarrow M': \\
    K_1 = \{(delta_1,f_1) \in M_1\},K_2 = \{(delta_2,f_2) \in M_2\}, \\
    K_a = K_1 \cap K_2, K_u =  (K_1 \cup K_2) - K_f : \\
    M_{u1}'= seleccio(M_1, (delta,f) \in K_u)\\
    M_{u2}'= seleccio(M_2, (delta,f) \in K_u)\\
    M_1 = seleccio(M_1, (delta,f) \in K_f) \\
    M_2 = seleccio(M_2, (delta,f) \in K_f) \\
    M_u = \{\forall R_1\in M_1,R_2\in M_2: unioR(R_1,R_2) |
       (delta_1,f_1) = (delta_2,f_2) \} \\
    M' =  M_{a} \cup  M'_{1}  \cup  M'_{2}     
\end{gather*}






\subsubsection{Fusió de multiresolució}

Tinc una sèrie temporal en una base de dades, i una altra sèrie temporal en una altra base de dades. Vull emmagatzemar-les totes dues en una mateixa base de dades amb una sèrie temporal multivaluada.


Fusió de dos discs resolució que tenen el mateix $\delta$ i $f$.
Sigui $R_1^*=(S_{B1},S_{D1},\delta,\tau_1,k_1,f)$ i
$R_2^*=(S_{B2},S_{D2},\delta,\tau_2,k_2,f)$
  \begin{gather*}
    \text{FusioR}: R_1^* \times R_2^* \longrightarrow R': \\
    k^M=\max(k_1,k_2), R' = (S'_B,S'_D,\delta,\max(\tau_1,\tau_2),k^M,f), \\
    S^M_D = S_{Di} | k_i = k_M,  S^m_D = S_{Di} | k_i \neq k^M   : \\
    S'_B = \text{fusio}^r(S_{B1},S_{B2})\\
    S'_D = \text{semifusio}^r(S^M_{D},S^m_{D})
\end{gather*}


Fusió de dues bases de dades multiresolució

\begin{gather*}
    \text{FusioM}: M_1 \times M_2 \longrightarrow M': \\
    K_1 = \{(delta_1,f_1) \in M_1\},K_2 = \{(delta_2,f_2) \in M_2\}, \\
    K_f = K_1 \cap K_2, K_u =  (K_1 \cup K_2) - K_f : \\
    M_1'=\text{afegeixMultivalor}(seleccio(M_1, (delta,f) \in K_u))\\
    M_2'=\text{afegeixMultivalor*} (seleccio(M_2, (delta,f) \in K_u))\\
    \text{afegeixMultivalor*: com a }v_1\\
    M_{f1} = seleccio(M_1, (delta,f) \in K_f) \\
    M_{f2} = seleccio(M_2, (delta,f) \in K_f) \\
    M_f = \{\forall R_1\in M_1,R_2\in M_2: fusioR(R_1,R_2) |
       (delta_1,f_1) = (delta_2,f_2) \in K_u  \} \\
    M' =  M_{f} \cup  M'_{1}  \cup  M'_{2}     
\end{gather*}\todo{repensar perquè $|S_{D1} \cup S_{D2}| = k$?} 




\subsection{Com treure profit de les operacions dels SGSTM}

Temes que després es poden aprofitar a les implementacions

* No hi ha updates --> les sèries temporals no s'han de canviar

* Per exemple, vull calcular la mitjana de  BDSTM(a,b] si tinc un disc resolució amb $\delta=b-a$ i $f=$mitjana aquest seria l'adequat en comptes de calcular mitjana(SerieTotal(M)(a,b])

%??
% No obstant, la base de dades multiresolució conté informació sobre la
% resolució de les subsèries i per tant aquesta operació és susceptible
% d'implementar-se aprofitant aquesta informació.  A tall d'exemple es
% defineix una operació per extreure de la base de dades multiresolució
% una sèrie temporal regular amb període $T$:


% \begin{definition}[Selecció de resolució regular]
%   \begin{gather*}
%     \text{ResolucióRegular}: M^* \times T \times r \longrightarrow S'\\
%     \forall (S_{Bi},S_{Di},\delta_i,\tau_i,k_i,f_i) \in M : \\
%     d_i = T - \delta_i , \\
%     0 \geq d_0 > d_1 \dots > d_a, 0 < d_{a+1} < \dots < d_d: \\
%     S'' = S_{D0} || S_{D1} || \dotsb || S_{Da}  ||  S_{Da+1} || \dotsb || S_{Dd}, \\
%     S' = S''[i]^r: i = {t|0+nT,n\in\mathbb{N}}
%   \end{gather*}
% \end{definition}

% Nota: les operacions no són equivalents, l'operació $\text{SerieTotal}(M)[i]^r$ és molt més potent que la $\text{ResolucióRegular}(M,T)$.







%%% Local Variables:
%%% TeX-master: "main"
%%% End:
% LocalWords:  SGSTM


\section{Funcions d'agregació d'atributs}
\label{sec:model:interpolador}
\label{sec:model:agregador}
\glsaddsection{not:sgstm:fdef} %%%%secció de model
\glsaddsection{not:sgstm:f} %%%%secció de model


Les funcions d'agregació d'atributs s'utilitzen en la consolidació
dels buffers per tal de compactar certa informació de la sèrie
temporals. Sigui $S$ una sèrie temporal i $t_a$ i $t_b$ dos instants
de temps, una funció d'agregació d'atributs $f$ calcula una mesura que
resumeix la informació de $S$ en un interval de temps $i=[t_a,t_b]$:
\[
f: \text{sèrie temporal} \glsdisp{not:times}{\times}
\text{interval de temps} \longrightarrow \text{mesura}
\]
\[
f: S=\{m_0,\dotsc,m_k\} \times i=[t_a,t_b] \longrightarrow  m'
\]


Generalment, $m'$ resulta d'aplicar dues operacions a $S$: 
\begin{enumerate}
\item una selecció d'una subsèrie $S'$ segons l'interval de temps $i$,
  per exemple $S' = S[t_a,t_b]$
\item i una agregació en aquesta subsèrie $m' =
  \glssymbol{not:sgst:aggregate}(S',m_i,\glssymbol{not:sgst:fagg})$ on
  $\glssymbol{not:sgst:fagg}$ i $m_i$ són els atributs d'aquesta agregació.
\end{enumerate}



Atès que hi ha maneres diferents de resumir la informació d'una sèrie
temporal, cal plantejar diferents funcions d'agregació d'atributs. Per
exemple, es poden calcular estadístics de la sèrie temporal, com el
valor màxim o la mitjana, o aplicar operacions de processament digital
del senyal, com fan \textcite{zhang11}. A més a més, la representació
de les sèries temporals (v.~\autoref{sec:model:repr}) pot afectar els
càlculs que es fan en l'agregació o bé es pot aprofitar l'agregació
per a tractar algunes de les patologies de les sèries temporals
(v.~\autoref{sec:sgst:patologies}).  Així doncs, es poden definir una
enorme varietat de funcions d'agregació d'atributs i no hi ha cap
assumpció global que es pugui fer, cada usuari ha d'interpretar quina
combinació d'agregació i representació s'adiu més amb el fenomen
mesurat. Com a conseqüència, els \gls{SGSTM} han de donar llibertat
als usuaris per a definir funcions d'agregació d'atributs
personalitzades.


Com a mostra de com dissenyar funcions d'agregació d'atributs, a
continuació descrivim algunes interpretacions possibles que se'n poden
fer, tant pel que fa al càlcul de l'instant de temps resultant de la
consolidació com pel que fa al càlcul amb representació de sèries
temporals, i descrivim com utilitzar-les per a tractar i validar dades
desconegudes en les sèries temporals.



\subsection{Interpretació de l'agregació}


L'agregació d'una sèrie temporal en un interval resulta en una mesura
$m'=(t',v')$. Així per a definir les operacions d'agregació cal
interpretar quin ha de ser el temps resultant $t'=T(m')$ i el valor
resultant $v'=V(m')$.


Podem definir patrons generals de funcions d'agregació d'atributs que
indiquin quina informació o estadístic es resumeix de la sèrie
temporal, és a dir patrons generals que indiquin com s'ha de calcular
el valor resultant $V(m')$ independentment del mètode de representació
que es vulgui associar a la sèrie temporal.  Tot i així, el temps
resultant $T(m')$ no queda definit sinó que s'ha interpretar
coherentment per a cada cas particular de representació.


A continuació mostrem alguns exemples de patrons generals per a
calcular el valor resultant $V(m')$ que resumeix atributs d'una sèrie
temporal $S$ en un interval $i=[t_a,t_B]$. Sigui $S^r(t)$ la funció de
representació de la sèrie temporal i $t\in T$ els instants de temps
en el domini de temps:
\begin{itemize}
\item màxim: $S \times i \mapsto m'$ on $V(m') = \max_{\forall t \in
    [t_a,t_b]}(S^r(t))$. Resumeix $S$ amb el màxim dels valors de les
  mesures a l'interval $i$.
\item darrer: $S \times i \mapsto m'$ on $V(m') = S^r(t_b)$. Resumeix
  $S$ amb el valor del darrer instant de temps de l'interval $i$.

\item mitjana: $S \times i \mapsto m'$ on $V(m') = \frac{1}{t_b-t_a}
  \int_{t_a}^{t_b} S^r(t)dt$. Resumeix $S$ amb la \emph{mitjana de la
    funció} a l'interval $i$. \emph{Nota:} La mitjana d'una
  funció \parencite{weisstein:averagefunction}, $\bar f=f(x^*)$,
  utilitza la propietat $\int_a^b f(x)dx = f(x^*)(b-a)$ quan $f$ és
  contínua a $[a,b]$.
  % Explicació:
  % If $f$ is continuous on a closed interval $[a,b]$, then there is at least one number $x^*$ in $[a,b]$ such that
  % $$
  % \int_a^b f(x)dx = f(x^*)(b-a)
  % $$

  % The average value of the function ($\bar f$)  on this interval is then given by  $f(x^*)$.
  % $S(t)$ ha de ser contínua en l'interval $i$.
\end{itemize}




En aquests patrons d'atributs es treballa sobre una funció $S^r(t)$,
que a cada cas serà una funció de representació concreta i el temps
resultant $T(m')$ serà interpretat coherentment.  A més, per a cada
representació concreta també cal interpretar amb matemàtica discreta
el càlcul del valor resultant $V(m')$, atès que aquests patrons estan
definits com a problemes d'anàlisi numèric però a cada cas $S^r(t)$ és
una funció que prové d'un conjunt de mesures i podem expressar els
operadors segons el model de \gls{SGST} descrit amb àlgebra discreta
matemàtica.  A continuació s'exemplifiquen algunes interpretacions
possibles per al càlcul de $T(m')$ i de $V(m')$.





\subsubsection{Temps d'agregació resultant}


L'objectiu de les funcions d'agregació d'atributs és determinar un
instant de temps $T(m')$ i un valor $V(m')$. Aquest càlcul del temps i
del valor es pot realitzar al mateix temps però també pot ser
independent. Així, en principi el temps resultant serà independent i
valdrà $T(m')=t_b$ per estar d'acord amb l'operació de consolidació
del buffer i no causar desfasament de la subsèrie resolució
(v.~\autoref{def:sgstm:desdsamentR}), però en alguns casos aquest
$T(m')$ serà dependent del valor calculat o estarà subjecte a una
interpretació adient com és el cas per les representacions a l'apartat
següent.


Un exemple de funció d'agregació on temps i valor són dependents és
una que retorni la primera mesura que troba, $\operatorname{primera}:
S \times i \mapsto m'$ on $m' = \min(S[t_a,t_b))$ i llavors el temps
resultant pot ser $t_a \leq T(m') < t_b$. En aquest cas la sèrie
temporal consolidada resultant no és regular.


Un exemple de funció d'agregació on temps i valor són independents i
on la subsèrie resolució resultant és regular però amb desfasament, és
una funció que fa la mitjana amb un desfasament de 5 unitats de temps.
La funció d'agregació $\operatorname{mitjanad5}$ s'ha utilitzat
anteriorment a l'\autoref{ex:model:bdm-desfasaments}, ara podem
definir-la contextualitzada en les funcions d'agregació d'atributs,
$\operatorname{mitjanad5}: S \times i \mapsto m'$ on $V(m')=
\glssymbol{not:sgst:mitjanav}(S[t_a-5,t_b-5))$ i $T(m')=t_b-5$.

%De què pot servir la mitjanad5? per calcular mitjanes centrades? estem fent una interpolació sobre la representació centrada en l'interval de la sèrie temporal?


%mitjana mòbil, MM
%moving average, MA




\subsubsection{Agregació amb representació}

La varietat de funcions de representació per les sèries temporals
indueix a una varietat de funcions d'agregació per a un mateix patró
d'atributs. Per exemple, la funció d'agregació per l'atribut de màxim
dóna com a resultat valors diferents si es considera una representació
lineal o una representació a trossos constant. A continuació mostrem
la interpretació dels patrons definits anteriorment per a tres mètodes de
representació: \gls{pd}, \gls{dd} i \gls{zohe}.


\paragraph{Parcial discreta.}
En els casos parcials, $S^r(t)$ no és totalment contínua en el temps,
però es pot resoldre l'agregació del valor resultant assumint que el
domini de temps $T$ es correspon als instants de temps que hi ha a la
sèrie temporal, és a dir $T=\glssymbol{not:sgst:project}_{t}(S)$.  El
temps resultant es pot interpretar segon descrit a l'apartat anterior,
per exemple $T(m')=t_b$, i a més també es pot interpretar l'interval
de temps d'agregació $i=[t_a,t_b]$. Així sigui $S$ la sèrie original,
el resultat es pot calcular sobre una subsèrie amb interval obert
$S'=S(t_a,t_b)$, tancat $S'=S(t_a,t_b]$, semiobert $S'=S(t_a,t_b]$ o
$S'=S[t_a,t_b)$, o altres combinacions com per exemple tenir
desfasaments $S'=S[t_a-d,t_b-d]$ on $d$ és una durada.  Així de forma
general podem definir les funcions d'agregació d'atributs amb
representació \gls{pd}, $f^{\gls{pd}}\in f$, com $f^{\gls{pd}}: S
\times [t_a,t_b] \mapsto m'$ on $m'=(t_b,v')$ i el valor resultant
depèn del l'atribut que es vulgui resumir calculat en l'interval
$S'=S[t_a,t_b]$, a continuació es mostren els patrons d'exemple
interpretats segons aquest criteri.

\begin{definition}[Agregació parcial discreta]
  Sigui $S=\{m_0,\dotsc,m_k\}$ una sèrie temporal, $i=[t_a,t_b]$ un
  interval de temps i $S'=S[t_a,t_b]$ un interval de la sèrie
  temporal, les funcions d'agregació \gls{pd} per als atributs màxim,
  darrer i mitjana són:
  \begin{itemize}

  \item $\operatorname{m\grave{a}xim}^{\gls{pd}}$: $S \times i \mapsto
    m'$ on $V(m') = \max_{\forall m \in S'}(V(m))$ i
    $T(m')=t_b$. Aquest càlcul de $V(m')$ es correspon amb l'operació
    $\glssymbol{not:sgst:maxv}(S')$ dels \gls{SGST}.

\item $\operatorname{darrer}^{\gls{pd}}$: $S \times i \mapsto m'$ on $V(m') =
  V(\max(S'))$ i $T(m')=t_b$.

\item $\operatorname{mitjana}^{\gls{pd}}$: $S \times i \mapsto m'$ on $V(m') =
  \frac{1}{|S'|} \sum\limits_{\forall m\in S'} V(m)$ i $T(m')=t_b$. Aquest càlcul de
  $V(m')$ es correspon amb l'operació $\glssymbol{not:sgst:mitjanav}(S')$
  dels \gls{SGST}, és a dir amb calcular la mitjana aritmètica dels
  valors de les mesures.
\end{itemize}

\end{definition}



\paragraph{Delta de Dirac.} 
Per a les funcions d'agregació delta de Dirac interpretem el temps
d'agregació resultant centrat en l'interval $T(m')=\frac{t_b+t_a}{2}$,
tot i que també es podrien considerar altres interpretacions com per
exemple $T(m')=t_b$. Així de forma general podem definir les funcions
d'agregació d'atributs amb representació \gls{dd}, $f^{\gls{dd}}\in
f$, com $f^{\gls{dd}}: S \times [t_a,t_b] \mapsto m'$ on
$m'=(\frac{t_b+t_a}{2},v')$ i el valor resultant depèn del l'atribut
que es vulgui resumir calculat en l'interval temporal \gls{dd}
$S'=S[t_a,t_b]^{\gls{dd}}$.


\begin{definition}[Agregació delta de Dirac]
  Sigui $S=\{m_0,\dotsc,m_k\}$ una sèrie temporal, $i=[t_a,t_b]$ un
  interval de temps i $S'=S[t_a,t_b]^{\gls{dd}}$ un interval temporal
  de la sèrie temporal, les funcions d'agregació \gls{dd} per als
  atributs màxim, darrer i mitjana són:
\begin{itemize}
\item \glssymboldef{not:sgstm:maxdd}: $S
  \times i \mapsto m'$ on $V(m') = \max\big(0,\max_{\forall m \in
    S'}(V(m))\big)$ i $T(m')=\frac{t_b+t_a}{2}$. 

\item $\operatorname{darrer}^{\gls{dd}}$: $S \times i \mapsto m'$ on $V(m') =
  V(\max(S'))$ i $T(m')=\frac{t_b+t_a}{2}$.

\item \glssymboldef{not:sgstm:mitjanadd}: $S \times i \mapsto m'$ on
  $V(m') = \frac{1}{t_b-t_a}\sum\limits_{\forall m \in S'} V(m)$ i
  $T(m')=\frac{t_b+t_a}{2}$. Nota: la funció delta de Dirac té la
  propietat fonamental $\int \delta(t)dt = 1$. 
\end{itemize}
\end{definition}



\paragraph{Zero-order hold enrere.}
Per a les funcions d'agregació \gls{zohe} interpretem sempre el temps
d'agregació resultant com $T(m')=t_b$, atès que la representació
\gls{zohe} es defineix amb funcions graó contínues per
l'esquerra. Així de forma general podem definir les funcions
d'agregació d'atributs amb representació \gls{zohe},
$f^{\gls{zohe}}\in f$, com $f^{\gls{zohe}}: S \times [t_a,t_b] \mapsto
m'$ on $m'=(t_b,v')$ i el valor resultant depèn de l'atribut que es
vulgui resumir calculat en l'interval temporal \gls{zohe}
$S'=S[t_a,t_b]^{\gls{zohe}}$.
\begin{definition}[Agregació zero-order hold enrere]
  Sigui $S=\{m_0,\dotsc,m_k\}$ una sèrie temporal, $i=[t_a,t_b]$ un
  interval de temps i $S'=S[t_a,t_b]^{\gls{zohe}}$ un interval
  temporal de la sèrie temporal, les funcions d'agregació \gls{zohe}
  per als atributs màxim, darrer i mitjana són:
  \begin{itemize}
  \item \glssymboldef{not:sgstm:maxzohe}: $S \times i \mapsto m'$ on
    $V(m') = \max_{\forall m \in S'}(V(m))$ i $T(m')=t_b$.

  \item $\operatorname{darrer}^{\gls{zohe}}$: $S \times i \mapsto m'$
    on $V(m') = V(\max(S'))$ i $T(m')=t_b$.

  \item \glssymboldef{not:sgstm:meanzohe}: $S \times i \mapsto m'$ on
    $V(m') = \frac{1}{t_b-t_a} \big[ (T(o)-t_a)V(o) +
    \sum\limits_{\forall m \in S''}( T(m)-
    T(\glssymbol{not:sgst:prev}_S (m)) )V(m) \big]$; $o=\min(S')$;
    $S''= S' - \{o\}$; i $T(m')=t_b$.
% \[
%   \begin{split}
%   V(m')  = & \frac{1}{t_b-T_0} 
%   \big[ (T(o)-T_0)V(o) -( T(n)-T_f)V(n) \\
%     & {}+\sum\limits_{\forall m \in S''}( T(m)- T(\prev_S m) )V(m) \big]   
%    \end{split}
%   \]
% Nota: s'aplica la definició $0 \times \infty = 0$ tal com es fa habitualment a la teoria de mesura, \cite{wiki:extendedreal}.
  \end{itemize}
\end{definition}




Un cop definits els tres exemples de famílies d'agregacions, podem
comparar-les en funció de com resumeixen la informació de la sèrie
temporal. Reprenent la consolidació dels buffers
(v.~\autoref{sec:model:buffer}), l'interval de consolidació es
correspon a $t_a=\tau$ i $t_b=\tau+\delta$ i és consolidable quan
existeix una mesura $T(m)\geq\tau+\delta$. A la
\autoref{fig:sgstm:agg} dibuixem les mesures d'una sèrie temporal en
vermell, un interval de consolidació del buffer en línies blaves i la
mesura resultant de consolidació en verd.  Així, sigui
$S=\{\dotsc,m_{a-1},m_{a+1},\dotsc,m_{b-1},m_{b+1}, \ldots\}$ una
sèrie temporal on $ T(m_{a-1}) < t_a < T(m_{a+1}) < \dotsc <
T(m_{b-1}) < t_b < T(m_{b+1})$ i la consolidació del buffer que
calcula la mesura resultant $m'=f(S,[t_a,t_b])$ amb la funció
d'agregació d'atributs $f$.  Assumim $T(m')=t_b$ per simplificar el
dibuix, de manera general el càlcul del valor resultant és una
agregació a partir de les mesures:
\begin{itemize}
\item $\{m_{a+1},\dotsc,m_{b-1}\}$ en el cas de les agregacions \gls{pd}
\item $\{(t_a,0),(\ldots,0),m_{a+1},\dotsc,(\ldots,0),\dotsc,m_{b-1},(\ldots,0),(t_b,0)\}$ en el cas de les
  agregacions \gls{dd}
\item $\{m_{a+1},\dotsc,m_{b-1},m_{b+1}\}$ en el cas de les
  agregacions \gls{zohe}
\end{itemize}






\begin{figure}[tp]
  \centering
 
    \begin{tikzpicture}
        \begin{axis}[
          % width=10cm,
%          scale only axis, height=3cm,
          ymin = 0,
          xmax = 50,
          xmin = 20,
          yticklabels= {},
          xticklabels={,,,$t_a$,,$t_b$},
          ]
          \addplot[ycomb,blue] coordinates {
            (30,10)
            (40,10)
          }; 
          
          \addplot[only marks,mark=*,red] coordinates {
            (25,5)
            (32,2)
            (35,4)
            (38,6)
            (45,8)
          };
          
          \addplot[only marks,mark=*,green] coordinates {
            (40,4)
          };
          
          \node[above] at (axis cs:26,5) {$m_{a-1}$};
          \node[below] at (axis cs:32,2) {$m_{a+1}$};
          \node[below] at (axis cs:35,4) {$\ldots$};
          \node[above] at (axis cs:38,6) {$m_{b-1}$};
          \node[above] at (axis cs:45,8) {$m_{b+1}$};
          \node[right] at (axis cs:40,4) {$m'$};
        \end{axis}
      \end{tikzpicture}

    
  \caption{Agregació d'un interval de la sèrie temporal}
  \label{fig:sgstm:agg}
\end{figure}






En conclusió, per una banda alguns exemples mostrats de patrons tenen
una interpretació semblant per a les representacions particulars, en
certa manera només es diferencien en la interpretació de l'interval on
s'ha de resumir la sèrie temporal. Per exemple la diferència principal
en els atributs de màxim i darrer per a les tres representacions rau
en la $S'$, tot i que en el cas del $\glssymbol{not:sgstm:maxdd}$
l'agregació a més ha de tenir en compte que en la funció de
representació hi ha valors intermitjos que valen zero.

Per altra banda, altres exemples són molt diferents, com és el cas de
l'atribut mitjana. En aquest cas, per a la \gls{pd} i la \gls{dd} és
el càlcul de la suma dels valors tot i que dividit per $|S'|$ en la
primera i per $t_b-t_a$ en la segona, i és una mitjana ponderada per
les durades de temps en la \gls{zohe}.  En general, es pot dissenyar
qualsevol operació d'agregació, com per exemple calcular la mitjana
aritmètica de l'interval \gls{zohe} amb
$\glssymbol{not:sgst:mitjanav}(S[t_a,t_b]^{\gls{zohe}})$, tot i que
llavors cal interpretar quin patró d'atribut li correspon o altrament
aquesta operació d'agregació pot no tenir sentit real.


\textcite{rrdtool} utilitza a RRDtool una funció d'agregació semblant
a la $\glssymboldef{not:sgstm:meanzohe}$ per a resumir la informació
conservant el comptatge total si les sèries temporals mesurades tenen
trets semàntics de comptador i són en forma de velocitat; així aquesta
agregació es pot veure com una consolidació que conserva l'àrea del
senyal original. 





% Notes:

% * Quan una sèrie temporal és regular, l'intepolador mitjana aritmètica i l'interpolador àrea valen el mateix en l'interval $(T_o,n\delta]$.




\subsection{Tractament i validació de dades}


En les patologies de les sèries temporals
(v.~\autoref{sec:sgst:patologies}) s'ha descrit el problema de les
dades desconegudes, les funcions d'agregació d'atributs poden cooperar
en els processos de validació i tractament de dades. Així, les
funcions d'agregació poden marcar o tractar dades desconegudes:
\begin{itemize}
\item Marcar dades com a desconegudes. És a dir determinar quan el
  resultat d'una agregació ha de ser desconegut perquè la sèrie
  temporal avaluada pateix una de les causes descrites: valors fora de
  rang, temps de termini excedit, etc.

\item Tractar dades que són desconegudes, ja sigui perquè d'origen són
  desconegudes o perquè les hem marcat abans com a desconegudes.
  Si una funció d'agregació rep valors que són desconeguts, des d'un
  punt de vista estricte el resultat de l'agregació ha de ser
  desconegut. No obstant això, es poden aplicar operacions que tractin
  aquest valors desconeguts: reconstrucció del senyal, ignorar els
  valors desconeguts, etc.
\end{itemize}

 
A continuació definim el procés que fan les funcions d'agregació per a
ambdós casos. Com a exemple de domini pels valors utilitzem els
nombres reals projectius \glssymbol{not:R*}, en els quals representem
el valor desconegut mitjançant l'element infinit ($\infty$), segons la
\autoref{def:model:mesura_valor_indefinit} de mesura de valor
indefinit. Això no obstant, el domini de valors podria tenir diversos
valors per a marcar diferents casos de dades desconegudes.

\paragraph{Tractament de dades desconegudes.}
Una funció d'agregació d'atributs $f^u \in f$ que tracti dades
desconegudes és aquella que pot calcular un resultat quan la sèrie
temporal original conté valors desconeguts
\[
f^u: S \times i \mapsto m' \text{ on } \exists m \in S: V(m)=\infty
\]

Per exemple, podem redefinir el patró de la funció d'agregació mitjana
en una $\operatorname{mitjana}^{u}$ que sigui capaç de tractar valors
desconeguts conservant l'àrea coneguda, és a dir, l'àrea total
coneguda quedarà escampada en l'interval de consolidació.
\begin{gather*}
  \operatorname{mitjana}^{cu}: S \times i \mapsto m' \text{ on }\\
  V(m') = \frac{1}{t_b-t_a}\int_{t_a}^{t_b} S^u(t)dt \text{ i }
  S^u(t)=
  \begin{cases}
    0 &\text{si }  S^r(t)=\infty\\
    S^r(t) & \text{altrament }
  \end{cases}
\end{gather*}


\paragraph{Marcatge de dades desconegudes.}
Una funció d'agregació d'atributs $f^{mu} \in f$ que marqui
dades desconegudes és aquella que pot retornar una mesura de valor
indefinit com a resultat
\[
f^{mu}: S \times i \mapsto m' \text{ on } V(m')\in \glssymbol{not:R*}
\]


Per exemple, podem definir un patró de funció d'agregació d'atribut
màxim que retorni valor desconegut
si hi ha un a mesura amb el valor més gran que 2; és a dir establim un
límit superior de 2 (L2). 
\begin{gather*}
  \operatorname{m\grave{a}xim}^{L2}: S \times i \mapsto m' \text{ on }\\
  m' = \begin{cases}
    (T(m''),\infty) &\text{si }  \exists m\in S[t_a,t_b]: V(m)>2\\
    m'' & \text{altrament }
  \end{cases} \text{ i } m''= \operatorname{m\grave{a}xim}(S,i)
\end{gather*}






%\todo{}
%hauria d'aparèixer algun exemple on es resolgués inframostreig. Potser també algun exemple on es veiés on els agregadors solucionen el problema de l'ultramostreig.



%Per exemple definim un termini, si les dades estan més espaiades que 2 es marca com a desconeguda
% Sigui $S=\{m_0,\ldots,m_k\}$ una sèrie temporal i $H$ un termini de temps, una mesura $m_i=(v_i,t_i)\in S$ és desconeguda si, donada la mesura anterior $m_{i-1}=(v_{i-1},t_{i-1})$, $t_i - t_{i-1} > H$.    





% Sigui $S=\{m_0,\ldots,m_k\}$ una sèrie temporal, $f$ un interpolador, $i=[T_0,T_f]$ un interval de temps i $\alpha$ un llindar, la mesura de consolidació calculada per l'interpolador $f$ és desconeguda ssi  
% \[
% \frac{t_d }{T_f - T_0} > \alpha :
% \]
% \[
% :t_d = t_{d0} + t_{df} + \sum\limits_{i=1}^{k-1}(t_i-t_{i-1}) : v_k = 'desconegut':
% \]
% \[
% : t_{d0} = \left\{\begin{array}{l} t_0-T_0 \text{ si } v_0 = 'desconegut' \\ 0\end{array}\right. ,
% t_{df} = \left\{\begin{array}{l} T_f-t_{k-1} \text{ si } v_k = 'desconegut' \\ 0\end{array}\right. :
% \]
% \[
% :k=|S|-1,(v_k,t_k)=m_k\in S' :S'= S_{T_0:T_f} \cup \{min(S_{T_f:\infty})\}
% \]



%operacions amb nan de octave i matlab
%http://biosig-consulting.com/matlab/NaN/
% The NaN-toolbox v2.0: A statistics and machine learning toolbox for Octave and Matlab
% for data with and w/o MISSING VALUES encoded as NaN's.











%%% Local Variables:
%%% TeX-master: "main"
%%% End:
% LocalWords: buffer buffers ZOHE









\part{Consideracions i reflexions sobre els models}
%-- POST-MODELS --

\chapter{Introducció a variacions del model}




\todo{estructura?}

Potser estructurar aquesta part amb:

1. SGSTM per a dispositius on l'emmagatzematge reduït i afitat és important
2. SGSTM per a emmagatzematges massius on calen consultes i visualitzacions ràpides
3. Qualitat dels SGSTM, teoria de la informació



Per a 2. i 3. ens farà falta raonar sobre un funció de multiresolució que presentem al capítol tal.







\todo{falta raonar sobre variacions importants}


El model de \gls{SGSTM} que hem definit té l'objectiu de ser genèric i senzill, és a dir sense entrar en detalls particulars o en aspectes que potser són útils a la pràctica però que compliquen l'estructura del model.


Particularment, hi ha una generalització en els buffers que a l'hora d'implementar-se pot resultar estranya. És el fet que de forma genèrica hem definit que en els buffers s'acumula tota la sèrie temporal original independentment en cadascun dels buffers. Això és útil en el model perquè permet definir-ho de forma molt abstracta i abastar diferents possibles variacions, però és bo d'explorar aquestes possibles variacions que podrà tenir en les implementacions.

\begin{itemize}
\item La funció d'agregació d'atributs treballa amb orientació a flux

\item Les resolucions estan encadenades

\item S'emmagatzema tota la sèrie temporal original en un \gls{SGST} i el
\gls{SGSTM} treballa sobre aquestes mesures, és a dir que realment els
buffers no les emmagatzemen sinó que seleccionen les que necessiten a
cada moment. En aquest cas pensem en \gls{SGST} d'emmagatzematge
massiu com els descrits a l'\autoref{art:massius}. A
la~\autoref{sec:multiresolucio:dual} explorarem l'estructura i les
aplicacions de sistemes \gls{SGST} i \gls{SGSTM} conjunts.

\end{itemize}

Per altra banda a la~\autoref{sec:multiresolucio:funcio} observarem com els \gls{SGSTM} es poden definir com una funció sobre una sèrie temporal. Aquests casos ja operaran directament sobre un \gls{SGST} amb tota la sèrie temporal emmagatzemada i per tant els buffers no emmagatzemaran temporalment sinó que treballaran sobre mesures ja emmagatzemades. 











\section{Estructures interessants}



\subsubsection{Discos enllaçats}


Aquesta pot ser útil per al filó distribuït. 

* Si quan s'entra una mesura aquesta ha d'anar a totes les subsèries resolució, aleshores el filó distribuït no té sentit perquè igualment han d'arribar les mesures noves al node central.

* Però si hi ha discos enllaçats aleshores només cal anar transferint la subsèrie consolidada cap amunt i llavors un cop es llança una consulta aquesta decideix si té prou resolució al node on és o ha d'anar a buscar-ne d'un altre. Aquesta decissió de la consulta pot fer-se a mà, consultes amb SerieDisc()  indiquen quin disc s'ha d'anar a buscar, per tant l'usuari tria quin vol i sap si s'haurà d'anar a buscar a fora. 


\paragraph{Exemple 2}

Les taules es poden veure a la \autoref{fig:model:mtsdb:cadena} a on la base de dades multiresolució és la vista-relació 
\begin{verbatim}
M_2 = ( ((M_2' RENAME S'_B AS S') JOIN (M^{series}_2 RENAME S AS S_B)) RENAME S'_D AS S') JOIN (M^{series}_2 RENAME S AS S_D)
\end{verbatim}



\begin{figure}[tp]
  \centering
  \begin{tabular}{|c|c|c|c|c|c|}
    \multicolumn{2}{c}{$M'_2$} \\ \hline
    $S'_B$  & $S'_D$ & $\tau$ & $\delta$ & $k$ & $f$ \\ \hline
    $S_{B1}$ & $S_{D1}$ & 45 & 5  & 2 & mitjana  \\
    $S_{D1}$ & $S_{D2}$ & 40 & 10 & 4 & mitjana  \\ \hline
  \end{tabular}\qquad
  \begin{tabular}{|c|c|c|}
    \multicolumn{3}{c}{$M^{series}_{2}$} \\ \hline
    \multirow{2}{*}{$S'$}  &  \multicolumn{2}{c|}{$S$} \\ \cline{2-3}
    & $t$      & $v$  \\ \hline
    \multirow{3}{*}{$S_{B1}$} & 46 & 0 \\ 
    & 48 & 0 \\ 
    & 49 & 0 \\ \hline
    \multirow{2}{*}{$S_{D1}$} & 40 & 0 \\ 
    & 45 & 0 \\ \hline
    \multirow{4}{*}{$S_{D2}$} & 10 & 0 \\ 
    & 20 & 0 \\ 
    & 30 & 0 \\ 
    & 40 & 0 \\ \hline
  \end{tabular}
  \caption{Taula d'una mtsdb en cadena}
  \label{fig:model:mtsdb:cadena}
\end{figure}

\todo{per a fer l'exemple falta conèixer els operadors estructurals}

\todo{falta definir qui són els buffer d'entrada de mesures}
definir una $M^{in}_2$.





Respecte a l'estructura general, l'estructura enllaçada restringeix
els períodes de consolidació de les sèries temporals: aquests són
múltiples dels discs anteriors.



\subsubsection{Data stream}



Base de dades multiresolució a on les sèries temporals dels buffers
només tenen una mesura; és a dir tenen cardinal afitat a 1.


Per a orientar a streams els buffers s'han de canviar els operadors
d'afegir i consolidar:

Es canvia l'operador d'afegir per tal que incorpori el càlcul orientat
a stream cada cop:
\[
\text{addB}^{\text{stream}}: B \times m \longrightarrow B' =
(streamB(S,m),\tau,\delta,f)
\]

Es canvia l'operador de consolidar per tal que reconegui la sèrie
temporal del buffer com a consolidada amb stream.

  \[
  \text{consolidaB}^{\text{stream}}: B \longrightarrow B' \times m'
  \]
  \[
  B'= (S',\tau+\delta,\delta,f)
  \]
  \[
  S' = S(\tau+\delta,\infty)
  \]
  \[
  m' \in S(\tau,\tau+\delta] 
  \]


Per a orientar a streams els buffers es defineix un nou operador
\[
\text{streamB}: S \times m \longrightarrow S' = \{f^{\text{stream}}(m_o,m)\}
\]
\[
m_o \in S
\]
\[
f^{\text{stream}} \text{ és un agregador d'atributs orientat a streams}
\]
 

Aleshores els agregadors d'atributs funcionen orientats a stream;
nota: no tots els agregadors d'atributs es poden definir com a
streams.


Per exemple l'interpolador mitjana orientat a stream:

\[
\text{mitjana}^{\text{stream}}: m_o \times m_n \longrightarrow m' = (T(m_n),v')
\]
\[
\text{a on } v' = (V(m_0) + V^1(m_n), V^2(m_n) + 1 )
\]






\subsubsection{Compartició de buffers}


Les diferents $f$ amb mateix $\delta$ poden compartir buffer.


Tenir un buffer únic per a totes les subsèries i que no s'esborrin mesures.

\subsubsection{Lapses de buffer o no lapses}

Podem tenir esquemes de multiresolució a on les diferents subsèries
resolució coincideixin en els temps recents o a on no coincideixin:
les subsèries més velles acabin on comencen les noves de manera
semblant a l'estructura de buffers enllaçats.

La primera opció pot servir per quan hi ha moltes dades tenir diferents resums preparats per a ser visualitzats, així permet triar ràpidament entre diferents zooms de les dades.

La segona opció serveix per aprofitar al màxim la resolució i l'espai d'emmagatzematge, sense que cap subsèrie desi informació per al mateix interval de temps. Així permet conservar una sèrie temporal al llarg del seu temps amb diferents resolucions. També pot servir per usar la informació d'altres buffers i no haver de repetir emmagatzematge de buffers.









\subsubsection{Push i pull}

Estudiar Push o pull aplicada als SGSTM i quines implicacions pot tenir.


\subsubsection{Rellotge}


\todo{atenció que no s'ha parlat enlloc del rellotge en els SGBDM? s'hauria de dir que en principi no es diu res sobre el rellotge i que si segueix l'esquema de consolidació proposat les mateixes mesures van entrant ordenades i van marcant el pas del temps}


\todo{alerta! el model sgstm s'ha fet pensant en els instants de consolidació periòdics. Què passa quan no ho són? }
per exemple event trigerred (només emmatgatzemem informació quan creiem que és interessant). Aleshores potser són casos molt especials i no es pot dibuixar l'esquema? Potser posar tot això com a comentari a la secció d'Altres Estructures: En el model de \gls{SGSTM} s'ha considerat que la consolidació es feia periòdicament però es podria fer quan es cregués oportú, aleshores no es pot preveure quin esquema de multiresolució hi haurà; són casos que requereixen un estudi més profund. Etc. \todo{potser cap a treball futur?}


Parlar de com ha de ser el rellotge en un SGSTM. Pot ser:

* intern (pull): el SGSTM té un rellotge que li va marcant cada quan consolidar; per tant és ell que va decidint quines mesures s'agafaven i quines no i com. En diem de tipus pull perquè d'alguna forma és el SGSTM qui tiba les mesures; en aquest cas fins i tot es podrien contemplar escenaris a on el SGSTM reclamés mesures quan li semblés que no disposa de prou informació.

* extern (push): el SGSTM no té rellotge sinó que les mesures que entren ja tenen un temps i s'assumeix que entren ordenades en el temps; aleshores d'aquesta entrada de mesures s'extreu el rellotge; és a dir que es van sabent els instants de consolidació a partir de les mesures. Això pot provocar un cert decalatge del temps del SGSTM amb el del rellotge real; ja que el primer només canvia quan té mesures noves. En diem de tipus push perquè les mesures són les que controlen el procés (bé el sistema de monitoratge extern). 

* discret: tipus la cpu marca el rellotge; en aquest cas no hi ha un rellotge real sinó que es compten esdeveniments i el rellotge es crea relativament a aquest comptatge. Per exemple és el cas de sistemes encapsulats a on no hi ha RTC i el clock diu cada quan s'ha de consolidar (en aquest moment es consoliden les mesures de què es disposi, les quals potser tampoc tenen temps i per tant el rellotge del SGSTM serà relatiu) o bé un altre exemple és quan la consolidació d'una subsèrie resolució depèn d'una altra subsèrie (p.ex. es consolida cada cop que l'altre ha expulsat quatre mesures).




\todo{}
* orientat a esdeveniments (event triggered)
* amb rellotge (time triggered)

* en cas de comptadors digitals, aquests cada cop que incrementen el valor poden fer un push a la base de dades. En comptadors analògics no ho poden fer perquè van incrementant contínuament i no discretament.







%%% Local Variables: 
%%% mode: latex
%%% TeX-master: "main"
%%% End: 







\chapter{Esquemes de multiresolució}

\todo{}


Definim una funció de multiresolució com una consulta sobre una sèrie
temporal que ens retorna una nova sèrie temporal resultant
d'aplicar-li un esquema de multiresolució. 

Aquesta funció de multiresolució ens permetrà:

\begin{itemize}
\item Plantejar problemes on veiem la multiresolució com una consulta
  sobre una sèrie temporal; és a dir com una operació per calcular
  parelles d'agregació d'atributs i resolucions temporals per a una
  sèrie temporal.
\item Oferir sistemes duals de \glspl{SGSTM} i \glspl{SGST} amb operacions
  de consulta multiresolució.
\item Estudiar implementacions per a la consulta de multiresolució, p.ex. para\l.lelisme. (vegeu secció implementació \todo{})
\item Estudiar la teoria de la informació per a l'esquema de multiresolució
\end{itemize}


\section{Funció de multiresolució}

En el model de \gls{SGSTM} hem definit un model de dades de \gls{SGBD}
per a gestionar sèries temporals multiresolució. Com a \gls{SGBD},
aquest model té una estructura i per tant emmagatzema informació d'una
sèrie temporal en una forma determinada: la de multiresolució.  La
definició com a \gls{SGBD} té com a objectius l'emmagatzematge
compacte de les dades i la selecció de la informació ja preparada per
a consultes posteriors. 

Així, aquest model té capacitats de computació
sincronitzada o en línia (\emph{online}) amb el temps i té
característiques dels sistemes que tracten fluxos de dades (\emph{data
  stream}); és a dir dades que s'estan adquirint contínuament i cal
anar computant al mateix temps que es van adquirint. Això no treu,
però, que de manera més simplificada també es pugui treballar amb un
\gls{SGSTM} en temps diferit (\emph{offline}), és a dir que
s'emmagatzemin les dades adquirides i en el moment que es vulgui
aplicar-hi la consolidació.




Això no obstant, podem simplificar el problema de càlcul de
multiresolució en temps diferit com una consulta en un \gls{SGST} de
transformació d'una sèrie temporal a una nova sèrie temporal.

És a dir, sigui $S$ una sèrie temporal, $M$ una sèrie temporal
multiresolució i $e = \{ (\delta_0,f_0,\tau_0,k_0), \ldots,
(\delta_d,f_d,\tau_d,k_d)\}$ els paràmetres de l'esquema de
multiresolució de $M$. Afegim totes les mesures de la sèrie temporal a
la multiresolució, $\forall m \in S:
M=\glssymbol{addM}(M,m)$\todo{matemàticament és correcte recursivitat
  M=f(M)?}, i la consolidem, $M=\glssymbol{consolidaM}(M')$ fins que
$M$ no sigui consolidable. Consultem la sèrie temporal multiresolució
amb les dues consultes bàsiques, les quals retornen sèries temporals,
$S'=\glssymbol{not:sgstm:serietotal}(M')$ i $S_{\delta
  f}'\glssymbol{not:sgstm:seriedisc}(M',\delta,f)$ on $\delta$ i $f$
són dos paràmetres de l'esquema de multiresolució de $M$ que van
associats amb els altres dos corresponents $\tau$ i $k$.



Plantegem les funcions de transformació de la sèrie temporal original
a les consultades. És a dir, les funcions que anomenem
$\glssymbol{not:sgstm:dmap}$ i $\glssymbol{not:sgstm:multiresolucio}$
i que ens permeten calcular:

\[
\glssymbol{not:sgstm:dmap}: S \times \delta \times f \times \tau \times k \longrightarrow
S'_{\delta f}
\]


\[
 \glssymbol{not:sgstm:multiresolucio}: S \times e  \longrightarrow S'
\]



Definim la consulta de selecció de disc dels \gls{SGSTM} a partir del
mapatge dels \gls{SGST} de manera que, en computació per temps
diferit, són equivalents
\[
\glssymbol{not:sgstm:seriedisc}(M',\delta,f) \equiv
\glssymboldef{not:sgstm:dmap}(S,\delta,f,\tau,k)
\]


\begin{definition}[mapa de \glssymbol{not:sgstm:seriedisc}]
  Sigui $S$ una sèrie temporal, $M$ una sèrie temporal multiresolució
  amb esquema $e$ i $(\delta,f,\tau,k)\in e$ els paràmetres de
  multiresolució d'una subsèrie resolució. L'expressió de
  $\glssymbol{not:sgstm:seriedisc}(M,\delta,f)$ com a mapa d'una sèrie
  temporal és $\glssymboldef{not:sgstm:dmap}(S,\delta,f,\tau,k)=
  \glssymbol{not:sgst:map}(S_I,\glssymbol{not:sgst:fmap})$ on
  \[
  \glssymbol{not:sgst:fmap}: m_i\mapsto f(S, [T(m_i)-\delta,T(m_i)]),
  \]
  \[
  S_I = \{ (t,\infty) | t\in T_I  \},\;  t_M = T(\max(S)),
  \]
  \[
  T_I = \{ t_I = \tau+n\delta | n\in\glssymbol{not:Z}, t_M - k\delta <
  t_I \leq t_M \}.
  \]
\end{definition}



\begin{example}
  \label{ex:multiresolucio:dmap}
  Sigui la sèrie temporal $S=\{(1,0),(3,1),(6,0),(10,1)\}$ i els
  paràmetres de multiresolució
  $((\delta,5),(f,\glssymbol{not:sgstm:maxdd}),(\tau,0),(k,2))$.  El
  mapa de \glssymbol{not:sgstm:seriedisc} és una sèrie temporal $S'=
  \glssymboldef{not:sgstm:dmap}(S,5,0,\glssymbol{not:sgstm:maxdd},2)$
  on $S'=\{(5,1),(10,1)\}$. Expressem el càlcul pas a pas, a la
  \autoref{fig:multiresolucio:dmap} es visualitzen en taula les sèries
  temporals corresponents:
  \begin{enumerate}
  \item El primer pas és obtenir els instants de temps que
    s'emmagatzemarien al disc d'una sèrie temporal
    multiresolució. Així, els instants de consolidació possibles són
    $T_I'=\{\tau+n\delta|n\in\glssymbol{not:Z}\}=
    \{\ldots,-5,0,5,10,15,\ldots\}$. Però un cop consolidat el disc
    només hi haurà els $k=2$ més recents abans de $t_M=T(\max(S))=10$,
    és a dir $T_I=\{t_I'\in T_I'|t_M - k\delta < t_I \leq
    t_M\}=\{5,10\}$.

  \item El segon pas és obtenir a partir de $T_I$ la sèrie temporal
    $S_I$ que es correspon amb la sèrie temporal que s'inicialitzaria
    al disc encara amb valors desconeguts,
    $S_I=\{(5,\infty),(10,\infty)\}$.



  \item El tercer pas és calcular la funció d'agregació a $S$ per a
    cada intervals de consolidació del disc de la forma
    $[T(m_i)-\delta,T(m_i)]$ on $m_i\in S_i$, és a dir $f(S,[0,5])$ i
    $f(S,[5,10])$. A tal efecte utilitzem el mapa sobre $S_I$ per a
    calcular la sèrie temporal resultant $S'=\{ (5,f(S,[0,5])),
    (10,f(S,[5,10])) \}$.

    Podríem calcular un pas entremig que es correspon amb les sèries
    temporals que hi hauria en el buffer abans de cada instant de
    consolidació. Així, per a cada $T(m_i)$ hi hauria la sèrie
    temporal $S[T(m_i)-\delta,T(m_i)]$, és a dir $S_B=\{
    (5,S[0,5],(10,S[5,10]) \}$.
  \end{enumerate}


  


\begin{figure}[tp]
  \centering
  \begin{tabular}[c]{|c|c|}
    \multicolumn{2}{c}{$S$} \\ \hline
    $t$  & $v$ \\ \hline
    1  & 0 \\
    3  & 1 \\
    6  & 0 \\
    10  & 1 \\ \hline
  \end{tabular} \qquad
  \begin{tabular}[c]{|c|c|}
    \multicolumn{2}{c}{$S_I$} \\ \hline
    $t$  & $v$ \\ \hline
    5  & $\infty$ \\
    10  & $\infty$ \\ \hline
  \end{tabular} \qquad
  \begin{tabular}[c]{|c|c|}
    \multicolumn{2}{c}{$S_B$} \\ \hline
    $t$  & $v$ \\ \hline
    5  & \begin{tabular}[c]{|c|c|}\hline $t$  & $v$ \\ \hline 1&0\\ 3&0 \\\hline  \end{tabular} \\\hline
    10  & \begin{tabular}[c]{|c|c|}\hline $t$  & $v$ \\ \hline 6&0\\ 10&1 \\\hline  \end{tabular} \\ \hline
  \end{tabular} \qquad
 \begin{tabular}[c]{|c|c|}
    \multicolumn{2}{c}{$S'$} \\ \hline
    $t$  & $v$ \\ \hline
    5  & 1 \\
    10  & 1\\ \hline
  \end{tabular}
  \caption{Taules de les sèries temporals per l'operació de mapa de  \glssymbol{not:sgstm:seriedisc}}
  \label{fig:multiresolucio:dmap}
\end{figure}
 
\end{example}





Definim la consulta de sèrie temporal total dels \gls{SGSTM} a partir
del plegament dels \gls{SGST} de manera que, en computació per temps
diferit, són equivalents
\[
\glssymbol{not:sgstm:serietotal}(M') \equiv \glssymbol{not:sgstm:multiresolucio}(S,e)
\]

\begin{definition}[plec de \glssymbol{not:sgstm:serietotal}]
  Sigui $S$ una sèrie temporal i $M$ una sèrie temporal multiresolució
  amb esquema $e = \{ (\delta_0,f_0,\tau_0,k_0), \ldots,
  (\delta_d,f_d,\tau_d,k_d)\}$, el qual es pot observar com una sèrie
  temporal multivaluada.  L'expressió de
  $\glssymbol{not:sgstm:serietotal}(M)$ com a plec d'una sèrie
  temporal és $\glssymboldef{not:sgstm:multiresolucio}(S,e)=
  \glssymbol{not:sgst:ofold}(e,\{\},\glssymbol{not:sgst:ffold},\min)$
  on $\glssymbol{not:sgst:ffold}: S_i \times (\delta_c,f_c,\tau_c,k_c)
  \mapsto S_i ||
  \glssymbol{not:sgstm:dmap}(S,\delta_c,f_c,\tau_c,k_c)$.

  Així, el plec de \glssymbol{not:sgstm:serietotal} és la concatenació de
  tots els \glssymbol{not:sgstm:dmap} possibles per l'esquema $e$
  ordenats per $\delta$, assumint que $e$ no conté $\delta$ repetits.
\end{definition}



En resum: en temps diferit, s'insereixen les mateixes mesures a un
\gls{SGST} i a un \gls{SGSTM}. Per una banda es consolida el
\gls{SGSTM} i s'obté la sèrie total i per altra banda es consulta la
multiresolució en el \gls{SGST}. Aleshores s'obté la mateixa sèrie
temporal.


\begin{example}
  Sigui la sèrie temporal $S=\{(1,0),(3,1),(6,0),(10,1)\}$ i l'esquema
  de multiresolució
  $e=\{\{(\delta,5),(f,\glssymbol{not:sgstm:maxdd}),(\tau,0),(k,2)\},
  \{(\delta,2),(f,\glssymbol{not:sgstm:maxdd}),(\tau,0),(k,3)\}\}$.
  El plec de $\glssymbol{not:sgstm:serietotal}$ és una sèrie temporal
  $S'= \glssymboldef{not:sgstm:multiresolucio}(S,e)$ on
  $S'=\{(5,1),(6,0),(8,0),(10,1)\}$. Expressem el càlcul pas a pas, a la
  \autoref{fig:multiresolucio:multiresolucio} es visualitzen en taula les sèries
  temporals corresponents:

  \begin{enumerate}
  \item En primer lloc es calcula la sèrie temporal pels paràmetres de
    multiresolució $\delta_1$:
    $S_{D1}=\glssymbol{not:sgstm:dmap}(5,\glssymbol{not:sgstm:maxdd}),0,2)=\{(5,1),(10,1)\}$,
    com ja s'ha vist a l'\autoref{ex:multiresolucio:dmap}.

  \item En segon lloc, es calcula la sèrie temporal pels paràmetres de
    multiresolució $\delta_2$:
    $S_{D2}=\glssymbol{not:sgstm:dmap}(2,\glssymbol{not:sgstm:maxdd}),0,3)=\{(6,0),(8,0),(10,1)\}$,
    de manera similar a $S_{D1}$.

  \item En tercer lloc es concatenen les sèries temporals per ordre de
    $\delta$: $\delta_2<\delta_1$. Així, $S'= S_{D2} || S_{D1}$.

  \end{enumerate}
  


\begin{figure}[tp]
  \centering
  \begin{tabular}[c]{|c|c|}
    \multicolumn{2}{c}{$S$} \\ \hline
    $t$  & $v$ \\ \hline
    1  & 0 \\
    3  & 1 \\
    6  & 0 \\
    10  & 1 \\ \hline
  \end{tabular} \qquad
  \begin{tabular}[c]{|c|c|}
    \multicolumn{2}{c}{$S_{D1}$} \\ \hline
    $t$  & $v$ \\ \hline
    5  & 1 \\
    10  & 1 \\ \hline
  \end{tabular} \qquad
  \begin{tabular}[c]{|c|c|}
    \multicolumn{2}{c}{$S_{D2}$} \\ \hline
    $t$  & $v$ \\ \hline
    6  & 0 \\
    8  & 0 \\
    10  & 1 \\ \hline
  \end{tabular} \qquad
  \begin{tabular}[c]{|c|c|}
    \multicolumn{2}{c}{$S'$} \\ \hline
    $t$  & $v$ \\ \hline
    5  & 1 \\
    6  & 0 \\
    8  & 0 \\
    10  & 1 \\ \hline
  \end{tabular}
  \caption{Taules de les sèries temporals per l'operació de plec de  \glssymbol{not:sgstm:serietotal}}
  \label{fig:multiresolucio:multiresolucio}
\end{figure}
 


\end{example}








\subsection{Demostració}

Cal demostrar l'equivalència formalment\todo{}





\section{Sistemes duals SGST+SGSTM}

Una sèrie temporal es pot emmagatzemar i gestionar en un \gls{SGST} o
en un \gls{SGSTM}. També es pot plantejar un sistema dual de
multiresolució on una sèrie temporal es tracti alhora en un \gls{SGST}
i en un \gls{SGSTM}.

Les equivalències entre els \gls{SGSTM} i les funcions de
multiresolució aplicades a un \gls{SGST} permeten dissenyar sistemes
duals que tinguin propietats complementàries.  Així, aquests sistemes
duals ofereixen altres utilitats a la multiresolució més enllà de
l'orientació de compressió amb pèrdua mostrada a la secció??\todo{ref
  model sgstm}. A continuació:
\begin{itemize}
\item Dissenyem l'estructura d'aquests sistemes duals
\item Avaluem conceptes relacionats en l'àmbit genèric dels \gls{SGBD}
\item Mostrem les aplicacions que permeten
\end{itemize}




\subsection{Estructura}

Un sistema dual de multiresolució està format per un \gls{SGST} i un
\gls{SGSTM} on s'emmagatzemen les mateixes sèries temporals. A
cadascun s'hi pot fer les consultes pertinents de cada model per a les
sèries temporal. A més, s'obté el mateix resultat en els dos sistemes
per a les consultes que segueixin les restriccions de la funció de
multiresolució.


A la \autoref{fig:multiresolucio:dual} es mostra l'estructura d'un
sistema dual de multiresolució. L'usuari percep aquest sistema com un
\gls{SGST} on emmagatzema una sèrie temporal $S$ i hi gestiona les
consultes, on algunes d'aquestes consultes són de multiresolució o
treballen sobre sèries temporals $S'$ que provenen de la
multiresolució.  Internament hi ha un \gls{SGST} i un \gls{SGSTM} que
comparteixen l'entrada de mesures de la sèrie temporal. Així, quan
l'usuari so\l.licita la multiresolució $S'$, el sistema dual tant pot
calcular-la a partir del \gls{SGST} amb l'operació de
$\glssymbol{not:sgstm:multiresolucio}(S)$ com a partir del \gls{SGSTM}
amb l'operació de $\glssymbol{not:sgstm:serietotal}(M)$.  La mateixa
estructura també pot servir per al cas de les operacions de
$\glssymbol{not:sgstm:dmap}$ i les de
$\glssymbol{not:sgstm:seriedisc}$.


\begin{figure}
  \centering
  %\usetikzlibrary{shapes,arrows,positioning}
\begin{tikzpicture}[scale=0.8, every node/.style={transform shape}]

      \tikzset{
        mynode/.style={rectangle,rounded corners,draw=black, 
          very thick, inner sep=1em, minimum size=3em, text centered,
          groc},
        myarrow/.style={->, shorten >=1pt, thick},
        mylabel/.style={text width=7em, text centered},
        groc/.style={top color=white, bottom color=yellow!50},
        verd/.style={top color=white, bottom color=green!50},
        roig/.style={top color=white, bottom color=red!50},
      }  






 \node[mynode] (m) {$S$};

 \node[right=1cm of m] (mdins) {};

 \node[mynode, verd, above right=0.25cm and 2cm of m] (tsms) {\glstext{SGST}};

 \node[mynode, verd, below right=0.25cm and 2cm of m] (mtsms) {\glstext{SGSTM}};

 \node[rectangle,draw,minimum height=6cm,minimum width=6.5cm,right=0.9cm of m] (dual) {};

\draw[shift=( dual.south west)]   
  node[above right] {\glstext{SGST} dual};





 \node[right=7cm of m] (tsdins) {};

 \node[mynode,right=8cm of m] (ts) {$S'$};



 \draw (m) -- (mdins.east);
 \draw[myarrow] (mdins.east) -- (tsms.west);
 \draw[myarrow] (mdins.east) -- (mtsms.west);


 \draw (tsms.east) -- (tsdins.west) node[above,midway,sloped]
 {$\glssymbol{not:sgstm:multiresolucio}(S)$}; 
 
 \draw (mtsms.east) -- (tsdins.west) node[above,midway,sloped]
 {$\glssymbol{not:sgstm:serietotal}(M)$};

 \draw[myarrow] (tsdins.west) -- (ts);


\end{tikzpicture}



%%% Local Variables:
%%% TeX-master: "../main"
%%% End:

  \caption{Sistema dual de multiresolució: \gls{SGST}+\gls{SGSTM}}
  \label{fig:multiresolucio:dual}
\end{figure}


Cal comentar que el model de \gls{SGSTM} està dissenyat en base al
model de \gls{SGST} i per tant aquests primers sempre depenen dels
segons. No obstant això, cal no confondre aquesta dependència amb el
sistema dual, el qual gestiona una mateixa sèrie temporal independentment
en un \gls{SGSTM} i en un \gls{SGST}.




Tot i que per al sistema dual és equivalent calcular la sèrie temporal
resultant a partir del \gls{SGST} o del \gls{SGSTM}, no pot seguir el
mateix procediment en cada cas. Per una banda, la
$\glssymbol{not:sgstm:multiresolucio}(S)$ és una operació computada en
temps diferit; cada cop que s'afegeix una nova mesura cal tornar a
calcular tot el resultat. Per altra banda, la
$\glssymbol{not:sgstm:serietotal}(M)$ és una operació computada en
línia; és a dir seguint el flux d'adicions de les mesures.


El sistema dual dissenyat funciona a partir de l'adició de mesures, de
la mateixa manera que els \gls{SGSTM}. L'ordre d'arribada d'aquestes
mesures és crític en el sistema dual ja que un cop el \gls{SGSTM} s'ha
consolidat les dades més antigues que arribin no seran tingudes en
compte i per tant l'equivalència entre les consultes de \gls{SGST} i
\gls{SGSTM} ja no serà certa. Així doncs, si es vol mantenir
l'equivalència, el sistema dual dissenyat té aquestes dues
restriccions: només permet operacions d'adició i l'ordre d'adició és
important.  Més endavant a les aplicacions descriurem l'abast
d'aquestes restriccions.







\subsection{Conceptes relacionats}


\textcite{marz13:nosql13, marz14:bigdata} generalitzen un concepte
similar al de \gls{SGST} dual, ho emmarquen en l'àmbit dels \gls{SGBD}
per a \emph{Big Data}.  Proposen \gls{SGBD} dissenyats amb tres
nivells, que anomenen arquitectura \emph{Lambda}:
\begin{itemize}
\item Nivell \emph{batch}: Emmagatzema totes les dades originals i
  permet realitzar qualsevol consulta sobre aquestes dades. Preveu que
  algunes consultes operen sobre dades consultades prèviament, per
  tant en aquest nivell es gestionen també aquestes consultes
  precomputades, les quals a més es poden obtenir amb computació
  para\l.lela com per exemple amb Hadoop. Particularment, es considera
  que les dades originals són immutables, és a dir que les bases de
  dades només permeten afegir però no modificar.

\item Nivell \emph{server}: Emmagatzema les consultes precomputades i
  n'ofereix les dades per a altres consultes. Les consultes
  precomputades s'han de tornar a calcular periòdicament i en el
  nivell \emph{server} sempre hi ha la versió calculada més
  recent. Per tant, es preveu que les consultes precomputades no
  ofereixen la informació actualitzada al moment, sinó que hi ha un
  cert temps des que es modifiquen les dades originals fins que té
  impacte en les consultes.

\item Nivell \emph{speed}: Precomputa les mateixes consultes que el
  nivell \emph{batch} però incrementalment, és a dir cada cop que
  s'afegeix una dada nova les dades de la consulta \emph{speed}
  s'actualitzen adequadament.  Aquest nivell només s'usa per a dades
  recents per tal complementar el problema de les dades
  desactualitzades en els nivells \emph{batch} i \emph{server}.
\end{itemize}


Les consultes precomputades d'aquests sistemes semblen una bona
solució per a la computació de les vistes dels \gls{SGBDR}.  Una vista
és un àlies per a una expressió relacional, és a dir una consulta, que
s'utilitza en altres consultes. Per tant, una vista $v$ és una
operació $\text{op1}$ sobre unes $\text{dades}$,
$v:=\text{op1}(\text{dades})$, i s'utilitza una altra consulta
$\text{op2}(v)$ de manera que és equivalent a executar la consulta
$\text{op2}(\text{op1}(\text{dades}))$. Així doncs, el model de vistes
és similar a les consultes que es basen en altres consultes proposat
per \citeauthor{marz14:bigdata} o a les sèries temporals precomputades
que proposem.


En el model relacional \cite[cap.~10. Views]{date04:introduction8} es
considera, conceptualment, que les vistes no s'avaluen quan es
defineixen sinó cada cop que s'executa una consulta que hi opera.  En
les implementacions les vistes poden ser precomputades, aleshores
s'anomenen \emph{snapshots} o \emph{materialized views}, per tal
d'aconseguir un emmagatzematge temporal dels mateixos càlculs per a
diverses consultes. En el context de sistemes de suport a les
decisions, la precomputació també es preveu en el càlcul de taules
resum per a agregacions de les dades \cite[cap.~22. Decision
support]{date04:introduction8}.  Això no obstant, la precomputació de
vistes no sempre comporta una millor eficiència; el concepte de vista
del model permet la substitució algebraica i per tant permet
l'optimització global de la consulta i l'operació continguda a la
vista.


Les vistes precomputades tenen associada una acció per actualitzar de
nou el seu valor, és a dir per a recalcular la consulta que contenen
quan les dades originals han canviat. En usar vistes precomputades cal
preveure el termini de validesa dels càlculs precomputats, com ocorre
en el nivell \emph{server} de l'arquitectura \emph{Lambda}. Així
doncs, les vistes precomputades es poden actualitzar de vàries
maneres:
\begin{itemize}

\item L'usuari decideix manualment quan s'han de tornar a computar. Per
  exemple, per a treballar durant un cert temps amb còpies immutables
  de les dades (\emph{snapshots}) sense haver de blocar la modificació
  de la base de dades \cite[\S{}10.5]{date04:introduction8}.

\item Es computen associades a un temps de termini a partir del qual
  si es tornen a utilitzar s'hauran de tornar a computar. \todo{}

\item Es computen periòdicament, com també es proposa en el nivell
  \emph{batch} de l'arquitectura \emph{Lambda}.

\item Es computen cada cop que es modifiquen les dades amb les quals
  operen, és a dir quan les dades originals reben una operació
  d'afegir, de modificar o d'actualitzar es torna a computar tota la
  vista.

\item Quan es modifiquen les dades, s'aplica la mateixa operació a la
  vista precomputada. És a dir, quan es modifiquen les dades originals
  amb una operació, $\text{mod}(\text{dades})$, es trasllada aquesta
  operació a la vista, $\text{mod}'(\text{dades})$, on cal determinar
  la relació entre $\text{mod}$ i $\text{mod}'$.  Aquesta translació
  és més senzilla quan només hi ha possibilitat d'operacions d'afegir
  noves dades però no de modificar-les: és el
  que es proposa en el nivell \emph{speed} de l'arquitectura
  \emph{Lambda} i el que admet el model de \gls{SGSTM} que
  proposem. \todo{s'ha d'estudiar la relació d'això amb els data stream}
% When data comes as an ordered sequence of instances it is called
% data stream, then specific \acro{DBMS} are designed to manage data
% stream data \cite{stonebraker05:sigmod}.  \acro{MTSMS} can take
% advantage of data stream orientation in order to simplify the
% consolidation process.  Assuming a time order acquisition of time
% series, the update of a \acro{MTSMS} only consists in the addition of
% new measures and the incremental consolidation of subseries. 
\end{itemize}



En resum, la sèrie temporal resultant dels sistemes duals pot ser
considerada com una vista calculada sobre les sèries temporals
originals. Aquesta vista pot ser precomputada, cosa que es pot fer de
diverses maneres: la funció de multiresolució l'ha de computar
totalment cada cop que s'afegeix una nova mesura i en canvi els
\gls{SGSTM} la computen incrementalment.  Aleshores aquestes vistes es
poden usar per a altres consultes que tinguin com a context
l'aproximació de multiresolució realitzada, o per a visualitzacions
gràfiques com les que ofereix RRDtool \cite{rrdtool}.




\subsection{Aplicacions}


Les sèries temporals són dades que s'adquireixen contínuament i per
tant cada cop és més gran el volum de dades que s'han d'emmagatzemar i
tractar. Aquest gran volum de dades és un problema per a operar amb
les sèries temporals i és un problema en els sistemes que tenen
l'emmagatzematge limitat. En aquest sentit, originalment hem
plantejat el model de \gls{SGSTM} per tal d'oferir una solució
d'emmagatzematge que comprimeix la informació seleccionant-ne una
multiresolució determinada.


Així, un \gls{SGSTM} implica un selecció d'informació i la informació
que no es considera importat és descartada. Aquests sistemes, per
tant, no són adequats quan totes les dades monitorades han de ser
emmagatzemades tal com s'adquireixen. Un cas d'aquests és quan no es
coneixen quines funcions d'agregació són les més escaients per a les
dades futures que s'adquiriran. Un altre cas és quan volem resoldre
consultes detallades sobre les dades, com per exemple: a quina hora
exacta ha ocorregut un esdeveniment.


Els sistemes duals de multiresolució ofereixen una solució per
emmagatzemar totes les dades però mantenint-ne una gestió de
multiresolució.  En el sistema dual, s'ha d'entendre el \gls{SGST} com
un emmagatzematge a llarg termini que no és consultat freqüentment;
així pot estar implementat com a \gls{SGBD} per a dades de gran volum
i basat en tècniques de compressió sense pèrdua, encara que tinguin
temps grans de descompressió. El \gls{SGSTM} s'ha d'entendre com un
emmagatzematge de compressió amb pèrdua que conté multiresolucions
precomputades de la sèries temporal.  El temps de còmput no és tant
crític en els \gls{SGSTM} perquè es reparteix al llarg del temps, és a
dir tal com es van adquirint les dades; més enllà del temps de còmput
de cada funció d'agregació d'atributs, el qual limita la quantitat de
multiresolucions diferents que pot gestionar un mateix \gls{SGSTM}.


En la compressió de dades multimèdia s'utilitza una tècnica de gestió
similar.  Les dades s'emmagatzemen inicialment amb compressió sense
pèrdua, a partir d'aquestes es generen dades amb compressió amb pèrdua
que ocupen menys i són més àgils per a treballar. En el cas que calgui
modificar les dades, es canvien les comprimides sense pèrdua i es
regeneren de nou comprimides amb pèrdua. Amb aquesta gestió s'evita el
problema que les compressions amb pèrdua acumulin pèrdua entre
successives modificacions.  \todo{citació?}



En resum, les aplicacions del sistema dual de multiresolució són les
següents:
\begin{itemize}
\item Sistemes on els \gls{SGSTM} precomputen en flux vistes que són
  consultes de multiresolució. És a dir, funcionen com a precomputació
  de consultes que es preveuen que es faran; per tant al llarg del
  temps es creen i eliminen vistes segons les necessitats que es
  preveuen. Aleshores els \gls{SGST} funcionen com a emmagatzematge a
  llarg termini que es consulta rarament.


\item Les dades emmagatzemades en els \gls{SGST} s'utilitzen per al
  farciment inicial dels \gls{SGSTM} gràcies a la funció de
  multiresolució que permet computar les sèries temporals dels discs a
  partir de l'operació \glssymbol{not:sgstm:dmap}. Pot tenir diversos
  objectius:
  \begin{itemize}
  \item Quan es creen les vistes precomputades anteriors, inicialment
    el \gls{SGSTM} contindrà sèries temporals amb valors desconeguts;
    amb la funció de multiresolució es poden inicialitzar amb els
    valors correctes.

  \item Es pot usar per a canviar l'esquema de multiresolució dels
    \gls{SGSTM}. En alguns canvis d'esquema, per exemple ampliar un
    disc, inicialment hi ha dades desconegudes però que es poden
    computar amb la funció de multiresolució.

  \item Es pot usar per a canviar d'un emmagatzematge de sèries
    temporals en \gls{SGST} a un emmagatzematge en \gls{SGSTM}. Cal
    notar que és un canvi irreversible perquè l'emmagatzematge en els
    \gls{SGSTM} és amb pèrdua.
  \end{itemize}

\item Es poden usar els \gls{SGST} per a experimentar amb diversos
  esquemes de multiresolució per a les dades adquirides i així
  observar-ne la idoneïtat i escollir-ne un de millor.

\item En el cas que no es compleixi la restricció d'ordre d'arribada
  de les mesures per a l'equivalència entre els \gls{SGST} i els
  \gls{SGSTM}, es podria refer la informació emmagatzemada en els
  \gls{SGSTM} a partir dels \gls{SGST}.

\end{itemize}


Com a contrapartida, però, en els sistemes duals apareix un \gls{SGST}
amb una gran quantitat de dades. Per tant, cal tenir en compte que si
la informació computada pels \gls{SGSTM} és suficient per a les
consultes que s'han de realitzar, aleshores la informació
emmagatzemada en els \gls{SGST} és redundant. Això no obstant, no és
senzill identificar i predir quan la informació emmagatzemada en el
\gls{SGSTM} serà totalment suficient.
\todo{a la secció de la
  teoria de la informació descriurem el problema d'identificar la
  informació que seleccionem i la que perdem i per tant podríem predir
  quan la tenim redundant?? }



Així, encara que l'objectiu final sigui l'emmagatzematge de les sèries
temporals comprimides amb pèrdua en un \gls{SGSTM}, és a dir el model
proposat originalment, els sistemes duals de multiresolució es poden
utilitzar mentre hi hagi dubtes sobre quin esquema de multiresolució
escollir i un eliminar-los un cop es consideri que l'esquema és
correcte. Aleshores, l'estructura de sistema dual serveix per a
observar clarament que els \gls{SGSTM} ofereixen una sèrie temporal
$S'$ que és una aproximació a una certa informació que hi ha en
l'original $S$ i que, per tant, permeten resoldre consultes
aproximades.






\section{Teoria de la informació}

\todo{}








% \section{Stream orientation}

% \todo{}
% * Dues variacions possibles interessants pels MTSMS:

 
%   - Buffers com a streams, sempre de mida fitada 
%   - Discos enllaçats





%%% Local Variables:
%%% TeX-master: "main"
%%% End:






%  LocalWords:  multiresolució multiresolucions

\chapter{Reflexions sobre la informació en la multiresolució}
\label{sec:multiresolucio:teoriainformacio}



En aquest capítol reflexionem sobre el problema de la qualitat en la
multiresolució de sèries temporals, és a dir sobre el problema
d'identificar la informació que selecciona un esquema de
multiresolució i per tant, alhora, d'observar quina informació no
queda emmagatzemada i es perd.  Contextualitzem aquest problema en
l'aplicació de la teoria de la informació per a l'esquema de
multiresolució.

En aquest context, els \gls{SGSTM} són un sistemes que emmagatzemen
dades comprimides amb pèrdua d'una certa part de la informació
original. Aleshores, les consultes que es resolen a partir d'un
\gls{SGSTM} són consultes aproximades a la informació total original,
llevat que mitjançant una anàlisi determinem que poden oferir
consultes exactes. A continuació reflexionem sobre com analitzar
l'error de les consultes de multiresolució respecte a la informació
original:
\begin{itemize}
\item Primer, descrivim de forma molt genèrica la teoria de la
  informació, la qual està relacionada amb la quantificació de la
  informació.
\item Segon, definim el problema de calcular l'error en la multiresolució.
\item Tercer, mostrem exemples d'anàlisi de la informació per alguns
  esquemes de multiresolució.
\end{itemize}






\section{Quantificació de la informació}

En altres àmbits, la teoria de la informació (\emph{information
  theory}) és la teoria de referència per a formalitzar la
quantificació de la informació. Aquest també és el cas de la
compressió de dades, un àmbit proper als \gls{SGSTM} en aquest context
d'informació.  Per al cas particular de compressió amb pèrdua s'aplica
un subconjunt de la teoria anomenat teoria de la taxa de
bit-distorsió %optimot: rate-distortion relationship
(\emph{rate–distortion theory}); la qual modela la percepció de la
distorsió i valora l'estètica dels resultats.


Per a quantificar la informació d'unes dades s'utilitza l'entropia, la
qual mesura la incertesa que hi ha en unes dades particulars. Així,
com més entropia més incertesa, a causa que les dades són més
aleatòries, i com menys entropia més facilitat de predir-les a causa
que són més redundants. Si l'entropia és zero aleshores les dades són
totalment predictibles; és a dir donat un valor es coneix exactament
quin és el següent.




La compressió redueix la mida original de les dades. En el cas de la
compressió sense pèrdua es conserva la informació però s'augmenta
l'entropia, ja que s'eliminen les dades redundants. En el cas de la
compressió amb pèrdua es descarta informació que es considera no
essencial. Un exemple de compressió amb pèrdua és eliminar detalls
d'una imatge que l'ull humà no pot apreciar. A més, la compressió amb
pèrdua també es pot utilitzar per a transformar les dades a un altre
domini on es percebi millor la informació, la qual cosa es coneix amb
el nom de codificació perceptiva (\emph{perceptual encoding}). Un
exemple de codificació perceptiva és transformar els sons al domini
freqüencial per a operacions d'equalització.



Els mètodes de compressió amb pèrdua se solen usar per a compressió de
multimèdia. L'objectiu és aconseguir menys volum de dades però que
conservin la mateixa percepció que les originals, o fins i tot amb una
pèrdua de qualitat perceptible mentre compleixi amb els requisits de
l'aplicació que se li vol donar. Així doncs, la compressió de
multimèdia sovint requereix valorar la percepció humana, per tal de
valorar la qualitat de percepció humana s'utilitzen testos subjectius
en què un humà ha d'intentar distingir entre multimèdia original i
multimèdia comprimit.
%  com per exemple el test
% ABX\todo{ref}.  %http://en.wikipedia.org/wiki/ABX_test
%http://web.archive.org/web/20070813001013/http://www.pcabx.com/


%  Lossy methods are most often used for compressing sound, images or videos. This is because these types of data are intended for human interpretation where the mind can easily "fill in the blanks" or see past very minor errors or inconsistencies – ideally lossy compression is transparent (imperceptible), which can be verified via an ABX test. http://en.wikipedia.org/wiki/ABX_test
% Flaws caused by lossy compression that are noticeable to the human eye or ear are known as compression artifacts.




% \todo{information algebra}
% \url{https://en.wikipedia.org/wiki/Information_algebra}

% algebra of information, describing basic modes of information processing. Such an algebra involves several formalisms of computer science, which seem to be different on the surface: relational databases, multiple systems of formal logic or numerical problems of linear algebra. It 



% El procediment
% que se segueix és: comprimir, emmagatzemar, descomprimir i
% visualitzar.
% compressed data must be decompressed to use -> en els sgstm no? bé si vull veure tota la sèrie temporal sí que he de concatenar el que hi tinc i per tant es pot veure com una descompressió (al marge que hi pot haver una compressió/descompressió en els agregadors usats, per exemple si és un so emmatgatzemar la freqüència i després s'haurà de descomprimir a amplitud al llarg del temps).
% The design of data compression schemes involves trade-offs among various factors, including the degree of compression, the amount of distortion introduced (e.g., when using lossy data compression), and the computational resources required to compress and uncompress the data.
% * No hi ha descompressió. Els algoritmes de compressió amb pèrdua normalment van associats amb les tècniques de compressió/descompressió; és a dir que les dades s'emmagatzemen amb estructures intermitges, que ocupen menys mida, i que cal descomprimir per recuperar-les. En els cas dels SGSTM les dades s'emmagatzemen com a subsèries temporals en els discs i per tant a l'hora de recuperar-les ja són sèries temporals, com a molt potser fa falta concatenar-les per a obtenir tota la sèrie temporal.



% The main drawback of lossy compression techniques is that they
% rely on specific patterns for providing a good approximation of the given time series.
% This is the main reason why lossy compression has been rarely applied to network
% monitoring contexts, where the patterns of time series can drastically change due to
% anomalous events or to transient networking issues.

%http://www.data-compression.com/theory.shtml


\section{Error en la informació de la multiresolució}

El model dels \gls{SGSTM} es basa en una compressió amb pèrdua, és a
dir descartar dades i emmagatzemar només aquella informació que es
consideri necessària o suficient. Així doncs, cal quantificar quin
error hi ha entre la informació emmagatzemada i la informació que
contenen les dades originals.


El problema genèric de la informació en la multiresolució és el següent.
\begin{definition}[Error en la informació de la multiresolució]
  \label{def:informacio:error}
  Sigui una sèrie temporal $S$ i una sèrie temporal multiresolució $M$
  resultant de l'emmagatzematge i consolidació de $S$. De la sèrie
  temporal multiresolució es pot consultar la sèrie temporal total
  $S'=\glssymbol{not:sgstm:serietotal}(M)$ o bé de forma equivalent,
  com s'ha descrit en \textref{cap:funciomultiresolucio},
  $S'=\glssymbol{not:sgstm:multiresolucio}(S,\glssymbol{not:esquemaM})$
  on $,\glssymbol{not:esquemaM}$ és l'esquema de $M$.  S'executa una
  operació de consulta, $o$, sobre la sèrie temporal original,
  $r_1=o(S)$, i la mateixa operació sobre la sèrie temporal de la
  multiresolució, $r_2=o(S')$. Es pot avaluar l'error de
  multiresolució $\epsilon=d(r_1,r_2)$ on $d$ és una funció que permet
  avaluar la distància entre els dos resultats i per tant considerem
  $\epsilon\geq 0$.
\end{definition}


Cal aclarir que es podria avaluar $d(S,S')$ com un problema
d'aproximació al senyal original. Aquest és un problema ben resolt en
altres models, com per exemple \textcite{last01,ogras06}, però no és
el problema que es vol resoldre en els \gls{SGSTM}.  L'objectiu dels
\gls{SGSTM} és comprimir les dades i seleccionar una determinada
informació, per tant és un problema de compressió amb pèrdua.  Així
doncs, ens interessa avaluar l'error de la multiresolució en aquest
context, en què les consultes als \gls{SGSTM} ($r_2$) haurien de
retornar resultats similars que si es fessin a les dades originals
($r_1$), sense que sigui necessari que $S$ i $S'$ es corresponguin.




Per a avaluar la distància $d(r_1,r_2)$, si els resultats d'ambdues
consultes són sèries temporals, es pot utilitzar per exemple mínims
quadrats com fa \textcite{last01}.  Ara bé, en la informació de la
multiresolució cal pensar també amb consultes qualitatives. Per
exemple una consulta podria ser $o=$`Creix la sèrie temporal?'
Aleshores no hi hauria error $\epsilon=d(r_1,r_2)$ quan la resposta
fos la mateixa per als dos casos, $r_1$ i $r_2$, i hi hauria error
quan les respostes diferissin.



Tot i que hem proposat d'aplicar la mateixa operació $o$ a la sèrie
temporal original $r_1=o(S)$ i a la sèrie temporal de la
multiresolució $r_2=o(S')$, pot ser que l'operació hagi de ser
diferent per a obtenir el mateix resultat. És a dir $r_1=o(S)$ però
$r_2=o'(S')$ on $o'$ és l'operació equivalent a $o$ que s'ha d'aplicar
després de la multiresolució.  Per exemple, una operació $o$ podria
ser el càlcul de la multiresolució,
$r_1=\glssymbol{not:sgstm:multiresolucio}(S,\glssymbol{not:esquemaM})$,
aleshores l'operació $o'$ equivalent és la identitat perquè el
\gls{SGSTM} ja ha calculat la multiresolució, $r_2=S'$. Aquest
exemple, de fet, és el cas que hem formulat a la
\autoref{sec:multiresolucio:funcio}, on hem descrit l'equivalència
entre la sèrie temporal total d'un \gls{SGSTM} i la funció
\glssymbol{not:sgstm:multiresolucio} d'una sèrie temporal; per tant
l'error $\epsilon =d(r_1,r_2)$ seria nul.



Així doncs, l'error en la informació de la multiresolució permet
quantificar la pèrdua d'informació dels \gls{SGSTM}. Un cop
quantificat l'error per a un determinat esquema de multiresolució es
pot saber per a quines consultes serà apropiat aquell esquema i per a
quines no. Per a quantificar aquest error es poden preveure diversos
contextos:
\begin{itemize}
\item Hi ha consultes que es poden resoldre a la perfecció, és a dir
  sense error. Per exemple és el cas descrit on l'operació que es vol
  fer a una sèrie temporal és precisament la consulta de
  multiresolució.
  %altres exemples, per exemple en el cas de comptadors per a consultar totals

\item Es pot calcular l'error mesurant la diferència entre la mateixa
  consulta aplicada a les dades originals que a les dades
  emmagatzemades en el \gls{SGSTM}.

\item Es pot avaluar subjectivament l'error mitjançant la
  visualització de les dades. És a dir, de manera semblant a la
  compressió amb pèrdua de multimèdia, l'usuari valora subjectivament
  si visualitza la mateixa informació en les dades originals com en
  les dades comprimides amb multiresolució. Per exemple, un dels
  criteris que recomana RRDtool per a establir un esquema de
  multiresolució és tenir en compte l'amplada de la pantalla on es
  visualitzen els resultats: no cal treballar amb una sèrie temporal
  amb molta resolució si no és possible observar-la \parencite[Rates,
  normalizing and consolidating]{vandenbogaerdt} .
  % RRDtool expliquen un criteri que és definir les k dels discs inferior a l'amplada en píxels de la pantalla. Aquest és un criteri basat en una consulta per a fer visualització immediatament. Llavors no té sentit tenir més nombre de dades que les que es poden visualitzar.  Per exemple si capturem dades cada 5 minuts al llarg d'un any obtenim 43800 punts; si no disposem d'un monitor amb aquest nombre de píxels d'amplada no els podrem percebre.
  %Això potser també té relació amb el camp de gràfics 3D, on la multiresolució s'aplica per a no treballar amb objectes que només seran percebuts com un píxel?

\end{itemize}






\section{Exemples d'anàlisi de la informació}


La \autoref{def:informacio:error} descriu el problema d'informació en
la multiresolució de forma genèrica i abstracta. Així, en cada context
d'aplicació de la multiresolució cal interpretar-ne el significat i
particularitzar-ne un mètode d'anàlisi adequat.  A tal efecte, a
continuació proposem alguns exemples que mostren casos particulars
d'anàlisi de la selecció o pèrdua d'informació que hi ha en un
\gls{SGSTM}.


Per a simplificar els exemples, proposem esquemes de multiresolució amb
només una resolució i amb funcions d'agregacions d'atributs que
seleccionen intervals independents de mesures. Per tal de referir-nos
amb comoditat als càlculs de les funcions d'agregació d'atributs,
anomenem les mesures i els valors amb els quals operen mitjançant la
notació següent.



\begin{definition}[Notació de les mesures i els valors amb els quals operen
  les funcions d'agregació d'atributs]
  \label{def:informacio:notaciovalors}
  Sigui $S=\{m_0,\dotsc,m_k\}$ una sèrie temporal, on recordem que les
  mesures tenen la forma $m=(t,v)$, i $\glssymbol{not:esquemaM}= \{
  (\delta, \glssymbol{not:sgstm:f} , \tau, \glssymbol{not:sgstm:k})
  \}$ un esquema de multiresolució amb una sola subsèrie
  resolució. Assumim $ \glssymbol{not:sgstm:k}=\infty$ per tal de
  negligir-ne l'efecte i $\tau=T(\min S)$. Obtenim la sèrie temporal
  resultant de la funció de multiresolució
  $S'=\glssymbol{not:sgstm:multiresolucio}(S,\glssymbol{not:esquemaM})$
  (\seeref{def:multiresolucio:plecmu}).  Així, aquesta sèrie temporal
  resultant conté mesures calculades a partir de l'agregació de $S$ en
  els intervals definits per $\tau$ i $\delta$ i té la forma
  \begin{align*}
    S'=  \{ & \glssymbol{not:sgstm:f}(S,\tau,\delta), \\
    & \glssymbol{not:sgstm:f}(S,\tau+\delta,\delta),\\
    &\dotsc, \\
    & \glssymbol{not:sgstm:f}(S,\tau+(n-1)\delta,\delta)\}
  \end{align*}
  on $\tau+n\delta\geq T(\max S)$ i $n\in\glssymbol{not:N}$.


  La funció d'agregació d'atributs $\glssymbol{not:sgstm:f}$ realitza
  una operació sobre les mesures corresponents a l'interval de temps
  definit (\seeref{sec:model:agregador}).  Així per a l'agregació en
  el primer interval $[\tau,\tau+\delta]$ escriurem de forma genèrica
  que utilitza les mesures$m_0,\dotsc,m_{i_1}$ de la sèrie temporal
  original, $m_0,\dotsc,m_{i_1} \in S$ on $i_1$ pot ser qualsevol
  índex, i per tant escriurem els valors corresponent a aquestes
  mesures com $v_0,\dotsc,v_{i_1}$. De la mateixa manera, escriurem
  $m_{i_1+1},\dotsc,m_{i_2}$ i $v_{i_1+1},\dotsc,v_{i_2}$ per a
  l'agregació en el segon interval,
  $[\tau+\delta,\tau+2\delta]$, i així successivament
  fins al darrer interval en què notem les mesures amb $m_{i_n},
  \dotsc, m_k$ i els valors amb $v_{i_n}, \dotsc, v_k$.
%és a dir s'utilitzen les mesures de forma independent a cada interval, segons les opcions descrites a les famílies d'agregacions aquí simplifiquem i no introduïm els casos en què una mesura s'utilitza en més d'un interval, per exemple en el ZOHE hauríem de formular com s'escampa la informació en més d'un interval

  Aleshores, a partir de la notació dels valors, podem expressar la
  sèrie temporal resultant amb la forma
  \begin{align*}
    S'=\{& (\tau+\delta, f'(v_0,\dotsc,v_{i_1})),\\
    & (\tau+2\delta, f'(v_{i_1+1},\dotsc,v_{i_2})),\\
    & \dotsc,\\
    & ((\tau+n\delta ,f'(v_{i_M},\dotsc, v_k)) \}
  \end{align*}
  on $f'$ és l'operació corresponent de l'atribut que resumeix
  $f$. En els exemples següents assenyalarem quin $f'$ correspon a
  cada $f$ que utilitzem.
\end{definition}



L'anàlisi que formulem és una introducció a la reflexió sobre l'error
en la informació de la multiresolució. Així, de forma simple,
analitzem si hi ha error o si no n'hi ha, sense pretendre avaluar
quantificacions més complicades. A més, ho analitzem en base a
l'esquema de multiresolució que s'utilitzi, particularment de quines
funcions d'agregació d'atributs s'utilitzin i de com siguin les
consultes posteriors.



\subsection{Mateixa consulta i funció d'agregació d'atributs}
\label{ex:multiresolucio:f=op}


En aquest apartat es formulen exemples que reflexionen sobre l'error
de multiresolució que hi pot haver quan una consulta s'aplica a tota
la sèrie temporal i l'operació de consulta es correspon amb la mateixa
funció d'agregació d'atributs que s'ha utilitzat en l'esquema de
multiresolució.


% Sigui $S=\{m_0,\dotsc,m_k\}$ una sèrie temporal, on $m_k=(t_k,v_k)$, i
% $e= \{ (\delta_0, f_0, \tau_0, k_0) \}$ un esquema de multiresolució
% amb una sola subsèrie resolució, suposem $k_0=\infty$ per tal de
% negligir-ne l'efecte i $\tau_0=T(\min(S))$. Obtenim la sèrie temporal
% resultant d'aplicar la funció de multiresolució
% $S'=\glssymbol{not:sgstm:multiresolucio}(S,e)$.  Així, aquesta sèrie
% temporal resultant contindrà mesures calculades a partir de
% l'agregació de $S$ en els intervals definits per $\tau_0$ i $\delta_0$
% i tindrà la forma $S'=\{ f_0(S,[\tau_0,\tau_0+\delta_0],
% f_0(S,[\tau_0+\delta_0,\tau_0+2\delta_0]),\dotsc,
% f_0(S,[\tau_0+(M-1)\delta_0,\tau_0+M\delta_0])\}$ on
% $\tau_0+M\delta\geq T(\max(S))$ i $M\in\glssymbol{not:N}$. Si escrivim
% de forma genèrica $v_0,\dotsc,v_{i_1}$ els valors originals usats en
% l'agregació del primer interval $f_0(S,[\tau_0,\tau_0+\delta_0])$,
% $v_{i_1+1},\dotsc,v_{i_2}$ els del segon interval, etc. aleshores
% podem expressar $S'=\{ (\tau_0+\delta_0, f'(v_0,\dotsc,v_{i_1})),
% (\tau_0+2\delta_0, f'(v_{i_1+1},\dotsc,v_{i_2})), \dotsc,
% ((\tau_0+M\delta_0 ,f'(v_{i_M}, \dotsc, v_k)) \}$ on $f'$ és
% l'aplicació corresponent de l'atribut que resumeix $f_0$.


El context del problema és el següent. S'aplica un operador de
consulta $o$ a les dues sèries temporals, $r_1=o(S)$ i $r_2=o(S')$.
Aquest operador $o$ és el mateix càlcul que fa la funció d'agregació
d'atributs $f$ però aplicat a totes les mesures de la sèrie temporal
original, $o(S)=V(f(S,[-\infty,\infty]))$. Analitzem l'error de
multiresolució entre $r_1$ i $r_2$ segons tres funcions d'agregació
d'atributs:

  \begin{itemize}
  \item Màxim: $f=\glssymbol{not:sgstm:maxpd}$, el qual es correspon
    a aplicar l'operació d'agregació dels \glspl{SGST}
    $o=\glssymbol{not:sgst:maxv}$ (\seeref{def:sgstm:maxpd}) i per
    tant a calcular l'atribut $f'=\max$ dels valors (\seeref{def:sgst:maxv}).

    D'una banda $S=\{m_0,\dotsc,m_k\}$ i aleshores
    $r_1=o(S)=\glssymbol{not:sgst:maxv}(S) = \max(v_0,\dotsc,
    v_k)$. D'altra banda, aplicant la notació de la
    \textref{def:informacio:notaciovalors}, $S'=\{ (\tau+\delta,
    \max(v_0,\dotsc,v_{i_1})),
    (\tau+2\delta,\max(v_{i_1+1},\dotsc,v_{i_2})), \dotsc,
    (\tau+n\delta,\max(v_{i_n}, \dotsc, v_k)) \}$ i aleshores
    $r_2=o(S')= \max\big( \max(v_0,\dotsc,v_{i_1}),
    \max(v_{i_1+1},\dotsc,v_{i_2}), \dotsc, \max(v_{i_n}, \dotsc v_k)
    \big)$.

    Atès que $\max(v_0,\dotsc, v_k) = \max\big(
    \max(v_0,\dotsc,v_{i_1}), \max(v_{i_1+1},\dotsc,v_{i_2}), \dotsc,
    \max(v_{i_n}, \dotsc v_k) \big)$ perquè $\max$ és una funció
    associativa: $\max(a,b,c,d,e) = \max( \max(a,b), \max(c,d,e))$,
    podem concloure que en aquest cas $r_1=r_2$ i per tant
    $\epsilon=0$.


    En aquest exemple hem negligit els temps resultants de
    l'agregació. És a dir, $r_1$ és correspon amb una o més d'una
    mesura $m\in S: V(m)=r_1$ i de la mateixa manera $n\in S':
    V(n)=r_2$ on hem conclòs que $V(m)=V(n)$. Això no obstant,
    en general els temps d'aquestes mesures no es correspondran,
    $T(m)\neq T(n)$ perquè $f=\glssymbol{not:sgstm:maxpd}$
    resumeix els atributs de temps segons l'interval de consolidació i
    al marge del resum de la informació en els valors.



  \item Mitjana aritmètica: $f=\glssymbol{not:sgstm:mitjanapd}$, el
    qual es correspon a aplicar l'operació d'agregació dels
    \glspl{SGST} $o=\glssymbol{not:sgst:mitjanav}$
    (\seeref{def:sgstm:mitjanapd}) i per tant a calcular l'atribut
    $f'=\operatorname{mitjana}$ (aritmètica) dels valors
    (\seeref{def:sgst:mitjanav}).

    De manera similar al cas anterior, els resultats són
    $r_1=\operatorname{mitjana}(v_0,\dotsc, v_k)$ i
    $r_2=\operatorname{mitjana}\big(
    \operatorname{mitjana}(v_0,\dotsc,v_{i_1}),
    \operatorname{mitjana}(v_{i_1+1},\dotsc,v_{i_2}), \dotsc,
    \operatorname{mitjana}(v_{i_n}, \dotsc v_k) \big)$.  Però en
    aquest cas hem de concloure que $\epsilon\geq 0$ perquè
    la mitjana no és una funció associativa:
    $\operatorname{mitjana}(a,b,c,d,e) \neq \operatorname{mitjana}(
    \operatorname{mitjana}(a,b), \operatorname{mitjana}(c,d,e))$.
 


  \item Total: definim una funció d'agregació d'atributs
    $f=\operatorname{total}$ que, negligint l'atribut de temps,
    resumeix la sèrie temporal amb la suma dels valors:
    $\operatorname{total}(S,\tau,\delta)=m'$  i $V(m') =
    \sum\limits_{\forall m\in S(\tau,\tau+\delta]} V(m)$. Així doncs, es
    correspon a aplicar l'operació d'agregació dels \glspl{SGST}
    $o=\glssymbol{not:sgst:sumav}$ i per tant a calcular l'atribut
    $f'=\sum$ dels valors (\seeref{def:sgst:sumav}).

    Aquest cas és similar al del màxim, $r_1=\sum(v_0,\dotsc, v_k)$ i
    $r_2=\sum\big( \sum(v_0,\dotsc,v_{i_1}),
    \sum(v_{i_1+1},\dotsc,v_{i_2}), \dotsc, \sum(v_{i_n}, \dotsc v_k)
    \big)$, i $\epsilon=0$ perquè la suma és una funció
    associativa.
    

  \end{itemize}


\subsection{Mitjana d'una sèrie temporal regular}

  Seguint \textref{ex:multiresolucio:f=op}, podem estudiar en quins
  casos la mitjana aritmètica no té error. Com ja s'ha dit, la mitjana
  no és una funció associativa. Però sí que esdevé associativa quan
  s'associen el mateix nombre d'elements:
  $\operatorname{mitjana}(a,b,c,d,e,f) = \operatorname{mitjana}(
  \operatorname{mitjana}(a,b), \operatorname{mitjana}(c,d),
  \operatorname{mitjana}(e,f)) = \operatorname{mitjana}(
  \operatorname{mitjana}(a,b,c), \operatorname{mitjana}(d,e,f))$.

  Per tal que s'associïn el mateix nombre d'elements cal que per cada
  interval de consolidació de la sèrie temporal hi hagi el mateix
  nombre de mesures:
  $|S[\tau,\tau+\delta]|=|S[\tau+\delta,\tau+2\delta]|=\dotsb
  = |S[\tau+(n-1)\delta,\tau+n\delta]|$.  Aquest cas es
  compleix, per exemple, quan la sèrie temporal és regular amb període
  $p$ i el pas de consolidació de l'esquema multiresolució és múltiple
  de la regularitat de la sèrie temporal, $\delta = kp$ on
  $k\in\glssymbol{not:N}$, o bé quan la sèrie temporal és de temps
  real amb període $p$ i iniciada a $t$ i la multiresolució n'és
  múltiple: $\delta = kp$ i $\tau = t+k\delta$
  (\seeref{sec:sgst:regularitat}).

  Aleshores, en aquest casos, sí que es podria concloure que
  $\epsilon=0$ per la multiresolució amb mitjana.



\subsection{Consulta d'un interval determinat}
\label{ex:multireoslucio:informacio-subresolucions}

En \textref{ex:multiresolucio:f=op} l'operació de consulta $o$
s'aplica a tota la sèrie temporal $S[-\infty,\infty]$. Ara proposem
d'aplicar-la a un interval concret de la sèrie temporal $S[s,t]$
on $s$ i $t$ són dos instants de temps.  Analitzem l'error de
multiresolució quan l'interval $[s,t]$ és múltiple dels intervals
de multiresolució consolidats, $s=\tau+k\delta$ i
$t=\tau+(k+l)\delta$ on $k,l\in\glssymbol{not:N}$, i quan no ho
és.


\paragraph{L'interval és múltiple dels intervals de multiresolució
  consolidats.} Aleshores $r_1=V(f(S,[s,t]))$ i
$r_2=f(S',[s,t])$ on $S'= \{ \dotsc, (s+\delta,
f'(v_{s},\dotsc,v_{s+1}) ), \dotsc, (t,
f'(v_{t-1},\dotsc,v_{t})), \dotsc \}$. En aquest darrer cas, només
assenyalem els valors que se seleccionen, $S'[s,t]$, els quals
notem amb $v_{s},\dotsc,v_{s+\delta}$ pels valors de les mesures en
$S[s,s+\delta]$ i amb $v_{t-\delta},\dotsc,v_{t}$ pels valor en
$S[t-\delta,t]$.

  Per tant, per a estudiar $d(r_1,r_2)$ es pot analitzar el
  comportament de la funció de resum de l'atribut per als dos casos
  $r_1=f'(v_{s},\dotsc,v_{s+\delta},\dotsc, v_{t-\delta},\dotsc,v_{t})$
  i $r_2=f'(f'(v_{s},\dotsc,v_{s+\delta}),\dotsc,
  f'(v_{t-\delta},\dotsc,v_{t}))$, cosa que és una situació similar a
  la de l'\autoref{ex:multiresolucio:f=op}.


  \paragraph{L'interval no és múltiple dels intervals de
    multiresolució consolidats.}  Per a simplificar la notació,
  suposem $s=\tau$ i $\tau+\delta < t \leq \tau+2\delta$.
  Aleshores $r_1=V(f(S,[s,t]))$ i $r_2=f(S',[s,t])$ on
  $S'= \{(\tau+\delta, f'(v_{i_0},\dotsc,v_{i_1}) ),(\tau+2\delta
  , f'(v_{i_1+1},\dotsc,v_{t} ,\dotsc,v_{i_2})), \dotsc \}$.  Els
  valors s'anoten com a la \autoref{def:informacio:notaciovalors} però
  s'afegeix el valor $v_t$ que assenyala una possible mesura en
  l'instant $t$.  A la
  \autoref{fig:multiresolucio:informacio-subresolucions} es mostren
  els instants de temps i els valors d'aquest exemple, l'interval
  temporal $[s,t]$ de la consulta i l'interval temporal
  $[t,\tau+2\delta]$ que mostra l'error en la consulta a partir
  de la informació emmagatzemada a la multiresolució.


\begin{figure}[tp]
  \centering
     \begin{tikzpicture}
        \begin{axis}[
          ymin = 0,
          yticklabels= {},
          xticklabels={,,$\underset{s}{\tau}$,,$\tau+\delta$,$t$,$\tau+2\delta$},
          ]
          \addplot[ycomb,blue] coordinates {
            (20,10)
            (30,10)
            (40,10)
          }; 
          
 
          \addplot[
          ybar interval, 
          fill=green!25,
          fill opacity=0.5,
          draw=none,
          ] plot coordinates
          {(20,7)(35,7)};

          \addplot[
          ybar interval, 
          fill=orange!75,
          fill opacity=0.5,
          draw=none,
          ] plot coordinates
          {(35,7)(40,7)};


          \addplot[red,mark=*,only marks] coordinates {
            (20,2)
            (30,2)
            (35,2)
            (40,2)
          }; 


          \node[right] at (axis cs:37,5) {$\epsilon$};

          \node[right] at (axis cs:20,3) {$v_{i0}$};
          \node[right] at (axis cs:30,3) {$v_{i1}$};
          \node[right] at (axis cs:30,4) {$v_{i1+1}$};
          \node[right] at (axis cs:35,3) {$v_{t}$};
          \node[left] at (axis cs:40,3) {$v_{i2}$};
        \end{axis}
      \end{tikzpicture}

      \caption{Sèrie temporal amb la consulta desitjada (verd) i l'error de la
        informació no coneguda (taronja)}
  \label{fig:multiresolucio:informacio-subresolucions}
\end{figure}



En aquest cas, cal tenir en compte que per a calcular
$f(S',[s,t])$ s'ha de resoldre la selecció de $S'$ en l'interval $[s,t]$, cosa que es pot realitzar:

  \begin{enumerate}
  \item amb una selecció d'interval,  $S'[s,t]=\{ (\tau+\delta,
    f'(v_{i_0},\dotsc,v_{i_1}) ) \}$,

  \item amb una selecció d'interval temporal,
    $S'[s,t]^r=(\tau+\delta, f'(v_{i_0},\dotsc,v_{i_1}), (t,
    f^r(f'(v_{i_1+1},\dotsc,v_{t} ,\dotsc,v_{i_2}))) ) $ on $f^r$ és
    la interpolació realitzada per la funció de representació $r$

  \item o amb altres casos, podem pensar per exemple amb una funció de
    representació que directament utilitzi el valor de tot el segon
    interval com a vàlid per a $t$, $S'[s,t]^r=(\tau+\delta,
    f'(v_{i_0},\dotsc,v_{i_1}), (t, f'(v_{i_1+1},\dotsc,v_{t}
    ,\dotsc,v_{i_2})) )$ on $f^r$ no hi és perquè seria la funció identitat.
   \end{enumerate}



   Així, per a estudiar $\epsilon=d(r_1,r_2)$ s'ha d'analitzar d'una
   banda $r_1=f'(v_{i_0},\dotsc,v_{i_1},v_{i_1+1},\dotsc,v_{t})$ i de l'altra, depenent de quina de les
   tres seleccions s'utilitzi,
   \begin{enumerate}
   \item $r_2=f'(f'(v_{i_0},\dotsc,v_{i_1}))$, 

   \item
     $r_2=f'(f'(v_{i_0},\dotsc,v_{i_1}),f^r(f'(v_{i_1+1},\dotsc,v_{t},\dotsc,v_{i_2})))$
     \item o $r_2=f'(f'(v_{i_0},\dotsc,v_{i_1}),f'(v_{i_1+1},\dotsc,v_{t},\dotsc,v_{i_2}))$.
\end{enumerate}

És a dir, en la multiresolució consolidada no hi ha disponible la
informació $f'(v_{i_1+1},\dotsc,v_{t})$ que és la que es voldria
consultar.  Per tant hem de concloure que en aquest cas generalment
$\epsilon=d(r_1,r_2)\geq 0$. Tot i així, en la segona selecció
proposada es pot observar que si es coneix exactament el comportament
de la sèrie temporal, per exemple s'estudia mitjançant la teoria del
senyal, aleshores pot ser possible de determinar una funció de
representació que compleixi $ f'(v_{i_1+1},\dotsc,v_{t}) =
f^r(f'(v_{i_1+1},\dotsc,v_{t},\dotsc,v_{i_2}))$ i per tant aconseguir
$r_2=f'(f'(v_{i_0},\dotsc,v_{i_1}), f'(v_{i_1+1},\dotsc,v_{t}) )$;
cosa que ja és una situació similar a la de
l'\autoref{ex:multiresolucio:f=op}..







Retornant, però, al cas que hi ha error $\epsilon=d(r_1,r_2)\geq 0$,
estudiem dos dels atributs descrits a
l'\autoref{ex:multiresolucio:f=op} per tal d'avaluar si és possible
afitar-ne l'error en aquests casos. Així, formulem el cas que caldria
calcular $f'(v_{i_1+1},\dotsc,v_{t})$ però la informació que hi ha
emmagatzemada per a aquest interval és
$f'(v_{i_1+1},\dotsc,v_{t},\dotsc,v_{i_2})$:
\begin{itemize}

\item Màxim. Cal calcular $\max(v_{i_1+1},\dotsc,v_{t})$ però hi ha
  emmagatzemat $\max(v_{i_1+1},\dotsc,v_{t},\dotsc,v_{i_2})$.  Per
  tant l'error en la consulta és $\epsilon=d(r_1,r_2)=
  d(\max(v_{i_1+1},\dotsc,v_{t}),\max(v_{i_1+1},\dotsc,v_{t},\dotsc,v_{i_2}))$. Si
  el màxim es troba a $[s+\delta,t]$ l'error és nul però si hi
  ha un màxim a $[t,s+2\delta]$ aleshores l'error és
  $\epsilon=d(\max(v_{i_1+1},\dotsc,v_{t}),\max(v_{t},\dotsc,v_{i_2}))$,
  el qual no és fitable perquè generalment no podem trobar cap relació
  entre $\max(v_{i_1+1},\dotsc,v_{t})$ i $\max(v_{t},\dotsc,v_{i_2})$.


\item Total: Cal calcular $\sum(v_{i_1+1},\dotsc,v_{t})$ però hi ha
  emmagatzemat $\sum(v_{i_1+1},\dotsc,v_{t},\dotsc,v_{i_2})$. Per tant
  l'error en la consulta és $\epsilon=d(r_1,r_2)=d(
  \sum(v_{i_1+1},\dotsc,v_{t},\dotsc,v_{i_2}),\sum(v_{i_1+1},\dotsc,v_{t}))
  = \sum(v_{t+1},\dotsc,v_{i_2})$. Però $\sum(v_{t+1},\dotsc,v_{i_2})$
  no és un valor emmagatzemat a la multiresolució i per tant no es pot
  saber l'error. Això no obstant, en el cas del total té sentit
  plantejar el cas que la variable mesurada és monòtona creixent (v. a
  continuació l'\autoref{ex:multiresolucio:comptadors}): aleshores es
  compleix que $\sum(v_{i_1+1},\dotsc,v_{t}) \leq
  \sum(v_{i_1+1},\dotsc,v_{t},\dotsc,v_{i_2})$ i per tant es pot fitar
  l'error $\epsilon=d(r_1,r_2) = \sum(v_{t+1},\dotsc,v_{i_2}) \leq
  \sum(v_{i_1+1},\dotsc,v_{t},\dotsc,v_{i_2})$; és a dir com a màxim
  es cometria un error del valor consolidat a $\tau+2\delta$ que
  significaria que en els pitjors del casos tota la mesura hauria
  ocorregut després de $t$.


\end{itemize}






\subsection{Conservació d'informació en comptadors}
  \label{ex:multiresolucio:comptadors}



Els comptadors són un dels trets semàntics que poden tenir les sèries
temporals (\seeref{sec:sgst:tretssemantics}) i se'n pot conservar la
informació aprofitant aquests trets. En aquest cas explorem com
conservar la mitjana de la funció dels comptadors.  Aquest exemple
prové d'una reflexió acurada de per què RRDtool té com a referent els
comptadors.



%\todo{parlar de comptadors monòtons creixents?? o de comptadors en general}

Un comptador monòton creixent és un aparell que mesura l'energia en un
determinat interval de temps. Entre dues lectures successives del
comptador la mesura de l'energia és exacta a diferència d'un aparell
que mesuri potència instantània. D'aquesta només es pot deduir
l'energia exacta si es considera que el senyal es pot reconstruir, per
exemple compleix la freqüència del teorema de Nyquist–Shannon tot i
que a la pràctica és complicat conèixer la freqüència de les variables
mesurades atès que solen ser aleatòries o canvien bruscament. A la
inversa també ocorre el mateix, a partir de la mesura de l'energia
només es pot deduir la potència instantània exacta si es considera que
el senyal es pot reconstruir.  


Els conceptes d'energia i potència solen anar associats a un
determinat tipus de variables físiques contínues; en altres comptadors
els conceptes equivalents són quantitat, comptatge total o increments
per a l'energia i el flux o la velocitat per a la potència.  No totes
les variables físiques són susceptibles de ser mesurades amb un
comptador. Els comptadors es poden aplicar per exemple per a mesurar
energia elèctrica (\seeref{ex:sgst:comptador-electricitat}),
aforaments de trànsit en una carretera, consum d'aigua, etc.

% ha d'aparèixer potència mitjana (que és la línia) i energia (que és l'àrea sota la línia). els aparells tant poden mesurar en un interval energia com potència mitjana i és el mateix, si la mesura mitjana és real i no a partir de la mitjana de diverses instantànies. 



  En resum, l'aparell condiciona la informació que es podrà extreure
  de la mesura; en aquest exemple ens centrem en la informació de
  l'energia i de quina manera la multiresolució és capaç de conservar
  exactament algunes propietats d'aquesta informació.  Per a aquest
  exemple suposem aparells de mesura ideals quan parlem de mesura
  exacta; és a dir que no tenim en compte l'error de precisió o
  d'exactitud de l'aparell.

 



\begin{figure}[tp]
  \centering

     \begin{tikzpicture}
        \begin{axis}[
          title={$E(t)=\int P(t) dt$},
          ymin=0,
          ymax=3,
          xmax=21,
          domain=0:20,
          ylabel=$P(t)$,
          xlabel=$t$,
          xtick=\empty,
          ytick=\empty,
          axis x line=left,
          axis y line=left,
          ]

          % \addplot[const plot, blue,fill=blue,smooth,
          % ] plot coordinates
          % {(0,4)(1,2)(2,6)(3,4)(4,4)};


          \addplot[pattern=crosshatch dots, pattern
          color=blue,draw=blue, samples=500] {2+sin(deg(x))/2} \closedcycle;



          \node at (axis cs:10,1) {$E(t)$};


     \end{axis}
      \end{tikzpicture}  

      \caption{Relació entre l'energia i la potència o la quantitat
        comptada i la velocitat}
  \label{fig:multiresolucio:energia-potencia}
\end{figure}




Així doncs, la definició del problema és la següent.  Sigui $E(t)$
l'energia d'un senyal i $P(t)$ la potència instantània del senyal, es
compleix la relació $E(t)=\int P(t) dt$, la qual es mostra a la
\autoref{fig:multiresolucio:energia-potencia}. %
% Aquesta és la mateixa
% relació $Q(t)=\int v(t) dt$ en altres termes de comptador on $Q$ és la
% quantitat comptada i $v$ és la velocitat instantània, o la mateixa per
% a un cas discret $\Delta Q = \bar{v} \Delta t$ on $\bar{v}$ és la
% velocitat mitjana mesurada durant $\Delta t$. %
%   %Potser el cas discret hauria de ser $\Delta Q = \sum \bar{v}_k \Delta t_k$
Siguin $[x,y]$ i $[y,z]$ dos intervals de temps de mesura, un
comptador mesura exactament el valor de $E_{s}^{t} =
\int_{x}^{y} P(t) dt$ i de $E_{y}^{z} =\int_{y}^{z} P(t) dt$. En canvi,
un aparell de mesura de potència instantània es capaç de mesurar
exactament $P(x)$, $P(y)$ i $P(z)$. Ara bé, a partir del
comptador no es poden deduir exactament $P(x)$, $P(y)$ ni $P(z)$
i a partir de la mesura de la potència instantània no es poden deduir
exactament $\int_{x}^{y} P(t)$ ni $\int_{y}^{z} P(t)$. Tampoc
a partir del comptador es poden deduir exactament energies que no
s'han mesurat, per exemple ni $\int_{(x+y)/2}^{y} P(t)$ ni
$\int_{(x+y)/2}^{z} P(t)$; ara bé sí que serà exacte el càlcul
$\int_{x}^{y} P(t)+\int_{y}^{z} P(t)$.
  

La multiresolució és capaç de conservar aquesta exactitud del
comptatge total.  El comptatge total es pot conservar en un esquema de
multiresolució amb funcions d'agregació per atributs de suma de totals
o bé per atributs de mitjana de la funció. Aquests darrers són els que
permeten, a més, expressar la sèrie temporal resultant de la
multiresolució de forma més coherent amb l'original
(\seeref{sec:sgst:tretssemantics}). Reprenent
l'\autoref{ex:multiresolucio:f=op}, avaluem els atributs de mitjana de
la funció, els quals són semblants als atributs de total però
considerant la sèrie temporal en la representació contínua:
  
  \begin{itemize}


  \item Mitjana de la funció: $f=\operatorname{mitjana}$, segons el
    patró general de mitjana de la funció mostrat a
    \textref{sec:sgstm:mitjanafuncio}, el qual es correspon a calcular
    la mitjana de la funció de representació de la sèrie temporal
    $\frac{1}{b-a} \int_{a}^{b} S(t)dt$ en l'interval tancat $[a,b]$.

    Sigui $t_M=T(\max(S))$ i $t_m=T(\min(S))$ i $S'= \{
    (\tau+\delta, \frac{1}{\delta}
    \int_{\tau}^{\tau+\delta} S(t) dt), (\tau+2\delta,
     \frac{1}{\delta}\int_{\tau+\delta}^{\tau+2\delta} S(t) dt), \dotsc,
    (\tau+n\delta, \frac{1}{\delta}
    \int_{\tau+(n-1)\delta}^{\tau+n\delta} S(t) dt )\}$. %
% , on
%     s'ha aplicat $\delta_0= (\tau_0+\delta_0)-(\tau_0)=\dotsb=
%     (\tau_0+M\delta_0)-(\tau_0+(M-1)\delta_0)$.    
    Els resultats que cal
    calcular són $r_1=\frac{1}{t_M-t_m} \int_{t_m}^{t_M} S(t)dt$ i
    $r_2 = \frac{1}{t_M-t_m} \int_{t_m}^{t_M} S'(t)$.

    Si suposem $\tau=t_m$ i $\tau+n\delta=t_M$ aleshores
    $\int_{t_m}^{t_M} S'(t) = \int_{\tau}^{\tau+\delta} S'(t) dt + \int_{\tau+\delta}^{\tau+2\delta} S'(t) dt + \dotsb + \int_{\tau+(n-1)\delta}^{\tau+n\delta}
    S'(t) dt$.  
    
    Si resolem per exemple el primer interval de consolidació: $
    \int_{\tau}^{\tau+\delta} S'(t) dt = \deltaV(m'_0) =
    \delta \frac{ \int_{\tau}^{\tau+\delta} S(t)
      dt}{\delta}$ on $m'_0$ és la mesura corresponent a la
    consolidació en el primer interval. Així doncs
    $\int_{\tau}^{\tau+\delta} S'(t) dt =
    \int_{\tau}^{\tau+\delta} S(t) dt$, de fet és la propietat
    que resumeix la mitjana de la funció, i per tant podem estendre-ho
    a $\int_{t_m}^{t_M} S'(t)= \int_{t_m}^{t_M} S(t)$. Podem
    concloure, doncs, que $r_1=r_2$.


    A la \autoref{fig:multiresolucio:comptador} es mostra un exemple
    amb valors concrets on $S=\{ (1,4),(2,2),(3,6),(4,4)\}$ i
    l'esquema de multiresolució és
    $\glssymbol{not:esquemaM}=\{(\delta=2,f=\glssymbol{not:sgstm:meanzohe},\tau=0,\glssymbol{not:sgstm:k}=\infty)\}$,
    segons la $\glssymbol{not:sgstm:meanzohe}$ de la
    \textref{def:sgstm:meanzohe}. La sèrie resultant de la
    consolidació de la multiresolució és $S'=\{ (2,3),(4,5)\}$. A la
    figura, les sèries temporals es representen amb representació
    \gls{zohe}, la superfície pintada en blau correspon a l'àrea de
    sota la corba de la sèrie temporal i els nombres interiors
    indiquen el valor d'aquesta àrea; en la sèrie temporal $S$ les
    àrees es corresponen amb els valors de la sèrie temporal perquè
    els intervals són d'una unitat de temps. Així doncs es pot
    observar a $S'$ com aquest esquema de multiresolució conserva el
    comptatge total en la nova resolució, és a dir $6=4+2$ en el
    primer interval consolidat i $10=6+4$ en el segon, i en una
    consulta del comptatge total per a tot l'interval $[0,4]$ s'obté
    $16$ tant en $S$ com a $S'$. Aquesta és la idea bàsica de
    conservació d'informació en els comptadors.



\begin{figure}[tp]
  \centering


     \begin{tikzpicture}
        \begin{axis}[
          width=0.5*\textwidth,
          title=$S$,
          ymin = 0,
          ymax=6,
          ]
          \addplot[
          ybar interval, 
          blue,fill=blue!30!white,
          ] plot coordinates
          {(0,4)(1,2)(2,6)(3,4)(4,4)};


    \node at (axis cs:0.5,2) {$4$};
    \node at (axis cs:1.5,1) {$2$};
    \node at (axis cs:2.5,3) {$6$};
    \node at (axis cs:3.5,2) {$4$};


     \end{axis}
      \end{tikzpicture}\qquad
     \begin{tikzpicture}
        \begin{axis}[
          width=0.5*\textwidth,
          title=$S'$,
          ymin = 0,
          ymax=6,
          ]

          \addplot[
          ybar interval, 
          blue,fill=blue!30!white,
          ] plot coordinates
          {(0,3)(2,5)(4,5)};

    \node at (axis cs:1,1.5) {$6$};
    \node at (axis cs:3,2.5) {$10$};


     \end{axis}
      \end{tikzpicture}


      \caption{Sèrie temporal amb àrea sota la corba i sèrie temporal
        resultant de la multiresolució amb agregació mitjana de la
        funció}
  \label{fig:multiresolucio:comptador}
\end{figure}






  \end{itemize}
  


La multiresolució, però, no pot conservar la resolució del comptatge,
així en l'exemple un cop s'ha consolidat $6=4+2$ no es pot tornar a
obtenir $4$ i $2$ llevat que es pogués reconstruir el senyal.  Com
s'ha exposat a
l'\autoref{ex:multireoslucio:informacio-subresolucions}, en consultes
en què l'interval no es correspongui amb les resolucions
emmagatzemades, els totals no seran els correctes que s'obtindrien de
calcular amb les dades originals.  De fet, és el mateix problema que
hem exposat que a partir d'un comptador no es poden deduir exactament
energies que no s'han mesurat.




% \todo{potser fer un exemple on es vegi com la multiresolució pot solucionar un problema d'inframostreig en els comptadors}




\subsection{Equivalències en l'agregació d'atributs}

Hi ha casos en què és el mateix aplicar una funció d'agregació
d'atributs que aplicar-ne una altra. En aquest apartat particularment
estudiem el cas de la $\glssymbol{not:sgstm:mitjanapd}$
(\seeref{def:sgstm:mitjanapd}) i el de la
$\glssymbol{not:sgstm:meanzohe}$ (\seeref{def:sgstm:meanzohe})
per a sèries temporals regulars.

Sigui $S=\{m_0,\dotsc,m_k\}$ una sèrie temporal
regular de període $p$ (\seeref{def:st:regular}) i sigui $[s,t]$
un interval de temps on $s=T(\min(S))-p=t_0-p$ i
$t=T(\max(S))=t_k$.  Demostrem que
$\glssymbol{not:sgstm:meanzohe}(S,[s,t]) =
\glssymbol{not:sgstm:mitjanapd}(S,[s,t])$ pel que fa al valor
resultant calculat negligint el temps resultant. És a dir en realitat demostrem
$V(\glssymbol{not:sgstm:meanzohe}(S,[s,t])) =
V(\glssymbol{not:sgstm:mitjanapd}(S,[s,t]))$ però no escrivim la
projecció de l'atribut de valor, $V()$, per a no complicar la notació.

La mitjana aritmètica de la sèrie temporal és
$\glssymbol{not:sgstm:mitjanapd}(S,[s,t])=\frac{v_0+\dotsb+v_k}{|S|}$.
%
La mitjana amb representació \gls{zohe} és
$\glssymbol{not:sgstm:meanzohe}(S,[s,t])= \frac{1}{t-s} (
v_0(t_0-s)+v_1(t_1-t_0)+\dotsb+ v_k(t_k-t_{k-1}) )$.

Per ser regular, $t_1-t_0 = \dotsb = t_k-t_{k-1} = p$. A més
$s=t_0-p$ i per tant $t_0-s = p$. També per ser regular, $t_k-t_0=
t_k +(- t_{k-1} + t_{k-1}) + \dotsb + (- t_1 + t_1) - t_0 = (|S|-1)p$
i per tant $t-s=t_k - (t_0 - p) = (|S|-1)d +p = |S|p$.

Reescrivint, $\glssymbol{not:sgstm:meanzohe}(S,[s,t])=
\frac{1}{|S|p} ( v_0p+v_1p+\dotsb+ v_kp ) = \frac{1}{|S|} (
v_0+v_1+\dotsb+ v_k ) = \glssymbol{not:sgstm:mitjanapd}(S,[s,t])$.



Així doncs, en una sèrie temporal regular es pot aplicar la
$\glssymbol{not:sgstm:mitjanapd}$ com a equivalent a la
$\glssymbol{not:sgstm:meanzohe}$; de fet la
\autoref{fig:multiresolucio:comptador} n'és un exemple. Un àmbit
d'aplicació d'aquestes equivalències pot ser el de simplificar els
càlculs en les consultes als \gls{SGSTM}, en els discs dels quals les
subsèries temporals normalment s'emmagatzemen regulars.
 








%%% Local Variables:
%%% TeX-master: "main"
%%% End:


%  LocalWords:  multiresolució



\part{Experimentació}
%-- Implementacions --

%\part{Experimentació}


\chapter{Introducció a les implementacions}


Es realitzen implementacions a alt nivell per a observar el funcionament a nivell acadèmic: Python.

Es realitzen implementacions a baix nivell d'estructures específiques: VHDL




















%%% Local Variables:
%%% TeX-master: "main"
%%% End:




\chapter{Implementació amb Python}


\todo{mirar}
\url{http://en.wikipedia.org/wiki/Design_Patterns}


La implementació de referència dels models de SGST i SGSTM es realitza
amb Python \parencite{python:doc2}. L'objectiu d'aquesta implementació
de referència és mantenir la fidelitat al model per tal de poder
experimentar-hi amb tota la potència matemàtica .

En la implementació s'afegeixen alguns operadors que en el model no
estaven explícitament definits perquè són propis de l'àlgebra de
conjunts. Alguns dels operadors principals que s'han d'afegir són els
relacionats amb la notació de creació de
conjunts, %set-builder notation (set comprehension)
els quals en els SGBD s'inclouen en el que es coneix com a
\emph{llenguatge de definició de dades} (DDL, de l'anglès \emph{data
  definition language}).
% El model s'ha descrit utilitzant àlgebra de conjunts. Així moltes operacions no s'han hagut de definir perquè formen part de la notació de conjunts: definició de nous conjunts (set builder notation), operació d'assignació, manipulació de les dades amb inserció, modificació, esborrar (perquè en àlgebra de conjunts es treballa amb immutabilitat ja que a l'àlgebra crear un nou conjunt no té cap cost), etc.  A les implementacions, però, totes aquestes operacions s'han de definir en cas que es vulguin.





Implementem els dos models de SGST i SGSTM com a dues biblioteques
diferents: \emph{Pytsms} i \emph{RoundRobinson} respectivament. La
RoundRobinson, però, té una forta dependència en la
Pytsms de la mateixa manera que hem definit els SGSTM en base
als SGST.


Utilitzem orientació a objectes. Ens permet fer explicita la relació
entre la implementació i el model. \todo{més explicat}

Utilitzem diagrames UML per a definir l'estructura de classes. Principalment volem mostrar les relacions entre les diverses classes.\todo{més explicat}


Implementem la part essencial, és a dir l'àlgebra definida en els
models lògics. No implementem els complements habituals dels SGBD, els
quals són necessaris en entorns d'explotació, com per exemple gestió
d'usuaris i permisos, còpies de seguretat, llenguatges estàndards de
consulta, etc.
Tampoc es tenen en compte paràmetres de rendiment; per exemple no es té en compte si hi ha dues subsèries resolució que tenen el mateix pas de consolidació i les funcions d'agregació d'atributs tenen la mateixa funció de representació --per exemple mitjanaZOHE i màximZOHE\todo{símbols}--,   podrien aprofitar la mateixa operació de selecció temporal d'interval ZOHE.



\section{Pytsms}

La biblioteca Pytsms implementa un SGST de referència. Així
doncs, seguint el model, els objectes principals són les mesures, les
sèries temporals i les representacions de les sèries temporals. Tots
tres s'implementen respectivament com a classes \emph{Measure},
\emph{TimeSeries} i \emph{Representation}.


\begin{figure}[tp]
  \centering
  \begin{tikzpicture}

  %Timeseries
  \umlclass[x=0,y=0] {TimeSeries}{}{}  

  % Measure
  \umlclass[x=-4] {Measure}{}{}
  \umluniaggreg[mult=0..*]  {TimeSeries}{Measure}
  \umlclass[x=-5.2,y=-2] {MFloat}{}{}
  \umlclass[x=-2.8,y=-2] {MChar}{}{}
  \umlinherit {MFloat}{Measure}
  \umlinherit {MChar}{Measure}

  %Repr
  \umlclass[x=4] {Representation}{}{} %,type=abstract
  \umlassoc[mult1=1,mult2=1]  {TimeSeries}{Representation}
  \umlclass[x=3,y=-2] {Zohe}{}{}
  \umlclass[x=5,y=-2] {Delta}{}{}
  \umlinherit {Zohe}{Representation}
  \umlinherit {Delta}{Representation}

  %Associacions
  \umlclass[x=-1.5,y=-4] {RegularProp}{}{}
  \umluniassoc  {RegularProp}{TimeSeries}
  \umlclass[x=1.5,y=-4] {Storage}{}{}
  \umluniassoc {Storage}{TimeSeries}

  %Dependencies
  \umlemptypackage[x=5,y=-5]{Matplotlib}
  \umldep{Zohe}{Matplotlib}
  \umldep{Delta}{Matplotlib}


  \end{tikzpicture}



  \caption{Diagrama UML de Pytsms}
  \label{fig:implementacio:pytsms-uml}
\end{figure}





\begin{figure}
  \centering

\begin{tikzpicture}

  %Timeseries
  \umlclass[x=0,y=0] {TimeSeries}{}{}  
  %Realisations 
  \umlclass[x=-3.5,y=-3] {Structure}{}{}
  \umlclass[x=-1.2,y=-3] {OpSet}{}{}
  \umlclass[x=1.2,y=-3] {OpSeq}{}{}
  \umlclass[x=3.5,y=-3] {OpFunc}{}{}
  %\umlreal[geometry=|-|]{Structure}{TimeSeries}
  \umlinherit[geometry=|-|]{TimeSeries}{Structure}
  \umlinherit[geometry=|-|]{TimeSeries}{OpSet}
  \umlinherit[geometry=|-|]{TimeSeries}{OpSeq}
  \umlinherit[geometry=|-|]{TimeSeries}{OpFunc}
  %Subrealisations
  \umlclass[x=-3,y=-6] {SetNoTemporal}{}{}
  \umlclass[x=0.7,y=-6] {SetTemporal}{}{}
  \umlclass[x=4,y=-6] {SetRelacional}{}{}
  \umlinherit[geometry=|-|]{OpSet}{SetNoTemporal}
  \umlinherit[geometry=|-|]{OpSet}{SetTemporal}
  \umlinherit[geometry=|-|]{OpSet}{SetRelacional}
  \umlclass[x=-6,y=-6,type=python] {set}{}{}
  \umlinherit{Structure}{set}

\end{tikzpicture}

  \caption{Diagrama UML de la realització de sèries temporals a Pytsms}
  \label{fig:implementacio:pytsms-uml-ts}
\end{figure}





La \autoref{fig:implementacio:pytsms-uml} mostra amb un diagrama
UML la relació entre aquests tres objectes principals. Així, per una
banda, una \emph{TimeSeries} té una relació d'agregació amb les
\emph{Measure}, és a dir que una sèrie temporal conté cap, una o més
d'una mesura.  Per altra banda, les sèries temporals i les
representacions són ortogonals i això s'implementa mitjançant una
relació d'associació bidireccional entre una \emph{TimeSeries} i una
\emph{Representation}, és a dir que una instància de sèrie temporal té
associada una representació i una instància de representació coneix la
sèrie temporal que representa.




Una \emph{TimeSeries} és un objecte amb una gran quantitat de
mètodes. Com a conseqüència, la implementació de funcionalitats
essencials s'ha dividit en diversos objectes, els quals es mostren a
la \autoref{fig:implementacio:pytsms-uml-ts}. Els mètodes que
implementen el model estructural i el model d'operacions bàsiques
s'han agrupat en objectes segons la seva funcionalitat. Així hi ha
l'objecte \emph{Structure} que implementa el model estructural de les
sèries temporals, l'\emph{OpSet} pel model d'operacions de conjunts,
l'\emph{OpSeq} pel model d'operacions de seqüències i l'\emph{OpFunc}
pel model d'operacions de funció temporal.  Aleshores l'objecte
\emph{TimeSeries} multihereta les funcionalitats d'aquests quatre
objectes, cosa que s'implementa a Python com a
\emph{Mixin} \parencite[\S 8.3.6, \S 20.17]{python:doc2}.  L'objecte
\emph{OpSet} també té una gran quantitat de mètodes i, de la mateixa
manera, hereta la funcionalitat de tres objectes: el
\emph{SetNoTemporal} per a les operacions basades en l'orde parcial de
les sèries temporals, el \emph{SetTemporal} basat en l'ordre temporal
i el \emph{SetRelacional} per a les operacions específiques de
l'àlgebra relacional. Pel que fa a l'\emph{Structure} hereta
funcionalitats dels objectes \emph{set}, que són un tipus predefinit a
Python \parencite[\S 5.7]{python:doc2}.


Les funcionalitats complementàries de les \emph{TimeSeries} s'han
implementat amb relacions d'associació unidireccionals, les quals es
mostren a la \autoref{fig:implementacio:pytsms-uml}. Així, hi ha dues
funcionalitats complementàries: \emph{RegularProperties} és un objecte
que agrupa les operacions relacionades amb la regularitat de les
sèries temporals i \emph{Storage} agrupa les operacions
d'emmagatzematge i de recuperació en fitxers. En aquests casos,
l'associació unidireccional indica que són objectes que treballen
sobre una \emph{TimeSeries} i s'implementa a Python seguint el patró
de disseny
\emph{Visitor} \parencite[cap.~14]{ziade08:expert_python_programming}. Amb
aquest patró les \emph{TimeSeries} esdevenen \emph{Visitable}, és a
dir accepten objectes \emph{Visitor} que aporten funcionalitats
extres. Així doncs, els objectes \emph{RegularProperties} i
\emph{Storage} són \emph{Visitor}.


La \autoref{fig:implementacio:pytsms-uml} mostra exemples
d'especialitzacions de les mesures i de les representacions.
%
Pel que fa a les \emph{Measure}, poden tenir especialitzacions segons
els tipus dels atributs de temps i de valor. Amb aquesta relació
implementem la propietat homogènia de les sèries temporals i la
definició de mesura indefinida i de valor indefinit, és a dir que
totes les mesures que conté una sèrie temporal són del mateix tipus i
cada tipus de mesura té uns valors de l'atribut temps que la
defineixen indefinida i uns valors de l'atribut valor que la
defineixen de valor indefinit.  Així, per defecte, una \emph{Measure}
defineix els reals $-\infty$ i $+\infty$ per a les mesures indefinida
negativa i positiva respectivament, i defineix el valor \emph{None} de
Python per a la mesura de valor indefinit. Aleshores, mitjançant
especialitzacions es poden definir altres tipus de mesures; per
exemple la \emph{MFloat} que defineix el real $\infty$ com a valor
indefinit o bé la \emph{MChar} que defineix mesures de tipus caràcter.




Pel que fa a les representacions, cada representació en concret és una
especialització de \emph{Representation}. Per exemple \emph{Zohe} i
\emph{Delta} implementen la funció de representació ZOHE i la delta
respectivament.
%\emph{Representation} és una classe abstracta?
Bàsicament, cada representació particular ha de definir l'operació que
calcula l'interval temporal i l'operació que permet trobar-ne el graf.
També cadascuna implementa una operació que dibuixi correctament el
gràfic de la sèrie temporal segons la representació; en les
representacions definides s'usa la biblioteca
\emph{Matplotlib} \parencite{python:matplotlib}  per a fer els gràfics.




\subsubsection{Encaix a Python}

\todo{}

Explicar més detalladament quins mètodes tenen les sèries temporals: unió, selecció, etc.

Heretem de sets, implementem els mètodes especials de sets

implementem els mètodes especials de seqüències







\section{RoundRobinson}

La biblioteca RoundRobinson implementa un SGSTM de referència. Així
doncs, seguint el model, els objectes principals són les sèries
temporals multiresolució, les subsèries resolució, els buffers, els
discs i les funcions d'agregació d'atributs. Respectivament
s'implementen com a classes \emph{MultiresolutionSeries},
\emph{Resolution}, \emph{Buffer}, \emph{Disc} i \emph{Function}.
Així, les funcions d'agregació d'atributs són realitzades per
\emph{Function} de Python.


\begin{figure}[tp]
  \centering

\begin{tikzpicture}

  %MultiTimeseries
  \umlclass[x=0,y=0] {MultiresolutionSeries}{}{}  
  \umlclass[x=-4,y=0,type=python] {set}{}{}
  \umlinherit{MultiresolutionSeries}{set}
  %Components 
  \umlclass[x=0,y=-3] {Resolution}{}{}
  \umluniaggreg  {MultiresolutionSeries}{Resolution}
  %SubComponents 
  \umlclass[x=-1.2,y=-6] {Buffer}{}{}
  \umlclass[x=1.2,y=-6] {Disc}{}{}
  \umlclass[x=-3,y=-9,template={s,i},type=interface] {Function}{}{}
  \umlunicompo[mult=1]  {Resolution}{Buffer}
  \umlunicompo[mult=1]  {Resolution}{Disc}
  \umluniassoc[mult=1]  {Buffer}{Function}

  %TimeSeries
  \begin{umlpackage}[x=1,y=-9]{Pytsms}
    \umlclass{TimeSeries}{}{}  
  \end{umlpackage}
  \umluniassoc[mult=1]  {Buffer}{TimeSeries}
  \umluniassoc[mult=1]  {Disc}{TimeSeries}

  %Associacions
  \umlclass[x=4,y=-3] {Storage}{}{}
  \umluniassoc {Storage}{MultiresolutionSeries}
  \umlclass[x=4,y=-1] {Plot}{}{}
  \umluniassoc {Plot}{MultiresolutionSeries}

\end{tikzpicture}

  \caption{Diagrama UML de RoundRobinson}
  \label{fig:implementacio:roundrobinson-uml}
\end{figure}



La \autoref{fig:implementacio:roundrobinson-uml} mostra amb un
diagrama UML la relació entre aquests cinc objectes principals. Així,
una \emph{MultiresolutionSeries} té una relació d'agregació amb les
\emph{Resolution}, és a dir que una sèrie temporal multiresolució
conté subsèries resolucions.  Una \emph{Resolution} té una relació de
composició amb un \emph{Buffer} i una altra amb un \emph{Disc}, és a
dir que cada subsèries resolució està formada exactament per un buffer
i un disc. Cada \emph{Buffer} té una relació d'associació amb una
\emph{TimeSeries}, és a dir amb la sèrie temporal del buffer; de
manera similar per la sèrie temporal del disc cada \emph{Disc}
s'associa a una \emph{TimeSeries}. A més, cada \emph{Buffer} també té
una relació d'associació amb una \emph{Function} que ha de tenir dos
paràmetres: la sèrie temporal (\emph{s}) i l'interval de consolidació
(\emph{i}).




Les \emph{MultiresolutionSeries} tenen funcionalitats complementàries
que s'han implementat amb relacions d'associació
unidireccionals. Així, hi ha dues funcionalitats complementàries:
\emph{Plot} per a les operacions relacionades amb la visualització
gràfica i \emph{Storage} per a operacions d'emmagatzematge i de
recuperació en fitxers.



Les \emph{MultiresolutionSeries} com a conjunts formats per
\emph{Resolution} s'han implementat heretant funcionalitats dels
\emph{Set} de Python.

\todo{}
explicar més detalladament les operacions de les Multiresolution: add, addResolution, consolidation, etc. i sobretot les totalConsult i les discConsult.





\section{Exemples d'ús}


\todo{}


Amb les biblioteques Pytsms i RoundRobinson podem treballar amb les sèries temporals i les sèries temporals multiresolució. 

Un cas d'exemple és demostrar l'equivalència entre l'operació multiresolució dels SGST sobre una sèrie temporal i la sèrie temporal resultant d'una consolidació dels SGSTM. 












%%% Local Variables:
%%% TeX-master: "main"
%%% End:
\chapter{Implementació amb para\l.lelisme}

En el capítol \todo{ref al capítol} hem definit la funció de
multiresolució com una funció sobre una sèrie temporal. Aquesta funció
té bàsicament dues parts: un plec sobre un esquema de multiresolució i
mapes sobre la sèrie temporal. Ara implementem
aquesta funció mitjançant computació para\l.lela.


Una tècnica de computació para\l.lela és
MapReduce \parencite{deanghemawat04:mapreduce}, la qual s'adequa bé al
problema ja que es basa en aplicar operacions de mapa (\emph{map}) i
posteriorment plegar-les (\emph{reduce}). Un sistema que es basa
exclusivament en aquesta tècnica és
Hadoop \parencite{hadoop}.

A continuació, en primer lloc, estudiem Hadoop i la tècnica MapReduce.
En segon lloc, implementem usant Hadoop un \gls{SGSTM} anomenat
\emph{RoundRobindoop}. Aquest és un \gls{SGSTM} específic amb
l'objectiu de mostrar una implementació que resolgui la multiresolució
d'una sèrie temporal en temps diferit (\emph{offline}) i computant
para\l.lelament.




\section{Hadoop i MapReduce}


Apache Hadoop, o simplement Hadoop, \parencite{hadoop} és un sistema
de computació distribuïda que permet processar grans volums de dades
amb diferents computadors en para\l.lel. El sistema inclou la gestió
de l'emmagatzematge per a distribuir les dades als diferents
computadors, la qual cosa s'anomena \gls{HDFS}; la gestió dels
diferents processos en els diversos computadors; i el model de
programació para\l.lela, el qual és MapReduce.


MapReduce \parencite{deanghemawat04:mapreduce,lammel08:mapreduce} és
un model de programació per processar algoritmes en para\l.lel. Es
basa en resoldre els algoritmes en dues etapes: primer en una etapa de
\emph{maps} i segon en una etapa de \emph{reduces}.  Aquestes dues
etapes són l'algoritme bàsic i per això s'anomena MapReduce, tot i que
hi ha variacions que afegeixen més etapes.  Els noms de \emph{map} i
\emph{reduce} també s'usen per a les operacions d'alt ordre, com les
mapa (\emph{map}) i plec (\emph{fold} o també \emph{reduce}), que hem
definit en el model dels \gls{SGST}, però
\textcite{lammel08:mapreduce} compara les de MapReduce amb les d'alt
ordre i conclou que no són exactament el mateix; aquí distingirem els
conceptes usant els noms en anglès map i reduce per a MapReduce.


A la~\autoref{fig:mapreduce:esquema}
es mostra l'esquema de funcionament de MapReduce, que és el següent:


\begin{figure}[tp]
  \centering
  \chapter{Implementació amb para\l.lelisme}

En el capítol \todo{ref al capítol} hem definit la funció de
multiresolució com una funció sobre una sèrie temporal. Aquesta funció
té bàsicament dues parts: un plec sobre un esquema de multiresolució i
mapes sobre la sèrie temporal. Ara implementem
aquesta funció mitjançant computació para\l.lela.


Una tècnica de computació para\l.lela és
MapReduce \parencite{deanghemawat04:mapreduce}, la qual s'adequa bé al
problema ja que es basa en aplicar operacions de mapa (\emph{map}) i
posteriorment plegar-les (\emph{reduce}). Un sistema que es basa
exclusivament en aquesta tècnica és
Hadoop \parencite{hadoop}.

A continuació, en primer lloc, estudiem Hadoop i la tècnica MapReduce.
En segon lloc, implementem usant Hadoop un \gls{SGSTM} anomenat
\emph{RoundRobindoop}. Aquest és un \gls{SGSTM} específic amb
l'objectiu de mostrar una implementació que resolgui la multiresolució
d'una sèrie temporal en temps diferit (\emph{offline}) i computant
para\l.lelament.




\section{Hadoop i MapReduce}


Apache Hadoop, o simplement Hadoop, \parencite{hadoop} és un sistema
de computació distribuïda que permet processar grans volums de dades
amb diferents computadors en para\l.lel. El sistema inclou la gestió
de l'emmagatzematge per a distribuir les dades als diferents
computadors, la qual cosa s'anomena \gls{HDFS}; la gestió dels
diferents processos en els diversos computadors; i el model de
programació para\l.lela, el qual és MapReduce.


MapReduce \parencite{deanghemawat04:mapreduce,lammel08:mapreduce} és
un model de programació per processar algoritmes en para\l.lel. Es
basa en resoldre els algoritmes en dues etapes: primer en una etapa de
\emph{maps} i segon en una etapa de \emph{reduces}.  Aquestes dues
etapes són l'algoritme bàsic i per això s'anomena MapReduce, tot i que
hi ha variacions que afegeixen més etapes.  Els noms de \emph{map} i
\emph{reduce} també s'usen per a les operacions d'alt ordre, com les
mapa (\emph{map}) i plec (\emph{fold} o també \emph{reduce}), que hem
definit en el model dels \gls{SGST}, però
\textcite{lammel08:mapreduce} compara les de MapReduce amb les d'alt
ordre i conclou que no són exactament el mateix; aquí distingirem els
conceptes usant els noms en anglès map i reduce per a MapReduce.


A la~\autoref{fig:mapreduce:esquema}
es mostra l'esquema de funcionament de MapReduce, que és el següent:


\begin{figure}[tp]
  \centering
  \chapter{Implementació amb para\l.lelisme}

En el capítol \todo{ref al capítol} hem definit la funció de
multiresolució com una funció sobre una sèrie temporal. Aquesta funció
té bàsicament dues parts: un plec sobre un esquema de multiresolució i
mapes sobre la sèrie temporal. Ara implementem
aquesta funció mitjançant computació para\l.lela.


Una tècnica de computació para\l.lela és
MapReduce \parencite{deanghemawat04:mapreduce}, la qual s'adequa bé al
problema ja que es basa en aplicar operacions de mapa (\emph{map}) i
posteriorment plegar-les (\emph{reduce}). Un sistema que es basa
exclusivament en aquesta tècnica és
Hadoop \parencite{hadoop}.

A continuació, en primer lloc, estudiem Hadoop i la tècnica MapReduce.
En segon lloc, implementem usant Hadoop un \gls{SGSTM} anomenat
\emph{RoundRobindoop}. Aquest és un \gls{SGSTM} específic amb
l'objectiu de mostrar una implementació que resolgui la multiresolució
d'una sèrie temporal en temps diferit (\emph{offline}) i computant
para\l.lelament.




\section{Hadoop i MapReduce}


Apache Hadoop, o simplement Hadoop, \parencite{hadoop} és un sistema
de computació distribuïda que permet processar grans volums de dades
amb diferents computadors en para\l.lel. El sistema inclou la gestió
de l'emmagatzematge per a distribuir les dades als diferents
computadors, la qual cosa s'anomena \gls{HDFS}; la gestió dels
diferents processos en els diversos computadors; i el model de
programació para\l.lela, el qual és MapReduce.


MapReduce \parencite{deanghemawat04:mapreduce,lammel08:mapreduce} és
un model de programació per processar algoritmes en para\l.lel. Es
basa en resoldre els algoritmes en dues etapes: primer en una etapa de
\emph{maps} i segon en una etapa de \emph{reduces}.  Aquestes dues
etapes són l'algoritme bàsic i per això s'anomena MapReduce, tot i que
hi ha variacions que afegeixen més etapes.  Els noms de \emph{map} i
\emph{reduce} també s'usen per a les operacions d'alt ordre, com les
mapa (\emph{map}) i plec (\emph{fold} o també \emph{reduce}), que hem
definit en el model dels \gls{SGST}, però
\textcite{lammel08:mapreduce} compara les de MapReduce amb les d'alt
ordre i conclou que no són exactament el mateix; aquí distingirem els
conceptes usant els noms en anglès map i reduce per a MapReduce.


A la~\autoref{fig:mapreduce:esquema}
es mostra l'esquema de funcionament de MapReduce, que és el següent:


\begin{figure}[tp]
  \centering
  \input{imatges/implementacio/mapreduce.tex}
  \caption{Esquema de funcionament de MapReduce}
  \label{fig:mapreduce:esquema}
\end{figure}



\begin{enumerate}

\item Hi ha unes dades originals que es poden partir en
  trossos. Hadoop està orientat a fitxers, mitjançant \gls{HDFS}, i
  per tant cada tros de dades és cadascun dels fitxers que es volen
  processar o bé conjunts de línies d'un fitxer.

\item Cada tros de les dades es processa mitjançant una operació
  map. Cada map es pot computar en para\l.lel i distribuït.

\item Cada operació map ha de retornar un nou conjunt de dades
  formats per parelles d'identificador i valor. Aquests conjunts de
  dades s'ordenen per identificador. 

\item Cada conjunt de dades amb el mateix identificador es processa
  mitjançant una operació reduce. Cada reduce es pot computar en
  para\l.lel i distribuït.

\item Cada reduce ha de retornar un tros del resultat final. És a dir,
  que unint les dades que retornen els reduce s'obtenen les dades
  finals. En l'orientació a fitxers de Hadoop, el resultat final és un
  fitxer, o bé un tros d'un fitxer, per a cada reduce.

\end{enumerate}



Per a resoldre un algoritme amb MapReduce, cal definir l'operació de
map i l'operació de reduce. El map ha de calcular un filtre sobre les
dades, sobretot establir grups de dades, i el reduce ha de calcular
agregacions o resums per a cada grup. MapReduce té una gran similitud
amb l'operació \emph{summarize} dels
\gls{SGBDR} \parencite[cap.~7]{date04:introduction8}, però separada
convenientment en les dues etapes.  A més, MapReduce imposa les
següents restriccions: les dades s'han de poder partir i l'algoritme
s'ha de poder expressar separat en les dues operacions de map i de
reduce.  \textcite{deanghemawat04:mapreduce} mostren exemples
d'algoritmes que es poden expressar amb MapReduce.



Un cop s'ha modelat un algoritme amb MapReduce, aleshores Hadoop ja és
capaç d'executar els maps i els reduces en para\l.lel i distribuïts. A
més, Hadoop també gestiona el compromís dels recursos entre el temps
de distribuir les dades, la quantitat de processos en para\l.lel que
s'han de crear i el temps afegit que suposa cada procés nou.








\section{RoundRobindoop}


\todo{}

RoundRobindoop implementa un \gls{SGSTM} específic que resol la funció
de multiresolució amb el model de programació MapReduce. 
\todo{atenció només resol l'operació de Dmap!}

Primer, dissenyem les operacions de MapReduce per a la multiresolució. Segon, 



\subsection{Multiresolució amb MapReduce}

L'algoritme que implementem amb MapReduce és el de la funció de
multiresolució definida a la secció \todo{ref secció
  sec:multiresolucio:funcio}. Aquesta funció principalment té dues
parts, una de mapa i una de plec, que com ja s'ha dit no es poden
correspondre exactament amb les operacions de map i de reduce. 
 Així
doncs, dissenyem les operacions de map i de reduce que tenen el mateix
efecte que calcular la funció de multiresolució, en què el resultat
final no és la sèrie temporal total sinó el resultat de totes les
funcions $\glssymbol{not:sgstm:dmap}$, és a dir el resultat final són
les sèries temporals dels discs del model de \gls{SGSTM} la
concatenació dels quals resulta en la sèrie temporal total.



Sigui $S=\{m_0,m_1,\dotsc,m_k\}$ una sèrie temporal, i $e = \{
(\delta_0,f_0,\tau_0,k_0),\ldots, (\delta_d,f_d,\tau_d,k_d)\}$ els
paràmetres d'un esquema de multiresolució, definim l'algoritme
MapReduce que calcula $\operatorname{mapreduce}(S,e) = \{
(\delta_0,f_0,
\glssymbol{not:sgstm:dmap}(S,\delta_0,f_0,\tau_0,k_0)),\dotsc,
(\delta_c,f_c,\glssymbol{not:sgstm:dmap}(S,\delta_c,f_c,\tau_c,k_c))
\}$. És a dir, calcula tots els $\glssymbol{not:sgstm:dmap}$ possibles
i els identifica amb el pas de consolidació $\delta$ i la funció
d'agregació d'atributs $f$, els quals identifiquen les subsèries
resolució assumint que no n'hi ha de repetits.
%i que per tant tenenassociats el cardinal màxim $k$ i un instant de consolidació $\tau$.





L'esquema de funcionament de RoundRobindoop és el de
la~\autoref{fig:roundrobindoop:esquema}, el qual és la implementació
particular de l'esquema de la~\autoref{fig:mapreduce:esquema} per a la
multiresolució. A continuació expliquem com són les dades originals,
les d'entremig i les finals, i per tant quina operació realitzen els
map i els reduce.





\begin{figure}[tp]
  \centering

  \begin{tikzpicture}

      \tikzset{
        mynode/.style={rectangle,rounded corners,draw=black, 
          very thick, inner sep=1em, minimum size=3em, text centered,
          groc},
        myarrow/.style={->, shorten >=1pt, thick},
        mylabel/.style={text width=7em, text centered},
        groc/.style={top color=white, bottom color=yellow!50},
        verd/.style={top color=white, bottom color=green!50},
        roig/.style={top color=white, bottom color=red!50},
      }  




 \node[mynode,verd] (m1) {map 1};
 \node (ml1) [below=of m1] {};
 \node[mynode,verd] (m2) [below=of ml1] {map 2};
 \node (ml2) [below=of m2] {};
 \node[mynode,verd,dotted] (mn) [below=of ml2] {map M};


 \node[mynode] (d1) [right=of m1] {
   \begin{tabular}{|c|cc|}\hline
     \multirow{2}{*}{$B$} & \multicolumn{2}{|c|}{valor} \\\cline{2-3}
       & \multicolumn{1}{|c|}{t} & v \\\hline
      & & \\
      & &\\\hdashline
      & &\\
      & &\\\hline
   \end{tabular}
 };

 \node[mynode] (d2) [right=of m2] {
   \begin{tabular}{|c|cc|}\hline
     \multirow{2}{*}{$B$} & \multicolumn{2}{|c|}{valor} \\\cline{2-3}
       & \multicolumn{1}{|c|}{t} & v \\\hline
      & &\\\hdashline
      & &\\\hline
   \end{tabular}
 };

 \node[mynode,dotted] (dn) [right=2cm of mn] {};



 \node[mynode,verd] (r1) [right=of d1] {reduce 1};
 \node[mynode,verd] (r2) [below=2cm of r1] {reduce 2};
 \node[mynode,verd,dotted] (rn) [right=2cm of dn] {};




 \node[mynode] (o) [above left=of m1, anchor=north east] {
   \begin{tabular}{|cc|}\hline
     t & v \\\hline
      & \\
      & \\
     \parbox[c][7cm][s]{0cm}{\vfill}& \\
      & \\
      & \\\hline
   \end{tabular}
 };
 \node [above=0cm of o] {$S$ original};
 \node[mynode,minimum height=10cm] (f) [above right=of r1, anchor=north west] {
   \begin{tabular}{|cc|}\hline
    $\delta\, f$ & t v \\\hline
      & \\
       & \\
     \parbox[c][7cm][s]{0cm}{\vfill}&  \\
       & \\
       & \\\hline
   \end{tabular}
};
 \node [above=0cm of f] {final};




 \draw[dashed] (ml1) -- (ml1-|o.west);
 \draw[dashed] (ml2) -- (ml2-|o.west);

 \draw[myarrow] (o.center|-m1) -- (m1)  node[sloped,above,near end] {$S'_1$};
 \draw[myarrow] (o.center|-m2) -- (m2) node[sloped,above,near end] {$S'_2$};
 \draw[myarrow,dotted] (o.center|-mn) -- (mn) node[sloped,above,near end] {$S'_M$};


 \draw[myarrow] (m1) -- (d1);
 \draw[myarrow] (m2) -- (d2);
 \draw[myarrow,dotted] (mn) -- (dn);

 \draw[myarrow] (d1.center) -- (r1);
 \node (d1b) [above=3ex of d1.280] {}; 
 \draw[myarrow] (d1b.center) -- (r2);

 \draw[myarrow] (d2.center) -- (r1);
 \node (d2b) [above=3ex of d2.280] {}; 
 \draw[myarrow] (d2b.center) -- (r2);

 \draw[myarrow,dotted] (dn) -- (r1) ;
 \draw[myarrow,dotted] (dn) -- (r2) ;
 \draw[myarrow,dotted] (dn) -- (rn) ;


 \node (r1l) [below=of r1] {};
 \draw[dashed] (r1l) -- (r1l-|f.east);
 \node (r2l) [below=of r2] {};
 \draw[dashed] (r2l) -- (r2l-|f.east);


 \draw[myarrow] (r1) -- (r1-|f.center) ;
 \draw[myarrow] (r2) -- (r2-|f.center) ;
 \draw[myarrow,dotted] (rn) -- (rn-|f.center) ;


  \end{tikzpicture}
  
  \caption{Esquema de funcionament de RoundRobindoop}
  \label{fig:roundrobindoop:esquema}
\end{figure}






Les dades originals són la sèrie temporal $S$ és a dir un conjunt de
parelles de temps i valor, a les quals anomenem mesures. Un sèrie
temporal es pot partir en trossos on cada tros és un subconjunt de
mesures, és a dir una subsèrie temporal. Per tant, cada operació map
rep una subsèrie temporal de l'original: $S'_1 =
\{m_0,\dotsc,m_{o1}\}, S'_2 = \{m_{o1+1},\dotsc,m_{o2}\}, \dotsc, S'_M
= \{\dotsc,m_{k}\}$ on $M$ són el nombre de maps i $o_1 < o_2 < k$.



Les dades finals són les sèries temporals dels discs, és a dir les
mesures consolidades per a cada subsèrie resolució. Així doncs, les
dades finals vistes com a conjunt són un conjunt de dades
consolidades $R=\{ D'_{0}, \dotsc, D'_r\}$ on cada dada consolidada
és un tuple $D'=(\delta,f,m')$ que indica quina mesura $m'=(t,v)$ és i a
quin disc pertany identificat per $\delta$ i $f$.  Per tant, cada
operació reduce, assumint que hi ha un
reduce per cada resolució, calcula un subconjunt de $R$: $R'_1 = \{(\delta_0,f_0,m'_0),\dotsc,(\delta_0,f_0,m'_{k1})\}$

$\{m'_0,\dotsc,m'_{r1}\}, R'_2 = \{m'_{r1+1},\dotsc,m'_{r2}\}, \dotsc,
R'_d = \{\dotsc,m'_{r}\}$ on $d$ són el nombre de reduces
i $(\delta_0,f_0,m'_0)$   $(\delta_0,f_0,m'_{k1})$  






, per tant  $d=|e|$ 


 i $r_1 < r_2
< r$.  Per a l'esquema de multiresolució $e$, i assumint que hi ha un
reduce per cada resolució, $d=|e|$ 

El nombre total de mesures consolidades $r$ és afitat ja
que per a l'esquema de multiresolució $e$ és $r+1 \leq k_0+\dotsb+k_d$.










\subsection{Execució de l'algoritme}

Proposem dues maneres per a executar l'algoritme implementat amb
MapReduce: Hadoop i shell.  \todo{també es pot executar fora de Hadoop
  via pipeline del shell}


A Hadoop:\todo{}
Només farem una configuració de Single Node Setup. 
Després es podria estendre de forma senzilla a una Cluster Setup, on
només caldria decidir com distribuir les dades i els processos de map
i reduce als diferents computadors.


%%% Local Variables:
%%% TeX-master: "main"
%%% End:

  \caption{Esquema de funcionament de MapReduce}
  \label{fig:mapreduce:esquema}
\end{figure}



\begin{enumerate}

\item Hi ha unes dades originals que es poden partir en
  trossos. Hadoop està orientat a fitxers, mitjançant \gls{HDFS}, i
  per tant cada tros de dades és cadascun dels fitxers que es volen
  processar o bé conjunts de línies d'un fitxer.

\item Cada tros de les dades es processa mitjançant una operació
  map. Cada map es pot computar en para\l.lel i distribuït.

\item Cada operació map ha de retornar un nou conjunt de dades
  formats per parelles d'identificador i valor. Aquests conjunts de
  dades s'ordenen per identificador. 

\item Cada conjunt de dades amb el mateix identificador es processa
  mitjançant una operació reduce. Cada reduce es pot computar en
  para\l.lel i distribuït.

\item Cada reduce ha de retornar un tros del resultat final. És a dir,
  que unint les dades que retornen els reduce s'obtenen les dades
  finals. En l'orientació a fitxers de Hadoop, el resultat final és un
  fitxer, o bé un tros d'un fitxer, per a cada reduce.

\end{enumerate}



Per a resoldre un algoritme amb MapReduce, cal definir l'operació de
map i l'operació de reduce. El map ha de calcular un filtre sobre les
dades, sobretot establir grups de dades, i el reduce ha de calcular
agregacions o resums per a cada grup. MapReduce té una gran similitud
amb l'operació \emph{summarize} dels
\gls{SGBDR} \parencite[cap.~7]{date04:introduction8}, però separada
convenientment en les dues etapes.  A més, MapReduce imposa les
següents restriccions: les dades s'han de poder partir i l'algoritme
s'ha de poder expressar separat en les dues operacions de map i de
reduce.  \textcite{deanghemawat04:mapreduce} mostren exemples
d'algoritmes que es poden expressar amb MapReduce.



Un cop s'ha modelat un algoritme amb MapReduce, aleshores Hadoop ja és
capaç d'executar els maps i els reduces en para\l.lel i distribuïts. A
més, Hadoop també gestiona el compromís dels recursos entre el temps
de distribuir les dades, la quantitat de processos en para\l.lel que
s'han de crear i el temps afegit que suposa cada procés nou.








\section{RoundRobindoop}


\todo{}

RoundRobindoop implementa un \gls{SGSTM} específic que resol la funció
de multiresolució amb el model de programació MapReduce. 
\todo{atenció només resol l'operació de Dmap!}

Primer, dissenyem les operacions de MapReduce per a la multiresolució. Segon, 



\subsection{Multiresolució amb MapReduce}

L'algoritme que implementem amb MapReduce és el de la funció de
multiresolució definida a la secció \todo{ref secció
  sec:multiresolucio:funcio}. Aquesta funció principalment té dues
parts, una de mapa i una de plec, que com ja s'ha dit no es poden
correspondre exactament amb les operacions de map i de reduce. 
 Així
doncs, dissenyem les operacions de map i de reduce que tenen el mateix
efecte que calcular la funció de multiresolució, en què el resultat
final no és la sèrie temporal total sinó el resultat de totes les
funcions $\glssymbol{not:sgstm:dmap}$, és a dir el resultat final són
les sèries temporals dels discs del model de \gls{SGSTM} la
concatenació dels quals resulta en la sèrie temporal total.



Sigui $S=\{m_0,m_1,\dotsc,m_k\}$ una sèrie temporal, i $e = \{
(\delta_0,f_0,\tau_0,k_0),\ldots, (\delta_d,f_d,\tau_d,k_d)\}$ els
paràmetres d'un esquema de multiresolució, definim l'algoritme
MapReduce que calcula $\operatorname{mapreduce}(S,e) = \{
(\delta_0,f_0,
\glssymbol{not:sgstm:dmap}(S,\delta_0,f_0,\tau_0,k_0)),\dotsc,
(\delta_c,f_c,\glssymbol{not:sgstm:dmap}(S,\delta_c,f_c,\tau_c,k_c))
\}$. És a dir, calcula tots els $\glssymbol{not:sgstm:dmap}$ possibles
i els identifica amb el pas de consolidació $\delta$ i la funció
d'agregació d'atributs $f$, els quals identifiquen les subsèries
resolució assumint que no n'hi ha de repetits.
%i que per tant tenenassociats el cardinal màxim $k$ i un instant de consolidació $\tau$.





L'esquema de funcionament de RoundRobindoop és el de
la~\autoref{fig:roundrobindoop:esquema}, el qual és la implementació
particular de l'esquema de la~\autoref{fig:mapreduce:esquema} per a la
multiresolució. A continuació expliquem com són les dades originals,
les d'entremig i les finals, i per tant quina operació realitzen els
map i els reduce.





\begin{figure}[tp]
  \centering

  \begin{tikzpicture}

      \tikzset{
        mynode/.style={rectangle,rounded corners,draw=black, 
          very thick, inner sep=1em, minimum size=3em, text centered,
          groc},
        myarrow/.style={->, shorten >=1pt, thick},
        mylabel/.style={text width=7em, text centered},
        groc/.style={top color=white, bottom color=yellow!50},
        verd/.style={top color=white, bottom color=green!50},
        roig/.style={top color=white, bottom color=red!50},
      }  




 \node[mynode,verd] (m1) {map 1};
 \node (ml1) [below=of m1] {};
 \node[mynode,verd] (m2) [below=of ml1] {map 2};
 \node (ml2) [below=of m2] {};
 \node[mynode,verd,dotted] (mn) [below=of ml2] {map M};


 \node[mynode] (d1) [right=of m1] {
   \begin{tabular}{|c|cc|}\hline
     \multirow{2}{*}{$B$} & \multicolumn{2}{|c|}{valor} \\\cline{2-3}
       & \multicolumn{1}{|c|}{t} & v \\\hline
      & & \\
      & &\\\hdashline
      & &\\
      & &\\\hline
   \end{tabular}
 };

 \node[mynode] (d2) [right=of m2] {
   \begin{tabular}{|c|cc|}\hline
     \multirow{2}{*}{$B$} & \multicolumn{2}{|c|}{valor} \\\cline{2-3}
       & \multicolumn{1}{|c|}{t} & v \\\hline
      & &\\\hdashline
      & &\\\hline
   \end{tabular}
 };

 \node[mynode,dotted] (dn) [right=2cm of mn] {};



 \node[mynode,verd] (r1) [right=of d1] {reduce 1};
 \node[mynode,verd] (r2) [below=2cm of r1] {reduce 2};
 \node[mynode,verd,dotted] (rn) [right=2cm of dn] {};




 \node[mynode] (o) [above left=of m1, anchor=north east] {
   \begin{tabular}{|cc|}\hline
     t & v \\\hline
      & \\
      & \\
     \parbox[c][7cm][s]{0cm}{\vfill}& \\
      & \\
      & \\\hline
   \end{tabular}
 };
 \node [above=0cm of o] {$S$ original};
 \node[mynode,minimum height=10cm] (f) [above right=of r1, anchor=north west] {
   \begin{tabular}{|cc|}\hline
    $\delta\, f$ & t v \\\hline
      & \\
       & \\
     \parbox[c][7cm][s]{0cm}{\vfill}&  \\
       & \\
       & \\\hline
   \end{tabular}
};
 \node [above=0cm of f] {final};




 \draw[dashed] (ml1) -- (ml1-|o.west);
 \draw[dashed] (ml2) -- (ml2-|o.west);

 \draw[myarrow] (o.center|-m1) -- (m1)  node[sloped,above,near end] {$S'_1$};
 \draw[myarrow] (o.center|-m2) -- (m2) node[sloped,above,near end] {$S'_2$};
 \draw[myarrow,dotted] (o.center|-mn) -- (mn) node[sloped,above,near end] {$S'_M$};


 \draw[myarrow] (m1) -- (d1);
 \draw[myarrow] (m2) -- (d2);
 \draw[myarrow,dotted] (mn) -- (dn);

 \draw[myarrow] (d1.center) -- (r1);
 \node (d1b) [above=3ex of d1.280] {}; 
 \draw[myarrow] (d1b.center) -- (r2);

 \draw[myarrow] (d2.center) -- (r1);
 \node (d2b) [above=3ex of d2.280] {}; 
 \draw[myarrow] (d2b.center) -- (r2);

 \draw[myarrow,dotted] (dn) -- (r1) ;
 \draw[myarrow,dotted] (dn) -- (r2) ;
 \draw[myarrow,dotted] (dn) -- (rn) ;


 \node (r1l) [below=of r1] {};
 \draw[dashed] (r1l) -- (r1l-|f.east);
 \node (r2l) [below=of r2] {};
 \draw[dashed] (r2l) -- (r2l-|f.east);


 \draw[myarrow] (r1) -- (r1-|f.center) ;
 \draw[myarrow] (r2) -- (r2-|f.center) ;
 \draw[myarrow,dotted] (rn) -- (rn-|f.center) ;


  \end{tikzpicture}
  
  \caption{Esquema de funcionament de RoundRobindoop}
  \label{fig:roundrobindoop:esquema}
\end{figure}






Les dades originals són la sèrie temporal $S$ és a dir un conjunt de
parelles de temps i valor, a les quals anomenem mesures. Un sèrie
temporal es pot partir en trossos on cada tros és un subconjunt de
mesures, és a dir una subsèrie temporal. Per tant, cada operació map
rep una subsèrie temporal de l'original: $S'_1 =
\{m_0,\dotsc,m_{o1}\}, S'_2 = \{m_{o1+1},\dotsc,m_{o2}\}, \dotsc, S'_M
= \{\dotsc,m_{k}\}$ on $M$ són el nombre de maps i $o_1 < o_2 < k$.



Les dades finals són les sèries temporals dels discs, és a dir les
mesures consolidades per a cada subsèrie resolució. Així doncs, les
dades finals vistes com a conjunt són un conjunt de dades
consolidades $R=\{ D'_{0}, \dotsc, D'_r\}$ on cada dada consolidada
és un tuple $D'=(\delta,f,m')$ que indica quina mesura $m'=(t,v)$ és i a
quin disc pertany identificat per $\delta$ i $f$.  Per tant, cada
operació reduce, assumint que hi ha un
reduce per cada resolució, calcula un subconjunt de $R$: $R'_1 = \{(\delta_0,f_0,m'_0),\dotsc,(\delta_0,f_0,m'_{k1})\}$

$\{m'_0,\dotsc,m'_{r1}\}, R'_2 = \{m'_{r1+1},\dotsc,m'_{r2}\}, \dotsc,
R'_d = \{\dotsc,m'_{r}\}$ on $d$ són el nombre de reduces
i $(\delta_0,f_0,m'_0)$   $(\delta_0,f_0,m'_{k1})$  






, per tant  $d=|e|$ 


 i $r_1 < r_2
< r$.  Per a l'esquema de multiresolució $e$, i assumint que hi ha un
reduce per cada resolució, $d=|e|$ 

El nombre total de mesures consolidades $r$ és afitat ja
que per a l'esquema de multiresolució $e$ és $r+1 \leq k_0+\dotsb+k_d$.










\subsection{Execució de l'algoritme}

Proposem dues maneres per a executar l'algoritme implementat amb
MapReduce: Hadoop i shell.  \todo{també es pot executar fora de Hadoop
  via pipeline del shell}


A Hadoop:\todo{}
Només farem una configuració de Single Node Setup. 
Després es podria estendre de forma senzilla a una Cluster Setup, on
només caldria decidir com distribuir les dades i els processos de map
i reduce als diferents computadors.


%%% Local Variables:
%%% TeX-master: "main"
%%% End:

  \caption{Esquema de funcionament de MapReduce}
  \label{fig:mapreduce:esquema}
\end{figure}



\begin{enumerate}

\item Hi ha unes dades originals que es poden partir en
  trossos. Hadoop està orientat a fitxers, mitjançant \gls{HDFS}, i
  per tant cada tros de dades és cadascun dels fitxers que es volen
  processar o bé conjunts de línies d'un fitxer.

\item Cada tros de les dades es processa mitjançant una operació
  map. Cada map es pot computar en para\l.lel i distribuït.

\item Cada operació map ha de retornar un nou conjunt de dades
  formats per parelles d'identificador i valor. Aquests conjunts de
  dades s'ordenen per identificador. 

\item Cada conjunt de dades amb el mateix identificador es processa
  mitjançant una operació reduce. Cada reduce es pot computar en
  para\l.lel i distribuït.

\item Cada reduce ha de retornar un tros del resultat final. És a dir,
  que unint les dades que retornen els reduce s'obtenen les dades
  finals. En l'orientació a fitxers de Hadoop, el resultat final és un
  fitxer, o bé un tros d'un fitxer, per a cada reduce.

\end{enumerate}



Per a resoldre un algoritme amb MapReduce, cal definir l'operació de
map i l'operació de reduce. El map ha de calcular un filtre sobre les
dades, sobretot establir grups de dades, i el reduce ha de calcular
agregacions o resums per a cada grup. MapReduce té una gran similitud
amb l'operació \emph{summarize} dels
\gls{SGBDR} \parencite[cap.~7]{date04:introduction8}, però separada
convenientment en les dues etapes.  A més, MapReduce imposa les
següents restriccions: les dades s'han de poder partir i l'algoritme
s'ha de poder expressar separat en les dues operacions de map i de
reduce.  \textcite{deanghemawat04:mapreduce} mostren exemples
d'algoritmes que es poden expressar amb MapReduce.



Un cop s'ha modelat un algoritme amb MapReduce, aleshores Hadoop ja és
capaç d'executar els maps i els reduces en para\l.lel i distribuïts. A
més, Hadoop també gestiona el compromís dels recursos entre el temps
de distribuir les dades, la quantitat de processos en para\l.lel que
s'han de crear i el temps afegit que suposa cada procés nou.








\section{RoundRobindoop}


\todo{}

RoundRobindoop implementa un \gls{SGSTM} específic que resol la funció
de multiresolució amb el model de programació MapReduce. 
\todo{atenció només resol l'operació de Dmap!}

Primer, dissenyem les operacions de MapReduce per a la multiresolució. Segon, 



\subsection{Multiresolució amb MapReduce}

L'algoritme que implementem amb MapReduce és el de la funció de
multiresolució definida a la secció \todo{ref secció
  sec:multiresolucio:funcio}. Aquesta funció principalment té dues
parts, una de mapa i una de plec, que com ja s'ha dit no es poden
correspondre exactament amb les operacions de map i de reduce. 
 Així
doncs, dissenyem les operacions de map i de reduce que tenen el mateix
efecte que calcular la funció de multiresolució, en què el resultat
final no és la sèrie temporal total sinó el resultat de totes les
funcions $\glssymbol{not:sgstm:dmap}$, és a dir el resultat final són
les sèries temporals dels discs del model de \gls{SGSTM} la
concatenació dels quals resulta en la sèrie temporal total.



Sigui $S=\{m_0,m_1,\dotsc,m_k\}$ una sèrie temporal, i $e = \{
(\delta_0,f_0,\tau_0,k_0),\ldots, (\delta_d,f_d,\tau_d,k_d)\}$ els
paràmetres d'un esquema de multiresolució, definim l'algoritme
MapReduce que calcula $\operatorname{mapreduce}(S,e) = \{
(\delta_0,f_0,
\glssymbol{not:sgstm:dmap}(S,\delta_0,f_0,\tau_0,k_0)),\dotsc,
(\delta_c,f_c,\glssymbol{not:sgstm:dmap}(S,\delta_c,f_c,\tau_c,k_c))
\}$. És a dir, calcula tots els $\glssymbol{not:sgstm:dmap}$ possibles
i els identifica amb el pas de consolidació $\delta$ i la funció
d'agregació d'atributs $f$, els quals identifiquen les subsèries
resolució assumint que no n'hi ha de repetits.
%i que per tant tenenassociats el cardinal màxim $k$ i un instant de consolidació $\tau$.





L'esquema de funcionament de RoundRobindoop és el de
la~\autoref{fig:roundrobindoop:esquema}, el qual és la implementació
particular de l'esquema de la~\autoref{fig:mapreduce:esquema} per a la
multiresolució. A continuació expliquem com són les dades originals,
les d'entremig i les finals, i per tant quina operació realitzen els
map i els reduce.





\begin{figure}[tp]
  \centering

  \begin{tikzpicture}

      \tikzset{
        mynode/.style={rectangle,rounded corners,draw=black, 
          very thick, inner sep=1em, minimum size=3em, text centered,
          groc},
        myarrow/.style={->, shorten >=1pt, thick},
        mylabel/.style={text width=7em, text centered},
        groc/.style={top color=white, bottom color=yellow!50},
        verd/.style={top color=white, bottom color=green!50},
        roig/.style={top color=white, bottom color=red!50},
      }  




 \node[mynode,verd] (m1) {map 1};
 \node (ml1) [below=of m1] {};
 \node[mynode,verd] (m2) [below=of ml1] {map 2};
 \node (ml2) [below=of m2] {};
 \node[mynode,verd,dotted] (mn) [below=of ml2] {map M};


 \node[mynode] (d1) [right=of m1] {
   \begin{tabular}{|c|cc|}\hline
     \multirow{2}{*}{$B$} & \multicolumn{2}{|c|}{valor} \\\cline{2-3}
       & \multicolumn{1}{|c|}{t} & v \\\hline
      & & \\
      & &\\\hdashline
      & &\\
      & &\\\hline
   \end{tabular}
 };

 \node[mynode] (d2) [right=of m2] {
   \begin{tabular}{|c|cc|}\hline
     \multirow{2}{*}{$B$} & \multicolumn{2}{|c|}{valor} \\\cline{2-3}
       & \multicolumn{1}{|c|}{t} & v \\\hline
      & &\\\hdashline
      & &\\\hline
   \end{tabular}
 };

 \node[mynode,dotted] (dn) [right=2cm of mn] {};



 \node[mynode,verd] (r1) [right=of d1] {reduce 1};
 \node[mynode,verd] (r2) [below=2cm of r1] {reduce 2};
 \node[mynode,verd,dotted] (rn) [right=2cm of dn] {};




 \node[mynode] (o) [above left=of m1, anchor=north east] {
   \begin{tabular}{|cc|}\hline
     t & v \\\hline
      & \\
      & \\
     \parbox[c][7cm][s]{0cm}{\vfill}& \\
      & \\
      & \\\hline
   \end{tabular}
 };
 \node [above=0cm of o] {$S$ original};
 \node[mynode,minimum height=10cm] (f) [above right=of r1, anchor=north west] {
   \begin{tabular}{|cc|}\hline
    $\delta\, f$ & t v \\\hline
      & \\
       & \\
     \parbox[c][7cm][s]{0cm}{\vfill}&  \\
       & \\
       & \\\hline
   \end{tabular}
};
 \node [above=0cm of f] {final};




 \draw[dashed] (ml1) -- (ml1-|o.west);
 \draw[dashed] (ml2) -- (ml2-|o.west);

 \draw[myarrow] (o.center|-m1) -- (m1)  node[sloped,above,near end] {$S'_1$};
 \draw[myarrow] (o.center|-m2) -- (m2) node[sloped,above,near end] {$S'_2$};
 \draw[myarrow,dotted] (o.center|-mn) -- (mn) node[sloped,above,near end] {$S'_M$};


 \draw[myarrow] (m1) -- (d1);
 \draw[myarrow] (m2) -- (d2);
 \draw[myarrow,dotted] (mn) -- (dn);

 \draw[myarrow] (d1.center) -- (r1);
 \node (d1b) [above=3ex of d1.280] {}; 
 \draw[myarrow] (d1b.center) -- (r2);

 \draw[myarrow] (d2.center) -- (r1);
 \node (d2b) [above=3ex of d2.280] {}; 
 \draw[myarrow] (d2b.center) -- (r2);

 \draw[myarrow,dotted] (dn) -- (r1) ;
 \draw[myarrow,dotted] (dn) -- (r2) ;
 \draw[myarrow,dotted] (dn) -- (rn) ;


 \node (r1l) [below=of r1] {};
 \draw[dashed] (r1l) -- (r1l-|f.east);
 \node (r2l) [below=of r2] {};
 \draw[dashed] (r2l) -- (r2l-|f.east);


 \draw[myarrow] (r1) -- (r1-|f.center) ;
 \draw[myarrow] (r2) -- (r2-|f.center) ;
 \draw[myarrow,dotted] (rn) -- (rn-|f.center) ;


  \end{tikzpicture}
  
  \caption{Esquema de funcionament de RoundRobindoop}
  \label{fig:roundrobindoop:esquema}
\end{figure}






Les dades originals són la sèrie temporal $S$ és a dir un conjunt de
parelles de temps i valor, a les quals anomenem mesures. Un sèrie
temporal es pot partir en trossos on cada tros és un subconjunt de
mesures, és a dir una subsèrie temporal. Per tant, cada operació map
rep una subsèrie temporal de l'original: $S'_1 =
\{m_0,\dotsc,m_{o1}\}, S'_2 = \{m_{o1+1},\dotsc,m_{o2}\}, \dotsc, S'_M
= \{\dotsc,m_{k}\}$ on $M$ són el nombre de maps i $o_1 < o_2 < k$.



Les dades finals són les sèries temporals dels discs, és a dir les
mesures consolidades per a cada subsèrie resolució. Així doncs, les
dades finals vistes com a conjunt són un conjunt de dades
consolidades $R=\{ D'_{0}, \dotsc, D'_r\}$ on cada dada consolidada
és un tuple $D'=(\delta,f,m')$ que indica quina mesura $m'=(t,v)$ és i a
quin disc pertany identificat per $\delta$ i $f$.  Per tant, cada
operació reduce, assumint que hi ha un
reduce per cada resolució, calcula un subconjunt de $R$: $R'_1 = \{(\delta_0,f_0,m'_0),\dotsc,(\delta_0,f_0,m'_{k1})\}$

$\{m'_0,\dotsc,m'_{r1}\}, R'_2 = \{m'_{r1+1},\dotsc,m'_{r2}\}, \dotsc,
R'_d = \{\dotsc,m'_{r}\}$ on $d$ són el nombre de reduces
i $(\delta_0,f_0,m'_0)$   $(\delta_0,f_0,m'_{k1})$  






, per tant  $d=|e|$ 


 i $r_1 < r_2
< r$.  Per a l'esquema de multiresolució $e$, i assumint que hi ha un
reduce per cada resolució, $d=|e|$ 

El nombre total de mesures consolidades $r$ és afitat ja
que per a l'esquema de multiresolució $e$ és $r+1 \leq k_0+\dotsb+k_d$.










\subsection{Execució de l'algoritme}

Proposem dues maneres per a executar l'algoritme implementat amb
MapReduce: Hadoop i shell.  \todo{també es pot executar fora de Hadoop
  via pipeline del shell}


A Hadoop:\todo{}
Només farem una configuració de Single Node Setup. 
Després es podria estendre de forma senzilla a una Cluster Setup, on
només caldria decidir com distribuir les dades i els processos de map
i reduce als diferents computadors.


%%% Local Variables:
%%% TeX-master: "main"
%%% End:

\chapter{Implementació relacional amb Tutorial~D}

Els models \gls{SGST} i \gls{SGSTM} es basen fortament en l'àlgebra
relacional.  Així doncs, n'explorem una implementació en un
\gls{SGBDR}. Per tal de mantenir la màxima fidelitat amb els models i
per les consideracions de \textcite[cap.~1--4]{date04:introduction8}
sobre la idoneïtat del llenguatge \gls{SQL} en el model relacional,
implementem els models amb el llenguatge acadèmic dels \gls{SGBDR}:
\emph{Tutorial~D} \parencite{date04:introduction8,date:thethirdmanifesto,date:tutoriald}. Com
a intèrpret per a aquest llenguatge utilitzem Rel \parencite{rel}, el
qual és un \gls{SGBDR} de propòsit acadèmic per a experimentar amb
Tutorial~D.


% Tutorial D is a specific D which is defined and used for illustration in The Third Manifesto. Implementations of D need not have the same syntax as Tutorial D. The purpose of Tutorial D is both educational and to show what a D might be like. Rel is an implementation of Tutorial D.
% Rel is an open source true relational database management system that implements a significant portion of Chris Date and Hugh Darwen's Tutorial D query language.
% Primarily intended for teaching purposes, Rel is written in the Java programming language.


\todo{}

Implementem la part essencial, és a dir l'àlgebra definida en els
models lògics. Com que són de nivell acadèmic, ni Tutorial D ni Rel
implementen els complements que tenen els SGBDR en entorns
d'explotació. Com a conseqüència, només es podran utilitzar per a
comprovar l'exactitud en l'àlgebra, però no amb comoditat per a
treballar amb dades reals en les quals es necessita sovint operacions per canviar-ne el format, per convertir-les, etc.\todo{}




\section{SGST relacional}


Tal com s'ha definit el model estructural de SGST, en el model
relacional les sèries temporals són relacions amb dos atributs:
\emph{t} pel temps i \emph{v} pels valors, on l'atribut \emph{t} fa de
clau primària en les variables relació.  Les mesures són els tuples
d'aquestes relacions. 

Principalment la implementació relacional consisteix en definir el
tipus sèrie temporal i totes les operacions de sèries
temporals. Aquestes definicions es basaran en Tutorial D, és a dir que
no cal cap llenguatge no relacional extern.

Tutorial D defineix estàticament els tipus a les operacions i això
dificulta generalitzar-ho per a qualsevol tipus de temps i valor com
s'ha fet a la implementació amb Python. Com a conseqüència, prefixarem
els tipus dels atributs temps i dels valors a reals, \emph{Rational} a
Tutorial D.

Així doncs, el valor de sèrie temporal es defineix directament
mitjançant una relació amb els atributs \emph{t} i \emph{v}. Per
exemple:
\begin{verbatim}
RELATION {
   TUPLE { t 2.0, v 3.0 },
   TUPLE { t 3.0, v 4.0 }
 }
\end{verbatim}


A partir d'aquest valors relació s'hauria de definir el tipus sèrie
temporal. No obstant això, la definició de tipus encara té un estat
massa experimental a Tutorial D, com es detalla a
l'apartat~\ref{sec:implementacio:tipus-relacional}. Així doncs,
definim la variable relació \emph{timeseries} per a utilitzar-la com a
equivalent del tipus en les definicions dels operadors:
\begin{verbatim}
VAR timeseries BASE RELATION
    { t RATIONAL, v RATIONAL }  KEY { t } ;
\end{verbatim}


En les sèries temporals com a relació, els tuples són les mesures del
model de SGST. Hi ha alguns operadors dels SGST que tenen mesures com
a paràmetres, però els tuples sense formar part d'una relació no són
un tipus vàlid a Tutorial D. Així caldria definir també un valor
relació per a les mesures. Per a simplificar, definim aquestes mesures
com a sèries temporals d'un sol element, per exemple:
\begin{verbatim}
RELATION {
   TUPLE { t 2.0, v 3.0 }
 }
\end{verbatim}


A partir d'aquestes mesures com a valors relació definim els operadors
que tenen mesures com a paràmetres. Així, els dos operadors bàsics
dels SGST per a obtenir els atributs de temps i de valor d'una mesura
són:
\begin{verbatim}
OPERATOR ts.t(m SAME_TYPE_AS  (timeseries)) RETURNS RATIONAL;
  return t FROM TUPLE FROM m;
END OPERATOR;

OPERATOR ts.v(m SAME_TYPE_AS  (timeseries)) RETURNS RATIONAL;
  return v FROM TUPLE FROM m;
END OPERATOR;
\end{verbatim}


Per exemple, per a obtenir l'atribut temps d'una mesura
\begin{verbatim}
WITH RELATION {
   TUPLE { t 2.0, v 3.0 }
  } AS m1: 
ts.t(m1)
\end{verbatim}
i per a obtenir el valor
\begin{verbatim}
WITH RELATION {
   TUPLE { t 2.0, v 3.0 }
  } AS m1: 
ts.v(m1)
\end{verbatim}




\subsection{Operadors}


Algunes operacions dels SGST es poden aplicar directament a les
relacions \emph{timeseries} amb les operacions de Tutorial D. 
És el cas de
\begin{itemize}
\item La projecció
\begin{verbatim}
WITH
 RELATION {
   TUPLE { t 2.0, v 3.0 },
   TUPLE { t 3.0, v 4.0 }
 } AS ts1:
ts1 {t}
\end{verbatim}

\item La selecció
\begin{verbatim}
WITH
 RELATION {
   TUPLE { t 2.0, v 3.0 },
   TUPLE { t 3.0, v 4.0 }
 } AS ts1:
ts1 where v>3.0
\end{verbatim}

\item El reanomena
\begin{verbatim}
WITH
 RELATION {
   TUPLE { t 2.0, v 3.0 },
   TUPLE { t 3.0, v 4.0 }
 } AS ts1:
ts1 rename ( v as temperatura )
\end{verbatim}
\end{itemize}





Les altres operacions s'han de definir. A continuació en definim
algunes amb Tutorial D.  Tots els operadors que definim els anomenem
prefixats amb \emph{ts.}.



L'operació d'unió és equivalent a unir la primera sèrie temporal amb
les mesures de la segona que no tenen un atribut de temps igual que
algun temps de la primera. Així, l'operador d'unió \emph{ts.union} és
\begin{verbatim}
OPERATOR ts.union(s1 SAME_TYPE_AS  (timeseries), s2 SAME_TYPE_AS  (timeseries)) RETURNS RELATION SAME_HEADING_AS  (timeseries);
  return s1 UNION (s2 JOIN (s2 {t} MINUS s1 {t}));
END OPERATOR;
\end{verbatim}

Per exemple:
\begin{verbatim}
WITH 
 RELATION {
   TUPLE { t 2.0, v 3.0 },
   TUPLE { t 4.0, v 2.0 },
   TUPLE { t 6.0, v 4.0 }
  } AS ts1,
 RELATION {
   TUPLE { t 1.0, v 2.0 },
   TUPLE { t 5.0, v 3.0 },
   TUPLE { t 6.0, v 5.0 },
   TUPLE { t 10.0, v 1.0 }
  } AS ts2: 
ts.union(ts1,ts2)
\end{verbatim}






\subsection{Sèries temporals multivaluades i dobles}


\todo{}

No funcionaran a les operacions perquè són operadors de relation {t,v} i les multivaluades són relation{t,relation{v1,v2}} -> potser es podria solucionar amb warps?


\begin{verbatim}
VAR timeseriesdouble BASE RELATION
    { t1 RATIONAL, v1 RATIONAL, t2 RATIONAL, v2 RATIONAL }  KEY { t1, t2 } ;
\end{verbatim}


% // multivalued2canonical -> potser es pot fer amb grup/ungroup o wrap/unwrap ?
% //extend r add ( (r rename (t as tr) where t=tr) {ALL BUT tr} as vr ) {t,vr} rename (vr as v)
% //multivalued2canonical
% //r group ({all but t} as v)
% //canonical2multivalued
% // ungroup (v)






\subsection{Quant a definir el tipus sèrie temporal}
\label{sec:implementacio:tipus-relacional}

\todo{}

* Les relacions no són exactament un tipus a Tutorial D però haurien de ser-ho

* S'hauria de poder heretar de les relacions




TYPE ts POSSREP {ts relation {t rational, v rational}};

però després no podríem aplicar-hi operacions relacionals com per exemple el projection perquè és projection(relation)

tampoc podríem definir multivaluades en la forma canònica \verb+ts(relation {tuple {t 1.0, v relation { tuple { v1 1.0, v2 2.0}}}})+ perquè el v no és un rational






La definició estàtica de tipus també dificulta definir tipus i operadors que admetin sèries temporals definides com a relacions on un dels atributs sigui t el temps. Hugh Darwen proposa millores a Tutorial D definit una capçalera com per exemple Relation \{ t Rational, * \} on especificaria un tipus de relació la capçalera de la qual contingui l'atribut temps.
\todo{veure}
Extending Tutorial D to Support User-Defined
Generic Relation and Tuple Operators
Hugh Darwen
File: User-defined relational operators in TD.doc
Printed at: 17:03 on Monday, 18 November, 2013
\url{www.dcs.warwick.ac.uk/~hugh/CHAP05.pdf}
\url{http://www.dcs.warwick.ac.uk/~hugh/TTM/User-defined-relational-operators-in-TD.pdf}


Aleshores podríem definir correctament el tipus sèrie temporal, tot i així encara faria falta que fos un subtipus de relation.

\begin{verbatim}
type timeseries
	POSSREP canonical {ts relation {t *, v * }}}
	POSSREP multivalued {multivalued relation {t * ,*}} 
	INIT  canonical (multivalued:= ts ungroup (v))
	      multivalued (ts:= multivalued group ({all but t} as v));
\end{verbatim}

% //Ex
% //with multivalued2(relation{ tuple{ t 1.0, v1 2.0, v2 3.0} }) as r:
% //THE_multivalued2(r)

% //with multivalued2(relation{ tuple{ t 1.0, v1 2.0, v2 3.0} }) as r:
% //THE_ts(r)

% //with ts(relation{ tuple{ t 1.0, v relation { tuple {v1 2.0, v2 3.0}} }}) as r:
% //THE_multivalued2(r)




% \subsection{Pertinença i inclusió}


% WITH RELATION {
% TUPLE { t 2.0, v 3.0 }
%  } AS m,
% RELATION {
% TUPLE { t 2.0, v 3.0 },
% TUPLE { t 5.0, v 3.0 },
% TUPLE { t 6.0, v 5.0 },
% TUPLE { t 10.0, v 1.0 }
%  } AS ts: 
% ts.in(m,ts)


% \subsection{Màxim i suprem}

% Tutorial D:
% \begin{verbatim}
% OPERATOR ts.max(s1 SAME_TYPE_AS  (timeseries)) RETURNS RELATION SAME_HEADING_AS  (timeseries);
% return s1 JOIN ( SUMMARIZE s1 {t} PER (s1 {}) ADD (MAX (t) AS t));
% END OPERATOR;

% OPERATOR ts.sup(s1 SAME_TYPE_AS  (timeseries)) RETURNS RELATION SAME_HEADING_AS  (timeseries);
% return ts.max(ts.union(s1,(RELATION { TUPLE {t -1.0/0.0, v 1.0/0.0} })));
% END OPERATOR;
% \end{verbatim}



% Exemple:
% \begin{verbatim}
% WITH RELATION {
% TUPLE { t 2.0, v 3.0 },
% TUPLE { t 4.0, v 2.0 },
% TUPLE { t 6.0, v 4.0 }
%  } AS ts1: 
% ts.max(ts1)
% \end{verbatim}
% \begin{verbatim}
% RELATION {
% TUPLE { t 1.0, v 2.0 },
% TUPLE { t 5.0, v 3.0 },
% TUPLE { t 6.0, v 5.0 },
% TUPLE { t 1.0/0.0, v 1.0 }  //1.0/0.0 infinit
%  } AS ts2: 
% ts.max(ts2)
% \end{verbatim}
% \begin{verbatim}
% WITH RELATION {
% TUPLE { t 2.0, v 3.0 },
% TUPLE { t 4.0, v 2.0 },
% TUPLE { t 6.0, v 4.0 }
%  } AS ts1: 
% ts.sup(ts1)
% \end{verbatim}
% \begin{verbatim}
% WITH RELATION {
% TUPLE { t 2.0, v 3.0 },
% TUPLE { t 4.0, v 2.0 },
% TUPLE { t 6.0, v 4.0 }
%  } AS ts1: 
% ts.sup(timeseries)
% \end{verbatim}




% \subsection{Producte}

% WITH RELATION {
% TUPLE { t 1.0, v 4.0 }
%  } AS s1,
% RELATION {
% TUPLE { t 2.0, v 3.0 },
% TUPLE { t 5.0, v 3.0 },
% TUPLE { t 6.0, v 5.0 },
% TUPLE { t 10.0, v 1.0 }
%  } AS s2: 
% ts.product(s1,s2)


% \subsection{Unió exclusiva}


% TutorialD:
% \begin{verbatim}
% OPERATOR ts.xunion(s1 SAME_TYPE_AS  (timeseries), s2 SAME_TYPE_AS  (timeseries)) RETURNS RELATION SAME_HEADING_AS  (timeseries);
% return ts.union(s1,s2) MINUS ts.intersect(s1,s2) ;
% END OPERATOR;
% \end{verbatim}


% Exemple:
% \begin{verbatim}
% WITH RELATION {
% TUPLE { t 2.0, v 3.0 },
% TUPLE { t 4.0, v 2.0 },
% TUPLE { t 6.0, v 4.0 }
%  } AS ts1,
% RELATION {
% TUPLE { t 1.0, v 2.0 },
% TUPLE { t 5.0, v 3.0 },
% TUPLE { t 6.0, v 5.0 },
% TUPLE { t 10.0, v 1.0 }
%  } AS ts2: 
% ts.xunion(ts1,ts2)
% \end{verbatim}


% \subsection{Selecció temporal}



% TutorialD:
% \begin{verbatim}
% OPERATOR ts.temporal.select.zohe(s SAME_TYPE_AS  (timeseries), l RATIONAL, h RATIONAL ) RETURNS RELATION SAME_HEADING_AS  (timeseries);
% BEGIN;
% VAR x RATIONAL init(0.0);
% VAR sp PRIVATE SAME_TYPE_AS ( timeseries) KEY { t };
% x := ts.v(ts.inf(s MINUS ts.interval.ni(s,h)));
% sp := RELATION {
% TUPLE {t h, v x}
% };
% return ts.union(ts.interval(s,l,h),sp);
% END;
% END OPERATOR;
% \end{verbatim}

% Exemple:
% \begin{verbatim}
% WITH RELATION {
% TUPLE { t 2.0, v 3.0 },
% TUPLE { t 4.0, v 2.0 },
% TUPLE { t 6.0, v 4.0 }
%  } AS ts1,
% RELATION {
% TUPLE { t 1.0, v 2.0 },
% TUPLE { t 5.0, v 3.0 },
% TUPLE { t 6.0, v 5.0 },
% TUPLE { t 10.0, v 1.0 }
%  } AS ts2:
% ts.temporal.select.zohe(ts1,1.0,5.0)
% \end{verbatim}
% \begin{verbatim}
% WITH RELATION {
% TUPLE { t 2.0, v 3.0 },
% TUPLE { t 4.0, v 2.0 },
% TUPLE { t 6.0, v 4.0 }
%  } AS ts1,
% RELATION {
% TUPLE { t 1.0, v 2.0 },
% TUPLE { t 5.0, v 3.0 },
% TUPLE { t 6.0, v 5.0 },
% TUPLE { t 10.0, v 1.0 }
%  } AS ts2:
% ts.temporal.select.zohe(ts1,-1.0/0.0,-1.0/0.0)
% \end{verbatim}



% \subsection{Map i fold}



% TutorialD:
% \begin{verbatim}
% WITH RELATION {
% TUPLE { t 2.0, v 3.0 },
% TUPLE { t 4.0, v 2.0 },
% TUPLE { t 6.0, v 4.0 }
% } AS ts1: 
% ts.map(ts1,'t','t*v/2.0')
% \end{verbatim}

% TutorialD:
% \begin{verbatim}
% WITH RELATION {
% TUPLE { t 2.0, v 3.0 },
% TUPLE { t 4.0, v 2.0 },
% TUPLE { t 6.0, v 4.0 }
%  } AS ts1,
% RELATION {
% TUPLE { t 0.0, v 0.0}
% } AS mi: 
% ts.fold(ts1,mi,'t','v+vi')
% \end{verbatim}

% antM TutorialD:
% \begin{verbatim}
% WITH RELATION {
% TUPLE { t 2.0, v 3.0 },
% TUPLE { t 4.0, v 2.0 },
% TUPLE { t 6.0, v 4.0 }
%  } AS ts1,
% RELATION {
% TUPLE { t 5.0, v 0.0}
% } AS m: 
% ts.sup(ts1 WHERE t < ts.t(m))
% \end{verbatim}

% sup fold TutorialD:
% \begin{verbatim}
% WITH RELATION {
% TUPLE { t 2.0, v 3.0 },
% TUPLE { t 4.0, v 2.0 },
% TUPLE { t 6.0, v 4.0 }
%  } AS ts1,
% RELATION {
% TUPLE { t -1.0/0.0, v 1.0/0.0}
% } AS mi: 
% ts.fold(ts1,mi,'max {t,ti}','v FROM TUPLE FROM (RELATION { TUPLE {t t, v v, e True}, TUPLE {t ti, v vi,  e False} } JOIN RELATION { TUPLE {e t > ti}})')
% \end{verbatim}

% predecessors mapfold TutorialD:
% \begin{verbatim}
% WITH RELATION {
% TUPLE { t 2.0, v 3.0 },
% TUPLE { t 4.0, v 2.0 },
% TUPLE { t 6.0, v 4.0 }
%  } AS ts1,
% EXTEND ts1 {t} ADD ( -1.0/0.0 as v)
% AS si: 
% ts.fold(ts1,si,'t','v FROM TUPLE FROM (RELATION { TUPLE {v ti, e True}, TUPLE {v v,  e False} } JOIN RELATION { TUPLE {e v < ti and ti<t}})')
% \end{verbatim}


% \section{SGSTM relacional}














%%% Local Variables:
%%% TeX-master: "main"
%%% End:

%\chapter{Implementació com a circuit digital o en VHDL}




Sistema integrat d'emmagatzematge multiresolució de sèries temporals

  Proposta d'emmagatzematge multiresolució de sèries temporals com a
  part d'un projecte que integri xarxes de sensors.



Les xarxes de sensors capturen dades de l'entorn, les quals s'han
d'emmagatzemar en bases de dades per a poder-les tractar
posteriorment. Hi ha models que descriuen com han de ser aquestes
bases de dades per a sèries temporals i esquemes que solucionen alguns
dels seus problemes. En aquest cas, es tracta d'integrar en una xarxa
de sensors un sistema d'emmagatzematge multiresolució per a sèries
temporals.




\section{Solució d'emmagatzematge multiresolució}

Una sèrie temporal és un conjunt de parelles de temps i valor que
provenen de l'evolució d'una variable al llarg del temps. 

A causa d'aquesta naturalesa de variable capturada al llarg del temps,
en l'adquisició i tractament de les sèries temporals apareixen
propietats problemàtiques que anomenem patologies.
Algunes d'aquestes patologies són:
\begin{itemize}
\item La sincronització dels rellotges en els diferents sistemes
  d'adquisició.
\item L'aparició de dades desconegudes perquè no s'han pogut adquirir
  o perquè són errònies.
\item La gestió d'una quantitat enorme de dades i que a més segueix
  creixent al llarg del temps.
\item L'operació amb dades que no s'han recollit de manera uniforme en
  el temps.
\end{itemize}


Els sistemes informàtics que saben emmagatzemar i tractar les sèries
temporals s'anomenen sistemes de gestió de bases de dades per a sèries
temporals (SGST). Els SGST han de saber gestionar les patologies de
les sèries temporals. 

Una solució per a aquestes patologies es pot aconseguir afegint
esquemes de multiresolució per a les sèries temporals. Aleshores
s'obtenen SGST específics anomenats SGST multiresolució (SGSTM).  La
multiresolució és un sistema d'emmagatzematge que selecciona la
informació prèviament a ser guardada i en descarta la que no es
considera important.


\begin{figure}[tp]
\centering
\input{imatges/model/mtsms-arquitectura_interna.tex}
\caption{Arquitectura dels SGSTM}
\label{fig:model:bdstm}
\end{figure}


Un SGSTM és una solució d'emmagatzematge per a sèries temporals a on,
resumint, la informació es distribueix mitjançant diferents
resolucions temporals.  Una sèrie temporal amb multiresolució és una
co\l.lecció de subsèries resolució, les quals acumulen temporalment
mesures en un buffer on són processades i finalment emmagatzemades
en un disc. El processament de les dades té per objectiu canviar els
intervals de temps entre les mesures per tal de compactar la
informació de les sèries temporals. D'aquesta manera, les sèries
temporals queden emmagatzemades en diferents resolucions temporals
distribuïdes en els discs.  L'arquitectura d'aquests sistemes es pot
veure a la figura~\ref{fig:model:bdstm}.

Els discs tenen la mida limitada i només poden contenir un nombre
fixat de mesures. Quan un disc no té més capacitat ha d'eliminar una
mesura. Com a conseqüència en un SGSTM la mida és fixada i les sèries
temporals hi queden emmagatzemades a trossos; és a dir com a subsèries
temporals.



\section{Implementació de la multiresolució}

Els SGSTM es poden implementar amb llenguatges d'alt nivell o de baix
nivell. Cadascun ofereix avantatges i inconvenients.

Els llenguatges d'alt nivell faciliten una implementació genèrica dels
SGSTM que n'incorpori totes les capacitats i sigui totalment flexible
a nous canvis i a treballar amb varis tipus de dades. Actualment
s'està desenvolupant amb Python una implementació genèrica dels SGSTM.


En baixar de nivell es fa més difícil aconseguir implementacions
genèriques i fidedignes als models lògics, però s'aconsegueix més
especificitat i eficiència en un determinat problema i àmbit d'aplicació.

RRDtool és una implementació de SGSTM específica per a sistemes
d'adquisició de dades que provenen bàsicament de comptadors. En
aquesta implementació específica, la multiresolució té un cert esquema
prefixat i té com a avantatges l'emmagatzematge eficient al disc de les
sèries temporals i una ràpida visualització.


\section{Proposta d'implementació}

Es poden realitzar altres implementacions específiques dels SGSTM, i
és en aquest sentit que proposem l'estudi i implementació d'un SGSTM
específic en xarxes de sensors. En aquest cas es tractaria d'una
implementació en baix nivell encastada en el sensor que podria seguir
l'esquema de la figura~\ref{fig:vhdl:resolucio}, el qual és per a cada
subsèrie resolució i per tant una sèrie multiresolució estaria formada
per múltiples subsèries d'aquestes.


\begin{figure}[htp]
\centering
\begin{tikzpicture}
\tikzset{
    maquina/.style={rectangle,rounded corners,draw=black, 
      very thick, inner sep=1em, minimum size=3em, text centered,
      groc},
    interficie/.style={rectangle,rounded corners,draw=black, 
       inner sep=0.2em, minimum size=1em, text centered,
      verd},
    modul/.style={rectangle,rounded corners,draw=black, 
      very thick, inner sep=1em, minimum size=3em, text centered,
      roig},   
    myarrow/.style={->, >=latex', shorten >=1pt, thick},
    fletxaswitch/.style={<->, >=latex',shorten >=10pt,shorten <=10pt, thick},
    mylabel/.style={text width=7em, text centered},
    groc/.style={top color=white, bottom color=yellow!50},
    verd/.style={top color=white, bottom color=green!50},
    roig/.style={top color=white, bottom color=red!50},
  }  

  
   \node (discres) [draw, dotted, minimum width=9.5cm, text depth=9cm, rectangle] {Subsèrie resolució};



  \node[modul,text depth=3cm,below right=1cm and 1.7cm of discres.north west] (buffer) {Buffer};  

  %entrades
  \node[above left=-1.5cm and 2.5cm of buffer.north west] (buffer_valor)   {};
  \draw[-] (buffer_valor) -- (buffer_valor-|buffer.west)
   node[near end,above]{valor};

   \node[below=0.5cm of buffer_valor] (buffer_nou)   {};
   \draw[-] (buffer_nou) -- (buffer_nou-|buffer.west)
   node[near end,above]{temps};

   \node[above left=-3.5cm and 1cm of buffer.north west] (buffer_consolida) {};
   \draw[-] (buffer_consolida) -- (buffer_consolida-|buffer.west)
   node[pos=0.2,above]{consolida};

   %sortides
   \node[above right=-1.5cm and 1.5cm of buffer.north east] (buffer_dada)   {};
  \draw[-] (buffer_dada) -- (buffer_dada-|buffer.east)
   node[pos=0,above]{atribut agregat};





  \node[modul,right=3cm of buffer,text depth=3cm] (disc)   {Disc}; 

  % entrades
  \node[above left=-1.5cm and 2cm of disc.north west] (disc_valor)   {};
  \draw[-] (disc_valor) -- (disc_valor-|disc.west)
   node[near end,above]{};

  \node[above left=-3.5cm and 1.96cm of disc.north west] (disc_consolida)   {};
  \draw[-] (disc_consolida) -- (disc_consolida-|disc.west)
   node[pos=0.58,above]{consolida};

   % sortides
   \node[above right=-1.5cm and 2.5cm of disc.north east] (disc_d0)   {};
   \draw[-] (disc_d0) -- (disc_d0-|disc.east)
   node[near end,above]{$D_0$};

   \node[below=0.5cm of disc_d0] (disc_d1)   {};
  \draw[-] (disc_d1) -- (disc_d1-|disc.east)
   node[near end,above]{$D_1$};

   \node[below=0.5cm of disc_d1] (disc_d2)   {};
  \draw[-] (disc_d2) -- (disc_d2-|disc.east)
   node[near end,above]{$\dots$};

   \node[below=0.5cm of disc_d2] (disc_d3)   {};
  \draw[-] (disc_d3) -- (disc_d3-|disc.east)
   node[near end,above]{$D_k$};




  \node[modul,below=1cm of buffer,text depth=1.5cm] (temps)   {Temps}; 

  % entrades
  \node[above left=-1cm and 2.5cm of temps.north west] (temps_rtc)   {};
  \draw[-] (temps_rtc) -- (temps_rtc-|temps.west)
   node[near end,above]{RTC};

  % sortides
   \node[above right=-1cm and 1cm of temps.north east] (temps_delta)   {};
   \draw[-] (temps_delta) -- (temps_delta-|temps.east)
   node[near end,above]{$\delta$};

   \node[above right=-2cm and 7cm of temps.north east] (temps_tau)   {};
   \draw[-] (temps_tau) -- (temps_tau-|temps.east)
   node[pos=0.96,above]{$\tau$};








   %connexions
   \draw[-] (temps_delta.west) -- (disc_consolida.east); 
   
%   \node[above left=0.3cm and 1cm of temps.north west] (tau_reset)   {};
   \node[below=1cm of buffer_consolida] (tau_reset)   {};
   \draw[-*,shorten >=-2pt] (tau_reset) -- (tau_reset-|disc_consolida.east);
   \draw[-] (tau_reset.east) -- (tau_reset.east|-buffer_consolida);


 \end{tikzpicture}
 
\caption{Esquema genèric d'una subsèrie resolució}
\label{fig:vhdl:resolucio}
\end{figure}

Aquesta implementació encastada es pot realitzar tant en un
microcontrolador que emmagatzemi els valors a la memòria seguint
l'esquema de multiresolució o bé en una FPGA aprofitant que l'esquema
multiresolució té una mida finita i és per tant implementable en
hardware. 

En aquesta implementació només fem referència a la part
d'emmagatzematge. La part de tractament i consultes s'hauria de
resoldre en un sistema a part, el qual tingués més flexibilitat en el
tractament de dades. Si bé caldria implementar un protocol per tal que
aquest sistema rebés les dades emmagatzemades, el qual de forma
senzilla es podria implementar com si la base de dades multiresolució
fos un perifèric de memòria.





%\section{Aplicacions de la multiresolució en el baix nivell}




\section{MODEL}








Un disc resolució està format per un buffer, un disc i un controlador del temps.


\begin{figure}[htp]
\centering
\begin{tikzpicture}
\tikzset{
    maquina/.style={rectangle,rounded corners,draw=black, 
      very thick, inner sep=1em, minimum size=3em, text centered,
      groc},
    interficie/.style={rectangle,rounded corners,draw=black, 
       inner sep=0.2em, minimum size=1em, text centered,
      verd},
    modul/.style={rectangle,rounded corners,draw=black, 
      very thick, inner sep=1em, minimum size=3em, text centered,
      roig},   
    myarrow/.style={->, >=latex', shorten >=1pt, thick},
    fletxaswitch/.style={<->, >=latex',shorten >=10pt,shorten <=10pt, thick},
    mylabel/.style={text width=7em, text centered},
    groc/.style={top color=white, bottom color=yellow!50},
    verd/.style={top color=white, bottom color=green!50},
    roig/.style={top color=white, bottom color=red!50},
  }  

  
   \node (discres) [draw, dotted, minimum width=9.5cm, text depth=9cm, rectangle] {Subsèrie resolució};



  \node[modul,text depth=3cm,below right=1cm and 1.7cm of discres.north west] (buffer) {Buffer};  

  %entrades
  \node[above left=-1.5cm and 2.5cm of buffer.north west] (buffer_valor)   {};
  \draw[-] (buffer_valor) -- (buffer_valor-|buffer.west)
   node[near end,above]{valor};

   \node[below=0.5cm of buffer_valor] (buffer_nou)   {};
   \draw[-] (buffer_nou) -- (buffer_nou-|buffer.west)
   node[near end,above]{temps};

   \node[above left=-3.5cm and 1cm of buffer.north west] (buffer_consolida) {};
   \draw[-] (buffer_consolida) -- (buffer_consolida-|buffer.west)
   node[pos=0.2,above]{consolida};

   %sortides
   \node[above right=-1.5cm and 1.5cm of buffer.north east] (buffer_dada)   {};
  \draw[-] (buffer_dada) -- (buffer_dada-|buffer.east)
   node[pos=0,above]{atribut agregat};





  \node[modul,right=3cm of buffer,text depth=3cm] (disc)   {Disc}; 

  % entrades
  \node[above left=-1.5cm and 2cm of disc.north west] (disc_valor)   {};
  \draw[-] (disc_valor) -- (disc_valor-|disc.west)
   node[near end,above]{};

  \node[above left=-3.5cm and 1.96cm of disc.north west] (disc_consolida)   {};
  \draw[-] (disc_consolida) -- (disc_consolida-|disc.west)
   node[pos=0.58,above]{consolida};

   % sortides
   \node[above right=-1.5cm and 2.5cm of disc.north east] (disc_d0)   {};
   \draw[-] (disc_d0) -- (disc_d0-|disc.east)
   node[near end,above]{$D_0$};

   \node[below=0.5cm of disc_d0] (disc_d1)   {};
  \draw[-] (disc_d1) -- (disc_d1-|disc.east)
   node[near end,above]{$D_1$};

   \node[below=0.5cm of disc_d1] (disc_d2)   {};
  \draw[-] (disc_d2) -- (disc_d2-|disc.east)
   node[near end,above]{$\dots$};

   \node[below=0.5cm of disc_d2] (disc_d3)   {};
  \draw[-] (disc_d3) -- (disc_d3-|disc.east)
   node[near end,above]{$D_k$};




  \node[modul,below=1cm of buffer,text depth=1.5cm] (temps)   {Temps}; 

  % entrades
  \node[above left=-1cm and 2.5cm of temps.north west] (temps_rtc)   {};
  \draw[-] (temps_rtc) -- (temps_rtc-|temps.west)
   node[near end,above]{RTC};

  % sortides
   \node[above right=-1cm and 1cm of temps.north east] (temps_delta)   {};
   \draw[-] (temps_delta) -- (temps_delta-|temps.east)
   node[near end,above]{$\delta$};

   \node[above right=-2cm and 7cm of temps.north east] (temps_tau)   {};
   \draw[-] (temps_tau) -- (temps_tau-|temps.east)
   node[pos=0.96,above]{$\tau$};








   %connexions
   \draw[-] (temps_delta.west) -- (disc_consolida.east); 
   
%   \node[above left=0.3cm and 1cm of temps.north west] (tau_reset)   {};
   \node[below=1cm of buffer_consolida] (tau_reset)   {};
   \draw[-*,shorten >=-2pt] (tau_reset) -- (tau_reset-|disc_consolida.east);
   \draw[-] (tau_reset.east) -- (tau_reset.east|-buffer_consolida);


 \end{tikzpicture}
 
\caption{Esquema genèric d'un disc resolució}
\label{fig:vhdl:disc-resolucio}
\end{figure}


Condicions:
\begin{itemize}
\item Àmbit: xarxes de sensors
\item No apte per processos industrials ni llaços de control, en els
  quals cal tenir totes les dades
\item Possibilitat de validació de dades del sensor (en el buffer) i
  generar avisos de funcionament incorrecte.
\item Interfície per a consulta les dades?
\end{itemize}





Estructures alternatives:
\begin{itemize}
\item timestamps absoluts: $\delta$ creats a partir de RTC
\item timestamps relatius: $\delta$ creats a partir del clock
\item timestamps creixents: l'RTC el marca el temps de la mesura
\item Memòria estàtica en comptes de volàtil. Útil per a: a) tenir dades permanents en cas d'apagada o b) per a poder consumir menys (?)   ->  tot això no seria millor fer-ho per programa (CPU) que implementat físicament (aleshores seria més reprogramable)?
\end{itemize}





Base de dades completa, conjunt de discs resolució:
\usetikzlibrary{shapes,arrows,positioning}

\begin{tikzpicture}
 \tikzset{
        myarrow/.style={->, >=latex',  thick},
      }
      

  \node[rectangle,draw,minimum height=6cm,minimum width=9cm] (m) {};
  \draw[shift=( m.south west)]   
  node[above right] {base de dades multiresolució};


  %discmig
  \node (m.center) (discr1) {...};

  %discr
  
  \node[ellipse,draw,minimum height=3.5cm,minimum width=2.5cm,alias=discr0] [left=of discr1] {};
  \node[above=0cm of discr0.north] {$R_0$};
  \node[below=0cm of discr0] {disc resolució};

  \node[cylinder, draw, shape border rotate=90, aspect=0.25,alias=buffer0] [below=3mm of discr0.north] {buffer};
  \node[circle, draw,alias=disc0]  [above=3mm of discr0.south] {disc} ;
  \draw [->] (disc0.center)++(.4:.4cm) arc(0:180:.4cm);
  \draw[myarrow] (buffer0.bottom) -- (disc0.north);


  %discrd

  \node[ellipse,draw,minimum height=3.5cm,minimum width=2.5cm,alias=discrd] [right=of discr1] {};
  \node[above=0cm of discrd] {$R_d$};
  \node[below=0cm of discrd] {disc resolució};

  \node[cylinder, draw, shape border rotate=90, aspect=0.25,alias=bufferd] [below=3mm of discrd.north] {buffer};
  \node[circle, draw,alias=discd]  [above=3mm of discrd.south] {disc} ;
  \draw [->] (discd.center)++(.4:.4cm) arc(0:180:.4cm);
  \draw[myarrow] (bufferd.bottom) -- (discd.north);



  %mesura 
  \node[above=1cm of m.north] (m0) {};

  \draw[myarrow] (m0) -- (m.north) 
  node[right,midway] {mesura};

  \draw[myarrow] (m.north) -- (buffer0);
  \draw[myarrow] (m.north) -- (bufferd);
  \draw[myarrow] (m.north) -- (discr1);

\end{tikzpicture}




Base de dades amb discs resolució enllaçats:

\begin{tikzpicture}
 \tikzset{
        myarrow/.style={->, >=latex',  thick},
      }
      

  \node[rectangle,draw,minimum height=6cm,minimum width=9cm] (m) {};
  \draw[shift=( m.south west)]   
  node[above right] {base de dades multiresolució};


  %discmig
  \node (m.center) (discr1) {...};

  %discr
  
  \node[ellipse,draw,minimum height=3.5cm,minimum width=2.5cm,alias=discr0] [left=of discr1] {};
  \node[above=0cm of discr0.north] {$R_0$};
  \node[below=0cm of discr0] {disc resolució};

  \node[cylinder, draw, shape border rotate=90, aspect=0.25,alias=buffer0] [below=3mm of discr0.north] {buffer};
  \node[circle, draw,alias=disc0]  [above=3mm of discr0.south] {disc} ;
  \draw [->] (disc0.center)++(.4:.4cm) arc(0:180:.4cm);
  \draw[myarrow] (buffer0.bottom) -- (disc0.north);


  %discrd

  \node[ellipse,draw,minimum height=3.5cm,minimum width=2.5cm,alias=discrd] [right=of discr1] {};
  \node[above=0cm of discrd] {$R_d$};
  \node[below=0cm of discrd] {disc resolució};

  \node[cylinder, draw, shape border rotate=90, aspect=0.25,alias=bufferd] [below=3mm of discrd.north] {buffer};
  \node[circle, draw,alias=discd]  [above=3mm of discrd.south] {disc} ;
  \draw [->] (discd.center)++(.4:.4cm) arc(0:180:.4cm);
  \draw[myarrow] (bufferd.bottom) -- (discd.north);



  %mesura 
  \node[above=1cm of m.north] (m0) {};

  \draw[myarrow] (m0) -- (m.north) 
  node[right,midway] {mesura};

  \draw[myarrow] (m.north) -- (buffer0);
  \draw[myarrow] (discr1.south) -- (bufferd);
  \draw[myarrow] (disc0) -- (discr1.north);

\end{tikzpicture}









% \begin{tikzpicture}[circuit logic IEC,
%   every circuit symbol/.style={
%     logic gate IEC symbol color=black,
%     fill=blue!20,draw=blue,very thick}]
%   \matrix[column sep=7mm]
%   {
%     \node (i0) {0}; &
%     & \\
%     & \node [and gate] (a1) {}; & \\
%     \node (i1) {0}; &
%     & \node [or gate] (o) {};\\
%     & \node [nand gate] (a2) {}; & \\
%     \node (i2) {1}; &
%     & \\
%   };
%   \draw (i0.east) -- ++(right:3mm) |- (a1.input 1);
%   \draw (i1.east) -- ++(right:3mm) |- (a1.input 2);
%   \draw (i1.east) -- ++(right:3mm) |- (a2.input 1);
%   \draw (i2.east) -- ++(right:3mm) |- (a2.input 2);
%   \draw (a1.output) -- ++(right:3mm) |- (o.input 1);
%   \draw (a2.output) -- ++(right:3mm) |- (o.input 2);
%   \draw (o.output) -- ++(right:3mm);
% \end{tikzpicture}








% \def\degr{${}^\circ$}
% \begin{tikztimingtable}
%   Clock 128\,MHz 0\degr    & H   12{2C} G \\ % ends with edge
%   Clock 128\,MHz 90\degr   & [C] 12{2C} C \\ % starts with edge
%   Clock 128\,MHz 180\degr  & C   12{2C} G \\ % ends with edge
%   Clock 128\,MHz 270\degr  &     12{2C} C \\
% \end{tikztimingtable}





\subsection{Possibles aplicacions}

Exemple d'aparell encastat molt petit, per exemple un aparell que
s'hagi de posar dins del cos per a mesurar la temperatura, a on hi ha
molt poc espai disponible i només permet emmagatzemar 100 dades. Es
poden desar 100 valors cada hora = 4 dies, o bé es pot aplicar una
solució de multiresolució i desa 75 valors cada hora = 2 dies i 25
valors cada dia = 25 dies. Així si no s'és a temps de recuperar les
dades en quatre dies sempre hi haurà alguna informació.

Exemple de la implementació de circuits integrats en materials molt prims i doblegables.




Bàsicament en les implementacions a baix nivell d'un SGSTM tenim dues opcions:

\begin{itemize}
\item implementar-ho amb llenguatge de baix nivell (assemblador, C) en un microcontrolador. Aquesta seria la manera típica perquè dóna molta flexibilitat: es pot reconfigurar l'esquema de multiresolució quan es vulgui, es poden utilitzar les operacions que es vulgui, canviar la mida, etc. 

\item implementar-ho amb hardware, per exemple dissenyar amb
  VHDL. Inconvenients: gens flexible (un cop implementat no es pot
  canviar), difícil de fer genèric (no es pot trobar un circuit que
  serveixi per a fer varis càlculs, per això caldria un
  microcontrolador). Avantatges: es pot compactar i encastar en
  llocs petits, estructures de càlcul paral·leles (la feina no recau
  en el microcontrolador), bases de dades distribuïdes. Per tant es fa
  difícil pensar de vendre BDM genèriques hardware però sí que es
  poden pensar algunes possibles aplicacions que només serien
  exclusives de hardware:

  \begin{itemize}
  \item Sensors inte\l.ligents o totalment encastats. Actualment es
    dissenyen sensors acompanyats de circuits digitals, tot encastat
    en un xip petit. Fan filtratge dels senyal, tenen un bus de
    comunicació senzill amb el controlador (I2C, 1-Wire, SPI, etc.), el
    controlador pot indicar quan s'ha d'iniciar la captura d'un no
    valor, s'emmagatzema el darrer valor capturat i el controlador el
    consulta quan vol, poden establir uns llindars d'alarma... Així
    doncs aquests sensors només emmagatzemen el darrer valor, es
    podria proposar que emmagatzemessin amb esquema multiresolució, el
    qual hauria de ser una mica configurable per exemple els períodes
    de consolidació. No obstant això, la configuració seria molt poc
    flexible i si es necessiten càlculs més complicats sempre és
    millor seguir amb l'esquema habitual del microcontrolador captura
    la dada i ell gestiona els càlculs. Ara bé, també es pot entendre
    el sensor inte\l.ligent com que forma part d'una base de dades
    distribuïda i ell té una part de l'emmagatzematge. 
    
  \item Esquema multiresolució encastat en perifèrics per a monitorar
    el seu funcionament: en una impressora en els comptadors
    d'impressions, en una antena de comunicació sense fils en els
    comptadors de bytes transmesos o la potència transmesa, etc.

  \item Implementació de circuits integrats en llocs mai vistos:
    materials molt prims i doblegables. Aquí segurament hi ha
    problemes de la mida dels circuits que es poden implementar, per
    tant poder-hi disposar d'esquemes multiresolució aniria bé.

  \item Possibilitat de disseny de FPGA casolans. Si mai això fos
    possible (sembla que ja és possible: exemple de FPGA a la
    raspberry, System-on-Chip (SoC)), els sensors intel·ligents o
    totalment encastats prendrien molt sentit ja que serien
    reconfigurables o que fos molt barat implementar en hardware un
    circuit i llençar-lo quan es volgués canviar de configuració. Això
    llavors encaixaria amb construir un xip per blocs: hi poso un bloc
    de comunicacions, un bloc de tal i un bloc de base de dades
    multiresolució amb tal esquema. Tot i així, sembla que en els SoC
    la idea és implementar-ho com a microcontrolador i fins i tot
    implementar-hi sistemes operatius.

  \item Buffers amb agregadors molt complexos: si l'agregador és complex cal implementar-ho en hardware, suposem per exemple que agrega imatge. Gestió de 1 milió de sèries temporals (article de iSAX): això com es fa? la solució podria ser sistemes hardware.


  \end{itemize}


\end{itemize}

En l'àmbit, no es veu clara la implementació de coses en VHDL. Sembla que es desaprofitar recursos perquè la feina ja la pot fer el microcontrolador. Potser l'aplicació útil i que s'ha de destacar és quan en els xips integrats no hi ha microcontrolador com poder-hi posar una base de dades?

Però sí que hi pot haver un problema de temps de computació i per tant
un aparell hardware seria necessari. Si prenem l'operació multiresolució sobre una sèrie temporal en un SGST, aquesta pot trigar molt a computar-se. Hi ha solucions en el món NoSQL, per exemple utilitzar MapReduce per a resoldre-ho en sistemes de computació para\l.lels. Però una altra solució de computació para\l.lela podria ser tenir el SGSTM implementat en hardware, aleshores aniria inserint les mesures en ordre temporal i al final obtindria el resultat; tot computat en un hardware para\l.lel. Tot i així cal notar que a més aquesta computació es pot fer en orientació stream si es coneix a priori l'operació de multiresolució que es vol fer.



%%% Local Variables:
%%% TeX-master: "main"
%%% End:

\chapter{Exemple}
\label{sec:implementacions:exemple}

Un exemple amb dades massives.


\part{Conclusions}

\begin{frame}{Conclusions}
\end{frame}

\begin{frame}{Treball futur}
\end{frame}



%%% Local Variables: 
%%% mode: latex
%%% TeX-master: "defensa"
%%% End: 
\chapter{Treball futur}
\label{sec:futur}


En aquest capítol es proposen treballs futurs a partir del que s'ha
dissertat en aquest document. El treball presentat obre la
possibilitat a nous temes de recerca, alguns dels quals creiem que són
reptes interessants. 


Proposem nous treballs futurs al voltant de tres temes: els models,
les implementacions i reflexions sobre la qualitat de la
multiresolució.


\section{Models}



En el model de \gls{SGST} s'ha definit un seguit d'operacions que són
les que considerem més bàsiques per a poder manipular les sèries
temporals. Algunes operacions són conseqüència directa d'altres
conceptes; per exemple la diferència a partir de la pertinença o la
intersecció a partir de la diferència. D'altres operacions són
conseqüència però cal prendre una decisió sobre el raonament; per
exemple en la unió i la unió temporal cal decidir quina prové de
l'ordre total i quina de l'ordre parcial.  I d'altres operacions són
passos previs per a altres operacions; per exemple la concatenació
temporal s'utilitza en els \gls{SGSTM} per a calcular la sèrie
temporal total en el context d'un mètode de representació.  Així
doncs, caldria establir clarament la motivació de cada operació i
cercar en cada cas el raonament sobre el qual es pot basar cada
operador. També, de forma breu, hem notat les propietats d'alguns
operadors com la commutativitat, però caldria explorar més
profundament les propietats de tots els operadors.


En les funcions d'agregació d'atributs dels \gls{SGSTM}, n'hem
proposat alguns exemples sobretot raonats a partir dels mètodes de
representació. Tot i així, hem proposat estadístics senzills
--mitjana, màxim i darrer-- atès que l'objectiu és mostrar el
comportament que tenen en la multiresolució.  En els treballs
d'anàlisi de sèries temporals es presenten multitud de mètodes i
d'algoritmes: per a extreure patrons de les sèries temporals, per a
cercar periodicitats, per a comparar dues sèries temporals,
agregacions en el domini freqüencial, per a fer prediccions, per a
validació de dades, etc. Per tant, es poden dissenyar més funcions
d'agregació d'atributs basant-se en qualsevol d'aquests algoritmes o
mètodes; només cal adaptar el problema per tal de retornar una mesura
que resumeixi la informació d'un interval de la sèrie temporal.



De les funcions d'agregació d'atributs n'hem notat la possibilitat
d'orientar-les a flux. El model de \gls{SGSTM} és adequat per a
computar-se en flux llevat de les funcions d'agregació d'atributs que
es defineixen genèriques. Creiem, doncs, que també és interessant
aplicar orientació a flux en aquestes funcions i caldria aprofundir en
aquests algoritmes, com per exemple els que proposa
\textcite{cormode08:pods}.



En la teoria de la mesura, la incertesa sol acompanyar les mesures. La
incertesa reflecteix probabilísticament els límits del que es coneix
sobre la quantitat mesurada.  Així doncs, seria interessant poder
incorporar la incertesa en els models.  Principalment la incertesa
hauria d'acompanyar els atributs de temps i de valor de les sèries
temporals, és a dir haurien de reflectir la incertesa que hi ha en
cada mesura a l'hora d'adquirir un valor un instant de
temps. Aleshores, caldria estudiar com aquesta incertesa afecta les
operacions, és a dir com es propaga la incertesa quan s'uneixen dues
sèries temporals, quan es representen, en les funcions d'agregació
d'atributs, etc.




Pel que fa a la consolidació de la multiresolució, aquesta s'ha pensat
sobretot periòdica per tal d'obtenir sèries temporals regulars.
Aleshores s'obtenen els esquemes de multiresolució periòdics que hem
analitzat. Això no obstant, altres escenaris de consolidació són
possibles; per exemple sistemes de monitoratge que adquireixen dades
només quan ho creuen interessant en base a esdeveniments.  En el model
de \gls{SGSTM} no estan previstos aquests casos i requereixen un
estudi més profund. Per exemple, en un cert moment es podria
considerar que una subsèrie resolució és molt significativa i que
s'han de mantenir aquestes dades, per tant en l'esquema de
multiresolució hi hauria una subsèrie fixada a un interval de temps en
el passat.





En els esquemes de multiresolució, s'ha notat la propietat de
desfasament d'una subsèrie resolució. Aquest desfasament és induït per
la funció d'agregació d'atributs. D'una banda, aquest desfasament pot
ser requerit per la naturalesa de la funció; per exemple en el cas
d'agregacions \gls{dd} o \gls{zoh} introdueixen un desfasament perquè
interpolen cap endavant, de fet el resultat de consolidació entre la
\gls{zoh} i la \gls{zohe} és similar però aquesta darrera per
naturalesa interpola cap enrere i no necessita desfasament.  D'altra
banda, aquest desfasament podria ser afegit intencionalment per a
controlar els solapament entre les subsèries resolució.  Aquest
solapament significa que en l'esquema de multiresolució les diverses
subsèries resolució coincideixen a emmagatzemar informació pels
mateixos instants de temps. Proposem dos escenaris de solapament:
\begin{itemize}

\item Les subsèries resolució se solapen totalment. Això pot servir
  per a disposar de diferents resums de la sèrie temporal preparats
  per a ser visualitzats immediatament. És a dir, el \gls{SGSTM}
  permet escollir ràpidament entre diferents \emph{zooms} de les
  dades.

\item Les subsèries resolució no se solapen, és a dir les subsèries
  amb menys resolució acaben allà on comencen les de més
  resolució. Això pot servir per a aprofitar al màxim la resolució i
  l'espai d'emmagatzematge, sense que cap subsèrie desi informació per
  al mateix interval de temps. A part d'introduir desfasament, per tal
  que no se solapin també es pot dissenyar un esquema de
  multiresolució on el pas de consolidació de cada subsèrie sigui
  exactament $\delta_j = k_i\delta_i$ on $\delta_i<\delta_j$. És a dir
  la subsèrie amb menys resolució ($j$) té un pas de consolidació
  exactament múltiple del pas de consolidació i el cardinal de la
  subsèrie superior en resolució ($i$). Aquestes restriccions es
  podrien considerar també en el cas de l'estructura de resolucions
  encadenades.
\end{itemize}


%També l'emmagatzematge de metades, es  pot fer perfectament amb els SGBDR.









\section{Implementacions}


En les implementacions hem treballat a nivell acadèmic, és a dir sense
objectius d'optimització del rendiment. De fet, aleshores les
implementacions s'allunyarien del model i per tant del nostre objectiu
de mantenir una forta correspondència entre la forma de la
implementació i del model. Aquesta correspondència és útil per a
manteniments futurs: qualsevol millorar en el model pot ser
traslladada immediatament a les implementacions o bé, a la inversa,
qualsevol error trobat en les implementacions pot ser localitzat
fàcilment i estudiat en el model.



%RoundRobinson


Tot i així, aquesta correspondència model-implementació no sempre és
senzilla de mantenir. A Pytsms i RoundRobinson, les implementacions
més completes que hem realitzat dels models, s'hauria de simplificar
la la definició de les mesures genèriques.  En el model abstracte
matemàtic és senzill de descriure uns valors genèrics, en canvi a les
implementacions això complica l'estructura. Així, s'ha hagut de
dissenyar mesures de diferents tipus que contenen el rang del domini
de temps i valors per tal de definir les mesures indefinides, el
suprem i l'ínfim, etc.  També s'hauria de repensar la gestió de la
homogeneïtat de les sèries temporals: en el model les sèries temporals
són homogènies i en les implementacions és difícil gestionar el
concepte de nova sèrie temporal amb el mateix tipus de mesures que les
originals.

En altres casos, però, s'ha trobar una solució adequada per a
implementar la genericitat del model.  Per exemple, és el cas de la
multitud d'operacions de les sèries temporals implementades amb
Mixins, el de les representacions com a objectes independents
associats a les sèries temporals, o bé els de les funcionalitats
complementàries com l'emmagatzematge i els gràfics implementades amb
el patró Visitor. Tanmateix, algunes parts encara no han estat prou
generalitzades; per exemple els gràfics de RoundRobinson sempre
utilitzen la representació \gls{zohe}.





En les altres implementacions, l'objectiu s'ha centrat en observar
altres paradigmes d'implementació dels models. Així, ens hem pres la
llibertat de no implementar tota la genericitat del model sinó casos
simplificats. Caldria avaluar fins a quin límit aquestes
implementacions es podrien apropar més al model, per exemple a
RoundRobindoop hem notat algunes limitacions a l'hora d'usar les
funcions d'agregació d'atributs.


%RoundRobindoop

A RoundRobindoop usem Hadoop com a intèrpret de la tècnica de
programació para\l.lela MapReduce. L'execució d'aquesta tècnica
implica un compromís a l'hora d'escollir el nombre de processos en
para\l.lel ja que cada un té un cost mínim de crear-se i a més cal
comptar el cost de distribuir les dades. Es podria experimentar més
amb Hadoop en aquest sentit, és a dir amb diferents quantitats de maps
i de reduces. De fet només hem provat amb un node de computació, però
Hadoop té la possibilitat de distribuir a més nodes.


Hi ha altres projectes que també utilitzen Hadoop com a sistema
d'emmagatzematge distribuït de sèries temporals. Aquest és el cas
d'OpenTSDB \parencite{opentsdb}, que utilitza Hadoop per a
emmagatzemar i recuperar ràpidament sèries temporals.  Aquest, però,
només no té en compte l'aplicació de consultes a les dades sinó només
recuperar les dades originals.  RoundRobindoop és una solució per a
computar la multiresolució a Hadoop. Així doncs, OpenTSDB i
RounbdRobindoop podrien treballar conjuntament: el primer per a
emmagatzemar distribuïdament les sèries temporals i el segon per a
calcular la multiresolució aprofitant que les dades ja estan
distribuïdes en diversos nodes.



%Relstsms


A Reltsms hem implementat el model d'\gls{SGST} seguint la programació
acadèmica del model relacional. Es podria seguir la mateixa
aproximació per a implementar també el model d'\acro{SGSTM}.  En
aquestes implementacions, es podria experimentar amb un dels punts
forts dels \gls{SGBDR}: l'optimització de les
consultes \parencite[\gls{capitol}~18
\emph{Optimization}]{date04:introduction8}. Les expressions
relacionals són d'alt nivell matemàtic i això permet trobar
expressions equivalents a una consulta. Aleshores, els sistemes poden
decidir quina expressió és la millor per a ser executada.  En aquest
sentit, hem definit els operador de Reltsms a partir dels operadors
relacionals. Això no obstant, s'hauria d'estudiar si a Tutorial~D les
funcionalitats d'optimització s'estenen automàticament als operadors
derivats dels primitius.



En un sentit relacional, també cal comparar Pytsms i RoundRobinson amb
Dee. Dee \parencite{dee} és la implementació amb Python d'un
llenguatge de bases de dades relacionals que compleixi amb les normes
D del Third Manifesto. A Pytsms i RoundRobinson hem utilitzat els
conjunts de Python com a objectes bàsics però es podrien utilitzar els
conjunts relacionals de Dee, els quals ja incorporen propietats i
mètodes de \gls{SGBD}. Si més no, el raonament que fa Dee com a gestor
de bases de dades també podria ser aplicat a Pytsms i
RoundRobinson.
  



%Experiment amb dades


\todo{fer} 


Amb l'experimentació amb dades reals hem pogut demostrar el correcte funcionament de la multiresolució. 

* Experimentació amb dades: només demostrem el correcte funcionament de la multiresolució. No avaluem rendiment dels recursos, per exemple no avaluem temps de computació només n'hem mostrat un a tall de referència orientativa.  Cal provar amb més diversitat: dades de diferent mida, de diversa naturalesa, diferents esquemes de multiresolució, dades de referència utilitzades en altres recerques (per exemple keogh?)





* Implementar variacions dels \gls{SGSTM}


  Una mostra que es pot millorar molt
el codi és que canviant mètode add de les TimeSeries amb
MeasureTotalEquality hem aconseguit una millora espectacular de temps
de computació: en la inserció cal cercar que no hi hagin temps
repetits i ara s'aprofita la cerca dels sets de Python mitjançant
hash.

* Sobre de Pytsms es podria implementar també la funció de multiresolució

*de fet falta demostrar l'equivalència entre el model de SGSTM i les funcions de multiresolució






\subsection{Sistemes de multiresolució  integrats en maquinari}

Es poden realitzar altres implementacions dels \gls{SGSTM} que siguin
molt específiques. Una bona implementació ja no és només aquella que
calcula en poc temps sinó que en alguns contextos també pot ser un
consum baix d'energia, ocupar poc espai, etc.  En aquest sentit,
pensem en sistemes específics integrats en xarxes de sensor. Així, es
podria implementar un \gls{SGSTM} integrat en el maquinari d'un sensor.

Aquesta implementació integrada es podria realitzar tant en un
microcontrolador que gestionés una memòria seguint l'esquema de
multiresolució o bé com a circuit digital, aprofitant que l'esquema de
multiresolució té una mida finita i és implementable en maquinari.


En la implementació en maquinari, es podria seguir l'esquema de la
\autoref{fig:vhdl:resolucio}. Aquest esquema és per a una subsèrie
resolució, per tant una sèrie temporal multiresolució seria un conjunt
d'aquests esquemes. Així, aquest esquema és la integració de l'esquema
de les subsèries resolució de la \autoref{fig:sgstm:bdsubserie}.  En
el buffer es van afegint les mesures --temps i valor-- i calcula la
mesura resultant --l'atribut agregat. Aquest atribut agregat
s'emmagatzema al disc a cada pas de consolidació --marcat per
l'esdeveniment consolida-- i el disc gestiona l'emmagatzematge afitat
--les dades consolidades $D_0,D_1,\dotsc,D_k$. Caldria afegir un mòdul
de temps que a partir del rellotge -- per exemple un real-time clock
(RTC)-- marqués els passos de consolidació i indiqués el darrer temps
de consolidació. 





\begin{figure}[htp]
\centering
\begin{tikzpicture}
\tikzset{
    maquina/.style={rectangle,rounded corners,draw=black, 
      very thick, inner sep=1em, minimum size=3em, text centered,
      groc},
    interficie/.style={rectangle,rounded corners,draw=black, 
       inner sep=0.2em, minimum size=1em, text centered,
      verd},
    modul/.style={rectangle,rounded corners,draw=black, 
      very thick, inner sep=1em, minimum size=3em, text centered,
      roig},   
    myarrow/.style={->, >=latex', shorten >=1pt, thick},
    fletxaswitch/.style={<->, >=latex',shorten >=10pt,shorten <=10pt, thick},
    mylabel/.style={text width=7em, text centered},
    groc/.style={top color=white, bottom color=yellow!50},
    verd/.style={top color=white, bottom color=green!50},
    roig/.style={top color=white, bottom color=red!50},
  }  

  
   \node (discres) [draw, dotted, minimum width=9.5cm, text depth=9cm, rectangle] {Subsèrie resolució};



  \node[modul,text depth=3cm,below right=1cm and 1.7cm of discres.north west] (buffer) {Buffer};  

  %entrades
  \node[above left=-1.5cm and 2.5cm of buffer.north west] (buffer_valor)   {};
  \draw[-] (buffer_valor) -- (buffer_valor-|buffer.west)
   node[near end,above]{valor};

   \node[below=0.5cm of buffer_valor] (buffer_nou)   {};
   \draw[-] (buffer_nou) -- (buffer_nou-|buffer.west)
   node[near end,above]{temps};

   \node[above left=-3.5cm and 1cm of buffer.north west] (buffer_consolida) {};
   \draw[-] (buffer_consolida) -- (buffer_consolida-|buffer.west)
   node[pos=0.2,above]{consolida};

   %sortides
   \node[above right=-1.5cm and 1.5cm of buffer.north east] (buffer_dada)   {};
  \draw[-] (buffer_dada) -- (buffer_dada-|buffer.east)
   node[pos=0,above]{atribut agregat};





  \node[modul,right=3cm of buffer,text depth=3cm] (disc)   {Disc}; 

  % entrades
  \node[above left=-1.5cm and 2cm of disc.north west] (disc_valor)   {};
  \draw[-] (disc_valor) -- (disc_valor-|disc.west)
   node[near end,above]{};

  \node[above left=-3.5cm and 1.96cm of disc.north west] (disc_consolida)   {};
  \draw[-] (disc_consolida) -- (disc_consolida-|disc.west)
   node[pos=0.58,above]{consolida};

   % sortides
   \node[above right=-1.5cm and 2.5cm of disc.north east] (disc_d0)   {};
   \draw[-] (disc_d0) -- (disc_d0-|disc.east)
   node[near end,above]{$D_0$};

   \node[below=0.5cm of disc_d0] (disc_d1)   {};
  \draw[-] (disc_d1) -- (disc_d1-|disc.east)
   node[near end,above]{$D_1$};

   \node[below=0.5cm of disc_d1] (disc_d2)   {};
  \draw[-] (disc_d2) -- (disc_d2-|disc.east)
   node[near end,above]{$\dots$};

   \node[below=0.5cm of disc_d2] (disc_d3)   {};
  \draw[-] (disc_d3) -- (disc_d3-|disc.east)
   node[near end,above]{$D_k$};




  \node[modul,below=1cm of buffer,text depth=1.5cm] (temps)   {Temps}; 

  % entrades
  \node[above left=-1cm and 2.5cm of temps.north west] (temps_rtc)   {};
  \draw[-] (temps_rtc) -- (temps_rtc-|temps.west)
   node[near end,above]{RTC};

  % sortides
   \node[above right=-1cm and 1cm of temps.north east] (temps_delta)   {};
   \draw[-] (temps_delta) -- (temps_delta-|temps.east)
   node[near end,above]{$\delta$};

   \node[above right=-2cm and 7cm of temps.north east] (temps_tau)   {};
   \draw[-] (temps_tau) -- (temps_tau-|temps.east)
   node[pos=0.96,above]{$\tau$};








   %connexions
   \draw[-] (temps_delta.west) -- (disc_consolida.east); 
   
%   \node[above left=0.3cm and 1cm of temps.north west] (tau_reset)   {};
   \node[below=1cm of buffer_consolida] (tau_reset)   {};
   \draw[-*,shorten >=-2pt] (tau_reset) -- (tau_reset-|disc_consolida.east);
   \draw[-] (tau_reset.east) -- (tau_reset.east|-buffer_consolida);


 \end{tikzpicture}
 
\caption{Esquema d'integració d'una subsèrie resolució}
\label{fig:vhdl:resolucio}
\end{figure}

En aquesta implementació només fem referència a la part
d'emmagatzematge.  Caldria implementar un protocol per tal de
consultar les dades emmagatzemades, si bé forma senzilla es podria
implementar com si els discs fossin un perifèric de memòria.  

% Podrien ser emmagatzematges volàtils depenent de la criticitat de les dades



Algunes aplicacions dels sistemes integrats de multiresolució podrien ser:

\begin{itemize}
\item Emmagatzematge de la multiresolució en perifèrics la informació
  dels quals no és essencial però pot ajudar a monitorar-ne el seu
  funcionament. Per exemple comptadors d'aparells de xarxa,
  temperatures dels components, etc.

\item Aparells integrats molt petits, en els quals hi ha molt poc
  espai per a l'emmagatzematge. %en materials molt prims i doblegables.

\item Com un complement més de sensors inte\l.ligents, que actualment
  ja integren diverses tasques: filtratge del senyal, busos de
  comunicacions, llindars d'alarma, etc.


\item Per a computar funcions d'agregació d'atributs complexes. En
  aquest cas els buffers podrien treballar directament amb components
  del maquinari. Per exemple per a agregacions de sèries temporals en
  què els valors fossin imatges.

\item Implementació de la multiresolució en Field Programmable Gate
  Arrays, és a dir en dispositius de maquinari configurables. Això
  permetria flexibilitat a l'hora de canviar els esquemes de
  multiresolució integrats.



\end{itemize}





% \subsection{RRDtool}

% \todo{fer}


% Ara hauríem de reprendre RRDtool i avaluar fins a quin punt compleix el nostre model. 
% Segur que hi ha punts en què no compleix... 

% Podem dir que RRDtool és la implementació productiva que actualment més s'apropa al concepte de multiresolució que hem formalitzat.


% Regarding other implementations,
% % \emph{RRDtool} can be seen as an specific case of \acro{MTSMS} and as
% % a NoSQL system, although Oetiker \cite{rrdtool} has not commented
% % it. However, regardless of the implementation backend, we have shown
% % how a generic model for \acro{MTSMS} can be defined firmly rooted on
% % \acro{DBMS} algebra theory.






% % RRDtool té una estructura multiresolució amb un buffer únic d'entrada
% % i buffers orientats a stream; segons havíem avaluat anteriorment \parencite{llusa11:tfm}.


% % S'ha d'estudiar com es fan les consultes a RRDtool

% % \url{http://en.wikipedia.org/wiki/RRD_Editor}



% % Podem considerar que:

% % 1. RRDtool és un SGBD NoSQL?
% % 2. Nosaltres n'hem formalitzat un model lògic?
% % 3. És el primer model lògic per a un producte NoSQL?
% % 4. Aquest model lògic es pot implementar tant en productes relacionals com amb NoSQL? i per tant es demostra que els models lògics són extremadament potents i necessaris?
% % 5. La implementació que fa RRDtool és molt eficient per a un determinat camp d'aplicació?
% % 6. La implementació relacional seria molt genèrica i propera al model però no tan eficient? més aviat subjecte a l'eficiència genèrica dels SGBDR?
% % 7. Els SGST són uns SGBD més simples? no tenen tantes actualitzacions de valors, no hi ha tantes relationships en l'esquema... Els SGST només es preocupen de sèries temporals i per tant només d'un tipus de dades en concret, això no obstant tal com s'ha dissenyat el model aquest tipus de dades es pot implementar en SGBD més complexos. 

% Hi ha RRDtool, que és un SGBD específic dissenyat per a dades monitorades. Les causes del seu disseny són:

% * Tobias Oetiker dissenyava un monitor de paràmetres de xarxes de comunicacions i en aquest monitor una part era la d'emmagatzematge de les dades. Per raons pràctiques i d'utilitat dissenya aquesta part amb un esquema inovadós. Finalment acaba separant aquesta part i la converteix independentment en RRDtool.

% * RRDtool té aquest model pràctic i a la pràctica és molt útil per a ser usat com a SGBD dels sistemes de monitoratge, sobretot en l'àmbit dels comptadors de xarxa on és l'estàndard de facto. 

% Això no obstant, no hi ha cap raonament teòric sobre el model de RRDtool ja que s'ha dissenyat per raons pràctiques. Per tant, entendre el funcionament de RRDtool és complicat, hi ha un nivell molt elevat per començar a fer-lo funcionar i molts conceptes no s'entenen perquè no estan ben definits. 

% Per això ens proposem de compendre i formalitzar el model de RRDtool, que acabarem anomenar model de multiresolució, en la teoria dels sitemes d'informació. A més RRDtool és molt específic pel camp de comptadors de xarxa i volem oferir un model genèric per a altres àmbits.  









\section{Reflexions sobre la qualitat}


El capítol d'aplicació de la teoria de la informació és només una
introducció al problema de la qualitat de la multiresolució.  La
teoria de la informació formalitza anàlisis més profundes per a la
compressió de dades que es podrien aplicar també a la multiresolució.
En el cas que es conegui més bé el context i el comportament de la
sèrie temporal a la qual s'aplica la multiresolució, es pot detallar
més bé la quantificació de l'error. És a dir, en termes de la teoria
de la informació aleshores hi ha més coneixement sobre la predicció
del comportament de les dades cosa que es pot utilitzar per a avaluar
característiques més concretes. Per exemple, una variable real
adquirida té limitat el rang de valors que pot prendre i fins i tot
pot tenir un comportament probabilístic determinat.



Cal notar que no avaluem la idoneïtat d'aplicar un estadístic o un
altre a unes dades ni quin és el que més bé va per a obtenir una
informació. Només expressem el cas que es vol aplicar una consulta amb
una agregació determinada a una sèrie temporal i avaluem l'error que
hi ha comparant l'aplicació de la consulta a les dades originals amb
les dades multiresolucionades. 

Per a determinar un esquema de multiresolució --la quantitat de
resolucions, els passos de consolidació de cadascuna, els cardinals,
\dots-- caldria analitzar cada problema particular en el seu context i
utilitzar els coneixements adequats. Per exemple, per a treballar amb
problemes de so la teoria del senyal formalitza tot de raonaments que
no es poden obviar a l'hora de definir-ne un esquema de
multiresolució.  O bé, també en la teoria del senyal, es poden trobar
anàlisis per a determinar bons passos de consolidació per a una
variable, per exemple de temperatura. Així i tot, en el cas dels
comptadors se'n pot fer un raonament a banda per a conservar la seva
informació genuïna de comptatge lligada a com l'adquireixen.


A partir de les reflexions fetes sobre la qualitat de la
multiresolució s'obren un seguit de qüestions més, cadascuna de les
quals és un repte futur:
\begin{itemize}
\item Quina redundància d'emmagatzematge hi ha entre diverses
  subsèries s'una mateixa sèrie temporal multiresolució.? Què ocorre
  quan hi ha més d'una resolució i són de diferents funcions
  d'agregació d'atributs?

\item En cas que es perdi una resolució, es podria reconstruir a
  partir de les altres? O bé en cas que es vulgui ampliar la mida d'un
  disc, es podria completar amb les dades d'altres resolucions?

\item Les resolucions encadenades són interessant perquè aprofiten les
  dades emmagatzemades però afegeixen més restriccions a la compressió
  d'informació. Com es poden barrejar diferents passos de consolidació
  i diferents funcions d'agregació d'atributs?


\item Gràcies a un esquema de multiresolució es pot emmagatzemar dades
  d'una sèrie temporals durant un llarg temps de forma
  comprimida. Aquesta durada de temps és llarga però finita, tot i que
  quan s'esgoti dinàmicament es pot afegir una nova subsèrie amb
  inferior resolució però amb més lapse. Inicialment aquesta subsèrie
  serà buida, però es podria utilitzar les dades ja emmagatzemades per
  a iniciar-la?

\end{itemize}




















% \subsubsection{Operacions habituals en les sèries temporals}


% \paragraph{Semblança de dues sèries temporals}


% Similarity Measures for Time Series

% Hi ha varis mètodes, [keogh08:vldb] n'avalua uns quants i els generalitza amb:

% Given two
% time series T1 and T2 , a similarity function Dist calcu-
% lates the distance between the two time series, denoted by
% Dist(T1 , T2 ).

% Exemplifiquem amb la distància euclídia, [keogh08:vldb] nota que és
% competitiva amb les altres.

% Distancia euclídia segons [faloutsous94-sigmod]


% \[
% D(S,Q) = \left( \sum_{i=1}^{l} (S[i]-Q[i])^2  \right)^{1/2}
% \]

% \begin{gather*}
%   D: S \times Q \longrightarrow v: \\
%   S' = map(fusio(S,Q),(t,v_1,v_2)\mapsto(t,(v_1-v_2)^2)), \\
%   S'' = fold(quad,(0,0),(t^1,v^1,t^2,v^2)\mapsto(t^1,v^1+v^2)), \\
%   v = \sqrt{V(m)}:m\in S''
% \end{gather*}


% S i Q haurien de ser regulars entre elles, sinó cal aplicar una fusió amb representació/interpretació.

% Amb la multiresolució la fusió es pot fer de forma eficient. Per altra banda, es podria crear un disc resolució amb agregador de semblança.


% \paragraph{Semblança de dues sèries temporals amb offset}

% Aquí es descriu la solució general del problema (SequentialScan),
% [faloutsous94-sigmod] n'estudia implementacions amb certes
% heurístiques que aconsegueixen més eficiència.





% \paragraph{Filtratge senzill per mitjana mòbil}

% Sigui $p$ la mida de la finestra mòbil
% \begin{gather*}
%   \text{MitMobil}: S \times \text{p} \longrightarrow S':\\
%   \text{map}(S,(t,v)\mapsto \text{mitjanaV}(S[t,t+p]))
% \end{gather*}


% Mitjana mòbil sobre la multiresolució



% \paragraph{Farciment de forats}

% Jo tinc una sèrie temporal i vull que entre dues mesures no hi hagi més d'un cert temps. Si no es compleix dic que té forats. 

% Sigui $S$ una sèrie temporal, aquesta té forats de més durada que $d$
% si alguna mesura compleix $\text{forats}(S,d) = \text{selecciona}(difT(S),v>d \bigwedge v\neq\infty)$ a on $difT(S) = \text{map}(\text{tpredecessors}(S),(t,v)\mapsto(t,t-v))$.

% Amb la multiresolució el farciment de forats és natural a l'estructura i és controlat per la funció agregadora d'atributs.


% * Com farciria els forats manualment a una sèrie temporal?

% 1. Passar-ho per un esquema de multiresolució

% 2. Treballar sobre la sèrie temporal:

% a partir del càlcul de forats anterior $\text{forats}(S,d)$ per
% exemple apliquem un farciment amb representació
% zohe. $\text{farciment}(S,d) = \text{unio}(S,S')$ a on fem la selecció
% de resolució $S' = S[T]^{\text{zohe}}$, $\forall (t,v) \in
% \text{forats}(S,d): T = \{ \tau = t - dn |
% \tau\in(t-v,t),n\in\mathbb{N} \}$.







% \subsubsection{Com treure profit de les operacions dels SGSTM}

% Temes que després es poden aprofitar a les implementacions

% * No hi ha updates --> les sèries temporals no s'han de canviar

% * Per exemple, vull calcular la mitjana de  BDSTM(a,b] si tinc un disc resolució amb $\delta=b-a$ i $f=$mitjana aquest seria l'adequat en comptes de calcular mitjana(SerieTotal(M)(a,b])

% %??
% % No obstant, la base de dades multiresolució conté informació sobre la
% % resolució de les subsèries i per tant aquesta operació és susceptible
% % d'implementar-se aprofitant aquesta informació.  A tall d'exemple es
% % defineix una operació per extreure de la base de dades multiresolució
% % una sèrie temporal regular amb període $T$:


% % \begin{definition}[Selecció de resolució regular]
% %   \begin{gather*}
% %     \text{ResolucióRegular}: M^* \times T \times r \longrightarrow S'\\
% %     \forall (S_{Bi},S_{Di},\delta_i,\tau_i,k_i,f_i) \in M : \\
% %     d_i = T - \delta_i , \\
% %     0 \geq d_0 > d_1 \dots > d_a, 0 < d_{a+1} < \dots < d_d: \\
% %     S'' = S_{D0} || S_{D1} || \dotsb || S_{Da}  ||  S_{Da+1} || \dotsb || S_{Dd}, \\
% %     S' = S''[i]^r: i = {t|0+nT,n\in\mathbb{N}}
% %   \end{gather*}
% % \end{definition}

% % Nota: les operacions no són equivalents, l'operació $\text{SerieTotal}(M)[i]^r$ és molt més potent que la $\text{ResolucióRegular}(M,T)$.




% \subsubsection{Semàntica de comportament}

% \todo{?}









%%% Local Variables:
%%% TeX-master: "main"
%%% End:
% LocalWords:  SGSTM multiresolució



%------- Annexos ------
%\appendix




\backmatter
\bookmarksetup{startatroot}%  per eliminar les parts
\addtocontents{toc}{\bigskip}% perhaps as well

%------- Bibliografia ------
\cleardoublepage
\phantomsection\addcontentsline{toc}{chapter}{\bibname}
%\pdfbookmark{\bibname}{bookmark:bibliografia}
\printbibliography
%----------------------------------------------


%------- Glossari ------
\cleardoublepage
%\phantomsection\addcontentsline{toc}{chapter}{\glossaryname}
%\pdfbookmark{\glossaryname}{bookmark:glossari}
\chapter{Abreviacions i nomenclatura}

\glsaddall[counter=subsection,types={abreviatura,\acronymtype}]

\printglossary[type=abreviatura,nonumberlist]

\printglossary[type=\acronymtype,title=Sigles,nonumberlist]


\renewcommand{\glossarypreamble}{Índex de notació segons nom, símbol i
  referència --pàgina o en negreta lloc on es defineix el concepte:
  definició, secció (\S) o exemple (ex.). Organitzada en símbols
  matemàtics que requereixen aclariment i notació dels
  components dels \gls{SGST} i dels \gls{SGSTM}.}

\printglossary[type=notation,style=estil-notation]



%------------- Índexs ------------
%ATENCIÓ, hi ha problemes quan s'activen molts índexs
\cleardoublepage\phantomsection\addcontentsline{toc}{chapter}{Índexs}
\chapter*{Índexs}
\begingroup
\let\clearpage\relax
\let\cleardoublepage\relax
\phantomsection\addcontentsline{toc}{section}{\listfigurename}\listoffigures
\phantomsection\addcontentsline{toc}{section}{\listtablename}\listoftables
\phantomsection\addcontentsline{toc}{section}{\lstlistlistingname}\lstlistoflistings %
\renewcommand\listtheoremname{Índex de definicions}
\phantomsection\addcontentsline{toc}{section}{\listtheoremname}\listoftheorems[ignoreall,show={definition}]
\renewcommand\listtheoremname{Índex d'exemples}
\phantomsection\addcontentsline{toc}{section}{\listtheoremname}\listoftheorems[ignoreall,show={example}]
\endgroup
%\listoftodos %%%%mode esborrany
%----------------------------------------------



\end{document}



%%%%%%%%%%%%%%%%%%%%%%%%%%%%%%%%%%%%%%%%%%%%%%%%%%%%%%%%%%%%%%%%%%%%%%%%%%  
% Memòria Tesi Doctoral. 
%
% Copyright (C) 2011-2014 Aleix Llusà Serra.
% 
% This LaTeX document is free software: you can redistribute it and/or
% modify it under the terms of the GNU General Public License as
% published by the Free Software Foundation, either version 3 of the
% License, or (at your option) any later version.
%
% This document is distributed in the hope that it will be useful, but
% WITHOUT ANY WARRANTY; without even the implied warranty of
% MERCHANTABILITY or FITNESS FOR A PARTICULAR PURPOSE. See the GNU
% General Public License for more details.
%
% You should have received a copy of the GNU General Public License
% along with this document. If not, see <http://www.gnu.org/licenses/>.
%
%
% Aleix Llusà Serra
% Departament de Disseny i Programació de Sistemes Electrònics de la Universitat Politècnica de Catalunya (DiPSE-UPC)
% Escola Politècnica Superior d'Enginyeria de Manresa (EPSEM)
% Av. de les Bases de Manresa, 61-73
% 08242 Manresa (Barcelona)
% PAÏSOS CATALANS 
%
% aleix (a) dipse.upc.edu
% 
% El codi font LaTeX del document es troba a 
% <http://escriny.epsem.upc.edu/projects/rrb/>
%%%%%%%%%%%%%%%%%%%%%%%%%%%%%%%%%%%%%%%%%%%%%%%%%%%%%%%%%%%%%%%%%%%%%%%%%% 
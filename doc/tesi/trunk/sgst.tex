\chapter{Model SGST}
\label{cap:model:sgst}
\glsaddchap{not:sgst} %%secció d'operacions
\glsreset{SGST}

En aquest capítol es defineix un model per als \glspl{SGST}. Aquest model
s'estructura en base a dos objectes principals, mesures i sèries temporals.
Ambdós tenen un atribut de temps, el qual requereix un tractament
adequat. El model de \gls{SGST} es dissenya en tres parts.

\begin{itemize}

\item Primer, es defineix el model d'estructura de les dades, és a
  dir, la forma com es descriuen les mesures i les sèries temporals.  

\item Segon, es defineix el model d'operacions sobre les dades, és a
  dir, els operadors bàsics que permeten modelar el comportament i la
  manipulació de les sèries temporals.

\item Tercer, es descriuen propietats de les sèries
  temporals. Les sèries temporals adquireixen propietats
  variades depenent del context on s'apliquin.

\end{itemize}





\section{Model estructural de dades}

L'estructura d'un \gls{SGST} està formada per quatre conceptes
principals: temps, valor, mesura i sèrie temporal. Al final d'aquesta
secció, mostrem alguns exemples de sèries temporals amb valors
concrets.

Una sèrie temporal és una relació de temps i valors. A cada parella
temps-valor l'anomenem mesura. Així doncs, una sèrie temporal és un
conjunt de mesures i una mesura es correspon amb un valor mesurat en
un instant de temps.





\subsection{Temps}
\label{sec:sgst:temps}
\glsaddsec{not:temps} %%%%secció de model

El temps és la variable que permet ordenar les mesures.  Anomenem
\emph{domini del temps} al conjunt \glssymbolsec{not:temps-domini} de
tots els possibles valors de temps. $\glssymbol{not:temps-domini}$ pot
ser tant un conjunt finit com infinit i normalment serà un conjunt
tancat %(compactificat?)
per a poder incloure les mesures indefinides (v.\
\autoref{def:model:mesura_indefinida}) com a límits.  Per tal de
facilitar la comprensió, en aquest document assumim que
$\glssymbol{not:temps-domini}$ és el conjunt estès de nombres reals
$\glssymbol{not:Rb} = \glssymbol{not:R} \cup
\{+\infty,-\infty\}$ \parencite{wiki:extendedreal,cantrell:extendedreal},
també anomenat recta real acabada, el qual és un conjunt tancat.


El conjunt estès de nombres reals té dos punts límits corresponents al
valor impropi infinit, aleshores en notació d'interval el conjunt
$\glssymbol{not:temps-domini}$ es pot escriure com $\glssymbol{not:Rb}
= [-\infty,+\infty]$. En referència amb el conjunt dels nombres reals
$\glssymbol{not:R}$, les relacions d'ordre i algunes operacions
aritmètiques s'estenen al conjunt $\glssymbol{not:Rb}$
\parencite{cantrell:extendedreal}.  Algunes expressions esdevenen
indefinides (p.ex.\ $0/0$) i altres depenen del context, com és el cas
de l'expressió indeterminada $0 \times \infty$ que per exemple en la
teoria de la mesura habitualment es defineix com $0 \times \infty =
0$ \parencite{wiki:extendedreal}.


El conjunt dels reals és un espai mètric ja que té definida una funció
distància (o mètrica), com per exemple la distància euclidiana. Com a
conseqüència, ens permet distingir entre instants de temps (els
elements del conjunt) i durades (la mètrica). Observant els instants
de temps com a punts en la recta real, les durades com a segments de
la recta real i especificant un instant de temps com a marc de
referència, es pot definir el temps com a sistema de
coordenades \parencite{iep:time-supplement,kopetz11:realtime}. A
continuació definim el temps de manera que puguem ordenar
esdeveniments, mesurar durades d'esdeveniments i establir quan
esdevenen; és una aproximació ingènua sense abastar detalls complicats
del concepte temps \parencite{iep:time}.


\begin{definition}[Temps]
  \label{def:model:temps}
  Sigui $\glssymbol{not:temps-domini}=\glssymbol{not:Rb}$ el domini del temps.
  %
  Anomenem un element $t\in\glssymbol{not:Rb}$ com a \emph{instant de temps}.
  L'element $0\in\glssymbol{not:Rb}$ és el \emph{marc de referència}.

  Siguin $s,t\in\glssymbol{not:Rb}$ dos instants de temps.  Definim la
  \emph{durada de temps} entre $s$ i $t$ com el valor $d \in\glssymbol{not:Rb}$
  que mesura la distància en unitats de temps entre tots dos instants
  de temps, és a dir $d= |s-t|$.


\end{definition}

% \begin{definition}[Temps]
%   \label{def:model:temps}
%   Siguin $t^i_i$ i $t^i_j$ dos instants de temps amb el mateix $t^R$
%   com a marc de referència, definim la quantitat de temps o la durada
%   $\glsdispdef{not:durada}{t^d}$ com un valor $t^d \in\glssymbol{not:Rb}$ que
%   mesura la distància en unitats de temps entre dos instants de temps
%   $t^d = d(t^i_i,t^i_j)$ a on $d$ és la mètrica del conjunt $T$. En el
%   cas que els instants de temps es defineixin com a reals, $t^i_i ,
%   t^i_j \in \glssymbol{not:Rb}$, aleshores $t^d = t^i_i - t^i_j$.

%   Sigui $T$ el domini del temps, definim un instant de temps $I$ com
%   un element del conjunt $I \in T$. Així, un instant de temps és
%   l'etiqueta d'un punt en la línia temporal. Seguint la definició de
%   sistema de coordenades i prenent els nombres reals com a domini del
%   temps, sigui $\glsdispdef{not:temps-referencia}{t^{R}}$ un instant de temps
%   marc de referència, aleshores els instants de temps es defineixen
%   com un valor $\glsdispdef{not:instant}{t^i} \in\glssymbol{not:Rb}$
%   que indica la distància de temps amb signe respecte a l'instant de
%   temps de referència $t^i= d(t^{R},I)$ a on $d$ és la mètrica del
%   conjunt $T$.

%   Siguin $t^i_i$ i $t^i_j$ dos instants de temps, la parella d'ambdós
%   $\glsdispdef{not:temps-interval}{[t^i_i,t^i_j]}$ s'anomena interval de
%   temps on $t^i_i \leq t^i_j$.
% \end{definition}

En resum, els instants de temps es poden veure com una seqüència de
valors reals que ordenen els esdeveniments i entre dos instants de
temps es pot determinar una durada.  El marc de referència és
l'instant de temps que correspon a l'\emph{origen} del sistema de
coordenades.  Expressem tant els instants de temps com les durades amb
un real que té unitats de temps. Aquestes unitats són 'segons' en
sistema internacional.



\subsubsection{Estàndards de temps}

Els estàndards de temps especifiquen com s'ha de mesurar el pas del
temps i com s'han d'assenyalar els instants de temps.
\textcite{allen:timescales} recull diferents estàndards de temps
que existeixen, dels quals a continuació comentem els més habituals.

Actualment l'estàndard de temps habitual per mesurar el pas del temps
és el \gls{TAI}, del qual se'n deriva un altre estàndard més conegut
que és el \gls{UTC}.  Ambdós estàndards assenyalen els instants de
temps segons el calendari gregorià i segons el dia julià. Actualment,
de forma genèrica s'utilitza \gls{UTC} per a sincronitzar rellotges,
tot i que en el futur es podria canviar per altres estàndards nous com
per exemple un anomenat Temps Internacional o simplement TI, el qual
també es basaria en el \gls{TAI}.

El dia julià utilitza un estàndard de comptar el temps com a
nombre de dies que han passat des d'una data concreta, la qual
s'anomena època. L'època es correspon amb el concepte d'instant de
temps marc de referència de la \autoref{def:model:temps}. Per defecte
l'època se situa a l'inici del Període Julià tot i que també se solen
utilitzar altres dates assenyalades.

Així, un estàndard semblant al julià és l'Hora POSIX o Hora Unix, el
qual compta el nombre de segons des de l'1 de gener de 1970 basant-se
en les mesures d'\gls{UTC}. L'Hora Unix és l'estàndard de temps
habitual en els sistemes operatius de la família Unix. No obstant
això, aquest estàndard presenta un problema d'ambigüitat a causa que
no té en compte els segons addicionals d'\gls{UTC}.





\subsubsection{Calendari}

Un cas particular del temps és el calendari. Els calendaris són
definicions pel domini temps que consisteixen en noms per als punts de
la línia de temps i regles per establir la durada entre ells per tal
que el temps tingui certa relació amb la rotació de la Terra. A
l'apartat anterior hem definit el domini temps de manera genèrica amb
el conjunt de reals, els quals exemplifiquen  el concepte
de sistema de coordenades de temps absolut sense entrar en detall en
conceptes de calendari.

\textcite{dreyer94} situen els calendaris i les seves operacions com a
essencials en els \gls{SGST}. Tanmateix, pot no ser necessari modelar les
dates i regles de calendari en el model de temps. Els calendaris es
poden observar com a noms que fan referència a instants de temps
quantificables, com els de la \autoref{def:model:temps}. Aleshores,
només cal una eina que sigui capaç de convertir els noms de calendari
a instants de temps.

El fet que un calendari sigui més o menys complicat no afecta al
model de \gls{SGST}, sols té incidència en les funcions de conversió
d'instant de temps a calendari i viceversa. Tampoc afecta que els
calendaris siguin ambigus (p.ex.\ dos noms per al mateix instant o
instants sense nom) o que continguin propietats impredictibles (p.ex.\
cas dels segons addicionals en \gls{UTC}) ja que aquests casos es
corresponen amb la bona definició dels sistemes de calendari.

Així doncs, els calendaris en el model de \gls{SGST} es poden implementar
com una extensió del model de temps. El tipus de dades ordinal de
calendari gregorià implementat per
\textcite[cap.~16]{date02:_tempor_data_relat_model} pot servir com a
guia per a la implementació dels calendaris en els \gls{SGST}.




\subsection{Valor}
\label{sec:sgst:valor}
\glsaddsec{not:valor} %%%%secció de model

El valor és la variable que indica la magnitud de la dada
mesurada. Per tal de no restringir el model a cap àmbit de mesura,
definim genèricament el concepte de valor. Així, el valor és qualsevol
element que és d'un tipus de dades, també anomenat domini i que
simbolitzem amb $\glssymbolsec{not:valor-domini}$; és a dir, un valor
és un objecte que pertany a un determinat conjunt de valors
$\glssymbol{not:valor-domini}$ i que té associades les operacions que
s'hi poden aplicar. Exemples de tipus de dades són els enters, els
reals, les cadenes de text i les estructures de dades com vectors,
llistes o relacions.



El valor quan forma part d'una sèrie temporal pot preveure una dada
que defineixi el valor indefinit.  Aquest valor indefinit és un valor
impropi, és a dir no vàlid, del conjunt de valors possibles. Així cada
tipus de dades pot anar associat amb un valor indefinit corresponent.
Seguint l'exemple amb nombres reals per a la variable temps, en aquest
document assumim que el domini $\glssymbol{not:valor-domini}$ és el
conjunt dels reals estès projectivament $\glssymbol{not:R*} =
\glssymbol{not:R} \cup
\{\infty\}$ \parencite{cantrell:projectivelyextendedreal}.  D'aquesta
manera, com a valor indefinit en aquest document usem el valor infinit
($\infty$), el qual és un valor impropi del conjunt dels reals.  En
altres sistemes, es podrien utilitzar més símbols per a precisar
diferents casos de valors impropis, com per exemple assenyalar casos
de valors no numèrics (NaN, \emph{not a number}).



El valor indefinit s'utilitza per identificar que el valor d'una
mesura és desconegut. Un valor és desconegut quan en el moment de fer
la mesura es desconeix o és erroni. També pot esdevenir desconegut
posteriorment si és marcat com a erroni o descartat després d'un
processament de les dades (v.~\autoref{sec:sgst:patologies}).





\subsection{Mesura}\label{sec:model:mesura} 

Una mesura és un valor mesurat en un determinat instant de temps. Per tant, 
és una parella de temps i valor.

\begin{definition}[Mesura]
  \label{def:model:mesura}
  Sigui $v\in\glssymbol{not:valor-domini}$ un valor i
  $t\in\glssymbol{not:temps-domini}$ un instant de temps. Definim una
  \emph{mesura} $m$ com el tuple
  $\glsdispdef{not:mesura}{m=(t\glsdisp{not:coma}{,}v)}$, en què $v$
  és el valor de la mesura i $t$ és l'instant de temps en que s'ha
  pres aquesta mesura.

  El domini d'una mesura $m$, notat com a
  $\glssymbol{dom}m$, és el domini del seu valor.
\end{definition}


Sigui $m=(t,v)$ una mesura, escrivim
$\glsdispdef{not:mesura-valor}{V(m)}$ per a referir-nos a $v$ i
$\glsdispdef{not:mesura-temps}{T(m)}$ per a referir-nos a $t$.


L'instant de temps indueix la relació d'ordre
entre les mesures. Definim dues relacions d'ordre diferents.

\begin{definition}[Ordre semitemporal]
  Siguin $m$ i $n$ dues mesures. Anomenem \emph{ordre semitemporal} a
  la relació binària $m\glssymboldef{not:mesura:ordreparcial} n$ que
  definim com $m\leq n \iff T(m) < T(n) \vee ( T(m) = T(n)\wedge V(m)
  = V(n))$.
\end{definition}


\begin{definition}[Ordre temporal]
  \label{def:model:mesura-relacio-ordre}
  Siguin $m$ i $n$ dues mesures. Anomenem \emph{ordre temporal} a la
  relació binària $m\glssymboldef{not:mesura:ordretotal} n$ que
  definim com $m\glssymbol{not:mesura:ordretotal} n \iff T(m) \leq
  T(n) $.
\end{definition}


Noteu que l'ordre semitemporal és un ordre parcial mentre que l'ordre
temporal és un ordre total.  Amb aquestes relacions d'ordre queden
definides les operacions de comparació i igualtat entre mesures: $m <
n$ i $m<^t n$, $m=n$ i $m=^t n$, $m > n$ i $m>^t n$, etc.
Precisament, les dues relacions d'ordre només es diferencien en
les operacions d'igualtat.




\subsubsection{Mesures multivaluades}

Les mesures poden contenir alhora més d'un fenomen mesurat quan
aquests comparteixen els instants de temps de mesura; és a dir que hi
ha una co\l.lecció de valors mesurats en el mateix instant de temps.
Aleshores les mesures poden esdevenir tuples n-dimensionals
$(t,v_1,\dotsc,v_n)$, de manera semblant a com ho defineix
\textcite{assfalg08:thesis}.  Anomenem aquestes mesures com a \emph{mesures
multivaluades}.



\subsubsection{Mesures indefinides}



En les definicions de temps i valor s'han estès els conjunts amb
valors impropis, concretament s'ha exemplificat amb el conjunt estès
$\glssymbol{not:Rb} = \glssymbol{not:R} \cup \{+\infty,-\infty\}$ pel
temps i amb el $\glssymbol{not:R*}=\glssymbol{not:R} \cup\{\infty\}$
pel valor. Aquesta extensió amb l'element impropi infinit ($\infty$)
dóna com a resultat unes mesures impròpies que anomenarem mesura de
valor indefinit i mesura indefinida.

\begin{definition}[Mesura de valor indefinit]
  \label{def:model:mesura_valor_indefinit}
  Definim \emph{mesura de valor indefinit} com el tuple
  $\glsdispdef{not:mesura-vindefinit}{(t,v)}$ en què el valor és
  indefinit $v=\infty$ i l'instant de temps és qualsevol
  $t\in\glssymbol{not:Rb}$.
\end{definition}

\begin{definition}[Mesura indefinida]
  \label{def:model:mesura_indefinida}
  Definim \emph{mesura indefinida} com el tuple
  $\glsdispdef{not:mesura-indefinida}{(t,v)}$ en què el valor és
  qualsevol $v\in\glssymbol{not:R*}$ i l'instant de temps és indefinit
  $t\in\{+\infty,-\infty\}$.
\end{definition}

Així doncs, sigui $m$ una mesura de valor indefinit, aquesta pren la
forma $m=(t,\infty)$. Sigui $m$ una mesura indefinida, aquesta pren la
forma $m=(+\infty,v)$ per la positiva i $m=(-\infty,v)$ per la
negativa, les quals normalment anotarem també amb valor indefinit:
$m=(+\infty,\infty)$ i $m=(-\infty,\infty)$ respectivament.








\subsection{Sèrie temporal}
\label{sec:model:serietemporal}

Una sèrie temporal és un conjunt de mesures del mateix fenomen.  Com a
conseqüència, en una sèrie temporal les mesures són homogènies, és a
dir els temps pertanyen al mateix domini i els valors pertanyen al
mateix domini.  Tradicionalment s'anomenen sèries temporals tot i que
alguns autors les anomenen seqüències temporals, per exemple
\textcite{last:hetland}.

\begin{definition}[Sèrie temporal]
  \label{def:serie_temporal}
  Sigui
  $\glsdispdef{not:serietemporal}{S=\{m_0\glsdisp{not:coma}{,}\ldots,m_k\}
    \subset \glssymbol{not:temps-domini} \times
    \glssymbol{not:valor-domini}}$ un conjunt finit de mesures del
  mateix tipus.  Aleshores, $S$ és una \emph{sèrie temporal} si i
  només si no hi ha temps repetits $\forall i,j\glsdisp{not::}{:}
  j\in[0,k] \wedge i \neq j : T(m_i)\neq T(m_j)$.

  Definim el domini d'una sèrie temporal $S$, notat com a
  $\glssymbol{dom}S$, com el domini de les seves mesures.
\end{definition}

% En una sèrie temporal no hi pot haver temps repetits perquè una mateixa mesura no pot haver estat mesurada en el mateix temps i tenir diferent valor (mesurada en el mateix temps, amb el mateix aparell...) seria una contradicció \todo{}

Com que en una sèrie temporal no hi ha temps repetits, no hi ha
discrepància en l'operació d'igualtat i per tant les mesures
contingudes tenen un ordre total. Això no obstant, donades dues sèries
temporals diferents, entre totes les mesures hi pot haver temps
repetits i per tant l'ordre temporal i el semitemporal indueixen a les
dues operacions d'igualtat com ja s'ha comentat.



Per ser un conjunt, les sèries temporals tenen mesura de cardinalitat.
\begin{definition}[Cardinal]
  \label{def:sgst:cardinal}
  Sigui $S$ una sèrie temporal, el cardinal de la sèrie temporal,
  notat com a $\glsdispdef{not:sgst:cardinal}{|S|}$, és el nombre de
  mesures que conté la sèrie temporal.
\end{definition}

Una sèrie temporal sense mesures és la sèrie temporal buida que notem
com a $\emptyset=\{\}$. És a dir que no té cap element i per tant
$|\emptyset|=0$.

 


\subsubsection{Formes d'una sèrie temporal}


Una sèrie temporal s'expressa com un conjunt i com a tal és
susceptible d'aplicar-hi els conceptes del model relacional dels
\glspl{SGBDR} (\seeref{sec:art:sgbd}), a continuació
expressem la forma de sèrie temporal seguint també el concepte de
relació. Diferenciem entre tres formes possibles d'una sèrie temporal:
canònica, multivaluada i doble.  

La forma bàsica d'una sèrie temporal és la de parelles de temps i
valor, l'anomenem forma canònica. 
\begin{definition}[Forma canònica]
  \label{def:sgst:forma-canonica}
  Sigui $\glsdispdef{not:sgst-canonica}{S = \{ m_0, m_1 , \dotsc, m_k
    \}}\subset \glssymbol{not:temps-domini} \times
  \glssymbol{not:valor-domini}$ una sèrie temporal, la forma canònica
  com a relació s'escriu com $S = ( \{t: \glssymbol{not:temps-domini}, v:
  \glssymbol{not:valor-domini} \}, \{ \{ (t,t_0),(v,v_0)\},
  \{(t,t_1),(v,v_1)\}, \dotsc, \{(t,t_k),(v,v_k)\} \} )$; és a dir és
  una parella amb la capçalera i el conjunt de valors certs.

  Així doncs, sigui $\emptyset=\{ \}$ una sèrie temporal buida,
  modelada com a relació s'escriu com $\emptyset = ( \{t:
  \glssymbol{not:temps-domini}, v: \glssymbol{not:valor-domini} \}, \{ \} )$.
\end{definition}


A causa del format esquemàtic de les sèries temporals, en simplifiquem
l'escriptura de la forma canònica com a conjunt de tuples $(t,v)$ en
què $t$ és el temps i $v$ és el valor. Així doncs quan no hi ha dubte
sobre els dominis ni els noms d'atributs, una sèrie temporal es pot
escriure de manera simplificada com a $S = \{ (t_0,v_0), (t_1,v_1),
\dotsc, (t_k,v_k) \}$, la qual es correspon amb la forma de la sèrie
temporal expressada inicialment a la \autoref{def:serie_temporal}.


Tal com s'utilitza en les relacions, les sèries temporals es poden
visualitzar com a taules. La sèrie temporal $S$ i la $\emptyset$
es visualitzen com a taula a la
\autoref{fig:model:serietemporal:taula}.
Per ser un conjunt de mesures, s'observa una sèrie temporal en la
forma canònica com una relació de grau dos on la capçalera conté els
atributs temps i valor. Ambdós atributs tenen els dominis de temps i
valor descrits a les seccions \ref{def:model:temps} i
\ref{sec:sgst:valor}, com per exemple el tipus de dades reals
estesos. Les relacions de sèries temporals inclouen algunes
particularitats:


\begin{itemize}
\item El predicat és similar a: «En el temps \emph{t} s'ha mesurat el
  valor \emph{v}»
\item Els temps no poden ser repetits: és una restricció que indica
  que l'atribut \emph{t} és la clau primària
\item Els valors mesurats han d'estar associats al mateix fenomen o
  fenòmens.
\end{itemize}

\begin{figure}[tp]
  \centering
  \begin{tabular}[c]{|c|c|}
    \multicolumn{2}{c}{$S$} \\ \hline
    $t$  & $v$ \\ \hline
    $t_0$  & $v_0$ \\
    $t_1$  & $v_1$ \\
    $\dots$  & $\dots$ \\ 
    $t_k$  & $v_k$ \\ \hline
  \end{tabular} \qquad
  \begin{tabular}[c]{|c|c|}
    \multicolumn{2}{c}{$\emptyset$} \\ \hline
    $t$  & $v$ \\ \hline
      &  \\ \hline
  \end{tabular}
  \caption{Visualització com a taula d'una sèrie temporal}
  \label{fig:model:serietemporal:taula}
\end{figure}




% Els temps no repetits indueixen un ordre temporal a les sèries
% temporals. Tot i així, les relacions, per ser conjunts, conserven la
% no definció d'un ordre dels elements. En el model relacional no hi ha
% ordre ni en les tuples ni en els atributs a diferència de les
% relacions matemàtiques que tenen un ordre d'esquerra a
% dreta \parencite[sec.\ 5.3]{date:introduction}.









Les sèries temporals poden mesurar alhora més d'un fenomen quan
aquests comparteixen els instants de temps de mesura; és a dir
contenir mesures multivaluades.  Anomenem aquestes sèries temporals
com a sèries temporal multivaluades.
\begin{definition}[Sèrie temporal multivaluada]
  Anomenem sèrie temporal multivaluada a una sèrie temporal que té més
  d'un atribut de valors. Així una sèrie temporal multivaluada té la
  forma simplificada $\glsdispdef{not:sgst-multivaluada}{\{ m_0,
    m_1 , \dotsc, m_k \}}\subset \glssymbol{not:temps-domini} \times
  \glssymbol{not:valor-domini}_1 \times \dotsb \times
  \glssymbol{not:valor-domini}_n$ on cada mesura $m_i$ és un tuple
  $m_i=(t,v_1,\dotsc,v_n)$ on $t$ és un instant de temps i $v_1$,
  \dots, $v_n$ són valors.

  La forma completa com a relació d'una sèrie temporal multivaluada
  buida és $\emptyset = ( \{t: \glssymbol{not:temps-domini}, v_1:
  \glssymbol{not:valor-domini}_1 , \dotsc, v_n:
  \glssymbol{not:valor-domini}_n\}, \{ \} )$
\end{definition}

Com ocorre en les relacions, el nom dels atributs d'una sèrie temporal
pot ser decidit per l'usuari. Per exemple, una sèrie temporal
multivaluada amb tres atributs anomenats és $\emptyset = ( \{t:
\glssymbol{not:Rb}, \text{temperatura}: \glssymbol{not:R*}, \text{consum}:
\glssymbol{not:R*}, \text{volum}: \glssymbol{not:R*}\}, \{ \} )$.



Una sèrie temporal multivaluada es pot escriure en forma canònica de
mesures $(t,v)$. D'aquesta manera, les mesures $m_i$ d'una sèrie
temporal multivaluada en forma canònica són tuples
$m_i=(t,(v_1,v_2,\dotsc,v_n))$.  Així, per al cas de la sèrie temporal
multivaluada buida $\emptyset$, en forma multivaluada canònica és
$\emptyset = ( \{t: \glssymbol{not:temps-domini}, v: V \}, \{ \} )$ on
el domini de l'atribut valor és de tipus relació $V = \{ v_1:
\glssymbol{not:valor-domini}_1 , \dotsc, v_n:
\glssymbol{not:valor-domini}_n\}$ amb restricció que els valors
relació que hi pertanyen només poden tenir un tuple $\forall x \in V:
\glsdisp{not:cardinal}{|x|} = 1$.


La forma canònica s'utilitza per a generalitzar les sèries temporals
multivaluades en les operacions on el valor multivaluat no és
rellevant. En altres operacions, per exemple la selecció o la junció,
el multivalor és rellevant per treballar-hi o per a retornar un
resultat on la sèrie temporal és multivaluada. En les sèries temporals
multivaluades cada atribut pot ser de tipus diferent, tot i que, com
en les sèries temporals canòniques, cada atribut de valor és homogeni.

  



Hi ha una forma no habitual de les sèries temporals que ocorre quan
tenen dos atributs de temps i que anomenem forma doble.

\begin{definition}[Sèrie temporal doble]
  \label{def:sgst:st-doble}
  Anomenem sèrie temporal doble a una sèrie temporal que té dos
  atributs de temps i dos atributs de valors. Sigui
  $\glsdispdef{not:sgst-doble}{\{m_0, \dotsc, m_k\}}$ una sèrie
  temporal és doble si cada mesura $m_i$ és un tuple
  $m_i=(t_1,v_1,t_2,v_2)$ on $t_1$ i $t_2$ són instants de temps i
  $v_1$ i $v_2$ són valors. De la mateixa manera, a aquesta mesura
  $m_i$ l'anomenem mesura doble.  Sigui $S$ una sèrie temporal doble, no té dues
  parelles de temps repetides $|\{(t_1,t_2)| (t_1,v_1,t_2,v_2)\in S\}| = |S|$.
\end{definition}

La sèrie temporal doble s'utilitza com a càlcul intermedi d'altres
operacions com per exemple la junció o el mapatge. En la forma de
relació una sèrie temporal doble buida es pot escriure com $\emptyset
= ( \{t_1: \glssymbol{not:temps-domini}_1,
v_1:\glssymbol{not:valor-domini}_1 , t_2:
\glssymbol{not:temps-domini}_2, v_2: \glssymbol{not:valor-domini}_2,
\{ \} )$.
% La seva forma canònica es
% correspondria de manera semblant a l'exemple de la
% \autoref{fig:model:serietemporal:serietemporal}.









\subsection{Exemples}


\begin{example}[Valors reals]
  Sèrie temporal $S_1$ on el temps i els valors pertanyen a
  $\glssymbol{not:Rb}$. Conté la mesura de valor 1 en el temps 2, la mesura de
  valor 3 en el temps 2 i la mesura de valor 1 en el temps 6.

En la forma canònica completa s'escriu com $S_1 = ( \{t:
\glssymbol{not:Rb}, v: \glssymbol{not:Rb}\}, \{ \{(t,2),(v,1)\}, \{(t,3),(v,3)\},
\{(t,6),(v,1)\} \} )$. També es pot escriure de manera simplificada com a
$S_1 = \{ (2,1), (3,3), (6,1) \}$.


La sèrie temporal $S_1$ es visualitza com a taula a la
\autoref{fig:model:serietemporal:real}, a la qual hi afegim una
visualització com diagrama de dispersió amb el temps a l'eix
horitzontal i el valor a l'eix vertical.

\begin{figure}[tp]
  \centering
  \begin{tabular}[c]{|c|c|}
    \multicolumn{2}{c}{$S_1$} \\ \hline
    $t$  & $v$ \\ \hline
    2  & 1 \\
    3  & 3 \\
    6  & 1 \\ \hline
  \end{tabular} \qquad
  \begin{tikzpicture}[baseline=(current bounding box.center)]
    \begin{axis}[
        timeseriesrel,
        title=$S_1$,
        ]
    \addplot[only marks,mark=*,blue] coordinates {
        (2,1)
        (3,3)
        (6,1)
    };
    \end{axis}
   \end{tikzpicture}
  \caption{Taula i gràfic d'una sèrie temporal amb valors reals}
  \label{fig:model:serietemporal:real}
\end{figure}
 
\end{example}


\begin{example}[Valors caràcters]
  Sèrie temporal $S_2$ on el temps pertany a $\glssymbol{not:Rb}$ i els valors
  són caràcters que pertanyen a $C=\{a,b,\dotsc,z,\infty\}$. Conté el
  caràcters $a$, $c$ i $a$ mesurats respectivament en els temps $2$,
  $3$ i $6$.

De manera simplificada s'escriu com $S_2 = \{ (2,a), (3,c), (6,a) \}$.
La sèrie temporal $S_2$ es visualitza com a taula a la
\autoref{fig:model:serietemporal:caracter}, a la qual hi afegim una
visualització com diagrama de dispersió amb el temps a l'eix
horitzontal i el valor a l'eix vertical no continu.

\begin{figure}[tp]
  \centering
  \begin{tabular}[c]{|c|c|}
    \multicolumn{2}{c}{$S_4$} \\ \hline
    $t$  & $v$ \\ \hline
    2  & a \\
    3  & c \\
    6  & a \\ \hline
  \end{tabular} \qquad
  \begin{tikzpicture}[baseline=(current bounding box.center)]
    \begin{axis}[
        timeseriesrel,
        title=$S_4$,
        yticklabels={0,0,a,b,c},
        ]
    \addplot[only marks,mark=*,blue] coordinates {
        (2,1)
        (3,3)
        (6,1)
    };
    \end{axis}
   \end{tikzpicture}
  \caption{Taula i gràfic d'una sèrie temporal amb valors caràcters}
  \label{fig:model:serietemporal:caracter}
\end{figure}

\end{example}



\begin{example}[Sèrie temporal multivaluada]
  Sèrie temporal $S_3$ on el temps pertany a $\glssymbol{not:Rb}$ i hi ha tres
  valors on cadascun pertany a $\glssymbol{not:Rb}$. En els temps $2$, $3$ i
  $6$ s'ha mesurat: a) un atribut \emph{temp} amb valors $1$, $2$ i
  $1$; b) un atribut \emph{cons} amb valors $2$, $1$ i $2$; i c) un
  atribut \emph{vol} amb valors $3$, $0$ i $3$.



En la forma multivaluada s'escriu com
\begin{align*}
  S_3  &= ( \{t: \glssymbol{not:Rb}, \text{ temp}: \glssymbol{not:Rb}, \text{
  cons}: \glssymbol{not:Rb},\text{ vol}: \glssymbol{not:Rb} \}, \{ \\
  & \{(t,2),(\text{ temp},1) ,( \text{ cons},2),(\text{ vol},3) \}, \\
  & \{(t,3),(\text{ temp},2) ,( \text{ cons},1),(\text{ vol},0) \}, \\
  & \{(t,6),(\text{ temp},1) , (\text{ cons},2),(\text{ vol},3) \} \\
\})
\end{align*}
També es pot escriure de manera simplificada com a $S_{3} = (
(t,\text{ temp},\text{ cons},\text{ vol}),\{ (2,1,2,3), (3,2,1,0),
(6,1,2,3) \})$.

La forma canònica és una sèrie temporal amb tuples $(t,v)$, és a dir
\begin{align*}
  S^C_{3} &= ( \{t: \glssymbol{not:Rb}, v: \{ \text{ temps}:
  \glssymbol{not:Rb}, \text{ cons}: \glssymbol{not:Rb},\text{ vol}:
  \glssymbol{not:Rb},\} \}, \{ \\
  & \{(t,2), (v, ( \{ \text{ temp}: \glssymbol{not:Rb}, \text{ cons}:
  \glssymbol{not:Rb},\text{ vol}: \glssymbol{not:Rb},\}, \{ (\text{ temp},1)
  ,  (\text{ cons},2),(\text{ vol},3) \}\} )), \\
 & \{(t,3), (v, ( \{ \text{ temp}: \glssymbol{not:Rb}, \text{ cons}:
  \glssymbol{not:Rb},\text{ vol}: \glssymbol{not:Rb},\}, \{ (\text{ temp},2)
  ,  (\text{ cons},1),(\text{ vol},0) \}\} )), \\
 & \{(t,6), (v, ( \{ \text{ temp}: \glssymbol{not:Rb}, \text{ cons}:
  \glssymbol{not:Rb},\text{ vol}: \glssymbol{not:Rb},\}, \{ (\text{ temp},1)
  ,  (\text{ cons},2),(\text{ vol},3) \}\} )) \\
  \} )
\end{align*}


La sèrie temporal $S_3$ i la seva forma canònica es visualitzen com a
taula a la \autoref{fig:model:serietemporal:caracter}, a la qual hi
afegim una visualització com diagrama de dispersió amb el temps a
l'eix horitzontal i els valor a l'eix vertical cadascun amb color
diferent.


\begin{figure}[tp]
  \centering
  \begin{tabular}[tp]{|c|c|c|c|}
   \multicolumn{4}{c}{$S_3$} \\ \hline
    $t$  & temp & cons & vol \\ \hline
    2  & 1 & 2 & 3 \\
    3  & 2 & 1 & 0 \\
    6 & 1 & 2 & 3 \\ \hline
  \end{tabular}\qquad
  \begin{tabular}{|c|ccc|}
    \multicolumn{4}{c}{$S_3^c$} \\ \hline
    \multirow{2}{*}{$t$}  & \multicolumn{3}{c|}{$v$} \\ \cline{2-4}
       & temp & cons & vol \\ \hline
    2  & 1 & 2 & 3 \\
    3  & 2 & 1 & 0 \\
    6  & 1 & 2 & 3 \\ \hline
  \end{tabular} 
  \begin{tikzpicture}[baseline=(current bounding box.center)]
    \begin{axis}[
        timeseriesrel,
        title=$S_3$,
        legend columns = 4,
        every axis legend/.append style={
          at={(1,-0.1)},
          anchor=north east,
          draw = none},
        ]
    \addplot[only marks,mark=*,blue] coordinates {
        (2,1)
        (3,2)
        (6,1)
    };
    \addplot[only marks,mark=*,red] coordinates {
        (2,2)
        (3,1)
        (6,2)
    };
    \addplot[only marks,mark=*,green] coordinates {
        (2,3)
        (3,0)
        (6,3)
    };
    \legend{temp,cons,vol}
    \end{axis}
  \end{tikzpicture}
  %El gràfic d'una multivaluada: es poden pintar dos eixos verticals quan les diferències d'escala siguin molt grans.
  \caption{Taula d'una sèrie temporal multivaluada}
  \label{fig:model:serietemporal:multivaluada}
\end{figure}


\end{example}



\begin{example}[Valors vectors]
  Sèrie temporal $S_4$ on el temps pertany a $\glssymbol{not:Rb}$ i el valor
  pertany a $\glssymbol{not:Rb}^3$; és a dir és un vector representat amb un
  tuple. Conté el valor $(1,2,3)$ en el temps $2$, el valor $(3,4,5)$
  en el temps $4$ i el valor $(1,2,3)$ en el temps $6$.

De manera simplificada s'escriu com $S_4 = \{ (2,(1,2,3)),
(3,(3,4,5)), (6,(1,2,3)) \}$ i es visualitza com a taula i com a
gràfic a la \autoref{fig:model:serietemporal:vector}.

\begin{figure}[tp]
  \centering
  \begin{tabular}{|c|c|}
    \multicolumn{2}{c}{$S_4$} \\ \hline
    $t$  & $v$ \\ \hline
    2  & (1,2,3) \\
    4  & (3,4,5) \\
    6  & (1,2,3) \\ \hline
  \end{tabular} \qquad
  \begin{tikzpicture}[baseline=(current bounding box.center)]
\begin{axis}[
        timeseriesrel,
        nodes near coords,
        title=$S_4$,
        yticklabels={},
        axis y line=none,
        ]
        \addplot+[only marks,mark=none,blue, point meta=explicit symbolic]
        coordinates {
          (2,0)  [\rotatebox{45}{(1,2,3)}]
          (4,0)  [\rotatebox{45}{(3,4,5)}]
          (5,1)  [\mbox{}]
          (6,0) [\rotatebox{45}{(1,2,3)}]
        };
      \end{axis}
    \end{tikzpicture}
    \caption{Taula d'una sèrie temporal amb valors vectors}
  \label{fig:model:serietemporal:vector}
\end{figure}



S'observa que una sèrie temporal amb valors vectors és diferent d'una
sèrie temporal multivaluada. El domini de la primera són vectors i el
de la segona són relacions d'un sol tuple en els que es pot operar
cada atribut per separat. En els vectors de forma general no es poden
operar cada component per separat sinó que formen una unitat
semàntica. Aquesta diferència de significat prové de si es considera
que es mesuren vectors o atributs diferents, el qual s'observa en la
visualització: el gràfic d'un vector és un espai $R^n$ en canvi el
gràfic d'una sèrie temporal multivaluada és un multigràfic, un gràfic
per a cada atribut.

\end{example}


\begin{example}[Valors sèrie temporal]\label{par:model:exemple-relvalues}
  Sèrie temporal $S_5$ on el temps pertany a $\glssymbol{not:Rb}$ i el valor
  és una sèrie temporal del mateix format que en l'exemple 1. Conté
  els tuples de $S_1$ com a valors en el temps $1$ i $2$.

De manera simplificada s'escriu com $S_5 = \{ (1,\{ (2,1), (3,3),
(6,1) \}), (2,\{ (2,1),$ $(3,3),$ $(6,1) \}) \}$ i es visualitza com a
taula a la \autoref{fig:model:serietemporal:serietemporal}.


\begin{figure}[tp]
  \centering
  \begin{tabular}{|c|c|}
    \multicolumn{2}{c}{$S_5$} \\ \hline
    $t$  & $v$ \\ \hline
    1 &   
       \begin{tabular}{|c|c|}
         \hline
         $t$  & $v$ \\ \hline
         2  & 1 \\
         3  & 3 \\
         6  & 1 \\ \hline
       \end{tabular} \\ \hline
    2 & 
       \begin{tabular}{|c|c|}
         \hline
         $t$  & $v$ \\ \hline
         2  & 1 \\
         3  & 3 \\
         6  & 1 \\ \hline
       \end{tabular} \\ \hline
  \end{tabular}
  \caption{Taula d'una sèrie temporal amb valors sèrie temporal}
  \label{fig:model:serietemporal:serietemporal}
\end{figure}

S'observa que la capçalera de $S_5$ és $\{t:\glssymbol{not:Rb},v:
\{t:\glssymbol{not:Rb},v:\glssymbol{not:Rb}\}\}$. És a dir, el valor
és una altra relació, com es descriu per \textcite[sec.\
6.4]{date04:introduction8}, on el temps i el valor pertanyen a
$\glssymbol{not:Rb}$. Per tant, el valor de $S_5$ és de tipus sèrie
temporal amb valors reals.  Això no obstant, el significat d'aquestes
sèries temporals és difícil d'interpretar, tot i que són interessants
com a exercici acadèmic de definir sèries temporals de sèries
temporals.

%  Tot i així, aquest és un valor especial i
% si es desplega la relació s'obté una sèrie temporal doble amb
% capçalera $\{t^1,t^2,v\}$.

% Cal insistir que \emph{tots} el valors de $S_5$ han de pertànyer al
% mateix domini \parencite[sec.\ 5.4]{date:introduction}, el qual és
% $\{temps:\glssymbol{not:Rb},valor:\glssymbol{not:Rb}\}$.


\end{example}




















%%% Local Variables:
%%% TeX-master: "main"
%%% End:







% LocalWords:  SGST multivaluada multivaluades

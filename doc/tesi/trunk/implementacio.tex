%\part{Experimentació}


\chapter{Introducció a les implementacions}
\label{sec:implementacions}

En aquest capítol implementem \glspl{SGST} i \glspl{SGSTM} segons els
models definits.



Idealment, en un \gls{SGBD} l'usuari executa una consulta i
l'optimitzador s'encarrega d'executar les operacions físiques més
eficients, en aquest sentit el model relacional permet trobar
expressions equivalents donada una determinada consulta. Això no
obstant, i com s'ha detallat a l'estat actual dels \gls{SGBD}
\todo{ref}, una implementació de \gls{SGBD} no pot abastar i ser
eficient en tots els àmbits i contextos. Així doncs, no és tan senzill
mantenir totalment la independència entre l'usuari i la implementació
ja que cal que aquest decideixi un \gls{SGBD} adequat per a cada
context i fins i tot que declari com resoldre algunes operacions.  Ara
bé, és possible mantenir la independència del nivell lògic respecte a
les implementacions, i és en aquest sentit que a continuació avaluem
diferents implementacions per als models definits de \gls{SGST} i de
\gls{SGSTM}, on cada implementació està pensada per a un context
determinat.


Anomenem de forma diferent cada implementació que dissenyem per tal de
distingir-les clarament. Són les següents:

\begin{itemize}
\item Pytsms i RoundRobinson: Implementacions a alt nivell per a
  observar el funcionament a nivell acadèmic, amb llenguatge
  Python. Aquesta és la nostra implementació de referència per als
  models de \gls{SGST} i \gls{SGSTM}, la qual a més usem per als
  experiments amb dades.

\item RoundRobindoop: Implementació específica per a la resolució en
  diferit i amb computació para\.lela de la multiresolució, amb model
  de programació MapReduce i en el sistema de computació distribuïda
  Hadoop.

\item Reltsms: Implementació a alt nivell en un \gls{SGBDR}, amb
  llenguatge Tutorial~D i en el sistema Rel.

% \item RoundRobinhard: Implementació a baix nivell d'estructures
%   específiques, amb disseny de circuits digitals o amb llenguatge
%   VHDL. \todo{aquesta va en un apèndix?}


%S'avaluen implementacions específiques com RRDtool?

\end{itemize}




\todo{codis?}  Hem d'oferir els codis complets de les implementacions?
Hem de posar un enllaç a l'escriny?


\todo{exemple d'implementacions?} Al final fem un exemple de multiresolucionar una sèrie temporal amb totes les implementacions?

% \todo{resum d'implementacions?} Al final fem un resum comparant les principals característiques de cada implementació?






\section{Particularitats de les implementacions}



Els models d'implementacions pertanyen al nivell físic dels \gls{SGBD}
i són una realització d'un model lògic, en el nostre cas dels models
lògics de \gls{SGST} i de \gls{SGSTM}. Per a les implementacions se
sol definir el nivell d'usuari, que és el llenguatge que serà visible
per als usuaris. Les implementacions que realitzem volen ser molt
properes al model lògic i per tant el nivell d'usuari que se'n deriva
és molt similar. Per a ser un llenguatge d'usuari complet es
requereixen facilitats i capacitats de llenguatge de programació
--bucles, condicionals, declarar variables, etc.-- per a la qual
cosa ens basem en els recursos particulars de cada implementació:
Python, Tutorial~D, etc. L'objectiu principal, però, d'aquestes
implementacions és acadèmic i per tant no considerem prioritari el
llenguatge d'usuari.

% Aquestes facilitats en el nivell lògic no s'expliciten ja que es
% consideren inherents a les matemàtiques.



En la implementació s'afegeixen alguns operadors que en el model
estructural no havíem definit explícitament perquè ja són propis de
l'àlgebra de conjunts. Així per exemple, alguns d'aquests operadors
que s'han d'implementar són els relacionats amb la notació de creació
de conjunts, %set-builder notation (set comprehension)
els quals en els \gls{SGBD} s'inclouen en el que es coneix com a
\gls{DDL}, o els relacionats amb la manipulació de les dades amb
assignació, inserció, modificació o esborrament, els quals en els
\gls{SGBD} es coneixen com a \gls{DML}.  En el cas dels \gls{SGSTM} sí
que hem definit en el model algunes operacions de \gls{DDL} i \gls{DML} per a
l'esquema multiresolució, ja que aquest requereix ser manipulat
coherentment.

% Pel que fa al llenguatge de consultes (\emph{query language}),
% s'implementa seguint el model d'operacions de consulta de cada cas.












%%% Local Variables:
%%% TeX-master: "main"
%%% End:

\chapter{Estat actual} 
\label{cap:estat}

%Base teòrica i avantguarda

En aquest capítol resumim la base teòrica per als models que proposem
i l'avantguarda de l'estat de la qüestió. S'estructura en tres parts:

\begin{itemize}

\item Sèries temporals. L'anàlisi, l'adquisició, el monitoratge,
  l'emmagatzematge i la gestió de sèries temporals.

\item \Glspl{SGBD}. El model relacional i sistemes
  actuals.

\item Sistemes i projectes similars. Altres sistemes que gestionen
  sèries temporals.

\end{itemize}



%Estat de l'art

% * no n'hi ha d'específic del tema, potser el que més s'hi assembla són els SGST que hi ha (Cougar, RRDtool, ...)

% * Hi ha temes colaterals (monitoratge,anàlisis)

% * Temes para\l.lels que ens serveixen d'inspiració (SGBD relacionals)

%Cal introduir bé el forat de coneixement que hi ha en els SGST. Forat entre les sèries temporals i els SGBD.



%Capítol:

% * Sèries temporals
  
%   - mineria
%   - aplicacions
%   - monitoratge de sèries temporals i problemes
%      * censura
%      * mostreig

%   - sgst: 
%       ficar aquí els sgbd per sèries temporals i més endavant ja es parlarà dels sgbd en general i com modelar-los i implementar-los.


% * SGBD
%  - model relacional
%  - implementacions
%  - temporal data


  % * Sèries temporals (històrics, predicció, diagnosis, prognosis, etc.)
  % * Mostreig: docs quan període de mostreig no regular
  % * Bases de dades (docs d'emmagatzematge quan la memòria és finita, docs quan període de mostreig no és regular, altres sistemes semblants (comercials,prototips))










%%% Local Variables: 
%%% mode: latex
%%% TeX-master: "main"
%%% End: 

% LocalWords:  monitoratge

\chapter*{Resum}

\todo{repassar la taula de notació que a vegades fa un salt de pàgina on no toca}

Actualment és possible d'adquirir una gran quantitat de dades,
principalment gràcies a la facilitat de disposar de sistemes de
monitoratge amb grans xarxes de sensors. Això no obstant, no és tan
senzill de gestionar posteriorment totes aquestes dades.  A més, també
cal tenir en compte com s'emmagatzemen aquestes dades.


D'una banda, l'adquisició de valors d'una variable al llarg del temps
es formalitza com a sèrie temporal. Així, hi ha multitud d'algoritmes
i metodologies d'anàlisi de sèries temporals que descriuen com
extreure informació de les dades. D'altra banda, l'emmagatzematge i la
gestió de les dades es formalitza com a \glspl{SGBD}. Així, hi ha
sistemes informàtics dedicats a inferir la informació que un usuari
vol consultar. Aquests sistemes són descrits per models lògics
formals, entre els quals el model relacional n'és la referència
principal.


En aquesta tesi dissertem sobre el fet d'emmagatzemar només aquella
part de les dades originals que conté una certa informació
seleccionada. Aquesta selecció de la informació es duu a terme
mitjançant el resum de diferents resolucions de les dades, cadascuna
de les quals bàsicament són agregacions de les dades a intervals de
temps periòdics. A aquesta tècnica l'anomenem multiresolució.



La multiresolució s'aplica a les sèries temporals. Com a resultat,
s'obtenen subsèries temporals de mida finita i amb la informació
resumida. Per tal de gestionar les sèries temporals, s'utilitzen
\gls{SGBD} específics anomenats \glspl{SGST}. Així doncs, proposem
\gls{SGST} amb capacitats de multiresolució i els anomenem
\glspl{SGSTM}. De la mateixa manera que en els \gls{SGBD}, formalitzem
un model pels \gls{SGST} i pels \gls{SGSTM}.



A causa de la naturalesa de variable capturada al llarg del temps, en
l'adquisició de les sèries temporals apareixen propietats
problemàtiques. Els \gls{SGSTM} tenen en compte algunes d'aquestes
propietats com:
\begin{itemize}
\item La sincronització dels rellotges en els diferents sistemes
  d'adquisició.
\item L'aparició de dades desconegudes perquè no s'han pogut adquirir
  o perquè són errònies.
\item La gestió d'una quantitat enorme de dades, i que a més segueix
  creixent al llarg del temps.
\item Les consultes amb dades que no s'han recollit de manera uniforme
  en el temps.
\end{itemize}


Ara bé, els \gls{SGSTM} són uns sistemes que emmagatzemen unes dades
segons una selecció d'informació i descarten les que no es consideren
importants. Per tant, prèviament a l'emmagatzematge, cal decidir els
paràmetres de selecció de la informació. Per tal d'avaluar la qualitat
d'aquests sistemes, depenent dels paràmetres que s'escullin, es pot
utilitzar la teoria de la informació. En aquest sentit, la
multiresolució es pot considerar com una tècnica de compressió amb
pèrdua. Així doncs, introduïm una reflexió sobre com avaluar l'error
que es comet amb la multiresolució en comparació amb disposar de totes
les dades originals.


Com es diu actualment en l'àmbit dels \gls{SGBD}, un mateix sistema no
pot ser adequat per a tots els contextos. A més, els sistemes han de
tenir en compte un bon rendiment en altres recursos a part del temps
de computació, com per exemple la capacitat finita, el consum
d'energia o la transmissió per la xarxa. Així doncs, dissenyem
diverses implementacions del model dels \gls{SGSTM}. Aquestes
implementacions exploren diverses tècniques de computació: computació
para\l.lela, computació en línia amb el flux de dades i computació relacional \todo{pensar bé quin és el nom tècnic}.


En resum, en aquesta tesi dissenyem els \gls{SGSTM} i en formalitzem
un model.  Els \gls{SGSTM} són útils per a emmagatzemar sèries
temporals en sistemes amb capacitat finita i per a precomputar la
multiresolució. D'aquesta manera, permeten disposar de consultes i
visualitzacions immediates de les sèries temporals de forma
resumida. Això no obstant, impliquen una selecció de la informació que
cal decidir prèviament. En aquesta tesi proposem consideracions i
reflexions sobre els límits de la multiresolució.





\chapter*{Abstract}






\glsreset{SGBD}
\glsreset{SGST}
\glsreset{SGSTM}


%%% Local Variables: 
%%% mode: latex
%%% TeX-master: "main"
%%% End: 

%  LocalWords:  multiresolució

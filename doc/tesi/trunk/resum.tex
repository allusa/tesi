\chapter*{Resum}

Actualment és possible d'adquirir una gran quantitat de dades,
principalment gràcies a la facilitat de disposar de sistemes de
monitoratge amb grans xarxes de sensors. Això no obstant, no és tan
senzill de gestionar posteriorment totes aquestes dades.  A més, també
cal tenir en compte com s'emmagatzemen aquestes dades.


D'una banda, l'adquisició de valors d'una variable al llarg del temps
es formalitza com a sèrie temporal. Així, hi ha multitud d'algoritmes
i metodologies d'anàlisi de sèries temporals que descriuen com
extreure informació de les dades. D'altra banda, l'emmagatzematge i la
gestió de les dades es formalitza com a \glspl{SGBD}. Així, hi ha
sistemes informàtics dedicats a inferir la informació que un usuari
vol consultar. Aquests sistemes són descrits per models lògics
formals, entre els quals el model relacional n'és la referència
principal.


En aquesta tesi dissertem sobre el fet d'emmagatzemar només aquella
part de les dades originals que conté una certa informació
seleccionada. Aquesta selecció de la informació es duu a terme
mitjançant el resum de diferents resolucions de les dades, cadascuna
de les quals bàsicament són agregacions de les dades a intervals de
temps periòdics. A aquesta tècnica l'anomenem \emph{multiresolució}.



La multiresolució s'aplica a les sèries temporals. Com a resultat,
s'obtenen subsèries temporals de mida finita i amb la informació
resumida. Per tal de gestionar les sèries temporals, s'utilitzen
\gls{SGBD} específics anomenats \glspl{SGST}. Així doncs, proposem
\gls{SGST} amb capacitats de multiresolució i els anomenem
\glspl{SGSTM}. De la mateixa manera que en els \gls{SGBD}, formalitzem
un model pels \gls{SGST} i pels \gls{SGSTM}.



A causa de la naturalesa de variable capturada al llarg del temps, en
l'adquisició de les sèries temporals apareixen propietats
problemàtiques. Els \gls{SGSTM} tenen en compte algunes d'aquestes
propietats com:
\begin{itemize}
\item La sincronització dels rellotges en els diferents sistemes
  d'adquisició.
\item L'aparició de dades desconegudes perquè no s'han pogut adquirir
  o perquè són errònies.
\item La gestió d'una quantitat enorme de dades, i que a més segueix
  creixent al llarg del temps.
\item Les consultes amb dades que no s'han recollit de manera uniforme
  en el temps.
\end{itemize}


Ara bé, els \gls{SGSTM} són uns sistemes que emmagatzemen unes dades
segons una selecció d'informació i descarten les que no es consideren
importants. Per tant, prèviament a l'emmagatzematge, cal decidir els
paràmetres de selecció de la informació. Per tal d'avaluar la qualitat
d'aquests sistemes, depenent dels paràmetres que s'escullin, es pot
utilitzar la teoria de la informació. En aquest sentit, la
multiresolució es pot considerar com una tècnica de compressió amb
pèrdua. Així doncs, introduïm una reflexió sobre com avaluar l'error
que es comet amb la multiresolució en comparació amb disposar de totes
les dades originals.




Com es diu actualment en l'àmbit dels \gls{SGBD}, un mateix sistema no
pot ser adequat per a tots els contextos. A més, els sistemes han de
tenir en compte un bon rendiment en altres recursos a part del temps
de computació, com per exemple la capacitat finita, el consum
d'energia o la transmissió per la xarxa. Així doncs, dissenyem
diverses implementacions del model dels \gls{SGSTM}. Aquestes
implementacions exploren diverses tècniques de computació: computació
incremental seguint el flux de dades, computació para\l.lela i
computació de bases de dades relacional.


En resum, en aquesta tesi dissenyem els \gls{SGSTM} i en formalitzem
un model.  Els \gls{SGSTM} són útils per a emmagatzemar sèries
temporals en sistemes amb capacitat finita i per a precomputar la
multiresolució. D'aquesta manera, permeten disposar de consultes i
visualitzacions immediates de les sèries temporals de forma
resumida. Això no obstant, impliquen una selecció de la informació que
cal decidir prèviament. En aquesta tesi proposem consideracions i
reflexions sobre els límits de la multiresolució.





\chapter*{Abstract}

Nowadays, it is possible to acquire a huge amount of data, mainly due
to the fact that it is easy to build monitoring systems together with
big sensor networks. However, data has to be managed accordingly,
which is not so trivial. Furthermore. the storage for all this data
also has to be considered.


On the one hand, time series is the formalisation for the process of
acquiring values from a variable along time. There is a great deal of
algorithms and methodologies for analysing time series which describe
how information can be extracted from data. On the other hand,
\gls{DBMS} are the formalisation for the systems that store and
manage data. That is, these computer systems are devoted to infer the
information that a given user may query.  These systems are formally
defined by logic models from which the relational model is the main
reference.

This thesis is a dissertation on the hypothesis to store only parts of
original data which contain selected information.  This information
selection involves summarising data with different resolutions, mainly
by aggregating data at periodic time intervals.
We name \emph{multiresolution} to this technique.


Multiresolution is operated on time series. The results are time
subseries that have bounded size and summaries of information.
Particular \gls{DBMS} are used for managing time series, then they are
called \gls{TSMS}. In this context, we define \gls{TSMS} with
multiresolution capabilities, which we call \gls{MTSMS}.  Similarly
to how it is done for \gls{DBMS}, we formalise a model for
\gls{TSMS} and for \gls{MTSMS}.



The acquisition of time series presents troublesome properties owing
to the fact of variable acquired along time.
In \gls{MTSMS} we consider some of this properties such as:
\begin{itemize}
\item The clock synchronisation for different acquisition systems. 
\item Unknown data when data has not been acquired or when it is
  erroneous.
\item A huge amount of data to be manage. Moreover it increases as
  more data gets acquired.
\item Queries with data that has not been acquired regularly along
  time.
\end{itemize}


\gls{MTSMS} are defined as systems to store data by selecting
information and so by discarding data that is not considered
important. Therefore, the parameters for selecting information must be
decided previously to storing data. The information theory is the base
for measuring the quality of theses systems, which depends on the
parameters chosen. Regarding this, multiresolution can be considered
as a lossy compression technique. We introduce some hypothesis on
measuring the error caused by multiresolution in comparison with the
case of having all the original data.


Paraphrasing a current opinion in \gls{DBMS}, the same system can not
be adequate for all the different contexts. In addition, systems must
consider performance in a variety of resources apart from computing
time, such as energy consumption, storage capacity or network
transmission. Concerning this, we design different implementations for
the model of \gls{MTSMS}. These implementations experiment with
various computing methodologies: incremental computing along the data
stream, parallel computing and relational databases computing.


Summarising, in this thesis we formalise a model for \gls{MTSMS}.
\gls{MTSMS} are useful for storing time series in bounded capacity
systems and in order to precompute the multiresolution. In this way
they can achieve immediate queries and graphical visualisations for
summarised time series.  However, they imply an information selection
that has to be considered previously to the storage. In this thesis we
consider the limits for the multiresolution technique.



\glsreset{SGBD}
\glsreset{SGST}
\glsreset{SGSTM}

\glsreset{DBMS}
\glsreset{TSMS}
\glsreset{MTSMS}

%%% Local Variables: 
%%% mode: latex
%%% TeX-master: "main"
%%% End: 

%  LocalWords:  multiresolució

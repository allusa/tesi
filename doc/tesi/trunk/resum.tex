\chapter*{Resum}


Les xarxes de sensors capturen dades de l'entorn, les quals s'han
d'emmagatzemar en bases de dades per a poder-les tractar
posteriorment. Hi ha models que descriuen com han de ser aquestes
bases de dades per a sèries temporals i esquemes que solucionen alguns
dels seus problemes. 


Una sèrie temporal és un conjunt de parelles de temps i valor que
provenen de l'evolució d'una variable al llarg del temps. 

A causa d'aquesta naturalesa de variable capturada al llarg del temps,
en l'adquisició i tractament de les sèries temporals apareixen
propietats problemàtiques que anomenem patologies.
Algunes d'aquestes patologies són:
\begin{itemize}
\item La sincronització dels rellotges en els diferents sistemes
  d'adquisició.
\item L'aparició de dades desconegudes perquè no s'han pogut adquirir
  o perquè són errònies.
\item La gestió d'una quantitat enorme de dades i que a més segueix
  creixent al llarg del temps.
\item L'operació amb dades que no s'han recollit de manera uniforme en
  el temps.
\end{itemize}


Els sistemes informàtics que saben emmagatzemar i tractar les sèries
temporals s'anomenen sistemes de gestió de bases de dades per a sèries
temporals (SGST). Els SGST han de saber gestionar les patologies de
les sèries temporals. 

Una solució per a aquestes patologies es pot aconseguir afegint
esquemes de multiresolució per a les sèries temporals. Aleshores
s'obtenen SGST específics anomenats SGST multiresolució (SGSTM).  La
multiresolució és un sistema d'emmagatzematge que selecciona la
informació prèviament a ser guardada i en descarta la que no es
considera important.




Un SGSTM és una solució d'emmagatzematge per a sèries temporals a on,
resumint, la informació es distribueix mitjançant diferents
resolucions temporals.  Una sèrie temporal amb multiresolució és una
co\l.lecció de subsèries resolució, les quals acumulen temporalment
mesures en un buffer on són processades i finalment emmagatzemades
en un disc. El processament de les dades té per objectiu canviar els
intervals de temps entre les mesures per tal de compactar la
informació de les sèries temporals. D'aquesta manera, les sèries
temporals queden emmagatzemades en diferents resolucions temporals
distribuïdes en els discs.  L'arquitectura d'aquests sistemes es pot
veure a la figura~\ref{fig:vhdl:bdstm}.

Els discs tenen la mida limitada i només poden contenir un nombre
fixat de mesures. Quan un disc no té més capacitat ha d'eliminar una
mesura. Com a conseqüència en un SGSTM la mida és fixada i les sèries
temporals hi queden emmagatzemades a trossos; és a dir com a subsèries
temporals.





* Ja no només importa el temps de computació, també tenir en compte altres recursos limitats --capacitat, transmissió per la xarxa, etc. Sobretot en xarxes de sensor i sistemes integrats petits


* Com es diu actualment en el món dels SGBD, un sistema no pot anar bé per tots els casos.(Com que com diu Stonebraker, one size does not fit all), dissenyem
diverses implementacions del nostre model de SGBD per explorar diverses naturaleses d'implementacions. Explorem altres
tècniques de computació: computació para\l.lela, computació de fluxos
de dades.




\chapter*{Abstract}







%%% Local Variables: 
%%% mode: latex
%%% TeX-master: "main"
%%% End: 

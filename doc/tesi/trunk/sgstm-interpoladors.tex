
\section{Funcions d'agregació d'atributs}
\label{sec:model:interpolador}
\label{sec:model:agregador}
\glsaddsection{not:sgstm:fdef} %%%%secció de model
\glsaddsection{not:sgstm:f} %%%%secció de model


Les funcions d'agregació d'atributs s'utilitzen en la consolidació
dels buffers per tal de compactar certa informació de la sèrie
temporals. Sigui $S$ una sèrie temporal i $t_a$ i $t_b$ dos instants
de temps, una funció d'agregació d'atributs $f$ calcula una mesura que
resumeix la informació de $S$ en un interval de temps $i=[t_a,t_b]$:
\[
f: \text{sèrie temporal} \glsdisp{not:times}{\times}
\text{interval de temps} \longrightarrow \text{mesura}
\]
\[
f: S=\{m_0,\dotsc,m_k\} \times i=[t_a,t_b] \longrightarrow  m'
\]


Generalment, $m'$ resulta d'aplicar dues operacions a $S$: 
\begin{enumerate}
\item una selecció d'una subsèrie $S'$ segons l'interval de temps $i$,
  per exemple $S' = S[t_a,t_b]$
\item i una agregació en aquesta subsèrie $m' =
  \glssymbol{not:sgst:aggregate}(S',m_i,\glssymbol{not:sgst:fagg})$ on
  $\glssymbol{not:sgst:fagg}$ i $m_i$ són els atributs d'aquesta agregació.
\end{enumerate}



Atès que hi ha maneres diferents de resumir la informació d'una sèrie
temporal, cal plantejar diferents funcions d'agregació d'atributs. Per
exemple, es poden calcular estadístics de la sèrie temporal, com el
valor màxim o la mitjana; aplicar operacions de processament digital
del senyal, com fan \textcite{zhang11}, o algoritmes per a detectar
comportaments aberrants, com fa \textcite{lisa00:brutlag}. A més a
més, la representació de les sèries temporals
(v.~\autoref{sec:model:repr}) pot afectar els càlculs que es fan en
l'agregació o bé es pot aprofitar l'agregació per a tractar algunes de
les patologies de les sèries temporals
(v.~\autoref{sec:sgst:patologies}).  Així doncs, es poden definir una
enorme varietat de funcions d'agregació d'atributs i no hi ha cap
assumpció global que es pugui fer, cada usuari ha d'interpretar quina
combinació d'agregació i representació s'adiu més amb el fenomen
mesurat. Com a conseqüència, els \gls{SGSTM} han de donar llibertat
als usuaris per a definir funcions d'agregació d'atributs
personalitzades.


Com a mostra de com dissenyar funcions d'agregació d'atributs, a
continuació descrivim algunes interpretacions possibles que se'n poden
fer, tant pel que fa al càlcul de l'instant de temps resultant de la
consolidació com pel que fa al càlcul amb representació de sèries
temporals, i descrivim com utilitzar-les per a tractar i validar dades
desconegudes en les sèries temporals.



\subsection{Interpretació de l'agregació}


L'agregació d'una sèrie temporal en un interval resulta en una mesura
$m'=(t',v')$. Així per a definir les operacions d'agregació cal
interpretar quin ha de ser el temps resultant $t'=T(m')$ i el valor
resultant $v'=V(m')$.


Podem definir patrons generals de funcions d'agregació d'atributs que
indiquin quina informació o estadístic es resumeix de la sèrie
temporal, és a dir patrons generals que indiquin com s'ha de calcular
el valor resultant $V(m')$ independentment del mètode de representació
que es vulgui associar a la sèrie temporal.  Tot i així, el temps
resultant $T(m')$ no queda definit sinó que s'ha interpretar
coherentment per a cada cas particular de representació.


A continuació mostrem alguns exemples de patrons generals per a
calcular el valor resultant $V(m')$ que resumeix atributs d'una sèrie
temporal $S$ en un interval $i=[t_a,t_B]$. Sigui $S^r(t)$ la funció de
representació de la sèrie temporal i $t\in T$ els instants de temps
en el domini de temps:
\begin{itemize}
\item màxim: $S \times i \mapsto m'$ on $V(m') = \max_{\forall t \in
    [t_a,t_b]}(S^r(t))$. Resumeix $S$ amb el màxim dels valors de les
  mesures a l'interval $i$.
\item darrer: $S \times i \mapsto m'$ on $V(m') = S^r(t_b)$. Resumeix
  $S$ amb el valor del darrer instant de temps de l'interval $i$.

\item mitjana: $S \times i \mapsto m'$ on $V(m') = \frac{1}{t_b-t_a}
  \int_{t_a}^{t_b} S^r(t)dt$. Resumeix $S$ amb la \emph{mitjana de la
    funció} a l'interval $i$. \emph{Nota:} La mitjana d'una
  funció \parencite{weisstein:averagefunction}, $\bar f=f(x^*)$,
  utilitza la propietat $\int_a^b f(x)dx = f(x^*)(b-a)$ quan $f$ és
  contínua a $[a,b]$. \label{sec:sgstm:mitjanafuncio}
  % Explicació:
  % If $f$ is continuous on a closed interval $[a,b]$, then there is at least one number $x^*$ in $[a,b]$ such that
  % $$
  % \int_a^b f(x)dx = f(x^*)(b-a)
  % $$

  % The average value of the function ($\bar f$)  on this interval is then given by  $f(x^*)$.
  % $S(t)$ ha de ser contínua en l'interval $i$.
\end{itemize}




En aquests patrons d'atributs es treballa sobre una funció $S^r(t)$,
que a cada cas serà una funció de representació concreta i el temps
resultant $T(m')$ serà interpretat coherentment.  A més, per a cada
representació concreta també cal interpretar amb matemàtica discreta
el càlcul del valor resultant $V(m')$, atès que aquests patrons estan
definits com a problemes d'anàlisi numèric però a cada cas $S^r(t)$ és
una funció que prové d'un conjunt de mesures i podem expressar els
operadors segons el model de \gls{SGST} descrit amb àlgebra discreta
matemàtica.  A continuació s'exemplifiquen algunes interpretacions
possibles per al càlcul de $T(m')$ i de $V(m')$.





\subsubsection{Temps d'agregació resultant}


L'objectiu de les funcions d'agregació d'atributs és determinar un
instant de temps $T(m')$ i un valor $V(m')$. Aquest càlcul del temps i
del valor es pot realitzar al mateix temps però també pot ser
independent. Així, en principi el temps resultant serà independent i
valdrà $T(m')=t_b$ per estar d'acord amb l'operació de consolidació
del buffer i no causar desfasament de la subsèrie resolució
(v.~\autoref{def:sgstm:desdsamentR}), però en alguns casos aquest
$T(m')$ serà dependent del valor calculat o estarà subjecte a una
interpretació adient com és el cas per les representacions a l'apartat
següent.


Un exemple de funció d'agregació on temps i valor són dependents és
una que retorni la primera mesura que troba, $\operatorname{primera}:
S \times i \mapsto m'$ on $m' = \min(S[t_a,t_b))$ i llavors el temps
resultant pot ser $t_a \leq T(m') < t_b$. En aquest cas la sèrie
temporal consolidada resultant no és regular.


Un exemple de funció d'agregació on temps i valor són independents i
on la subsèrie resolució resultant és regular però amb desfasament, és
una funció que fa la mitjana amb un desfasament de 5 unitats de temps.
La funció d'agregació $\operatorname{mitjanad5}$ s'ha utilitzat
anteriorment a l'\autoref{ex:model:bdm-desfasaments}, ara podem
definir-la contextualitzada en les funcions d'agregació d'atributs,
$\operatorname{mitjanad5}: S \times i \mapsto m'$ on $V(m')=
\glssymbol{not:sgst:mitjanav}(S[t_a-5,t_b-5))$ i $T(m')=t_b-5$.

%De què pot servir la mitjanad5? per calcular mitjanes centrades? estem fent una interpolació sobre la representació centrada en l'interval de la sèrie temporal?


%mitjana mòbil, MM
%moving average, MA




\subsubsection{Agregació amb representació}

La varietat de funcions de representació per les sèries temporals
indueix a una varietat de funcions d'agregació per a un mateix patró
d'atributs. Per exemple, la funció d'agregació per l'atribut de màxim
dóna com a resultat valors diferents si es considera una representació
lineal o una representació a trossos constant. A continuació mostrem
la interpretació dels patrons definits anteriorment per a tres mètodes de
representació: \gls{pd}, \gls{dd} i \gls{zohe}.


\paragraph{Parcial discreta.}
En els casos parcials, $S^r(t)$ no és totalment contínua en el temps,
però es pot resoldre l'agregació del valor resultant assumint que el
domini de temps $T$ es correspon als instants de temps que hi ha a la
sèrie temporal, és a dir $T=\glssymbol{not:sgst:project}_{t}(S)$.  El
temps resultant es pot interpretar segon descrit a l'apartat anterior,
per exemple $T(m')=t_b$, i a més també es pot interpretar l'interval
de temps d'agregació $i=[t_a,t_b]$. Així sigui $S$ la sèrie original,
el resultat es pot calcular sobre una subsèrie amb interval obert
$S'=S(t_a,t_b)$, tancat $S'=S(t_a,t_b]$, semiobert $S'=S(t_a,t_b]$ o
$S'=S[t_a,t_b)$, o altres combinacions com per exemple tenir
desfasaments $S'=S[t_a-d,t_b-d]$ on $d$ és una durada.  Així de forma
general podem definir les funcions d'agregació d'atributs amb
representació \gls{pd}, $f^{\gls{pd}}\in f$, com $f^{\gls{pd}}: S
\times [t_a,t_b] \mapsto m'$ on $m'=(t_b,v')$ i el valor resultant
depèn del l'atribut que es vulgui resumir calculat en l'interval
$S'=S[t_a,t_b]$, a continuació es mostren els patrons d'exemple
interpretats segons aquest criteri.

\begin{definition}[Agregació parcial discreta]
  Sigui $S=\{m_0,\dotsc,m_k\}$ una sèrie temporal, $i=[t_a,t_b]$ un
  interval de temps i $S'=S[t_a,t_b]$ un interval de la sèrie
  temporal, les funcions d'agregació \gls{pd} per als atributs màxim,
  darrer i mitjana són:
  \begin{itemize}

  \item $\glssymboldef{not:sgstm:maxpd}: S \times i \mapsto
    m'$ on $V(m') = \max_{\forall m \in S'}(V(m))$ i
    $T(m')=t_b$. Aquest càlcul de $V(m')$ es correspon amb l'operació
    $\glssymbol{not:sgst:maxv}(S')$ dels \gls{SGST}.  \label{def:sgstm:maxpd}

\item $\operatorname{darrer}^{\gls{pd}}$: $S \times i \mapsto m'$ on $V(m') =
  V(\max(S'))$ i $T(m')=t_b$.

\item $\glssymboldef{not:sgstm:mitjanapd}: S \times i \mapsto m'$ on $V(m') =
  \frac{1}{|S'|} \sum\limits_{\forall m\in S'} V(m)$ i $T(m')=t_b$. Aquest càlcul de
  $V(m')$ es correspon amb l'operació $\glssymbol{not:sgst:mitjanav}(S')$
  dels \gls{SGST}, és a dir amb calcular la mitjana aritmètica dels
  valors de les mesures. \label{def:sgstm:mitjanapd}
\end{itemize}

\end{definition}



\paragraph{Delta de Dirac.} 
Per a les funcions d'agregació delta de Dirac interpretem el temps
d'agregació resultant centrat en l'interval $T(m')=\frac{t_b+t_a}{2}$,
tot i que també es podrien considerar altres interpretacions com per
exemple $T(m')=t_b$. Així de forma general podem definir les funcions
d'agregació d'atributs amb representació \gls{dd}, $f^{\gls{dd}}\in
f$, com $f^{\gls{dd}}: S \times [t_a,t_b] \mapsto m'$ on
$m'=(\frac{t_b+t_a}{2},v')$ i el valor resultant depèn del l'atribut
que es vulgui resumir calculat en l'interval temporal \gls{dd}
$S'=S[t_a,t_b]^{\gls{dd}}$.


\begin{definition}[Agregació delta de Dirac]
  \label{def:sgstm:maxdd}
  Sigui $S=\{m_0,\dotsc,m_k\}$ una sèrie temporal, $i=[t_a,t_b]$ un
  interval de temps i $S'=S[t_a,t_b]^{\gls{dd}}$ un interval temporal
  de la sèrie temporal, les funcions d'agregació \gls{dd} per als
  atributs màxim, darrer i mitjana són:
\begin{itemize}
\item \glssymboldef{not:sgstm:maxdd}: $S
  \times i \mapsto m'$ on $V(m') = \max\big(0,\max_{\forall m \in
    S'}(V(m))\big)$ i $T(m')=\frac{t_b+t_a}{2}$. 

\item $\operatorname{darrer}^{\gls{dd}}$: $S \times i \mapsto m'$ on $V(m') =
  V(\max(S'))$ i $T(m')=\frac{t_b+t_a}{2}$.

\item \glssymboldef{not:sgstm:mitjanadd}: $S \times i \mapsto m'$ on
  $V(m') = \frac{1}{t_b-t_a}\sum\limits_{\forall m \in S'} V(m)$ i
  $T(m')=\frac{t_b+t_a}{2}$. Nota: la funció delta de Dirac té la
  propietat fonamental $\int \delta(t)dt = 1$. 
\end{itemize}
\end{definition}



\paragraph{Zero-order hold enrere.}
Per a les funcions d'agregació \gls{zohe} interpretem sempre el temps
d'agregació resultant com $T(m')=t_b$, atès que la representació
\gls{zohe} es defineix amb funcions graó contínues per
l'esquerra. Així de forma general podem definir les funcions
d'agregació d'atributs amb representació \gls{zohe},
$f^{\gls{zohe}}\in f$, com $f^{\gls{zohe}}: S \times [t_a,t_b] \mapsto
m'$ on $m'=(t_b,v')$ i el valor resultant depèn de l'atribut que es
vulgui resumir calculat en l'interval temporal \gls{zohe}
$S'=S[t_a,t_b]^{\gls{zohe}}$.
\begin{definition}[Agregació zero-order hold enrere]
  \label{def:sgstm:agregadorszohe}
  Sigui $S=\{m_0,\dotsc,m_k\}$ una sèrie temporal, $i=[t_a,t_b]$ un
  interval de temps i $S'=S[t_a,t_b]^{\gls{zohe}}$ un interval
  temporal de la sèrie temporal, les funcions d'agregació \gls{zohe}
  per als atributs màxim, darrer i mitjana són:
  \begin{itemize}
  \item \glssymboldef{not:sgstm:maxzohe}: $S \times i \mapsto m'$ on
    $V(m') = \max_{\forall m \in S'}(V(m))$ i $T(m')=t_b$.

  \item $\operatorname{darrer}^{\gls{zohe}}$: $S \times i \mapsto m'$
    on $V(m') = V(\max(S'))$ i $T(m')=t_b$.

  \item \glssymboldef{not:sgstm:meanzohe}: $S \times i \mapsto m'$ on
    $V(m') = \frac{1}{t_b-t_a} \big[ (T(o)-t_a)V(o) +
    \sum\limits_{\forall m \in S''}( T(m)-
    T(\glssymbol{not:sgst:prev}_S (m)) )V(m) \big]$; $o=\min(S')$;
    $S''= S' - \{o\}$; i $T(m')=t_b$.  \label{def:sgstm:meanzohe}
% \[
%   \begin{split}
%   V(m')  = & \frac{1}{t_b-T_0} 
%   \big[ (T(o)-T_0)V(o) -( T(n)-T_f)V(n) \\
%     & {}+\sum\limits_{\forall m \in S''}( T(m)- T(\prev_S m) )V(m) \big]   
%    \end{split}
%   \]
% Nota: s'aplica la definició $0 \times \infty = 0$ tal com es fa habitualment a la teoria de mesura, \cite{wiki:extendedreal}.
  \end{itemize}
\end{definition}




Un cop definits els tres exemples de famílies d'agregacions, podem
comparar-les en funció de com resumeixen la informació de la sèrie
temporal. Reprenent la consolidació dels buffers
(v.~\autoref{sec:model:buffer}), l'interval de consolidació es
correspon a $t_a=\tau$ i $t_b=\tau+\delta$ i és consolidable quan
existeix una mesura $T(m)\geq\tau+\delta$. A la
\autoref{fig:sgstm:agg} dibuixem les mesures d'una sèrie temporal en
vermell, un interval de consolidació del buffer en línies blaves i la
mesura resultant de consolidació en verd.  Així, sigui
$S=\{\dotsc,m_{a-1},m_{a+1},\dotsc,m_{b-1},m_{b+1}, \ldots\}$ una
sèrie temporal on $ T(m_{a-1}) < t_a < T(m_{a+1}) < \dotsc <
T(m_{b-1}) < t_b < T(m_{b+1})$ i la consolidació del buffer que
calcula la mesura resultant $m'=f(S,[t_a,t_b])$ amb la funció
d'agregació d'atributs $f$.  Assumim $T(m')=t_b$ per simplificar el
dibuix, de manera general el càlcul del valor resultant és una
agregació a partir de les mesures:
\begin{itemize}
\item $\{m_{a+1},\dotsc,m_{b-1}\}$ en el cas de les agregacions \gls{pd}
\item $\{(t_a,0),(\ldots,0),m_{a+1},\dotsc,(\ldots,0),\dotsc,m_{b-1},(\ldots,0),(t_b,0)\}$ en el cas de les
  agregacions \gls{dd}
\item $\{m_{a+1},\dotsc,m_{b-1},m_{b+1}\}$ en el cas de les
  agregacions \gls{zohe}
\end{itemize}






\begin{figure}[tp]
  \centering
 
    \begin{tikzpicture}
        \begin{axis}[
          % width=10cm,
%          scale only axis, height=3cm,
          ymin = 0,
          xmax = 50,
          xmin = 20,
          yticklabels= {},
          xticklabels={,,,$t_a$,,$t_b$},
          ]
          \addplot[ycomb,blue] coordinates {
            (30,10)
            (40,10)
          }; 
          
          \addplot[only marks,mark=*,red] coordinates {
            (25,5)
            (32,2)
            (35,4)
            (38,6)
            (45,8)
          };
          
          \addplot[only marks,mark=*,green] coordinates {
            (40,4)
          };
          
          \node[above] at (axis cs:26,5) {$m_{a-1}$};
          \node[below] at (axis cs:32,2) {$m_{a+1}$};
          \node[below] at (axis cs:35,4) {$\ldots$};
          \node[above] at (axis cs:38,6) {$m_{b-1}$};
          \node[above] at (axis cs:45,8) {$m_{b+1}$};
          \node[right] at (axis cs:40,4) {$m'$};
        \end{axis}
      \end{tikzpicture}

    
  \caption{Agregació d'un interval de la sèrie temporal}
  \label{fig:sgstm:agg}
\end{figure}






En conclusió, per una banda alguns exemples mostrats de patrons tenen
una interpretació semblant per a les representacions particulars, en
certa manera només es diferencien en la interpretació de l'interval on
s'ha de resumir la sèrie temporal. Per exemple la diferència principal
en els atributs de màxim i darrer per a les tres representacions rau
en la $S'$, tot i que en el cas del $\glssymbol{not:sgstm:maxdd}$
l'agregació a més ha de tenir en compte que en la funció de
representació hi ha valors intermitjos que valen zero.

Per altra banda, altres exemples són molt diferents, com és el cas de
l'atribut mitjana. En aquest cas, per a la \gls{pd} i la \gls{dd} és
el càlcul de la suma dels valors tot i que dividit per $|S'|$ en la
primera i per $t_b-t_a$ en la segona, i és una mitjana ponderada per
les durades de temps en la \gls{zohe}.  En general, es pot dissenyar
qualsevol operació d'agregació, com per exemple calcular la mitjana
aritmètica de l'interval \gls{zohe} amb
$\glssymbol{not:sgst:mitjanav}(S[t_a,t_b]^{\gls{zohe}})$, tot i que
llavors cal interpretar quin patró d'atribut li correspon o altrament
aquesta operació d'agregació pot no tenir sentit real.


\textcite{rrdtool} utilitza a RRDtool una funció d'agregació semblant
a la $\glssymboldef{not:sgstm:meanzohe}$ per a resumir la informació
conservant el comptatge total si les sèries temporals mesurades tenen
trets semàntics de comptador i són en forma de velocitat; així aquesta
agregació es pot veure com una consolidació que conserva l'àrea del
senyal original. 





% Notes:

% * Quan una sèrie temporal és regular, l'intepolador mitjana aritmètica i l'interpolador àrea valen el mateix en l'interval $(T_o,n\delta]$.




\subsection{Tractament i validació de dades}


En les patologies de les sèries temporals
(v.~\autoref{sec:sgst:patologies}) s'ha descrit el problema de les
dades desconegudes, les funcions d'agregació d'atributs poden cooperar
en els processos de validació i tractament de dades. Així, les
funcions d'agregació poden marcar o tractar dades desconegudes:
\begin{itemize}
\item Marcar dades com a desconegudes. És a dir determinar quan el
  resultat d'una agregació ha de ser desconegut perquè la sèrie
  temporal avaluada pateix una de les causes descrites: valors fora de
  rang, temps de termini excedit, etc.

\item Tractar dades que són desconegudes, ja sigui perquè d'origen són
  desconegudes o perquè les hem marcat abans com a desconegudes.
  Si una funció d'agregació rep valors que són desconeguts, des d'un
  punt de vista estricte el resultat de l'agregació ha de ser
  desconegut. No obstant això, es poden aplicar operacions que tractin
  aquest valors desconeguts: reconstrucció del senyal, ignorar els
  valors desconeguts, etc.
\end{itemize}

 
A continuació definim el procés que fan les funcions d'agregació per a
ambdós casos. Com a exemple de domini pels valors utilitzem els
nombres reals projectius \glssymbol{not:R*}, en els quals representem
el valor desconegut mitjançant l'element infinit ($\infty$), segons la
\autoref{def:model:mesura_valor_indefinit} de mesura de valor
indefinit. Això no obstant, el domini de valors podria tenir diversos
valors per a marcar diferents casos de dades desconegudes.

\paragraph{Tractament de dades desconegudes.}
Una funció d'agregació d'atributs $f^u \in f$ que tracti dades
desconegudes és aquella que pot calcular un resultat quan la sèrie
temporal original conté valors desconeguts
\[
f^u: S \times i \mapsto m' \text{ on } \exists m \in S: V(m)=\infty
\]

Com ja hem comentat, treballar amb valors desconeguts estrictament
hauria de resultar en valors desconeguts. Això no obstant, les dades
desconegudes es poden tractar mitjançant tècniques de reconstrucció,
d'interpolació, d'aproximació, etc. L'usuari, però, s'ha d'assegurar i
estudiar en cada context que la tècnica que apliqui per a tractar
dades desconegudes sigui vàlida. Altrament, només podrà considerar el
resultat com a desconegut.


Per exemple, podem redefinir el patró de la funció d'agregació mitjana
en una $\operatorname{mitjana}^{u}$ que sigui capaç de tractar valors
desconeguts conservant l'àrea coneguda, és a dir, l'àrea total
coneguda quedarà escampada en l'interval de consolidació.
\begin{gather*}
  \operatorname{mitjana}^{cu}: S \times i \mapsto m' \text{ on }\\
  V(m') = \frac{1}{t_b-t_a}\int_{t_a}^{t_b} S^u(t)dt \text{ i }
  S^u(t)=
  \begin{cases}
    0 &\text{si }  S^r(t)=\infty\\
    S^r(t) & \text{altrament }
  \end{cases}
\end{gather*}


\paragraph{Marcatge de dades desconegudes.}
Una funció d'agregació d'atributs $f^{mu} \in f$ que marqui
dades desconegudes és aquella que pot retornar una mesura de valor
indefinit com a resultat
\[
f^{mu}: S \times i \mapsto m' \text{ on } V(m')\in \glssymbol{not:R*}
\]


Per exemple, podem definir un patró de funció d'agregació d'atribut
màxim que retorni valor desconegut
si hi ha un a mesura amb el valor més gran que 2; és a dir establim un
límit superior de 2 (L2). 
\begin{gather*}
  \operatorname{m\grave{a}xim}^{L2}: S \times i \mapsto m' \text{ on }\\
  m' = \begin{cases}
    (T(m''),\infty) &\text{si }  \exists m\in S[t_a,t_b]: V(m)>2\\
    m'' & \text{altrament }
  \end{cases} \text{ i } m''= \operatorname{m\grave{a}xim}(S,i)
\end{gather*}






%\todo{}
%hauria d'aparèixer algun exemple on es resolgués inframostreig. Potser també algun exemple on es veiés on els agregadors solucionen el problema de l'ultramostreig.



%Per exemple definim un termini, si les dades estan més espaiades que 2 es marca com a desconeguda
% Sigui $S=\{m_0,\ldots,m_k\}$ una sèrie temporal i $H$ un termini de temps, una mesura $m_i=(v_i,t_i)\in S$ és desconeguda si, donada la mesura anterior $m_{i-1}=(v_{i-1},t_{i-1})$, $t_i - t_{i-1} > H$.    





% Sigui $S=\{m_0,\ldots,m_k\}$ una sèrie temporal, $f$ un interpolador, $i=[T_0,T_f]$ un interval de temps i $\alpha$ un llindar, la mesura de consolidació calculada per l'interpolador $f$ és desconeguda ssi  
% \[
% \frac{t_d }{T_f - T_0} > \alpha :
% \]
% \[
% :t_d = t_{d0} + t_{df} + \sum\limits_{i=1}^{k-1}(t_i-t_{i-1}) : v_k = 'desconegut':
% \]
% \[
% : t_{d0} = \left\{\begin{array}{l} t_0-T_0 \text{ si } v_0 = 'desconegut' \\ 0\end{array}\right. ,
% t_{df} = \left\{\begin{array}{l} T_f-t_{k-1} \text{ si } v_k = 'desconegut' \\ 0\end{array}\right. :
% \]
% \[
% :k=|S|-1,(v_k,t_k)=m_k\in S' :S'= S_{T_0:T_f} \cup \{min(S_{T_f:\infty})\}
% \]



%operacions amb nan de octave i matlab
%http://biosig-consulting.com/matlab/NaN/
% The NaN-toolbox v2.0: A statistics and machine learning toolbox for Octave and Matlab
% for data with and w/o MISSING VALUES encoded as NaN's.











%%% Local Variables:
%%% TeX-master: "main"
%%% End:
% LocalWords: buffer buffers ZOHE






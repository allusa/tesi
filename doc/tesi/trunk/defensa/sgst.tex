\subsection[SGST]{Sistemes de gestió de sèries temporals (SGST)}




\begin{frame}{SGST - Estructura}
\begin{columns}

  \column{5cm}

  Conceptes:
  \begin{itemize}

  \item Instant de temps
  \item Valor
  \item Mesura
  \item Sèrie temporal

  \end{itemize}

  Propietats:
  \begin{itemize}

  \item Sense temps repetits
  \item Ordre de les mesures: el temps indueix l'ordre, però pot ser
    semitemporal o temporal.

  \end{itemize} 

  
  \column{5cm}
  Exemple:
  \begin{center}
    $S=\{(3,1), (4,3), (6,2), (9,1)\}$\\[1em]

    \begin{tabular}[h]{|c|c|}\hline
      $t$ & $v$ \\\hline
      3 & 1 \\
      4 & 3 \\
      6 & 2 \\
      9 & 1 \\\hline
    \end{tabular}

  \end{center}

\end{columns}
\end{frame}




\begin{frame}{SGST - Operacions}

  Tres grups:
  \begin{itemize}
    
  \item Conjunt: pertinença, inclusió, màxim, suprem, unió,
    diferència, intersecció, diferència simètrica, selecció,
    projecció, reanomena, producte, junció i computacionals.
  \item Seqüència: interval, successió i concatenació.
  \item Funció temporal: interval temporal, selecció temporal,
    concatenació temporal i junció temporal.

  \end{itemize}


\end{frame}


\begin{frame}{SGST: unió i concatenació}

\begin{center}

Unió, dues possiblilitats:

\def\escala{0.9}

\def\nodeA{node [above left=0.5cm and 0.1cm] {$S_1$}}
\def\nodeB{node [above right=0.5cm and 0.1cm] {$S_2$}}
\def\nodeT{}
% Definition of circles
\def\firstcircle{(0,0) circle (1.5cm)}
\def\secondcircle{(0:2cm) circle (1.5cm)}
\def\thirdcircle{(0:1cm) circle (1.11cm)}

\colorlet{circle edge}{blue!50}
\colorlet{circle area}{blue!20}

\tikzset{
  filled/.style={fill=circle area, draw=circle edge, thick},
  outline/.style={draw=circle edge, thick},
  every node/.style={transform shape}
}

%\setlength{\parskip}{5mm}



%Set A or B
\begin{tikzpicture}[scale=\escala]
  \draw[filled] \firstcircle \nodeA;
    \begin{scope}
        \clip \secondcircle;
        \draw[filled, even odd rule] \firstcircle \nodeA
                                 \secondcircle 
                                 \thirdcircle;
   \end{scope}
    \draw[outline] \firstcircle
                   \secondcircle \nodeB
                   \thirdcircle \nodeT;

   %\node[anchor=south] at (current bounding box.north) {$S_1 \cup S_2$};
\end{tikzpicture}
%Set temporal A or B
\begin{tikzpicture}[scale=\escala]
    \draw[filled, even odd rule] \firstcircle \nodeA
                                 \secondcircle \nodeB
                                 \thirdcircle \nodeT;
    %\node[anchor=south] at (current bounding box.north) {$S_1 \cup^t S_2$};
\end{tikzpicture}



Concatenació: unió per seqüències que exclou l'interval de la segona
  \begin{tikzpicture}[baseline=(current bounding box.center)]
    \begin{axis}[
        timeseriesrel,
        width=0.5\textwidth,
        title=,
        xmin=0,
        xmax=11,
        ymin=0,
        xtickmin=0,
        xtickmax=0,
        ytickmin=0,
        ytickmax=0,
        try min ticks=6,
        ]
    \addplot[mark=*,blue] coordinates { 
        (3,2) 
        (4,3)
        (6,2)
        (9,1.5)
    };
    \addlegendentry{$S_1$};

    \addplot[mark=*,orange] coordinates { 
        (-1,1) 
        (2,0.5)
        (3,1)
    };
    \addlegendentry{$S_2$};

    \addplot[mark=*,orange] coordinates { 
        (9,0.75) 
        (11,1)
        (13,0.5)
    };

    \addplot[mark=*,lightgray] coordinates { 
        (3,1) 
        (6,0.5)
        (9,0.75)
    };


    \end{axis}
   \end{tikzpicture}


\end{center}


\end{frame}



\begin{frame}{SGST - Graf i funció temporal de representació}

  Funció temporal de representació $S(t): \text{Temps} \longrightarrow \text{Valors}$

  Graf: $\{\dotsc,(3,1),(4,3),(5,2),\dotsc\}$\\\medskip


  Diversos mètodes:

  \begin{columns}

  \column{3cm}
  \begin{tikzpicture}[baseline=(current bounding box.center)]
    \begin{axis}[
        timeseriesrel,
        title=$S^\text{dd}$,
        xmin=0,
        xmax=11,
        xtickmin=0,
        xtickmax=10,
        try min ticks=6,
        ]
    \addplot[mark=*,blue,ycomb] coordinates { 
        (3,1) 
        (4,3)
        (6,2)
        (9,1)
    };

    \addplot[mark=o,blue,only marks] coordinates { 
        (3,0)
        (4,0)
        (6,0)
        (9,0)
    };


    \pgfplotsextra{%
      \pgfpathmoveto{\pgfplotspointaxisxy{0.5}{0}}%
      \pgfpathlineto{\pgfplotspointaxisxy{12}{0}}%
      \pgfsetarrowsstart{latex}
      \pgfsetarrowsend{latex}
      \pgfsetcolor{blue}
      \pgfusepath{stroke}%
    }


    \end{axis}
   \end{tikzpicture}


  \column{3cm}
  \begin{tikzpicture}[baseline=(current bounding box.center)]
    \begin{axis}[
        timeseriesrel,
        title=$S^{\text{zohe}}$,
        xmin=0,
        xmax=11,
        xtickmin=0,
        xtickmax=10,
        try min ticks=6,
        ]
    \addplot[mark=*,blue,const plot mark right] coordinates { 
        (3,1) 
        (4,3)
        (6,2)
        (9,1)
    };

    \addplot[mark=o,blue,only marks] coordinates { 
        (3,3)
        (4,2)
        (6,1)
        (9,0)
    };

    \pgfplotsextra{%
      \pgfpathmoveto{\pgfplotspointaxisxy{9}{1}}%
      \pgfpathlineto{\pgfplotspointaxisxy{9}{0}}%
      \pgfsetcolor{blue}
      \pgfusepath{stroke}%
    }

    \pgfplotsextra{%
      \pgfpathmoveto{\pgfplotspointaxisxy{9}{0}}%
      \pgfpathlineto{\pgfplotspointaxisxy{12}{0}}%
      \pgfsetarrowsend{latex}
      \pgfsetcolor{blue}
      \pgfusepath{stroke}%
    }

    \pgfplotsextra{%
      \pgfpathmoveto{\pgfplotspointaxisxy{3}{1}}%
      \pgfpathlineto{\pgfplotspointaxisxy{0.5}{1}}%
      \pgfsetarrowsend{latex}
      \pgfsetcolor{blue}
      \pgfusepath{stroke}%
    }

    \end{axis}
   \end{tikzpicture}


  \column{3cm}
  \begin{tikzpicture}[baseline=(current bounding box.center)]
    \begin{axis}[
        timeseriesrel,
        title=$S^\text{foh}$,
        xmin=0,
        xmax=11,
        ymin=0,
        xtickmin=0,
        xtickmax=10,
        try min ticks=6,
        ]
    \addplot[mark=*,blue] coordinates { 
        (3,1) 
        (4,3)
        (6,2)
        (9,1)
    };

    \addplot[lightgray] coordinates { 
        (3,1) 
        (4,0)
    };
     \addplot[lightgray] coordinates { 
        (3,0)
        (4,3)
        (6,0)
    };
    \addplot[lightgray] coordinates { 
        (4,0)
        (6,2)
        (9,0)
    };
    \addplot[lightgray] coordinates { 
        (6,0)
        (9,1)
    };



    \pgfplotsextra{%
      \pgfpathmoveto{\pgfplotspointaxisxy{9}{1}}%
      \pgfpathlineto{\pgfplotspointaxisxy{12}{1}}%
      \pgfsetarrowsend{latex}
      \pgfsetcolor{blue}
      \pgfusepath{stroke}%
    }

    \pgfplotsextra{%
      \pgfpathmoveto{\pgfplotspointaxisxy{3}{1}}%
      \pgfpathlineto{\pgfplotspointaxisxy{0.5}{1}}%
      \pgfsetarrowsend{latex}
      \pgfsetcolor{blue}
      \pgfusepath{stroke}%
    }

    \end{axis}
   \end{tikzpicture}

 \end{columns}
 
\end{frame}




\begin{frame}{SGST: Junció i junció temporal }

\begin{center}

    \begin{tabular}[h]{|c|c|}
    \multicolumn{2}{c}{$S_1$} \\ \hline
      $t$ & $v$ \\\hline
      1 & 1 \\
      2 & 1 \\\hline
    \end{tabular}
    \begin{tabular}[h]{|c|c|}
    \multicolumn{2}{c}{$S_2$} \\ \hline
      $t$ & $v$ \\\hline
      1 & 2 \\
      2 & 2 \\\hline
    \end{tabular}$\quad\longrightarrow\quad$
    \begin{tabular}[h]{|c|c|c|}
    \multicolumn{3}{c}{Junció} \\ \hline
      $t$ & $v_1$ & $v_2$ \\\hline
      1 & 1 & 2  \\
      2 & 1 & 2\\\hline
    \end{tabular}

\bigskip

    \begin{tabular}[h]{|c|c|}
    \multicolumn{2}{c}{$S_1$} \\ \hline
      $t$ & $v$ \\\hline
      1 & 1 \\
      2 & 1 \\\hline
    \end{tabular}
    \begin{tabular}[h]{|c|c|}
    \multicolumn{2}{c}{$S_2$} \\ \hline
      $t$ & $v$ \\\hline
      1 & 2 \\
      3 & 2 \\\hline
    \end{tabular}$\quad\longrightarrow\quad$
    \begin{tabular}[h]{|c|c|c|}
    \multicolumn{3}{c}{Junció temporal} \\ \hline
      $t$ & $v_1$ & $v_2$ \\\hline
      1 & 1 & 2  \\
      2 & 1 & ?  \\
      3 & ? & 2\\\hline
    \end{tabular}


  \end{center}
  
Cal utilitzar mètode de representació per a calcular ?, o bé deixar-lo indefinit.

\end{frame}


\begin{frame}{SGST: Patologies}


  \begin{itemize}

    \item Sincronització de rellotges \parencite{kopetz11:realtime}
    \item Dades desconegudes: Validació, rebuig, reconstrucció\dots
    \item Quantitat enorme de dades
    \item Regularitat de les sèries temporals
  \end{itemize}
  

\end{frame}



%%% Local Variables: 
%%% mode: latex
%%% TeX-master: "defensa"
%%% End: 

\begin{frame}{Conclusions dels models}

  \begin{itemize}
  \item Hem formalitzat el model dels SGSTM, conjuntament amb el model dels SGST
  \item La formalització és sòlida en l’àlgebra de conjunts i l’àlgebra relacional

\item Els SGST inclouen comportament de conjunt, seqüència i funció temporal
\item Els SGST són extensibles mitjançant els mètodes de representació
\item Els SGSTM compacten les sèries temporals en diverses resolucions, amb més pes a les més recents
\item Cal configurar la quantitat de resolucions i quatre paràmetres per a cada una: pas de consolidació, instant inicial, capacitat i funció d’agregació d’atributs
\item Els SGSTM són extensibles mitjançant les funcions d’agregació d’atributs i per tant s’adapten a diferents contexts
\item Les funcions d’agregació d’atributs inclouen la validació i la regularització de les sèries temporals

  \end{itemize}


\end{frame}

\begin{frame}{Conclusions de les implementacions}
  \begin{itemize}
  \item 
\item La multiresolució encaixa amb el problema de grans volums de dades (big data)
\item Les dades es comprimeixen directament com a subsèries temporals: visualitzacions immediates i consultes precomputades
\item En les implementacions incrementals el còmput es reparteix durant l’adquisició contínua
\item Es tenen en compte altres recursos: capacitat fitada, emmagatzematge distribuït en els nodes de captura, distribució de les dades agregades en xarxa
\item Les implementacions paral·leles possibiliten l’experimentació ràpida
\item Els sistemes duals permeten mantenir les dades originals quan són crítiques
  \end{itemize}
\end{frame}

\begin{frame}{Conclusions de la qualitat de la multiresolució}

  \begin{itemize}
  \item Els SGSTM comprimeixen les dades amb pèrdua
  \item Els SGSTM resolen consultes aproximades
  \item Hem definit una manera d’avaluar l’error que comet la multiresolució
  \item Hem analitzat alguns escenaris particulars
  \item L’avaluació de la qualitat de forma genèrica és difícil
  \item Cal obrir noves línies de recerca i contextualitzar-ho en la teoria de la informació
  \end{itemize}
\end{frame}

\begin{frame}{Treball futur}

  \begin{itemize}
  \item Fer implementacions productives, definir nivell d’usuari (llenguatges de bases de dades
  \end{itemize}

\end{frame}



%%% Local Variables: 
%%% mode: latex
%%% TeX-master: "defensa"
%%% End: 
\subsection{Objectius i característiques}
\begin{frame}{Objectius principals}

  \begin{itemize}
  \item Sèries temporals: multitud d'algoritmes d'anàlisi
  \item Sistemes de gestió de bases de dades (SGBD): gestió i tractament de dades
  \end{itemize} 

\begin{block}{Multiresolució}
  Compressió de certa informació de la sèrie temporal
    \begin{itemize}
    \item diverses resolucions (períodes regulars)
    \item diverses agregacions d'atributs 
    \end{itemize}
\end{block}


\end{frame}


\begin{frame}{Característiques de la multiresolució}

  \begin{itemize}
  \item Formalització: àlgebra de conjunts, àlgebra relacional
  \item Compressió amb pèrdua, consultes aproximades
  \item Mida emmagatzemada fitada
  \item Regularització de les sèries temporals
  \item Computació incremental
  \item Precomputació de consultes i visualitzacions immediates
  \item Aplicacions: sistemes encastats, xarxes de sensors, big data\dots
  \end{itemize}

\end{frame}


\begin{frame}{Assoliments de la tesi}

  \begin{itemize}
  \item Formalització del model dels sistemes multiresolució sòlidament en l’àlgebra de conjunts i particularment en l’àlgebra relacional.
  \item Funcions d’agregació d’atributs genèriques i independents i funcions de representació per considerar el comportament en diversos contexts.
  \item Solució de compressió amb pèrdua i mida fitada. Apta per a sitemes petits
  \end{itemize}

\end{frame}



\subsection{Estat de la qüestió}
% \begin{frame}{Introducció: estat SGBD}

%   \begin{itemize}
%   \item Model relacional: teoria de conjunts (àlgebra relacional) i lògica de predicats (càlcul relacional)
%   \item Sistemes SQL
%   \item Sistemes NoSQL
%   \item Sistemes NewSQL
%   \item The Third Manifesto 
%   \item \emph{One size does not fit all} \parencite{stonebraker09}
%   \end{itemize}

% \end{frame}


% \begin{frame}{Introducció: estat sistemes sèries temporals}
%   \begin{itemize}

%   \item Emmagatzematge massiu:
%     \begin{itemize}
%      \item Sistemes per matrius: SciDB i SciQL
%      \item Sistemes d'emmagatzematge distribuït: OpenTSDB
%     \end{itemize}
%   \item Compressió de dades:
%     \begin{itemize}
%     \item amb pèrdua: RRDtool
%     \item sense pèrdua: Tsdb
%     \item representació aproximada òptima: iSAX
%     \item  emmagatzematges en memòries flash  %Dou   
%     \end{itemize}

%   \item Flux de dades (\emph{data stream}): 
%     \begin{itemize}
%     \item Xarxes de sensors: Cougar, TinyDB
%     \item Algoritmes de més pes a les dades recents \parencite{cormode08:pods}
%     \end{itemize}


% \end{itemize}


% \emph{Nota:} SGBD per dades bitemporals diferents als de sèries temporals \parencite{schmidt95}


% \end{frame}

\begin{frame}{Estat actual i motivació}
  \begin{itemize}
  \item  SGBD per a dades bitemporals (històrics) i per a sèries temporals no són el mateix
  \item Multitud de tècniques d’anlàsis de sèries temporals, representacions aproximades òptimes: cal poder-les integrar en els SGBD %(candidats a funcions d’agregació)
  \item Dades voluminoses: per què no una tècnica de compressió amb pèrdua?
  \item Més pes a les dades més recents:  RRDtool%, cormode, flash storages (en àmbits particulars, oferim més genericitat i ho integrem en un SGBD)
  \item Emmagatzematge massiu:  però després com analitzem les dades?
  \item Computació paral·lela però també cal pensar en sistemes amb recursos restringits (temps, processament, capacitat, energia)
  \item Computació incremental per a fluxos de dades (\emph{data stream})
  \item Irregularitat de sèries temporal
  \end{itemize}

\end{frame}


\subsection{Preliminars}
\begin{frame}{Preliminars als models}
\begin{columns}

  \column{9cm}

  \begin{itemize}
  \item Arquitectura 3 nivells: independència entre nivells
  \item Model abstracte sense semàntica
  \item Nucli de la formalització: estructura, operacions i integritat
  \item Model relacional: teoria de conjunts (àlgebra relacional) i lògica de predicats (càlcul relacional)
  \item The Third Manifesto \parencite{date:thethirdmanifesto} 
  \item \emph{One size does not fit all} \parencite{stonebraker09}
  \end{itemize}

    \column{2cm}
\begin{center}
Nivells:\\[2ex]

\begin{tikzpicture}[scale=0.6, every node/.style={transform shape}]

      \tikzset{
        mynode/.style={rectangle,rounded corners,draw=black, 
          very thick, inner sep=1em, minimum size=3em, text centered,
          groc},
        myarrow/.style={->, shorten >=1pt, thick},
        mylabel/.style={text width=7em, text centered},
        groc/.style={top color=white, bottom color=yellow!50},
        verd/.style={top color=white, bottom color=green!50},
        roig/.style={top color=white, bottom color=red!50},
      }  


 \node[mynode] (u) {\parbox{2.2cm}{Usuari \\ (Llenguatge)}};
 \node[mynode, below=0.5cm of u] (l) {\parbox{2.2cm}{Lògic (Model abstracte)}};
 \node[mynode, below=0.5cm of l] (i) {\parbox{2.2cm}{Físic (Implementacions)}};

\end{tikzpicture}
\end{center}

\end{columns}
\end{frame}



%%% Local Variables: 
%%% mode: latex
%%% TeX-master: "defensa"
%%% End: 
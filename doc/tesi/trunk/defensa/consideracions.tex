
\begin{frame}{Casos d'ús}

  \begin{enumerate}
  \item Funció de multiresolució
  \item Sistemes duals
  \item Avaluació de la qualitat de la multiresolució
  \end{enumerate}

\end{frame}


\subsection*{Variacions}

\begin{frame}{SGSTM vs.\ funció de multiresolució}


Model sistema de gestió de bases de dades (conserva estat):


\begin{center}
\begin{tikzpicture}[scale=0.6, every node/.style={transform shape}]

      \tikzset{
        mynode/.style={rectangle,rounded corners,draw=black, 
          very thick, inner sep=1em, minimum size=3em, text centered,
          groc},
        myarrow/.style={->, shorten >=1pt, thick},
        mylabel/.style={text width=7em, text centered},
        groc/.style={top color=white, bottom color=yellow!50},
        verd/.style={top color=white, bottom color=green!50},
        roig/.style={top color=white, bottom color=red!50},
      }  






 \node[mynode] (S) {Sèrie temporal};
 \node[mynode, verd, right=2cm of S] (sgstm) {SGSTM};
 \node[mynode, above right=-0.5cm and 2cm of sgstm] (Sp) {Nova sèrie temporal};
 \node[mynode, below=1.5cm of Sp.west, anchor=west] (S1) {subsèries};

 \draw[->] (S) -- (sgstm) node[above,midway,sloped] {afegeix} node[below,midway,sloped] {mesura}; 
 \draw[->] (sgstm) -- (Sp) node[above,midway,sloped] {total}; 
 \draw[->] (sgstm) -- (S1) node[above,midway,sloped] {disc}; 
 \node[left=2.1cm of Sp.north] (consultes) {consultes};

\end{tikzpicture}
\end{center}




Model d'operació de multiresolució (sobre un SGST):


\begin{center}
\begin{tikzpicture}[scale=0.6, every node/.style={transform shape}]

      \tikzset{
        mynode/.style={rectangle,rounded corners,draw=black, 
          very thick, inner sep=1em, minimum size=3em, text centered,
          groc},
        myarrow/.style={->, shorten >=1pt, thick},
        mylabel/.style={text width=7em, text centered},
        groc/.style={top color=white, bottom color=yellow!50},
        verd/.style={top color=white, bottom color=green!50},
        roig/.style={top color=white, bottom color=red!50},
      }  






 \node[mynode] (S) {Sèrie temporal};
 \node[mynode, right=4cm of S] (Sp) {Nova sèrie temporal};
 \draw[->] (S) -- (Sp) node[above,midway] {multiresolució}; 

 \node[mynode, below right=0.5cm and 1cm of S] (S1) {subsèries};
 \draw[->] (S) -- (S1) node[above,midway,sloped] {mapa}; 
 \draw[->] (S1) -- (Sp) node[above,midway,sloped] {plec}; 


\end{tikzpicture}
\end{center}


\end{frame}




% \begin{frame}{Usos de la multiresolució}

% Possibilitats de computació:

%   \begin{itemize}
%   \item Computació offline
%   \item Computació incremental
%   \item Computació para\l.lela (offline)
%   \item Computació distribuïda (xarxes de sensors)
%   \end{itemize}


% Arquitectures dels sistemes:
%   \begin{itemize}

%   \item Únicament SGSTM: emmagatzematge amb pèrdua, aproximació a l'històric, dades originals no són necessàries.
%   \item SGST+SGSTM: dipòsit a llarg termini poc consultat i sistema per a resoldre consultes habituals o visualitzacions immediates
%   \item Funció de multiresolució (para\l.lela): experimentació 
%   \end{itemize}


% \end{frame}



\begin{frame}{Sistemes duals}




\begin{tikzpicture}[scale=0.8, every node/.style={transform shape}]

      \tikzset{
        mynode/.style={rectangle,rounded corners,draw=black, 
          very thick, inner sep=1em, minimum size=3em, text centered,
          groc},
        myarrow/.style={->, shorten >=1pt, thick},
        mylabel/.style={text width=7em, text centered},
        groc/.style={top color=white, bottom color=yellow!50},
        verd/.style={top color=white, bottom color=green!50},
        roig/.style={top color=white, bottom color=red!50},
      }  






 \node[mynode] (m) {$S$};

 \node[right=2cm of m] (mdins) {};

 \node[mynode, verd, above right=0.6cm and 1cm of mdins] (tsms) {SGST};

 \node[mynode, verd, below right=0.6cm and 1cm of mdins] (mtsms) {SGSTM};

 \node[rectangle,draw,minimum height=6cm,minimum width=9.5cm,right=-0.25cm of mdins] (dual) {};

\draw[shift=( dual.south west)]   
  node[above right] {sistema dual de multiresolució};






 \node[mynode,right=3cm of mtsms] (ts) {$S'$};



 \draw (m.east) -- (mdins.east) node[above right,at start]
 {afegeix$(m)$};

 \draw[myarrow] (mdins.east) -- (tsms.west);
 \draw[myarrow] (mdins.east) -- (mtsms.west);


 \draw[myarrow] (tsms) -- (ts) node[above,midway,sloped]
 {$multiresolucio(S,\text{esquema})$}; 
 
 \draw[myarrow] (mtsms) -- (ts) node[above,midway,sloped]
 {$SerieTotal(M)$};




 \node[right=6cm of tsms] (consdins) {};

 \draw (tsms) -- (consdins.center);
 \draw (ts) -- (consdins.center);

 \node[right=2.5cm of consdins] (consultes) {};
 \draw[myarrow] (consdins.center) -- (consultes) node[above,midway,sloped]
 {consultes};



\end{tikzpicture}

Conceptes relacionats: 
\begin{itemize}
\item Arquitectura Lambda \parencite{marz14:bigdata} 
\item Precomputació de vistes \parencite{date04:introduction8}
\end{itemize}





\end{frame}




\subsection*{Teoria de la informació}
\begin{frame}{Avançaments en determinar la qualitat de la multiresolució}

  Quantificació de l'error que es comet en la selecció de la informació
  per a la muliresolució. 

\begin{block}{Definició del problema}
\centering
consulta(sèrie temporal original) vs.\ consulta(multiresolució)
\end{block}


Alguns dels escenaris particulars analitzats:
\begin{itemize}
\item Consulta a tota la sèrie temporal i mateixa consulta i atribut:
  \begin{itemize}
  \item Màxim i total no tenen error
  \item Mitjana té error
  \item Mitjana en el cas d'una sèrie temporal regular no té error 
  \end{itemize}

\item Consultes d'un interval determinat:
  \begin{itemize}
  \item Interval múltiple de les resolucions consolidades: similar al punt anterior
  \item No múltiple: hi ha error. Fitat per a variables monòtones creixents.
  \end{itemize}

% \item Conservació dels totals en comptadors

% \item Equivalència de l'atribut mitjana en sèries temporals regulars per a diversos mètodes de representació.

\end{itemize}

\end{frame}




%%% Local Variables: 
%%% mode: latex
%%% TeX-master: "defensa"
%%% End: 
%  LocalWords:  multiresolució

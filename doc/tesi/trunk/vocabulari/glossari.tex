


%Temes lingüístics

% \newglossaryentry{escandall}{name={escandall}, description={ Sobre escandall/escandallar/escandallatge com a alternativa a les aberracions lingüístiques \emph{mostrejar/mostreig}   Lluís Marquet, ESCANDALLAR I ESCANDIR (original gener 1991) \emph{Novetat i Llenguatge}, 2011, volum 4, p. 10.  }
% }



%TERMES


%temps real
% \newglossaryentry{TempsReal}{name={temps real}, description={(\emph{real time}), sistemes que han de respondre amb un temps determinat. A vegades també s'utilitza el terme com a adjectiu per a designar sincronització real amb el rellotge o per a indicar que l'usuari no percep retards. Allà on pugui causar confusió, utilitzarem sincronitzat o en línia (\emph{online}) per al segon significat.}
% }





% \newglossaryentry{SistemaGestioBaseDades}{name={sistema de gesti{ó} de base de dades}, description={(\emph{Data Base Management System})} }




%terme:SGBDR

% \newglossaryentry{terme:SGBDR}{name={sistema de gestió de base de dades relacional}, description={(\emph{Relational Data Base Management System}). 
% També anomenat 'object/relational' DBMS \parencite{date06}.
% Totes les definicions són coherents amb \textcite{date:introduction} } }



%model, implementació
%Els SGBD es basen en teories matemàtiques que reben el nom de model de dades, un SGBD és una implementació d'un model de dades.
%Segons \citeauthor{date:introduction}, ``un model de dades és una definició abstracta, auto continguda i lògica dels objectes, de les operacions i  de la resta que conjuntament constitueixen la màquina abstracta amb la que els usuaris interactuen. Els objectes permeten modelar l'estructura de les dades. Les operacions permeten modelar el comportament''. Ara bé, \citeauthor{date:introduction} avisa que el concepte model de dades també s'usa per a definir una estructura persistent de dades concreta i per tant cal distingir adequadament la confusió entre els dos conceptes.
% Tal com fa Date, parlarem de model de dades en el primer sentit de màquina abstracta i a vegades ho abreviarem com a model.


% %tipus,valor,variable,operador

% \newglossaryentry{terme:SGBDR:domini}{see={terme:SGBDR},name={domini}, description = {(\emph{domain}), equivalent a tipus de dades.
% Conjunt de valors. Cada domini té associat un conjunt d'operadors, en alguns casos fins i tot s'entén que el domini inclou els operadors (concepte de classe a orientació a objecte). Els tipus tenen una representació (estructura) o més d'una, és a dir els seus valors poden estar denotats per més d'un literal} }
% \newglossaryentry{terme:SGBDR:tipus}{see={terme:SGBDR:domini}, name={tipus de dades}, description = {(\emph{data type}), a vegades solament 'tipus' (\emph{type}) o bé 'tipus de dades abstracte' (\emph{abstract data type}). Segons \textcite{date:introduction} en el context de model tots els tipus de dades han de ser abstractes} }

% \newglossaryentry{terme:SGBDR:escalar}{parent={terme:SGBDR:domini}, name={escalar}, description = {Un tipus és escalar (\emph{scalar}) quan no té components visibles a l'usuari i és no escalar (\emph{nonscalar}) en cas contrari; no obstant, tant els escalars com els no escalars tenen representació, la qual pot contenir components} }


% \newglossaryentry{terme:SGBDR:valor}{see={terme:SGBDR},name={valor}, description = {(\emph{value}), equivalent a objecte i instància.
% 'Constant individual' que és d'un tipus de dades. A vegades s'utilitza 'constant' per designar una  variable que mai canvia de valor, però aquest no és el cas d'aquesta definició} }
% \newglossaryentry{terme:SGBDR:objecte}{see={terme:SGBDR:valor}, name={objecte}, description = {(\emph{object})} }
% \newglossaryentry{terme:SGBDR:instancia}{see={terme:SGBDR:valor}, name={instància}, description = {(\emph{instance})} }

% \newglossaryentry{terme:SGBDR:literal}{see={terme:SGBDR},name={literal}, description = {(\emph{literal}).
% Símbol que denota un valor. Un valor pot estar denotat per més d'un literal. Segons aquesta definició literal no és equivalent a valor} }


% \newglossaryentry{terme:SGBDR:variable}{see={terme:SGBDR},name={variable}, description = {(\emph{variable}).
% Contenidor d'una aparició d'un valor. El valor que conté la variable pot ser canviat mitjançant l'operador d'assignació. En canvi els valors, per si mateixos, no poden ser actualitzats} } %A l'esquerra de l'operador d'assignació sempre hi ha variables, tot i que s'admeten simplificacions mitjançant expressions que són pseudovariables (p.ex. s[1] := 3 és equivalent a s := [s[0],3,s[2],..]).
% %Les variables tenen adreces (\emph{addresses}) i per tant es pot apuntar (\emph{point to}) a les variables mitjançant els operadors de referència (\emph{referencing}), el qual retorna l'adreça d'una variable, i de desreferència (\emph{dereferencing}), el qual retorna la variable a partir de l'adreça. Els valors adreces pertanyen al tipus apuntador, però el model relacional prohibeix els valors de tipus apuntador i per tant no té REF ni DEREF; les relvar s'identifiquen pel seu nom i no cal que tinguin adreça. (Compte que en orientació a objectes una variable és el contenidor d'un valor que és un ID d'objecte, és a dir és el contenidor d'una referència).



% %relació
% \newglossaryentry{terme:SGBDR:relacio}{%
%   see={terme:SGBDR},%
%   name={relació},%
%   plural={relacions},%
%   sort={relacio},%
%   description = {(\emph{relation}). Pot referir-se tant a tipus,
%     valor, literal o variable relació. És l'objecte principal d'estudi
%     en els SGBDR i de manera popular s'anomena taula. \emph{Nota}:
%     hi ha certes diferències lògiques entre les relacions del model
%     relacional i les relacions tal com es defineixen en matemàtiques.
%   }%
% }










%terme:tipus

% %reals projectius
% \newglossaryentry{terme:tipus:real-projectiu}{%
%   see={terme:SGBDR:tipus},%
%   name={real projectiu},%
%   plural={reals projectius},%
%   symbol={\ensuremath{\bar\mathbb{R}}},%
%   description = {(\emph{projective extended real
%       numbers}). 
% %$\bar\mathbb{R}\in\mathbb{R}\cup$
% %\{-\infty,+\infty\}$.
%   }%
% }





% [date2005]
% The original version of the model also omitted a few things I now consider vital. For example, it excluded any
% mention—at least, any explicit mention—of all of the following: predicates, constraints (other than candidate
% and foreign key constraints), relation variables, relational comparisons, relation type inference and associated
% features, certain algebraic operators (especially rename, extend, summarize, semijoin, and semidifference),
% and the important relations TABLE_DUM and TABLE_DEE.




%pendent: falta posar el name

% \newglossaryentry{SGBD-model}{ description = {Un model és}, name={Model de SGBD} }



% \newglossaryentry{SBDR-cap}{ description = {La capçalera d'un SGBDR}, name={Capçalera}, parent={SGBD-model} }



% \newglossaryentry{heading}{ description = {Equivalent to intension and relation schema} }
% \newglossaryentry{intension}{ description = {}, see=heading }
% \newglossaryentry{relation schema}{ description = {}, see=heading }

% \newglossaryentry{body}{ description = {Equivalent to extension} }
% \newglossaryentry{extension}{ description = {buit}, see=body}


% \newglossaryentry{DBMS data model}{ description = {A data model (first sense) is an abstract, self-contained, logical definition of the
% objects, operators, and so forth, that together constitute the abstract machine with which
% users interact. The objects allow us to model the structure of data. The operators allow us
% to model its behavior.\cite{date}. Sometimes it is referred as architecture.
% } }

% \newglossaryentry{data model}{ description = { A data model (second sense) is a model of the persistent data of some particular
% enterprise. [date06]. }}


% \newglossaryentry{DBMS implementation}{ description = {An implementation of a given data model is a physical realization on a real
% machine of the components of the abstract machine that together constitute that model.\cite{date}} }


% \newglossaryentry{data independence}{ description = {model and implementation kept separated}}




% \newglossaryentry{relationships}{
% description={relationships are semantic. relationships are entities.}}






% %terme:noms propis

% \newglossaryentry{RRDtool}{name={RRDtool}, description={Sistema de
%     gestió de base de dades per a sèries temporals basat en la tècnica
%     de Round Robin \parencite{rrdtool}. S'autodefineix com a estàndard
%     de programari lliure industrial i el seu nom prové de \emph{Round
%       Robin Database management system}.} }



%%% Local Variables: 
%%% mode: latex
%%% TeX-master: "../main"
%%% End: 

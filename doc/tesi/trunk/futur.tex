\chapter{Treball futur}
\label{sec:futur}


En aquest capítol es proposen treballs futurs a partir del que s'ha  dissertat en aquest document.

% * El temps pot ser vist com un valor amb incertesa. Aleshores què passa amb les operacions?




% We have showed some aggregation functions examples with simple
% aggregation statistics, mean and maximum, and simple representation
% methods, Delta and \zohe{}. More attribute aggregation functions could
% be designed based on methods from other fields such as data streaming
% or time series data mining, especially it would be interesting aggregations with uncertain data.

% * Agregacions en el domini freqüencial


% * Aclariment sobre les operacions: algunes són perfectament conseqüència de conceptes (per exemple la diferència té un raonament a partir de la pertinença, la intersecció a partir de la diferència) altres hi ha conseqüència però s'ha de prendre una decisió (per exemple en el cas de la unió i la unió temporal s'ha de decidir quina és quina) i altres les hem definit com a passos intermedis per a altres oepracions (per exemple la concatenació temporal). En aquest darrers casos cal, doncs, cercar-ne si hi ha algun raonament o teoria sobre el qual es poden basar.

* Implementar variacions dels \gls{SGSTM}


% * el model sgstm s'ha fet pensant en els instants de consolidació
% periòdics. Què passa quan no ho són?  per exemple event trigerred
% (només emmatgatzemem informació quan creiem que és
% interessant). Aleshores potser són casos molt especials i no es pot
% dibuixar l'esquema? Potser posar tot això com a comentari a la secció
% d'Altres Estructures: En el model de \gls{SGSTM} s'ha considerat que
% la consolidació es feia periòdicament però es podria fer quan es
% cregués oportú, aleshores no es pot preveure quin esquema de
% multiresolució hi haurà; són casos que requereixen un estudi més
% profund. Etc. 


% \subsubsection{Operacions habituals en les sèries temporals}


% \paragraph{Semblança de dues sèries temporals}


% Similarity Measures for Time Series

% Hi ha varis mètodes, [keogh08:vldb] n'avalua uns quants i els generalitza amb:

% Given two
% time series T1 and T2 , a similarity function Dist calcu-
% lates the distance between the two time series, denoted by
% Dist(T1 , T2 ).

% Exemplifiquem amb la distància euclídia, [keogh08:vldb] nota que és
% competitiva amb les altres.

% Distancia euclídia segons [faloutsous94-sigmod]


% \[
% D(S,Q) = \left( \sum_{i=1}^{l} (S[i]-Q[i])^2  \right)^{1/2}
% \]

% \begin{gather*}
%   D: S \times Q \longrightarrow v: \\
%   S' = map(fusio(S,Q),(t,v_1,v_2)\mapsto(t,(v_1-v_2)^2)), \\
%   S'' = fold(quad,(0,0),(t^1,v^1,t^2,v^2)\mapsto(t^1,v^1+v^2)), \\
%   v = \sqrt{V(m)}:m\in S''
% \end{gather*}


% S i Q haurien de ser regulars entre elles, sinó cal aplicar una fusió amb representació/interpretació.

% Amb la multiresolució la fusió es pot fer de forma eficient. Per altra banda, es podria crear un disc resolució amb agregador de semblança.


% \paragraph{Semblança de dues sèries temporals amb offset}

% Aquí es descriu la solució general del problema (SequentialScan),
% [faloutsous94-sigmod] n'estudia implementacions amb certes
% heurístiques que aconsegueixen més eficiència.





% \paragraph{Filtratge senzill per mitjana mòbil}

% Sigui $p$ la mida de la finestra mòbil
% \begin{gather*}
%   \text{MitMobil}: S \times \text{p} \longrightarrow S':\\
%   \text{map}(S,(t,v)\mapsto \text{mitjanaV}(S[t,t+p]))
% \end{gather*}


% Mitjana mòbil sobre la multiresolució



% \paragraph{Farciment de forats}

% Jo tinc una sèrie temporal i vull que entre dues mesures no hi hagi més d'un cert temps. Si no es compleix dic que té forats. 

% Sigui $S$ una sèrie temporal, aquesta té forats de més durada que $d$
% si alguna mesura compleix $\text{forats}(S,d) = \text{selecciona}(difT(S),v>d \bigwedge v\neq\infty)$ a on $difT(S) = \text{map}(\text{tpredecessors}(S),(t,v)\mapsto(t,t-v))$.

% Amb la multiresolució el farciment de forats és natural a l'estructura i és controlat per la funció agregadora d'atributs.


% * Com farciria els forats manualment a una sèrie temporal?

% 1. Passar-ho per un esquema de multiresolució

% 2. Treballar sobre la sèrie temporal:

% a partir del càlcul de forats anterior $\text{forats}(S,d)$ per
% exemple apliquem un farciment amb representació
% zohe. $\text{farciment}(S,d) = \text{unio}(S,S')$ a on fem la selecció
% de resolució $S' = S[T]^{\text{zohe}}$, $\forall (t,v) \in
% \text{forats}(S,d): T = \{ \tau = t - dn |
% \tau\in(t-v,t),n\in\mathbb{N} \}$.







% \subsubsection{Com treure profit de les operacions dels SGSTM}

% Temes que després es poden aprofitar a les implementacions

% * No hi ha updates --> les sèries temporals no s'han de canviar

% * Per exemple, vull calcular la mitjana de  BDSTM(a,b] si tinc un disc resolució amb $\delta=b-a$ i $f=$mitjana aquest seria l'adequat en comptes de calcular mitjana(SerieTotal(M)(a,b])

% %??
% % No obstant, la base de dades multiresolució conté informació sobre la
% % resolució de les subsèries i per tant aquesta operació és susceptible
% % d'implementar-se aprofitant aquesta informació.  A tall d'exemple es
% % defineix una operació per extreure de la base de dades multiresolució
% % una sèrie temporal regular amb període $T$:


% % \begin{definition}[Selecció de resolució regular]
% %   \begin{gather*}
% %     \text{ResolucióRegular}: M^* \times T \times r \longrightarrow S'\\
% %     \forall (S_{Bi},S_{Di},\delta_i,\tau_i,k_i,f_i) \in M : \\
% %     d_i = T - \delta_i , \\
% %     0 \geq d_0 > d_1 \dots > d_a, 0 < d_{a+1} < \dots < d_d: \\
% %     S'' = S_{D0} || S_{D1} || \dotsb || S_{Da}  ||  S_{Da+1} || \dotsb || S_{Dd}, \\
% %     S' = S''[i]^r: i = {t|0+nT,n\in\mathbb{N}}
% %   \end{gather*}
% % \end{definition}

% % Nota: les operacions no són equivalents, l'operació $\text{SerieTotal}(M)[i]^r$ és molt més potent que la $\text{ResolucióRegular}(M,T)$.




% \subsubsection{Semàntica de comportament}

% \todo{?}





%%% Local Variables:
%%% TeX-master: "main"
%%% End:
% LocalWords:  SGSTM

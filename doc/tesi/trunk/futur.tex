\chapter{Treball futur}
\label{sec:futur}


\todo{fer intro}

En aquest capítol es proposen treballs futurs a partir del que s'ha  dissertat en aquest document.

També algunes manies que hem tingut, pex la de la forta correspondència model-implementació, i la motivació que tenen pensant en el futur.


* Models

* Implementacions

* Reflexions de la qualitat



\section{Models}



En el model de \gls{SGST} s'ha definit un seguit d'operacions que són
les que considerem més bàsiques per a poder manipular les sèries
temporals. Algunes operacions són conseqüència directa d'altres
conceptes; per exemple la diferència a partir de la pertinença o la
intersecció a partir de la diferència. D'altres operacions són
conseqüència però cal prendre una decisió sobre el raonament; per
exemple en la unió i la unió temporal cal decidir quina prové de
l'ordre total i quina de l'ordre parcial.  I d'altres operacions són
passos previs per a altres operacions; per exemple la concatenació
temporal s'utilitza en els \gls{SGSTM} per a calcular la sèrie
temporal total en el context d'un mètode de representació.  Així
doncs, caldria establir clarament la motivació de cada operació i
cercar en cada cas el raonament sobre el qual es pot basar cada
operador. També, de forma breu, hem notat les propietats d'alguns
operadors com la commutativitat, però caldria explorar més
profundament les propietats de tots els operadors.


En les funcions d'agregació d'atributs dels \gls{SGSTM}, n'hem
proposat alguns exemples sobretot raonats a partir dels mètodes de
representació. Tot i així, hem proposat estadístics senzills
--mitjana, màxim i darrer-- atès que l'objectiu és mostrar el
comportament que tenen en la multiresolució.  En els treballs
d'anàlisi de sèries temporals es presenten multitud de mètodes i
d'algoritmes: per a extreure patrons de les sèries temporals, per a
cercar periodicitats, per a comparar dues sèries temporals,
agregacions en el domini freqüencial, per a fer prediccions, per a
validació de dades, etc. Per tant, es poden dissenyar més funcions
d'agregació d'atributs basant-se en qualsevol d'aquests algoritmes o
mètodes; només cal adaptar el problema per tal de retornar una mesura
que resumeixi la informació d'un interval de la sèrie temporal.



De les funcions d'agregació d'atributs n'hem notat la possibilitat
d'orientar-les a flux. El model de \gls{SGSTM} és adequat per a
computar-se en flux llevat de les funcions d'agregació d'atributs que
es defineixen genèriques. Creiem, doncs, que també és interessant
aplicar orientació a flux en aquestes funcions i caldria aprofundir en
aquests algoritmes, com per exemple els que proposa
\textcite{cormode08:pods}.



En la teoria de la mesura, la incertesa sol acompanyar les mesures. La
incertesa reflecteix probabilísticament els límits del que es coneix
sobre la quantitat mesurada.  Així doncs, seria interessant poder
incorporar la incertesa en els models.  Principalment la incertesa
hauria d'acompanyar els atributs de temps i de valor de les sèries
temporals, és a dir haurien de reflectir la incertesa que hi ha en
cada mesura a l'hora d'adquirir un valor un instant de
temps. Aleshores, caldria estudiar com aquesta incertesa afecta les
operacions, és a dir com es propaga la incertesa quan s'uneixen dues
sèries temporals, quan es representen, en les funcions d'agregació
d'atributs, etc.




Pel que fa a la consolidació de la multiresolució, aquesta s'ha pensat
sobretot periòdica per tal d'obtenir sèries temporals regulars.
Aleshores s'obtenen els esquemes de multiresolució periòdics que hem
analitzat. Això no obstant, altres escenaris de consolidació són
possibles; per exemple sistemes de monitoratge que adquireixen dades
només quan ho creuen interessant en base a esdeveniments.  En el model
de \gls{SGSTM} no estan previstos aquests casos i requereixen un
estudi més profund. Per exemple, en un cert moment es podria
considerar que una subsèrie resolució és molt significativa i que
s'han de mantenir aquestes dades, per tant en l'esquema de
multiresolució hi hauria una subsèrie fixada a un interval de temps en
el passat.





En els esquemes de multiresolució, s'ha notat la propietat de
desfasament d'una subsèrie resolució. Aquest desfasament és induït per
la funció d'agregació d'atributs. D'una banda, aquest desfasament pot
ser requerit per la naturalesa de la funció; per exemple en el cas
d'agregacions \gls{dd} o \gls{zoh} introdueixen un desfasament perquè
interpolen cap endavant, de fet el resultat de consolidació entre la
\gls{zoh} i la \gls{zohe} és similar però aquesta darrera per
naturalesa interpola cap enrere i no necessita desfasament.  D'altra
banda, aquest desfasament podria ser afegit intencionalment per a
controlar els solapament entre les subsèries resolució.  Aquest
solapament significa que en l'esquema de multiresolució les diverses
subsèries resolució coincideixen a emmagatzemar informació pels
mateixos instants de temps. Proposem dos escenaris de solapament:
\begin{itemize}

\item Les subsèries resolució se solapen totalment. Això pot servir
  per a disposar de diferents resums de la sèrie temporal preparats
  per a ser visualitzats immediatament. És a dir, el \gls{SGSTM}
  permet escollir ràpidament entre diferents \emph{zooms} de les
  dades.

\item Les subsèries resolució no se solapen, és a dir les subsèries
  amb menys resolució acaben allà on comencen les de més
  resolució. Això pot servir per a aprofitar al màxim la resolució i
  l'espai d'emmagatzematge, sense que cap subsèrie desi informació per
  al mateix interval de temps. A part d'introduir desfasament, per tal
  que no se solapin també es pot dissenyar un esquema de
  multiresolució on el pas de consolidació de cada subsèrie sigui
  exactament $\delta_j = k_i\delta_i$ on $\delta_i<\delta_j$. És a dir
  la subsèrie amb menys resolució ($j$) té un pas de consolidació
  exactament múltiple del pas de consolidació i el cardinal de la
  subsèrie superior en resolució ($i$). Aquestes restriccions es
  podrien considerar també en el cas de l'estructura de resolucions
  encadenades.
\end{itemize}


%També l'emmagatzematge de metades, es  pot fer perfectament amb els SGBDR.









\section{Implementacions}


En les implementacions hem treballat a nivell acadèmic, és a dir sense
objectius d'optimització del rendiment. De fet, aleshores les
implementacions s'allunyarien del model i per tant del nostre objectiu
de mantenir una forta correspondència entre la forma de la
implementació i del model. Aquesta correspondència és útil per a
manteniments futurs: qualsevol millorar en el model pot ser
traslladada immediatament a les implementacions o bé, a la inversa,
qualsevol error trobat en les implementacions pot ser localitzat
fàcilment i estudiat en el model.



%RoundRobinson


Tot i així, aquesta correspondència model-implementació no sempre és
senzilla de mantenir. A Pytsms i RoundRobinson, les implementacions
més completes que hem realitzat dels models, s'hauria de simplificar
la la definició de les mesures genèriques.  En el model abstracte
matemàtic és senzill de descriure uns valors genèrics, en canvi a les
implementacions això complica l'estructura. Així, s'ha hagut de
dissenyar mesures de diferents tipus que contenen el rang del domini
de temps i valors per tal de definir les mesures indefinides, el
suprem i l'ínfim, etc.  També s'hauria de repensar la gestió de la
homogeneïtat de les sèries temporals: en el model les sèries temporals
són homogènies i en les implementacions és difícil gestionar el
concepte de nova sèrie temporal amb el mateix tipus de mesures que les
originals.

En altres casos, però, s'ha trobar una solució adequada per a
implementar la genericitat del model.  Per exemple, és el cas de la
multitud d'operacions de les sèries temporals implementades amb
Mixins, el de les representacions com a objectes independents
associats a les sèries temporals, o bé els de les funcionalitats
complementàries com l'emmagatzematge i els gràfics implementades amb
el patró Visitor. Tanmateix, algunes parts encara no han estat prou
generalitzades; per exemple els gràfics de RoundRobinson sempre
utilitzen la representació \gls{zohe}.





En les altres implementacions, l'objectiu s'ha centrat en observar
altres paradigmes d'implementació dels models. Així, ens hem pres la
llibertat de no implementar tota la genericitat del model sinó casos
simplificats. Caldria avaluar fins a quin límit aquestes
implementacions es podrien apropar més al model, per exemple a
RoundRobindoop hem notat algunes limitacions a l'hora d'usar les
funcions d'agregació d'atributs.


%RoundRobindoop

A RoundRobindoop usem Hadoop com a intèrpret de la tècnica de
programació para\l.lela MapReduce. L'execució d'aquesta tècnica
implica un compromís a l'hora d'escollir el nombre de processos en
para\l.lel ja que cada un té un cost mínim de crear-se i a més cal
comptar el cost de distribuir les dades. Es podria experimentar més
amb Hadoop en aquest sentit, és a dir amb diferents quantitats de maps
i de reduces. De fet només hem provat amb un node de computació, però
Hadoop té la possibilitat de distribuir a més nodes.


Hi ha altres projectes que també utilitzen Hadoop com a sistema
d'emmagatzematge distribuït de sèries temporals. Aquest és el cas
d'OpenTSDB \parencite{opentsdb}, que utilitza Hadoop per a
emmagatzemar i recuperar ràpidament sèries temporals.  Aquest, però,
només no té en compte l'aplicació de consultes a les dades sinó només
recuperar les dades originals.  RoundRobindoop és una solució per a
computar la multiresolució a Hadoop. Així doncs, OpenTSDB i
RounbdRobindoop podrien treballar conjuntament: el primer per a
emmagatzemar distribuïdament les sèries temporals i el segon per a
calcular la multiresolució aprofitant que les dades ja estan
distribuïdes en diversos nodes.



%Relstsms


A Reltsms hem implementat el model d'\gls{SGST} seguint la programació
acadèmica del model relacional. Es podria seguir la mateixa
aproximació per a implementar també el model d'\acro{SGSTM}.  En
aquestes implementacions, es podria experimentar amb un dels punts
forts dels \gls{SGBDR}: l'optimització de les
consultes \parencite[\gls{capitol}~18
\emph{Optimization}]{date04:introduction8}. Les expressions
relacionals són d'alt nivell matemàtic i això permet trobar
expressions equivalents a una consulta. Aleshores, els sistemes poden
decidir quina expressió és la millor per a ser executada.  En aquest
sentit, hem definit els operador de Reltsms a partir dels operadors
relacionals. Això no obstant, s'hauria d'estudiar si a Tutorial~D les
funcionalitats d'optimització s'estenen automàticament als operadors
derivats dels primitius.



En un sentit relacional, també cal comparar Pytsms i RoundRobinson amb
Dee. Dee \parencite{dee} és la implementació amb Python d'un
llenguatge de bases de dades relacionals que compleixi amb les normes
D del Third Manifesto. A Pytsms i RoundRobinson hem utilitzat els
conjunts de Python com a objectes bàsics però es podrien utilitzar els
conjunts relacionals de Dee, els quals ja incorporen propietats i
mètodes de \gls{SGBD}. Si més no, el raonament que fa Dee com a gestor
de bases de dades també podria ser aplicat a Pytsms i
RoundRobinson.
  



%Experiment amb dades


\todo{fer} 


Amb l'experimentació amb dades reals hem pogut demostrar el correcte funcionament de la multiresolució. 

* Experimentació amb dades: només demostrem el correcte funcionament de la multiresolució. No avaluem rendiment dels recursos, per exemple no avaluem temps de computació només n'hem mostrat un a tall de referència orientativa.  Cal provar amb més diversitat: dades de diferent mida, de diversa naturalesa, diferents esquemes de multiresolució, dades de referència utilitzades en altres recerques (per exemple keogh?)





* Implementar variacions dels \gls{SGSTM}


  Una mostra que es pot millorar molt
el codi és que canviant mètode add de les TimeSeries amb
MeasureTotalEquality hem aconseguit una millora espectacular de temps
de computació: en la inserció cal cercar que no hi hagin temps
repetits i ara s'aprofita la cerca dels sets de Python mitjançant
hash.









\subsection{Sistemes de multiresolució  integrats en maquinari}

Es poden realitzar altres implementacions dels \gls{SGSTM} que siguin
molt específiques. Una bona implementació ja no és només aquella que
calcula en poc temps sinó que en alguns contextos també pot ser un
consum baix d'energia, ocupar poc espai, etc.  En aquest sentit,
pensem en sistemes específics integrats en xarxes de sensor. Així, es
podria implementar un \gls{SGSTM} integrat en el maquinari d'un sensor.

Aquesta implementació integrada es podria realitzar tant en un
microcontrolador que gestionés una memòria seguint l'esquema de
multiresolució o bé com a circuit digital, aprofitant que l'esquema de
multiresolució té una mida finita i és implementable en maquinari.


En la implementació en maquinari, es podria seguir l'esquema de la
\autoref{fig:vhdl:resolucio}. Aquest esquema és per a una subsèrie
resolució, per tant una sèrie temporal multiresolució seria un conjunt
d'aquests esquemes. Així, aquest esquema és la integració de l'esquema
de les subsèries resolució de la \autoref{fig:sgstm:bdsubserie}.  En
el buffer es van afegint les mesures --temps i valor-- i calcula la
mesura resultant --l'atribut agregat. Aquest atribut agregat
s'emmagatzema al disc a cada pas de consolidació --marcat per
l'esdeveniment consolida-- i el disc gestiona l'emmagatzematge afitat
--les dades consolidades $D_0,D_1,\dotsc,D_k$. Caldria afegir un mòdul
de temps que a partir del rellotge -- per exemple un real-time clock
(RTC)-- marqués els passos de consolidació i indiqués el darrer temps
de consolidació. 





\begin{figure}[htp]
\centering
\begin{tikzpicture}
\tikzset{
    maquina/.style={rectangle,rounded corners,draw=black, 
      very thick, inner sep=1em, minimum size=3em, text centered,
      groc},
    interficie/.style={rectangle,rounded corners,draw=black, 
       inner sep=0.2em, minimum size=1em, text centered,
      verd},
    modul/.style={rectangle,rounded corners,draw=black, 
      very thick, inner sep=1em, minimum size=3em, text centered,
      roig},   
    myarrow/.style={->, >=latex', shorten >=1pt, thick},
    fletxaswitch/.style={<->, >=latex',shorten >=10pt,shorten <=10pt, thick},
    mylabel/.style={text width=7em, text centered},
    groc/.style={top color=white, bottom color=yellow!50},
    verd/.style={top color=white, bottom color=green!50},
    roig/.style={top color=white, bottom color=red!50},
  }  

  
   \node (discres) [draw, dotted, minimum width=9.5cm, text depth=9cm, rectangle] {Subsèrie resolució};



  \node[modul,text depth=3cm,below right=1cm and 1.7cm of discres.north west] (buffer) {Buffer};  

  %entrades
  \node[above left=-1.5cm and 2.5cm of buffer.north west] (buffer_valor)   {};
  \draw[-] (buffer_valor) -- (buffer_valor-|buffer.west)
   node[near end,above]{valor};

   \node[below=0.5cm of buffer_valor] (buffer_nou)   {};
   \draw[-] (buffer_nou) -- (buffer_nou-|buffer.west)
   node[near end,above]{temps};

   \node[above left=-3.5cm and 1cm of buffer.north west] (buffer_consolida) {};
   \draw[-] (buffer_consolida) -- (buffer_consolida-|buffer.west)
   node[pos=0.2,above]{consolida};

   %sortides
   \node[above right=-1.5cm and 1.5cm of buffer.north east] (buffer_dada)   {};
  \draw[-] (buffer_dada) -- (buffer_dada-|buffer.east)
   node[pos=0,above]{atribut agregat};





  \node[modul,right=3cm of buffer,text depth=3cm] (disc)   {Disc}; 

  % entrades
  \node[above left=-1.5cm and 2cm of disc.north west] (disc_valor)   {};
  \draw[-] (disc_valor) -- (disc_valor-|disc.west)
   node[near end,above]{};

  \node[above left=-3.5cm and 1.96cm of disc.north west] (disc_consolida)   {};
  \draw[-] (disc_consolida) -- (disc_consolida-|disc.west)
   node[pos=0.58,above]{consolida};

   % sortides
   \node[above right=-1.5cm and 2.5cm of disc.north east] (disc_d0)   {};
   \draw[-] (disc_d0) -- (disc_d0-|disc.east)
   node[near end,above]{$D_0$};

   \node[below=0.5cm of disc_d0] (disc_d1)   {};
  \draw[-] (disc_d1) -- (disc_d1-|disc.east)
   node[near end,above]{$D_1$};

   \node[below=0.5cm of disc_d1] (disc_d2)   {};
  \draw[-] (disc_d2) -- (disc_d2-|disc.east)
   node[near end,above]{$\dots$};

   \node[below=0.5cm of disc_d2] (disc_d3)   {};
  \draw[-] (disc_d3) -- (disc_d3-|disc.east)
   node[near end,above]{$D_k$};




  \node[modul,below=1cm of buffer,text depth=1.5cm] (temps)   {Temps}; 

  % entrades
  \node[above left=-1cm and 2.5cm of temps.north west] (temps_rtc)   {};
  \draw[-] (temps_rtc) -- (temps_rtc-|temps.west)
   node[near end,above]{RTC};

  % sortides
   \node[above right=-1cm and 1cm of temps.north east] (temps_delta)   {};
   \draw[-] (temps_delta) -- (temps_delta-|temps.east)
   node[near end,above]{$\delta$};

   \node[above right=-2cm and 7cm of temps.north east] (temps_tau)   {};
   \draw[-] (temps_tau) -- (temps_tau-|temps.east)
   node[pos=0.96,above]{$\tau$};








   %connexions
   \draw[-] (temps_delta.west) -- (disc_consolida.east); 
   
%   \node[above left=0.3cm and 1cm of temps.north west] (tau_reset)   {};
   \node[below=1cm of buffer_consolida] (tau_reset)   {};
   \draw[-*,shorten >=-2pt] (tau_reset) -- (tau_reset-|disc_consolida.east);
   \draw[-] (tau_reset.east) -- (tau_reset.east|-buffer_consolida);


 \end{tikzpicture}
 
\caption{Esquema d'integració d'una subsèrie resolució}
\label{fig:vhdl:resolucio}
\end{figure}

En aquesta implementació només fem referència a la part
d'emmagatzematge.  Caldria implementar un protocol per tal de
consultar les dades emmagatzemades, si bé forma senzilla es podria
implementar com si els discs fossin un perifèric de memòria.  

% Podrien ser emmagatzematges volàtils depenent de la criticitat de les dades



Algunes aplicacions dels sistemes integrats de multiresolució podrien ser:

\begin{itemize}
\item Emmagatzematge de la multiresolució en perifèrics la informació
  dels quals no és essencial però pot ajudar a monitorar-ne el seu
  funcionament. Per exemple comptadors d'aparells de xarxa,
  temperatures dels components, etc.

\item Aparells integrats molt petits, en els quals hi ha molt poc
  espai per a l'emmagatzematge. %en materials molt prims i doblegables.

\item Com un complement més de sensors inte\l.ligents, que actualment
  ja integren diverses tasques: filtratge del senyal, busos de
  comunicacions, llindars d'alarma, etc.


\item Per a computar funcions d'agregació d'atributs complexes. En
  aquest cas els buffers podrien treballar directament amb components
  del maquinari. Per exemple per a agregacions de sèries temporals en
  què els valors fossin imatges.

\item Implementació de la multiresolució en Field Programmable Gate
  Arrays, és a dir en dispositius de maquinari configurables. Això
  permetria flexibilitat a l'hora de canviar els esquemes de
  multiresolució integrats.



\end{itemize}





\subsection{RRDtool}

\todo{fer}


Ara hauríem de reprendre RRDtool i avaluar fins a quin punt compleix el nostre model. 
Segur que hi ha punts en què no compleix... 

Podem dir que RRDtool és la implementació productiva que actualment més s'apropa al concepte de multiresolució que hem formalitzat.


Regarding other implementations,
% \emph{RRDtool} can be seen as an specific case of \acro{MTSMS} and as
% a NoSQL system, although Oetiker \cite{rrdtool} has not commented
% it. However, regardless of the implementation backend, we have shown
% how a generic model for \acro{MTSMS} can be defined firmly rooted on
% \acro{DBMS} algebra theory.






% RRDtool té una estructura multiresolució amb un buffer únic d'entrada
% i buffers orientats a stream; segons havíem avaluat anteriorment \parencite{llusa11:tfm}.


% S'ha d'estudiar com es fan les consultes a RRDtool

% \url{http://en.wikipedia.org/wiki/RRD_Editor}



% Podem considerar que:

% 1. RRDtool és un SGBD NoSQL?
% 2. Nosaltres n'hem formalitzat un model lògic?
% 3. És el primer model lògic per a un producte NoSQL?
% 4. Aquest model lògic es pot implementar tant en productes relacionals com amb NoSQL? i per tant es demostra que els models lògics són extremadament potents i necessaris?
% 5. La implementació que fa RRDtool és molt eficient per a un determinat camp d'aplicació?
% 6. La implementació relacional seria molt genèrica i propera al model però no tan eficient? més aviat subjecte a l'eficiència genèrica dels SGBDR?
% 7. Els SGST són uns SGBD més simples? no tenen tantes actualitzacions de valors, no hi ha tantes relationships en l'esquema... Els SGST només es preocupen de sèries temporals i per tant només d'un tipus de dades en concret, això no obstant tal com s'ha dissenyat el model aquest tipus de dades es pot implementar en SGBD més complexos. 

Hi ha RRDtool, que és un SGBD específic dissenyat per a dades monitorades. Les causes del seu disseny són:

* Tobias Oetiker dissenyava un monitor de paràmetres de xarxes de comunicacions i en aquest monitor una part era la d'emmagatzematge de les dades. Per raons pràctiques i d'utilitat dissenya aquesta part amb un esquema inovadós. Finalment acaba separant aquesta part i la converteix independentment en RRDtool.

* RRDtool té aquest model pràctic i a la pràctica és molt útil per a ser usat com a SGBD dels sistemes de monitoratge, sobretot en l'àmbit dels comptadors de xarxa on és l'estàndard de facto. 

Això no obstant, no hi ha cap raonament teòric sobre el model de RRDtool ja que s'ha dissenyat per raons pràctiques. Per tant, entendre el funcionament de RRDtool és complicat, hi ha un nivell molt elevat per començar a fer-lo funcionar i molts conceptes no s'entenen perquè no estan ben definits. 

Per això ens proposem de compendre i formalitzar el model de RRDtool, que acabarem anomenar model de multiresolució, en la teoria dels sitemes d'informació. A més RRDtool és molt específic pel camp de comptadors de xarxa i volem oferir un model genèric per a altres àmbits.  









\section{Reflexions sobre la qualitat}


El capítol d'aplicació de la teoria de la informació és només una
introducció al problema de la qualitat de la multiresolució.  La
teoria de la informació formalitza anàlisis més profundes per a la
compressió de dades que es podrien aplicar també a la multiresolució.
En el cas que es conegui més bé el context i el comportament de la
sèrie temporal a la qual s'aplica la multiresolució, es pot detallar
més bé la quantificació de l'error. És a dir, en termes de la teoria
de la informació aleshores hi ha més coneixement sobre la predicció
del comportament de les dades cosa que es pot utilitzar per a avaluar
característiques més concretes. Per exemple, una variable real
adquirida té limitat el rang de valors que pot prendre i fins i tot
pot tenir un comportament probabilístic determinat.



Cal notar que no avaluem la idoneïtat d'aplicar un estadístic o un
altre a unes dades ni quin és el que més bé va per a obtenir una
informació. Només expressem el cas que es vol aplicar una consulta amb
una agregació determinada a una sèrie temporal i avaluem l'error que
hi ha comparant l'aplicació de la consulta a les dades originals amb
les dades multiresolucionades. 

Per a determinar un esquema de multiresolució --la quantitat de
resolucions, els passos de consolidació de cadascuna, els cardinals,
\dots-- caldria analitzar cada problema particular en el seu context i
utilitzar els coneixements adequats. Per exemple, per a treballar amb
problemes de so la teoria del senyal formalitza tot de raonaments que
no es poden obviar a l'hora de definir-ne un esquema de
multiresolució.  O bé, també en la teoria del senyal, es poden trobar
anàlisis per a determinar bons passos de consolidació per a una
variable, per exemple de temperatura. Així i tot, en el cas dels
comptadors se'n pot fer un raonament a banda per a conservar la seva
informació genuïna de comptatge lligada a com l'adquireixen.


A partir de les reflexions fetes sobre la qualitat de la
multiresolució s'obren un seguit de qüestions més, cadascuna de les
quals és un repte futur:
\begin{itemize}
\item Quina redundància d'emmagatzematge hi ha entre diverses
  subsèries s'una mateixa sèrie temporal multiresolució.? Què ocorre
  quan hi ha més d'una resolució i són de diferents funcions
  d'agregació d'atributs?

\item En cas que es perdi una resolució, es podria reconstruir a
  partir de les altres? O bé en cas que es vulgui ampliar la mida d'un
  disc, es podria completar amb les dades d'altres resolucions?

\item Les resolucions encadenades són interessant perquè aprofiten les
  dades emmagatzemades però afegeixen més restriccions a la compressió
  d'informació. Com es poden barrejar diferents passos de consolidació
  i diferents funcions d'agregació d'atributs?


\item Gràcies a un esquema de multiresolució es pot emmagatzemar dades
  d'una sèrie temporals durant un llarg temps de forma
  comprimida. Aquesta durada de temps és llarga però finita, tot i que
  quan s'esgoti dinàmicament es pot afegir una nova subsèrie amb
  inferior resolució però amb més lapse. Inicialment aquesta subsèrie
  serà buida, però es podria utilitzar les dades ja emmagatzemades per
  a iniciar-la?

\end{itemize}




















% \subsubsection{Operacions habituals en les sèries temporals}


% \paragraph{Semblança de dues sèries temporals}


% Similarity Measures for Time Series

% Hi ha varis mètodes, [keogh08:vldb] n'avalua uns quants i els generalitza amb:

% Given two
% time series T1 and T2 , a similarity function Dist calcu-
% lates the distance between the two time series, denoted by
% Dist(T1 , T2 ).

% Exemplifiquem amb la distància euclídia, [keogh08:vldb] nota que és
% competitiva amb les altres.

% Distancia euclídia segons [faloutsous94-sigmod]


% \[
% D(S,Q) = \left( \sum_{i=1}^{l} (S[i]-Q[i])^2  \right)^{1/2}
% \]

% \begin{gather*}
%   D: S \times Q \longrightarrow v: \\
%   S' = map(fusio(S,Q),(t,v_1,v_2)\mapsto(t,(v_1-v_2)^2)), \\
%   S'' = fold(quad,(0,0),(t^1,v^1,t^2,v^2)\mapsto(t^1,v^1+v^2)), \\
%   v = \sqrt{V(m)}:m\in S''
% \end{gather*}


% S i Q haurien de ser regulars entre elles, sinó cal aplicar una fusió amb representació/interpretació.

% Amb la multiresolució la fusió es pot fer de forma eficient. Per altra banda, es podria crear un disc resolució amb agregador de semblança.


% \paragraph{Semblança de dues sèries temporals amb offset}

% Aquí es descriu la solució general del problema (SequentialScan),
% [faloutsous94-sigmod] n'estudia implementacions amb certes
% heurístiques que aconsegueixen més eficiència.





% \paragraph{Filtratge senzill per mitjana mòbil}

% Sigui $p$ la mida de la finestra mòbil
% \begin{gather*}
%   \text{MitMobil}: S \times \text{p} \longrightarrow S':\\
%   \text{map}(S,(t,v)\mapsto \text{mitjanaV}(S[t,t+p]))
% \end{gather*}


% Mitjana mòbil sobre la multiresolució



% \paragraph{Farciment de forats}

% Jo tinc una sèrie temporal i vull que entre dues mesures no hi hagi més d'un cert temps. Si no es compleix dic que té forats. 

% Sigui $S$ una sèrie temporal, aquesta té forats de més durada que $d$
% si alguna mesura compleix $\text{forats}(S,d) = \text{selecciona}(difT(S),v>d \bigwedge v\neq\infty)$ a on $difT(S) = \text{map}(\text{tpredecessors}(S),(t,v)\mapsto(t,t-v))$.

% Amb la multiresolució el farciment de forats és natural a l'estructura i és controlat per la funció agregadora d'atributs.


% * Com farciria els forats manualment a una sèrie temporal?

% 1. Passar-ho per un esquema de multiresolució

% 2. Treballar sobre la sèrie temporal:

% a partir del càlcul de forats anterior $\text{forats}(S,d)$ per
% exemple apliquem un farciment amb representació
% zohe. $\text{farciment}(S,d) = \text{unio}(S,S')$ a on fem la selecció
% de resolució $S' = S[T]^{\text{zohe}}$, $\forall (t,v) \in
% \text{forats}(S,d): T = \{ \tau = t - dn |
% \tau\in(t-v,t),n\in\mathbb{N} \}$.







% \subsubsection{Com treure profit de les operacions dels SGSTM}

% Temes que després es poden aprofitar a les implementacions

% * No hi ha updates --> les sèries temporals no s'han de canviar

% * Per exemple, vull calcular la mitjana de  BDSTM(a,b] si tinc un disc resolució amb $\delta=b-a$ i $f=$mitjana aquest seria l'adequat en comptes de calcular mitjana(SerieTotal(M)(a,b])

% %??
% % No obstant, la base de dades multiresolució conté informació sobre la
% % resolució de les subsèries i per tant aquesta operació és susceptible
% % d'implementar-se aprofitant aquesta informació.  A tall d'exemple es
% % defineix una operació per extreure de la base de dades multiresolució
% % una sèrie temporal regular amb període $T$:


% % \begin{definition}[Selecció de resolució regular]
% %   \begin{gather*}
% %     \text{ResolucióRegular}: M^* \times T \times r \longrightarrow S'\\
% %     \forall (S_{Bi},S_{Di},\delta_i,\tau_i,k_i,f_i) \in M : \\
% %     d_i = T - \delta_i , \\
% %     0 \geq d_0 > d_1 \dots > d_a, 0 < d_{a+1} < \dots < d_d: \\
% %     S'' = S_{D0} || S_{D1} || \dotsb || S_{Da}  ||  S_{Da+1} || \dotsb || S_{Dd}, \\
% %     S' = S''[i]^r: i = {t|0+nT,n\in\mathbb{N}}
% %   \end{gather*}
% % \end{definition}

% % Nota: les operacions no són equivalents, l'operació $\text{SerieTotal}(M)[i]^r$ és molt més potent que la $\text{ResolucióRegular}(M,T)$.




% \subsubsection{Semàntica de comportament}

% \todo{?}









%%% Local Variables:
%%% TeX-master: "main"
%%% End:
% LocalWords:  SGSTM multiresolució

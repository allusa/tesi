\documentclass{scrartcl}
\usepackage[utf8]{inputenc}
\usepackage[catalan]{babel}

\usepackage{url}

\usepackage{todonotes}

\begin{document}
\section*{Informe del doctorand per al dipòsit de la tesi}

\begin{enumerate}
\item DEPARTAMENT/INSTITUT UNIVERSITARI RESPONSABLE DEL PROGRAMA:

Institut  d’Organització i Control de Sistemes Industrials i Departament d'Enginyeria de Sistemes, Automàtica i Informàtica Industrial


\item PROGRAMA QUE HA SEGUIT EL DOCTORAND:

Automàtica, Robòtiva i Visió


\item  TÍTOL DE LA TESI PRESENTADA:


Disseny i modelització d'un sistema de gestió multiresolució per a sèries temporals




\item NOM DEL DOCTORAND QUE PRESENTA LA TESI:

Aleix Llusà Serra

\end{enumerate}





\subsection*{5. ORIGINALITAT DEL TREBALL}


Aquesta recerca té per objectiu l'estudi de les necessitats
específiques que comporta l'emmagatzematge i gestió de dades amb
naturalesa de sèrie temporal i la proposta d'un model de sistema de
gestió de bases de dades (SGBD) que satisfaci aquestes
necessitats. Els punts destacats del treball són els següents:

\begin{itemize}
\item Formalitzem el concepte de multiresolució de sèries temporals,
  com una tècnica d'emmagatzematge compacte de diferents atributs i
  resolucions de les dades. Com a solució de compressió amb pèrdua,
  s'ha de definir un esquema de multiresolució per a cada aplicació.


\item La formalització es basa en la teoria de conjunts i s'inspira
  fortament en la teoria relacional de SGBD.

\item Definim el model de forma genèrica per tal que els usuaris
  puguin configurar-lo segons els requisits d'un context determinat.
  El concepte de funció d'agregació i el de funció temporal de
  representació gràfica d'una sèrie temporal són configurables
  depenent del significat que hagin de tenir en una aplicació
  concreta.

\item El model està preparat perquè pugui ser computat
  incrementalment, seguint una arribada seqüencial d'un flux de dades.
  Incorpora la idea de preveure i de precomputar diversos atributs i
  resolucions de les dades per tal que es puguin visualitzar immediatament
  quan es consultin.

\item La formalització té en compte les irregularitats en el mostratge de
  sèries temporals. A més, les operacions definides són coherents amb
  la dimensió temporal de les sèries temporals. 


\item Reflexionem sobre el problema d'avaluar la qualitat de la
  informació emmagatzemada en un sistema de multiresolució.


\item Explorem vàries necessitats a l'hora d'implementar el
  model. Experimentem amb una computació incremental a partir del flux
  de dades, amb una computació para\l.lela i amb una
  computació de bases de dades relacionals.


\end{itemize}



\subsection*{6. OBJECTIUS ASSOLITS AMB LA TESI EN RELACIÓ AMB ELS PROPOSATS INICIALMENT EN EL PROJECTE DE TESI}


\begin{itemize}

\item Estudi de les aplicacions en què les dades són sèries temporals
  amb la finalitat de determinar quines són les propietats i problemes
  comuns que planteja la seva gestió i emmagatzematge.

\item Estudi dels models de SGBD existents. Segons es desprèn de la
  formalització de Date%\textcite{date:introduction}
  el model principal és el model relacional, el qual es fonamenta en
  dos conceptes: relacions i tipus de dades.


\item Disseny d'un model de SGBD per a les sèries temporals
  multiresolució. El model consisteix en la definició de l'estructura de les
  sèries temporals i les operacions bàsiques que necessiten.
  L'assoliment d'aquest objectiu té tres parts:

  \begin{enumerate}
  \item Disseny d'un model per a la gestió bàsica de les sèries
    temporals, el qual anomenem model de SGBD per a sèries temporals
    (SGST).  
    Prenent com a base el model de SGST, el qual és un model general
    per a les sèries temporals, s'hi poden incloure altres models per
    a propietats més específiques de les sèries temporals.

  \item Disseny d'un model específic en base del model de
    SGST. Concretament es dissenya un model pels SGST multiresolució
    (SGSTM).  En el model de SGSTM s'hi poden incloure propietats de
    les sèries temporals relacionades amb la resolució que s'han
    observat en les aplicacions pràctiques de les sèries temporals:
    regularització, canvis de resolució mitjançant agregacions,
    reconstrucció de forats, etc.
 
  \item Disseny de la metodologia per a definir funcions d'agregació
    d'atributs. Estudi de la relació de les funcions d'agregació
    d'atributs amb la representació gràfica com a funció d'una sèrie
    temporal.


  \end{enumerate}

\item Consideracions sobre la multiresolució.  El model de SGSTM
  proposat està preparat per a rebre contínuament dades en flux
  (\emph{data stream}). Això no obstant, la multiresolució també es
  pot aplicar en temps diferit (\emph{offline}) per a dades
  emmagatzemades. S'ha formulat una funció que permet determinar com
  d'una sèrie temporal s'obté una nova sèrie temporal resultant
  d'haver aplicat un esquema de multiresolució.

    

\item Reflexió sobre la qualitat de la multiresolució. Determinar
  quina selecció i quina pèrdua d'informació té un esquema de
  multiresolució en particular.



\item Implementació de referència dels models de SGST i SGSTM. Per una
  banda, aquesta implementació, a nivell acadèmic, serveix com a
  exemple per a futurs desenvolupaments de sistemes de gestió,
  acadèmics o productius. Per altra banda, serveix per a
  exemplificar-ne els seu funcionament amb unes dades de prova.

\item Implementació específica del model per a una determinada
  aplicació de sèries temporals. Experimentació de com una estructura
  de SGSTM pot ser implementada per a aconseguir una aplicació
  concreta. Aquesta implementació treballa amb conceptes de
  para\l.lelisme.

\item Implementació genèrica del model mitjançant un sistema
  relacional. Aquest experiment avalua a nivell acadèmic les capacitats dels sistemes relacionals per a implementar un SGSTM.




\end{itemize} 




\subsection*{7. METODOLOGIA EMPRADA}

\begin{itemize}
\item Inicialment, ens vam interessar en les aplicacions del sistema
  de gestió de bases de dades RRDtool de Oetiker. Vam notar que tenia
  particularitzacions que en dificultaven la comprensió.  Això va
  motivar a analitzar-lo profundament i a basar el nostre estudi en la
  seva idea de tractament de sèries temporals, que és el que hem
  anomenat multiresolució.


\item Un dels reptes principals ha estat la formalització d'un model
  abstracte matemàtic basant-nos en el model relacional com a teoria
  dels sistemes de gestió de bases de dades. La formalització es basa
  en la teoria de conjunts. Ens hem inspirat fortament en el model
  relacional definit per Codd i divulgat per Date i Darwen,
  especialment la seva proposta \emph{The Third Manifesto}, del qual
  s'han seguit les darreres novetats a través de la llista de
  distribució en què debaten les noves propostes.

\item A partir de la teoria de la informació i la compressió de dades
  amb pèrdua, hem reflexionat sobre com es podrien aplicar anàlisis similars
  per a la multiresolució d'una sèrie temporal per tal d'avaluar-ne la
  qualitat. A causa d'això, hem trobat necessari formular la
  multiresolució com una funció sobre una sèrie temporal que retorna
  una nova sèrie temporal.

\item Un cop formalitzat el nostre model, l'hem utilitzat per a
  avaluar-ne diferents consideracions. Per una banda, el concepte
  d'agregació de \emph{data streams} com el de Cormode, a banda
  d'ajudar-nos a caracteritzar l'aspecte de computació incremental de
  la multiresolució, també ens ha servit per a definir funcions
  d'agregació d'atributs que computin seguint el flux de dades.  Per
  altra banda, a partir del concepte de vistes relacionals i de
  l'arquitectura \emph{Lambda} de Marz, s'ha considerat oportú dissenyar
  sistemes duals de SGST i SGSTM que tinguin en compte altres
  escenaris per a la multiresolució.

\item Seguint el lema de Stonebraker <<one size does not fit all>> per
  als SGBD, s'ha considerat interessant implementar els models amb
  llenguatges de programació i amb sistemes de gestió de bases de
  dades de diversa naturalesa. Així s'han experimentat tres
  implementacions: una implementació de referència amb orientació a
  objectes, un sistema de computació para\l.lela mitjançant la tècnica
  MapReduce i Hadoop, i un sistema de computació relacional mitjançant
  el llenguatge Tutorial D i amb Rel.



% \item Experimentació amb dades. S'ha provat la implementació dels
%   models amb dades experimentals per a una aplicació concreta. S'han
%   aconseguit dades reals gràcies al projecte \emph{i-Sense} (FP7-
%   ICT-270428) que són sèries temporals i amb les quals s'ha pogut
%   experimentar.




\end{itemize}





\subsection*{8. RESULTATS}

\begin{itemize}


\item Report de recerca conjuntament amb la
  doctora Teresa Escobet Canal i el doctor Sebastià Vila Marta,
  publicat el 17 de desembre de 2012 al departament de Disseny i
  Programació de Sistemes Electrònics de la Universitat Politècnica de
  Catalunya. És un report on consten les mancances que tenen els
  SGBD per a les sèries temporals, les propietats i requisits
  que haurien de complir i la idea bàsica de la proposta d'un nou
  model multiresolució per a sèries temporals.

\item Ponència a congrés  conjuntament amb la
  doctora Teresa Escobet Canal i el doctor Sebastià Vila Marta, realitzada
  a l'\emph{International Conference on Artificial Intelligence, Knowledge
  Engineering and Data Bases} (AIKED '13) a Cambridge, UK, els dies
  20--22 de febrer de 2013.  Es dóna a conèixer de forma resumida el
  model multiresolució que dissenyem.


\item Article en revisió a \emph{Information Systems} presentat
  conjuntament amb la doctora Teresa Escobet Canal i el doctor
  Sebastià Vila Marta. Un cop s'ha completat el disseny del model de
  SGSTM, s'ha escrit compactament en format article per a
  l'àmbit de les bases de dades.  En aquest article també s'ha inclòs
  el disseny de la implementació de referència dels model i la
  motivació del treball comparada amb recerques similars.
  Actualment ha passat dos processos de revisió.

\item Les implementacions de les metodologies són  programari lliure
  i es poden trobar a
  \url{http://escriny.epsem.upc.edu/projects/rrb/repository/show/src}. Tot
  i que són experimentals i tenen nivell acadèmic, mostren el
  funcionament correcte del model.

\end{itemize}


\subsection*{9. EL TREBALL DE TESI S’HA REALITZAT EN EL MARC D’ALGUN PROJECTE O CONTRACTE DE RECERCA AMB FINANÇAMENT ESPECÍFIC}


La recerca s'ha dut a terme amb el suport de la Universitat
Politècnica de Catalunya (UPC) mitjançant una beca predoctoral FPU-UPC
adscrita al departament d'Enginyeria del Disseny i Programació de
Sistemes Electrònics (DiPSE).

S'ha participat en el projectes d'investigació subvencionats pel
Ministerio de Economia y Competitividad TEC2012-35571 \emph{Nuevas
  aplicaciones del principio superregenerativo a comunicaciones por
  radiofrecuencia}, DPI2011-26243 \emph{System Health Management and
  Reliable Control of Complex Systems} i DPI2014-58104-R \emph{Control
  basado en la saluda y la resilencia de infraestructuras críticas y
  sistemas complejos}.


\section*{10. ANÀLISI DE L’IMPACTE DELS RESULTATS DE LA TESI I DE LES METODOLOGIES DESENVOLUPADES EN L’ENTORN SOCIO-ECONÒMIC}
  

Actualment és possible d'adquirir una gran quantitat de dades,
principalment gràcies a la facilitat de disposar de sistemes de
monitoratge amb grans xarxes de sensors i també gràcies a
l'abaratiment del maquinari informàtic que hi ha al voltant de la
gestió d'aquesta informació.  Com a conseqüència, han aparegut grans
conjunts de dades digitals per a les quals es requereixen equips
informàtics que les capturin, emmagatzemin, manipulin, cerquin i
visualitzin. Aquesta nova tecnologia ha quedat associada al nom de
\emph{Big Data}.  La gestió d’aquestes dades té aplicació en àmbits
diversos com la mobilitat, l’energia, o les  anomenades
\emph{smart cities}.




En el marc d'aquesta nova situació, la multiresolució permet gestionar
les dades de manera que no cal emmagatzemar totes les que s'hagin
capturat, sinó que permet seleccionar només aquells atributs i
aquelles resolucions que posteriorment hauran de ser consultades.  A
més, la multiresolució restringeix el temps
d'emmagatzematge i permet oferir més resolució a les dades recents i
eliminar dades conforme esdevenen antigues.



El model de sistema de multiresolució resultant admet la computació
incremental seguint el flux d'adquisició d'una sèrie
temporal. D'aquesta manera el temps de computació queda repartit
durant el temps d'adquisició i permet reduir els sistemes de
processament a l'hora d'obtenir els resultats de les consultes.  A
més, aquesta precomputació de les dades permet disposar de
visualitzacions immediates de diversos resums de les dades i per tant
permet observar senzillament  com evolucionen
diversos paràmetres al llarg del temps.

Els sistemes de multiresolució són adequats per a les xarxes de
sensors, en les quals cal que els sistemes tinguin en compte un bon
rendiment en altres recursos a part del temps de computació, com per
exemple la capacitat finita, el consum d'energia o la transmissió per
la xarxa.  Atès que les xarxes de sensors tenen naturalesa
distribuïda, la multiresolució està pensada tant per a computar-se en
un node central que reculli totes les dades com en els mateixos
sensors. D'aquesta darrera manera es redueix la transmissió de dades i
només es transmeten els resums de la informació i les alarmes.  El
model de multiresolució que hem definit és genèric pel que fa a les
funcions d'agregació i per tant es pot adaptar per a qualsevol context
en què s'hagin de gestionar quantitats enormes de dades.


El manteniment dels sistemes de multiresolució és molt petit gràcies a
tres motius principals. Primer, hi ha un emmagatzematge
compacte en què la mida de les dades està afitada. Segon, la
consolidació de les dades es pot aplicar en flux amb
l'adquisició. Tercer, el model preveu que es puguin gestionar algunes
de les irregularitats que els sistemes d'adquisició causin.

Hem desenvolupat una implementació de referència que ha de servir tant
per a experiment acadèmics amb la multiresolució com per a guia de
desenvolupament de nous sistemes productius que vulguin treballar
seguint aquesta tècnica.  Això no obstant, hem notat que les
necessitats actuals de computació són molt variades i cal estudiar una
implementació adequada per a cada context. En aquest sentit, hem
experimentat amb altres possibles implementacions del model per tal
d'observar com la multiresolució es podria adequar a algunes
d'aquestes necessitats.






\end{document}

%  LocalWords:  multiresolució SGST SGSTM

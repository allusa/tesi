

\chapter{Introducció}




% This paper focuses on Data Base Management Systems (DBMS) that store
% and treat data as time series. Traditional DBMS, as is ones derived
% from relational model, are not adequate for these cases as they do not
% have enough facilities to manage and retrieve time series
% information.


% This paper focuses on Data Base Management Systems (DBMS) that store
% and treat data as time series.   Other DBMS are not adequate for these cases as they do not have enough facilities to manage and retrieve time series
% information \parencite{schmidt95}.

% DBMS are based from formal models that define the objects and
% operations of the abstract machine to which users interact, such is
% the relational model \parencite{date}. TSMS lack a consolidated formal
% model, although special properties and requirements for a TSMS
% have been proposed \parencite{dreyer94}.



% In time series analysis there are some common generic operations.
% Most of these operations deal with the time given the nature of data.
% Usual operations include querying time intervals, finding time
% correlations, or calculating distances between two time series. In
% all these operations \acro{TSMS} must consider the temporal coherence
% of the time series.  In the context of statistics, aggregation of time
% series is also a common operation. Aggregate means to summarise a time
% series subset by a smaller set of measures. Statistic indicators like
% the mean, the maximum, or the mode, for instance, summarise time
% series into one only measure.

% In the discrete context, a time series is defined as a set of value
% and time pairs. Furthermore, a time series has a continuous nature as
% it comes from a phenomena evolution along time. As a result,
% \acro{TSMS} operations may deal with this time series nature by
% methods of interpolation or approximation.




% A \acro{TSMS} is a special purpose \acro{DBMS} devoted to store and
% manage time series.  The main objective of \acro{TSMS} is to gather
% two areas of study: time series analysis and \acro{DBMS}.  Time series
% analysis formalises a great amount of algorithms and methodologies
% that apply to time series, with a main focus on improving
% efficiency. \acro{DBMS} theory formalises systems that store and
% operate with data, currently the relational model is the referent
% \cite{date:introduction}.


% A \acro{MTSMS} proposes a \acro{TSMS} with multiresolution
% capabilities.  A \acro{MTSMS} schema represents a time series using a
% set of different resolutions.  The multiresolution concept comes from
% thoroughly analysis of \emph{RRDtool} \cite{rrdtool}. Our objectives
% have been to formalise the main concepts into an abstract model and to
% include more genericity in order to describe \acro{MTSMS} as fully
% \acro{TSMS}.



% * Disseny del model de TSMS, aleshores veurem si una TSMS pot ser implementada com a camp d'una altra DBMS o si els DBMS no són capaços de manipular TS adequadament i cal implementar TSMS específics.




% \subsubsection{Intro de l'estat actual}




% Els SGST actuals bàsicament resolen alguns problemes d'anàlisis de sèries temporals.
% Però no solen atendre la relació entre la base de dades i el sistema de monitoratge, és a dir la manera com s'adquireixen les dades. En aquest pas intermig hi ha un sèrie de problemes, com per exemple forats, dades falses o irregularitat en els temps de mostreig, que cal gestionar correctament. Concretament un dels problemes que no s'atén és el de mostreig irregular ja que es considera que les mostres estan a intervals regulars (equi-espaiades) encara que els sistemes de monitoratge informàtics sovint no són capaços de complir-ho amb exactitud sinó que presenten una certa variació en els temps de mesura. 

% RRDtool n'és una excepció ja que, per ser un sistema productiu, el processament de dades i emmagatzematge és més proper als sistemes de monitoratge. No obstant, està centrat en un tipus de dades particulars, les magnituds i els comptadors, i no té tantes operacions generals per les sèries temporals com els altres SGST.

% També Cougar i TinyDB que exploren l'encaix dels SGBD en entorns distribuïts de xarxes de sensors. Proposen noves estratègies de comunicació amb l'objectiu d'ajustar el consum d'energia. 


% SciQL, un model recent per SGBD  basat en matrius, és el que més es pot considerar com a SGST, ja que s'està desenvolupant per complir-ne algunes propietats.

% \todo{comentar també}
% \cite{dou14:historic_queries_flash_storage}


% On the other hand, compression techniques for time series are
% considered in the form of approximation to the original signal in
% order to compute analysis such as similarity or pattern search
% \cite{fu11,keogh01,last01} or in the form of compression and
% aggregation approaches for massive data streams
% \cite{cormode08:pods,bonnet01}. However, treating time series as data
% streams does not consider adequately the time dimension nor computes
% the evolution of aggregated parameters along time, which is
% interesting for monitoring purposes.  On a similar approach,
% \emph{RRDtool} \cite{rrdtool} is a system that stores time series
% aggregated in different resolutions in order to compact data and to do
% faster visualisations. However, \emph{RRDtool} is very specific and
% has limited aggregation operations to applications of network
% counters.





% \section{Context derivat d'aquesta tesi}


% * report12
% * aiked13
% * if14 en revisió

% * beca FPU-UPC
% * participació projecte NAPS
% * Dades iSENSE





% \section{Estructura del document}

% El model de dades per a sèries temporals es dissenya en el
% capítol~\ref{cap:model:sgst}. El disseny d'aquest model és necessari
% per a comprendre i construir el model multiresolució.  El model de
% dades multiresolució es dissenya en el capítol~\ref{cap:model:sgstm}.





%%% Local Variables: 
%%% mode: latex
%%% TeX-master: "main"
%%% End: 

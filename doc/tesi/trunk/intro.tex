

\chapter{Introducció}




% This paper focuses on Data Base Management Systems (DBMS) that store
% and treat data as time series.   Other DBMS are not adequate for these cases as they do not have enough facilities to manage and retrieve time series
% information \parencite{schmidt95}.

% DBMS are based from formal models that define the objects and
% operations of the abstract machine to which users interact, such is
% the relational model \parencite{date}. TSMS lack a consolidated formal
% model, although special properties and requirements for a TSMS
% have been proposed \parencite{dreyer94}.








\section{Intro de l'estat actual}




Els SGST actuals bàsicament resolen alguns problemes d'anàlisis de sèries temporals.
Però no solen atendre la relació entre la base de dades i el sistema de monitoratge, és a dir la manera com s'adquireixen les dades. En aquest pas intermig hi ha un sèrie de problemes, com per exemple forats, dades falses o irregularitat en els temps de mostreig, que cal gestionar correctament. Concretament un dels problemes que no s'atén és el de mostreig irregular ja que es considera que les mostres estan a intervals regulars (equi-espaiades) encara que els sistemes de monitoratge informàtics sovint no són capaços de complir-ho amb exactitud sinó que presenten una certa variació en els temps de mesura. 

RRDtool n'és una excepció ja que, per ser un sistema productiu, el processament de dades i emmagatzematge és més proper als sistemes de monitoratge. No obstant, està centrat en un tipus de dades particulars, les magnituds i els comptadors, i no té tantes operacions generals per les sèries temporals com els altres SGST.

També Cougar i TinyDB que exploren l'encaix dels SGBD en entorns distribuïts de xarxes de sensors. Proposen noves estratègies de comunicació amb l'objectiu d'ajustar el consum d'energia. 


SciQL, un model recent per SGBD  basat en matrius, és el que més es pot considerar com a SGST, ja que s'està desenvolupant per complir-ne algunes propietats.

\todo{comentar també}
\cite{dou14:historic_queries_flash_storage}


On the other hand, compression techniques for time series are
considered in the form of approximation to the original signal in
order to compute analysis such as similarity or pattern search
\cite{fu11,keogh01,last01} or in the form of compression and
aggregation approaches for massive data streams
\cite{cormode08:pods,bonnet01}. However, treating time series as data
streams does not consider adequately the time dimension nor computes
the evolution of aggregated parameters along time, which is
interesting for monitoring purposes.  On a similar approach,
\emph{RRDtool} \cite{rrdtool} is a system that stores time series
aggregated in different resolutions in order to compact data and to do
faster visualisations. However, \emph{RRDtool} is very specific and
has limited aggregation operations to applications of network
counters.







%%% Local Variables: 
%%% mode: latex
%%% TeX-master: "main"
%%% End: 

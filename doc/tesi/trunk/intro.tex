

\chapter{Introducció}

En els darrers anys, el cost dels sensors de monitoratge s'ha reduït
considerablement i com a conseqüència cada cop n'hi ha més al nostre
voltant.  Per exemple, recentment ha aparegut el concepte de ciutat
inte\l.ligent (\emph{smart city}) --una ciutat plena de sensors i per
tant amb una gran potència de monitoratge-- o bé el concepte de
comptador inte\l.ligent (\emph{smart meter}) --un sensor amb capacitat
d'enregistrar dades, d'analitzar-les i amb sistemes de comunicació
integrats. Així doncs, és possible de recollir grans quantitats de
dades per a monitorar i controlar sistemes complexos. 

Però, cal emmagatzemar totes les dades que aquests sistemes de
monitoratge són capaços de recollir?, cal emmagatzemar-les
indefinidament en el temps? O bé es pot emmagatzemar només la
informació que es desitgi monitorar o controlar i rebutjar la resta?,
de manera que no calgui gestionar un volum de dades massiu que
realment no s'utilitzarà? En aquesta tesi dissertem sobre un mètode
capaç de seleccionar i emmagatzemar una determinada informació de les
dades, com gestionar aquesta informació i fins a quin límit això es pot
aplicar. L'anomenem multiresolució.


Contextualitzem la multiresolució en dos àmbits: les sèries temporals
i els \glspl{SGBD}. En aquesta tesi unim conceptes d'aquest dos àmbits
i proposem solucions per alguns dels problemes que en resulten.



\paragraph{Les sèries temporals}
Les sèries temporals es defineixen com a co\l.lecions d'observacions
d'un mateix fenomen al llarg del temps.  Les sèries temporals són
útils per a formalitzar dades com les que adquireixen els sensors, és
a dir conjunts de valors mesurats en un instant de temps
determinat. Aquestes sèries temporals contenen informació de l'entorn
monitorat i han de ser analitzades per a predir l'evolució dels
sistemes, per a detectar patrons de comportament, per a detectar
anomalies, per a reconstruir els històrics, etc.


La recerca en anàlisi de sèries temporals ha augmentat en la darrera
dècada, tal com explica \textcite{fu11}. Actualment, és un camp de
gran interès, sobretot pel que fa a processar grans volums de dades.
Com es mostra en \textref{cap:estat} d'aquesta memòria, hi ha multitud
de metodologies i algoritmes que proposen solucions a aquests
problemes. Un dels problemes de la gestió de sèries temporals és
conseqüència del fet que són dades voluminoses i per tant són
complicades d'emmagatzemar i de tractar \cite{fu11,keogh08:isax}.
Aquest problema és especialment crític en el disseny de sistemes
integrats petits \cite{yaogehrke02}, els recursos dels quals són
limitats: capacitat, energia, processament i comunicacions.  Una altre
problema ocorre quan les dades no han estat adquirides equi-espaiades
en el temps, ja que molts algoritmes d'anàlisi de sèries temporals ho
requereixen.

  Another problem is that the procedure of
processing and synthesising information becomes difficult if data is
not equi-time spaced.




* Hi ha molts mètodes per a tractar les sèries temporals. Nosaltres les tractarem de forma molt genèrica. Altres mètodes són més particulars, per exemple un senyal d'àudio es pot entendre com una sèrie temporal, car és una variable adquirida al llarg del temps; això no obstant hi ha mètodes més particulars per a tractar amb els senyals d'àudio que els genèrics de sèries temporals atès que aquests senyals solen ser regulars, el timestamps no és important, etc.

* Nosaltres ens ocuparem de les sèries temporals des de la visió més
  genèrica possible; és a dir una sèrie temporal com a conjunt de
  dades que s'han adquirit en uns certs instants de temps.  En aquest
  sentit, les sèries temporals poden representar dades molt variades i
  que pertanyen a àmbits molt diferents. Alguns d'aquests àmbits tenen
  estudis més particulars, focalitzant i simplificant l'estudi en les
  propietats específiques de les dades. Per exemple la teoria del
  senyal es focalitza en dades adquirides regularment i amb components
  de periodicitat o, més particularment encara, l'estudi del so es
  focalitza en una seqüència de valors que han de complir una relació.
  Així doncs, l'estudi genèric de les sèries temporals no pretén
  substituir aquests estudis propis de cada àmbit sinó que pretén
  oferir una visió més àmplia i comuna a totes aquestes dades i oferir
  un estudi per a aquelles dades que no tenen un comportament
  clarament definit. Aquest és el cas, per exemple, de les dades
  adquirides en un monitoratge en entorns no controlats d'una variable
  física: aquestes variables són aleatòries, el temps d'adquisició pot
  ser irregular i per tant cal estudiar-les genèricament com a sèries
  temporals.  Cal dir, que a vegades les sèries temporals es redueixen
  a seqüències temporals; és a dir a estudis de dades on només importa
  l'ordre en què s'han adquirit i el període d'adquisició es
  constant. Nosaltres no fem aquesta reducció sinó que tractem les
  sèries temporals des del punt de vista més genèric on cal saber
  també la posició de temps absoluta que ocupen i la distància de
  temps entre els valors.





\paragraph{Els \gls{SGBD}}

* Dir que plantegem els SGBD per a tractar dades genèriques i una d'aquestes dades poden ser sèries temporals. En el treball plantegem específicament la part que tracta amb les sèries temporals: ho plantegem com un SGBD específic per a sèries temporals. Les metadades de les sèries temporals s'han d'emmagatzemar en un SGBD genèric. Fer un dibuix on és vegi un SGBD i a dins metadades de sèries temporals i un SGST que gestiona les sèries temporals.

En els SGBDR les bitemporal data han tingut molts anys de recerca i ara tenen un estatus ben consolidat. No és així per altres tipus de dades, com per exemple les sèries temporals. Per tant volem elevar les sèries temporals a una consolidació similar.

% This paper focuses on Data Base Management Systems (DBMS) that store
% and treat data as time series.   Other DBMS are not adequate for these cases as they do not have enough facilities to manage and retrieve time series
% information \parencite{schmidt95}.


% DBMS are based from formal models that define the objects and
% operations of the abstract machine to which users interact, such is
% the relational model \parencite{date}. TSMS lack a consolidated formal
% model, although special properties and requirements for a TSMS
% have been proposed \parencite{dreyer94}.

% A \acro{TSMS} is a special purpose \acro{DBMS} devoted to store and
% manage time series.  The main objective of \acro{TSMS} is to gather
% two areas of study: time series analysis and \acro{DBMS}.  Time series
% analysis formalises a great amount of algorithms and methodologies
% that apply to time series, with a main focus on improving
% efficiency. \acro{DBMS} theory formalises systems that store and
% operate with data, currently the relational model is the referent
% \cite{date:introduction}.

En els SGBD es contempla l'estudi formal dels seus conceptes
mitjançant models basats en teories matemàtiques, del qual n'és una
referència el model relacional \parencite{date:introduction}.  Dins
del model relacional hi ha hagut un estudi profund relacionat amb els
intervals temporals, considerats com a dades
temporals \parencite{date02:_tempor_data_relat_model}, que resol el
problema dels històrics temporals en els SGBD. Les sèries temporals
necessiten un estudi similar ja que no poden ser tractades com a
intervals temporals per pertànyer a una categoria diferent de dades
temporals \parencite{assfalg08:thesis,schmidt95}.



\paragraph{Problemes actuals}

Recentment s'ha observat que hi ha necessitat de coneixement entre els
SGBD i les aplicacions de les sèries
temporals \parencite{stonebraker09:scidb,zhang11}.  Hi alguns sistemes
que es poden considerar com a SGST, com per exemple
\emph{RRDtool} \parencite{rrdtool} o \emph{Cougar} \parencite{cougar},
però en ells no s'ha definit clarament un model de SGST.  Dels
treballs publicats en els que es mostra la necessitar d'estudiar
conceptes de model es destaquen el treball de \textcite{dreyer94}, en
el qual presenten l'estructura bàsica que han de tenir els SGST; els
estudis de \textcite{bonnet01} per a xarxes de sensors; i els exemples
de consultes de \textcite{zhang11} per algunes de les propietats de
les sèries temporals.



Rendiment i eficiència

L'eficiència i el rendiment ja no és només calcular en poc temps, sinó que en al
guns contextos pot ser un consum baix d'energia, ocupar poc espai, etc.

És a dir, que hi ha contextos i en cadascun el rendiment vol dir una cosa o una 
altra.


A més, hi ha la computació en flux. Aquesta ens permet anar calcular a mesura que van arribant les dades en comptes d'acumular-ho tot i fer-ho de cop. Si potdem utilitzar un esquema de flux aleshores els càlculs quedem repartits en el temps per la mateixa naturalesa del flux. (per altra banda també estudiaderem càlculs en temps diferent i oferirem una solució amb computació para\l.lela).


També hi ha l'emmagatzematge limitat per a sistemes petits.

Una de les grans mancances en l'àmbit de les sèries temporals és no
tenir un model que faci d'enllaç entre les diferents aplicacions i les
diferents implementacions, el que s'anomena com a necessitat de
coneixement entre els SGBD i les sèries
temporals \parencite{zhang11,stonebraker09:scidb}. Un model permetria,
entre d'altres, estudiar les propietats abstractes de les sèries
temporals i poder comparar diferents sistemes de gestió.
L'estudi d'un model de SGBD per a sèries temporals té dues vessants.

Per una banda, després d'una mirada general a algunes aplicacions de
les sèries temporals, s'observa que hi ha molts algoritmes específics
i dissenyats a mida per al seu tractament. A més, s'observa que en el
tractament de les sèries temporals es duen a terme un conjunt de
tasques comunes: canvis de resolució, reconstrucció de forats,
reducció del volum de dades, etc.  Tot i que hi ha sistemes dissenyats
específicament per a tractar-hi, com per exemple
\emph{RRDtool} \parencite{rrdtool} o \emph{Cougar} \parencite{fung02},
no s'ha identificat un model general per tractar aquests problemes.

Per altra banda, en el model relacional de SGBD es troba una
formalització per a tractar qualsevol tipus de dades. Disposar d'un
model matemàtic consolidat, com és el cas del relacional, ha estat una
fita important en l'àmbit dels SGBD.  No obstant, el model relacional
només presenta els conceptes bàsics dels SGBD, quedant per resoldre la
formalització dels tipus de dades. Concretament dins d'aquest àmbit
destaca la publicació del model per a intervals
temporals \parencite{date02:_tempor_data_relat_model}, el qual
s'utilitza per a formalitzar els històrics en els SGBD.

En l'àmbit dels SGBD també s'hi inclou l'estudi d'implementacions que
permetin obtenir un bon rendiment. Un cop s'ha formalitzat el model es
poden avaluar les implementacions més adequades per a cada
aplicació. Per exemple en el cas de les sèries temporals es pot
aplicar la gestió mitjançant \emph{data
  streams} \parencite{babcock02}.



Time series can be stored and managed by Structured Query Language
(\acro{SQL}) relational database management systems. However, some
authors \cite{dreyer94,schmidt95,stonebraker09:scidb,zhang11} notice
that the use of \acro{SQL} systems as a time series backend suffers
some drawbacks.  On the one hand, \emph{NoSQL} or \emph{NewSQL}
products are being developed in order to increase the performance and
flexibility of \acro{SQL} systems
\cite{atzeni13:relational_model_dead,stonebraker10,stonebraker09:scidb,zhang11},
however the continue acquisition nature of time series is an issue for
storing and analysing offline all the data \cite{keogh97}.


On the other hand, compression techniques for time series are
considered in the form of approximation to the original signal in
order to compute analysis such as similarity or pattern search
\cite{fu11,keogh01,last01} or in the form of compression and
aggregation approaches for massive data streams
\cite{cormode08:pods,bonnet01}. However, treating time series as data
streams does not consider adequately the time dimension nor computes
the evolution of aggregated parameters along time, which is
interesting for monitoring purposes.  On a similar approach,
\emph{RRDtool} \cite{rrdtool} is a system that stores time series
aggregated in different resolutions in order to compact data and to do
faster visualisations. However, \emph{RRDtool} is very specific and
has limited aggregation operations to applications of network
counters.




% Els SGST actuals bàsicament resolen alguns problemes d'anàlisis de sèries temporals.
% Però no solen atendre la relació entre la base de dades i el sistema de monitoratge, és a dir la manera com s'adquireixen les dades. En aquest pas intermig hi ha un sèrie de problemes, com per exemple forats, dades falses o irregularitat en els temps de mostreig, que cal gestionar correctament. Concretament un dels problemes que no s'atén és el de mostreig irregular ja que es considera que les mostres estan a intervals regulars (equi-espaiades) encara que els sistemes de monitoratge informàtics sovint no són capaços de complir-ho amb exactitud sinó que presenten una certa variació en els temps de mesura. 

% RRDtool n'és una excepció ja que, per ser un sistema productiu, el processament de dades i emmagatzematge és més proper als sistemes de monitoratge. No obstant axiò, està centrat en un tipus de dades particulars, les magnituds i els comptadors, i no té tantes operacions generals per les sèries temporals com els altres SGST.

% També Cougar i TinyDB que exploren l'encaix dels SGBD en entorns distribuïts de xarxes de sensors. Proposen noves estratègies de comunicació amb l'objectiu d'ajustar el consum d'energia. 


% SciQL, un model recent per SGBD  basat en matrius, és el que més es pot considerar com a SGST, ja que s'està desenvolupant per complir-ne algunes propietats.

% \todo{comentar també}
% \cite{dou14:historic_queries_flash_storage}


% On the other hand, compression techniques for time series are
% considered in the form of approximation to the original signal in
% order to compute analysis such as similarity or pattern search
% \cite{fu11,keogh01,last01} or in the form of compression and
% aggregation approaches for massive data streams
% \cite{cormode08:pods,bonnet01}. However, treating time series as data
% streams does not consider adequately the time dimension nor computes
% the evolution of aggregated parameters along time, which is
% interesting for monitoring purposes.  On a similar approach,
% \emph{RRDtool} \cite{rrdtool} is a system that stores time series
% aggregated in different resolutions in order to compact data and to do
% faster visualisations. However, \emph{RRDtool} is very specific and
% has limited aggregation operations to applications of network
% counters.







\paragraph{La multiresolució}



% A \acro{MTSMS} proposes a \acro{TSMS} with multiresolution
% capabilities.  A \acro{MTSMS} schema represents a time series using a
% set of different resolutions.  The multiresolution concept comes from
% thoroughly analysis of \emph{RRDtool} \cite{rrdtool}. Our objectives
% have been to formalise the main concepts into an abstract model and to
% include more genericity in order to describe \acro{MTSMS} as fully
% \acro{TSMS}.


% In time series analysis there are some common generic operations.
% Most of these operations deal with the time given the nature of data.
% Usual operations include querying time intervals, finding time
% correlations, or calculating distances between two time series. In
% all these operations \acro{TSMS} must consider the temporal coherence
% of the time series.  In the context of statistics, aggregation of time
% series is also a common operation. Aggregate means to summarise a time
% series subset by a smaller set of measures. Statistic indicators like
% the mean, the maximum, or the mode, for instance, summarise time
% series into one only measure.



* La multiresolució implica una pèrdua d'informació i per tant l'usuari n'ha d'estar al cas, és a dir que no pot ser que un sistema autònomament decideixi emmagatzemar una sèrie temporal amb multiresolució sense avisar a l'usuari. És més, l'usuari és el responsable de decidir un esquema de multiresolució adequat al context en què el vulgui aplicar.  Aquest és el motiu principal per a dissenyar un model dels sistemes de multiresolució: perquè tenen un comportament diferent a altres sistemes d'emmagatzematge i per tal d'explicar bé a un usuari com els ha de fer funcionar.


* Bàsicament hi ha dues aproximacions per a fer la multiresolució: Processar cad
a resolució independentment i en para\l.lel (tal com fa RRDtool) o bé processar 
les resolucions encadenades, és a dir a mesura que les dades es fan velles salte
n de resolució. El model que presentem és genèric per a tots dos casos.








\paragraph{RRDtool}



Hi ha RRDtool, que és un SGBD específic dissenyat per a dades monitorades. Les c
auses del seu disseny són:

* Tobias Oetiker dissenyava un monitor de paràmetres de xarxes de comunicacions i en aquest monitor una part era la d'emmagatzematge de les dades. Per raons pràctiques i d'utilitat dissenya aquesta part amb un esquema inovadós. Finalment acaba separant aquesta part i la converteix independentment en RRDtool.

* RRDtool té aquest model pràctic i a la pràctica és molt útil per a ser usat com a SGBD dels sistemes de monitoratge, sobretot en l'àmbit dels comptadors de xarxa on és l'estàndard de facto. 

Això no obstant, no hi ha cap raonament teòric sobre el model de RRDtool ja que s'ha dissenyat per raons pràctiques. Per tant, entendre el funcionament de RRDtool és complicat, hi ha un nivell molt elevat per començar a fer-lo funcionar i molts conceptes no s'entenen perquè no estan ben definits. 

Per això ens proposem de compendre i formalitzar el model de RRDtool, que acabarem anomenar model de multiresolució, en la teoria dels sitemes d'informació. A més RRDtool és molt específic pel camp de comptadors de xarxa i volem oferir un model genèric per a altres àmbits.  



% * Disseny del model de TSMS, aleshores veurem si una TSMS pot ser implementada com a camp d'una altra DBMS o si els DBMS no són capaços de manipular TS adequadament i cal implementar TSMS específics.




% \subsubsection{Intro de l'estat actual}






\section{Objectius}


\todo{revisar}

Aquesta recerca té per objectiu l'estudi de les necessitats
específiques que comporta l'emmagatzematge i gestió de dades amb
naturalesa de sèrie temporal i la proposta d'un model de SGBD que
satisfaci aquestes necessitats. Aquest objectiu es divideix en els
següents subobjectius més concrets:

\begin{itemize}

\item Estudi de les aplicacions en que les dades són sèries temporals
  amb la finalitat de determinar quines són les propietats i problemes
  comuns que planteja la seva gestió i emmagatzematge.

\item Estudi dels models de SGBD existents. Segons es desprèn de la
  formalització de Date%\textcite{date:introduction}
  el model principal és el model relacional, el qual es fonamenta en
  dos conceptes: relacions i tipus de dades.

\item Una àrea de treball important en els SGBD és la incorporació de
  nous tipus de dades complexos. És important estudiar com es modifica
  el model de dades d'un SGBD quan s'afegeix un nou tipus de dades
  complex.  Les sèries temporals es poden d'entendre com a tipus
  complex ja que presenten diferents propietats característiques i
  necessiten operadors addicionals.  

\item Disseny d'un model de SGBD per a les sèries temporals. D'aquesta
  manera els SGBD podran tractar dades amb instants de temps que
  mostrin l'evolució de variables en funció del temps. El model
  consisteix en la definició de l'estructura de les sèries temporals i
  les operacions bàsiques que necessiten.

  L'assoliment d'aquest objectiu té tres parts:

  \begin{enumerate}
  \item Disseny d'un model per a la gestió bàsica de les sèries
    temporals, el qual anomenem model de SGBD per a sèries temporals
    (SGST).  L'estructura d'aquest model és similar a l'utilitzat en
    els intervals
    temporals. %\parencite{date02:_tempor_data_relat_model}.
    Prenent com a base el model de SGST, el qual és un model general
    per a les sèries temporals, s'hi poden incloure altres models per
    a propietats més específiques de les sèries temporals.

  \item Disseny d'un model específic en base del model de
    SGST. Concretament es dissenya un model pels SGST multiresolució
    (SGSTM).  En el model de SGSTM s'hi poden incloure propietats de
    les sèries temporals relacionades amb la resolució que s'han
    observat en les aplicacions pràctiques de les sèries temporals:
    regularització, canvis de resolució mitjançant agregacions,
    reconstrucció de forats, etc.



 \item Avaluació dels esquemes de multiresolució.  El model de SGSTM
    proposat està preparat per a rebre contínuament dades en flux
    (\emph{data stream}). Això no obstant, la multiresolució també es
    pot aplicar en temps diferit (\emph{offline}) per a dades
    emmagatzemades. Cal una funció que permeti determinar com d'una
    sèrie temporal s'obté una nova sèrie temporal resultant d'haver
    aplicat un esquema de multiresolució. A més, cal determinar quina
    selecció i quina pèrdua d'informació té un esquema de
    multiresolució en particular.

  \end{enumerate}

\item Implementació de referència dels models de SGST i SGSTM. Per una
  banda, aquesta implementació, a nivell acadèmic, ha de servir com a
  exemple per a futurs desenvolupaments de sistemes de gestió,
  acadèmics o productius. Per altra banda, ha de servir per a
  exemplificar-ne els seu funcionament amb unes dades de prova.

\item Implementació específica del model per a una determinada
  aplicació de sèries temporals. Exemplificació de com una estructura
  de SGSTM pot ser implementada per a aconseguir una aplicació
  concreta. Aquesta implementació treballa amb conceptes de
  para\l.lelisme.

\end{itemize} 


% item[Tasca 5.b] Recerca en els esquemes de multiresolució.  A partir
%   de la multiresolució expressada com una funció simplificada, que
%   calcula en temps diferit, convé explorar noves recerques. En primer
%   lloc, cal avaluar formalment aquesta funció de multiresolució
%   respecte a tot el model proposat de SGSTM. En segon lloc, en aquest
%   càlcul en temps diferit apareix la possibilitat de computar
%   para\l.lelament. En tercer lloc, a partir d'aquesta funció pot ésser
%   més senzill estudiar la selecció o pèrdua d'informació que implica
%   usar un model de multiresolució.


% \item[Tasca 6.b] S'ha començat a dissenya una nova implementació de la
%   multiresolució en Hadoop basada en conceptes de computació
%   para\l.lela en els SGBD. Un cop ben establert el model d'aquest
%   esquema de multiresolució cal acabar d'implementar-lo i provar-lo
%   amb les mateixes dades experimentals que s'usen per a la
%   implementació de referència.

% \item[Tasca 7.] Experimentació amb dades. Es disposa d'unes dades
%   noves amb irregularitats interessants per a aplicar-hi el model de
%   multiresolució. Són unes dades de sensors reals en un sistema de
%   dipòsits d'aigües. Aquestes dades pertanyen a sensors
%   interrelacionats, cosa que les fa més interessants per analitzar-hi
%   l'efecte que hi un SGSTM i com pot ajudar a emmagatzemar-les de
%   forma compacta.


% \item[Tasca 9.] La nova tasca 5.b plantejada ofereix un nou tema en
%   l'àmbit de la multiresolució que, si l'experimentació que hem
%   plantejat és satisfactòria, pot conduir a la redacció d'un nou
%   article. Aquesta redacció, però, s'allargaria més enllà de l'entrega
%   de la tesi.




\subsection{Contribució d'aquesta tesi}




Publicacions realitzades:

\begin{itemize}

\item Report de recerca \parencite{llusa12:report} conjuntament amb
  la Teresa Escobet-Canal i el Sebastià Vila-Marta, publicat el 17 de
  desembre de 2012 al departament de Disseny i Programació de
  Sistemes Electrònics de la Universitat Politècnica de Catalunya.
 informe de
  recerca a on constés les mancances que observem que tenen els SGBD
  per a les sèries temporals, les propietats i requisits que han de
  complir i la idea bàsica de la nostra proposta d'un nou model
  multiresolució per a sèries temporals. 

\item Ponència a congrés \parencite{llusa13:aiked} conjuntament amb la
  Teresa Escobet-Canal i el Sebastià Vila-Marta, realitzada al
  ``International Conference on Artificial Intelligence, Knowledge
  Engineering and Data Bases'' (AIKED '13) a Cambridge (UK) els dies
  20--22 de febrer de 2013.
 donar a conèixer de
  forma resumida el model multiresolució que volem dissenyar.

\end{itemize}



% * report12
% * aiked13
% * if14 en revisió
Redacció d'articles i informes. Un cop s'ha completat
  el disseny del model de SGBD per a sèries temporals, s'ha escrit
  compactament en format article per a l'àmbit de les bases de dades.
  En aquest article també s'ha inclòs el disseny de la implementació
  de referència dels model.  Aquesta tasca s'ha allargat més del
  previst: ha calgut fer una tasca intensa de recerca de revistes i
  congressos que siguin afins a l'àmbit on treballem, cosa que també
  ha causat una gran feina de motivació i de recerca de treballs
  relacionats en l'actualitat de la temàtica seleccionada.  L'article
  s'ha escrit conjuntament amb la Teresa Escobet Canal i el Sebastià
  Vila Marta i està pendent per presentar-lo a la revista
  \emph{Information Systems}.


% * beca FPU-UPC
% * participació projecte NAPSAprovada la concessió del projecte d'investigació
%   TEC2012-35571 ''Nuevas aplicaciones del principio superregenerativo
%   a comunicaciones por radiofrecuencia (NASP)'' per als pròxims 3 anys
%   en finalitzar l'anterior projecte AVIC. Investigador principal Pere
%   Palà Schönwälder.

% * Dades iSENSE




% S'ha estat subscrit a les llistes de debat de:

% * RRDtool
% * thirdmanifesto

% que és on aquesta gent discuteix els seus problemes



% S'ha desenvolupat amb control de versions a

% S'ha desenvolupat programari lliure, disponible a , amb manuals de documentació disponibles a


\section{Estructura del document}

% El model de dades per a sèries temporals es dissenya en el
% capítol~\ref{cap:model:sgst}. El disseny d'aquest model és necessari
% per a comprendre i construir el model multiresolució.  El model de
% dades multiresolució es dissenya en el capítol~\ref{cap:model:sgstm}.






%\subsection{Formats i qüestions lingüístiques}

%Algun format en especial? els codis python, els fitxers?

%Algunes qüestions lingüístiques remarcables?



%%% Local Variables: 
%%% mode: latex
%%% TeX-master: "main"
%%% End: 

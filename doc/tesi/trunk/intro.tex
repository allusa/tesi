

\chapter{Introducció}
\label{sec:intro}

En els darrers anys, el cost dels sensors de monitoratge s'ha reduït
considerablement i com a conseqüència cada cop n'hi ha més al nostre
voltant.  Per exemple, recentment ha aparegut el concepte de ciutat
inte\l.ligent (\emph{smart city}) --una ciutat plena de sensors i per
tant amb una gran potència de monitoratge-- o bé el concepte de
comptador inte\l.ligent (\emph{smart meter}) --un sensor amb capacitat
d'enregistrar dades, d'analitzar-les i amb sistemes de comunicació
integrats. Així doncs, és possible de recollir grans quantitats de
dades per a monitorar i controlar sistemes complexos. 

Però, cal emmagatzemar totes les dades que aquests sistemes de
monitoratge són capaços de recollir?, cal emmagatzemar-les
indefinidament en el temps? O bé es pot emmagatzemar només la
informació que es desitgi monitorar o controlar i rebutjar la resta?,
de manera que no calgui gestionar un volum de dades massiu que
realment no s'utilitzarà? En aquesta tesi dissertem sobre un mètode
capaç de seleccionar i emmagatzemar una determinada informació de les
dades, com gestionar aquesta informació i fins a quin límit això es pot
aplicar. L'anomenem multiresolució.


Contextualitzem la multiresolució en dos àmbits: les sèries temporals
i els \glspl{SGBD}. En aquesta tesi unim conceptes d'aquest dos àmbits
i proposem solucions per alguns dels problemes que en resulten.



\paragraph{Les sèries temporals}
Les sèries temporals es defineixen com a co\l.lecions d'observacions
d'un mateix fenomen al llarg del temps.  Les sèries temporals són
útils per a formalitzar dades com les que adquireixen els sensors, és
a dir conjunts de valors mesurats en un instant de temps
determinat. Aquestes sèries temporals contenen informació de l'entorn
monitorat i han de ser analitzades per a predir l'evolució dels
sistemes, per a detectar patrons de comportament, per a detectar
anomalies, per a reconstruir els històrics, etc.


La recerca en anàlisi de sèries temporals ha augmentat en la darrera
dècada, tal com explica \textcite{fu11}. Actualment, és un camp de
gran interès, sobretot pel que fa a processar grans volums de dades.
Com es mostra en \textref{cap:estat} d'aquesta memòria, hi ha multitud
de metodologies i algoritmes que proposen solucions a aquests
problemes. Un dels problemes de la gestió de sèries temporals és
conseqüència del fet que són dades voluminoses i per tant són
complicades d'emmagatzemar i de tractar \cite{fu11,keogh08:isax}.
Aquest problema és especialment crític en el disseny de sistemes
integrats petits \cite{yaogehrke02}, els recursos dels quals són
limitats: capacitat d'emmagatzematge, consum d'energia, temps de
processament i capacitat de les comunicacions.  Una altre problema
ocorre quan les dades no han estat adquirides equi-espaiades en el
temps, ja que alguns algoritmes d'anàlisi de sèries temporals ho
requereixen.

Contextualitzem el nostre estudi de les sèries temporals en l'àmbit
del monitoratge en què les variables adquirides són aleatòries, el
temps d'adquisició pot ser irregular, etc.  En l'anàlisi de sèries
temporals a vegades reben estudis més particulars, focalitzant i
simplificant l'estudi en algunes propietats específiques de les
dades. Per exemple la teoria del senyal es focalitza en dades
adquirides regularment i amb components de periodicitat o bé a vegades
les sèries temporals es redueixen a seqüències temporals en què només
importa l'ordre en què s'han adquirit i el període d'adquisició es
constant. El nostre estudi tracta les sèries temporals des del punt de
vista genèric temporal en què, a més, cal saber la posició de temps
absoluta que ocupa cada dada i la distància de temps entre els valors.






\paragraph{Els SGBD}
Els \gls{SGBD} són els sistemes informàtics encarregats d'emmagatzemar
i gestionar dades de forma genèrica, la qual cosa s'inscriu en l'àmbit
dels sistemes d'informació. Els conceptes dels \gls{SGBD} es basen en
models matemàtics formals que defineixen l'estructura dels objectes i
les operacions que els usuaris perceben de forma abstracta. El model
formal de referència dels \gls{SGBD} és el model
relacional \parencite{date04:introduction8}.


Els \gls{SGBD} habituals fins a l'actualitat han estat els que
disposen d'un llenguatge de consultes anomenat \gls{SQL}, tot i que en
els darrers anys han aparegut altres sistemes, anomenats \emph{NoSQL}
i \emph{NewSQL}, per tal de millorar el rendiment i la flexibilitat
dels
\gls{SQL} \parencite{atzeni13:relational_model_dead,stonebraker10,stonebraker09:scidb,zhang11}. Alguns
autors \parencite{dreyer94,schmidt95,stonebraker09:scidb,zhang11}
noten els aspectes en què cal millorar els \gls{SGBD} per tal de
tractar les sèries temporals i fan èmfasi en el fet que és necessari
un model que enllaci el coneixement dels \gls{SGBD} amb el de
l'anàlisi de sèries temporals. Sobretot, l'adquisició contínua de nous
valors és el un repte a l'hora d'emmagatzemar i analitzar les sèries
temporals \parencite{keogh97}. 

% Dels
% treballs publicats en els que es mostra la necessitar d'estudiar
% conceptes de model es destaquen el treball de \textcite{dreyer94}, en
% el qual presenten l'estructura bàsica que han de tenir els SGST; els
% estudis de \textcite{bonnet01} per a xarxes de sensors; i els exemples
% de consultes de \textcite{zhang11} per algunes de les propietats de
% les sèries temporals.


També cal que els \gls{SGBD} gestionin adequadament i amb coherència
l'atribut temporal de les sèries temporals.  En els \gls{SGBD} també
es gestionen històrics temporals, és a dir l'evolució de les dades al
llarg del temps. Aquest és un problema similar a les sèries temporals
pel que fa a que tots dos treballen amb atributs temporals, tot i així
pertanyen a dues categories diferents de dades i no poden ser tractats
de la mateixa manera \parencite{assfalg08:thesis,schmidt95}.  Ara bé,
els històrics temporals han estat llargament estudiats en els
\gls{SGBD} i finalment s'han formalitzat com a intervals temporals
dins del model relacional \parencite{date02:_tempor_data_relat_model}.
Les sèries temporals necessiten consolidar un model similar en què se
n'estudiïn les propietats i els requisits
específics \parencite{dreyer94,segev87:sigmod}.




En aquest treball estudiem els \gls{SGBD} que treballen
específicament amb sèries temporals, aleshores els anomenem
\glspl{SGST}. De la mateixa manera, dissenyem \gls{SGST} amb
capacitats de multiresolució, els quals anomenem
\glspl{SGSTM}. Formulem el model dels \glspl{SGSTM} per la qual cosa
necessitem també formular el model dels \gls{SGST}.




%Les metadades de les sèries temporals s'han d'emmagatzemar en un SGBD genèric. Fer un dibuix on és vegi un SGBD i a dins metadades de sèries temporals i un SGST que gestiona les sèries temporals.




\paragraph{La multiresolució}
La multiresolució resumeix la informació d'una sèrie temporal
mitjançant un conjunt de resolucions. Cada resolució correspon a un
atribut i a un període de temps de la sèrie temporal. El concepte de
multiresolució prové d'un estudi profund d'un sistema anomenat
RRDtool \parencite{rrdtool}. El nostre objectiu és descriure els
conceptes principals de la multiresolució de manera abstracta per a
obtenir un model formal dels \gls{SGSTM}.

La multiresolució implica una selecció d'informació i, per tant,
alhora implica una pèrdua d'informació. Com a conseqüència, l'usuari
ha de determinar un esquema de multiresolució per a cada sèrie
temporal que vulgui gestionar amb un \gls{SGSTM}. A partir del model
dels \gls{SGSTM} reflexionarem sobre com gestionen la informació i
sobre quins efectes tenen diferents esquemes de multiresolució.
%l'usuari n'ha d'estar al cas, és a dir que no pot ser que un sistema autònomament decideixi emmagatzemar una sèrie temporal amb multiresolució sense avisar a l'usuari. 


El model de \gls{SGSTM} es dissenya de forma genèrica per a permetre'n
implementacions vàries. Així, hi ha vàries aproximacions possibles per
a computar la multiresolució: limitar l'emmagatzematge pensant en
sistemes petits, precomputar el resultat per a disposar-ne
visualitzacions immediates, acumular totes les dades i processar-les
en temps diferit aprofitant computació para\l.lela, repartir el temps
de computació durant el mateix procés d'adquisició, etc. En aquest
document explorarem aquestes possibilitats dels \gls{SGSTM}.



% \todo{refer} La multiresolució és una tècnica de compressió amb pèrdua
% de les sèries temporals. Hi altres tècniques de compressió que
% s'apliquen a les sèries temporals. D'una banda hi ha tècniques
% d'aproximació al senyal original per a analitzar similituds entre
% sèries temporals o cercar-hi patrons \parencite{fu11,keogh01,last01}.
% D'altra banda hi ha tècniques per a comprimir i aproximar amb
% agregacions per a fluxos de dades
% massius \parencite{cormode08:pods,bonnet01}.  També hi ha tècniques
% d'emmagatzematge massiu
% \parencite{opentsdb,zhang11,stonebraker09:scidb} en què les sèries
% temporal en què s'aprofiten altres tècniques de \gls{SGBD} per a grans
% volums de dades. %OpenTSDB SciQL SciDB


% Això no obstant, aquestes tècniques o bé són massa genèriques i no
% consideren adequadament la dimensió temporal, o bé no analitzen
% l'evolució dels paràmetres al llarg del temps, o bé no atenen la
% relació entre els \gls{SGBD} i els sistemes de monitoratge, la qual
% pot causar alguns problemes --irregularitats en el temps de mostreig,
% dades falses, forats sense dades, etc.-- que cal gestionar
% adequadament.

% Alguns sí que ho tenen en compte com Cougar \parencite{fung02} 
% % També Cougar i TinyDB que exploren l'encaix dels SGBD en entorns distribuïts de xarxes de sensors. Proposen noves estratègies de comunicació amb l'objectiu d'ajustar el consum d'energia. 
% però molt particularitzat per les xarxes de sensors, o \textcite{dou14:historic_queries_flash_storage} però molt particularitzat per a l'emmagatzematge en memòries flash, o bé RRDtool que
% és una implementació de la multiresolució però molt específica i
% limitada a l'àmbit dels comptadors de xarxa. De fet, un dels objectius
% del model de multiresolució que formulem és oferir agregadors genèrics
% i resums per a qualsevol sèrie temporal.











\section{Contribució d'aquesta tesi}


A continuació resumim les principals contribucions d'aquesta tesi:

\begin{itemize}

\item Un \gls{SGSTM} emmagatzema les sèries temporals de forma
  comprimida. És un resum de l'evolució dels atributs de la sèrie
  temporal al llarg del temps i amb diferents períodes temporals. És
  una solució de compressió amb pèrdua i per tant l'esquema de
  multiresolució s'ha de decidir prèviament per a cada context.

\item Per a resumir cada atribut de la sèrie temporal s'utilitzen
  funcions d'agregació i funcions de representació, els quals es
  formalitzen com a objectes independents en el model. D'aquesta
  manera, els usuaris poden definir diferents operadors considerant la
  semàntica de les sèries temporals en cada context diferent.  Per
  exemple, els atributs es poden calcular mitjançant funcions
  d'agregació d'estadístics com la mitjana o el màxim.

\item El model opera coherentment amb la dimensió temporal de les
  sèries temporals. A més, té en compte les irregularitats del
  mostreig, el tractament i validació de dades i diverses
  interpretacions de les sèries temporals

\item El model es basa fermament en l'àlgebra de conjunts i la del
  model relacional com a teoria formal dels sistemes d'informació.

\item Es tenen en compte les diverses possibilitats de computació de
  la multiresolució i es desenvolupen diverses implementacions del
  model. Es desenvolupa una implementació de referència, una per a
  computació para\l.lela i una amb llenguatge acadèmic relacional.


\item S'introdueix el problema d'avaluar la qualitat de la
  multiresolució. És a dir el problema de determinar quina selecció i
  quina pèrdua d'informació hi ha quan s'aplica la multiresolució a
  una sèrie temporal i com realitzar consultes aproximades a la
  informació original.


\end{itemize}






Aquesta tesi ha derivat en altres publicacions, que detallem a continuació:

\begin{itemize}

\item Report de recerca \parencite{llusa12:report} conjuntament amb la
  doctora Teresa Escobet Canal i el doctor Sebastià Vila Marta,
  publicat el 17 de desembre de 2012 al departament de Disseny i
  Programació de Sistemes Electrònics de la Universitat Politècnica de
  Catalunya. És un report on consten les mancances que tenen els
  \gls{SGBD} per a les sèries temporals, les propietats i requisits
  que haurien de complir i la idea bàsica de la proposta d'un nou
  model multiresolució per a sèries temporals.

\item Ponència a congrés \parencite{llusa13:aiked} conjuntament amb la
  doctora Teresa Escobet Canal i el doctor Sebastià Vila Marta, realitzada
  a l'\emph{International Conference on Artificial Intelligence, Knowledge
  Engineering and Data Bases} (AIKED '13) a Cambridge, UK, els dies
  20--22 de febrer de 2013.  Es dóna a conèixer de forma resumida el
  model multiresolució que dissenyem.


\item Article pendent d'acceptació presentat conjuntament amb la
  doctora Teresa Escobet Canal i el doctor Sebastià Vila Marta. Un cop
  s'ha completat el disseny del model de \gls{SGSTM}, s'ha escrit
  compactament en format article per a l'àmbit de les bases de dades.
  En aquest article també s'ha inclòs el disseny de la implementació
  de referència dels model i la motivació del treball comparada amb
  recerques similars.
  % està pendent a la revista \emph{Information Systems}.

\item Les implementacions són de programari lliure i es poden trobar a
  \url{http://escriny.epsem.upc.edu/projects/rrb/repository/show/src}. Tot
  i que són experimentals i tenen nivell acadèmic, mostren el
  funcionament correcte del model.

\end{itemize}





% * beca FPU-UPC
% * participació projecte NAPSAprovada la concessió del projecte d'investigació
%   TEC2012-35571 ''Nuevas aplicaciones del principio superregenerativo
%   a comunicaciones por radiofrecuencia (NASP)'' per als pròxims 3 anys
%   en finalitzar l'anterior projecte AVIC. Investigador principal Pere
%   Palà Schönwälder.

% * Dades iSENSE




% S'ha estat subscrit a les llistes de debat de:

% * RRDtool
% * thirdmanifesto

% que és on aquesta gent discuteix els seus problemes



% S'ha desenvolupat amb control de versions a

% S'ha desenvolupat programari lliure, disponible a , amb manuals de documentació disponibles a


\section{Estructura del document}


Aquest document és el resultat de la recerca en un
model de multiresolució per a les sèries temporals. S'estructura en
cinc parts principals.

En una primera part, que inclou aquest \autoref{sec:intro}
d'introducció, es presenta el context i els objectius de la
recerca. En el \autoref{cap:estat} es descriu l'estat actual de la
recerca i treballs similars.

En una segona part es dissenya i es formula el model.  En el
\autoref{sec:model} s'introdueixen els conceptes de model de
\gls{SGBD} i l'abast de les sèries temporals i la multiresolució que
utilitzem.  El model es dissenya en dues parts: el model dels
\gls{SGST} en el \autoref{cap:model:sgst} i el model dels \gls{SGSTM}
en el \autoref{cap:model:sgstm}.


En una tercera part s'exploren variacions i reflexions sobre el model
presentat. En el \autoref{sec:variacions} s'introdueixen les
consideracions que es faran sobre el model i s'exploren petites
variacions en l'emmagatzematge de les sèries temporals. En el
\autoref{cap:funciomultiresolucio} es formula la multiresolució com
una funció que a partir d'una sèrie temporal resulta en una nova sèrie
temporal o conjunts de sèries temporals comprimides. En el
\autoref{sec:multiresolucio:dual} es dissenyen sistemes que utilitzin
alhora \gls{SGST} i \gls{SGSTM}.  En el
\autoref{sec:multiresolucio:teoriainformacio} es reflexiona sobre el
problema de la qualitat de la multiresolució, és a dir s'avalua quina
selecció d'informació fa un determinat esquema de multiresolució.



En una quarta part s'experimenta i es dissenyen les implementacions
del model. En el \autoref{sec:implementacions} s'introdueixen les
diferents implementacions i es comenten les particularitats de cada
una. En el \autoref{sec:implementacio:python} es dissenya la
implementació de referència amb llenguatge Python. En el
\autoref{sec:implementacio:mapreduce} es dissenya una implementació
amb computació distribuïda i para\l.lela amb la tècnica MapReduce i en
el sistema Hadoop. En el \autoref{sec:implementacio:relacional} es
dissenya una implementació amb el llenguatge acadèmic relacional
Tutorial~D. En el \autoref{sec:implementacions:exemple} s'exemplifica
l'ús de les implementacions amb dades reals.


En una cinquena part es conclou el document. Particularment en el
\autoref{sec:conclusions} es resumeix la dissertació i s'expressen les
conclusions que se'n poden treure, i en el \autoref{sec:futur} es
proposen altres investigacions que serien interessants a partir
d'aquesta recerca.




Finalment, com a materials de referència es recull la bibliografia
citada, les abreviacions i la nomenclatura utilitzades i s'ofereixen
índexs per a les entitats remarcables --figures, llistats, definicions
i exemples-- les quals numerem precedides del número de secció per a
facilitar-ne la localització.



% \todo{hi ha algun annex?}


%\subsection{Formats i qüestions lingüístiques}

%Algun format en especial? els codis python, els fitxers?

%Algunes qüestions lingüístiques remarcables?









% \section{Objectius}

% Aquesta recerca té per objectiu l'estudi de les necessitats
% específiques que comporta l'emmagatzematge i gestió de dades amb
% naturalesa de sèrie temporal i la proposta d'un model de SGBD que
% satisfaci aquestes necessitats. Aquest objectiu es divideix en els
% següents subobjectius més concrets:

% \begin{itemize}

% \item Estudi de les aplicacions en que les dades són sèries temporals
%   amb la finalitat de determinar quines són les propietats i problemes
%   comuns que planteja la seva gestió i emmagatzematge.

% \item Estudi dels models de SGBD existents. Segons es desprèn de la
%   formalització de Date%\textcite{date:introduction}
%   el model principal és el model relacional, el qual es fonamenta en
%   dos conceptes: relacions i tipus de dades.

% \item Una àrea de treball important en els SGBD és la incorporació de
%   nous tipus de dades complexos. És important estudiar com es modifica
%   el model de dades d'un SGBD quan s'afegeix un nou tipus de dades
%   complex.  Les sèries temporals es poden d'entendre com a tipus
%   complex ja que presenten diferents propietats característiques i
%   necessiten operadors addicionals.  

% \item Disseny d'un model de SGBD per a les sèries temporals. D'aquesta
%   manera els SGBD podran tractar dades amb instants de temps que
%   mostrin l'evolució de variables en funció del temps. El model
%   consisteix en la definició de l'estructura de les sèries temporals i
%   les operacions bàsiques que necessiten.

%   L'assoliment d'aquest objectiu té dues parts:

%   \begin{enumerate}
%   \item Disseny d'un model per a la gestió bàsica de les sèries
%     temporals, el qual anomenem model de SGBD per a sèries temporals
%     (SGST).  L'estructura d'aquest model és similar a l'utilitzat en
%     els intervals
%     temporals. %\parencite{date02:_tempor_data_relat_model}.
%     Prenent com a base el model de SGST, el qual és un model general
%     per a les sèries temporals, s'hi poden incloure altres models per
%     a propietats més específiques de les sèries temporals.

%   \item Disseny d'un model específic en base del model de
%     SGST. Concretament es dissenya un model pels SGST multiresolució
%     (SGSTM).  En el model de SGSTM s'hi poden incloure propietats de
%     les sèries temporals relacionades amb la resolució que s'han
%     observat en les aplicacions pràctiques de les sèries temporals:
%     regularització, canvis de resolució mitjançant agregacions,
%     reconstrucció de forats, etc.

%   \end{enumerate}


% \item Variacions del model. Petites variacions que condueixen a altres aplicacions. El model de SGSTM
%     proposat està preparat per a rebre contínuament dades en flux
%     (\emph{data stream}). Això no obstant, la multiresolució també es
%     pot aplicar en temps diferit (\emph{offline}) per a dades
%     emmagatzemades. Cal una funció que permeti determinar com d'una
%     sèrie temporal s'obté una nova sèrie temporal resultant d'haver
%     aplicat un esquema de multiresolució. En aquest
%   càlcul en temps diferit apareix la possibilitat de computar
%   para\l.lelament.

% \item Reflexió sobre la informació en la multiresolució.  A partir
%   de la multiresolució expressada com una funció simplificada, que
%   calcula en temps diferit, convé explorar noves recerques. En primer
%   lloc, cal avaluar formalment aquesta funció de multiresolució
%   respecte a tot el model proposat de SGSTM.  A partir d'aquesta funció pot ésser
%   més senzill estudiar la selecció o pèrdua d'informació que implica
%   usar un model de multiresolució.

% \item Implementació de referència dels models de SGST i SGSTM. Per una
%   banda, aquesta implementació, a nivell acadèmic, ha de servir com a
%   exemple per a futurs desenvolupaments de sistemes de gestió,
%   acadèmics o productius. Per altra banda, ha de servir per a
%   exemplificar-ne els seu funcionament amb unes dades de prova.

% \item Implementació específica del model per a una determinada
%   aplicació de sèries temporals. Exemplificació de com una estructura
%   de SGSTM pot ser implementada per a aconseguir una aplicació
%   concreta. Aquesta implementació treballa amb conceptes de
%   para\l.lelisme. S'ha començat a dissenya una nova implementació de la
%   multiresolució en Hadoop basada en conceptes de computació
%   para\l.lela en els SGBD. Un cop ben establert el model d'aquest
%   esquema de multiresolució cal acabar d'implementar-lo i provar-lo
%   amb les mateixes dades experimentals que s'usen per a la
%   implementació de referència.



% \item Experimentació amb dades. Es disposa d'unes dades
%   noves amb irregularitats interessants per a aplicar-hi el model de
%   multiresolució. Són unes dades de sensors reals en un sistema de
%   dipòsits d'aigües. Aquestes dades pertanyen a sensors
%   interrelacionats, cosa que les fa més interessants per analitzar-hi
%   l'efecte que hi un SGSTM i com pot ajudar a emmagatzemar-les de
%   forma compacta.


% \end{itemize} 




%%% Local Variables: 
%%% mode: latex
%%% TeX-master: "main"
%%% End: 

%  LocalWords:  multiresolució

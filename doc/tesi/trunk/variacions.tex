\chapter{Introducció a consideracions sobre el model}


Què fem en aquesta part?\todo{intro}

Ara que hem definit els models de \gls{SGST} i \gls{SGSTM}, podem fer algunes consideracions i reflexions:


\todo{estructura?}

Potser estructurar aquesta part amb:

1. Exposem SGSTM per a dispositius on l'emmagatzematge reduït i afitat és important, aquest és bàsicament el model que hem presentat però ara farem algunes consideracions
2. Exposem SGSTM per a emmagatzematges massius on calen consultes i visualitzacions ràpides
3. Formulem el problema de la qualitat dels SGSTM, teoria de la informació



Per a 2. i 3. ens farà falta raonar sobre un funció de multiresolució que presentem al capítol tal.
 a la~\autoref{sec:multiresolucio:funcio} observarem com els \gls{SGSTM} es poden definir com una funció sobre una sèrie temporal. Aquests casos ja operaran directament sobre un \gls{SGST} amb tota la sèrie temporal emmagatzemada i per tant els buffers no emmagatzemaran temporalment sinó que treballaran sobre mesures ja emmagatzemades. 













\section{Variacions en l'estructura dels SGSTM}


En el model de \gls{SGSTM} hem definit l'estructura al més genèrica i
senzilla possible per a encabir-hi diferents supòsits de
multiresolució. 
Particularment, en el model hem generalitzat els buffers de manera que
s'acumula tota la sèrie temporal original independentment en cadascun
dels buffers. Aquesta estructura és útil en el model perquè permet
definir de forma molt abstracta el comportament dels \gls{SGSTM} i
abastar-ne diferents possibles variacions.  Però en algunes
implementacions del model pot ser útil utilitzar altres aproximacions,
és a dir algunes variacions podrien resultar útils en el nivell físic
on no es gaudeixen dels avantatges abstractes matemàtics del nivell
lògic.



\todo{revisar paràgraf}

De forma senzilla podem pensar en implementacions que comparteixin els
buffers, per exemple les diferents $f$ amb mateix $\delta$ poden
compartir buffer. De forma més elaborada podem exposar implementacions
del model en què s'emmagatzemi tota la sèrie temporal original en un
\gls{SGST} i el \gls{SGSTM} treballi sobre aquestes mesures, és a dir
que realment els buffers no les emmagatzemen sinó que seleccionen les
que necessiten a cada moment. En aquest cas pensem en \gls{SGST}
d'emmagatzematge massiu com els descrits a l'\autoref{art:massius} i
dels quals a la~\autoref{sec:multiresolucio:dual} n'explorarem
aplicacions mitjançant sistemes \gls{SGST} i \gls{SGSTM} conjunts.



En aquesta secció considerem algunes petites variacions que poden
conduir cap a altres aplicacions.  Presentem tres variacions de
l'estructura:
\begin{itemize}
\item Resolucions encadenades

\item Funcions d'agregació d'atributs orientades a flux

\item El rellotge de consolidació
\end{itemize}









\subsection{Resolucions encadenades}


Una sèrie temporal multiresolució amb estructura de resolucions
encadenades té la mateixa estructura que la presentada en el model
(\seeref{cap:model:sgstm}) llevat que hi ha buffers que
reben les mesures d'altres discs en comptes de l'entrada comuna de
mesures.  És a dir, que una subsèrie resolució depèn dels valors
consolidats a una altra subsèrie resolució, cosa que anomenen
resolucions encadenades.


La \autoref{fig:sgstm:encadenats} mostra l'arquitectura d'una base de
dades multiresolució ja presentada a la \autoref{fig:model:bdstm} però
ara modificada amb les resolucions encadenades.  En aquest cas, les
mesures del disc de $R_0$ s'utilitzen en altres buffers i les mesures
del buffer de $R_d$ provenen d'un altre disc. En un cas simple de
resolucions encadenades, podem considerar que el buffer d'una
resolució és exactament el disc de l'altra. En un cas més elaborat,
podem considerar que quan el disc d'una resolució descarta una mesura,
s'afegeix al buffer de l'altra.


\begin{figure}[tp]
  \centering
  \begin{tikzpicture}
 \tikzset{
        myarrow/.style={->, >=latex',  thick},
      }
      

  \node[rectangle,draw,minimum height=6cm,minimum width=9cm] (m) {};
  \draw[shift=( m.south west)]   
  node[above right] {sèrie temporal multiresolució};


  %discmig
  \node (m.center) (discr1) {...};

  %discr
  
  \node[ellipse,draw,minimum height=3.5cm,minimum width=2.5cm,alias=discr0] [left=of discr1] {};
  \node[above=0cm of discr0.north] {$R_0$};
  \node[below=0cm of discr0] {subsèrie resolució};

  \node[cylinder, draw, shape border rotate=90, aspect=0.25,alias=buffer0] [below=3mm of discr0.north] {buffer};
  \node[circle, draw,alias=disc0]  [above=3mm of discr0.south] {disc} ;
  \draw [->] (disc0.center)++(.4:.4cm) arc(0:180:.4cm);
  \draw[myarrow] (buffer0.bottom) -- (disc0.north);


  %discrd

  \node[ellipse,draw,minimum height=3.5cm,minimum width=2.5cm,alias=discrd] [right=of discr1] {};
  \node[above=0cm of discrd] {$R_d$};
  \node[below=0cm of discrd] {subsèrie resolució};

  \node[cylinder, draw, shape border rotate=90, aspect=0.25,alias=bufferd] [below=3mm of discrd.north] {buffer};
  \node[circle, draw,alias=discd]  [above=3mm of discrd.south] {disc} ;
  \draw [->] (discd.center)++(.4:.4cm) arc(0:180:.4cm);
  \draw[myarrow] (bufferd.bottom) -- (discd.north);



  %mesura 
  \node[above=1cm of m.north] (m0) {};

  \draw[myarrow] (m0) -- (m.north) 
  node[right,midway] {mesura};

  \draw[myarrow] (m.north) -- (buffer0);
  \draw[myarrow] (discr1.south) -- (bufferd);
  \draw[myarrow] (disc0) -- (discr1.north);

\end{tikzpicture}
  \caption{Arquitectura encadenada d'una base de dades multiresolució}
  \label{fig:sgstm:encadenats}
\end{figure}


Respecte a l'estructura general, l'estructura encadenada restringeix
els passos de consolidació dels buffers i els cardinals màxims dels
discs. Els buffers que depenen d'una altra resolució han de tenir un
pas de consolidació múltiple de l'altra resolució i han de tenir un
període de buffer que estigui inclòs en el lapse de l'altra resolució.
A més les resolucions encadenades també han de ser coherents en la
funció d'agregació d'atributs, la qual pot ser que hagi de ser la
mateixa funció. Les resolucions encadenades requereixen un estudi més
profund que l'estructura general i poden encadenar pèrdues successives
significatives, com per exemple és el cas de calcular la mitjana
successivament que, per no ser associativa, no és el mateix que
calcular-la en dos buffers independents.


L'estructura de resolucions encadenades pot ser útil per a aplicacions
que necessitin distribuir l'emmagatzematge de les sèries temporals
multiresolució.  En l'estructura genèrica del model, cada mesura que
s'insereix a una base de dades ha d'inserir-se a totes les subsèries
resolució, és a dir que en cas d'un emmagatzematge distribuït tota la
sèrie temporal original s'ha de distribuir a cada subsèrie.  En canvi
en l'estructura de resolucions encadenades, la sèrie temporal original
primer es resumeix en una subsèrie resolució i és només aquest resum
el que es distribueix a la següent subsèrie resolució.  D'aquesta
manera l'emmagatzematge de les resolucions queda distribuït en
diferents nodes i a l'hora de respondre a una consulta només cal
recollir les resolucions ja resumides que es necessitin.
\textcite{deligiannakis07} proposen una estratègia similar de
disseminació de la informació per a xarxes de sensors.



%\subsubsection{Estructura d'exemple}


A continuació mostrem, mitjançant un exemple, la variació que comporten
les resolucions encadenades en el model de \gls{SGSTM}.


\begin{example} [Sèrie temporal multiresolució amb resolucions encadenades]
  \label{ex:multiresolucio:encadenada}

  Per a definir una sèrie temporal multiresolució amb resolucions
  encadenades és útil reprendre l'\autoref{ex:model:bdm-vistes} en què
  s'exemplifica una sèrie temporal multiresolució organitzada en
  vistes.

  En aquest cas la relació de sèries temporals i noms
  $M_2^{\text{series}}$ segueix sent la mateixa $M_2^{\text{series}}=
  ((S':\text{nom},S:\text{sèrie temporal}),\{
  (S_{B1},\{(26,0),(29,0)\}), (S_{D1},\{(10,0), (15,0), (20,0),
  (25,0)\}), (S_{D2},\{(10,0), (20,0)\} )\})$ llevat que no hi ha
  $S_{B2}$, i la sèrie temporal multiresolució amb noms com a domini
  dels atributs de sèries temporals $M_2^{\text{noms}}$ també és la
  mateixa $M_2^{\text{noms}}= ((S'_B:\text{nom},S'_D:\text{nom},
  \tau:\glssymbol{not:Rb}, \delta:\glssymbol{not:Rb},
  k:\glssymbol{not:N}, f:\text{funció} ),\{ (S_{B1},S_{D1},25 ,5 ,4
  ,\text{mitjana} ), ( \mathbf{S_{D1}},S_{D2},20 , 10 ,3 ,
  \text{mitjana} ) \})$ excepte que el buffer de la segona resolució
  és el disc de la primera $S_{D1}$, el qual el destaquem en negreta.
  Mostrem $M_2^{\text{series}}$ i $M_2^{\text{noms}}$ en forma de
  taula a la \autoref{fig:multiresolucio:exencadenat}.

  \begin{figure}[tp]
    \centering
    \begin{tabular}{|c|c|c|c|c|c|}
      \multicolumn{2}{c}{$M_2^{\text{noms}}$} \\ \hline
      $S'_B$  & $S'_D$ & $\tau$ & $\delta$ & $k$ & $f$ \\ \hline
      $S_{B1}$ & $S_{D1}$ & 25 & 5  & 4 & mitjana  \\
      $\mathbf{S_{D1}}$ & $S_{D2}$ & 20 & 10 & 3 & mitjana  \\ \hline
    \end{tabular}\qquad
    \begin{tabular}{|c|c|c|}
      \multicolumn{3}{c}{$M_2^{\text{series}}$} \\ \hline
      \multirow{2}{*}{$S'$}  &  \multicolumn{2}{c|}{$S$} \\ \cline{2-3}
      & $t$      & $v$  \\ \hline
      \multirow{2}{*}{$S_{B1}$} 
      & 26 & 0 \\ 
      & 29 & 0 \\ \hline
      \multirow{4}{*}{$S_{D1}$} 
      & 10 & 0 \\ 
      & 15 & 0 \\ 
      & 20 & 0 \\ 
      & 25 & 0 \\ \hline
      \multirow{2}{*}{$S_{D2}$} 
      & 10 & 0 \\ 
      & 20 & 0 \\ \hline
    \end{tabular}
    \caption{Taula d'una sèrie temporal multiresolució amb resolucions encadenades}
    \label{fig:multiresolucio:exencadenat}
  \end{figure}




  \begin{figure}[tp]
    \centering
    %\usetikzlibrary{shapes,arrows,positioning}
\begin{tikzpicture}
 \tikzset{
        myarrow/.style={->, >=latex',  thick},
      }
      

  \node[rectangle,draw,minimum height=6cm,minimum width=9cm] (m) {};
  \draw[shift=( m.south west)]   
  node[above right] {base de dades multiresolució};


  %discmig
  \node (m.center) (discr1) {};

  %discr
  
  \node[ellipse,draw,minimum height=3.5cm,minimum width=2.5cm,alias=discr0] [left=of discr1] {};
  \node[above=0cm of discr0.north] {$R_1$};
  \node[below=0cm of discr0] {subsèrie resolució};

  \node[cylinder, draw, shape border rotate=90, aspect=0.25,alias=buffer0] [below=3mm of discr0.north] {mitjana};
  \node[circle, draw,alias=disc0,minimum width=1.2cm]  [above=3mm of discr0.south] {5} ;
  \draw [->] (disc0.center)++(.2:.2cm) arc(0:180:.2cm);
  \draw[myarrow] (buffer0.bottom) -- (disc0.north);

  \node[circle,minimum width=9mm] (d0boles) [below=0mm of disc0.center,anchor=center] {};
  \node[below=0mm of d0boles.north,anchor=center] {$\circ$};
  \node[below=0mm of d0boles.east,anchor=center] {$\circ$};
  \node[below=0mm of d0boles.south,anchor=center] {$\circ$};
  \node[below=0mm of d0boles.west,anchor=center] {$\circ$};


  %discrd

  \node[ellipse,draw,minimum height=3.5cm,minimum width=2.5cm,alias=discrd] [right=of discr1] {};
  \node[above=0cm of discrd] {$R_2$};
  \node[below=0cm of discrd] {subsèrie resolució};

  \node[cylinder, draw, shape border rotate=90, aspect=0.25,alias=bufferd] [below=3mm of discrd.north] {mitjana};
  \node[circle, draw,alias=discd,minimum width=1.2cm]  [above=3mm of discrd.south] {10} ;
  \draw [->] (discd.center)++(.3:.3cm) arc(0:180:.3cm);
  \draw[myarrow] (bufferd.bottom) -- (discd.north);

  \node[circle,minimum width=9mm] (d1boles) [below=0mm of discd.center,anchor=center] {};
  \node[below=0mm of d1boles.north,anchor=center] {$\circ$};
  \node[below=0mm of d1boles.south east,anchor=center] {$\circ$};
  \node[below=0mm of d1boles.south west,anchor=center] {$\circ$};


  %mesura 
  \node[above=1cm of m.north] (m0) {};

  \draw[myarrow] (m0) -- (m.north) 
  node[right,midway] {mesura};

  \draw[myarrow] (m.north) -- (buffer0);
  
  %encadenada
  \draw[myarrow] (disc0.north) -- (bufferd);


\end{tikzpicture}
    \caption{Arquitectura de la base de dades multiresolució
      particular per l'\autoref{ex:multiresolucio:encadenada}}
    \label{fig:multiresolucio:ex-arqu-encadenada}
  \end{figure}



  Se segueix aplicant la mateixa operació de $\text{vista } M_2$ que a
  l'\autoref{ex:model:bdm-vistes} per a obtenir la sèrie temporal
  multiresolució. A la \autoref{fig:multiresolucio:ex-arqu-encadenada}
  particularitzem l'arquitectura de la \autoref{fig:sgstm:encadenats}
  per a la base de dades d'aquest exemple. Cal, però, tenir dues
  consideracions en les operacions estructurals dels \gls{SGSTM} per a
  les resolucions encadenades: 

  \begin{itemize}
  \item L'operació d'inserció de mesures, $\glssymbol{addM}(M,m)$, no
    pot inserir la mesura a tots els buffers de les subsèries
    resolució sinó només a aquells que no estiguin encadenats. En el
    cas de l'exemple només al buffer $B_1$.  Aquests buffers, als
    quals podem anomenar buffers d'entrada, es poden expressar amb
    l'operació $ \glssymbol{not:sgst:project}_{\{S'_B\}}(
    M_2^{\text{noms}} ) -
    \glssymbol{not:sgst:rename}_{S'_D/S'_B}\left(
      \glssymbol{not:sgst:project}_{\{S'_D\}}( M_2^{\text{noms}}
      )\right)$.


  \item Només es poden eliminar les mesures dels buffers que no siguin
    encadenats, és a dir dels buffers d'entrada. Les resolucions
    encadenades només poden llegir les dades dels altres discs però
    no hi tenen control.

  \end{itemize}







\end{example}









\subsection{Funcions d'agregació amb orientació a flux}


Les funcions d'agregació d'atributs definides a
\textref{sec:model:agregador} operen sobre un interval de la sèrie
temporal i retornen una mesura que en resumeix un atribut. Aquesta
definició genèrica implica que els buffers han d'emmagatzemar
temporalment un conjunt de mesures de la sèrie temporal original i un
cop resumides les poden eliminar.

Això no obstant, es poden utilitzar els algoritmes d'orientació a
flux, com els que proposen \textcite{cormode08:pods}, per tal d'afitar
els cardinals dels buffers. Tot i així no totes les funcions
d'agregació d'atributs es poden implementar amb orientació a flux.




Definim una funció d'agregació amb orientació a flux, $f^\text{flux}$,
com aquella que implementa el comportament equivalent a una funció
d'agregació d'atributs $f$. A diferència de les $f$, treballen sobre
dues mesures $m'=f^\text{flux}(m_f,m,i)$ per a retornar la mesura
resultant $m'$, on $m$ és la nova mesura que s'ha d'incorporar al
flux, $m_f$ és el flux anterior ja processat i $i=[t_a,t_b]$ és
l'interval de consolidació de $f$.  Per exemple, redefinim les
funcions d'agregació \gls{dd} màxim i mitjana per tal que tinguin
orientació a flux:

\begin{itemize}
\item $\glssymbol{not:sgstm:maxdd}^\text{flux}: m_f \times m \times i \mapsto
  m'$ on $V(m')=\max(V(m_f),V(m))$ i $T(m')=\frac{t_b+t_a}{2}$. 


\item $\glssymbol{not:sgstm:mitjanadd}^\text{flux}: m_f \times m
  \times i \mapsto m'$ on $V(m') = V(m_f) + \frac{V(m)}{t_b-t_a}$ i
  $T(m')=\frac{t_b+t_a}{2}$.

\end{itemize}


Així, sigui $S=\{m_0,\dotsc,m_k\}$ una sèrie temporal i $i=[t_a,t_b]$
un interval de dos instants de temps, calcular en flux $m'_0 =
\glssymbol{not:sgstm:mitjanadd}^\text{flux}((0,0),m_0,i), m'_1 =
\glssymbol{not:sgstm:mitjanadd}^\text{flux}(m'_0,m_1,i), \dotsc, m'_k
= \glssymbol{not:sgstm:mitjanadd}^\text{flux}(m'_{k-1},m_k,i)$ és
equivalent a $m_k'=\glssymbol{not:sgstm:mitjanadd}(S,i)$, on $(0,0)$
és una mesura inicial per al flux.




Per a utilitzar en els \gls{SGSTM} les funcions d'agregació d'atributs
amb orientació a flux s'han de canviar els operadors d'afegir i de
consolidar dels buffers:



\begin{itemize}
\item Sigui la \autoref{def:sgstm:addB}, se'n modifica el comportament
  perquè $B=(m_B,\tau,\delta,f)$ sigui un buffer que emmagatzema una
  mesura $m_B$ en comptes d'una sèrie temporal i l'operació d'afegir
  sigui $\glssymbol{addB}(B,m)= (m'_B,\tau,\delta,f)$ on $m'_B =
  f^\text{flux}(m_B,m,[\tau,\tau+\delta])$ i $f^\text{flux}$ és una
  funció d'agregació d'atributs orientada a flux.


\item Sigui la \autoref{def:model:consolidacio-buffer}, se'n modifica
  el comportament perquè essent el buffer modificat
  $B=(m_B,\tau,\delta,f)$ l'operació de consolidar sigui
  $\glssymbol{consolidaB}(B) = (B',m_B)$ on
  $B'=(m'_B,\tau+\delta,\delta,f)$ i $m'_B$ ha de ser l'element
  d'identitat de $f^\text{flux}$. Per exemple $m'_B=(0,0)$ per a l'atribut
  de mitjana i $m'_B=(0,\min(\glssymbol{not:valor-domini}))$ per a
  l'atribut de màxim on $\glssymbol{not:valor-domini}$ és el domini
  dels valors. 

\end{itemize}

Així doncs, en l'orientació a flux de les funcions d'agregació
d'atributs, la mesura resultant es computa durant l'operació d'afegir
noves mesures al buffer i quan s'ha de consolidar el buffer el
resultat ja està disponible, només cal determinar l'element que actua
com a identitat per a la funció d'agregació amb flux.  En aquest cas,
no té sentit parlar de l'eliminació de mesures antigues en el buffer.










\subsection{Rellotge de consolidació}


En el model de \gls{SGSTM} no hi ha definit el concepte de rellotge,
és a dir no s'explicita quan s'ha de computar l'operació de
consolidar, si bé s'ha definit quan les sèries temporals
multiresolució esdevenen consolidables.  Les mesures tenen l'atribut
de temps i, si s'insereixen ordenades, ja marquen el pas del temps.
Tot i així, segons com sigui el rellotge i quan es computi l'operació
de consolidació hi pot haver els escenaris següents:

\begin{itemize}
\item Extern. Ho anomenem rellotge extern o \emph{push} perquè les
  mesures són les que controlen el procés de consolidació, de fet el
  controla un sistema de monitoratge extern.  El \gls{SGSTM} no té
  rellotge sinó que s'utilitza l'atribut de temps de les mesures per
  conèixer l'instant actual.  És el cas que hem definit en el model,
  en què una sèrie temporal multiresolució esdevé consolidable segons
  els instants de temps de les mesures adquirides i llavors ja pot ser
  consolidada. Per saber quan esdevé consolidable es pot consultar
  periòdicament o en base a esdeveniments, per exemple cada cop que
  s'insereixi una nova mesura.  Ja que el temps observat pel
  \gls{SGSTM} només canvia quan té mesures noves, això pot causar un
  cert decalatge de la consolidació de l'esquema amb un rellotge real,
  sobretot quan hi hagi inframostreig. 
  % es poden omplir els buffers?

\item Intern. Ho anomenem rellotge intern o \emph{pull} perquè el
  \gls{SGSTM} té un rellotge que controla el procés de consolidació.
  En aquests cas, la consolidació actua al marge del temps que
  indiquin les mesures i es computa quan ho marca el rellotge.  Això
  causa que la consolidació de l'esquema estigui totalment
  sincronitzada amb el rellotge real. En aquest cas s'hi poden
  incloure \gls{SGSTM} que controlin el procés d'adquisició, és a dir
  que ordenin quan s'han d'adquirir noves mesures. També es pot pensar
  en \gls{SGSTM} i istemes de monitoratge que no tinguin una bona
  mesura del temps actual, per exemple sense sincronització de
  rellotge, i en què l'objectiu dels \gls{SGSTM} sigui informar de
  l'evolució de les variables situant-les relativament a partir de
  l'instant en què es fa la consulta.




\end{itemize}

En el cas de les resolucions encadenades, com que depenen de la
consolidació d'un altre disc aquest aquest pot servir per a marcar el
rellotge. 


% * en cas de comptadors digitals, aquests cada cop que incrementen el valor poden fer un push a la base de dades. En comptadors analògics no ho poden fer perquè van incrementant contínuament i no discretament.







% \subsubsection{RRDtool}

% \todo{rrdtool?}
% Dibuixar l'esquema bàsic de RRDtool i comentar-lo aquí?






%%% Local Variables: 
%%% mode: latex
%%% TeX-master: "main"
%%% End: 







\section{Sistemes i projectes similars}



\todo{revisar}

Similars a què? Similars als models que presentem? explicar-ho


Hi ha hagut vàries implementacions de sistemes específics per a sèries
temporals. Algunes són només l'aplicació d'un algoritme d'anàlisi per
un problema concret de sèries temporals però altres són més elaborades
i es defineixen com a SGBD per a sèries temporals.  En aquest apartat
resumim algunes aplicacions que considerem que implementem conceptes
dels SGST.







\todo{} També hi ha molts sistemes propis d'empreses que van lligats
amb els seus productes. Ara bé ofereixen molt poques capacitats de
SGST i les que ofereixen són molt restringides a l'àmbit a on estan
dirigits els productes; és a dir que no són genèrics i són més aviat
controladors del procés d'adquisició. Per exemple Keller
\url{http://www.catsensors.com/ca/productes/varis__software/logger_4x}, permet desar dades cada un cert període amb estructura d'anell (és a dir eliminant les més antigues quan és ple) però només té un anell. A banda permet detectar certs esdeveniments i aleshores canviar el període de mostreig. A banda permet també emmatgazemar alguns estadístics de les dades: mitjana i rang cada certs segons.






On the other hand, compression techniques for time series are
considered in the form of approximation to the original signal in
order to compute analysis such as similarity or pattern search
\cite{fu11,keogh01,last01} or in the form of compression and
aggregation approaches for massive data streams
\cite{cormode08:pods,bonnet01}. However, treating time series as data
streams does not consider adequately the time dimension nor computes
the evolution of aggregated parameters along time, which is
interesting for monitoring purposes.  On a similar approach,
\emph{RRDtool} \cite{rrdtool} is a system that stores time series
aggregated in different resolutions in order to compact data and to do
faster visualisations. However, \emph{RRDtool} is very specific and
has limited aggregation operations to applications of network
counters.


\url{http://en.wikipedia.org/wiki/Time_series_database}




 







\todo{wavelet}

També hi ha l'anàlisi de les sèries temporals amb wavelet analysis. Aquest es basa en anàlisis de la freqüència dels senyals. 

A multiresolution analysis (MRA) or multiscale approximation (MSA) is the design method of most of the practically relevant discrete wavelet transforms (DWT) and the justification for the algorithm of the fast wavelet transform (FWT). It was introduced in this context in 1988/89 by Stephane Mallat and Yves Meyer and has predecessors in the microlocal analysis in the theory of differential equations (the ironing method) and the pyramid methods of image processing as introduced in 1981/83 by Peter J. Burt, Edward H. Adelson and James Crowley.






\todo{}






\subsection{Sistemes genèrics}


La recerca en dades bitemporals formalitza de forma adequada els
\gls{SGBD} per a poder tractar històrics i esdeveniments
temporals \parencite{jensen99:temporaldata,date02:_tempor_data_relat_model}. Això
no obstant, com ja hem notat, les sèries temporals i les dades
bitemporals no són exactament el mateix i no poden ser tractats de la
mateixa manera \parencite{schmidt95}. Hi ha, però, certes similituds
que es poden tenir en compte, per exemple les nocions de temps
discret. A més, formalitzarem les sèries temporals de manera similar a
com les dades bitemporals es formalitzen en els \gls{SGBDR}.



Per altra banda, alguns autors descriuen sistemes genèrics per a
tractar sèries temporals, és a dir amb un model adequat per a sèries temporals però sense cap tècnica específica per a processar-les\todo{?}. A continuació en descrivim alguns breument.




\begin{description}


\item[TDM] \textcite{segev87:sigmod} presenten un model, que anomenen
  \emph{Temporal Data Management} (TDM), per a dades temporals amb un
  llenguatge molt semblant a \gls{SQL}. Les seqüències temporals que
  presenten són similars a les que definim en el model de
  \gls{SGST}, inclouen la noció de regularitat i representació
  temporal, tot i que molt lligades a un tercer atribut que indica
  l'objecte de referència. Principalment estudien les operacions
  d'agregació sobre les sèries temporals.


\item[Calanda] \textcite{dreyer94} proposen els requeriments de
  propòsit específic que han de complir els \gls{SGST} i basen el
  model en quatre elements estructurals bàsics: esdeveniments, sèries
  temporals, grups i metadades, a banda de les bases de dades per
  sèries temporals. Implementen un \gls{SGST} anomenat
  Calanda \parencite{dreyer94b,dreyer95,dreyer95b} que té operacions
  de calendari, pot agrupar sèries temporals i respondre consultes
  simples i ho exemplifiquen amb dades econòmiques. A \cite{schmidt95}
  es compara Calanda amb els \gls{SGBD} per a dades bitemporals.




\item[Pandas] Pandas \parencite{pandas} és una eina d'anàlisis de
  dades. Tot i no ser un \gls{SGBD} sí que en té una forta orientació
  ja que gestiona les dades a partir d'una estructura tabular i amb
  molts conceptes relacionals.  Una de les principals aplicacions és
  en les sèries temporals, inclou per exemple la regularització de
  sèries temporals. Així, Pandas és semblant a altres eines d'anàlisi
  estadística per a computació científica però incorpora la gestió de
  sèries temporals i dades similars. Un sistema similar d'anàlisi de
  sèries temporals, scikits.timeseries \parencite{pytseries}, ja no es
  desenvolupa més i està previst que s'incorpori a Pandas.



\end{description}



%SETL http://setl.org/setl/ un llenguatge de programació d'alt nivell que té els conjunts i els mapes de primer ordre com a parts fonamentals. Els tipus bàsics són conjunts, conjunts desordenats i seqüències (també anomenades tuples). Els mapes són conjunts de parelles (tuples de mida dos). Les operacions bàsiques inclouen la pertinença, la unió, la intersecció, etc.




\subsection{Tècniques de compressió i aproximació}



% As \acro{TSMS} suffer from problematic properties of time
% series, like the ones we describe in
% Section~\ref{sec:model:properties} mainly the huge data volume,
% compression techniques are used.  Next, we summarise some current work
% in \acro{TSMS} with compression.

Els \gls{SGST} han de gestionar les propietats problemàtiques de les
sèries temporals, com les descrites a
la~\autoref{sec:art:problemes}. Principalment, el gran volum de dades
comporta que s'explorin tècniques de compressió de les dades o de
treballar amb dades que s'aproximin a la informació original.  La
compressió i aproximació es pot explorar tant amb emmagatzematge amb
pèrdua de les dades originals o sense pèrdua o fins i tot intentant
resoldre el problema de trobar el compromís entre les mínimes dades
que poden reconstruir el senyal original amb el mínim error. A
continuació descrivim breument els projectes que exploren la
compressió i aproximació de sèries temporals.



\begin{description}


\item[T-Time] \textcite{assfalg08:thesis} mostra un sistema que pot
  cercar similituds entre sèries temporals, calculades segons funcions
  de distàncies entre sèries temporals. Principalment, dues sèries
  temporals es marquen com a similars si la seva distància és menor
  a un llindar per cada interval de temps. A partir d'aquest mètode dissenya
  algoritmes eficients que implementa en un programa anomenat
  T-Time \parencite{assfalg08:ttime}.

 
\item[iSAX] \textcite{keogh08:isax,keogh10:isax} estudien l'anàlisi i
  l'indexat de co\l.lecions massives de sèries temporals. Descriuen
  que el problema principal del tractament rau en l'indexat de les
  sèries temporals i proposen mètodes per calcular-lo de manera
  eficient. El mètode principal que proposen està basat en
  l'aproximació a trossos de la sèrie temporal \parencite{keogh00}.
  Ho implementen en una estructura de gestió de dades que anomenen
  \emph{indexable Symbolic Aggregate approXimation}
  (iSAX) \parencite{isax}. Les representacions de sèries temporals que
  s'obtenen amb aquesta eina permeten reduir l'espai emmagatzemat i
  indexar tant bé com altres mètodes de representació més complexos.
  Aquestes tècniques de compressió són candidates per a ser usades com
  a funcions d'agregació d'atributs en el model de \gls{SGSTM} que
  definim, així seria interessant poder definir agregacions en el
  domini freqüencial de les sèries temporals.

% Piecewise Aggregate Approximation (PAA) \cite{keogh00}: aproxima una sèrie temporal partint-la en segments de la mateixa mida i emmagatzemant la mitjana dels punts que cauen dins del segment. Redueix de dimensió $n$ a dimensió $N$

% Adaptive Piecewise Constant Approximation (APCA) \cite{keogh01}: com el PAA però amb segments de mida variable.




\item[RRDtool]
  \parencite{rrdtool,lisa98:oetiker} desenvolupa un \gls{SGBD}
  anomenat RRDtool que és molt usat per la comunitat de programari
  lliure en l'àmbit dels sistemes de monitoratge. A causa d'això es
  focalitza en unes dades en particular, les magnituds i els
  comptadors, i hi manquen operacions genèriques de sèries temporals.
  La principal característica és l'emmagatzematge de les dades amb la
  tècnica que anomenen Round Robin, la qual consisteix en emmagatzemar
  més resolució per als temps recents i en perdre resolució per als
  temps més antics tot gestionant els registres d'emmagatzematge de
  manera circular.

  Hi ha diversos projectes que utilitzen RRDtool com a \gls{SGBD}, en
  els quals hi ha sistemes de monitoratge professionals, també en
  l'àmbit de programari lliure, com
  Nagios/Icinga \parencite{nagios,icinga} o el Multi Router Traffic
  Grapher (MRTG) \parencite{mrtg}. Aquests monitors transfereixen a
  RRDtool la responsabilitat de gestionar l'emmagatzematge i d'operar
  amb les dades, i així es poden centrar en l'adquisició de dades i la
  gestió d'alarmes. Altres projectes adapten la tècnica de RRDtool en
  altres llenguatges, com per exemple
  \emph{JRobin} \parencite{jrobin}. També és destacable l'ús emergent
  de RRDtool en entorns d'experimentació, com és el cas de
  \textcite{zhang07} i \textcite{chilingaryan10} que hi emmagatzemen
  dades experimentals per posteriorment predir o validar-les.  A causa
  del gran ús que es fa de RRDtool, sobretot en la comunitat de
  programari lliure, ens ha inspirat per a desenvolupar un model a
  partir de les principals característiques, la qual cosa és el que
  anomenem multiresolució.


  En l'evolució de RRDtool hi ha dues millores destacables. En primer
  lloc, \textcite{lisa98:oetiker} va separar el sistema de gestió de
  RRDtool d'un sistema de monitoratge particular, MRTG, i el va
  dissenyar amb l'estructura característica de Round Robin. En segon
  lloc, \textcite{lisa00:brutlag} va estendre RRDtool amb algoritmes
  de predicció i detecció de comportaments aberrants.  Actualment,
  s'està estudiant l'eficiència i rapidesa de RRDtool en processar les
  sèries temporals.  RRDtool pot emmagatzemar múltiples resolucions de
  les dades, però \textcite{lisa07:plonka} troben limitacions de
  rendiment quan s'han d'emmagatzemar grans quantitats de sèries
  temporals diferents. Una solució que observen per a aquest problema
  és l'aplicació de \emph{cache} dissenyada per
  \textcite{carder:rrdcached}, anomenada \emph{rrdcached}, que permet
  fer funcionar simultàniament sistemes amb grans quantitats de bases
  de dades RRDtool.




\item[Whisper] Una eina per a visualitzar gràfics de dades que tenen
  forma de sèries temporals és Graphite \parencite{graphite}. Graphite
  utilitza un \gls{SGBD} anomenat Whisper que té un disseny molt
  similar a RRDtool, de fet inicialment Graphite usava RRDtool com a
  sistema d'emmagatzematge.





\item[Tsdb]
Alerta que no és el mateix tsdb que opentsdb??

\todo{aquest potser hauria d'anat a compressió}

Diuen que OpenTSDB té una arquitectura massa complicada i només és útil per a sistemes distribuïts.

Han fet anàlisis del rendiment de RRDtool, MySQL i la seva implementació TSDB i conclouen que RRDtool és el que pitjor funciona per a sèries temporals. La seva implementació, TSDB, es basa en la compressió de dades. Assumeixen que les sèries temporals són regulars i totes tenen el mateix patró de mostreig, fet que els permet implementar les gestió de les sèries temporals de manera més senzilla.
\todo{Potser aquest va a compressió de dades doncs?}

Deri et al.\ \cite{deri12:tsdb_compressed_database} present
\emph{Tsdb}, a lossless compression storage \acro{TSMS} for time
series that share the same time instants of acquisition. Different
time series are stored grouped by the time of acquisition instead of
each time series isolated.  They compare \emph{Tsdb} with \emph{RRDtool} and
with a relational product. As a consequence of \emph{Tsdb} structure,
they achieve a better measure addition time but a worse global
retrieval time as data has to be contiguously regrouped. However, when
measures have same time this is seen as the same time series in a
\acro{MTSMS}, so it would be interesting to use this implementation
architecture of shared time arrays in \acro{MTSMS} for resolution subseries
with same delta time in order to achieve better performance requirements
when having much equal acquired time series.





\item[Emmagatzematge en memòries Flash]
  \textcite{dou14:historic_queries_flash_storage} se centren en
  l'àmbit de l'emmagatzematge de sèries temporals en memòries de tipus
  Flash, de les quals noten que tenen propietats diferents a
  l'emmagatzematge tradicional en discs.  Proposen emmagatzemar
  informació de cada sèrie temporal per a poder resoldre tres tipus de
  consultes: agregacions temporals, històrics basats en mostrejos
  aleatoris i cerca de patrons similars.  La tècnica d'agregacions
  temporals que utilitzen és molt semblant a la de RRDtool, és a dir
  agregar i emmagatzemar les dades amb diferents resolucions, tot i
  que implementada i particularitzada per a les memòries Flash, amb
  registre i punters. Per a la cerca de patrons similars indexen les
  sèries temporals de manera similar als algoritmes de iSAX.


\end{description}



\subsection{Processament en flux}


Les sèries temporals també es tracten com a fluxos de dades
(\emph{data stream}) per tal de resoldre consultes d'agregació
estadística de les dades mitjançant consultes aproximades.  Com a
fluxos de dades, s'exploren tècniques per a processar les consultes de
forma incremental cada cop que arriba una dada nova.
\textcite{cormode08:pods} exploren tècniques d'agregació en flux per a
sèries temporals que consideren donar més pes a les dades més recents,
és a dir de manera molt similar a la multiresolució que proposem però
només per a una resolució i per a unes funcions d'agregació
determinades.

El processament en flux  s'usa sobretot en l'àmbit de les xarxes de
sensors, del qual a continuació en descrivim alguns projectes.


\begin{description}

\item[Cougar] \textcite{cougar,fung02} proposen Cougar com un
  \gls{SGBD} per a xarxes de sensors (\emph{sensor database
    systems}). El sistema té dues estructures \parencite{bonnet01}:
  una per a les característiques dels sensors emmagatzemades com a
  taules relacionals i una altra per a les sèries temporals dels
  sensor emmagatzemades com a seqüències de dades.  Les consultes es
  processen de manera distribuïda. Cada sensor és un node amb
  capacitat de processament que pot resoldre una part de la consulta i
  fusionar-la amb les altres. D'aquesta manera les dades
  s'emmagatzemen distribuïdes en els sensors i les consultes es
  resolen combinant les dades amb orientació de flux, cosa que millora
  el rendiment del processament i es minimitza l'ús de les
  comunicacions.  Això no obstant, l'estructura i l'estratègia de
  comunicació dels nodes esdevé una part crítica a configurar en
  aquests sistemes \parencite{demers03}.


\item[TinyDB] Un altre prototip de \gls{SGBD} per a xarxes de sensors
  desenvolupat para\l.lelament a Cougar és
  TinyDB \parencite{tinyDB,madden05}. A més de les característiques
  descrites per Cougar, aquest sistema s'implica i modifica el procés
  d'adquisició de les dades com ara els instants de temps, la
  freqüència o l'ordre de mostreig. Per exemple donada una consulta
  que vol correlacionar les dades de dos sensors diferents, el sistema
  indica als sensors implicats que han d'adquirir amb la mateixa
  freqüència.


\end{description}




% \url{http://2013.nosql-matters.org/bcn/abstracts/#abstract_gianmarco}

% Streaming data analysis in real time is becoming the fastest and most efficient way to obtain useful knowledge from what is happening now, allowing organizations to react quickly when problems appear or to detect new trends helping to improve their performance. In this talk, we present SAMOA, an upcoming platform for mining big data streams. SAMOA is a platform for online mining in a cluster/cloud environment. It features a pluggable architecture that allows it to run on several distributed stream processing engines such as S4 and Storm. SAMOA includes algorithms for the most common machine learning tasks such as classification and clustering. 




\subsection{Emmagatzematge massiu}

Hi ha sistemes que aborden l'emmagatzematge massiu de les
sèries temporals, és a dir de grans volums de dades, seguint
l'enfocament de les \gls{VLDB}.  A continuació en descrivim alguns
projectes.



\begin{description}



\item[TSDS] \textcite{weigel10} noten la necessitat de mostrar les
  dades en tot el seu rang temporal i no només en un subconjunt com
  molts altres sistemes ofereixen. Desenvolupen el paquet informàtic
  \emph{Time Series Data Server} (TSDS) \parencite{tsds} en què es
  poden introduir les dades de sèries temporals per posteriorment
  consultar-les per rangs temporals o aplicant-hi filtres i
  operacions. La particularitat de TSDS és el fet que incorpora un
  sistema de \emph{cache} per a les consultes que, de forma similar a
  la tècnica descrita per RRDtool, emmagatzema els resultats de les
  consultes segons la resolució i agregació realitzada. D'aquesta
  manera els resultats es poden aprofitar per a altres consultes
  similars. Això no obstant, aquestes consultes s'han de basar en els
  operadors predefinits de TSDS.







\item[SciDB] \textcite{stonebraker09:scidb} estudien l'emmagatzematge
  de dades científiques en \gls{SGBD} basats en models de matrius.
  Les sèries temporals són les dades científiques per exce\l.lència, i
  per tant són les aplicacions que principalment exploren.  Dissenyen
  SciDB, un \gls{SGBD} que implementa les sèries temporals com
  a matrius i permet aconseguir anàlisis multidimensonals amb més bon
  rendiment. Les altres dades que acompanyen les sèries temporals les
  emmagatzemen en taules. 
\todo{més?}


% However,
% difference between tables and arrays seems too physical and leads to
% ambiguity when representing time series.  
% Our TSMS model proposes time
% series as firmly based on relational algebra, clarifying this
% ambiguity and describing them coherently in terms of information
% systems theory.



\item[SciQL] \textcite{kersten11,zhang11} descriuen SciQL, un
  llenguatge per a \gls{SGBD} de dades científiques basades en
  matrius, del qual n'estan desenvolupant un
  prototip \parencite{sciql}. És molt semblant a la proposta de SciDB,
  però a diferència SciQL defineix les sèries temporals com una mescla
  de matrius, conjunts i seqüències. A més mostren com gestionar
  algunes característiques de sèries temporals com per exemple la
  regularitat, la interpolació o les consultes de correlació.
\todo{} 







\item[OpenTSDB] OpenTSDB \parencite{opentsdb} és un sistema
  d'emmagatzematge distribuït de sèries temporals. Basa
  l'emmagatzematge en Apache Hadoop i HBase, els quals permeten
  distribuir les dades ens diferents nodes. Gràcies a aquests
  sistemes, pot emmagatzemar totes les dades originals ja que és una
  estructura en què és ràpid d'escriure-hi i localitzar les dades, cal
  destacar que HBase crea uns índex potents de les dades i això
  s'aprofita per a indexar l'atribut de temps de les sèries temporals.
  Per a consultar les dades defineixen el concepte d'agregadors, tot i
  que només per a interpolacions lineals, i les operacions d'agregació
  es processen en el mateix moment d'executar la consulta.  Així
  doncs, si bé pot recuperar les dades de forma molt ràpida,
  restringeix les consultes a intervals temporals petits per tal que
  les execucions siguin ràpides. Per tant, és un sistema útil sobretot
  per a visualitzar i comparar intervals temporals petits de diferents
  sèries temporals.




\end{description}





% http://stackoverflow.com/questions/4814167/storing-time-series-data-relational-or-non















%%% Local Variables: 
%%% mode: latex
%%% TeX-master: "main"
%%% End: 



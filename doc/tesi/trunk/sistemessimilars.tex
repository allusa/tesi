
\section{Sistemes i projectes similars}



\todo{revisar}

Similars a què? Similars als models que presentem? explicar-ho


Hi ha hagut vàries implementacions de sistemes específics per a sèries
temporals. Algunes són només l'aplicació d'un algoritme d'anàlisi per
un problema concret de sèries temporals però altres són més elaborades
i es defineixen com a SGBD per a sèries temporals.  En aquest apartat
resumim algunes aplicacions que considerem que implementem conceptes
dels SGST.







\todo{} També hi ha molts sistemes propis d'empreses que van lligats
amb els seus productes. Ara bé ofereixen molt poques capacitats de
SGST i les que ofereixen són molt restringides a l'àmbit a on estan
dirigits els productes; és a dir que no són genèrics i són més aviat
controladors del procés d'adquisició. Per exemple Keller
\url{http://www.catsensors.com/ca/productes/varis__software/logger_4x}, permet desar dades cada un cert període amb estructura d'anell (és a dir eliminant les més antigues quan és ple) però només té un anell. A banda permet detectar certs esdeveniments i aleshores canviar el període de mostreig. A banda permet també emmatgazemar alguns estadístics de les dades: mitjana i rang cada certs segons.






On the other hand, compression techniques for time series are
considered in the form of approximation to the original signal in
order to compute analysis such as similarity or pattern search
\cite{fu11,keogh01,last01} or in the form of compression and
aggregation approaches for massive data streams
\cite{cormode08:pods,bonnet01}. However, treating time series as data
streams does not consider adequately the time dimension nor computes
the evolution of aggregated parameters along time, which is
interesting for monitoring purposes.  On a similar approach,
\emph{RRDtool} \cite{rrdtool} is a system that stores time series
aggregated in different resolutions in order to compact data and to do
faster visualisations. However, \emph{RRDtool} is very specific and
has limited aggregation operations to applications of network
counters.




\url{http://en.wikipedia.org/wiki/Time_series_database}





\subsection{Database approaches}

Some authors treat \acro{TSMS} as a particular \acro{DBMS} field
\cite{last01}.  Segev and Shoshani \cite{segev87:sigmod} propose an
structured language for querying \acro{TSMS}. Their time series
structures include the notion of regularity and temporal
representation and their operations are \acro{SQL}-like.  Dreyer et
al.\ \cite{dreyer94} propose the requirements of special purpose
\acro{TSMS} and base the model on five basic structural elements:
events, time series, groups, metadata and time series bases. They
implement a \acro{TSMS} called \emph{Calanda} which includes calendar
operations, allows grouping time series and operating with simple
queries. They exemplify it with financial data. In \cite{schmidt95}
\emph{Calanda} is compared with temporal systems designed for time
series.



 
Others implement \acro{TSMS} with array database approaches.
\emph{SciDB} \cite{stonebraker09:scidb} and \emph{SciQL}
\cite{zhang11} are array database systems intended for science
applications, in which time series play a principal role. They
structure time series into arrays in order to achieve multidimensional
analysis and they store other data into tables.  \emph{SciDB} is based
on arrays which, according to the authors, allow to represent time
series.  In contrast, \emph{SciQL} defines time series as a mixture of
array, set, and sequence properties and exhibits some time series
managing characteristics that include time series regularities,
interpolation or correlation queries.
% However,
% difference between tables and arrays seems too physical and leads to
% ambiguity when representing time series.  
% Our TSMS model proposes time
% series as firmly based on relational algebra, clarifying this
% ambiguity and describing them coherently in terms of information
% systems theory.





Bitemporal \acro{DBMS}, sometimes referred directly as temporal data,
is a database field related with time. Bitemporal data manages
historical data and events in databases by associating pairs of
\emph{valid} and \emph{transaction} time intervals to data.
Bitemporal data and time series data are not exactly the same and so
can not be treated interchangeably \cite{schmidt95}, however, there
are some similarities that can be considered. Moreover, \acro{DBMS}
research represents bitemporal data as relations extended with time
intervals attributes and extends relational operations in order to
deal with related time aspects
\cite{jensen99:temporaldata,date02:_tempor_data_relat_model}.  We
formalise time series similarly as how bitemporal data is formalised
for relational \acro{DBMS}.
% On the other hand, some bitemporal time concepts might be taken
% into account by \acro{TSMS}, such as the discussions about time
% granularities.



\subsection{Compression techniques}


% As \acro{TSMS} suffer from problematic properties of time
% series, like the ones we describe in
% Section~\ref{sec:model:properties} mainly the huge data volume,
% compression techniques are used.  Next, we summarise some current work
% in \acro{TSMS} with compression.



\emph{RRDtool} from Oetiker, \cite{rrdtool,lisa98:oetiker}, is a free
software database management system. It is designed to be used for
monitoring systems. Because of this, it is focused to a particular
kind of data, gauges and counters, and it lacks general time series
operations. \emph{RRDtool} can store multiple time resolution data,
however Plonka et al.\ \cite{lisa07:plonka} evaluated \emph{RRDtool}
performance and found limitations when storing huge number of
different time series. They propose a caching system on top of
\emph{RRDtool} as a solution.  \emph{RRDtool} is extremely used by the
free software community so it inspired us to develop a model from its
main characteristics, that is now what we call multiresolution. A
similar approach is done by \cite{weigel10} in a system called
\emph{TSDS} that caches queries by aggregate parameters. They notice
that data needs to be shown over its full time range and not only
subsets of data as it is usually provided.  They develop the software
package \emph{TSDS} where time series are stored fully and then
requested by date ranges or by applying different filters and
operations to the time series data.  Our \acro{MTSMS} model is a generic
approach to the multiresolution features, we define it open so that
users can define any attribute aggregate functions.


Deri et al.\ \cite{deri12:tsdb_compressed_database} present
\emph{Tsdb}, a lossless compression storage \acro{TSMS} for time
series that share the same time instants of acquisition. Different
time series are stored grouped by the time of acquisition instead of
each time series isolated.  They compare \emph{Tsdb} with \emph{RRDtool} and
with a relational product. As a consequence of \emph{Tsdb} structure,
they achieve a better measure addition time but a worse global
retrieval time as data has to be contiguously regrouped. However, when
measures have same time this is seen as the same time series in a
\acro{MTSMS}, so it would be interesting to use this implementation
architecture of shared time arrays in \acro{MTSMS} for resolution subseries
with same delta time in order to achieve better performance requirements
when having much equal acquired time series.


There are other lossy compression techniques for time series devoted
to the optimal approximation representation, that is finding the
compromise between least data that can reconstruct the original signal
with least error. Keogh et al.\ \cite{keogh01} cite some possible
approximation representations for time series such as Fourier
transforms, wavelets, symbolic mappings or piecewise linear
representation. They remark this last one as very usual due to its
simplicity and develop a system called \emph{iSAX}
\cite{keogh08:isax,keogh10:isax} in order to analyse and index massive
collections of time series. They describe that the main problem is in
the indexing of time series and they propose methods for processing
efficiently. The first method proposed is based on a constant
piecewise approximation. The time series representation obtained with
\emph{iSAX} allows reducing the stored space and indexing faster with
the same quality as other more complex representation methods.  These
compression techniques are candidates for being used as attribute
aggregate functions in the \acro{MTSMS} model, as instance it would be
interesting to define aggregations in the frequency domain of time
series.


 


\subsection{Data stream}



There are other \acro{TSMS} specifically designed for a particular
field requirements.  \emph{Cougar} \cite{bonnet01} is a sensor
database system that has two main structures: one for sensor
properties stored into relational tables and another for time series
stored into data sequences from sensors. Time series have specific
operations and can combine relations and sequences. \emph{Cougar}
target field is sensor networks, where data is stored distributed in
different locations. Queries are resolved combining sensor data in a
data stream abstraction that improves processing performance.

Time series as data streams are also considered when aggregating
statistically data in order to do fast approximate queries with
compressed data. Cormode et al.\ \cite{cormode08:pods} develop
aggregation techniques that consider giving more weight to recent
information.  Our \acro{MTSMS} model applies a similar approach of
weighting more recent data but specifically to time series, with
multiple aggregations and considering time irregularities.  Dou et
al.\ \cite{dou14:historic_queries_flash_storage} create index
structures as multiresolution aggregates, like average, count, or top,
for historical data managed in flash storage; they consider a specific
storage solution based on register with pointers similar to the
multiresolution storage in \emph{RRDtool} \cite{lisa98:oetiker}.







\todo{wavelet}

També hi ha l'anàlisi de les sèries temporals amb wavelet analysis. Aquest es basa en anàlisis de la freqüència dels senyals. 

A multiresolution analysis (MRA) or multiscale approximation (MSA) is the design method of most of the practically relevant discrete wavelet transforms (DWT) and the justification for the algorithm of the fast wavelet transform (FWT). It was introduced in this context in 1988/89 by Stephane Mallat and Yves Meyer and has predecessors in the microlocal analysis in the theory of differential equations (the ironing method) and the pyramid methods of image processing as introduced in 1981/83 by Peter J. Burt, Edward H. Adelson and James Crowley.






\todo{}






\subsection{Sistemes de bases de dades genèrics}




\begin{description}


\item[Calanda] \textcite{dreyer94} proposen els requeriments de
  propòsit específic que han de complir els \gls{SGST} i basen el
  model en quatre elements estructurals bàsics: esdeveniments, sèries
  temporals, grups i metadades, a banda de les bases de dades per
  sèries temporals. Implementen un \gls{SGST} anomenat
  Calanda \parencite{dreyer94b,dreyer95,dreyer95b} que té operacions
  de calendari, pot agrupar sèries temporals i respondre consultes
  simples i ho exemplifiquen amb dades econòmiques. A \cite{schmidt95}
  es compara Calanda amb els \gls{SGBD} per a dades bitemporals.


\item[TDM] \textcite{segev87:sigmod} presenten un model, que anomenen
  \emph{Temporal Data Management} (TDM), per a dades temporals amb un
  llenguatge molt semblant a \gls{SQL}. Les seqüències temporals que
  presenten són molt similars a les que definim en el model de
  \gls{SGST}, tot i que molt lligades a un tercer atribut que indica
  l'objecte de referència. Principalment estudien les operacions
  d'agregació sobre les sèries temporals.



\item[SciDB]
\textcite{stonebraker09:scidb} estudien els SGBD científiques amb models  de dades basats en matrius. Estan desenvolupant SciDB \parencite{scidb}, un SGBD productiu i optimitzat per treballar amb matrius.


\item[SciQL]
\textcite{kersten11} descriuen SciQL, un llenguatge per a SGBD científiques basades en matrius. Hi ha un prototip en desenvolupament de SciQL \parencite{sciql}.




\item[Pandas i altres]
Aquests potser no són sistemes de gestió de bases de dades però sí que gestionen sèries temporals.

\url{http://pandas.pydata.org/pandas-docs/stable/index.html}

\url{http://pytseries.sourceforge.net/}




\end{description}



%SETL http://setl.org/setl/ un llenguatge de programació d'alt nivell que té els conjunts i els mapes de primer ordre com a parts fonamentals. Els tipus bàsics són conjunts, conjunts desordenats i seqüències (també anomenades tuples). Els mapes són conjunts de parelles (tuples de mida dos). Les operacions bàsiques inclouen la pertinença, la unió, la intersecció, etc.




\subsection{Tècniques de compressió i aproximació}




\begin{description}


\item[T-Time] \textcite{assfalg08:thesis} mostra un sistema que pot
  cercar similituds entre sèries temporals, calculades segons funcions
  de distàncies entre sèries temporals. Principalment, dues sèries
  temporals es marquen com a similars si la seva distància és menor
  a un llindar per cada interval de temps. A partir d'aquest mètode dissenya
  algoritmes eficients que implementa en un programa anomenat
  T-Time \parencite{assfalg08:ttime}.

 
\item[iSAX] \textcite{keogh08:isax,keogh10:isax} estudien l'anàlisi i
  l'indexat de co\l.lecions massives de sèries temporals. Descriuen
  que el problema principal del tractament rau en l'indexat de les
  sèries temporals i proposen mètodes per calcular-lo de manera
  eficient. El mètode principal que proposen està basat en
  l'aproximació a trossos de la sèrie
  temporal \parencite{keogh00}.  Ho implementen en una estructura de
  gestió de dades que anomenen \emph{indexable Symbolic Aggregate
    approXimation} (iSAX) \parencite{isax}. Les representacions de
  sèries temporals que s'obtenen amb aquesta eina permeten reduir
  l'espai emmagatzemat i indexar tant bé com altres mètodes de
  representació més complexos.
% Piecewise Aggregate Approximation (PAA) \cite{keogh00}: aproxima una sèrie temporal partint-la en segments de la mateixa mida i emmagatzemant la mitjana dels punts que cauen dins del segment. Redueix de dimensió $n$ a dimensió $N$

% Adaptive Piecewise Constant Approximation (APCA) \cite{keogh01}: com el PAA però amb segments de mida variable.




\item[TSDS]
\textcite{weigel10} noten la necessitat de mostrar les dades en tot el seu rang temporal i no només en un subconjunt com normalment s'ofereixen. Desenvolupen el paquet informàtic \emph{Time Series Data Server} (TSDS) \parencite{tsds} a on es poden introduir les dades de sèries temporals per posteriorment consultar-les per rangs temporals o aplicant-hi filtres i operacions.




\item[RRDtool]
RRDtool \parencite{rrdtool} {é}s un SGBD molt usat per la comunitat de programari lliure. Projectes en diversos camps l'utilitzen com a SGBD, en els quals hi ha sistemes de monitoratge professionals, també en l'àmbit de programari lliure, com Nagios/Icinga \parencite{nagios,icinga} o el Multi Router Traffic Grapher (MRTG) \parencite{mrtg}. Aquests monitors transfereixen a RRDtool la responsabilitat de gestionar l'emmagatzematge i d'operar amb les dades, i així es poden centrar en l'adquisició de dades i la gestió d'alarmes. 
En l'evolució de RRDtool hi ha dues millores destacables. En primer lloc, \textcite{lisa98:oetiker} va separar el sistema de gestió de RRDtool de MRTG i el va dissenyar amb una estructura característica de Round Robin. En segon lloc,  \textcite{lisa00:brutlag} va estendre RRDtool amb algoritmes de predicció i detecció de comportaments aberrants. 

Actualment, s'està estudiant l'eficiència i rapidesa de RRDtool a processar sèries temporals. \textcite{carder:rrdcached} ha dissenyat una aplicació, \emph{rrdcached}, que millora el rendiment de RRDtool amb la qual s'aconsegueix fer funcionar  simultàniament sistemes amb grans quantitats de bases de dades RRDtool \parencite{lisa07:plonka}. \textcite{jrobin} han dissenyat una adaptació de RRDtool anomenada \emph{JRobin}. 
Finalment, és destacable l'ús emergent de RRDtool en entorns d'experimentació, com és el cas de \textcite{zhang07} i \textcite{chilingaryan10} que hi emmagatzemen dades experimentals per posteriorment predir o validar-les.


\item[Whisper] Whisper és el sistema de base de dades que utilitza Graphite. És molt semblant a RRDtool, de fet inicialment Graphite usava RRDtool com a sistema d'emmagatzematge.

\url{http://graphite.wikidot.com/whisper}
\url{http://graphite.wikidot.com/} graphite, semblant a RRDtool



\end{description}



\subsection{Processament en flux}



\begin{description}

\item[Cougar]
\textcite{cougar,fung02} proposen Cougar com un SGBD per xarxes de sensors (\emph{sensor database systems}). El sistema té dues estructures \parencite{bonnet01}: una basada en relacions per les característiques dels sensors i una basada en seqüències per les dades dels sensors, les quals són sèries temporals.
Les consultes es processen de manera distribuïda: cada sensor és un node amb capacitat de processament que pot resoldre una part de la consulta i fusionar-la amb les altres. D'aquesta manera es minimitza l'ús de comunicacions però l'estructura i estratègia de comunicació dels nodes esdevé una part crítica a configurar \parencite{demers03}.


\item[TinyDB]
Un altre prototip de SGBD per xarxes de sensors desenvolupat para\l.lelament a Cougar és TinyDB \parencite{tinyDB,madden05}. A part de les característiques descrites per Cougar, aquest sistema  modifica i s'implica en parts del procés d'adquisició de les dades com és el temps, la freqüència o l'ordre de mostreig. Per exemple donada una consulta que vol correlacionar les dades de dos sensors, el sistema indica als sensors implicats que han d'adquirir amb la mateixa freqüència.


\end{description}




% \url{http://2013.nosql-matters.org/bcn/abstracts/#abstract_gianmarco}

% Streaming data analysis in real time is becoming the fastest and most efficient way to obtain useful knowledge from what is happening now, allowing organizations to react quickly when problems appear or to detect new trends helping to improve their performance. In this talk, we present SAMOA, an upcoming platform for mining big data streams. SAMOA is a platform for online mining in a cluster/cloud environment. It features a pluggable architecture that allows it to run on several distributed stream processing engines such as S4 and Storm. SAMOA includes algorithms for the most common machine learning tasks such as classification and clustering. 




\subsection{Emmagatzematge massiu}

\todo{} Hi ha sistemes que aborden l'emmagatzematge massiu de les
sèries temporals, és a dir de grans volums de dades, seguint
l'enfocament de \gls{VLDB}. 


\begin{description}

\item[OpenTSDB]
\url{http://opentsdb.net/docs/build/html/user_guide/query/aggregators.html}  Aquest també defineix el concepte d'agregadors, però només sap interpolar linealment. Sobretot per a fer visualització. Basa l'emmagatzematge en Apache Hadoop i HBase, emmagatzema les dades originals i calcula totes les operacions durant l'execució de les consultes. És a dir, emmagatzema totes les dades originals i es basa en una estructura distribuïda d'emmagatzematge en què és ràpid d'escriure-hi les dades i localitzar-les, ja que HBase té uns índex potents per a les dades.  Permet fer consultes amb agregacions en què les agregacions siguin per a un interval temporal petit de les dades.



OpenTSDB \cite{deri12:tsdb_compressed_database}
\url{http://opentsdb.net/}
Han fet anàlisis del rendiment de RRDtool, MySQL i la seva implementació TSDB i conclouen que RRDtool és el que pitjor funciona per a sèries temporals. La seva implementació, TSDB, es basa en la compressió de dades. Assumeixen que les sèries temporals són regulars i totes tenen el mateix patró de mostreig, fet que els permet implementar les gestió de les sèries temporals de manera més senzilla.
\todo{Potser aquest va a compressió de dades doncs?}


\end{description}





% http://stackoverflow.com/questions/4814167/storing-time-series-data-relational-or-non















%%% Local Variables: 
%%% mode: latex
%%% TeX-master: "main"
%%% End: 



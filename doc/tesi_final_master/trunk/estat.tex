\chapter{Estat actual}
\label{cap:estat}

En aquest capítol se situen els sistemes de gestió de bases de dades (SGBD) per sèries temporals en el context de la mineria de dades de sèries temporals (\emph{time series data mining}), el qual també es considerat com mineria de dades per  detectar automàticament coneixement (\emph{knowledge discovery databases}). Els SGBD de model Round Robin (RRD) pertanyen a aquest context ja que  emmagatzemen sèries temporals  de les quals es vol aconseguir informació rellevant.


El capítol comença resumint l'estat de les sèries temporals en aquest camp de mineria; és a dir d'emmagatzematge i tractament. A continuació es llisten algunes aplicacions informàtiques que han implementat models de la mineria de sèries temporals. Finalment, es descriu l'estat actual de l'aplicació RRDtool, la qual també es classifica en aquest camp.



\section{Mineria de sèries temporals}

L'anàlisi de sèries temporals abasta camps molt diferents com ara la predicció econòmica, la medicina, la meteorologia, la qualitat industrial, etc. En aquest context,  la mineria de sèries temporals tracta de gestionar co\l.leccions cronològiques de dades que tenen una mida gran i contínuament estan en creixement. 
Aquest apartat se centra en  l'estat actual de la mineria de sèries temporals, àmbit que, en la darrera dècada, ha experimentat un important increment de la recerca.


En un article molt recent de Tak-chung Fu,~\cite{fu11}, es fa esment d'aquest increment de recerca i es resumeix l'estat actual de forma exhaustiva. Fu conclou que la recerca s'ha centrat en tasques de mineria però no s'ha resolt del tot el problema de la representació de sèries temporals.

Segons Keogh i Kasetty,~\cite{keogh02}, les quatre tasques que centren l'atenció de la recerca actual de sèries temporals són l'indexat, l'agrupament, la classificació i la segmentació. A més, Keogh compara els experiments duts a terme en aquests camps.
Un pas comú previ a aquestes quatre tasques és el de representació de la sèrie temporal. 
Keogh \emph{et al.},~\cite{keogh97,keogh98}, investiguen la representació de sèries temporals a trossos lineals (PLR, \emph{Piecewise Linear Representation}). Keogh fa notar que la representació PLR és l'habitual degut a que la visió de l'ésser humà segmenta les corbes en línies rectes.
Més tard, Keogh,~\cite{keogh00,keogh01}, explora la representació de sèries temporals per tal de reduir la dimensió d'una sèrie temporal i poder-la indexar més fàcilment  i proposa dues tècniques eficients en el càlcul: la PAA (\emph{Piecewise Aggregate Aproximation}) i  la APCA (\emph{Adaptive Piecewise Constant Approximation}), ambdues basades en la representació a trossos constants de la sèrie temporal. 
D'aquestes dues tècniques Keogh conclou que mantenen una bona aproximació a la sèrie temporal i que a més  tenen molt menys cost de càlcul que altres de més complicades, com ara la \emph{Discrete Fourier Transform} (DFT),  la  \emph{Singular Value Decomposition} (SVD) o la \emph{Discrete Wavelet Transform} (DWT).



Tal com expliquen Puig \emph{et al.},~\cite{puig10}, en un sistema complex de telecontrol hi ha una gran quantitat d'informació a manipular que s'obté de diversos sensors distribuïts pel camp de mesura, aquesta informació s'anomena variables mesurades. Un SCADA (\emph{Supervisory Control And Data Acquisition})  és el sistema encarregat de recollir i centralitzar les variables de manera periòdica en el temps. En el moment de reco\l.lecció de dades apareixen dos problemes: valors que en un instant de temps prefixat no s'han pogut recollir i valors que són falsos. Les tècniques de bases de dades no poden emmagatzemar les dades amb aquests dos tipus de problema ja que aleshores els registres històrics quedarien falsejats. Així doncs, cal comprovar que les dades emmagatzemades són correctes, segons un procés de validació, i modificar-les en el cas que siguin falses, segons un procés de reconstrucció. Puig,~\cite{puig10}, aplica aquests processos a xarxes de distribució d'aigua.

Els mètodes de validació i reconstrucció es poden basar en anàlisis senzilles del senyal o en comparacions del valor real amb models de predicció de dades. Quan les dades es tracten com a sèries temporals, hi ha mètodes de predicció específics.
Tot i que la teoria de sèries temporals permet establir aquests mètodes de predicció i reconstrucció, els SGBD habituals, com ara els de model relacional, no ho faciliten.  
Per tal d'aplicar aquests mètodes a les sèries temporals de manera eficient, els SGBD s'han d'especialitzar en el tractament de sèries temporals.



\section{SGBD per sèries temporals}


Per poder analitzar les dades de manera eficient cal disposar de bases de dades específiques, a més cada cop el volum de dades a tractar és més crític degut a que hi ha més facilitat a capturar-les i més capacitat per emmagatzemar-les. 
La diferència principal de les sèries temporals amb altres tipus de dades és que els valors són dependents d'una variable: el temps. Com a conseqüència, qualsevol base de dades que hi vulgui tractar no ho pot fer de manera independent pels valors i pel temps; ha de conservar la coherència temporal.

Tal com diu A{\ss}falg,~\cite{assfalg08:_advan_analy_temp}, la coherència temporal pot ser vista des de dues vessants. La primera, a la qual anomena \emph{bitemporal data}, consisteix en expressar el temps vàlid durant el qual un esdeveniment és cert i el temps de transacció durant el qual l'esdeveniment és guardat a la base de dades, és a dir consisteix a descriure dos estats, cert o fals, per cada observació. La segona, a la qual anomena \emph{time series data}, consisteix a descriure co\l.leccions de dades en funció del temps. A més diu que les primeres poden ser expressades amb les segones.

Els SGBD relacionals són capaços d'implementar el primer tipus de coherència, les \emph{bitemporal data}; llavors es classifiquen sota el nom de bases de dades temporals, \cite{Date,wiki:temporal_database}. Però el model relacional no és suficient pel segon tipus: les sèries temporals. Tot i que en principi no hi hauria cap problema a utilitzar una base de dades relacional per a sèries temporals, enteses com a dades històriques, la pròpia naturalesa dels sistemes relacionals  dificulta les operacions necessàries. 
Aquestes operacions per sèries temporals es basen en rangs de temps i precisen conversions de fusos horaris i rotacions dels registres de les taules, sinó el nombre de files creixeria de forma indefinida. 

Els SGBD que implementen operacions per a sèries temporals es poden anomenar \emph{Time Series Database Systems} (TSDS),~\cite{tsds}. Les TSDS Estan optimitzades per gestionar les dades segons les operacions de temps i rotació, les quals són molt comunes en la gestió de les sèries temporals.  A més també cal controlar el creixement de la base de dades i la consulta ha de ser flexible i d'alta velocitat,~\cite{keogh10:isax}. Per exemple, s'han de poder visualitzar les evolucions tant d'una setmana com d'un any sense haver de fer càlculs complicats amb els valors emmagatzemats. 
A continuació es llisten dues bases de dades optimitzades per a sèries temporals.

A{\ss}falg,~\cite{assfalg08:_advan_analy_temp}, presenta un TSDS que és capaç de
cercar similituds, també anomenades distàncies, entre sèries temporals. Principalment utilitza llindars per comparar en cada interval si les dues sèries temporals s'assemblen. A partir d'aquest mètode desenvolupa algoritmes que calculen de manera eficient per a les sèries temporals i en concret els implementa en una aplicació anomenada T-Time, la qual descriu a~\cite{assfalg08:ttime}.

Keogh i Camerra~\cite{keogh08:isax,keogh10:isax}, 
estudien l'anàlisi i l'indexat de co\l.lecions massives de sèries temporals. Descriuen que el problema principal del tractament rau en l'indexat de les sèries temporals i proposen mètodes per calcular-lo de manera eficient. El mètode principal que desenvolupen està basat en l'aproximació a trossos constants de la sèrie temporal (PAA,~\cite{keogh00}) i ho implementen en una estructura de dades que anomenen iSAX (\emph{indexable Symbolic Aggregate approXimation}),~\cite{isax}. Amb aquesta eina s'obtenen representacions de sèries temporals que permeten reduir l'espai emmagatzemat i indexar tant bé com altres mètodes de representació més complexos.




En resum, aquests SGBD per sèries temporals bàsicament resolen els problemes d'anàlisis de sèries temporals.
Però cap d'aquestes sol atendre la relació entre la base de dades i el sistema de monitoratge, és a dir la manera com s'adquireixen les dades. En aquest pas intermig hi ha un sèrie de problemes, com per exemple forats, dades falses, irregularitat en els temps de mostreig, que cal gestionar correctament. Concretament un dels problemes que no s'atén és el de mostreig irregular ja que es considera que les mostres estan a intervals regulars (equi-espaiades) encara que els sistemes de monitoratge informàtics sovint no són capaços de complir-ho amb exactitud sinó que presenten una certa variació en els temps de mesura. 

Així doncs, quan es prenen mesures d'un sistema productiu, aquests problemes apareixen i són de difícil solució.
Les bases de dades RRDtool tenen en compte aquests problemes intermitjos entre el sistema de monitoratge i el sistema d'emmagatzematge i tractament. 






\section{Base de dades RRDtool}

En aquest apartat es presenta el TSDS anomenat RRDtool. Aquest sistema, que serà objecte d'un estudi acurat en els capítols \ref{cap:rrdtool} i~\ref{cap:rrdtool-etapes}, s'ha pres com a referència en aquest treball.

RRDtool és un SGBD per a sèries temporals que despunta en l'àmbit de programari lliure. Hi ha una llista de projectes que utilitzen RRDtool que poden trobar-se indicats a l'apartat \emph{Projects using RRDtool} de~\cite{rrdtool}.
Entre d'altres, s'utilitza en sistemes de monitoratge professionals com per exemple Nagios,~\cite{nagios}, o Icinga,~\cite{icinga}, també populars dins del programari lliure, o en el montior MRTG (The Multi Router Traffic Grapher),~\cite{mrtg}, del mateix creador que RRDtool. Aquests monitors fan un ús complet de les possibilitats de RRDtool i li cedeixen tot el control de l'emmagatzematge de mesures i el posterior tractament i representació gràfica de les dades. 
L'ús de RRDtool permets a aquestes aplicacions centrar-se plenament en la problemàtica de l'adquisició de dades i la gestió d'alarmes.

En l'evolució de RRDtool destaquen dues millores significatives.
La primera, descrita per Oetiker a~\cite{lisa98:oetiker}, va consistir en independitzar la base de dades RRDtool del sistema de monitoratge MRTG i dissenyar-la amb l'estructura Round Robin que la caracteritza. La segona, feta per Brutlag,~\cite{lisa00:brutlag}, ha aportat la possibilitat de fer prediccions i detecció de comportaments aberrants basant-se en algoritmes de predicció exponencials i de Holt-Winters. 


L'evolució actual de RRDtool se centra en aspectes informàtics i consisteix a millorar la rapidesa i eficiència en el processament de les sèries temporals. És el cas de Plonka i Carder que a~\cite{carder:rrdcached,lisa07:plonka} dissenyen l'aplicació \verb+rrdcached+ per incrementar el rendiment de RRDtool, la qual demostren en un sistema de monitoratge amb moltes bases de dades funcionant simultàniament.  També \verb+JRobin+,~\cite{jrobin}, que és una implementació en Java de RRDtool que millora els accessos de lectura i escriptura a la base de dades i té una eina de gràfics més perfeccionada.
És significatiu l'ús incipient d'aquest sistema en experimentació. Zhang,~\cite{zhang07}, i Chilingaryan,~\cite{chilingaryan10}, per exemple, usen RRDtool per emmagatzemar de dades experimentals i posteriorment fer predicció o validació.
  

En l'àmbit dels SGBD els sistemes relacionals van fixar una fita que ha tingut una transcendència posterior de  primer ordre. En bona part aquest èxit dels SGBD relacionals es deu al fet que es basen en un model matemàtic sòlid,~\cite{Date}.
En el cas de RRDtool no existeix  cap model que descrigui el sistema i es objectiu d'aquest treball proposar-ne un. El model per a SGBD Round Robin es dissenya  al capítol~\ref{cap:model-rrd}.






%%% Local Variables: 
%%% mode: latex
%%% TeX-master: "memoria"
%%% End: 

% LocalWords:  monitoratge RRDtool SGBD RRD Round databases mining SCADA And
% LocalWords:  Supervisory Acquisition bitemporal Time Database Systems TSDS
% LocalWords:  Nagios Icinga Grapher Holt-Winters

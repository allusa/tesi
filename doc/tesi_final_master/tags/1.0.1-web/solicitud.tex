\documentclass[a4paper,12pt,oneside,catalan]{article}
\usepackage[catalan]{babel}
\usepackage[utf8]{inputenc}

\parindent0cm 
\parskip0.2cm

\begin{document}


\title{So\l.licitud de presentació de Tesi de Màster}
\author{Aleix Llusà Serra\\
Directors: Teresa Escobet Canal i Sebastià Vila-Marta}
\date{Màster U.\ en Automàtica i Robòtica, \today}

\maketitle

\thispagestyle{empty}

{\bfseries Títol Provisional:} Estudi i modelització de les bases de dades Round Robin pel tractament de sèries temporals.



A l'inici del projecte s'havia plantejat estudiar l'emmagatzematge de dades tractades com a sèries temporals amb la tècnica de bases de dades Round Robin (RRD). L'objectiu era comprovar el funcionament d'aquestes bases de dades per a validació i reconstrucció ja que en aquestes operacions hi ha un gran tractament amb sèries temporals.

No obstant, no s'ha trobat un model de bases de dades Round Robin i la documentació existent es redueix als manuals tècnics d'usuari de l'única implementació coneguda: RRDtool. Això complica i impossibilita l'estudi detallat de mètodes implementats a sobre d'aquestes bases de dades.

A causa d'això s'ha hagut de centrar el projecte a estudiar en detall els manuals tècnics i a partir d'aquests proposar un model per a aquestes bases de dades. S'ha redactat una proposta d'un model bàsic per a les bases de dades Round Robin a on s'analitza com es tracten les sèries temporals.

A més a més, un cop obtingut el model s'ha pogut implementar utilitzant el llenguatge Python. Hi ha disponible una primera versió d'aquest programa a on es pot fer un tractament bàsic d'una sèrie temporal.

Finalment, queda per acabar de redactar la informació recopilada a la memòria del projecte, així com documentar la implementació del model amb Python.
 
\vspace{2cm}

Alumne: \hfill Directors: \hfill \mbox{}

\end{document}
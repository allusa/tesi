\begin{frame}{Conclusions}

  \begin{itemize}
  \item S'ha situat l'estat actual de l'emmagatzematge i del tractament de sèries temporals.

  \item S'ha estudiat el sistema de gestió de bases de dades RRDtool.

  \item S'ha dissenyat un model que descriu l'estructura i el
    comportament dels SGBD Round Robin per a sèries temporals.
  
  \item S'ha proposat una implementació de referència del model dissenyat.


 \item Un SGBD Round Robin és un sistema informàtic d'emmagatzematge d'una sèrie temporal compactada i repartida segons diferents funcions d'interpolació i períodes de mostreig.

  \end{itemize}

\end{frame}




\begin{frame}{Treball futur}

\begin{itemize}

\item A partir del model RRD es poden estudiar l'aplicació de tècniques d'anàlisis de sèries temporals  i es poden dissenyar SGBD RRD assegurant el funcionament desitjat.


\item Tractament de dades desconegudes. Interpoladors específics.

\item Operacions de consulta: extreure dades, fusionar sèries temporals, visualització, predicció, cerca de patrons, etc.

\item Configurant els interpoladors i els períodes de mostreig s'aconsegueixen reduccions diferents de les sèries temporals.

\item Comprovació amb dades experimentals, (\cite{keogh02}).

\item Variacions en la implementació del model per complir restriccions.

\end{itemize}
\end{frame}


\begin{frame}

\begin{center}
{\huge
Gràcies per l'atenció!
}
\end{center}

\end{frame}

%%% Local Variables: 
%%% mode: latex
%%% TeX-master: "presentacio"
%%% End: 
